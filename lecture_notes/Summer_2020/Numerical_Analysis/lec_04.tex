
\documentclass[class=article, crop=false]{standalone} 
\usepackage{amsmath, amsfonts, amsthm, amssymb, graphicx}
\usepackage[utf8]{inputenc}
\usepackage[english]{babel}
	
\theoremstyle{plain}
\newtheorem{thm}{Theorem}
\newtheorem{lemma}{Lemma}
\newtheorem{coro}{Corollary}
\newtheorem{prop}{Proposition}

\renewcommand*{\thefootnote}{\fnsymbol{footnote}}
\renewcommand\qedsymbol{}

\theoremstyle{remark} 
\newtheorem{case}{Case}

\newcommand{\rr}{\ensuremath{\mathbb{R}}}
\newcommand{\zz}{\ensuremath{\mathbb{Z}}}
\newcommand{\qq}{\ensuremath{\mathbb{Q}}}
\newcommand{\nn}{\ensuremath{\mathbb{N}}}
\newcommand{\ff}{\ensuremath{\mathbb{F}}}

\providecommand{\norm}[1]{\lVert#1\rVert}

\pdfsuppresswarningpagegroup=1


\begin{document}
\section{Integration}
\subsection{Trapezoidal Rule}
We can build a degree one Lagrange polynomial for every two points.
\begin{equation*}
\begin{split}
	A_1&=\frac{h}{2}(f_0+f_1)\\
	\ldots	
\end{split}
\end{equation*}
\subsection{Simpson's Rule}
Now use every three points for Lagrange polynomial.
\[
	P_2 = \frac{(x-x_1) (x-x_2) }{(x_0-x_1)(x_0-x_2) }f_0 + \frac{(x-x_0) (x-x_2) }{(x_1-x_0)(x_1-x_2) }f_1 + \frac{(x-x_0) (x-x_1) }{(x_2-x_0)(x_2-x_1) }f_2
.\] 
We can use substitution $s = \frac{x-x_1}{h}$, so the equation simplifies to:
 \[
	 P_2 = \frac{1}{2}s(s-1) f_0 - (s+1) (s-1) f_1 + \frac{1}{2}(s+1) s f_2
.\] 
After integration we obtain
\[
	A_1 = \frac{h}{3} (f_0+ 4 f_1 +f_2) 
.\]
\[
	\int_a^b f(x)dx = \frac{h}{3}[f_0+4f_1+2f_2+4f_3+\ldots+2 f_{n-2} + 4 f_{n-1} + f_n]
.\] 


\subsection{Newton-Cotes}
After substitution $s = \frac{x-x_0}{h}$, and resetting $x=s$, consider
\[
	\int_0^1 f(x)dx = a f(0)  + b f(1)  
.\] 
Expand $f(x) $ as 
\[
	f(x) =f(0) + x f'(0) + \frac{x^2}{2}f''(0) + \ldots 
.\]
Integrate this expansion from 0 to 1, and expand $f(1) $, we can solve for $a,b$, we obtain  $a=b=\frac{1}{2}$.
This leads to the trapezoidal rule.

Similarly, consider
\[
	\int_{-1}^1 f(x) dx=a f(-1) + b f(0) + c f(1)  
.\] 
By doing the "grand accounting" again, we obtain the Simpson's rule, with error of $\mathcal{O}(h^5) $proportional to the 4th derivatives.

\subsection{Example}
Consider:
\[
\int_0^1 e^{(-x^2)}dx
.\]

\end{document}

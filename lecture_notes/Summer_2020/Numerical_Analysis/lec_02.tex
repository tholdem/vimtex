\documentclass[class=article, crop=false]{standalone} 
\usepackage{amsmath, amsfonts, amsthm, amssymb, graphicx}
\usepackage[utf8]{inputenc}
\usepackage[english]{babel}
	
\theoremstyle{plain}
\newtheorem{thm}{Theorem}
\newtheorem{lemma}{Lemma}
\newtheorem{coro}{Corollary}
\newtheorem{prop}{Proposition}

\renewcommand*{\thefootnote}{\fnsymbol{footnote}}
\renewcommand\qedsymbol{}

\theoremstyle{remark} 
\newtheorem{case}{Case}

\newcommand{\rr}{\ensuremath{\mathbb{R}}}
\newcommand{\zz}{\ensuremath{\mathbb{Z}}}
\newcommand{\qq}{\ensuremath{\mathbb{Q}}}
\newcommand{\nn}{\ensuremath{\mathbb{N}}}
\newcommand{\ff}{\ensuremath{\mathbb{F}}}

\providecommand{\norm}[1]{\lVert#1\rVert}

\pdfsuppresswarningpagegroup=1

\begin{document}
\section{Horner's Method} 
\begin{equation*}
\begin{split}
	P(x)&=(x-z)Q_0(z)+R_0 \\
	    &=(x-z) [b_nx^{n-1}+ \ldots +b_1] +b_0\\
	   a_n x^n + \ldots + a_0  &=(b_n x^n + b_{n-1}x^{b-1}+ \ldots +b_1 x )- z(b_n x^{n-1} + \ldots+b_1 )) +b_0 \\
\end{split}
\end{equation*}
\begin{equation*}
\begin{split}
	b_n&=a_n\\
	b_{n-1}&=a_{n-1}+zb_n\\
	\ldots\\
	b_1&=a_1+z b_2\\
	b_0 &=a_0
\end{split}
\end{equation*}
Let $b_{n+1}=0$\\
Do this again
\begin{equation*}
\begin{split}
	c_n&=b_n\\
	\ldots\\
	c_1&=b_1+z c_2
\end{split}
\end{equation*}
Let $c_{n+1}=0$, $c_k=b_k+z c_{k+1} $.\\

Root deflation: roots found in the end suffer more from numerical errors.\\

\section{Chapter 3 Interpolation: Lagrange Polynomials}

Let 
\[
	P(x) = \sum_{k=1}^n  P_k(x) = \sum_{k=0}^n L_{n,k} f(x_k)     
.\] 
where
\[
L_{n,k} =  \frac{ (x-x_0)(x-x_1)\ldots(x-x_{k-1})(x-x_{k+1}\ldots (x-x_{n}))}{(x_k -x_0)\ldots (x_k-x_{k-1})(x_k-x_{k+1}) (x_k-x_{n})))} 
.\] 
\[
	e=\frac{f^{n+1}(\xi)}{n+1}! (x-x_0)(x-x_1)\ldots(x-x_n)      
.\] 

\section{Neville's Method}

\section{Cubic Spline}
Linear doesn't work because it's not smooth at the junctions.
Quadratic misses one condition for the 6 parameters.
Cubic is the sweet spot where 8 parameters have 8 reasonable conditions.
Conditions for two consecutive cubic polynomials:
\begin{itemize}
	\item matches function values at two end points, for both functions
	\item matches each other's values at the middle point 
	\item matches derivatives at the middle point
	\item matches 2nd derivatives at the overlapped points
\end{itemize}
\end{document}

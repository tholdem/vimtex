\documentclass[class=article, crop=false]{standalone} 
\usepackage{amsmath, amsfonts, amsthm, amssymb, graphicx}
\usepackage[utf8]{inputenc}
\usepackage[english]{babel}
	
\theoremstyle{plain}
\newtheorem{thm}{Theorem}
\newtheorem{lemma}{Lemma}
\newtheorem{coro}{Corollary}
\newtheorem{prop}{Proposition}

\renewcommand*{\thefootnote}{\fnsymbol{footnote}}
\renewcommand\qedsymbol{}

\theoremstyle{remark} 
\newtheorem{case}{Case}

\newcommand{\rr}{\ensuremath{\mathbb{R}}}
\newcommand{\zz}{\ensuremath{\mathbb{Z}}}
\newcommand{\qq}{\ensuremath{\mathbb{Q}}}
\newcommand{\nn}{\ensuremath{\mathbb{N}}}
\newcommand{\ff}{\ensuremath{\mathbb{F}}}

\providecommand{\norm}[1]{\lVert#1\rVert}

\pdfsuppresswarningpagegroup=1


\begin{document}
\section{Direct method}
Consider
\[
	\int_{-1}^1 f(x)dx = a f(-1) +b f(0) +c f(1)   
.\] 
Now pretend $f(x)=1,x,x^2,\ldots$, keep plugging in the next order until we get inconsistency. Then we obtain the same coefficient as Simpson's rule.

But we don't have to stick with $x=-1,0,1$. If we let  $x=-\frac{2}{3,0,\frac{2}{3}}$, then we get something different.

We can generalize even further. Consider
\[
	\int_{-1}^1 f(x) \sin \frac{\pi}{2}x dx =  a f(-1) +b f(0) +c f(1)
.\] 
Repeat the same procedure and we obtain the weighted values.

Transforming integrals:
Let $t = \frac{2x-a-b}{b-a}$, hence
\[
	\int_a^b f(x) dx = \int_{-1}^1 f\left( \frac{t(b-a)+a+b}{2} \right) \frac{b-a}{2} dt 
.\] 

\section{Gaussian Quadrature}
We want to find:
\[
	\int_{-1}^1 f(x)dx = \sum_{i=1}^n c_i f(x_{i}) 
.\] 
$c_i and x_{i}$ give us $2n$ parameters to choose, so the polynomial is at most  $2n-1$ degree.

\subsection{Legendre Polynomials}
They are orthogonal with respect to the inner product $\int_{-1}^1 P(x)P_n(x)dx $, where $P_n(x)$ is the nth Legendre polynomial.

Example:
\[
\int_0^1 e^{(-x^2)}dx = \frac{1}{2}\int_{-1}^1 e^{(-\frac{t+1}{2})^2}dt
.\] 
This is a lot less work than Simpson's.

Advantage: good accuracy\\
Disadvantage: uneven spacing, so if we don't know $f(x)$ there might be too much interpolation.

\section{Improper Integrals}

Consider the integration of functions with a singularity at $x=a$ (left endpoint) of the form:
 \[
	 f(x) = \frac{g(x) }{(x-a)^{p}} 
.\] 
where $g(x) $ is continuous on $[a,b]$. And we want $\int_a^b f(x) $. Note this converges iff $0<p<1$. 

For right singularity, you can just flip the expression. For middle singularity, break it into two parts.

If infinity appears, change variable to $t=\frac{1}{x}$.
\end{document}

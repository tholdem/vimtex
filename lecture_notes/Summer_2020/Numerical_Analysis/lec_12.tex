\documentclass[class=article,crop=false]{standalone} 
\usepackage{amsmath, amsfonts, amsthm, amssymb, graphicx}
\usepackage[utf8]{inputenc}
\usepackage[english]{babel}
	
\theoremstyle{plain}
\newtheorem{thm}{Theorem}
\newtheorem{lemma}{Lemma}
\newtheorem{coro}{Corollary}
\newtheorem{prop}{Proposition}

\renewcommand*{\thefootnote}{\fnsymbol{footnote}}
\renewcommand\qedsymbol{}

\theoremstyle{remark} 
\newtheorem{case}{Case}

\newcommand{\rr}{\ensuremath{\mathbb{R}}}
\newcommand{\zz}{\ensuremath{\mathbb{Z}}}
\newcommand{\qq}{\ensuremath{\mathbb{Q}}}
\newcommand{\nn}{\ensuremath{\mathbb{N}}}
\newcommand{\ff}{\ensuremath{\mathbb{F}}}

\providecommand{\norm}[1]{\lVert#1\rVert}

\pdfsuppresswarningpagegroup=1


\begin{document}
\section{Heat conduction problem}
Consider a rod with length $L$.
\[
\rho c \frac{\partial T}{\partial t} =k \frac{\partial^2 T}{\partial x^2} 
.\] 
Where $\rho$ is the density,  $c$ is the heat capacity, and  $k$ is the heat conductivity.
Initial condition: $T(x,t)|_{t=0}=f(x)$\\
Boundary conditions: $T(x,t)|_{x=0}=T_0$, $T(x,t)|_{x=L}=T_0$

We want to nondimensionalize the problem to reduce complexity:
Let $\alpha=\frac{k}{\rho c}$ be the thermal diffusivity, $\overline{x}=\frac{x}{L}$, $U=\frac{T-T_0}{T_r}$, so $x=\overline{x}L$, $T=UT_r + T_0$.

\begin{align*}
	\frac{\partial UT_r + T_0}{\partial t} &= \alpha \frac{\partial^2 UTr+T_0}{\partial {\overline{x}L}^2}  \\
	T_r \frac{\partial U}{\partial t} &= \frac{T_r \alpha}{L^2} \frac{\partial^2 U}{\partial x^2}  \\
	\frac{\partial U}{\partial \frac{t \alpha}{L^2}} &= \frac{\partial^2 U}{\partial {\overline{x}}^2} 
\end{align*}

Let $\tau=\frac{t \alpha}{L^2}=\frac{t}{t_{ref}}$, so $t_{ref}=\frac{L^2}{\alpha}$, and we obtain:
\[
	\frac{\partial U}{\partial \tau} = \frac{\partial^2 U}{\partial {\overline{x}}^2} 
.\] 

\begin{align*}
	U(\overline{x},\tau)|_{\tau=0} &= \frac{f(x)-T_0}{T_r} := F(x) \\
	U(\overline{x},\tau)|_{\overline{x}=0} &= 0 \\
	U(\overline{x},\tau)|_{\overline{x}=1} &= 0
\end{align*}

\[
U(\overline{x},\tau)=\sin(\pi\overline{x}) e^{-t\pi^2}
.\]

Make a grid in the $\overline{x}-\tau$ plane.
Let $k=\Delta \tau$, $h=\Delta \overline{x}$.
Using finite difference approximation:
\[
	\frac{U_{i,j+1}-U_{ij}}{k} = \frac{U_{i+1,j}-2U_{ij} + U_{i-1,j}}{h^2}
.\] 
the LHS is the forward difference in time, and the RHS is the central difference in space.
\begin{align*}
	U_{i,j+1}&= U_{ij} +\frac{k}{h^2}(U_{i+1,j}-2U_{ij} +U_{i-1,j}) \\
		 &= (1-2r) U_{ij}  + r(U_{i+1,j}+U_{i-1,j}) 
\end{align*}
where $r=\frac{k}{h^2}=\frac{\Delta \tau}{\Delta x^2}$.

Set $r=\frac{1}{4}$, $\Delta x = 0.2$, then $k = 0.01$. 
Set $r=1$,  $\Delta x = 0.2$, then  $k = 0.04$. Starts to fall apart a little bit.
Set $r=2.5$,$\Delta x = 0.2$, then $k=0.1$. Results violate physics.

\subsection{Crank-Nicolson Method}
We can fix the problem above by using an implicit method. We want to average $\frac{\partial^2 U}{\partial {x}^2} $ at time  $j$ and $j+1$. Three points in the past and three points in the future.
 \begin{align*}
	 \frac{U_{i,j+1} -U_{ij} }{k} = \frac{1}{2} \left(  \frac{U_{i+1,j} - 2 U_{ij} + U_{i-1,j}}{h^2} +   \frac{U_{i+1,j+1} - 2 U_{i,j+1} + U_{i-1,j+1}}{h^2}  \right)  \\
	 -r U_{i+1,j+1}+(2+2r) U_{i,j+1} - r U_{i-1,j+1} = r U_{i+1,j} + (2-2r) U_{ij} + r U_{i-1,j}   	 	
\end{align*}

This is much more stable.
Symmetry gives us two equations for two unknowns.
\end{document}

\documentclass[class=article, crop=false]{standalone} 
\usepackage{amsmath, amsfonts, amsthm, amssymb, graphicx}
\usepackage[utf8]{inputenc}
\usepackage[english]{babel}
	
\theoremstyle{plain}
\newtheorem{thm}{Theorem}
\newtheorem{lemma}{Lemma}
\newtheorem{coro}{Corollary}
\newtheorem{prop}{Proposition}

\renewcommand*{\thefootnote}{\fnsymbol{footnote}}
\renewcommand\qedsymbol{}

\theoremstyle{remark} 
\newtheorem{case}{Case}

\newcommand{\rr}{\ensuremath{\mathbb{R}}}
\newcommand{\zz}{\ensuremath{\mathbb{Z}}}
\newcommand{\qq}{\ensuremath{\mathbb{Q}}}
\newcommand{\nn}{\ensuremath{\mathbb{N}}}
\newcommand{\ff}{\ensuremath{\mathbb{F}}}

\providecommand{\norm}[1]{\lVert#1\rVert}

\pdfsuppresswarningpagegroup=1



\begin{document}	 
\section{Bisection}
\section{Fixed point}
\[
	e_n = r- x_n = g(r) - g(x_{n-1}) = (r-x_{n-1})g'(\xi) = e_{n-1}g'(\xi) \approx e_{n-1}g'(r)
.\] 
where $\zeta \in (x_n{n-1},r)$.
\[
	g(r) = g(x_{n-1}) + (r-x_{n-1})g'(x_{n-1}) + \ldots
.\] 
\[
.\] 
This is linear convergence.

\[
	\frac{e_{n+2}}{e_{n+1}} \approx \frac{e_{n+1}}{e_n}
.\] 

\section{Secant Method}
$p_{n+1} = p_n - \frac{f( _{n+1})( p_{n+1} - p_n)}{f( p_{n+1}) -f( p_n)  }$

"False Position"

\subsection{"Sins"}
\begin{itemize}
	\item Do not subtract numbers that has very small differences
	\item Do not divide with a piece of "garbage"
	\item Do not set stopping criterion to equal 0, use $\epsilon$
\end{itemize}

\section{Newton's Method}
\[
	p_{n+1} = p_n - \frac{f(p_{n})}{f'(p_n)}
.\]
Problems with this method:
\begin{itemize}
	\item As  $p_n \to p*$, $f(p_n) \to 0$ . So the convergence gets slower.
	\item really small slope near the r
	\item local minimum
	\item inflection point
\end{itemize}
Convergence:
\[
	e_n = \frac{f''(\xi)}{2f'(r)}e_{n-1}^2 \approx \frac{f''(r)}{2f'(r)}e_{n-1}^2
.\] 
If $e_{n+1}/e_n^\alpha = \lambda$, we call $\alpha$ the order of convergence.

Aitken's $Delta^2$ Method

Steffensen's Method
\begin{itemize}
	\item Pick $p_0$
	\item $p_1= g(p_0)$
	\item $p_2=g(p_1)$
	\item compute $\hat{p_0}$ 
	\item $p_0=\hat{p_0}$ 
	repeat
\end{itemize}

If denominator $p_{n+2}-2p_{n+1}+p_n \approx 0$, then let $p_0=p_2$ in the next iteration instead.

Consider $f(x)=ax^2+bx+c$. If $b^2 >> 4ac$ then error occurred in two roots might differ significantly. To compute the "small" root more accurately, we multiply the root equation by 1 using the conjugate.
Assume $b>0$, the small root should be computed by
\[
	x=\frac{2c}{b+(b^2-4ac)^{ .5}}
.\] 

Instead of using secant, wouldn't it be better to use a parabola?
Try $s(x)=a(x-x_2)^2+b(x-x_2)+c  $ 
\[
	f_0=a(x_0-x_2)^2+b(x_0-x_2)+c
	f_1=\ldots
	f_2=c
.\] 
Now we want the root $s(x_3)=0 $ that is closer to $x_2$, which is the "small" root.
\[
	x_3-x_2=-\frac{2c}{b+sgn(b)\sqrt{b^2-4ac}}
.\] 
This is cubic convergence.

\[
	p(x)=(x-z)[( x-z)Q_1(x)+R_1 ]+R_0 
.\] 
$R_0=p(z) $
\end{document}

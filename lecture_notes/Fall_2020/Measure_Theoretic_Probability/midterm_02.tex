\documentclass[12pt]{article}
%Fall 2020
% Some basic packages
\usepackage{standalone}[subpreambles=true]
\usepackage[utf8]{inputenc}
\usepackage[T1]{fontenc}
\usepackage{textcomp}
\usepackage[english]{babel}
\usepackage{url}
\usepackage{graphicx}
\usepackage{float}
\usepackage{enumitem}


\pdfminorversion=7

% Don't indent paragraphs, leave some space between them
\usepackage{parskip}

% Hide page number when page is empty
\usepackage{emptypage}
\usepackage{subcaption}
\usepackage{multicol}
\usepackage[dvipsnames]{xcolor}


% Math stuff
\usepackage{amsmath, amsfonts, mathtools, amsthm, amssymb}
% Fancy script capitals
\usepackage{mathrsfs}
\usepackage{cancel}
% Bold math
\usepackage{bm}
% Some shortcuts
\newcommand{\rr}{\ensuremath{\mathbb{R}}}
\newcommand{\zz}{\ensuremath{\mathbb{Z}}}
\newcommand{\qq}{\ensuremath{\mathbb{Q}}}
\newcommand{\nn}{\ensuremath{\mathbb{N}}}
\newcommand{\ff}{\ensuremath{\mathbb{F}}}
\newcommand{\cc}{\ensuremath{\mathbb{C}}}
\renewcommand\O{\ensuremath{\emptyset}}
\newcommand{\norm}[1]{{\left\lVert{#1}\right\rVert}}
\renewcommand{\vec}[1]{{\mathbf{#1}}}
\newcommand\allbold[1]{{\boldmath\textbf{#1}}}

% Put x \to \infty below \lim
\let\svlim\lim\def\lim{\svlim\limits}

%Make implies and impliedby shorter
\let\implies\Rightarrow
\let\impliedby\Leftarrow
\let\iff\Leftrightarrow
\let\epsilon\varepsilon

% Add \contra symbol to denote contradiction
\usepackage{stmaryrd} % for \lightning
\newcommand\contra{\scalebox{1.5}{$\lightning$}}

% \let\phi\varphi

% Command for short corrections
% Usage: 1+1=\correct{3}{2}

\definecolor{correct}{HTML}{009900}
\newcommand\correct[2]{\ensuremath{\:}{\color{red}{#1}}\ensuremath{\to }{\color{correct}{#2}}\ensuremath{\:}}
\newcommand\green[1]{{\color{correct}{#1}}}

% horizontal rule
\newcommand\hr{
    \noindent\rule[0.5ex]{\linewidth}{0.5pt}
}

% hide parts
\newcommand\hide[1]{}

% si unitx
\usepackage{siunitx}
\sisetup{locale = FR}

% Environments
\makeatother
% For box around Definition, Theorem, \ldots
\usepackage[framemethod=TikZ]{mdframed}
\mdfsetup{skipabove=1em,skipbelow=0em}

%definition
\newenvironment{defn}[1][]{%
\ifstrempty{#1}%
{\mdfsetup{%
frametitle={%
\tikz[baseline=(current bounding box.east),outer sep=0pt]
\node[anchor=east,rectangle,fill=Emerald]
{\strut Definition};}}
}%
{\mdfsetup{%
frametitle={%
\tikz[baseline=(current bounding box.east),outer sep=0pt]
\node[anchor=east,rectangle,fill=Emerald]
{\strut Definition:~#1};}}%
}%
\mdfsetup{innertopmargin=10pt,linecolor=Emerald,%
linewidth=2pt,topline=true,%
frametitleaboveskip=\dimexpr-\ht\strutbox\relax
}
\begin{mdframed}[]\relax%
\label{#1}}{\end{mdframed}}


%theorem
%\newcounter{thm}[section]\setcounter{thm}{0}
%\renewcommand{\thethm}{\arabic{section}.\arabic{thm}}
\newenvironment{thm}[1][]{%
%\refstepcounter{thm}%
\ifstrempty{#1}%
{\mdfsetup{%
frametitle={%
\tikz[baseline=(current bounding box.east),outer sep=0pt]
\node[anchor=east,rectangle,fill=blue!20]
%{\strut Theorem~\thethm};}}
{\strut Theorem};}}
}%
{\mdfsetup{%
frametitle={%
\tikz[baseline=(current bounding box.east),outer sep=0pt]
\node[anchor=east,rectangle,fill=blue!20]
%{\strut Theorem~\thethm:~#1};}}%
{\strut Theorem:~#1};}}%
}%
\mdfsetup{innertopmargin=10pt,linecolor=blue!20,%
linewidth=2pt,topline=true,%
frametitleaboveskip=\dimexpr-\ht\strutbox\relax
}
\begin{mdframed}[]\relax%
\label{#1}}{\end{mdframed}}


%lemma
\newenvironment{lem}[1][]{%
\ifstrempty{#1}%
{\mdfsetup{%
frametitle={%
\tikz[baseline=(current bounding box.east),outer sep=0pt]
\node[anchor=east,rectangle,fill=Dandelion]
{\strut Lemma};}}
}%
{\mdfsetup{%
frametitle={%
\tikz[baseline=(current bounding box.east),outer sep=0pt]
\node[anchor=east,rectangle,fill=Dandelion]
{\strut Lemma:~#1};}}%
}%
\mdfsetup{innertopmargin=10pt,linecolor=Dandelion,%
linewidth=2pt,topline=true,%
frametitleaboveskip=\dimexpr-\ht\strutbox\relax
}
\begin{mdframed}[]\relax%
\label{#1}}{\end{mdframed}}

%corollary
\newenvironment{coro}[1][]{%
\ifstrempty{#1}%
{\mdfsetup{%
frametitle={%
\tikz[baseline=(current bounding box.east),outer sep=0pt]
\node[anchor=east,rectangle,fill=CornflowerBlue]
{\strut Corollary};}}
}%
{\mdfsetup{%
frametitle={%
\tikz[baseline=(current bounding box.east),outer sep=0pt]
\node[anchor=east,rectangle,fill=CornflowerBlue]
{\strut Corollary:~#1};}}%
}%
\mdfsetup{innertopmargin=10pt,linecolor=CornflowerBlue,%
linewidth=2pt,topline=true,%
frametitleaboveskip=\dimexpr-\ht\strutbox\relax
}
\begin{mdframed}[]\relax%
\label{#1}}{\end{mdframed}}

%proof
\newenvironment{prf}[1][]{%
\ifstrempty{#1}%
{\mdfsetup{%
frametitle={%
\tikz[baseline=(current bounding box.east),outer sep=0pt]
\node[anchor=east,rectangle,fill=SpringGreen]
{\strut Proof};}}
}%
{\mdfsetup{%
frametitle={%
\tikz[baseline=(current bounding box.east),outer sep=0pt]
\node[anchor=east,rectangle,fill=SpringGreen]
{\strut Proof:~#1};}}%
}%
\mdfsetup{innertopmargin=10pt,linecolor=SpringGreen,%
linewidth=2pt,topline=true,%
frametitleaboveskip=\dimexpr-\ht\strutbox\relax
}
\begin{mdframed}[]\relax%
\label{#1}}{\qed\end{mdframed}}


\theoremstyle{definition}

\newmdtheoremenv[nobreak=true]{definition}{Definition}
\newmdtheoremenv[nobreak=true]{prop}{Proposition}
\newmdtheoremenv[nobreak=true]{theorem}{Theorem}
\newmdtheoremenv[nobreak=true]{corollary}{Corollary}
\newtheorem*{eg}{Example}
\theoremstyle{remark}
\newtheorem*{case}{Case}
\newtheorem*{notation}{Notation}
\newtheorem*{remark}{Remark}
\newtheorem*{note}{Note}
\newtheorem*{problem}{Problem}
\newtheorem*{observe}{Observe}
\newtheorem*{property}{Property}
\newtheorem*{intuition}{Intuition}


% End example and intermezzo environments with a small diamond (just like proof
% environments end with a small square)
\usepackage{etoolbox}
\AtEndEnvironment{vb}{\null\hfill$\diamond$}%
\AtEndEnvironment{intermezzo}{\null\hfill$\diamond$}%
% \AtEndEnvironment{opmerking}{\null\hfill$\diamond$}%

% Fix some spacing
% http://tex.stackexchange.com/questions/22119/how-can-i-change-the-spacing-before-theorems-with-amsthm
\makeatletter
\def\thm@space@setup{%
  \thm@preskip=\parskip \thm@postskip=0pt
}

% Fix some stuff
% %http://tex.stackexchange.com/questions/76273/multiple-pdfs-with-page-group-included-in-a-single-page-warning
\pdfsuppresswarningpagegroup=1


% My name
\author{Jaden Wang}



\begin{document}
\centerline {\textsf{\textbf{\LARGE{Midterm 2}}}}
\centerline {Jaden Wang}
\vspace{.15in}

\begin{problem}[1]
We wish to show inequality in both directions to prove equality.

$(\geq)$: Since  $ f \geq 0$ is measurable, there exists a sequence of simple measurable functions $ (f_n)$ such that $ 0\leq f_n \nearrow f$. This means that given $ f_n, \omega$, $ f_n(\omega) \leq f(\omega)$.
\begin{align*}
	\int_{\Omega} f\ d \mu &= \int_{\Omega} \lim_{ n \to \infty} f_n\ d \mu \\
			       &=  \lim_{ n \to \infty}\sum_i a_{n_i} \mu(A_{n_i}) \\
                               &=  \limsup_{  n}  \sum_i a_{n_i} \mu(A_{n_i}) \\
			       &\leq \limsup_{  n} \sum_i \left[ \inf_{\omega \in A_{n_i}} f(\omega)\right] \mu(A_{n_i})\\
			       &\leq \sup \sum_i \left[ \inf_{\omega \in A_i} f(\omega) \right] \mu(A_i)  
\end{align*}

$ (\leq):$ Since the supremum of a set is always greater or equal to the supremum of its subset,
 \begin{align*}
	\int_{\Omega} f\ d \mu &= \sup_{0\leq s \leq f} s\ d \mu\\
			       &= \sup \sum_i a_i \mu(A_i) \text{ where } a_i \leq f(w) \text{ for } \omega \in A_i \\
			       &\geq \sup \sum_i \left[ \inf_{\omega \in A_i} f(\omega) \right]  \mu(A_i) \text{ since } \inf \text{ is a special case of }a_i   
\end{align*}

By inequality in both direction we show that they are equal.
\end{problem}

\begin{problem}[2]
\begin{case}[1]
$ f,g$ are simple. Then  let $ f= \sum_{ i= 1}^{ n} a_i I_{A_i}$ and $ g= \sum_{ j= 1}^{ m} b_j I_{B_j}$ where $ A_i, B_j$ are respectively disjoint partitions of $ \Omega$. Then
\begin{align*}
	gf &= \left( \sum_{ j= 1}^{ m} b_j I_{B_j} \right) \left( \sum_{ i= 1}^{ n} a_i I_{A_i} \right)  \\
	&= \sum_{ j= 1}^{ m} \sum_{ i= 1}^{ n} b_j a_i I_{A_i \cap B_j}
\end{align*}
Note that if a given $\omega \not\in B_j$ or $ \omega \not\in A_i$, then the product $ b_j a_i I_{B_j} I_{A_i}$ will be zero. Thus the only terms remain are the ones with $ I_{A_i \cap B_j}$. Now consider
\begin{align*}
	\int_{\Omega} g\ d \nu &= \sum_{ j= 1}^{ m} b_j \nu(B_j) \\
	&= \sum_{ j= 1}^{ m} b_j \int_{B_j} f d\ \mu \\
	&= \sum_{ j= 1}^{ m} b_j \sum_{ i= 1}^{ n} a_{i} \mu(A_i \cap B_j) \\
	&= \sum_{ j= 1}^{ m} \sum_{ i= 1}^{ n} b_j a_i \mu (A_i \cap B_j) \\
	&= \int_{\Omega} gf\ d \mu
\end{align*}
\end{case}
\begin{case}[2]
	$ f,g$ are not simple. Since $ f,g \geq 0$ are measurable, there exists sequences of simple measurable functions $ (f_n), (g_m)$ such that  $0\leq f_n \nearrow f , 0\leq g_m \nearrow g$. By Case 1, we know that for each pair of $ f_n, g_m$, 
	\[
	\int_{\Omega} g_m \ d \nu = \int_{\Omega} g_m f_n \ d \mu
	.\] 
Taking $ n,m \to \infty$, by Lebesgue's Monotone Convergence Theorem,
\begin{align*}
	\lim_{ n,m \to \infty} \int_{\Omega} g_m d\ \nu &= \lim_{ n,m \to \infty} \int_{\Omega} g_m f_n \ d \mu\\
	\int_{\Omega} g\ d\nu &= \int_{\Omega} gf\ d \mu \\
\end{align*}
\end{case}
\end{problem}

\begin{problem}[3]
	$ (\implies)$: Suppose $ E(X) =0$, we want to show that  $ P(X=0)= 1$ by showing $ P((X=0)^{c})=0$. Since $ X\geq 0$, the complement of  $ X=0$ is  $ X>0$. Let  $ B = \{\omega: X(\omega) > 0\} $, it suffices to show that $ P(B) = 0$.

	Note that since $ X\geq 0$, $ 0\leq X I_B \leq X$. Therefore,
	\[
		0 \leq \int X I_B\ dP \leq \int X\ dP = E(X) = 0
	.\]
	It follows that $ \int X I_B\ dP = \int_B X\ dP = 0$. Recall that  $ X(\omega) >0 \ \forall \ \omega \in B$, then it must be that $ P(B) = P(X>0) = 0$ as required.

	$ (\impliedby)$: Suppose $ P(X=0)=1 \implies P(X > 0) = 0$. $ X=0$ is clearly a simple function where  $ X(\omega) = 0 \cdot I_{\Omega}$. 
	\begin{align*}
		E(X) &= \int X\ dP \\
		     &= 0 \cdot  P(\Omega)\\
		&= 0 \\
	\end{align*}
\end{problem}
\end{document}

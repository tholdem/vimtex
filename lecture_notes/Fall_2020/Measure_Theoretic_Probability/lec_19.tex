\documentclass[class=article,crop=false]{standalone} 
%Fall 2020
% Some basic packages
\usepackage{standalone}[subpreambles=true]
\usepackage[utf8]{inputenc}
\usepackage[T1]{fontenc}
\usepackage{textcomp}
\usepackage[english]{babel}
\usepackage{url}
\usepackage{graphicx}
\usepackage{float}
\usepackage{enumitem}


\pdfminorversion=7

% Don't indent paragraphs, leave some space between them
\usepackage{parskip}

% Hide page number when page is empty
\usepackage{emptypage}
\usepackage{subcaption}
\usepackage{multicol}
\usepackage[dvipsnames]{xcolor}


% Math stuff
\usepackage{amsmath, amsfonts, mathtools, amsthm, amssymb}
% Fancy script capitals
\usepackage{mathrsfs}
\usepackage{cancel}
% Bold math
\usepackage{bm}
% Some shortcuts
\newcommand{\rr}{\ensuremath{\mathbb{R}}}
\newcommand{\zz}{\ensuremath{\mathbb{Z}}}
\newcommand{\qq}{\ensuremath{\mathbb{Q}}}
\newcommand{\nn}{\ensuremath{\mathbb{N}}}
\newcommand{\ff}{\ensuremath{\mathbb{F}}}
\newcommand{\cc}{\ensuremath{\mathbb{C}}}
\renewcommand\O{\ensuremath{\emptyset}}
\newcommand{\norm}[1]{{\left\lVert{#1}\right\rVert}}
\renewcommand{\vec}[1]{{\mathbf{#1}}}
\newcommand\allbold[1]{{\boldmath\textbf{#1}}}

% Put x \to \infty below \lim
\let\svlim\lim\def\lim{\svlim\limits}

%Make implies and impliedby shorter
\let\implies\Rightarrow
\let\impliedby\Leftarrow
\let\iff\Leftrightarrow
\let\epsilon\varepsilon

% Add \contra symbol to denote contradiction
\usepackage{stmaryrd} % for \lightning
\newcommand\contra{\scalebox{1.5}{$\lightning$}}

% \let\phi\varphi

% Command for short corrections
% Usage: 1+1=\correct{3}{2}

\definecolor{correct}{HTML}{009900}
\newcommand\correct[2]{\ensuremath{\:}{\color{red}{#1}}\ensuremath{\to }{\color{correct}{#2}}\ensuremath{\:}}
\newcommand\green[1]{{\color{correct}{#1}}}

% horizontal rule
\newcommand\hr{
    \noindent\rule[0.5ex]{\linewidth}{0.5pt}
}

% hide parts
\newcommand\hide[1]{}

% si unitx
\usepackage{siunitx}
\sisetup{locale = FR}

% Environments
\makeatother
% For box around Definition, Theorem, \ldots
\usepackage[framemethod=TikZ]{mdframed}
\mdfsetup{skipabove=1em,skipbelow=0em}

%definition
\newenvironment{defn}[1][]{%
\ifstrempty{#1}%
{\mdfsetup{%
frametitle={%
\tikz[baseline=(current bounding box.east),outer sep=0pt]
\node[anchor=east,rectangle,fill=Emerald]
{\strut Definition};}}
}%
{\mdfsetup{%
frametitle={%
\tikz[baseline=(current bounding box.east),outer sep=0pt]
\node[anchor=east,rectangle,fill=Emerald]
{\strut Definition:~#1};}}%
}%
\mdfsetup{innertopmargin=10pt,linecolor=Emerald,%
linewidth=2pt,topline=true,%
frametitleaboveskip=\dimexpr-\ht\strutbox\relax
}
\begin{mdframed}[]\relax%
\label{#1}}{\end{mdframed}}


%theorem
%\newcounter{thm}[section]\setcounter{thm}{0}
%\renewcommand{\thethm}{\arabic{section}.\arabic{thm}}
\newenvironment{thm}[1][]{%
%\refstepcounter{thm}%
\ifstrempty{#1}%
{\mdfsetup{%
frametitle={%
\tikz[baseline=(current bounding box.east),outer sep=0pt]
\node[anchor=east,rectangle,fill=blue!20]
%{\strut Theorem~\thethm};}}
{\strut Theorem};}}
}%
{\mdfsetup{%
frametitle={%
\tikz[baseline=(current bounding box.east),outer sep=0pt]
\node[anchor=east,rectangle,fill=blue!20]
%{\strut Theorem~\thethm:~#1};}}%
{\strut Theorem:~#1};}}%
}%
\mdfsetup{innertopmargin=10pt,linecolor=blue!20,%
linewidth=2pt,topline=true,%
frametitleaboveskip=\dimexpr-\ht\strutbox\relax
}
\begin{mdframed}[]\relax%
\label{#1}}{\end{mdframed}}


%lemma
\newenvironment{lem}[1][]{%
\ifstrempty{#1}%
{\mdfsetup{%
frametitle={%
\tikz[baseline=(current bounding box.east),outer sep=0pt]
\node[anchor=east,rectangle,fill=Dandelion]
{\strut Lemma};}}
}%
{\mdfsetup{%
frametitle={%
\tikz[baseline=(current bounding box.east),outer sep=0pt]
\node[anchor=east,rectangle,fill=Dandelion]
{\strut Lemma:~#1};}}%
}%
\mdfsetup{innertopmargin=10pt,linecolor=Dandelion,%
linewidth=2pt,topline=true,%
frametitleaboveskip=\dimexpr-\ht\strutbox\relax
}
\begin{mdframed}[]\relax%
\label{#1}}{\end{mdframed}}

%corollary
\newenvironment{coro}[1][]{%
\ifstrempty{#1}%
{\mdfsetup{%
frametitle={%
\tikz[baseline=(current bounding box.east),outer sep=0pt]
\node[anchor=east,rectangle,fill=CornflowerBlue]
{\strut Corollary};}}
}%
{\mdfsetup{%
frametitle={%
\tikz[baseline=(current bounding box.east),outer sep=0pt]
\node[anchor=east,rectangle,fill=CornflowerBlue]
{\strut Corollary:~#1};}}%
}%
\mdfsetup{innertopmargin=10pt,linecolor=CornflowerBlue,%
linewidth=2pt,topline=true,%
frametitleaboveskip=\dimexpr-\ht\strutbox\relax
}
\begin{mdframed}[]\relax%
\label{#1}}{\end{mdframed}}

%proof
\newenvironment{prf}[1][]{%
\ifstrempty{#1}%
{\mdfsetup{%
frametitle={%
\tikz[baseline=(current bounding box.east),outer sep=0pt]
\node[anchor=east,rectangle,fill=SpringGreen]
{\strut Proof};}}
}%
{\mdfsetup{%
frametitle={%
\tikz[baseline=(current bounding box.east),outer sep=0pt]
\node[anchor=east,rectangle,fill=SpringGreen]
{\strut Proof:~#1};}}%
}%
\mdfsetup{innertopmargin=10pt,linecolor=SpringGreen,%
linewidth=2pt,topline=true,%
frametitleaboveskip=\dimexpr-\ht\strutbox\relax
}
\begin{mdframed}[]\relax%
\label{#1}}{\qed\end{mdframed}}


\theoremstyle{definition}

\newmdtheoremenv[nobreak=true]{definition}{Definition}
\newmdtheoremenv[nobreak=true]{prop}{Proposition}
\newmdtheoremenv[nobreak=true]{theorem}{Theorem}
\newmdtheoremenv[nobreak=true]{corollary}{Corollary}
\newtheorem*{eg}{Example}
\theoremstyle{remark}
\newtheorem*{case}{Case}
\newtheorem*{notation}{Notation}
\newtheorem*{remark}{Remark}
\newtheorem*{note}{Note}
\newtheorem*{problem}{Problem}
\newtheorem*{observe}{Observe}
\newtheorem*{property}{Property}
\newtheorem*{intuition}{Intuition}


% End example and intermezzo environments with a small diamond (just like proof
% environments end with a small square)
\usepackage{etoolbox}
\AtEndEnvironment{vb}{\null\hfill$\diamond$}%
\AtEndEnvironment{intermezzo}{\null\hfill$\diamond$}%
% \AtEndEnvironment{opmerking}{\null\hfill$\diamond$}%

% Fix some spacing
% http://tex.stackexchange.com/questions/22119/how-can-i-change-the-spacing-before-theorems-with-amsthm
\makeatletter
\def\thm@space@setup{%
  \thm@preskip=\parskip \thm@postskip=0pt
}

% Fix some stuff
% %http://tex.stackexchange.com/questions/76273/multiple-pdfs-with-page-group-included-in-a-single-page-warning
\pdfsuppresswarningpagegroup=1


% My name
\author{Jaden Wang}



\begin{document}
\section*{Chapter 3: Integration}

$ (\Omega,\mathcal{F},\mu)$, $ f:\Omega \to \rr$ measurable. 
\begin{notation}
	\[ \int f d \mu = \int_{\Omega} f(\omega) d \mu(\omega) = \int_{\Omega} f(\omega) \mu(d\omega).\]
\end{notation}

\begin{defn}[integration]
	For a simple function $ f(\omega)= \sum_{ i= 1}^{ n} a_i I_{A_i}(\omega)$, $ A_i \in \mathcal{F}$, we define
	\[
		\int_{\Omega} f d \mu = \sum_{ i= 1}^{ n} a_i \mu(A_i)
	.\] 
	For any $ B \in \mathcal{F}$, define
	\[
		\int_B f d \mu = \sum_{ i= 1}^{ n} a_i \mu(A_i \cap B)
	.\] 
\end{defn}

\begin{eg}[]
~\begin{enumerate}[label=\arabic*)]
	\item $ \int_{\Omega} I_A d \mu = \mu(A)$.
	\item $ f = \sum_{ i= 1}^{ n} a_i I_{A_i}$, then $ \int f \ d\lambda$ is the Riemann integral.
	\item Recall Theorem 12.4 ($ F$ non-decreasing, right continuous, real-valued, there exists a unique measure $ \mu$ on $ \mathcal{B}(\rr)$ satisfying $ \mu((a,b])=F(b)-F(a)$. And $ \mu((a,b))=\mu((a,b ^{-}))= F(b ^{-}) - F(a)$ ). $ \mu$ is called the \allbold{Lebesgue-Stietjes meausre} given by $ F$. Suppose  $ f$ is a non-negative Riemann integrable function, and suppose $ F$ is defined by  $ F(x)=\int_{-\infty}^x f(y)\ dy$. Then for $ a<b$,
		 \[
			 \int_{\rr} I_{(a,b]} d \mu = \mu((a,b]) = F(b)-F(a) = \int_a^b f(x) \ dx
		.\]
		Moreover,
		\[
			\int_{\rr} I_{([a,b])} =\mu((-\infty,b])-\mu((-\infty,a^-]) = F(b)-F(a^-)
		.\] 
\end{enumerate}
\end{eg}

~\begin{defn}[]
	$ (\Omega,\mathcal{F},\mu), f:\Omega \to \rr$ measurable. If $ f$ is non-negative, define, for any  $ A \in \mathcal{F}$,
	\[
	\int_{A} f \ d \mu = \sup \int_A s\ d \mu
	\]
	where the supremum is taken over all simple functions $ s$ where  $ 0\leq s(\omega) \leq f(\omega) \ \forall \ \omega \in A$.
\end{defn}
\begin{note}[]
	This is well-defined since $ s(\omega) =0 \ \forall \ \omega \in \Omega$ is one element in the set. If the supremum is infinite, we say either "$ f$ is not integrable over  $ A$" or "$ f$ has infinite integral over  $ A$".
\end{note}

Facts:
\begin{enumerate}[label=\arabic*)]
	\item $ 0\leq f \leq g \implies \int_A f \  d \mu \leq \int_A g\ d \mu$.
	\item $ A \subseteq B \implies \int_A f\ d \mu \leq \int_B f\ d \mu$.
		\begin{prf}
			Take any simple $s: \Omega \to \rr $ such that $ 0<s<f$. Then $ s$ can be written as  $ s(\omega)=\sum_{ i= 1}^{ n} a_i I_{A_i} (\omega)$  for some partition $ A_1, \ldots, A_n$ of $ A$, assuming $ a_i$ are distinct. Then
			\[
				A \subseteq B \implies A_i \cap A \subseteq A_i \cap B \implies \mu(A_i \cap A) \leq \mu(A_i \cap B)
			\] 
			so
			\begin{align*}
				\int_A s\ d \mu&= \sum_{ i= 1}^{ n} a_i \mu(A_i \cap A) \\
					      &\leq \sum_{ i= 1}^{ n} a_i \mu(A_i \cap B)  \\
					      &= \int_B s\ d \mu \\
			\end{align*}
			Since $ s$ is arbitrary, this relationship should hold for the suprema:
			 \[
			\int_A f \ d \mu = \sup_{0\leq s \leq f} \int_A s \ d \mu \leq \sup_{0\leq s\leq f} \int_B s \ d \mu = \int_B f \ d \mu 
			.\] 
		\end{prf}
	\item For $ c>0$ constant, then  $ \int_A cf d \mu = c \int_A f d \mu$.

		~\begin{prf}
		$ 0\leq s_1 \leq c f$. Define $ s_2 = \frac{s_1}{c }$ is still simple. And constants can be taken out of supremum.
		\end{prf}
	\item $ \mu(A)=0 \implies \int_A f d \mu=0$.
	\item $ \int_A f d \mu = \int_\Omega I_A \cdot  f d \mu$.
\end{enumerate}

More: Let $ S_1,S_2$ be simple, then
\begin{enumerate}[label=\arabic*)]
	\item $ \int_A (s_1 + s_2) d \mu = \int_A s_1 d \mu + \int_A s_2 d \mu$.
	\item Define $ \nu (A) = \int_A s\ d \mu$. Then can show that $ \nu$ is another measure on  $ \Omega,\mathcal{F}$.
\end{enumerate}

\begin{thm}[Lebesgue's Monotone Convergence Theorem]
	Let $ (f_n)$ be a sequence of measurable functions on $(\Omega,\mathcal{F},\mu) $. Suppose $ 0\leq f_1(\omega)\leq f_2(\omega)\leq \ldots \ \forall \ \omega \in \Omega$ and that $ \lim_{ n \to \infty} f_n(\omega) = f(\omega) \ \forall \ \omega \in \Omega$, then
	\[
	\lim_{ n \to \infty} \int f_n \ d \mu=\int f\ d \mu
	.\] 
\end{thm}
~\begin{prf}
We have three cases.
\begin{enumerate}[label=\arabic*)]
	\item Some $ f_n$ are not integrable. 

		That is, $ \int f_n \ d \mu = \infty$. Then given $ M>0$, there exists a simple  $ s$ with  $ 0\leq s \leq f_n$ and $ \int s d \mu > M$. Since $ f_n \nearrow f$, then $ 0\leq s \leq f_n \leq f$. Hence, $ \int f d \mu = \infty = \lim_{ n \to \infty} \int f_n \ d \mu $. 
	\item All $ f_n$ are integrable but $ (\int f_n\ d \mu ) $ diverges.
		
		If we assume divergence, then for any constant $ M >0$, there exists  $ N$  such that $ \int f_n d \mu > M+1 \ \forall \ n\geq M$. So $ \lim_{ n \to \infty} \int f_n \ d \mu = \infty$. By the definition of $ \int f_n d \mu$, there exists a simple $ s, 0\leq s \leq f_n$ such that $ \int s d \mu>M \ \forall \ n\geq N$. Since $ 0\leq s \leq f_n \leq f$,  this $ s$ can make  $ \int f d \mu$ as large as we want. Hence $ \int f \ d \mu = \lim_{ n \to \infty}  \int f_n d \mu = \infty$.
	\item All $ f_n$ are integrable and $ (\int f_n \ d \mu)$ converges.

		$ f_n \leq f_{n+1} \implies \int f_n d \mu \leq \int f_{n+1} d \mu \implies \lim_{ n \to \infty} \int f_n d \mu = \sup_n \left\{\int f_n d \mu \right\} \equiv c$. We need to show that $ f$ is integrable and the integral equals  $ c$.
		
		Let $ s$ be simple with $ 0\leq s \leq f$. Let  $ b$ be any constant  $ (0,1)$. Define
		 \[
			 A_n = \{\omega: f_n(\omega) \geq b \cdot  s(\omega)\} 
		.\] 
		Note that $ A_n \in \mathcal{F}$ because both $ f_n, bs$ are both measurable and by the last theorem in lecture 17.
\end{enumerate}
\end{prf}
\end{document}

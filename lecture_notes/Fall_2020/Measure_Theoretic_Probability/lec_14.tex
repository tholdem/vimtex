\documentclass[class=article,crop=false]{standalone} 
%Fall 2020
% Some basic packages
\usepackage{standalone}[subpreambles=true]
\usepackage[utf8]{inputenc}
\usepackage[T1]{fontenc}
\usepackage{textcomp}
\usepackage[english]{babel}
\usepackage{url}
\usepackage{graphicx}
\usepackage{float}
\usepackage{enumitem}


\pdfminorversion=7

% Don't indent paragraphs, leave some space between them
\usepackage{parskip}

% Hide page number when page is empty
\usepackage{emptypage}
\usepackage{subcaption}
\usepackage{multicol}
\usepackage[dvipsnames]{xcolor}


% Math stuff
\usepackage{amsmath, amsfonts, mathtools, amsthm, amssymb}
% Fancy script capitals
\usepackage{mathrsfs}
\usepackage{cancel}
% Bold math
\usepackage{bm}
% Some shortcuts
\newcommand{\rr}{\ensuremath{\mathbb{R}}}
\newcommand{\zz}{\ensuremath{\mathbb{Z}}}
\newcommand{\qq}{\ensuremath{\mathbb{Q}}}
\newcommand{\nn}{\ensuremath{\mathbb{N}}}
\newcommand{\ff}{\ensuremath{\mathbb{F}}}
\newcommand{\cc}{\ensuremath{\mathbb{C}}}
\renewcommand\O{\ensuremath{\emptyset}}
\newcommand{\norm}[1]{{\left\lVert{#1}\right\rVert}}
\renewcommand{\vec}[1]{{\mathbf{#1}}}
\newcommand\allbold[1]{{\boldmath\textbf{#1}}}

% Put x \to \infty below \lim
\let\svlim\lim\def\lim{\svlim\limits}

%Make implies and impliedby shorter
\let\implies\Rightarrow
\let\impliedby\Leftarrow
\let\iff\Leftrightarrow
\let\epsilon\varepsilon

% Add \contra symbol to denote contradiction
\usepackage{stmaryrd} % for \lightning
\newcommand\contra{\scalebox{1.5}{$\lightning$}}

% \let\phi\varphi

% Command for short corrections
% Usage: 1+1=\correct{3}{2}

\definecolor{correct}{HTML}{009900}
\newcommand\correct[2]{\ensuremath{\:}{\color{red}{#1}}\ensuremath{\to }{\color{correct}{#2}}\ensuremath{\:}}
\newcommand\green[1]{{\color{correct}{#1}}}

% horizontal rule
\newcommand\hr{
    \noindent\rule[0.5ex]{\linewidth}{0.5pt}
}

% hide parts
\newcommand\hide[1]{}

% si unitx
\usepackage{siunitx}
\sisetup{locale = FR}

% Environments
\makeatother
% For box around Definition, Theorem, \ldots
\usepackage[framemethod=TikZ]{mdframed}
\mdfsetup{skipabove=1em,skipbelow=0em}

%definition
\newenvironment{defn}[1][]{%
\ifstrempty{#1}%
{\mdfsetup{%
frametitle={%
\tikz[baseline=(current bounding box.east),outer sep=0pt]
\node[anchor=east,rectangle,fill=Emerald]
{\strut Definition};}}
}%
{\mdfsetup{%
frametitle={%
\tikz[baseline=(current bounding box.east),outer sep=0pt]
\node[anchor=east,rectangle,fill=Emerald]
{\strut Definition:~#1};}}%
}%
\mdfsetup{innertopmargin=10pt,linecolor=Emerald,%
linewidth=2pt,topline=true,%
frametitleaboveskip=\dimexpr-\ht\strutbox\relax
}
\begin{mdframed}[]\relax%
\label{#1}}{\end{mdframed}}


%theorem
%\newcounter{thm}[section]\setcounter{thm}{0}
%\renewcommand{\thethm}{\arabic{section}.\arabic{thm}}
\newenvironment{thm}[1][]{%
%\refstepcounter{thm}%
\ifstrempty{#1}%
{\mdfsetup{%
frametitle={%
\tikz[baseline=(current bounding box.east),outer sep=0pt]
\node[anchor=east,rectangle,fill=blue!20]
%{\strut Theorem~\thethm};}}
{\strut Theorem};}}
}%
{\mdfsetup{%
frametitle={%
\tikz[baseline=(current bounding box.east),outer sep=0pt]
\node[anchor=east,rectangle,fill=blue!20]
%{\strut Theorem~\thethm:~#1};}}%
{\strut Theorem:~#1};}}%
}%
\mdfsetup{innertopmargin=10pt,linecolor=blue!20,%
linewidth=2pt,topline=true,%
frametitleaboveskip=\dimexpr-\ht\strutbox\relax
}
\begin{mdframed}[]\relax%
\label{#1}}{\end{mdframed}}


%lemma
\newenvironment{lem}[1][]{%
\ifstrempty{#1}%
{\mdfsetup{%
frametitle={%
\tikz[baseline=(current bounding box.east),outer sep=0pt]
\node[anchor=east,rectangle,fill=Dandelion]
{\strut Lemma};}}
}%
{\mdfsetup{%
frametitle={%
\tikz[baseline=(current bounding box.east),outer sep=0pt]
\node[anchor=east,rectangle,fill=Dandelion]
{\strut Lemma:~#1};}}%
}%
\mdfsetup{innertopmargin=10pt,linecolor=Dandelion,%
linewidth=2pt,topline=true,%
frametitleaboveskip=\dimexpr-\ht\strutbox\relax
}
\begin{mdframed}[]\relax%
\label{#1}}{\end{mdframed}}

%corollary
\newenvironment{coro}[1][]{%
\ifstrempty{#1}%
{\mdfsetup{%
frametitle={%
\tikz[baseline=(current bounding box.east),outer sep=0pt]
\node[anchor=east,rectangle,fill=CornflowerBlue]
{\strut Corollary};}}
}%
{\mdfsetup{%
frametitle={%
\tikz[baseline=(current bounding box.east),outer sep=0pt]
\node[anchor=east,rectangle,fill=CornflowerBlue]
{\strut Corollary:~#1};}}%
}%
\mdfsetup{innertopmargin=10pt,linecolor=CornflowerBlue,%
linewidth=2pt,topline=true,%
frametitleaboveskip=\dimexpr-\ht\strutbox\relax
}
\begin{mdframed}[]\relax%
\label{#1}}{\end{mdframed}}

%proof
\newenvironment{prf}[1][]{%
\ifstrempty{#1}%
{\mdfsetup{%
frametitle={%
\tikz[baseline=(current bounding box.east),outer sep=0pt]
\node[anchor=east,rectangle,fill=SpringGreen]
{\strut Proof};}}
}%
{\mdfsetup{%
frametitle={%
\tikz[baseline=(current bounding box.east),outer sep=0pt]
\node[anchor=east,rectangle,fill=SpringGreen]
{\strut Proof:~#1};}}%
}%
\mdfsetup{innertopmargin=10pt,linecolor=SpringGreen,%
linewidth=2pt,topline=true,%
frametitleaboveskip=\dimexpr-\ht\strutbox\relax
}
\begin{mdframed}[]\relax%
\label{#1}}{\qed\end{mdframed}}


\theoremstyle{definition}

\newmdtheoremenv[nobreak=true]{definition}{Definition}
\newmdtheoremenv[nobreak=true]{prop}{Proposition}
\newmdtheoremenv[nobreak=true]{theorem}{Theorem}
\newmdtheoremenv[nobreak=true]{corollary}{Corollary}
\newtheorem*{eg}{Example}
\theoremstyle{remark}
\newtheorem*{case}{Case}
\newtheorem*{notation}{Notation}
\newtheorem*{remark}{Remark}
\newtheorem*{note}{Note}
\newtheorem*{problem}{Problem}
\newtheorem*{observe}{Observe}
\newtheorem*{property}{Property}
\newtheorem*{intuition}{Intuition}


% End example and intermezzo environments with a small diamond (just like proof
% environments end with a small square)
\usepackage{etoolbox}
\AtEndEnvironment{vb}{\null\hfill$\diamond$}%
\AtEndEnvironment{intermezzo}{\null\hfill$\diamond$}%
% \AtEndEnvironment{opmerking}{\null\hfill$\diamond$}%

% Fix some spacing
% http://tex.stackexchange.com/questions/22119/how-can-i-change-the-spacing-before-theorems-with-amsthm
\makeatletter
\def\thm@space@setup{%
  \thm@preskip=\parskip \thm@postskip=0pt
}

% Fix some stuff
% %http://tex.stackexchange.com/questions/76273/multiple-pdfs-with-page-group-included-in-a-single-page-warning
\pdfsuppresswarningpagegroup=1


% My name
\author{Jaden Wang}



\begin{document}
\begin{coro}[]
	Let $ \mathcal{P}$ be a $\pi$-system. Suppose that $ \mu_1, \mu_2$ are finite measures on $ \sigma(\mathcal{P})$ and that $ \Omega$ is a countable union of sets in $ \mathcal{P}$. If $ \mu_1, \mu_2$ agree on $ \mathcal{P}$, they agree on $ \sigma(\mathcal{P})$.
\end{coro}
\begin{intuition}[]
	We sort of replace the $ \sigma$-finiteness of $ \mu$ with $ \Omega$ as a countable union of sets.
\end{intuition}
\begin{defn}[general outer-measure]
Let $\Omega$ be a non-empty set. An \allbold{outer measure} on $ \Omega$ is a function
\[
	\mu^* : \text{ Power Set of }\Omega \to [0, \infty] 
\] 
that satisfies
\begin{enumerate}[label=(\roman*)]
	\item $ \mu^* ( \O) = 0$
	\item monotone
	\item countable subadditivity
\end{enumerate}
\end{defn}

\begin{defn}[]
	Let $ \mu^* $ be an outer measure on $ \Omega$. A set $ A \subseteq \Omega$ is said to be \allbold{$ \mu^*$-measurable} if
	\[
		\mu^* (A \cap E) + \mu^* (A^{c} \cap E) = \mu^* (E) \ \forall \ E \subseteq \Omega
	.\] 
\end{defn}

Equivalently, by finite subadditivity of $ \mu^* $:
	\[
		\mu^* (E) = \mu^* ((A \cap E) \cup (A^{c} \cap  E)) \leq \mu^* (A \cap E) + \mu^* (A^{c} \cap E)
	.\] 

\begin{thm}[]
	A set $ A$ is  $ \mu^* $-measurable if $ \mu^* (A \cap E)+ \mu^* (A^{c} \cap E) \leq \mu^* (E)$.
\end{thm}

Let $ \mathcal{M}(\mu^*) =$ the collection of $ \mu^* $-measurable sets.

\begin{thm}[11.1]
$ \mu^* $ is an outer measure implies:
\begin{enumerate}[label=(\roman*)]
	\item $ \mathcal{M}(\mu^* )$ is a $\sigma$-field.
	\item $ \mu^* $, restricted to $ \mathcal{M}(u^* )$, is a measure on $ \mathcal{M}(\mu^* )$
\end{enumerate}
\end{thm}
(The same as the 5 lemma proof for probability outer measure.)

\begin{thm}[11.2]
	A measure $ \mu$ on a field has an extension to the generated $\sigma$-field.
\end{thm}

\begin{claim}[]
	If the original measure is finite or $ \sigma$-finite on the field (cover $ \Omega$ with a countable collection $ \mathcal{F}$-sets with finite measure), then the extension is unique.
\end{claim}

\begin{defn}[semiring]
A class $ \mathcal{A}$ of subsets of $ \Omega$ is called a \allbold{semiring} if
\begin{enumerate}[label=(\roman*)]
	\item $ \O \in \mathcal{A}$.
	\item closed under finite intersections.
	\item For any  $ A,B \in \mathcal{A}$ such that $ A \subseteq B$, there exists disjoint sets $ A_1,A_2,\ldots \in \mathcal{A}$ such that $ B \setminus A = \bigcup_{ i= 1}^{ n} A_i$. Equivalently, $ B = A \cup A_1 \cup  \ldots \cup A_n$ all disjoint.
\end{enumerate}
\end{defn}
\begin{note}[]
	For (iii), think of a picture with $ A$ in  $ B$ and the "ring" formed by the boundaries of  $ A$ and  $ B$ can be partitioned into finite disjoint sets.

A semiring doesn't have additive inverses compared to a ring.

In the set theory context, addition becomes $ A \Delta B$ (symmetric difference) and multiplication becomes $ A \cap B $.
\end{note}

\begin{eg}[]
	Any field is a semiring. 
	\begin{enumerate}[label=(\roman*)]
		\item Field (i) and (ii) 
		\item Field (ii) and (iii) 
		\item Take any $ A,B \in \mathcal{F}$ with $ A \subseteq B$, then
			\begin{align*}
				B &= A \cap  (B\setminus A) \\
				  &= A \cup (B \cap A^{c}) \in \mathcal{F} 
			\end{align*}
	\end{enumerate}
\end{eg}

\begin{eg}[]
$ \rr, \mathcal{A}=$ collection of half-open intervals of the form $ \{x: a<x\leq b\} $ for any $ a,b \in \rr$. Then $ \mathcal{A}$ is a semiring.
\begin{enumerate}[label=(\roman*)]
	\item Take $ b<a$.
	\item  $ A,B \in A \implies A = \{x: a< x \leq b\} , B = \{x: c<x\leq d\} $. Then $ a \cap B = \{x: \max(a,c)< x \leq \min(d,b)\} $ 
	\item Take $ A,B \in \mathcal{A}$ with $ A \subseteq B$, then $ A=(a,b], B = (c,d]$ with $ c\leq a<b\leq d$. It follows that $ B \setminus A = (c,a] \\cup (b,d]$ disjoint. 
\end{enumerate}
\end{eg}
\begin{eg}[]
	$ \rr^{k}$ and half-open rectangles  $ \{(x_1,\ldots,x_k): a_i<x_i\leq b_i \text{ for } i=1,\ldots,k \} $ is a semiring because any Cartesian product of semiring is a semiring.
\end{eg}

\begin{thm}[11.3]
	Suppose that $ \mu$ is a measure on a semiring $ \mathcal{A}$. Then $ \mu$ extends to a measure on the $ \sigma(\mathcal{A})$.
\end{thm}
\begin{note}[]
Theorem 11.2 is a special case of this.

Only need finite additivity of $ \mu$, as opposed to countable additivity.
\end{note}

\begin{prf}
\begin{claim}[]
$ \mu$ is monotone on $ \mathcal{A}$.
\end{claim}

We need to prove this because a semiring might not contain complements, which is what we used to prove monotonicity previously. Since $ \mathcal{A}$ is a semiring, there exist disjoint $ A_1, \ldots, A_n$ such that $ B = A \cup A_1 \cup \ldots \cup A_n$, where all these sets are in $ \mathcal{A}$. By finite additivity of $ \mu$,
\[
	\mu(B) = \mu(A) + \sum_{ i= 1}^{ n} \mu(A_n) \geq \mu(A)
.\] 

Define an outer measure, for any $ A \subseteq \Omega$. Using infimum like probability measure might not be well-defined because there might not exists $ \mathcal{A}$-covering of $ A$, since $ \mathcal{A}$ might not contain $ \Omega$, yet previously we can choose $ A_1 = \Omega, A_2 = \O, \ldots$ to cover $ A$. Therefore, we modify the definition to be
\begin{equation*}
	u^* (A)=
\begin{cases}
\infty,\qquad  \qquad  \text{ if a } \mathcal{A} \text{ covering doesn't exist for } A \\
\inf\left\{\sum_n \mu(A_n): A_n \in \mathcal{A}, A \subseteq \bigcup _{ n} A_n \right\},  \text{ otherwise} \\
\end{cases}
\end{equation*}
\begin{claim}[]
$ \mu^* $ is an outer measure.
\end{claim}
Let $ \mathcal{M}(\mu^* )$ be the class of $ \mu^* $-measurable sets.
\begin{claim}[]
	$ \mathcal{A} \subseteq \mathcal{M}(\mu^* )$.
\end{claim}
Take any $ A \in \mathcal{A}$ and $ E \in \Omega$, we want to show that
\begin{equation}
	\mu^* (A \cap E) + \mu^* (A^{c} \cap E) \leq \mu^* (E) 
\end{equation}
\begin{case}[]
	$ \mu^* (E) = \infty$. This makes (1) trivially true. 
\end{case}
\begin{case}[]
	$ \mu^* (E) < \infty$. This implies that there exists a $ \mathcal{A}$-covering of $ E$. Let  $ \epsilon>0$. Choose $ \mathcal{A}$-sets $ A_1,A_2,\ldots$ such that $ E \subseteq \bigcup_{ n} A_n $ and 
\[
	\sum_n \mu(A_n) < \mu^* (E) + \epsilon
\] 
by the definition of infimum. Since $ \mathcal{A}$ is a semiring, and $ A, A_n \in \mathcal{A}$, define $ B_n \coloneqq A \cap  A_n \in \mathcal{A} \ \forall \ n \in \nn$. Note that $ B_n \subseteq A_n$. So by definition of a semiring, there exists $ C_{n,1},C_{n,2},\ldots,C_{n,m_n}$ such that 
\[
A_n \setminus B_n = \bigcup_{ i= 1}^{m_n} C_{n,i} \ \forall \ n \in \nn 
.\]

Notice that $ A_n \cap A^{c} = A_n \setminus B_n = \bigcup_{ i= 1}^{m_n} C_{n,i} $ by how we define $ B_n$. Then $ A_n = B_n \cup \bigcup_{ i= 1}^{m_n} C_{n,i} $. Thus by finite additivity of $ \mu$,
\[
	\mu(A_n) = \mu(B_n) + \sum_{ i= 1}^{ m_n} \mu(C_{n,i}) 
.\]
Since $ E$ is covered by  $ A_n$, $ A \cap E \subseteq A \cap \bigcup_{ n} A_n = \bigcup_{ n}(A \cap A_n) = \bigcup_{ n} B_n  $, and $ A^{c} \cap E \subseteq A^{c} \cap \bigcup_{ n} A_n = \bigcup_{ n} (A^{c} \cap A_n) = \bigcup_{ n} \bigcup_{ i= 1}^{m_n} C_{n,i}    $. Notice both are $ \mathcal{A}$-covering of the sets. By definition of infimum,
\begin{align*}
	\mu^* (A \cap E) + \mu^* (A^{c} \cap E) &\leq \sum_n \mu (B_n) + \sum_n \sum_{ i= 1}^{ m_n} \mu (C_{n,i}) \\
						&= \sum_n \left( \mu(B_n) + \sum_{ i= 1}^{ m_n} \mu (C_{n,i}) \right) \\
						&= \sum_n \mu(A_n) \\
						&< \mu^* (E) + \epsilon
\end{align*}
Take $ \epsilon \to 0$, we have (1) as required. Thus, $ \mathcal{A} \subseteq  \mathcal{M}(\mu^* )$.
\end{case}

\begin{claim}[]
	$ \mu^* $ is a measure on $ \mathcal{M}(\mu^* )$
\end{claim}
by Theorem 11.1.

\begin{claim}[]
$ \mu$ and $ \mu^* $ agree on $ \mathcal{A}$.
\end{claim}

Take any $ A \in \mathcal{A}$.  

Since $ A_1 = A, A_2 = \O,\ldots$ is a $ \mathcal{A}$-covering of $ A$, $ \mu^* (A) \leq \mu(A)$ follows from $ \mu^* (A)$ being an infimum.

For the other direction, we want to show that $ \mu(A)$ is a lower bound. let $ (A_n)$ be an $ \mathcal{A}$-covering of $ A$. This is well-defined because we at least have the covering above. Since $ \mathcal{A}$ is a semiring, $ A \cap A_n \in \mathcal{A} \ \forall \ n \geq 1$. Since $ A \subseteq \bigcup_n A_n $, we have
\[
	\bigcup_{ n =1}^{\infty} (A \cap A_n) = A \cap \bigcup_{ n =1}^{\infty} A_n = A
.\]

By countable subadditivity of $ \mu$,
\[
	\mu(A) = \mu\left( \bigcup_{ n =1}^{\infty}  A \cap A_n \right) \leq \sum_{ n= 1}^{\infty} \mu(A \cap A_n) 
.\] 
Since $ \mu^* (A)$ is the infimum of the RHS, and $ \mu(A)$ is a lower bound of the RHS, we have $ \mu(A) \leq \mu*(A)$.

Since $ \mathcal{A} \subseteq \mathcal{M}(u^* )$ and $ \mathcal{M}(\mu^* )$ is a $\sigma$-field by Theorem 11.1, $ \mathcal{A} \subseteq \sigma(\mathcal{A}) \subseteq \mathcal{M}(\mu^* )$. Since $ \mu^* $ is a measure on $ \mathcal{M}(\mu^* )$, then $ \mu^* $ is a measure on $ \sigma(\mathcal{A})$ and it agrees with $ \mu$ for all sets in $ \mathcal{A}$.


\end{prf}
\end{document}

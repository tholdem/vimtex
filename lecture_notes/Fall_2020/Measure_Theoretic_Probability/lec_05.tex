\documentclass[class=article,crop=false]{standalone} 
%Fall 2020
% Some basic packages
\usepackage{standalone}[subpreambles=true]
\usepackage[utf8]{inputenc}
\usepackage[T1]{fontenc}
\usepackage{textcomp}
\usepackage[english]{babel}
\usepackage{url}
\usepackage{graphicx}
\usepackage{float}
\usepackage{enumitem}


\pdfminorversion=7

% Don't indent paragraphs, leave some space between them
\usepackage{parskip}

% Hide page number when page is empty
\usepackage{emptypage}
\usepackage{subcaption}
\usepackage{multicol}
\usepackage[dvipsnames]{xcolor}


% Math stuff
\usepackage{amsmath, amsfonts, mathtools, amsthm, amssymb}
% Fancy script capitals
\usepackage{mathrsfs}
\usepackage{cancel}
% Bold math
\usepackage{bm}
% Some shortcuts
\newcommand{\rr}{\ensuremath{\mathbb{R}}}
\newcommand{\zz}{\ensuremath{\mathbb{Z}}}
\newcommand{\qq}{\ensuremath{\mathbb{Q}}}
\newcommand{\nn}{\ensuremath{\mathbb{N}}}
\newcommand{\ff}{\ensuremath{\mathbb{F}}}
\newcommand{\cc}{\ensuremath{\mathbb{C}}}
\renewcommand\O{\ensuremath{\emptyset}}
\newcommand{\norm}[1]{{\left\lVert{#1}\right\rVert}}
\renewcommand{\vec}[1]{{\mathbf{#1}}}
\newcommand\allbold[1]{{\boldmath\textbf{#1}}}

% Put x \to \infty below \lim
\let\svlim\lim\def\lim{\svlim\limits}

%Make implies and impliedby shorter
\let\implies\Rightarrow
\let\impliedby\Leftarrow
\let\iff\Leftrightarrow
\let\epsilon\varepsilon

% Add \contra symbol to denote contradiction
\usepackage{stmaryrd} % for \lightning
\newcommand\contra{\scalebox{1.5}{$\lightning$}}

% \let\phi\varphi

% Command for short corrections
% Usage: 1+1=\correct{3}{2}

\definecolor{correct}{HTML}{009900}
\newcommand\correct[2]{\ensuremath{\:}{\color{red}{#1}}\ensuremath{\to }{\color{correct}{#2}}\ensuremath{\:}}
\newcommand\green[1]{{\color{correct}{#1}}}

% horizontal rule
\newcommand\hr{
    \noindent\rule[0.5ex]{\linewidth}{0.5pt}
}

% hide parts
\newcommand\hide[1]{}

% si unitx
\usepackage{siunitx}
\sisetup{locale = FR}

% Environments
\makeatother
% For box around Definition, Theorem, \ldots
\usepackage[framemethod=TikZ]{mdframed}
\mdfsetup{skipabove=1em,skipbelow=0em}

%definition
\newenvironment{defn}[1][]{%
\ifstrempty{#1}%
{\mdfsetup{%
frametitle={%
\tikz[baseline=(current bounding box.east),outer sep=0pt]
\node[anchor=east,rectangle,fill=Emerald]
{\strut Definition};}}
}%
{\mdfsetup{%
frametitle={%
\tikz[baseline=(current bounding box.east),outer sep=0pt]
\node[anchor=east,rectangle,fill=Emerald]
{\strut Definition:~#1};}}%
}%
\mdfsetup{innertopmargin=10pt,linecolor=Emerald,%
linewidth=2pt,topline=true,%
frametitleaboveskip=\dimexpr-\ht\strutbox\relax
}
\begin{mdframed}[]\relax%
\label{#1}}{\end{mdframed}}


%theorem
%\newcounter{thm}[section]\setcounter{thm}{0}
%\renewcommand{\thethm}{\arabic{section}.\arabic{thm}}
\newenvironment{thm}[1][]{%
%\refstepcounter{thm}%
\ifstrempty{#1}%
{\mdfsetup{%
frametitle={%
\tikz[baseline=(current bounding box.east),outer sep=0pt]
\node[anchor=east,rectangle,fill=blue!20]
%{\strut Theorem~\thethm};}}
{\strut Theorem};}}
}%
{\mdfsetup{%
frametitle={%
\tikz[baseline=(current bounding box.east),outer sep=0pt]
\node[anchor=east,rectangle,fill=blue!20]
%{\strut Theorem~\thethm:~#1};}}%
{\strut Theorem:~#1};}}%
}%
\mdfsetup{innertopmargin=10pt,linecolor=blue!20,%
linewidth=2pt,topline=true,%
frametitleaboveskip=\dimexpr-\ht\strutbox\relax
}
\begin{mdframed}[]\relax%
\label{#1}}{\end{mdframed}}


%lemma
\newenvironment{lem}[1][]{%
\ifstrempty{#1}%
{\mdfsetup{%
frametitle={%
\tikz[baseline=(current bounding box.east),outer sep=0pt]
\node[anchor=east,rectangle,fill=Dandelion]
{\strut Lemma};}}
}%
{\mdfsetup{%
frametitle={%
\tikz[baseline=(current bounding box.east),outer sep=0pt]
\node[anchor=east,rectangle,fill=Dandelion]
{\strut Lemma:~#1};}}%
}%
\mdfsetup{innertopmargin=10pt,linecolor=Dandelion,%
linewidth=2pt,topline=true,%
frametitleaboveskip=\dimexpr-\ht\strutbox\relax
}
\begin{mdframed}[]\relax%
\label{#1}}{\end{mdframed}}

%corollary
\newenvironment{coro}[1][]{%
\ifstrempty{#1}%
{\mdfsetup{%
frametitle={%
\tikz[baseline=(current bounding box.east),outer sep=0pt]
\node[anchor=east,rectangle,fill=CornflowerBlue]
{\strut Corollary};}}
}%
{\mdfsetup{%
frametitle={%
\tikz[baseline=(current bounding box.east),outer sep=0pt]
\node[anchor=east,rectangle,fill=CornflowerBlue]
{\strut Corollary:~#1};}}%
}%
\mdfsetup{innertopmargin=10pt,linecolor=CornflowerBlue,%
linewidth=2pt,topline=true,%
frametitleaboveskip=\dimexpr-\ht\strutbox\relax
}
\begin{mdframed}[]\relax%
\label{#1}}{\end{mdframed}}

%proof
\newenvironment{prf}[1][]{%
\ifstrempty{#1}%
{\mdfsetup{%
frametitle={%
\tikz[baseline=(current bounding box.east),outer sep=0pt]
\node[anchor=east,rectangle,fill=SpringGreen]
{\strut Proof};}}
}%
{\mdfsetup{%
frametitle={%
\tikz[baseline=(current bounding box.east),outer sep=0pt]
\node[anchor=east,rectangle,fill=SpringGreen]
{\strut Proof:~#1};}}%
}%
\mdfsetup{innertopmargin=10pt,linecolor=SpringGreen,%
linewidth=2pt,topline=true,%
frametitleaboveskip=\dimexpr-\ht\strutbox\relax
}
\begin{mdframed}[]\relax%
\label{#1}}{\qed\end{mdframed}}


\theoremstyle{definition}

\newmdtheoremenv[nobreak=true]{definition}{Definition}
\newmdtheoremenv[nobreak=true]{prop}{Proposition}
\newmdtheoremenv[nobreak=true]{theorem}{Theorem}
\newmdtheoremenv[nobreak=true]{corollary}{Corollary}
\newtheorem*{eg}{Example}
\theoremstyle{remark}
\newtheorem*{case}{Case}
\newtheorem*{notation}{Notation}
\newtheorem*{remark}{Remark}
\newtheorem*{note}{Note}
\newtheorem*{problem}{Problem}
\newtheorem*{observe}{Observe}
\newtheorem*{property}{Property}
\newtheorem*{intuition}{Intuition}


% End example and intermezzo environments with a small diamond (just like proof
% environments end with a small square)
\usepackage{etoolbox}
\AtEndEnvironment{vb}{\null\hfill$\diamond$}%
\AtEndEnvironment{intermezzo}{\null\hfill$\diamond$}%
% \AtEndEnvironment{opmerking}{\null\hfill$\diamond$}%

% Fix some spacing
% http://tex.stackexchange.com/questions/22119/how-can-i-change-the-spacing-before-theorems-with-amsthm
\makeatletter
\def\thm@space@setup{%
  \thm@preskip=\parskip \thm@postskip=0pt
}

% Fix some stuff
% %http://tex.stackexchange.com/questions/76273/multiple-pdfs-with-page-group-included-in-a-single-page-warning
\pdfsuppresswarningpagegroup=1


% My name
\author{Jaden Wang}



\begin{document}
\begin{defn}[]
If $A_1 \subseteq A_2 \subseteq A_3 \subseteq  \ldots$ and $A = \bigcup_{ n =1}^{\infty} A_n$, we write $A_n \uparrow A$. 
\end{defn}

\begin{defn}[]
If $ A_1 \supseteq A_2 \supseteq \ldots$ and $ A= \bigcap_{ n=1}^{\infty} A_n $, we write $A_n \downarrow A$.
\end{defn}

\begin{eg}[]
Given $A_1,A_2,\ldots \in \mathcal{F}$, consider $\bigcup_{ k =1}^{\infty} A_k, \bigcup_{ k =2}^{\infty} A_k \ldots  $ as $n \to \infty$, 
\[
\bigcup_{ k =n}^{\infty} A_k \downarrow \bigcap_{ m=1}^{\infty} \bigcup_{k=m}^{\infty} A_m = \limsup_{ n} A_n
.\] 
Also,
\[
\bigcap_{ k =n}^{\infty} A_k \uparrow \bigcup_{ m=1}^{\infty} \bigcap_{k=m}^{\infty} A_m = \liminf_{ n} A_n
.\] 
\end{eg}

\begin{note}[]
\[
\liminf_{ n} A_n \subseteq \limsup_{ n} A_n
.\] 
"("all but finitely many $A_n$") $\subseteq $ ("infinitely many of the $A_n$")" (this is not a proof).
\end{note}
\begin{prf}
	Take $\omega \in \liminf_{ n} A_n = \bigcup_{ n= 1}^{\infty} \bigcap_{k= n}^{\infty} A_k \implies \omega \in \bigcap_{ k =n}^{\infty} A_k$ for at least one $n$. Then there exists a  $N \geq 1$  such that $\omega \in A_N, A_{N+1},\ldots \implies \omega \in \bigcup_{ k =1}^{\infty} A_k, \omega \in \bigcup_{ k =2}^{\infty} A_k,\ldots$. Hence, it's in all of them (the intersection) so it's in limsup.
\end{prf}

\begin{defn}[common value]
If $\liminf_{ n} A_n = \limsup_{ n} A_n$, define $\lim_n A_n$ to be the \allbold{common value} .
\end{defn}

\begin{lem}[]
\[
	\left( \limsup_{ n} A_n \right)^{c} =  \liminf_{ n} A_n^{c}
.\] 
by De Morgan's law.
\end{lem}


In probability, "sets" represent "events". liminf and limsup are also "events".
\begin{enumerate}[label=\arabic*)]
	\item $\limsup_{ n} A_n =$ "the event that infinitely many of the events $A_n$ occur" / "$A_n$ occurs infinitely often" / "$A_n$ i.o.".
		\[
			P(A_n \text{i.o.} ) = P(\limsup_{ n} A_n)
		.\] 
	\item $\liminf_{ n} A_n = $ "$A_n$ occurs almost always"/"$A_n$ a.a.".
\end{enumerate}

\begin{eg}[]
Let $A_n = \{0,\frac{1}{n}, \frac{2}{n},\ldots, \frac{n^2}{n}\} $. So for example $A_2 \not \subseteq A_3$.\\

Claim: $\limsup_{ n} A_n = \qq_+$.
\begin{prf}
	Clearly $\limsup_{ n} A_n \subseteq \qq_+$. On the other hand, take any positive rational $\frac{a}{b}$. Assume $b \neq 0$,  $a,b$ are non-negative integers, and are coprime (have no common factors). 
\begin{case}[1]
$a=0 \implies \frac{a}{b} = 0 \implies \frac{a}{b} \in A_n \quad \forall n \geq 1$.
\end{case}
\begin{case}[2]
$b=1 \implies \frac{a}{b}=a \implies \frac{a}{b} =a \in A_n \quad \forall n \geq a$.
\end{case}
\begin{case}[3]
Otherwise, in order for $\frac{a}{b}$ to be in $A_n$, choose $n$ large enough, so $n$ has to be a multiple of $b$, \emph{i.e.} $n = kb$. Thus,  $\frac{a}{b} = \frac{ka}{kb} = \frac{ka}{n}$. To get $\frac{a}{b} \in A_n$, we need $ka \in \{0,1,2,\ldots,n^2\} $. That is, need $ a \in \{ 0, \frac{1}{k} , \frac{2}{k},\ldots, \frac{n^2}{k}\} $. This will happen if $\frac{n^2}{k}\geq a \implies n\geq \sqrt{ka} \implies \frac{a}{b} \in A_n$ for infinitely many $n > \sqrt{ka}$. So, any positive rational $\frac{a}{b}$ is in all $A_n$ for $n$ large enough. 
\[
\frac{a}{b} \in \limsup_{ n} A_n = \bigcap_{ n= 1}^{\infty} \bigcup_{k= n}^{\infty} A_k
.\] 
\end{case}
\end{prf}
\end{eg}

\begin{eg}[]
Claim: $\liminf_{ n} A_n = nn$.
\end{eg}

\begin{thm}[continuity of probabilities]
\begin{enumerate}[label=(\roman*)]
	\item if $A_n \uparrow A$, then $\lim_{ n \to \infty} P(A_n) = P(A)$.
	\item if $A_n \downarrow A$, then $\lim_{ n \to \infty} P(A_n) =P(A)$.
\end{enumerate}
\end{thm}

\begin{prf}
	For (i), rewrite  $A$ as a disjoint union,  $A = \bigcup_{ n =1}^{\infty} B_n$ where $B_1=A_1, B_2 = A_2 \setminus A_1 = A_2 \cap A_1^{c} $. Then
	\begin{align*}
		P(A) &= P(\bigcup_{ n =1}^{\infty} B_n)\\
		     &= \sum_{ n=1}^{\infty} P(B_n) \qquad \text{ by countable additivity} \\
		     &= \lim_{ m \to \infty} \sum_{ n=1}^{ m} P(B_n) \\
		     &= \lim_{ m \to \infty} P\left( \bigcup_{ n=1}^{m} B_n \right)  \\
		     &= \lim_{ m \to \infty} P\left( \bigcup_{ n=1}^{m} A_n \right)  \\
		     &= \lim_{ m \to \infty} P(A_m) 
	\end{align*}
	Since $ A_1 \subseteq A_2 \subseteq \ldots$.
\end{prf}

\begin{thm}[4.1]
\begin{enumerate}[label=(\roman*)]
	\item For any sequence $(A_n) \subseteq \mathcal{F}$. 
		\[
			P\left( \liminf_{ n} A_n \right) \leq \liminf_{ n} P(A_n) \leq \limsup_{ n} P(A_n) \leq P(\limsup_{ n} A_n)
		.\] 
	\item If $\lim_{ n \to \infty} A_n= A$, then
		\[
			\lim_{ n \to \infty} P(A_n) = P(A)
		.\] 
\end{enumerate}
\end{thm}
\begin{defn}[]
	If $(x_n)$ is a sequence in $\rr$,
	\[
		\liminf_{ n} x_n = \lim_{ n \to \infty} \left( \inf_{k\geq n} x_k \right) 
	.\] 
	\[
		\limsup_{ n} x_n = \lim_{ n \to \infty} \left( \sup_{k\geq n} x_k \right) 
	.\] 
\end{defn}

\begin{prf}
	(ii) follows from (i) by the "squeeze theorem".\\

(i): let $B_n = \bigcap_{ k =n}^{\infty} A_k$, $B = \bigcup_{ n =1}^{\infty} B_n$. So $B_1 \subseteq  B_2 \subseteq \ldots$ and $B_n \uparrow B$.
\begin{align*}
	P(\liminf_{ n} A_n) &= P\left( \bigcup_{ n= 1}^{\infty} \bigcap_{k= n}^{\infty} A_n \right)  \\
			    &= P(B) \\
			    &= \lim_{ n \to \infty} P(B_n) \text{ by continuity of probabilities}  \\
			    &= \lim_{ n \to \infty} P\left( \bigcap_{ k =n}^{\infty}A_k  \right)  \\
			    &= \liminf_{ n} P\left( \bigcap_{ k =n}^{\infty} A_k \right)  \\
			    &\leq \liminf_{ n} P(A_n)  \text{ by monotonicity of } P 
\end{align*}
\end{prf}

\begin{defn}[independent events]
	Let $(\Omega,\mathcal{F},P)$ be a probability space. Let $(A_n) \subseteq \mathcal{F}$. Then $A_1,A_2,\ldots$ are mutually independent if for any $j \in \{2,3,\ldots,n\} $ and any indices $1\leq k_1 < \ldots < k_j \leq n$,
	\[
		P(A_{k_1} \cap A_{k_2} \cap \ldots \cap A_{k_j}) = P(A_{k_1}) P(A_{k_2}) \ldots P(A_{k_j})
	.\] 
\end{defn}
\end{document}

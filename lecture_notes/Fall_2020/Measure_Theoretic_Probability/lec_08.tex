\documentclass[class=article,crop=false]{standalone} 
%Fall 2020
% Some basic packages
\usepackage{standalone}[subpreambles=true]
\usepackage[utf8]{inputenc}
\usepackage[T1]{fontenc}
\usepackage{textcomp}
\usepackage[english]{babel}
\usepackage{url}
\usepackage{graphicx}
\usepackage{float}
\usepackage{enumitem}


\pdfminorversion=7

% Don't indent paragraphs, leave some space between them
\usepackage{parskip}

% Hide page number when page is empty
\usepackage{emptypage}
\usepackage{subcaption}
\usepackage{multicol}
\usepackage[dvipsnames]{xcolor}


% Math stuff
\usepackage{amsmath, amsfonts, mathtools, amsthm, amssymb}
% Fancy script capitals
\usepackage{mathrsfs}
\usepackage{cancel}
% Bold math
\usepackage{bm}
% Some shortcuts
\newcommand{\rr}{\ensuremath{\mathbb{R}}}
\newcommand{\zz}{\ensuremath{\mathbb{Z}}}
\newcommand{\qq}{\ensuremath{\mathbb{Q}}}
\newcommand{\nn}{\ensuremath{\mathbb{N}}}
\newcommand{\ff}{\ensuremath{\mathbb{F}}}
\newcommand{\cc}{\ensuremath{\mathbb{C}}}
\renewcommand\O{\ensuremath{\emptyset}}
\newcommand{\norm}[1]{{\left\lVert{#1}\right\rVert}}
\renewcommand{\vec}[1]{{\mathbf{#1}}}
\newcommand\allbold[1]{{\boldmath\textbf{#1}}}

% Put x \to \infty below \lim
\let\svlim\lim\def\lim{\svlim\limits}

%Make implies and impliedby shorter
\let\implies\Rightarrow
\let\impliedby\Leftarrow
\let\iff\Leftrightarrow
\let\epsilon\varepsilon

% Add \contra symbol to denote contradiction
\usepackage{stmaryrd} % for \lightning
\newcommand\contra{\scalebox{1.5}{$\lightning$}}

% \let\phi\varphi

% Command for short corrections
% Usage: 1+1=\correct{3}{2}

\definecolor{correct}{HTML}{009900}
\newcommand\correct[2]{\ensuremath{\:}{\color{red}{#1}}\ensuremath{\to }{\color{correct}{#2}}\ensuremath{\:}}
\newcommand\green[1]{{\color{correct}{#1}}}

% horizontal rule
\newcommand\hr{
    \noindent\rule[0.5ex]{\linewidth}{0.5pt}
}

% hide parts
\newcommand\hide[1]{}

% si unitx
\usepackage{siunitx}
\sisetup{locale = FR}

% Environments
\makeatother
% For box around Definition, Theorem, \ldots
\usepackage[framemethod=TikZ]{mdframed}
\mdfsetup{skipabove=1em,skipbelow=0em}

%definition
\newenvironment{defn}[1][]{%
\ifstrempty{#1}%
{\mdfsetup{%
frametitle={%
\tikz[baseline=(current bounding box.east),outer sep=0pt]
\node[anchor=east,rectangle,fill=Emerald]
{\strut Definition};}}
}%
{\mdfsetup{%
frametitle={%
\tikz[baseline=(current bounding box.east),outer sep=0pt]
\node[anchor=east,rectangle,fill=Emerald]
{\strut Definition:~#1};}}%
}%
\mdfsetup{innertopmargin=10pt,linecolor=Emerald,%
linewidth=2pt,topline=true,%
frametitleaboveskip=\dimexpr-\ht\strutbox\relax
}
\begin{mdframed}[]\relax%
\label{#1}}{\end{mdframed}}


%theorem
%\newcounter{thm}[section]\setcounter{thm}{0}
%\renewcommand{\thethm}{\arabic{section}.\arabic{thm}}
\newenvironment{thm}[1][]{%
%\refstepcounter{thm}%
\ifstrempty{#1}%
{\mdfsetup{%
frametitle={%
\tikz[baseline=(current bounding box.east),outer sep=0pt]
\node[anchor=east,rectangle,fill=blue!20]
%{\strut Theorem~\thethm};}}
{\strut Theorem};}}
}%
{\mdfsetup{%
frametitle={%
\tikz[baseline=(current bounding box.east),outer sep=0pt]
\node[anchor=east,rectangle,fill=blue!20]
%{\strut Theorem~\thethm:~#1};}}%
{\strut Theorem:~#1};}}%
}%
\mdfsetup{innertopmargin=10pt,linecolor=blue!20,%
linewidth=2pt,topline=true,%
frametitleaboveskip=\dimexpr-\ht\strutbox\relax
}
\begin{mdframed}[]\relax%
\label{#1}}{\end{mdframed}}


%lemma
\newenvironment{lem}[1][]{%
\ifstrempty{#1}%
{\mdfsetup{%
frametitle={%
\tikz[baseline=(current bounding box.east),outer sep=0pt]
\node[anchor=east,rectangle,fill=Dandelion]
{\strut Lemma};}}
}%
{\mdfsetup{%
frametitle={%
\tikz[baseline=(current bounding box.east),outer sep=0pt]
\node[anchor=east,rectangle,fill=Dandelion]
{\strut Lemma:~#1};}}%
}%
\mdfsetup{innertopmargin=10pt,linecolor=Dandelion,%
linewidth=2pt,topline=true,%
frametitleaboveskip=\dimexpr-\ht\strutbox\relax
}
\begin{mdframed}[]\relax%
\label{#1}}{\end{mdframed}}

%corollary
\newenvironment{coro}[1][]{%
\ifstrempty{#1}%
{\mdfsetup{%
frametitle={%
\tikz[baseline=(current bounding box.east),outer sep=0pt]
\node[anchor=east,rectangle,fill=CornflowerBlue]
{\strut Corollary};}}
}%
{\mdfsetup{%
frametitle={%
\tikz[baseline=(current bounding box.east),outer sep=0pt]
\node[anchor=east,rectangle,fill=CornflowerBlue]
{\strut Corollary:~#1};}}%
}%
\mdfsetup{innertopmargin=10pt,linecolor=CornflowerBlue,%
linewidth=2pt,topline=true,%
frametitleaboveskip=\dimexpr-\ht\strutbox\relax
}
\begin{mdframed}[]\relax%
\label{#1}}{\end{mdframed}}

%proof
\newenvironment{prf}[1][]{%
\ifstrempty{#1}%
{\mdfsetup{%
frametitle={%
\tikz[baseline=(current bounding box.east),outer sep=0pt]
\node[anchor=east,rectangle,fill=SpringGreen]
{\strut Proof};}}
}%
{\mdfsetup{%
frametitle={%
\tikz[baseline=(current bounding box.east),outer sep=0pt]
\node[anchor=east,rectangle,fill=SpringGreen]
{\strut Proof:~#1};}}%
}%
\mdfsetup{innertopmargin=10pt,linecolor=SpringGreen,%
linewidth=2pt,topline=true,%
frametitleaboveskip=\dimexpr-\ht\strutbox\relax
}
\begin{mdframed}[]\relax%
\label{#1}}{\qed\end{mdframed}}


\theoremstyle{definition}

\newmdtheoremenv[nobreak=true]{definition}{Definition}
\newmdtheoremenv[nobreak=true]{prop}{Proposition}
\newmdtheoremenv[nobreak=true]{theorem}{Theorem}
\newmdtheoremenv[nobreak=true]{corollary}{Corollary}
\newtheorem*{eg}{Example}
\theoremstyle{remark}
\newtheorem*{case}{Case}
\newtheorem*{notation}{Notation}
\newtheorem*{remark}{Remark}
\newtheorem*{note}{Note}
\newtheorem*{problem}{Problem}
\newtheorem*{observe}{Observe}
\newtheorem*{property}{Property}
\newtheorem*{intuition}{Intuition}


% End example and intermezzo environments with a small diamond (just like proof
% environments end with a small square)
\usepackage{etoolbox}
\AtEndEnvironment{vb}{\null\hfill$\diamond$}%
\AtEndEnvironment{intermezzo}{\null\hfill$\diamond$}%
% \AtEndEnvironment{opmerking}{\null\hfill$\diamond$}%

% Fix some spacing
% http://tex.stackexchange.com/questions/22119/how-can-i-change-the-spacing-before-theorems-with-amsthm
\makeatletter
\def\thm@space@setup{%
  \thm@preskip=\parskip \thm@postskip=0pt
}

% Fix some stuff
% %http://tex.stackexchange.com/questions/76273/multiple-pdfs-with-page-group-included-in-a-single-page-warning
\pdfsuppresswarningpagegroup=1


% My name
\author{Jaden Wang}



\begin{document}
\begin{thm}[Kolmogoro's 0-1 Theorem]
	Suppose $ A_1,\ldots$ are independent events. For any $ A$ in the generated tail  $\sigma$-field, we have either $ P(A) =0$ or  $ P(A)=1$.
\end{thm}
\section{Simple Random Variables}
\begin{defn}[simple random variable]
Let $ (\Omega,\mathcal{F},P)$ be a probability space. A function $ X:\Omega \to \rr$ is a \allbold{simple random variable} if 
	\begin{enumerate}[label=(\roman*)]
		\item it assumes finitely many values (finite range).
		\item For any $ x \in \rr$, 
			\[
				\{X=x\}\coloneqq \{\omega: X(\omega)=x\} \in \mathcal{F}  
			.\]
			"The inverse image of $ x$ under  $ X$ is in  $ \mathcal{F}$". We say that "$ X$ is measurable wrt  $ \mathcal{F}$", or $ X$ is $ \mathcal{F}$-measurable.
	\end{enumerate}
\end{defn}

\begin{defn}[indicator random variables]
	Let $ A$ be a set. The \allbold{indicator random variable}
\begin{equation*}
	I_A(\omega)=
\begin{cases}
	1 &\text{ if } \omega \in A\\
	0 & \text{ if } \omega \not\in A 
\end{cases}
\end{equation*}
\end{defn}

\begin{thm}
Suppose $ A_1,A_2,\ldots,A_n \in \mathcal{F}$ form a finite partition of $ \Omega$. Define
\[
	X(\omega)=\sum_{ i= 1}^{ n} a_i I_{A_i} (\omega)
.\] 
for some fixed $ a_1,\ldots,a_n \in \rr$. Then $ X$ is a simple r.v.
\end{thm}
\begin{note}[]
Every simple r.v. can be written in this way.
\end{note}

\begin{thm}[]
	An indicator r.v. $ X(\omega) = I_A(\omega)$ is measurable wrt $ \mathcal{F} \implies A \in \mathcal{F}$.
\end{thm}

\begin{defn}[]
	$ (\Omega,\mathcal{F},P)$, $ X$ a simple r.v. on  $ (\Omega,\mathcal{F},P)$. Let $ \mathcal{ G} \subseteq \mathcal{F}$ be another $\sigma$-field. We say that \allbold{$ X$ is measurable wrt  $ \mathcal{ G}$} if for any $ x \in \rr$, $ \{w:X(\omega)=x\} \in \mathcal{ G} $ .
\end{defn}

\begin{note}[]
$ \mathcal{ G} \subseteq \mathcal{F}$. If $ X$ is measurable wrt  $ \mathcal{ G}$. Then for any $ H \subseteq \rr$,
\[
	\{\omega:X(\omega) \in H\} = \bigcup_{x \in H} \{\omega: X(\omega) = x\} \in \mathcal{ G}
.\] 
which might be an uncountable union unless $ X$ is simple so it has finite range.
\end{note}

\begin{defn}[the sigma-field generated by $ X$]
	The \allbold{ $\sigma$-field generated by $ X$}, denoted $ \sigma(X)$, is the smallest $\sigma$-field wrt which $ X$ is measurable. 
\end{defn}
\begin{note}[]
It is the intersection of a $\sigma$-field wrt which $ X$ is measurable.
\end{note}

\begin{defn}[sigma-field generated by a sequence of r.v.]
	For a finite or infinite sequence $ X_1,X_2,\ldots$ of simple r.v., $\sigma(X_1,X_2,\ldots)$ is the smallest $\sigma$-field wrt which each $ X_i$ is measurable.
\end{defn}
\begin{thm}[5.1]
	Let $ X_1,\ldots,X_n$ be a finite sequence of simple r.v.s.
	\begin{enumerate}[label=(\roman*)]
		\item The $\sigma$-field, $ \sigma(X_1,\ldots,X_n)$, consists of all subsets of $ \Omega$ of the form 
			\[
				\{\omega:(X_1(\omega),\ldots,X_n(\omega)) \in H\} \text{ for any } H \subseteq \rr^{n} 
			.\] 
		\item A simple r.v. $ Y$ is measurable wrt  $ \sigma(X_1,\ldots,X_n)$ if and only if $ Y=f(X_1,\ldots,X_n)$ for some $ f: \rr^{n} \to \rr$.
	\end{enumerate}
\end{thm}
\end{document}

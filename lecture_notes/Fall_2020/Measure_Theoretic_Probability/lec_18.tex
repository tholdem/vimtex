\documentclass[class=article,crop=false]{standalone} 
%Fall 2020
% Some basic packages
\usepackage{standalone}[subpreambles=true]
\usepackage[utf8]{inputenc}
\usepackage[T1]{fontenc}
\usepackage{textcomp}
\usepackage[english]{babel}
\usepackage{url}
\usepackage{graphicx}
\usepackage{float}
\usepackage{enumitem}


\pdfminorversion=7

% Don't indent paragraphs, leave some space between them
\usepackage{parskip}

% Hide page number when page is empty
\usepackage{emptypage}
\usepackage{subcaption}
\usepackage{multicol}
\usepackage[dvipsnames]{xcolor}


% Math stuff
\usepackage{amsmath, amsfonts, mathtools, amsthm, amssymb}
% Fancy script capitals
\usepackage{mathrsfs}
\usepackage{cancel}
% Bold math
\usepackage{bm}
% Some shortcuts
\newcommand{\rr}{\ensuremath{\mathbb{R}}}
\newcommand{\zz}{\ensuremath{\mathbb{Z}}}
\newcommand{\qq}{\ensuremath{\mathbb{Q}}}
\newcommand{\nn}{\ensuremath{\mathbb{N}}}
\newcommand{\ff}{\ensuremath{\mathbb{F}}}
\newcommand{\cc}{\ensuremath{\mathbb{C}}}
\renewcommand\O{\ensuremath{\emptyset}}
\newcommand{\norm}[1]{{\left\lVert{#1}\right\rVert}}
\renewcommand{\vec}[1]{{\mathbf{#1}}}
\newcommand\allbold[1]{{\boldmath\textbf{#1}}}

% Put x \to \infty below \lim
\let\svlim\lim\def\lim{\svlim\limits}

%Make implies and impliedby shorter
\let\implies\Rightarrow
\let\impliedby\Leftarrow
\let\iff\Leftrightarrow
\let\epsilon\varepsilon

% Add \contra symbol to denote contradiction
\usepackage{stmaryrd} % for \lightning
\newcommand\contra{\scalebox{1.5}{$\lightning$}}

% \let\phi\varphi

% Command for short corrections
% Usage: 1+1=\correct{3}{2}

\definecolor{correct}{HTML}{009900}
\newcommand\correct[2]{\ensuremath{\:}{\color{red}{#1}}\ensuremath{\to }{\color{correct}{#2}}\ensuremath{\:}}
\newcommand\green[1]{{\color{correct}{#1}}}

% horizontal rule
\newcommand\hr{
    \noindent\rule[0.5ex]{\linewidth}{0.5pt}
}

% hide parts
\newcommand\hide[1]{}

% si unitx
\usepackage{siunitx}
\sisetup{locale = FR}

% Environments
\makeatother
% For box around Definition, Theorem, \ldots
\usepackage[framemethod=TikZ]{mdframed}
\mdfsetup{skipabove=1em,skipbelow=0em}

%definition
\newenvironment{defn}[1][]{%
\ifstrempty{#1}%
{\mdfsetup{%
frametitle={%
\tikz[baseline=(current bounding box.east),outer sep=0pt]
\node[anchor=east,rectangle,fill=Emerald]
{\strut Definition};}}
}%
{\mdfsetup{%
frametitle={%
\tikz[baseline=(current bounding box.east),outer sep=0pt]
\node[anchor=east,rectangle,fill=Emerald]
{\strut Definition:~#1};}}%
}%
\mdfsetup{innertopmargin=10pt,linecolor=Emerald,%
linewidth=2pt,topline=true,%
frametitleaboveskip=\dimexpr-\ht\strutbox\relax
}
\begin{mdframed}[]\relax%
\label{#1}}{\end{mdframed}}


%theorem
%\newcounter{thm}[section]\setcounter{thm}{0}
%\renewcommand{\thethm}{\arabic{section}.\arabic{thm}}
\newenvironment{thm}[1][]{%
%\refstepcounter{thm}%
\ifstrempty{#1}%
{\mdfsetup{%
frametitle={%
\tikz[baseline=(current bounding box.east),outer sep=0pt]
\node[anchor=east,rectangle,fill=blue!20]
%{\strut Theorem~\thethm};}}
{\strut Theorem};}}
}%
{\mdfsetup{%
frametitle={%
\tikz[baseline=(current bounding box.east),outer sep=0pt]
\node[anchor=east,rectangle,fill=blue!20]
%{\strut Theorem~\thethm:~#1};}}%
{\strut Theorem:~#1};}}%
}%
\mdfsetup{innertopmargin=10pt,linecolor=blue!20,%
linewidth=2pt,topline=true,%
frametitleaboveskip=\dimexpr-\ht\strutbox\relax
}
\begin{mdframed}[]\relax%
\label{#1}}{\end{mdframed}}


%lemma
\newenvironment{lem}[1][]{%
\ifstrempty{#1}%
{\mdfsetup{%
frametitle={%
\tikz[baseline=(current bounding box.east),outer sep=0pt]
\node[anchor=east,rectangle,fill=Dandelion]
{\strut Lemma};}}
}%
{\mdfsetup{%
frametitle={%
\tikz[baseline=(current bounding box.east),outer sep=0pt]
\node[anchor=east,rectangle,fill=Dandelion]
{\strut Lemma:~#1};}}%
}%
\mdfsetup{innertopmargin=10pt,linecolor=Dandelion,%
linewidth=2pt,topline=true,%
frametitleaboveskip=\dimexpr-\ht\strutbox\relax
}
\begin{mdframed}[]\relax%
\label{#1}}{\end{mdframed}}

%corollary
\newenvironment{coro}[1][]{%
\ifstrempty{#1}%
{\mdfsetup{%
frametitle={%
\tikz[baseline=(current bounding box.east),outer sep=0pt]
\node[anchor=east,rectangle,fill=CornflowerBlue]
{\strut Corollary};}}
}%
{\mdfsetup{%
frametitle={%
\tikz[baseline=(current bounding box.east),outer sep=0pt]
\node[anchor=east,rectangle,fill=CornflowerBlue]
{\strut Corollary:~#1};}}%
}%
\mdfsetup{innertopmargin=10pt,linecolor=CornflowerBlue,%
linewidth=2pt,topline=true,%
frametitleaboveskip=\dimexpr-\ht\strutbox\relax
}
\begin{mdframed}[]\relax%
\label{#1}}{\end{mdframed}}

%proof
\newenvironment{prf}[1][]{%
\ifstrempty{#1}%
{\mdfsetup{%
frametitle={%
\tikz[baseline=(current bounding box.east),outer sep=0pt]
\node[anchor=east,rectangle,fill=SpringGreen]
{\strut Proof};}}
}%
{\mdfsetup{%
frametitle={%
\tikz[baseline=(current bounding box.east),outer sep=0pt]
\node[anchor=east,rectangle,fill=SpringGreen]
{\strut Proof:~#1};}}%
}%
\mdfsetup{innertopmargin=10pt,linecolor=SpringGreen,%
linewidth=2pt,topline=true,%
frametitleaboveskip=\dimexpr-\ht\strutbox\relax
}
\begin{mdframed}[]\relax%
\label{#1}}{\qed\end{mdframed}}


\theoremstyle{definition}

\newmdtheoremenv[nobreak=true]{definition}{Definition}
\newmdtheoremenv[nobreak=true]{prop}{Proposition}
\newmdtheoremenv[nobreak=true]{theorem}{Theorem}
\newmdtheoremenv[nobreak=true]{corollary}{Corollary}
\newtheorem*{eg}{Example}
\theoremstyle{remark}
\newtheorem*{case}{Case}
\newtheorem*{notation}{Notation}
\newtheorem*{remark}{Remark}
\newtheorem*{note}{Note}
\newtheorem*{problem}{Problem}
\newtheorem*{observe}{Observe}
\newtheorem*{property}{Property}
\newtheorem*{intuition}{Intuition}


% End example and intermezzo environments with a small diamond (just like proof
% environments end with a small square)
\usepackage{etoolbox}
\AtEndEnvironment{vb}{\null\hfill$\diamond$}%
\AtEndEnvironment{intermezzo}{\null\hfill$\diamond$}%
% \AtEndEnvironment{opmerking}{\null\hfill$\diamond$}%

% Fix some spacing
% http://tex.stackexchange.com/questions/22119/how-can-i-change-the-spacing-before-theorems-with-amsthm
\makeatletter
\def\thm@space@setup{%
  \thm@preskip=\parskip \thm@postskip=0pt
}

% Fix some stuff
% %http://tex.stackexchange.com/questions/76273/multiple-pdfs-with-page-group-included-in-a-single-page-warning
\pdfsuppresswarningpagegroup=1


% My name
\author{Jaden Wang}



\begin{document}
\begin{defn}[simple function]
	$ f: \Omega \to \rr$ is a \allbold{simple function} if it can be written as
	\[
		f(\omega)=\sum_{ i= 1}^{ n} x_i I_{A_i}(\omega)
	\] 
	where $ A_1,\ldots,A_n$ form a partition of $ \Omega$ (assuming disjoint).
\end{defn}

\begin{prop}[]
$ f: \Omega \to \rr$ is measurable iff $ A_i \in \mathcal{F}$ for $ i=1,2,\ldots,n$.
\end{prop}

\begin{thm}[]
	If $ f: \Omega \to \rr$ is measurable, there exists a sequence $ (f_n)$ of simple measurable functions such that 
	\begin{align*}
		0\leq f_n(\omega) \nearrow f(\omega) & \text{ if } f(\omega)\geq 0\\
		0\geq f_n(\omega) \searrow f(\omega) &\text{ if } f(\omega)<0\\ 
	\end{align*}
\end{thm}

\begin{prf}
Break up the y-axis at $ 0,\frac{1}{2^{n}}, \ldots,\frac{n 2^{n}}{2^{n} }$ and the negatives of all these. Define
\begin{equation*}
	f_n(\omega)=
\begin{cases}
	-n &\text{ if } f(\omega)\leq - n\\
	-(k-1)2^{-n} & \text{ if } -k2^{-n}\leq f(\omega) \leq -(k-1)2^{-n} \\
	(k-1)2^{-n} & \text{ if } (k-1)2^{-n} \leq f(\omega) \leq k 2^{-n}\\
	n &  \text{ if } f(\omega)\geq n 
\end{cases}
\end{equation*}
Clearly $ f_n$ are simple functions that are measurable since $ f$ is measurable and the interval is a Borel set so the inverse image is in $ \mathcal{F}$. 
\end{prf}
\begin{intuition}
	The idea is that $ f_n(\omega)$ will expand until its range totally covers $ f(\omega)$. Then after that the intervals will just get refined as $ n \to \infty$. 
\end{intuition}
\subsection*{Transformations of Measures}

~\begin{thm}[]
	$ (\Omega,\mathcal{F}),(\Omega',\mathcal{F}')$, $ T: \Omega \to \Omega'$ measurable $ F / F'$. Let  $ \mu$ be a measure on $ \mathcal{F}$. Define a function for $ \mathcal{F}'$ to $ \rr$, call it $ \mu T^{-1}$, as
	\[
		\mu T^{-1}(A') = \mu(T^{-1}(A')) \ \forall \ A' \in \mathcal{F}'
	.\] 
	Then $ \mu T^{-1}$ is a measure on $ \mathcal{F}'$.
\end{thm}

\begin{note}[]
This is the measure-theoretic version of probability distribution transformation using the Jacobian.
\end{note}

\section*{14: Distribution Functions}

Let $ (\Omega,\mathcal{F},P)$ be a probability space and let $ X: \Omega \to \rr$ be a r.v.

\begin{defn}[distribution]
	The \allbold{distribution} (law) of $ X$ is the probability measure $ P_X$, defined as
	\[
		P_X(A)=P(X \in A)= P(\{\omega: X(\omega) \in A\}) 
	.\]
\end{defn}

\begin{defn}[distribution function]
The \allbold{distribution function} for $ X$ is denoted/defined as 
 \[
	 F_X(x) = P_X ((-\infty,x]) = P(X\leq x)
.\] 
\end{defn}

\begin{property}[$ F_X$]
~\begin{enumerate}[label=\arabic*)]
	\item Non-decreasing: from monotonicity of $ P$ or  $ P_X$.
	\item Right-continuous: from continuity from above for  $ P_X$.
	\item  $ \lim_{ x \to -\infty} F_X(x)=0, \lim_{ x \to \infty} F_X(x)=1 $.
	\item $ F_X$ can have, at most, countably many points of discontinuity.
\end{enumerate}

\begin{prop}[]
	If a measure $ \mu$ is $\sigma$-finite on a space $ (\Omega,\mathcal{F})$, then $ \mathcal{F}$ cannot contain an uncountable collection of disjoint sets with positive $ \mu$ measure.
\end{prop}
\begin{intuition}
	How does this relate to (4)? Let $ x=a$ be a point of discontinuity of  $ F_X$.

	 \[
		 P_X(\{x\} )=P_X((-\infty,a])-P_X((-\infty,a)) = F(a)-F(a^{-})>0
	.\]
	So we see that proving this claim gives us (4), since discontinuities are disjoint singleton sets with positive $ \mu$ measure.
\end{intuition}
\end{property}

\begin{prf}
Suppose that $ \{A_s\}_{s \in S} $ is a disjoint collection of sets each with positive $ \mu$ measure. We want to show that $ S$ must be countable. 

Since $ \mu$ is $\sigma$-finite, there exists a cover $ \{B_n\} $ of $ \Omega$ of $ \mathcal{F}$-sets with $ \mu(B_n)< \infty \ \forall \ n$. For a given $ n$, consider the set of indices:
 \[
	 S_n = \{s \in S: \mu(A_s \cap B_n)>0\} 
.\] 
\begin{claim}[]
For each $ n$,  $ S_n$ is countable.
\end{claim}
Let $ A$ be any set in  $ \mathcal{F}$ ($ \mathcal{F}$-set) with $ \mu(A) < \infty$. Fix $ \epsilon>0$, suppose that $ s_1,s_2,\ldots,s_k$ are distinct indices such that 
\[
	\mu(A_{s_i} \cap A) \geq \epsilon > 0
.\]
Then
\begin{align*}
	k \epsilon &\leq \sum_{ i= 1}^{ k} \mu(A_{s_i} \cap A) \leq \mu(A) \text{ sum of measure of disjoint sets, might not cover }A \\
	k&\leq \frac{\mu(A)}{ \epsilon} < \infty \text{ by assumption}\\ 
\end{align*}
Therefore, the index set must be finite for each $ \epsilon>0$ (using the same rationals are dense trick). So apply the claim to $ B_n$ and union over all the $ n$,
\[
\bigcup_{ n= 1}^{ \infty} S_n \text{ is countable} 
.\] 
Since $ \mu(A_s)>0, \{B_n\} $ is a cover of $ \Omega$, there exists a $ n$  such that  $ \mu(A_s \cap B_n)>0$. \emph{i.e.} $ s \in S \implies s \in S_n$. Thus $ S \subseteq \bigcup_{ n =1}^{\infty} S_n$ must also be countable.
\end{prf}

~\begin{thm}[14.1]
	Suppose $ F$ is non-decreasing, right-continuous, with  $ \lim_{ x \to -\infty} F(x)=0$ and $ \lim_{ x \to \infty} F(x)=1$. Then $ F$ is the distribution function of some r.v.
\end{thm}

\begin{prf}
	Motivation (pre measure theory): Suppose $ X$ has cdf  $ F$ that is invertible. Then  $ F(X) \sim \text{Unif}(0,1)$. 

	Consider the probability space $(\Omega,\mathcal{F},P)$ with $ \Omega=(0,1)$, $ \mathcal{F}=\mathcal{B}((0,1))$, and $ P$ being the Lebesgue measure. 

	For intuition, let's first suppose that  $ F$ is invertible. For  $ \omega \in \Omega$, define $ X(\omega) = F^{-1}(\omega)$. Then this $ X$ is the r.v. we seek, because
	 \begin{enumerate}[label=\arabic*)]
		\item $ X$ is a r.v. measurable  $ \mathcal{F}$.
			
			Take $ x \in \rr$, since $ F$ is invertible and non-decreasing, $ F^{-1}$ is strictly increasing. So
\begin{align*}
\{\omega: X(\omega)\leq x\} &= \{\omega: F^{-1}(\omega)\leq x\}  \\
&= \{\omega: \omega \leq F(x)\}  \text{ by increasing} \\
&= (0,F(x)] \in \mathcal{B}((0,1)) =\mathcal{F} 
\end{align*}
		\item $ P(X\leq x) = F(x)$.

			\begin{align*}
				P(X\leq x) &= P(\{\omega:X(\omega)\leq x\} )\\
					    &= P(\{\omega: F^{-1}(\omega) \leq x\} ) \\
					    &= P(\{\omega: \omega\leq F(x)\} ) \\
					    &= \lambda((0,F(x))) \\
					    &= F(x)-0= F(x) \\
			\end{align*}
	\end{enumerate}

	Now let's define a generalized inverse;
	\[
		F^{-1}(\omega) \coloneqq \inf\left\{x: \omega\leq F(x) \right\} 
	.\] 
\begin{intuition}
	This is sort of the "leftmost" $ x$ that makes $ F(x)$ greater or equal to  $ \omega$.
\end{intuition}
Choose $ X(\omega) = F^{-1}(\omega)$. Notice that $ \inf\left\{x:\omega \leq F(x) \right\} \leq x $ as long as $ \omega \leq F(x)$ by definition of infimum. So 
\begin{align*}
	\{\omega: X(\omega) \leq x\} &= \{\omega: F^{-1}(\omega) \leq x \} \\
				     &= \{\omega: \inf\left\{x: \omega \leq F(x) \right\} \leq x \}  \\
	&= \{\omega:\omega\leq F(x)\}  	
\end{align*}
Then the same proof applies.
\end{prf}

\subsection{Weak Convergence (Convergence in Distribution)}
~\begin{defn}[weak convergence]
If $ F_n, n=1,2,\ldots$ and $ F$ are distribution functions, then  $ F_n$ converges weakly to $ F$ if
\[
	\lim_{ n \to \infty} F_n(x)=F(x)
\] 
at all points of continuity $ x$ of  $ F$. We say that the corresponding r.v.'s converge in distribution (convergence in law).  We denote it as $ X_n \xrightarrow{ d}X $.
\end{defn}
\begin{note}[]
It doesn't mean that the r.v.'s are getting close at all, only their distributions.
\end{note}

\begin{eg}[]
	Given $ X_1,X_2,\ldots$ i.i.d $ \sim N(\mu, \sigma^2)$, and $ X \sim N(\mu, \sigma^2)$ independent of $ (X_n)$. Then $ X_n \xrightarrow{ d} X $ but $ X_n $ doesn't converge to $ X$ since they are independent.
\end{eg}

\begin{thm}[]
\[ X_n \xrightarrow{ a.s.} X  \implies X_n \xrightarrow{ p} X \implies X_n \xrightarrow{ d} X .\]
\end{thm}

\begin{prf}
Let $ F_n$ be the distribution function of $ X_n$, and $ F$ for  $ X$. Take $ \epsilon>0$. Let $ x$  be a point of continuity of $ F$. 
\begin{align*}
	F_n(x)&= P(X_n \leq x) = P(\{\omega:X_n(\omega) \leq x\} ) \\
	      &= P(\{X_n \leq x ,  X\leq x + \epsilon\} \cup  \{X_n \leq x ,  X>x+ \epsilon \}   ) \\
	      &= P(X_n \leq x ,X\leq x + \epsilon ) +P( X_n \leq x ,  X>x + \epsilon   ) \text{ by disjoint} \\
	      &\leq P(X \leq x + \epsilon ) + P(|X_n - X| > \epsilon) \text{ monotonicity, easy to see from drawing} 
\end{align*}

The other direction gives
\begin{align*}
	F(X\leq x- \epsilon) &=   P(X \leq x - \epsilon, X_n \leq x ) +P( X \leq x - \epsilon ,  X_n > x ) \\
			     &\leq P(X_n \leq x ) + P(|X_n - X| > \epsilon)  \\
\end{align*}
Take $ n \to \infty$, $ X_n \xrightarrow{ p} X \implies \lim_{ n \to \infty} P(|X_n -X|> \epsilon) =0 $. So
\begin{align*}
	F(x- \epsilon) - \lim_{ n \to \infty} P(|X_n -X| > \epsilon) &\leq \lim_{ n \to \infty}  F_n(x) \text{ if it exists }  \leq F(x+ \epsilon) + \lim_{ n \to \infty}  P(|X_n - X|> \epsilon)\\
	F(x- \epsilon) &\leq \lim_{ n \to \infty} F_n(x) \leq F(x+ \epsilon) \\
\end{align*}
Finally, take $ \epsilon \to 0$, since $ x$ is a point of continuity of  $ F$, by Squeeze Theorem the limit of  $ F_n(x)$ exists and equals to $ F(x)$.

 \begin{note}[]
More formally, we would squeeze twice for both $ \liminf$ and $  \limsup $ and show that they are equal.
\end{note}
\end{prf}

\begin{note}[]
The converse is false.
\end{note}
\begin{eg}[counterexample]
	$ X_1,X_2,\ldots$ independent with $ P(X_i = \pm 1) = \frac{1}{2}$. Suppose $ X$ is independent of the  $ X_i$ with $ P(X=\pm 1)=\frac{1}{2}$. Then $ X_n \xrightarrow{ d} X$ because they have the same distribution functions. However, 
	\begin{align*}
		P(|X_n-X|\geq 2) &= P(|X_n -X|=2) \\
				 &= P(X_n=1, X=-1) + P(X_n=-1, X=1) \\
				 &= P()P() + P()P() \text{ by independence}  \\
				 &= \frac{1}{2} \cdot \frac{1}{2} + \frac{1}{2} \cdot \frac{1}{2} =\frac{1}{2} \neq 0\\
	\end{align*} 
\end{eg}

\begin{eg}[]
$ X_n$ with cdfs $ F_n$, where
\begin{equation*}
	\lim_{ n \to \infty} F_n(x) =
\begin{cases}
	0,&x<c\\
	1,& x\geq c\\
\end{cases}
\end{equation*}
So $ X_n \xrightarrow{ d} X $ where $ X=c$ w.p. 1.
\end{eg}

\begin{eg}[]
\begin{equation*}
	\lim_{ n \to \infty} F_n(x)
\begin{cases}
	0, &x\leq c\\
	1, & x\geq c\\
\end{cases}
\end{equation*}
This is not a valid cdf but matches at all points of continuities. Then $ X_n \xrightarrow{ d}X $ where $ X=c$ w.p. 1.
\end{eg}
\end{document}

\documentclass[class=article,crop=false]{standalone} 
%Fall 2020
% Some basic packages
\usepackage{standalone}[subpreambles=true]
\usepackage[utf8]{inputenc}
\usepackage[T1]{fontenc}
\usepackage{textcomp}
\usepackage[english]{babel}
\usepackage{url}
\usepackage{graphicx}
\usepackage{float}
\usepackage{enumitem}


\pdfminorversion=7

% Don't indent paragraphs, leave some space between them
\usepackage{parskip}

% Hide page number when page is empty
\usepackage{emptypage}
\usepackage{subcaption}
\usepackage{multicol}
\usepackage[dvipsnames]{xcolor}


% Math stuff
\usepackage{amsmath, amsfonts, mathtools, amsthm, amssymb}
% Fancy script capitals
\usepackage{mathrsfs}
\usepackage{cancel}
% Bold math
\usepackage{bm}
% Some shortcuts
\newcommand{\rr}{\ensuremath{\mathbb{R}}}
\newcommand{\zz}{\ensuremath{\mathbb{Z}}}
\newcommand{\qq}{\ensuremath{\mathbb{Q}}}
\newcommand{\nn}{\ensuremath{\mathbb{N}}}
\newcommand{\ff}{\ensuremath{\mathbb{F}}}
\newcommand{\cc}{\ensuremath{\mathbb{C}}}
\renewcommand\O{\ensuremath{\emptyset}}
\newcommand{\norm}[1]{{\left\lVert{#1}\right\rVert}}
\renewcommand{\vec}[1]{{\mathbf{#1}}}
\newcommand\allbold[1]{{\boldmath\textbf{#1}}}

% Put x \to \infty below \lim
\let\svlim\lim\def\lim{\svlim\limits}

%Make implies and impliedby shorter
\let\implies\Rightarrow
\let\impliedby\Leftarrow
\let\iff\Leftrightarrow
\let\epsilon\varepsilon

% Add \contra symbol to denote contradiction
\usepackage{stmaryrd} % for \lightning
\newcommand\contra{\scalebox{1.5}{$\lightning$}}

% \let\phi\varphi

% Command for short corrections
% Usage: 1+1=\correct{3}{2}

\definecolor{correct}{HTML}{009900}
\newcommand\correct[2]{\ensuremath{\:}{\color{red}{#1}}\ensuremath{\to }{\color{correct}{#2}}\ensuremath{\:}}
\newcommand\green[1]{{\color{correct}{#1}}}

% horizontal rule
\newcommand\hr{
    \noindent\rule[0.5ex]{\linewidth}{0.5pt}
}

% hide parts
\newcommand\hide[1]{}

% si unitx
\usepackage{siunitx}
\sisetup{locale = FR}

% Environments
\makeatother
% For box around Definition, Theorem, \ldots
\usepackage[framemethod=TikZ]{mdframed}
\mdfsetup{skipabove=1em,skipbelow=0em}

%definition
\newenvironment{defn}[1][]{%
\ifstrempty{#1}%
{\mdfsetup{%
frametitle={%
\tikz[baseline=(current bounding box.east),outer sep=0pt]
\node[anchor=east,rectangle,fill=Emerald]
{\strut Definition};}}
}%
{\mdfsetup{%
frametitle={%
\tikz[baseline=(current bounding box.east),outer sep=0pt]
\node[anchor=east,rectangle,fill=Emerald]
{\strut Definition:~#1};}}%
}%
\mdfsetup{innertopmargin=10pt,linecolor=Emerald,%
linewidth=2pt,topline=true,%
frametitleaboveskip=\dimexpr-\ht\strutbox\relax
}
\begin{mdframed}[]\relax%
\label{#1}}{\end{mdframed}}


%theorem
%\newcounter{thm}[section]\setcounter{thm}{0}
%\renewcommand{\thethm}{\arabic{section}.\arabic{thm}}
\newenvironment{thm}[1][]{%
%\refstepcounter{thm}%
\ifstrempty{#1}%
{\mdfsetup{%
frametitle={%
\tikz[baseline=(current bounding box.east),outer sep=0pt]
\node[anchor=east,rectangle,fill=blue!20]
%{\strut Theorem~\thethm};}}
{\strut Theorem};}}
}%
{\mdfsetup{%
frametitle={%
\tikz[baseline=(current bounding box.east),outer sep=0pt]
\node[anchor=east,rectangle,fill=blue!20]
%{\strut Theorem~\thethm:~#1};}}%
{\strut Theorem:~#1};}}%
}%
\mdfsetup{innertopmargin=10pt,linecolor=blue!20,%
linewidth=2pt,topline=true,%
frametitleaboveskip=\dimexpr-\ht\strutbox\relax
}
\begin{mdframed}[]\relax%
\label{#1}}{\end{mdframed}}


%lemma
\newenvironment{lem}[1][]{%
\ifstrempty{#1}%
{\mdfsetup{%
frametitle={%
\tikz[baseline=(current bounding box.east),outer sep=0pt]
\node[anchor=east,rectangle,fill=Dandelion]
{\strut Lemma};}}
}%
{\mdfsetup{%
frametitle={%
\tikz[baseline=(current bounding box.east),outer sep=0pt]
\node[anchor=east,rectangle,fill=Dandelion]
{\strut Lemma:~#1};}}%
}%
\mdfsetup{innertopmargin=10pt,linecolor=Dandelion,%
linewidth=2pt,topline=true,%
frametitleaboveskip=\dimexpr-\ht\strutbox\relax
}
\begin{mdframed}[]\relax%
\label{#1}}{\end{mdframed}}

%corollary
\newenvironment{coro}[1][]{%
\ifstrempty{#1}%
{\mdfsetup{%
frametitle={%
\tikz[baseline=(current bounding box.east),outer sep=0pt]
\node[anchor=east,rectangle,fill=CornflowerBlue]
{\strut Corollary};}}
}%
{\mdfsetup{%
frametitle={%
\tikz[baseline=(current bounding box.east),outer sep=0pt]
\node[anchor=east,rectangle,fill=CornflowerBlue]
{\strut Corollary:~#1};}}%
}%
\mdfsetup{innertopmargin=10pt,linecolor=CornflowerBlue,%
linewidth=2pt,topline=true,%
frametitleaboveskip=\dimexpr-\ht\strutbox\relax
}
\begin{mdframed}[]\relax%
\label{#1}}{\end{mdframed}}

%proof
\newenvironment{prf}[1][]{%
\ifstrempty{#1}%
{\mdfsetup{%
frametitle={%
\tikz[baseline=(current bounding box.east),outer sep=0pt]
\node[anchor=east,rectangle,fill=SpringGreen]
{\strut Proof};}}
}%
{\mdfsetup{%
frametitle={%
\tikz[baseline=(current bounding box.east),outer sep=0pt]
\node[anchor=east,rectangle,fill=SpringGreen]
{\strut Proof:~#1};}}%
}%
\mdfsetup{innertopmargin=10pt,linecolor=SpringGreen,%
linewidth=2pt,topline=true,%
frametitleaboveskip=\dimexpr-\ht\strutbox\relax
}
\begin{mdframed}[]\relax%
\label{#1}}{\qed\end{mdframed}}


\theoremstyle{definition}

\newmdtheoremenv[nobreak=true]{definition}{Definition}
\newmdtheoremenv[nobreak=true]{prop}{Proposition}
\newmdtheoremenv[nobreak=true]{theorem}{Theorem}
\newmdtheoremenv[nobreak=true]{corollary}{Corollary}
\newtheorem*{eg}{Example}
\theoremstyle{remark}
\newtheorem*{case}{Case}
\newtheorem*{notation}{Notation}
\newtheorem*{remark}{Remark}
\newtheorem*{note}{Note}
\newtheorem*{problem}{Problem}
\newtheorem*{observe}{Observe}
\newtheorem*{property}{Property}
\newtheorem*{intuition}{Intuition}


% End example and intermezzo environments with a small diamond (just like proof
% environments end with a small square)
\usepackage{etoolbox}
\AtEndEnvironment{vb}{\null\hfill$\diamond$}%
\AtEndEnvironment{intermezzo}{\null\hfill$\diamond$}%
% \AtEndEnvironment{opmerking}{\null\hfill$\diamond$}%

% Fix some spacing
% http://tex.stackexchange.com/questions/22119/how-can-i-change-the-spacing-before-theorems-with-amsthm
\makeatletter
\def\thm@space@setup{%
  \thm@preskip=\parskip \thm@postskip=0pt
}

% Fix some stuff
% %http://tex.stackexchange.com/questions/76273/multiple-pdfs-with-page-group-included-in-a-single-page-warning
\pdfsuppresswarningpagegroup=1


% My name
\author{Jaden Wang}



\begin{document}

\section*{Integrals for General Measurable Functions}

Consider $ f: \Omega \to [-\infty, \infty]$. Write $ f=f^{+} - f^{-}$ where $ f^{+} = \max\{f,0\}$ and $ f^{-}= -\min \{f,0\} $.
\begin{note}
~\begin{enumerate}[label=\arabic*)]
	\item $ f^{+}, f^{-} \geq 0$.
	\item $ |f| = f^{+}+ f^{-}$.
\end{enumerate}
\end{note}

\begin{defn}[integrable]
	$ f: \Omega \to [-\infty,\infty]$ is \allbold{integrable} if $ f^{+}, f^{-}$ are integrable. In this case,
	\[
	\int_{A} f\ d \mu = \int_A f^{+}\ d \mu - \int_{  A} f^{-} \ d  \mu 
	.\] 
\end{defn}

\begin{note}
$ f^{\pm}$ integrable $ \implies $ $ f^{\pm}$ measurable $ \implies f$ is measurable. 
\end{note}

\begin{thm}
Suppose $ f$ is measurable. Then  $ f$ is integrable if and only if  $ |f|$ is integrable.
\end{thm}

\begin{prf}
	($ \implies$): Suppose $ f$ is integrable, then  $ \int  f^{+} \ d  \mu < \infty$ and $ \int  f^{-} \ d  \mu <\infty$ so $ \int  |f| \ d  \mu = \int  f^{+} \ d  \mu + \int  f^{-} \ d  \mu < \infty$.

	$ (\impliedby)$: Suppose $ |f|$ is integrable, then  $ |f|$ is measurable. Write $ f^{+}= \frac{1}{2}(f+ |f|) \implies f^{+}$ is measurable. Moreover, 
	\[
	f^{+}\leq |f| \implies \int  f^{+} \ d  \mu  \leq \int  |f| \ d  \mu <\infty 
	.\] 
	So $ f^{+}$ is integrable. Likewise $ f^{-}$ is integrable. So $ f=f^{+}-f^{-}$ is integrable.
\end{prf}

\begin{claim}
If $ f,g$ are measurable, then
 \begin{enumerate}[label=\arabic*)]
	 \item $\min \{f,g\} , \max \{f,g\} $ are measurable. Think $ \{\omega: \max(f(\omega),g(\omega)) \leq x\} $.
	\item $ -f$ is measurable.
\end{enumerate}
\end{claim}

\begin{remark}
Billingsley defines the integral as 
\[
	\int f d \mu = \sup \sum_{ i= 1}^{\infty} \left[ \inf_{A_i} f  \right]  \mu (A_i)
.\] 
where the sup is taken of all finite partitions of $ \Omega$ into $ \mathcal{F}$-sets $ A_i$.
\end{remark}

\subsection*{Adventure in "Almost Everywhere" Properties}

~\begin{defn}[almost everywhere]
A property holds \allbold{almost everywhere} if it holds for all sets except for possible some sets of measure zero. 
\end{defn}

\begin{enumerate}[label=\arabic*)]
	\item $ f=0$ a.e.  $ \implies \int f d \mu = 0$ (need $ f$ measurable).
	\item  $ f=g$ a.e.  $ \implies \int f \ d \mu = \int g \ d \mu$.
	\item $ f\leq g$ a.e.,  $ f,g$ measurable  $ \implies \int f \ d \mu \leq \int g\ d \mu$.
	\item Suppose $ f$ is integrable and  $ \int_{  A}  f \ d  \mu \geq 0$ for every $ A \in \mathcal{F}$. Then $ f \geq 0$ a.e.
		 \begin{prf}
			 Let $ B = \{\omega: f(\omega) < 0\} $. We want to show that $ \mu(B)=0$. Then we have
			 \begin{equation*}
				 I_{B}(\omega) = 
			 \begin{cases}
				 0 & \text{ if } f(\omega)\geq 0\\
				 f(\omega) & \text{ if } f(\omega)<0 
			 \end{cases}
			 \end{equation*}
			 So $ f(\omega) I_{B}(\omega) \leq f(\omega)$. Since $ f(\omega) I_B $ is non-positive, we have
			 \begin{align*}
				 nf(\omega) I_{B}(\omega) &\leq f(\omega) \ \forall \ n \in \nn \\
				 \int n f I_B\ d \mu &\leq \int f\ d\mu\\
				 \int_B f\ d\mu &\leq \frac{1}{n} \int f\ d\mu  
			 \end{align*}
			 Taking $ n \to \infty$, we have $ \int_B f d \ \mu \leq 0$.
			 Since by assumption, $ \int_B f \ d \mu \geq 0$, we have $ \int_B f\ d \mu = 0$. And since we define $ f(\omega)<0 \ \forall \ \omega \in B$, it must be that $ \mu(B) = 0 \implies f\geq 0$ a.e.
		\end{prf}
\end{enumerate}

\section*{20: Random Variables}

$ (\Omega,\mathcal{F},P)$.

\begin{defn}[random variable]
A \allbold{random variable} is a measurable function $ X: \Omega \to \rr$. 
\end{defn}

\begin{defn}[random vector]
A \allbold{random vector} $ X$ is a measurable function $ X: \Omega \to \rr^{k}$. It necessarily has the form
\[
	X(\omega)=(X_1(\omega),\ldots,X_k(\omega))
.\] 
\end{defn}

\begin{claim}
$ X$ is a random vector if and only if each  $ X_i$ is measurable.
\end{claim}

\begin{thm}[20.1]
	Let $ X=(X_1,\ldots,X_k)$ be a random vector.
	\begin{enumerate}[label=(\roman*)]
		\item $ \sigma(X) = \sigma(X_1,\ldots,X_k)$ consists precisely of the sets $ \{\omega: X(\omega) \in H\} \ \forall \ H \in \mathcal{B}(\rr^{k})$.
		\item A r.v. $ Y$ is measurable wrt  $ \sigma(X)$ if and only if there exists a measurable $ f: \rr^{k} \to \rr$ such that $ Y(\omega) = f(X_1(\omega),\ldots,X_k(\omega)) \ \forall \ \omega \in \Omega$. 
	\end{enumerate}
\end{thm}

\begin{note}
This is Theorem 5.1 but with general random variables.
\end{note}

\begin{prf}
~\begin{enumerate}[label=(\roman*)]
	\item same as 5.1 by defining $ \mathcal{G}= \{X^{-1}(H): H \in \mathcal{B}( \rr^{k})\} $ and show equivalence.
	\item ($ \impliedby$) Suppose there exists a measurable $ f: \rr^{k} \to \rr$ such that $ Y(\omega) = f(X(\omega)) \ \forall \ \omega \in \Omega$. Then by Theorem 13.1 (ii), composite measurable function is measurable $ \implies Y$ measurable wrt $ \sigma(X)$.

		$ (\implies)$: Suppose $ Y: \Omega \to \rr$ is measurable wrt $ \sigma(X)$. Consider the following cases:
		\begin{case}[1]
			$ Y$ is simple, \emph{i.e.} $ Y(\omega) = \sum_{ i= 1}^{ n} a_i I_{A_i}$ where $ A_i$s are disjoint. We want to find a $ f$ measurable $ \mathcal{B}(\rr^{k})$ such that $ Y(\omega) = f(X(\omega)) \ \forall \ \omega \in \Omega$. 

			$ Y$ is measurable wrt  $ \sigma(X)$ implies that
			\[
				A_i = Y^{-1}(\{a_i\} ) \in \sigma(X)
			.\]
			By part (i) of this theorem, we know $ A_i $ has the form
			 \[
				 A_i = \{\omega: X(\omega) \in H_i\} \text{ for some } H_i \in \mathcal{B}(\rr^{k}) 
			.\] 
			Now let's define $ f:\rr^{k} \to \rr$ to be
				\[
					f(x) = \sum_{ i= 1}^{ n} a_i I_{H_i}(x)
				.\]
				Note that $ f$ is measurable (since the inverse image would give us $ H_i$ or $ \O$, both in $ \mathcal{ B}(\rr^{k})$). Therefore,
				\begin{align*}
					f(X(\omega)) &= \sum_{ i= 1}^{ n} a_i I_{H_i}(X(\omega))\\
						     &= \sum_{ i= 1}^{ n} a_i I_{A_i}(\omega) \\
						     &= Y(\omega)
				\end{align*}
		\end{case}
		\begin{case}[2]
			$ Y$ is simple. Then we want to approximate  $ Y$ with simple functions  $ Y_n$, \emph{i.e.} $ 0\leq Y_n(\omega) \nearrow Y(\omega)$ if $ Y(\omega) \geq 0$ and $ 0\geq Y_n(\omega) \searrow Y(\omega)$ if $ Y(\omega) < 0$ (see details in Theorem 13.5). We can use Case 1 to find measurable $ f: \rr^{k} \to \rr$ such that $ f_n(X(\omega)) = Y_n(\omega) \ \forall \ \omega \in \Omega$.

			Now consider the set $ A = \{x: f_n(x) \text{ converges} \} $. Then by the theorem right after 13.4 ($ A = \{\liminf_{  n} f_n(x) = \limsup_{  n} f_n(x)\} $ where LHS and RHS are both measurable), we know $ A \in \mathcal{B}(\rr^{k})$. Let's define $ f: \rr^{k} \to \rr$ as
			\begin{equation*}
				f(x)=
			\begin{cases}
				\lim_{ n \to \infty} f_n(x) & ,x \in A\\
				0 & , x \not\in A\\
			\end{cases}
			\end{equation*}
			Note that $ f = \left( \lim_{ n \to \infty} f_n \right) \cdot I_A = \lim_{ n \to \infty} (f_n \cdot  I_A)$. Since $ f_n, I_A$ are measurable, $ f_n \cdot I_A$ is also measurable, and the limit is also measurable by 13.4. Thus we showed that $ f$ is measurable and thus $ f(X(\omega)) = Y(\omega)$ is what we seek.
		\end{case}
\end{enumerate}
\end{prf}
\end{document}

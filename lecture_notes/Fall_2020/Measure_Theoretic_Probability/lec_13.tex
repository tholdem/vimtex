\documentclass[class=article,crop=false]{standalone} 
%Fall 2020
% Some basic packages
\usepackage{standalone}[subpreambles=true]
\usepackage[utf8]{inputenc}
\usepackage[T1]{fontenc}
\usepackage{textcomp}
\usepackage[english]{babel}
\usepackage{url}
\usepackage{graphicx}
\usepackage{float}
\usepackage{enumitem}


\pdfminorversion=7

% Don't indent paragraphs, leave some space between them
\usepackage{parskip}

% Hide page number when page is empty
\usepackage{emptypage}
\usepackage{subcaption}
\usepackage{multicol}
\usepackage[dvipsnames]{xcolor}


% Math stuff
\usepackage{amsmath, amsfonts, mathtools, amsthm, amssymb}
% Fancy script capitals
\usepackage{mathrsfs}
\usepackage{cancel}
% Bold math
\usepackage{bm}
% Some shortcuts
\newcommand{\rr}{\ensuremath{\mathbb{R}}}
\newcommand{\zz}{\ensuremath{\mathbb{Z}}}
\newcommand{\qq}{\ensuremath{\mathbb{Q}}}
\newcommand{\nn}{\ensuremath{\mathbb{N}}}
\newcommand{\ff}{\ensuremath{\mathbb{F}}}
\newcommand{\cc}{\ensuremath{\mathbb{C}}}
\renewcommand\O{\ensuremath{\emptyset}}
\newcommand{\norm}[1]{{\left\lVert{#1}\right\rVert}}
\renewcommand{\vec}[1]{{\mathbf{#1}}}
\newcommand\allbold[1]{{\boldmath\textbf{#1}}}

% Put x \to \infty below \lim
\let\svlim\lim\def\lim{\svlim\limits}

%Make implies and impliedby shorter
\let\implies\Rightarrow
\let\impliedby\Leftarrow
\let\iff\Leftrightarrow
\let\epsilon\varepsilon

% Add \contra symbol to denote contradiction
\usepackage{stmaryrd} % for \lightning
\newcommand\contra{\scalebox{1.5}{$\lightning$}}

% \let\phi\varphi

% Command for short corrections
% Usage: 1+1=\correct{3}{2}

\definecolor{correct}{HTML}{009900}
\newcommand\correct[2]{\ensuremath{\:}{\color{red}{#1}}\ensuremath{\to }{\color{correct}{#2}}\ensuremath{\:}}
\newcommand\green[1]{{\color{correct}{#1}}}

% horizontal rule
\newcommand\hr{
    \noindent\rule[0.5ex]{\linewidth}{0.5pt}
}

% hide parts
\newcommand\hide[1]{}

% si unitx
\usepackage{siunitx}
\sisetup{locale = FR}

% Environments
\makeatother
% For box around Definition, Theorem, \ldots
\usepackage[framemethod=TikZ]{mdframed}
\mdfsetup{skipabove=1em,skipbelow=0em}

%definition
\newenvironment{defn}[1][]{%
\ifstrempty{#1}%
{\mdfsetup{%
frametitle={%
\tikz[baseline=(current bounding box.east),outer sep=0pt]
\node[anchor=east,rectangle,fill=Emerald]
{\strut Definition};}}
}%
{\mdfsetup{%
frametitle={%
\tikz[baseline=(current bounding box.east),outer sep=0pt]
\node[anchor=east,rectangle,fill=Emerald]
{\strut Definition:~#1};}}%
}%
\mdfsetup{innertopmargin=10pt,linecolor=Emerald,%
linewidth=2pt,topline=true,%
frametitleaboveskip=\dimexpr-\ht\strutbox\relax
}
\begin{mdframed}[]\relax%
\label{#1}}{\end{mdframed}}


%theorem
%\newcounter{thm}[section]\setcounter{thm}{0}
%\renewcommand{\thethm}{\arabic{section}.\arabic{thm}}
\newenvironment{thm}[1][]{%
%\refstepcounter{thm}%
\ifstrempty{#1}%
{\mdfsetup{%
frametitle={%
\tikz[baseline=(current bounding box.east),outer sep=0pt]
\node[anchor=east,rectangle,fill=blue!20]
%{\strut Theorem~\thethm};}}
{\strut Theorem};}}
}%
{\mdfsetup{%
frametitle={%
\tikz[baseline=(current bounding box.east),outer sep=0pt]
\node[anchor=east,rectangle,fill=blue!20]
%{\strut Theorem~\thethm:~#1};}}%
{\strut Theorem:~#1};}}%
}%
\mdfsetup{innertopmargin=10pt,linecolor=blue!20,%
linewidth=2pt,topline=true,%
frametitleaboveskip=\dimexpr-\ht\strutbox\relax
}
\begin{mdframed}[]\relax%
\label{#1}}{\end{mdframed}}


%lemma
\newenvironment{lem}[1][]{%
\ifstrempty{#1}%
{\mdfsetup{%
frametitle={%
\tikz[baseline=(current bounding box.east),outer sep=0pt]
\node[anchor=east,rectangle,fill=Dandelion]
{\strut Lemma};}}
}%
{\mdfsetup{%
frametitle={%
\tikz[baseline=(current bounding box.east),outer sep=0pt]
\node[anchor=east,rectangle,fill=Dandelion]
{\strut Lemma:~#1};}}%
}%
\mdfsetup{innertopmargin=10pt,linecolor=Dandelion,%
linewidth=2pt,topline=true,%
frametitleaboveskip=\dimexpr-\ht\strutbox\relax
}
\begin{mdframed}[]\relax%
\label{#1}}{\end{mdframed}}

%corollary
\newenvironment{coro}[1][]{%
\ifstrempty{#1}%
{\mdfsetup{%
frametitle={%
\tikz[baseline=(current bounding box.east),outer sep=0pt]
\node[anchor=east,rectangle,fill=CornflowerBlue]
{\strut Corollary};}}
}%
{\mdfsetup{%
frametitle={%
\tikz[baseline=(current bounding box.east),outer sep=0pt]
\node[anchor=east,rectangle,fill=CornflowerBlue]
{\strut Corollary:~#1};}}%
}%
\mdfsetup{innertopmargin=10pt,linecolor=CornflowerBlue,%
linewidth=2pt,topline=true,%
frametitleaboveskip=\dimexpr-\ht\strutbox\relax
}
\begin{mdframed}[]\relax%
\label{#1}}{\end{mdframed}}

%proof
\newenvironment{prf}[1][]{%
\ifstrempty{#1}%
{\mdfsetup{%
frametitle={%
\tikz[baseline=(current bounding box.east),outer sep=0pt]
\node[anchor=east,rectangle,fill=SpringGreen]
{\strut Proof};}}
}%
{\mdfsetup{%
frametitle={%
\tikz[baseline=(current bounding box.east),outer sep=0pt]
\node[anchor=east,rectangle,fill=SpringGreen]
{\strut Proof:~#1};}}%
}%
\mdfsetup{innertopmargin=10pt,linecolor=SpringGreen,%
linewidth=2pt,topline=true,%
frametitleaboveskip=\dimexpr-\ht\strutbox\relax
}
\begin{mdframed}[]\relax%
\label{#1}}{\qed\end{mdframed}}


\theoremstyle{definition}

\newmdtheoremenv[nobreak=true]{definition}{Definition}
\newmdtheoremenv[nobreak=true]{prop}{Proposition}
\newmdtheoremenv[nobreak=true]{theorem}{Theorem}
\newmdtheoremenv[nobreak=true]{corollary}{Corollary}
\newtheorem*{eg}{Example}
\theoremstyle{remark}
\newtheorem*{case}{Case}
\newtheorem*{notation}{Notation}
\newtheorem*{remark}{Remark}
\newtheorem*{note}{Note}
\newtheorem*{problem}{Problem}
\newtheorem*{observe}{Observe}
\newtheorem*{property}{Property}
\newtheorem*{intuition}{Intuition}


% End example and intermezzo environments with a small diamond (just like proof
% environments end with a small square)
\usepackage{etoolbox}
\AtEndEnvironment{vb}{\null\hfill$\diamond$}%
\AtEndEnvironment{intermezzo}{\null\hfill$\diamond$}%
% \AtEndEnvironment{opmerking}{\null\hfill$\diamond$}%

% Fix some spacing
% http://tex.stackexchange.com/questions/22119/how-can-i-change-the-spacing-before-theorems-with-amsthm
\makeatletter
\def\thm@space@setup{%
  \thm@preskip=\parskip \thm@postskip=0pt
}

% Fix some stuff
% %http://tex.stackexchange.com/questions/76273/multiple-pdfs-with-page-group-included-in-a-single-page-warning
\pdfsuppresswarningpagegroup=1


% My name
\author{Jaden Wang}



\begin{document}
\begin{thm}[Markov's inequality]
For any $ c,r>0$, 
 \[
	 P(|X|\geq c) \leq \frac{E[|X|^{r}]}{c^{(r)} }
.\] 
\end{thm}
\begin{prf}
\[
	P(|X|\geq c) \iff P(|X|^{r} \geq c ^{r}) \leq \frac{E[|X|^{r}]}{c^{r} }
.\] 
by generalized Markov's inequality.
\end{prf}

\begin{thm}[Chebyshev's inequality]
	Suppose $ X$ is a r.v. with mean  $ \mu = E[X]$, $ \sigma^2 = Var[X]$. For any $ k>0$, then
	 \[
		 P(|X-\mu| < k \sigma) \geq 1-\frac{1}{k^2}
	.\] 
	or
	\[
		P(|X-\mu| \geq k \sigma) \leq \frac{1}{k^2}
	.\] 
\end{thm}
Proved by Markov:
\begin{prf}
\begin{align*}
	P(|X- \mu| \geq k \sigma) = P((X- \mu)^2 \geq k^2 \sigma^2) &\leq \frac{E(X-\mu)^2}{k^2 \sigma^2 } \\
								    &= \frac{\var[X]}{k^2 \sigma^2 } \\
								    &=  \frac{\sigma^2}{k^2 \sigma^2}\\
								    &= \frac{1}{k^2} 
\end{align*}
\end{prf}

\begin{thm}[Jensen's inequality]
If $ g$ is a convex function, then 
 \[
	 g(E[X]) \leq E[g(X)]
.\] 
If $ g$ is concave,  $ -g$ is convex. The sign thus flips.
\end{thm}

Prove by picture. Compare the tangent and graph of $ g(\mu)$.

\begin{thm}[Holder's inequality]
	Take $ p,g \geq 1$  such that $ \frac{1}{p}+\frac{1}{q}=1$ (Holder conjugates). Then
	\[
E[|XY|] \leq (E[|X|^{p}])^{\frac{1}{p}} \cdot (E[|Y|^{q}])^{\frac{1}{q}}
	.\] 
\end{thm}
We allow $ p=1$ and  $ q= \infty$ or vice versa.
\newpage
\section*{6. Law of Large Numbers}

~\begin{defn}[distribution of r.v.]
	The \allbold{distribution} of a r.v. $ X$ is the probability measure on  $ \rr$, denoted by $ P_X( \cdot )$, defined $ \ \forall \ A \subseteq \rr$ as
	\[
		P_X(A)= P(\{\omega: X(\omega) \in A\} )
	.\] 
\end{defn}
Show that this is a probability measure. (empty set is 0, countable additivity, outputs $ [0,1]$).

\begin{thm}[Strong Law of Large Numbers (SLLN)]
	Suppose that $ (X_n)$ is a sequence of independent and identically distributed r.v. (i.i.d.) with $ \mu = E[X_n]$ and $ E[X_n^{4}]< \infty$. Then
	\[
	\overline{X_n} \coloneqq \frac{1}{n} \sum_{ i= 1}^{ n} X_i \xrightarrow{ a.s.} \mu 
	.\] 
	\emph{i.e.} 
	\[
		P\left(\lim_{ n \to \infty} \overline{X_n}=\mu \right)=1
	.\] 
\end{thm}
\begin{note}[]
The last condition is always true for simple r.v.
\end{note}


\begin{prf}
	Transform $ X$ so that $ \mu=0$. Consider $ E[S_n^{4}]$. Most terms are just 0, except for
	\begin{itemize}
		\item $ E[X_i^2 X_j^2] = ( \sigma^2)^2$. There are $ 3n(n-1)$ choices for $ 4$ different sets of indices. 
		\item $ E[X_i^4] < C < \infty$. There are $ n$ choices.
	\end{itemize}
	So
	\begin{align*}	
		E[S_n^{4}] &\leq C \cdot  n + 3n(n-1)( \sigma^2)^2\\
			  &= C \cdot  n + 3n^2( \sigma^2)^2 - 3n ( \sigma^2)^2 \\
			  &\leq C \cdot n + 3n^2( \sigma^2)^2\\
			  &\leq C \cdot n^2 + 3n^2 ( \sigma^2)^2\\
			  &=kn^2 \qquad \text{ where } k=C + 3( \sigma^2)^2 < \infty
	\end{align*}
	Let $ \epsilon>0$, use generalized Markov inequality with $ g(x) = x^{4}$,
	\begin{align*}
		P(|S_n|\geq n \epsilon) &= P(|S_n|^{4} \geq n^{4} \epsilon^{4})\\
					&= P(S_n^{4}\geq n^{4} \epsilon^{4}) \\
					&\leq \frac{E[S_n^{4}]}{n^{4} \epsilon^{4} }\\
					&\leq \frac{kn^2}{n^{4} \epsilon^{4} }\\
					&= \frac{k}{n^2 \epsilon^{4}}
	\end{align*}
	So by Chebyshev, 
	\begin{align*}
		P(| \overline{X_n} - 0| \geq \epsilon) &= P(|\overline{X_n}| \geq \epsilon) \\
						       &= P(|S_n|\geq n \epsilon) \\
						       &\geq \frac{k}{n^2 \epsilon^{4}} \to 0 \text{ as } n \to \infty 
	\end{align*}
	Hence, $ \overline{X_n} \xrightarrow{ p} 0 $.

Note that
\[
	\sum_{ n= 1}^{\infty} P(|S_n|\geq n \epsilon) \geq \frac{k}{ \epsilon^{4}} \sum_{ n= 1}^{\infty} \frac{1}{n^2} < \infty
.\] 
By Borel-Cantelli (i), $ P(\limsup_{  n} A_n)=0$ or $ P(A_n\ i.o.)=0$, or $ P(|S_n|\geq n \epsilon \ i.o.) =0 \implies P(|\overline{X_n}| \geq \epsilon \ i.o.)=0$ for all $ \epsilon>0$. This is equivalent to
\[
	P(\lim_{ n \to \infty} \overline{X_n}=0) =1
.\] 
Thus, $ \overline{X_n} \xrightarrow{ a.s.} 0 $.

\end{prf}
\begin{note}[]
The weak law is just convergence in probability instead.
\end{note}
\newpage
\section*{Measure}
~\begin{defn}[Borel sets]
	In $ \rr^{n}$, the $\sigma$-field generated by the open rectangles $ \{(x_1,\ldots,x_n): a_i < x_i < b, i=1,2,\ldots,n\} $ is called the \allbold{Borel sets on $ \rr^{n}$}, denoted by $ \mathcal{B}( \rr^{n})$. 
\end{defn}

\begin{defn}[]
Let $ \mathcal{A}$ be a class/collection of sets in $ \Omega$. Let $ \Omega_0 \subseteq \Omega$ be a set of points. 
\[
\mathcal{A} \cap \Omega_0 \coloneqq \{A \cap \Omega_0: A \in \mathcal{A}\} 
.\] 
\end{defn}
\begin{note}[]
This is another collection of sets of points.
\end{note}

\begin{thm}[10.1]
Let $ \Omega_0 \subseteq \Omega$.
\begin{enumerate}[label=(\roman*)]
	\item $ \mathcal{F}$ is a $\sigma$-field on $ \Omega \implies \mathcal{F}_0 \coloneqq \mathcal{F} \cap  \Omega_0$ is a $\sigma$-field on $ \Omega_0$.
	\item If $ \mathcal{F}= \sigma(\mathcal{A})$ on $ \Omega \implies \mathcal{F}_0 \coloneqq \mathcal{F} \cap \Omega_0 = \sigma(\mathcal{A} \cap \Omega_0)$.
\end{enumerate}
\end{thm}
\begin{eg}[]
	$ \omega = \rr, \Omega_0 = [0,1]$. This theorem implies that $ \mathcal{B}([0,1]) = \mathcal{B}( \rr) \cap [0,1]$.
\end{eg}
\begin{prf}
~\begin{enumerate}[label=(\roman*)]
	\item By definition.
	\item By double containment. 

		$ \subseteq $: define
		\[
			\mathcal{G} = \{A \subseteq \Omega: A \cap  \Omega_0 \in \sigma(\mathcal{A} \cap  \Omega_0)\} 
		.\] 
		and want to show $ \mathcal{F} \subseteq \mathcal{G}$.
		\begin{claim}[]
		$ \mathcal{A} \subseteq \mathcal{G}$.
		\end{claim}
		$ A \in \mathcal{A} \implies A \cap \Omega_0 \in \mathcal{A} \cap \Omega_0 \subseteq \sigma(\mathcal{A} \cap \Omega_0)$.
		\begin{claim}[]
		$ \mathcal{G}$ is a $\sigma$-field on $ \Omega$.
		\end{claim}
		The goal is to show that $ \mathcal{F}$ is the smallest $\sigma$-field containing $ \mathcal{A}$, so $ \mathcal{G}$ as another $\sigma$-field containing $ \mathcal{A}$ must contain $ \mathcal{F}$. 
		\begin{enumerate}[label=(\roman*)]
			\item $ \Omega \cap \Omega_0 = \Omega_0 \in \sigma(\mathcal{A} \cap \Omega_0)$ since it is a $\sigma$-field on $ \Omega_0$.
			\item Take $ A \in \mathcal{G}$, 
				\begin{align*}
					(\Omega \setminus A) \cap \Omega_0 &= \Omega_0 \setminus (A \cap \Omega_0) \text{ by a picture} \\
									   & \in \sigma(\mathcal{A} \cap \Omega_0) \text{ as the complement in } \Omega_0 
				\end{align*}
			\item Take $ A_1, A_2, \ldots \in \mathcal{G}$.
				\[
					\left( \bigcup_{ n =1}^{\infty} A_n \right) \cap \Omega_0 = \bigcup_{ n =1}^{\infty} (A_n \cap \Omega_0) \in \sigma(\mathcal{A} \cap \Omega_0)
				.\] 
		\end{enumerate}
$ \supseteq :$
$ \mathcal{A} \cap \Omega_0 \in \sigma(\mathcal{A}) \cap \Omega_0$ since $ \mathcal{A} \subseteq \sigma(\mathcal{A}) $. Then $ \mathcal{A} \cap \Omega_0 \in \mathcal{F} \cap \Omega_0 = \mathcal{F}_0$ which is a $\sigma$-field on $ \Omega_0$ by part (i). So as the smallest $\sigma$-field containing $ \mathcal{A} \cap \Omega_0$, $ \sigma(\mathcal{A} \cap \Omega_0) \subseteq \mathcal{F}_0 = \sigma(\mathcal{A}) \cap \Omega_0$.
\end{enumerate}
\end{prf}

\begin{defn}[]
	Suppose $ (\Omega, \mathcal{F})$. A \allbold{general measure} $ \mu: \mathcal{F} \to [0, \infty]$ satisfies 
	\begin{enumerate}[label=(\roman*)]
		\item $ \mu( \O) = 0$.
		\item countable additivity of disjoint $ A_1,A_2,\ldots \in \mathcal{F}$.
	\end{enumerate}
\end{defn}

\begin{defn}[sigma-finite]
	The measure space $ (\Omega,\mathcal{F},\mu)$ is a \allbold{ $ \sigma$-finite space} if $ \Omega$ can be written as a countable union of $ \mathcal{F}$-sets, $ A_1, A_2,\ldots$ (not necessarily disjoint), with $ \mu(A_n) < \infty$ for all $ n$. Then we say  $ \mu$ is $ \sigma$-finite. 
\end{defn}

\begin{note}[]
	finite measure $ \implies$ $ \sigma$-finite.

	$ \sigma$-finite $ \not \implies$ finite.
	\begin{eg}[]
		$ (\rr, \mathcal{B}, \lambda)$, $ \lambda ( \rr) = \infty$. But $ \rr = \bigcup_{ n =- \infty}^{\infty} [n,n+1) \implies \lambda([n,n+1))=1 < \infty$.
	\end{eg}
\end{note}

\begin{defn}[]
	$ \mu$ is \allbold{concentrated} on $ A \in \mathcal{F}$ if $ \mu(A^{c})=0$. 
\end{defn}
\begin{note}[]
$ \mu$ is concentrated on the support of $ \mu$.
\end{note}

\begin{defn}[discrete measure]
	A measure $ \mu$ is \allbold{discrete} if $ \Omega$ is discrete and if, for any $ A \in \mathcal{F}$, $ \mu(A) = \sum_{\omega \in A} \mu(\{\omega\} )$ .
\end{defn}
\subsection{properties of a general measure $ \mu$}
\begin{enumerate}[label=\arabic*)]
	\item monotone: $ A \subseteq B \implies \mu(A) \leq \mu(B)$. Since $ \mu(B) = \mu(A) + \mu(B \setminus A)$.
	\item finite subadditivity.
	\item countable subadditivity.
	\item continuity from below: $ A_1, A_2,\ldots \in \mathcal{F}, A \in \mathcal{F}$, then $ A_n \uparrow A \implies \mu(A_n) \uparrow \mu(A)$.
	\item continuity from above: $ A_n \downarrow A$ and $ \mu(A_1) < \infty$, then $ \mu(A_n) \downarrow \mu(A)$.
		\begin{eg}[]
			$ \mu(A_1) = \mu(A_2) + \mu(A_1 \setminus A_2) \implies \mu(A_1 \setminus A_2) = \mu(A_1) - \mu(A_2)$, which only makes sense if $ \mu(A_1) < \infty$. Then 
			\[
				\mu(A_1) - \mu(A_n) = \mu(A_1 \setminus A_n) \uparrow \mu(A_1 \setminus A) \text{ by 4.} = \mu(A_1 \setminus A) = \mu(A_1) - \mu(A) 
			.\] 
		\end{eg}
\end{enumerate}	

\begin{thm}[inclusion-exclusion]
	\[ \mu\left( \bigcup _{ i= 1}^{ n} A_n \right) = \sum_{ i= 1}^{ n} \mu(A_i) - \sum_{ i<j}^{ n} \mu(A_i \cap A_j) + \ldots+ (-1)^{n-1} \mu(A_1 \cap \ldots \cap A_n)\]
\end{thm}

\begin{thm}[10.3]
	Let $ \mathcal{P}$ be a $\pi$-system. Suppose that $ \mu_1$ and $ \mu_2$ are two measures on $ \sigma(\mathcal{P})$ that are $ \sigma$-finite on $ \mathcal{P}$ and agree on $ \mathcal{P}$, then they agree on $ \sigma(\mathcal{P})$.
\end{thm}

~\begin{prf}
	Let $ A \in \mathcal{P}$. Define
\[
	\mathscr{L}_A \coloneqq \{B \in \sigma(\mathcal{P}): \mu_1(A \cap B) = \mu_2(A \cap B)\} 
.\] 
\begin{claim}[]
$ \mathcal{P} \subseteq \mathscr{L}_A$.
\end{claim}
Take any $ B \in \mathcal{P}$. Then $ A \cap B \in \mathcal{P}$ by $\pi$-system. Then $ \mu_1(A \cap B) = \mu_2(A \cap B) \implies B \in \mathscr{L}_A$.

\begin{claim}[]
$ \mathscr{L}_A$ is a $\lambda$-system.
\end{claim}
Since $ \mathcal{P} \subseteq \mathscr{L}_A$, by Dynkin's Theorem, $ \sigma(\mathcal{P}) \subseteq \mathscr{L}_A$.

	$ \mu_1,\mu_2$ are $ \sigma$-finite on $ \mathcal{P} \implies \ \exists \ A_1, A_2, \ldots \in \mathcal{P}$ such that $ \mathcal{P}=\bigcup_{ n =1}^{\infty} A_n$ and $ \mu_1(A_n)=\mu_2(A_n) < \infty$.
	Then by inclusion-exclusion,
	\begin{align*}
		\mu_{\alpha} \left( \bigcup_{ i= 1}^{ n} (A_i \cap B) \right) &= \ldots \\ 
	\end{align*}
	for $ \alpha=1,2$. Take any $ B \in \sigma(\mathcal{P})$. Then $ B \in \mathscr{L}_A$ by $ \sigma(\mathcal{P}) \subseteq \mathscr{L}_A$. Since the intersections of $ A_i$ is in $ \mathcal{P}$ as it is a $\pi$-system, this implies that the RHS of inclusion-exclusion agree for $ \alpha=1,2$. Then LHS also agree:
	\[
		\mu_1\left( \bigcup_{ i= 1}^{ n} (A_i \cap B) \right) = \mu_2\left( \bigcup_{ i= 1}^{ n} (A_i \cap B) \right) 
	.\] 
	Denote the union as $ C_n$. Since $ A_n$ cover $ \mathcal{P}$, and $ C_n \uparrow B$, by continuity from below, TODO $ \mu_1(B) = \mu_2(B)$.
\end{prf}

\end{document}

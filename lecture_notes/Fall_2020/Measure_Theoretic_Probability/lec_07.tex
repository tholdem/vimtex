\documentclass[class=article,crop=false]{standalone} 
%Fall 2020
% Some basic packages
\usepackage{standalone}[subpreambles=true]
\usepackage[utf8]{inputenc}
\usepackage[T1]{fontenc}
\usepackage{textcomp}
\usepackage[english]{babel}
\usepackage{url}
\usepackage{graphicx}
\usepackage{float}
\usepackage{enumitem}


\pdfminorversion=7

% Don't indent paragraphs, leave some space between them
\usepackage{parskip}

% Hide page number when page is empty
\usepackage{emptypage}
\usepackage{subcaption}
\usepackage{multicol}
\usepackage[dvipsnames]{xcolor}


% Math stuff
\usepackage{amsmath, amsfonts, mathtools, amsthm, amssymb}
% Fancy script capitals
\usepackage{mathrsfs}
\usepackage{cancel}
% Bold math
\usepackage{bm}
% Some shortcuts
\newcommand{\rr}{\ensuremath{\mathbb{R}}}
\newcommand{\zz}{\ensuremath{\mathbb{Z}}}
\newcommand{\qq}{\ensuremath{\mathbb{Q}}}
\newcommand{\nn}{\ensuremath{\mathbb{N}}}
\newcommand{\ff}{\ensuremath{\mathbb{F}}}
\newcommand{\cc}{\ensuremath{\mathbb{C}}}
\renewcommand\O{\ensuremath{\emptyset}}
\newcommand{\norm}[1]{{\left\lVert{#1}\right\rVert}}
\renewcommand{\vec}[1]{{\mathbf{#1}}}
\newcommand\allbold[1]{{\boldmath\textbf{#1}}}

% Put x \to \infty below \lim
\let\svlim\lim\def\lim{\svlim\limits}

%Make implies and impliedby shorter
\let\implies\Rightarrow
\let\impliedby\Leftarrow
\let\iff\Leftrightarrow
\let\epsilon\varepsilon

% Add \contra symbol to denote contradiction
\usepackage{stmaryrd} % for \lightning
\newcommand\contra{\scalebox{1.5}{$\lightning$}}

% \let\phi\varphi

% Command for short corrections
% Usage: 1+1=\correct{3}{2}

\definecolor{correct}{HTML}{009900}
\newcommand\correct[2]{\ensuremath{\:}{\color{red}{#1}}\ensuremath{\to }{\color{correct}{#2}}\ensuremath{\:}}
\newcommand\green[1]{{\color{correct}{#1}}}

% horizontal rule
\newcommand\hr{
    \noindent\rule[0.5ex]{\linewidth}{0.5pt}
}

% hide parts
\newcommand\hide[1]{}

% si unitx
\usepackage{siunitx}
\sisetup{locale = FR}

% Environments
\makeatother
% For box around Definition, Theorem, \ldots
\usepackage[framemethod=TikZ]{mdframed}
\mdfsetup{skipabove=1em,skipbelow=0em}

%definition
\newenvironment{defn}[1][]{%
\ifstrempty{#1}%
{\mdfsetup{%
frametitle={%
\tikz[baseline=(current bounding box.east),outer sep=0pt]
\node[anchor=east,rectangle,fill=Emerald]
{\strut Definition};}}
}%
{\mdfsetup{%
frametitle={%
\tikz[baseline=(current bounding box.east),outer sep=0pt]
\node[anchor=east,rectangle,fill=Emerald]
{\strut Definition:~#1};}}%
}%
\mdfsetup{innertopmargin=10pt,linecolor=Emerald,%
linewidth=2pt,topline=true,%
frametitleaboveskip=\dimexpr-\ht\strutbox\relax
}
\begin{mdframed}[]\relax%
\label{#1}}{\end{mdframed}}


%theorem
%\newcounter{thm}[section]\setcounter{thm}{0}
%\renewcommand{\thethm}{\arabic{section}.\arabic{thm}}
\newenvironment{thm}[1][]{%
%\refstepcounter{thm}%
\ifstrempty{#1}%
{\mdfsetup{%
frametitle={%
\tikz[baseline=(current bounding box.east),outer sep=0pt]
\node[anchor=east,rectangle,fill=blue!20]
%{\strut Theorem~\thethm};}}
{\strut Theorem};}}
}%
{\mdfsetup{%
frametitle={%
\tikz[baseline=(current bounding box.east),outer sep=0pt]
\node[anchor=east,rectangle,fill=blue!20]
%{\strut Theorem~\thethm:~#1};}}%
{\strut Theorem:~#1};}}%
}%
\mdfsetup{innertopmargin=10pt,linecolor=blue!20,%
linewidth=2pt,topline=true,%
frametitleaboveskip=\dimexpr-\ht\strutbox\relax
}
\begin{mdframed}[]\relax%
\label{#1}}{\end{mdframed}}


%lemma
\newenvironment{lem}[1][]{%
\ifstrempty{#1}%
{\mdfsetup{%
frametitle={%
\tikz[baseline=(current bounding box.east),outer sep=0pt]
\node[anchor=east,rectangle,fill=Dandelion]
{\strut Lemma};}}
}%
{\mdfsetup{%
frametitle={%
\tikz[baseline=(current bounding box.east),outer sep=0pt]
\node[anchor=east,rectangle,fill=Dandelion]
{\strut Lemma:~#1};}}%
}%
\mdfsetup{innertopmargin=10pt,linecolor=Dandelion,%
linewidth=2pt,topline=true,%
frametitleaboveskip=\dimexpr-\ht\strutbox\relax
}
\begin{mdframed}[]\relax%
\label{#1}}{\end{mdframed}}

%corollary
\newenvironment{coro}[1][]{%
\ifstrempty{#1}%
{\mdfsetup{%
frametitle={%
\tikz[baseline=(current bounding box.east),outer sep=0pt]
\node[anchor=east,rectangle,fill=CornflowerBlue]
{\strut Corollary};}}
}%
{\mdfsetup{%
frametitle={%
\tikz[baseline=(current bounding box.east),outer sep=0pt]
\node[anchor=east,rectangle,fill=CornflowerBlue]
{\strut Corollary:~#1};}}%
}%
\mdfsetup{innertopmargin=10pt,linecolor=CornflowerBlue,%
linewidth=2pt,topline=true,%
frametitleaboveskip=\dimexpr-\ht\strutbox\relax
}
\begin{mdframed}[]\relax%
\label{#1}}{\end{mdframed}}

%proof
\newenvironment{prf}[1][]{%
\ifstrempty{#1}%
{\mdfsetup{%
frametitle={%
\tikz[baseline=(current bounding box.east),outer sep=0pt]
\node[anchor=east,rectangle,fill=SpringGreen]
{\strut Proof};}}
}%
{\mdfsetup{%
frametitle={%
\tikz[baseline=(current bounding box.east),outer sep=0pt]
\node[anchor=east,rectangle,fill=SpringGreen]
{\strut Proof:~#1};}}%
}%
\mdfsetup{innertopmargin=10pt,linecolor=SpringGreen,%
linewidth=2pt,topline=true,%
frametitleaboveskip=\dimexpr-\ht\strutbox\relax
}
\begin{mdframed}[]\relax%
\label{#1}}{\qed\end{mdframed}}


\theoremstyle{definition}

\newmdtheoremenv[nobreak=true]{definition}{Definition}
\newmdtheoremenv[nobreak=true]{prop}{Proposition}
\newmdtheoremenv[nobreak=true]{theorem}{Theorem}
\newmdtheoremenv[nobreak=true]{corollary}{Corollary}
\newtheorem*{eg}{Example}
\theoremstyle{remark}
\newtheorem*{case}{Case}
\newtheorem*{notation}{Notation}
\newtheorem*{remark}{Remark}
\newtheorem*{note}{Note}
\newtheorem*{problem}{Problem}
\newtheorem*{observe}{Observe}
\newtheorem*{property}{Property}
\newtheorem*{intuition}{Intuition}


% End example and intermezzo environments with a small diamond (just like proof
% environments end with a small square)
\usepackage{etoolbox}
\AtEndEnvironment{vb}{\null\hfill$\diamond$}%
\AtEndEnvironment{intermezzo}{\null\hfill$\diamond$}%
% \AtEndEnvironment{opmerking}{\null\hfill$\diamond$}%

% Fix some spacing
% http://tex.stackexchange.com/questions/22119/how-can-i-change-the-spacing-before-theorems-with-amsthm
\makeatletter
\def\thm@space@setup{%
  \thm@preskip=\parskip \thm@postskip=0pt
}

% Fix some stuff
% %http://tex.stackexchange.com/questions/76273/multiple-pdfs-with-page-group-included-in-a-single-page-warning
\pdfsuppresswarningpagegroup=1


% My name
\author{Jaden Wang}



\begin{document}

\begin{thm}[4.2]
extends to an infinite number of $ \mathcal{A}_i$ and even an uncountable collection.
\end{thm}
\begin{coro}[]
	Suppose that $ A_{11}, A_{12},\ldots, A_{21}, A_{22}$ are independent events. 
Let $ \mathcal{F}_i$ be the $\sigma$-field generated by the $ i$th row. Then  $ \mathcal{F}_1,\mathcal{F}_2,\ldots$ are independent.
\end{coro}
\begin{prf}
	Define $ \mathcal{A}_i$ as the collection of all finite intersections of $ A_{i1}, A_{i2},\ldots$. Note that $ \mathcal{A}_i$ is a $ \pi$-system and $ \sigma(\mathcal{A}_i) = \mathcal{F}_i$.
\end{prf}

\begin{lem}[The Borel-Cantelli Lemmas]
\begin{enumerate}[label=\arabic*)]
	\item Let $ A_1, A_2,\ldots \in \mathcal{F}$. If $ \sum_{ n= 1}^{\infty} P(A_n) < \infty$, then $ P(\limsup_{  n} A_n) = 0$.
	\item Let $ A_1,A_2,\ldots \in \mathcal{F}$. If $ \sum_{ n= 1}^{\infty} = \infty$ and if the $ A_n$s are independent, then $ P\left( \limsup_{  n} A_n \right) =1$.
\end{enumerate}
\end{lem}

\begin{prf}[1]
	Suppose the sum is finite. Note that the "tail sums" $ infsum:	P(A_n) \to 0 $ as $ n \to \infty$. 
	\begin{align*}
		P(\limsup_{  n} A_n) &= P\left( \bigcap_{ m= 1}^{\infty} \bigcup_{k= m}^{\infty} A_k \right)  \\
				     &\leq P\left( \bigcup_{ k=m}^{\infty} A_k \right) \qquad \text{ monotonicity} \\
				     &\leq infsum:km P(A_k) \to 0 \text{ as } m \to \infty \text{ countable subadditivity}
	\end{align*}
	\[
		P(\limsup_{  n} A_n) = \lim_{ m \to \infty} P(\limsup_{  n} A_n) \leq \lim_{ m \to \infty} infsum:km P(A_k) =0
	.\] 
\end{prf}
\begin{prf}[2]
	Let $ A = \limsup_{  n} A_n = \bigcap_{ m= 1}^{\infty} \bigcup_{k= m}^{\infty} A_k $. Want to show $ P(A)=1$ and will show  $ P(A^{c})=0$.\\
\begin{note}[]
$ A_1,A_2$ independent $ \implies A_1^{c} and A_2^{c}$  independent. Also $ \implies A_1$ and $ A_2^{c}$ are independent. \\

Also, $ A^{c} = \bigcup_{ m= 1}^{\infty} \bigcap_{k= m}^{\infty} A_k^{c} $.

Also. $ e^{-x} \geq 1-x$ for $ x\geq 0$.
\end{note}

For fixed $ m$, 
 \begin{align*}
	 P\left( \bigcap_{ k=m }^{\infty} A_k^{c} \right) &\leq P\left( \bigcap_{ k=m}^{m+l} A_k^{c} \right) \\
							  &= \Pi_{k=m}^{m+1} P(A_k^{c}) \qquad \text{ by independence} \\
							  &= \Pi_{k=m}^{m+l} (1- P(A_k)) \\
							  &\leq \Pi_{k=m}^{m+l} e^{-P(A_k)}\\
							  &= e^{- \sum_{ k= m}^{ m+l} P(A_k)} \to 0 \text{ as } l \to \infty 
\end{align*}
Since $ \sum_{ n= 1}^{\infty} P(A_n) = \infty$. So
\begin{align*}
	P(A^{c}) &= P(\bigcup_{ m=1}^{\infty} \bigcap_{k=m}^{\infty} A_k^{c})\\
		 &\leq \sum_{ m= 1}^{\infty} P(\bigcap_{ m =1}^{\infty} P\left( \bigcap_{ k =m}^{\infty} A_k^{c} \right) )
		 &\leq \sum_{ m= 1}^{\infty} e^{-\sum_{ k=m}^{ m+l} P(A_k)}
\end{align*}
So 
\[
	P(A^{c}) = \lim_{ l \to \infty} P(A^{c}) \leq \lim_{ l \to \infty} \sum e^{\sum}
.\] 

\end{prf}

\begin{eg}[of why we need independence]
	Given $ (\Omega, \mathcal{F},P)$. Take any $ A \in \mathcal{F}$ with $ 0<P(A)<1$. Define  $ A_n = A$ for all $ n$ so it's not independent. Then 
	 \begin{align*}
		 \sum_{ n= 1}^{\infty} P(A_n) = \sum_{ n= 1}^{\infty} P(A) = \infty
	\end{align*}
	But
	\[
		P(A_n i.o.) = 
	.\] 
\end{eg}
\end{document}

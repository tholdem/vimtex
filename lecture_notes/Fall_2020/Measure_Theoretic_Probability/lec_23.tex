\documentclass[class=article,crop=false]{standalone} 
%Fall 2020
% Some basic packages
\usepackage{standalone}[subpreambles=true]
\usepackage[utf8]{inputenc}
\usepackage[T1]{fontenc}
\usepackage{textcomp}
\usepackage[english]{babel}
\usepackage{url}
\usepackage{graphicx}
\usepackage{float}
\usepackage{enumitem}


\pdfminorversion=7

% Don't indent paragraphs, leave some space between them
\usepackage{parskip}

% Hide page number when page is empty
\usepackage{emptypage}
\usepackage{subcaption}
\usepackage{multicol}
\usepackage[dvipsnames]{xcolor}


% Math stuff
\usepackage{amsmath, amsfonts, mathtools, amsthm, amssymb}
% Fancy script capitals
\usepackage{mathrsfs}
\usepackage{cancel}
% Bold math
\usepackage{bm}
% Some shortcuts
\newcommand{\rr}{\ensuremath{\mathbb{R}}}
\newcommand{\zz}{\ensuremath{\mathbb{Z}}}
\newcommand{\qq}{\ensuremath{\mathbb{Q}}}
\newcommand{\nn}{\ensuremath{\mathbb{N}}}
\newcommand{\ff}{\ensuremath{\mathbb{F}}}
\newcommand{\cc}{\ensuremath{\mathbb{C}}}
\renewcommand\O{\ensuremath{\emptyset}}
\newcommand{\norm}[1]{{\left\lVert{#1}\right\rVert}}
\renewcommand{\vec}[1]{{\mathbf{#1}}}
\newcommand\allbold[1]{{\boldmath\textbf{#1}}}

% Put x \to \infty below \lim
\let\svlim\lim\def\lim{\svlim\limits}

%Make implies and impliedby shorter
\let\implies\Rightarrow
\let\impliedby\Leftarrow
\let\iff\Leftrightarrow
\let\epsilon\varepsilon

% Add \contra symbol to denote contradiction
\usepackage{stmaryrd} % for \lightning
\newcommand\contra{\scalebox{1.5}{$\lightning$}}

% \let\phi\varphi

% Command for short corrections
% Usage: 1+1=\correct{3}{2}

\definecolor{correct}{HTML}{009900}
\newcommand\correct[2]{\ensuremath{\:}{\color{red}{#1}}\ensuremath{\to }{\color{correct}{#2}}\ensuremath{\:}}
\newcommand\green[1]{{\color{correct}{#1}}}

% horizontal rule
\newcommand\hr{
    \noindent\rule[0.5ex]{\linewidth}{0.5pt}
}

% hide parts
\newcommand\hide[1]{}

% si unitx
\usepackage{siunitx}
\sisetup{locale = FR}

% Environments
\makeatother
% For box around Definition, Theorem, \ldots
\usepackage[framemethod=TikZ]{mdframed}
\mdfsetup{skipabove=1em,skipbelow=0em}

%definition
\newenvironment{defn}[1][]{%
\ifstrempty{#1}%
{\mdfsetup{%
frametitle={%
\tikz[baseline=(current bounding box.east),outer sep=0pt]
\node[anchor=east,rectangle,fill=Emerald]
{\strut Definition};}}
}%
{\mdfsetup{%
frametitle={%
\tikz[baseline=(current bounding box.east),outer sep=0pt]
\node[anchor=east,rectangle,fill=Emerald]
{\strut Definition:~#1};}}%
}%
\mdfsetup{innertopmargin=10pt,linecolor=Emerald,%
linewidth=2pt,topline=true,%
frametitleaboveskip=\dimexpr-\ht\strutbox\relax
}
\begin{mdframed}[]\relax%
\label{#1}}{\end{mdframed}}


%theorem
%\newcounter{thm}[section]\setcounter{thm}{0}
%\renewcommand{\thethm}{\arabic{section}.\arabic{thm}}
\newenvironment{thm}[1][]{%
%\refstepcounter{thm}%
\ifstrempty{#1}%
{\mdfsetup{%
frametitle={%
\tikz[baseline=(current bounding box.east),outer sep=0pt]
\node[anchor=east,rectangle,fill=blue!20]
%{\strut Theorem~\thethm};}}
{\strut Theorem};}}
}%
{\mdfsetup{%
frametitle={%
\tikz[baseline=(current bounding box.east),outer sep=0pt]
\node[anchor=east,rectangle,fill=blue!20]
%{\strut Theorem~\thethm:~#1};}}%
{\strut Theorem:~#1};}}%
}%
\mdfsetup{innertopmargin=10pt,linecolor=blue!20,%
linewidth=2pt,topline=true,%
frametitleaboveskip=\dimexpr-\ht\strutbox\relax
}
\begin{mdframed}[]\relax%
\label{#1}}{\end{mdframed}}


%lemma
\newenvironment{lem}[1][]{%
\ifstrempty{#1}%
{\mdfsetup{%
frametitle={%
\tikz[baseline=(current bounding box.east),outer sep=0pt]
\node[anchor=east,rectangle,fill=Dandelion]
{\strut Lemma};}}
}%
{\mdfsetup{%
frametitle={%
\tikz[baseline=(current bounding box.east),outer sep=0pt]
\node[anchor=east,rectangle,fill=Dandelion]
{\strut Lemma:~#1};}}%
}%
\mdfsetup{innertopmargin=10pt,linecolor=Dandelion,%
linewidth=2pt,topline=true,%
frametitleaboveskip=\dimexpr-\ht\strutbox\relax
}
\begin{mdframed}[]\relax%
\label{#1}}{\end{mdframed}}

%corollary
\newenvironment{coro}[1][]{%
\ifstrempty{#1}%
{\mdfsetup{%
frametitle={%
\tikz[baseline=(current bounding box.east),outer sep=0pt]
\node[anchor=east,rectangle,fill=CornflowerBlue]
{\strut Corollary};}}
}%
{\mdfsetup{%
frametitle={%
\tikz[baseline=(current bounding box.east),outer sep=0pt]
\node[anchor=east,rectangle,fill=CornflowerBlue]
{\strut Corollary:~#1};}}%
}%
\mdfsetup{innertopmargin=10pt,linecolor=CornflowerBlue,%
linewidth=2pt,topline=true,%
frametitleaboveskip=\dimexpr-\ht\strutbox\relax
}
\begin{mdframed}[]\relax%
\label{#1}}{\end{mdframed}}

%proof
\newenvironment{prf}[1][]{%
\ifstrempty{#1}%
{\mdfsetup{%
frametitle={%
\tikz[baseline=(current bounding box.east),outer sep=0pt]
\node[anchor=east,rectangle,fill=SpringGreen]
{\strut Proof};}}
}%
{\mdfsetup{%
frametitle={%
\tikz[baseline=(current bounding box.east),outer sep=0pt]
\node[anchor=east,rectangle,fill=SpringGreen]
{\strut Proof:~#1};}}%
}%
\mdfsetup{innertopmargin=10pt,linecolor=SpringGreen,%
linewidth=2pt,topline=true,%
frametitleaboveskip=\dimexpr-\ht\strutbox\relax
}
\begin{mdframed}[]\relax%
\label{#1}}{\qed\end{mdframed}}


\theoremstyle{definition}

\newmdtheoremenv[nobreak=true]{definition}{Definition}
\newmdtheoremenv[nobreak=true]{prop}{Proposition}
\newmdtheoremenv[nobreak=true]{theorem}{Theorem}
\newmdtheoremenv[nobreak=true]{corollary}{Corollary}
\newtheorem*{eg}{Example}
\theoremstyle{remark}
\newtheorem*{case}{Case}
\newtheorem*{notation}{Notation}
\newtheorem*{remark}{Remark}
\newtheorem*{note}{Note}
\newtheorem*{problem}{Problem}
\newtheorem*{observe}{Observe}
\newtheorem*{property}{Property}
\newtheorem*{intuition}{Intuition}


% End example and intermezzo environments with a small diamond (just like proof
% environments end with a small square)
\usepackage{etoolbox}
\AtEndEnvironment{vb}{\null\hfill$\diamond$}%
\AtEndEnvironment{intermezzo}{\null\hfill$\diamond$}%
% \AtEndEnvironment{opmerking}{\null\hfill$\diamond$}%

% Fix some spacing
% http://tex.stackexchange.com/questions/22119/how-can-i-change-the-spacing-before-theorems-with-amsthm
\makeatletter
\def\thm@space@setup{%
  \thm@preskip=\parskip \thm@postskip=0pt
}

% Fix some stuff
% %http://tex.stackexchange.com/questions/76273/multiple-pdfs-with-page-group-included-in-a-single-page-warning
\pdfsuppresswarningpagegroup=1


% My name
\author{Jaden Wang}



\begin{document}
\begin{thm}[distributions of transformations]
	Let $ X=(X_1, \ldots, X_k)$ be a r.vec. Let $ g: \rr^{k} \to \rr^{i}$ be a measurable function. Define $ Y=g(X)$ (is measurable because it's composition of measurable functions). Let $ P_X$ be the distribution of  $ X$. Then
	\[
		P_Y(A) = P(Y \in A) = P(g(X) \in A) = P(X \in g^{-1}(A))= P_X(g^{-1}(A))
	.\] 
	\emph{i.e.} $ P_Y = P_X \circ g^{-1}$. 
\end{thm}

It's also called pushforward operation.


\subsection*{Projection Maps}
Consider the measurable spaces $ (\Omega_1,\mathcal{F}_1),\ldots,(\Omega_k,\mathcal{F}_k)$. Define $ \Omega= \Omega_1 \times \ldots \times \Omega_k$. Define
\[
	\mathcal{F} = \sigma(\mathcal{F}_1 \times \cdots \times \mathcal{F}_k)
.\]
Let $ \pi_i: \Omega \to \Omega_i$ be the projection map
\[
	\pi_i(\omega_1,\ldots,\omega_k) = \omega_i
.\] 
\begin{note}[]
	If $ A = A_1 \times \cdots \times A_k$ then $ \pi_i(A) = A_i$.
\end{note}

Let $ \pi_i ^{-1} (\mathcal{F}_i) = \{A \subseteq \Omega: \pi_i(A) \in \mathcal{F}_i\} $. Then
\[
	\mathcal{F}= \sigma \left( \bigcap_{ i= 1}^{k} \pi_{i}^{-1}(\mathcal{F}_i)  \right)
.\]
\begin{note}[]
The thing in the parenthesis do not equal. Their $\sigma$-fields equal.
\end{note}

Moreover (TODO requires proof!),
\[
	\pi_i ^{-1}(\mathcal{F}_i) \subseteq \mathcal{F} \implies \pi_i: \Omega \to \Omega_i \text{ is } \mathcal{F} /\mathcal{F}_i \text{ measurable} 
.\]

\subsection{Marginal Distribution}

Notice $ \pi_i(X) = X_i$ is a composition of measurable functions hence it's measurable.

Its distribution is $ P_{X_i} = P_X \circ \pi_i ^{-1}$. This is called \allbold{marginal distribution}.

Question: is there a corresponding marginal density? That is, we are looking for a function $ f_i$ such that $ P_{X_i}(A) = \int_A f_i(t) dt$. 

Guess: If we integrate the density over all the other r.v. To do this we need to change the order of integration using Fibini's Theorem. That is, we guess
\[
	f_i(x_i) = \int_{-\infty}^{\infty} \cdots \int_{-\infty}^{\infty} f(x_1,\ldots,x_k) dx_1 \ldots d x_{i-1} d x_{i+1} \ldots d x_k  
.\] 
And we would want to check that
\begin{align*}
	\int_{A} f_i (x_i) dx_i &= \int_{A}  \int_{-\infty}^{\infty} \cdots \int_{-\infty}^{\infty} f(x_1,\ldots,x_k) dx_1 \ldots d x_{i-1} d x_{i+1} \ldots d x_k d x_i  \\
				&\overset{\mathrm{Fibini}}{=}  \int_{-\infty}^{\infty} \cdots \int_{A} \cdots \int_{-\infty}^{\infty} f(x_1,\ldots,x_k) dx_1 \ldots d x_{i-1} d x_{i+1} \ldots d x_k \\
				&= P_{X} ( (- \infty, \infty) \times \cdots \times A \times \cdots \times (-\infty,\infty)) \\
				&= P_{X_i} (A) \\
\end{align*}

\begin{defn}[product space]
Let $ (\Omega_1, \mathcal{F}_1, \mu_1)$ and $ (\Omega_2,\mathcal{F}_2, \mu_2)$ be measure spaces. Let's define the product  $\sigma$-field $ \mathcal{F} \coloneqq \sigma(\mathcal{F}_1 \times \mathcal{F}_2)$. Define $ \Omega = \Omega_1 \times \Omega_2$. Then $ (\Omega,\mathcal{F})$ is called the \allbold{product space}. 
\end{defn}

\begin{prop}[]
Let $ \mathcal{F}$ be the product $\sigma$-field. Let $ A \in \mathcal{F}$. Define the "slices":
\[
	A \big|_{\omega_1} \coloneqq \{\omega_2 \in \Omega_2: (\omega_1, \omega_2) \in A\} 
.\]
Then $ A\big|_{\omega_1}$ and $ A\big|_{\omega_2}$ are in $ \mathcal{F}_2$ and $ \mathcal{F}_1$, respectively. 
\end{prop}

\begin{prf}
	Let $ \mathcal{G} = \{ B \subseteq \Omega: B\big|_{\omega_1} \in \mathcal{F}_2 \text{ and } B\big|_{\omega_2} \in \mathcal{F}_1 \ \forall \ \omega_1,\omega_2 \in \Omega\} $. We wish to show that $ \mathcal{G}$ is a $\sigma$-field.
	\begin{enumerate}[label=(\roman*)]
		\item Slices of $ \Omega$ are $ \Omega_1 \in \mathcal{F}_1, \Omega_2 \in \mathcal{F}_2 $. 
		\item Take $ B \in \mathcal{G}$. Notice
			\begin{align*}
				(B \big|_{\omega_1})^{c} &= \{\omega_2 \in \Omega_2: (\omega_1, \omega_2) \in B\}^{c}  \\
							 &= \{\omega_2 \in \Omega_2: (\omega_1,\omega_2) \in B^{c}\} \\
							 &= B^{c}\big|_{\omega_1}
			\end{align*}
			And we know $ (B\big|_{\omega_1})^{c} \in \mathcal{F}_2$ since $ \mathcal{F}_2$ is a $\sigma$-field.
		\item same technique as above.
	\end{enumerate}

	It remains to show that $ \mathcal{F}$ is a subset of $ \mathcal{G}$. Given $ A_i \in \mathcal{F}_i$, notice
	\begin{align*}
		(A_1 \times A_2 )\big|_{w_1} &= \{\omega_2 \in \Omega_2: (\omega_1,\omega_2) \in A_1 \times A_2\}  \\
					     &= \begin{cases}
						     A_2 &\text{ if } \omega_1 \in A_1\\
						     \O & \text{ if } \omega_1  \not\in A_1\\
					     \end{cases} \in \mathcal{F}_2 
	\end{align*}
	Likewise for the other. Thus $ A_1 \times A_2 \in \mathcal{G}$. Since $ \mathcal{F}$ is the smallest $\sigma$-field containing sets of the form $ A_1 \times A_2$, it follows that $ \mathcal{F} \subseteq \mathcal{G}$.
\end{prf}


\begin{thm}[]
\[
	\mathcal{B}(\rr^{n+m}) = \sigma(\mathcal{B}(\rr^{n})\times \mathcal{B}(\rr^{m}))
.\]
\end{thm}

\begin{defn}[product measure]
	Given two spaces $ (\Omega_1, \mathcal{F}_1,\mu_1), (\Omega_2, \mathcal{F}_2, \mu_2)$. Define a \allbold{product measure} as a measure $ \mu= \mu_1 \times \mu_2$ on $ (\Omega,\mathcal{F})$ such that 
	\[
		\mu(A_1 \times A_2) = \mu_1(A_1) \cdot \mu_2(A_2) \ \forall \ A_i \in \mathcal{F}_i
	.\] 
\end{defn}

\begin{thm}[Fubini's Theorem]
Given two $\sigma$-finite measure spaces and their product measure space. If $ f: \Omega \to \rr$ is a measurable function such that $ \int_{\Omega} |f|\ d \mu < \infty$, then
\begin{enumerate}[label=\arabic*)]
	\item For almost all $ \omega_1 \in \Omega_1$, $ f(\omega_1,\omega_2)$ is an integrable function of $ \omega_2$ (vice-versa).
	\item There exists an integrable $ h:\Omega_1 \to \rr$ such that 
		\[
			h(\omega_1) = \int_{\Omega_2} f(\omega_1,\omega_2) d \omega_2
		\]
		for almost all $ \omega_1$ (vice versa).
	\item \begin{align*}
			\int_{\Omega}f\ d \mu = \int_{\Omega_1 \times \Omega_2} f\ d( \mu_1 \times \mu_2) &= \int_{\Omega_1} \left[ \int_{\Omega_2} f(\omega_1,\omega_2) d \omega_2 \right] d \omega_1\\
													  &= \int_{\Omega_2} \left[ \int_{\Omega_1} f(\omega_1, \omega_2) d \omega_1 \right] d\omega_2 \\
		.\end{align*} 
\end{enumerate}
\end{thm}

\begin{lem}[1]
	Suppose $ \mu_1$ and $ \mu_2$ are $\sigma$-finite. For every $ A \in \mathcal{F} = \sigma(\mathcal{F}_1 \times \mathcal{F}_2)$, the function $ h: \Omega_1 \to \rr$ defined as 
	\[
		h(\omega_1) = \mu_2(A\big|_{\omega_1})
	\] 
is $ \mathcal{F}_1$-measurable.
\end{lem}

\begin{prf}
\begin{case}[1]
	$ \mu_1 , \mu_2$ are finite. Define $ \mathcal{G}= \{B \in \mathcal{F}: h_{B} \text{ is }\mathcal{F}_1 - \text{measurable}  \} $. We wish to show that $ \mathcal{G}$ is a $\lambda$-system.
	\begin{enumerate}[label=(\roman*)]
		\item $ \Omega \in \mathcal{G}$ because 
			\[
				h_{\Omega}(\omega_1) = \mu_2(\Omega\big|_{\omega_1}) = \mu_2(\Omega_2) < \infty
			.\] 
			Then $ h_\Omega $ is a finite constant function which is measurable as we have shown in Lecture 9.
		\item Take $ B \in \mathcal{G}$, 
			\begin{align*}
				h_{B^{c}}(\omega_1) &= \mu_2(B^{c}\big|_{\omega1}) \\
						    &= \mu_2((B\big|_{\omega_1})^{c}) \\
						    &= \mu_2(\Omega_2) - \mu_2(B\big|_{\omega_1} 
			\end{align*}
			Since a constant function minus a measurable function is still $ \mathcal{F}_1$ measurable, $ B^{c} \in \mathcal{G}$.
		\item Given $ B_n$ disjoint, using the same technique, we can show 
\[
	h_{\bigcup_{n} B_i } (\omega_1)= \sum_n h_{B_i} (\omega_1)
.\] 
The sum is measurable. Since the limit of measurable functions is measurable, taking $ n \to \infty$ gives us a $ \mathcal{F}_1$ measurable function.
			closure under countable disjoint union.
	\end{enumerate}
\end{case}
\end{prf}
\end{document}

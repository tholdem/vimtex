\documentclass[class=article,crop=false]{standalone} 
%Fall 2020
% Some basic packages
\usepackage{standalone}[subpreambles=true]
\usepackage[utf8]{inputenc}
\usepackage[T1]{fontenc}
\usepackage{textcomp}
\usepackage[english]{babel}
\usepackage{url}
\usepackage{graphicx}
\usepackage{float}
\usepackage{enumitem}


\pdfminorversion=7

% Don't indent paragraphs, leave some space between them
\usepackage{parskip}

% Hide page number when page is empty
\usepackage{emptypage}
\usepackage{subcaption}
\usepackage{multicol}
\usepackage[dvipsnames]{xcolor}


% Math stuff
\usepackage{amsmath, amsfonts, mathtools, amsthm, amssymb}
% Fancy script capitals
\usepackage{mathrsfs}
\usepackage{cancel}
% Bold math
\usepackage{bm}
% Some shortcuts
\newcommand{\rr}{\ensuremath{\mathbb{R}}}
\newcommand{\zz}{\ensuremath{\mathbb{Z}}}
\newcommand{\qq}{\ensuremath{\mathbb{Q}}}
\newcommand{\nn}{\ensuremath{\mathbb{N}}}
\newcommand{\ff}{\ensuremath{\mathbb{F}}}
\newcommand{\cc}{\ensuremath{\mathbb{C}}}
\renewcommand\O{\ensuremath{\emptyset}}
\newcommand{\norm}[1]{{\left\lVert{#1}\right\rVert}}
\renewcommand{\vec}[1]{{\mathbf{#1}}}
\newcommand\allbold[1]{{\boldmath\textbf{#1}}}

% Put x \to \infty below \lim
\let\svlim\lim\def\lim{\svlim\limits}

%Make implies and impliedby shorter
\let\implies\Rightarrow
\let\impliedby\Leftarrow
\let\iff\Leftrightarrow
\let\epsilon\varepsilon

% Add \contra symbol to denote contradiction
\usepackage{stmaryrd} % for \lightning
\newcommand\contra{\scalebox{1.5}{$\lightning$}}

% \let\phi\varphi

% Command for short corrections
% Usage: 1+1=\correct{3}{2}

\definecolor{correct}{HTML}{009900}
\newcommand\correct[2]{\ensuremath{\:}{\color{red}{#1}}\ensuremath{\to }{\color{correct}{#2}}\ensuremath{\:}}
\newcommand\green[1]{{\color{correct}{#1}}}

% horizontal rule
\newcommand\hr{
    \noindent\rule[0.5ex]{\linewidth}{0.5pt}
}

% hide parts
\newcommand\hide[1]{}

% si unitx
\usepackage{siunitx}
\sisetup{locale = FR}

% Environments
\makeatother
% For box around Definition, Theorem, \ldots
\usepackage[framemethod=TikZ]{mdframed}
\mdfsetup{skipabove=1em,skipbelow=0em}

%definition
\newenvironment{defn}[1][]{%
\ifstrempty{#1}%
{\mdfsetup{%
frametitle={%
\tikz[baseline=(current bounding box.east),outer sep=0pt]
\node[anchor=east,rectangle,fill=Emerald]
{\strut Definition};}}
}%
{\mdfsetup{%
frametitle={%
\tikz[baseline=(current bounding box.east),outer sep=0pt]
\node[anchor=east,rectangle,fill=Emerald]
{\strut Definition:~#1};}}%
}%
\mdfsetup{innertopmargin=10pt,linecolor=Emerald,%
linewidth=2pt,topline=true,%
frametitleaboveskip=\dimexpr-\ht\strutbox\relax
}
\begin{mdframed}[]\relax%
\label{#1}}{\end{mdframed}}


%theorem
%\newcounter{thm}[section]\setcounter{thm}{0}
%\renewcommand{\thethm}{\arabic{section}.\arabic{thm}}
\newenvironment{thm}[1][]{%
%\refstepcounter{thm}%
\ifstrempty{#1}%
{\mdfsetup{%
frametitle={%
\tikz[baseline=(current bounding box.east),outer sep=0pt]
\node[anchor=east,rectangle,fill=blue!20]
%{\strut Theorem~\thethm};}}
{\strut Theorem};}}
}%
{\mdfsetup{%
frametitle={%
\tikz[baseline=(current bounding box.east),outer sep=0pt]
\node[anchor=east,rectangle,fill=blue!20]
%{\strut Theorem~\thethm:~#1};}}%
{\strut Theorem:~#1};}}%
}%
\mdfsetup{innertopmargin=10pt,linecolor=blue!20,%
linewidth=2pt,topline=true,%
frametitleaboveskip=\dimexpr-\ht\strutbox\relax
}
\begin{mdframed}[]\relax%
\label{#1}}{\end{mdframed}}


%lemma
\newenvironment{lem}[1][]{%
\ifstrempty{#1}%
{\mdfsetup{%
frametitle={%
\tikz[baseline=(current bounding box.east),outer sep=0pt]
\node[anchor=east,rectangle,fill=Dandelion]
{\strut Lemma};}}
}%
{\mdfsetup{%
frametitle={%
\tikz[baseline=(current bounding box.east),outer sep=0pt]
\node[anchor=east,rectangle,fill=Dandelion]
{\strut Lemma:~#1};}}%
}%
\mdfsetup{innertopmargin=10pt,linecolor=Dandelion,%
linewidth=2pt,topline=true,%
frametitleaboveskip=\dimexpr-\ht\strutbox\relax
}
\begin{mdframed}[]\relax%
\label{#1}}{\end{mdframed}}

%corollary
\newenvironment{coro}[1][]{%
\ifstrempty{#1}%
{\mdfsetup{%
frametitle={%
\tikz[baseline=(current bounding box.east),outer sep=0pt]
\node[anchor=east,rectangle,fill=CornflowerBlue]
{\strut Corollary};}}
}%
{\mdfsetup{%
frametitle={%
\tikz[baseline=(current bounding box.east),outer sep=0pt]
\node[anchor=east,rectangle,fill=CornflowerBlue]
{\strut Corollary:~#1};}}%
}%
\mdfsetup{innertopmargin=10pt,linecolor=CornflowerBlue,%
linewidth=2pt,topline=true,%
frametitleaboveskip=\dimexpr-\ht\strutbox\relax
}
\begin{mdframed}[]\relax%
\label{#1}}{\end{mdframed}}

%proof
\newenvironment{prf}[1][]{%
\ifstrempty{#1}%
{\mdfsetup{%
frametitle={%
\tikz[baseline=(current bounding box.east),outer sep=0pt]
\node[anchor=east,rectangle,fill=SpringGreen]
{\strut Proof};}}
}%
{\mdfsetup{%
frametitle={%
\tikz[baseline=(current bounding box.east),outer sep=0pt]
\node[anchor=east,rectangle,fill=SpringGreen]
{\strut Proof:~#1};}}%
}%
\mdfsetup{innertopmargin=10pt,linecolor=SpringGreen,%
linewidth=2pt,topline=true,%
frametitleaboveskip=\dimexpr-\ht\strutbox\relax
}
\begin{mdframed}[]\relax%
\label{#1}}{\qed\end{mdframed}}


\theoremstyle{definition}

\newmdtheoremenv[nobreak=true]{definition}{Definition}
\newmdtheoremenv[nobreak=true]{prop}{Proposition}
\newmdtheoremenv[nobreak=true]{theorem}{Theorem}
\newmdtheoremenv[nobreak=true]{corollary}{Corollary}
\newtheorem*{eg}{Example}
\theoremstyle{remark}
\newtheorem*{case}{Case}
\newtheorem*{notation}{Notation}
\newtheorem*{remark}{Remark}
\newtheorem*{note}{Note}
\newtheorem*{problem}{Problem}
\newtheorem*{observe}{Observe}
\newtheorem*{property}{Property}
\newtheorem*{intuition}{Intuition}


% End example and intermezzo environments with a small diamond (just like proof
% environments end with a small square)
\usepackage{etoolbox}
\AtEndEnvironment{vb}{\null\hfill$\diamond$}%
\AtEndEnvironment{intermezzo}{\null\hfill$\diamond$}%
% \AtEndEnvironment{opmerking}{\null\hfill$\diamond$}%

% Fix some spacing
% http://tex.stackexchange.com/questions/22119/how-can-i-change-the-spacing-before-theorems-with-amsthm
\makeatletter
\def\thm@space@setup{%
  \thm@preskip=\parskip \thm@postskip=0pt
}

% Fix some stuff
% %http://tex.stackexchange.com/questions/76273/multiple-pdfs-with-page-group-included-in-a-single-page-warning
\pdfsuppresswarningpagegroup=1


% My name
\author{Jaden Wang}



\begin{document}
\section{Measures}
\begin{defn}[measurable space]
	Let $\Omega$ be a non-empty set, $\mathcal{F}$ be a $\sigma$-field on $\Omega$. $(\Omega,\mathcal{F})$ is a \allbold{measurable space}. 
\end{defn}

\begin{defn}[measure]
	A \allbold{measure} on this space is a function $\mu: \mathcal{F} \to [0,\infty]$ with the properties:
	\begin{enumerate}[label=(\roman*)]
		\item  $\mu(\O)=0$
		\item $A_1,A_2,\ldots \in \mathcal{F}$, disjoint $\implies \mu\left( \bigcup_{n= 1}^{\infty} A_n \right) = \sum_{ n=1}^{\infty} \mu(A_n)$ (countable additivity)
	\end{enumerate}
\end{defn}

\begin{defn}[probability meausre (defined on a field)]
	A set function $P$ on a  field $\mathcal{F}$ is a \allbold{probability measure} if it satisfies the following conditions:
\begin{enumerate}[label=(\roman*)]
	\item $0\leq P(A) \leq$ 1for $A \in \mathcal{F}$.
	\item $P(\O) = 0, P(\Omega)=1$.
	\item if $(A_n)$ is a disjoint sequence of  $\mathcal{F}$-sets and if $\bigcup_{ n =1}^{\infty} A_n \in \mathcal{F}$, then
		\[
			P\left( \bigcup_{ n =1}^{\infty} A_n \right) = \sum_{ n=1}^{\infty} P(A_n)
		.\] 
\end{enumerate}
\end{defn}

\begin{defn}[probability measure (defined on measurable space)]
	Let $(\Omega,\mathcal{F})$ be a measurable space. A \allbold{probability measure} is a function $P: \mathcal{F} \to [0,1]$ and has $P(\Omega) =1$.

\end{defn}

\begin{notation}
	$\mu=$ general measure, $P=$ probability measure,  $\lambda=$ Lebesgue measure.
\end{notation}

\begin{note}
\begin{enumerate}
	\item countable additivity $\implies$ finite additivity: if $A_1,\ldots,A_n \in \mathcal{F}$ disjoint. $\mu\left( \bigcup_{i= 1}^{ n} A_i \right) = \sum_{ i=1}^{ n} \mu(A_i)$.
	\item For any $A\in \mathcal{F}$, $1=P(\Omega)=P(A \cup A^{c}) = P(A) + P(A^{c}) \implies P(A^{c})=1-P(A)$.
	\item If $A,B \in \mathcal{F}$ and $A \subset B$, then $P(A) \leq P(B)$. Since $B=A \cup (B \setminus A) = A \cup (B \cap A^{c})$ which are disjoint.
	\item A measure is "countably subadditive". \emph{i.e.}: $A_1,\ldots \in \mathcal{F}$ are not necessarily disjoint, 
		\[
			\mu\left( \bigcup_{n= 1}^{\infty} A_n  \right) \leq \sum_{ n=1}^{\infty} \mu(A_n)
		.\] 
\end{enumerate}
\end{note}

\begin{prf}
\begin{align*}
	\mu\left( \bigcup_{n= 1}^{\infty} A_n \right) &= \mu(A_1 \cup \ldots) \\
						      &= \mu(A_1 \cup (A_2 \setminus A_1) \cup (A_3 \setminus A_2 \setminus A_1) \ldots) \\
						      &= \mu(A_1) + \mu(A_2 \setminus A_1) + \ldots \qquad  \text{(countable additivity)}  \\
						      &\leq \mu(A_1) + \mu(A_2) + \ldots 
\end{align*}
\end{prf}


\begin{defn}[probability space]
	Let $(\Omega,\mathcal{F})$ be a measurable space. Let $P$ be a probability measure on  $(\Omega,\mathcal{F})$. The triple $(\Omega,\mathcal{F},P)$ is called a \allbold{probability space}.
\end{defn}


\begin{eg}[]
	$\Omega=[0,1]$, $\mathcal{F}=\mathcal{B}([0,1]) =$ all Borel sets on $[0,1]$. Let $P =$ Lebesgue measure. Then $(\Omega,\mathcal{F},P)$ is a probability space.  
\end{eg}

\section{Existence and Extensions of Measures}

Let $\Omega$ be a non-empty set. Let $\mathcal{F}_0$ be a field on $\Omega$. Let $\mathcal{F} = \sigma(\mathcal{F}_0)$. Suppose $P$ is a probability measure on  $\mathcal{F}_0$. $(P(\bigcup_{n= 1}^{\infty} A_n) = \sum_{ n=1}^{\infty} P(A_n)$ only holds if $\bigcup_{n= 1}^{\infty} A_n \in \mathcal{F}_0$, disjointed.

Let's extend $P$ from  $\mathcal{F}_0$ to $\mathcal{F}$.

\begin{defn}[outer measure $P^* $]
	The \allbold{outer measure} is defined as 
	\[
		P^* (A) = \inf\left\{\sum_{ n=1}^{\infty} P(A_n): A_n \in \mathcal{F}_0, A \subset \bigcup_{n= 1}^{\infty} A_n\right\} \qquad \text{for any } A \in \Omega 
	\]
\end{defn}
\begin{note}[]
This is well-defined since there exists at least one cover $A \subset  \bigcup_{n= 1}^{\infty} A_n$ \emph{i.e.}:  $A_1 = \Omega, A_n=\O$ for $n \geq 2$.
\end{note}

\subsection{Properties of $P^* $}
\begin{enumerate}
	\item $P^* (\O)=0$. Pf: Take $A_n=\O$, $P(A_n)=0$. The infimum cannot go lower than 0.
	\item For any $A \subset  B$, $P^*(A) \leq P^* (B)$ (monotone). Pf: Any cover of $B$ is a cover of $A$.
	\item For any  $A \in \mathcal{F}_0$, $P^* (A) \leq P(A)$. Pf: Take $A_1=A, A_n=\O$ for $n\geq 2$. 
	\item "countable subadditivity":  \emph{i.e.} for any $A_1,\ldots,A_n \in \mathcal{F}$ 
		\[
			P^* \left( \bigcup_{n= 1}^{\infty} A_n \right) \leq \sum_{ n=1}^{\infty} P^* (A_n)
		.\]
\end{enumerate}
\begin{prf}
	Let $A=\bigcup_{n= 1}^{\infty} A_n$. Then we want to prove that $P^* (A) \leq \sum_{ n=1}^{\infty} P^* (A_n)$. For each $A_n$, $P^* (A_n) = \inf\{ \sum_{ k=1}^{\infty} P(A_{n_k}): A_{n_k} \in \mathcal{F}_0 \text{ and } A_n \subset \bigcup_{k=1}^{\infty} A_{n_k} \}$. Take any particular sequence that covers $A_n$. 
	\[
		A_n \subset \bigcup_{k= 1}^{\infty} A_{n_k}^*  
	.\] 
	Given $\epsilon>0$, by the definition of infimum, $\sum_{ k=1}^{\infty} P(A_{n_k}^* ) < P^* (A_n)+ \epsilon$.
	Do this with $\epsilon_n=\frac{\epsilon}{2^{n}}$.
	\[
		\sum_{ k=1}^{\infty} P(A_{n_k}^* ) < P^* (A_n) + \frac{\epsilon}{2^{n}}
	.\] 
	Sum both sides over $n$:
	 \begin{align*}
		 \sum_{ n=1}^{\infty} \sum_{ k=1}^{\infty} P(A_{n_k}^*) &< \sum_{ n=1}^{\infty} P^* (A_n) + \epsilon \\
		 \sum_{n,k} P(A_{n_k}^* ) &< \sum_{ n=1}^{\infty} P^*(A_n) +\epsilon
	\end{align*}
	Note that $A_n \subset \bigcup_{k= 1}^{\infty} A_{n_k}^*  \implies A := \bigcup_{n= 1}^{\infty} A_n \subset \bigcup_{n,k} A_{n_k}^*$. Therefore, $ P(A) \leq P\left(\bigcup_{n,k} A_{n_k}^*\right) \leq \sum_{n,k} P(A_{n_k}^* )$ by monotone and countable subadditivity of $P$. 
	By Property 3 of $P^*$, $P^*(A) \leq P(A)$. Putting all together, we obtain
\begin{align*}
	P^*(A) &\leq P(A)  \\
	       &\leq \sum_{ n=1}^{\infty} \sum_{ k=1}^{\infty} P(A_{n_k})  \\
	       &< \sum_{ n=1}^{\infty} \left( P^*(A_n) + \frac{\epsilon}{2^{n}}\right) \\
	       &= \sum_{ n=1}^{\infty} P^*(A_n) + \epsilon\\
\end{align*}
Let $\epsilon \to 0$, then $P^* (A) \leq \sum_{ n=1}^{\infty} P^*(A_n)$.
\end{prf}


\begin{defn}[inner measure]
\[
	P_*(A) := 1 - P^* (A^{c})
.\] 
\end{defn}

\begin{note}[]
	The inner and outer measures agree if $P^*(A^{c})=1-P^* (A)$. We proceed in defining $P$ for  $A \in \mathcal{F}$ as $P^* $ whenever this holds.
\end{note}

Consider all sets for which $P^* (A^{c})=1-P^* (A)$ and ..
\[
	\mathcal{M} := \{A \subset  \Omega: P^* (A \cap E) + P^* (A^{c} \cap E) = P^* (E) \quad \forall  E \subset  \Omega\}
.\] 
Fact: $P^* (\Omega) = 1$.
If we take $E=\Omega$, then we obtain the agreement. $\mathcal{M}$ is the collection of $P^* $-measurable sets.

\begin{note}[]
For any $A,E \subset \Omega$, 
\[
	P^* (E) = P^* ((A \cap E) \cup (A^{c} \cap E))
	\leq P^* (A \cap E) + P^* (A^{c} \cap E) \qquad \text{count add} 
.\] 
So $\mathcal{M}$ can be defined equivalently as $\mathcal{M}=\{A \subset \Omega: P^* (A \cap E) + P^* (A^{c} \cap E) \leq P^*(E) \quad \forall  E \subset  \Omega\}$ because $E$ is always smaller than the union on the LHS.
\end{note}
\end{document}

\documentclass[class=article,crop=false]{standalone} 
%Fall 2020
% Some basic packages
\usepackage{standalone}[subpreambles=true]
\usepackage[utf8]{inputenc}
\usepackage[T1]{fontenc}
\usepackage{textcomp}
\usepackage[english]{babel}
\usepackage{url}
\usepackage{graphicx}
\usepackage{float}
\usepackage{enumitem}


\pdfminorversion=7

% Don't indent paragraphs, leave some space between them
\usepackage{parskip}

% Hide page number when page is empty
\usepackage{emptypage}
\usepackage{subcaption}
\usepackage{multicol}
\usepackage[dvipsnames]{xcolor}


% Math stuff
\usepackage{amsmath, amsfonts, mathtools, amsthm, amssymb}
% Fancy script capitals
\usepackage{mathrsfs}
\usepackage{cancel}
% Bold math
\usepackage{bm}
% Some shortcuts
\newcommand{\rr}{\ensuremath{\mathbb{R}}}
\newcommand{\zz}{\ensuremath{\mathbb{Z}}}
\newcommand{\qq}{\ensuremath{\mathbb{Q}}}
\newcommand{\nn}{\ensuremath{\mathbb{N}}}
\newcommand{\ff}{\ensuremath{\mathbb{F}}}
\newcommand{\cc}{\ensuremath{\mathbb{C}}}
\renewcommand\O{\ensuremath{\emptyset}}
\newcommand{\norm}[1]{{\left\lVert{#1}\right\rVert}}
\renewcommand{\vec}[1]{{\mathbf{#1}}}
\newcommand\allbold[1]{{\boldmath\textbf{#1}}}

% Put x \to \infty below \lim
\let\svlim\lim\def\lim{\svlim\limits}

%Make implies and impliedby shorter
\let\implies\Rightarrow
\let\impliedby\Leftarrow
\let\iff\Leftrightarrow
\let\epsilon\varepsilon

% Add \contra symbol to denote contradiction
\usepackage{stmaryrd} % for \lightning
\newcommand\contra{\scalebox{1.5}{$\lightning$}}

% \let\phi\varphi

% Command for short corrections
% Usage: 1+1=\correct{3}{2}

\definecolor{correct}{HTML}{009900}
\newcommand\correct[2]{\ensuremath{\:}{\color{red}{#1}}\ensuremath{\to }{\color{correct}{#2}}\ensuremath{\:}}
\newcommand\green[1]{{\color{correct}{#1}}}

% horizontal rule
\newcommand\hr{
    \noindent\rule[0.5ex]{\linewidth}{0.5pt}
}

% hide parts
\newcommand\hide[1]{}

% si unitx
\usepackage{siunitx}
\sisetup{locale = FR}

% Environments
\makeatother
% For box around Definition, Theorem, \ldots
\usepackage[framemethod=TikZ]{mdframed}
\mdfsetup{skipabove=1em,skipbelow=0em}

%definition
\newenvironment{defn}[1][]{%
\ifstrempty{#1}%
{\mdfsetup{%
frametitle={%
\tikz[baseline=(current bounding box.east),outer sep=0pt]
\node[anchor=east,rectangle,fill=Emerald]
{\strut Definition};}}
}%
{\mdfsetup{%
frametitle={%
\tikz[baseline=(current bounding box.east),outer sep=0pt]
\node[anchor=east,rectangle,fill=Emerald]
{\strut Definition:~#1};}}%
}%
\mdfsetup{innertopmargin=10pt,linecolor=Emerald,%
linewidth=2pt,topline=true,%
frametitleaboveskip=\dimexpr-\ht\strutbox\relax
}
\begin{mdframed}[]\relax%
\label{#1}}{\end{mdframed}}


%theorem
%\newcounter{thm}[section]\setcounter{thm}{0}
%\renewcommand{\thethm}{\arabic{section}.\arabic{thm}}
\newenvironment{thm}[1][]{%
%\refstepcounter{thm}%
\ifstrempty{#1}%
{\mdfsetup{%
frametitle={%
\tikz[baseline=(current bounding box.east),outer sep=0pt]
\node[anchor=east,rectangle,fill=blue!20]
%{\strut Theorem~\thethm};}}
{\strut Theorem};}}
}%
{\mdfsetup{%
frametitle={%
\tikz[baseline=(current bounding box.east),outer sep=0pt]
\node[anchor=east,rectangle,fill=blue!20]
%{\strut Theorem~\thethm:~#1};}}%
{\strut Theorem:~#1};}}%
}%
\mdfsetup{innertopmargin=10pt,linecolor=blue!20,%
linewidth=2pt,topline=true,%
frametitleaboveskip=\dimexpr-\ht\strutbox\relax
}
\begin{mdframed}[]\relax%
\label{#1}}{\end{mdframed}}


%lemma
\newenvironment{lem}[1][]{%
\ifstrempty{#1}%
{\mdfsetup{%
frametitle={%
\tikz[baseline=(current bounding box.east),outer sep=0pt]
\node[anchor=east,rectangle,fill=Dandelion]
{\strut Lemma};}}
}%
{\mdfsetup{%
frametitle={%
\tikz[baseline=(current bounding box.east),outer sep=0pt]
\node[anchor=east,rectangle,fill=Dandelion]
{\strut Lemma:~#1};}}%
}%
\mdfsetup{innertopmargin=10pt,linecolor=Dandelion,%
linewidth=2pt,topline=true,%
frametitleaboveskip=\dimexpr-\ht\strutbox\relax
}
\begin{mdframed}[]\relax%
\label{#1}}{\end{mdframed}}

%corollary
\newenvironment{coro}[1][]{%
\ifstrempty{#1}%
{\mdfsetup{%
frametitle={%
\tikz[baseline=(current bounding box.east),outer sep=0pt]
\node[anchor=east,rectangle,fill=CornflowerBlue]
{\strut Corollary};}}
}%
{\mdfsetup{%
frametitle={%
\tikz[baseline=(current bounding box.east),outer sep=0pt]
\node[anchor=east,rectangle,fill=CornflowerBlue]
{\strut Corollary:~#1};}}%
}%
\mdfsetup{innertopmargin=10pt,linecolor=CornflowerBlue,%
linewidth=2pt,topline=true,%
frametitleaboveskip=\dimexpr-\ht\strutbox\relax
}
\begin{mdframed}[]\relax%
\label{#1}}{\end{mdframed}}

%proof
\newenvironment{prf}[1][]{%
\ifstrempty{#1}%
{\mdfsetup{%
frametitle={%
\tikz[baseline=(current bounding box.east),outer sep=0pt]
\node[anchor=east,rectangle,fill=SpringGreen]
{\strut Proof};}}
}%
{\mdfsetup{%
frametitle={%
\tikz[baseline=(current bounding box.east),outer sep=0pt]
\node[anchor=east,rectangle,fill=SpringGreen]
{\strut Proof:~#1};}}%
}%
\mdfsetup{innertopmargin=10pt,linecolor=SpringGreen,%
linewidth=2pt,topline=true,%
frametitleaboveskip=\dimexpr-\ht\strutbox\relax
}
\begin{mdframed}[]\relax%
\label{#1}}{\qed\end{mdframed}}


\theoremstyle{definition}

\newmdtheoremenv[nobreak=true]{definition}{Definition}
\newmdtheoremenv[nobreak=true]{prop}{Proposition}
\newmdtheoremenv[nobreak=true]{theorem}{Theorem}
\newmdtheoremenv[nobreak=true]{corollary}{Corollary}
\newtheorem*{eg}{Example}
\theoremstyle{remark}
\newtheorem*{case}{Case}
\newtheorem*{notation}{Notation}
\newtheorem*{remark}{Remark}
\newtheorem*{note}{Note}
\newtheorem*{problem}{Problem}
\newtheorem*{observe}{Observe}
\newtheorem*{property}{Property}
\newtheorem*{intuition}{Intuition}


% End example and intermezzo environments with a small diamond (just like proof
% environments end with a small square)
\usepackage{etoolbox}
\AtEndEnvironment{vb}{\null\hfill$\diamond$}%
\AtEndEnvironment{intermezzo}{\null\hfill$\diamond$}%
% \AtEndEnvironment{opmerking}{\null\hfill$\diamond$}%

% Fix some spacing
% http://tex.stackexchange.com/questions/22119/how-can-i-change-the-spacing-before-theorems-with-amsthm
\makeatletter
\def\thm@space@setup{%
  \thm@preskip=\parskip \thm@postskip=0pt
}

% Fix some stuff
% %http://tex.stackexchange.com/questions/76273/multiple-pdfs-with-page-group-included-in-a-single-page-warning
\pdfsuppresswarningpagegroup=1


% My name
\author{Jaden Wang}



\begin{document}
\section{}
\begin{lem}[1]
	$\mathcal{M}$ is a field.
\end{lem}
\begin{prf}
\begin{enumerate}[label=(\roman*)]
	\item Since $\quad \forall E \subset \Omega$, 
		 \begin{align*}
			 P^* (\O \cap E) + P^* (\O^{c} \cap E) &= P^* (\O) + P^* (E) \\
							       &= P^* (E) 
		\end{align*}
		So $\O \in \mathcal{M}$.
	\item $A \in \mathcal{M} \implies A^{c} \in \mathcal{M}$ by the symmetry of the definition of $\mathcal{M}$.
	\item Given $A_1,A_2 \in \mathcal{M}$. Want $A_1 \cup A_2 \in \mathcal{M}$. We can show that $A_1 \cap A_2 \in \mathcal{M}$.
		\begin{align*}
			& \quad  P^* ((A_1 \cap A_2) \cap E) + P^* ((A_1 \cap A_2)^{c} \cap E) \\
			&= P^* (A_1 \cap A_2 \cap E) + P^* ((A_1 \cap E  \cap A_2^{c}) \\ 
										      &\quad \cup (A_2 \cap E \cap A_1^{c}) \cup (A_1^{c} \cap A_2^{c} \cap E) ) \qquad \qquad \qquad   (1) \\
	&\leq P^* (A_1 \cap A_2 \cap E) + P^* (A_1 \cap E \cap A_2^{c})\\
	&\quad  + P^* (A_2 \cap E \cap A_1^{c}) + P^* (A_1^{c} \cap A_2^{c} \cap E) \\
	&= A+B+C+D 
		\end{align*}
		\begin{align*}
			A+B &= P^* (A_2 \cap (A_1 \cap E)) + P^* (A_2^{c} \cap (A_1 \cap E)) \\
			    & \leq P^* (A_1 \cap E) \text{ since }  A_2 \in \mathcal{M} \\
			C+D &= P^* (A_2 \cap (E \cap A_1^{c})) + P^* (A_2^{c} \cap (E \cap A_1^{c}))\\ 
			    &\leq P^* (A_1^{c} \cap E) \\
		\end{align*}
		So $(1)\leq (A+B)+(C+D) \leq P^* (A_1 \cap E) + P^* (A_1^{c} \cap E) \leq P^* (E)$ since $A_1 \in \mathcal{M}$.
\end{enumerate}
\end{prf}
\begin{lem}[2]
Countable additivity holds for sets in $\mathcal{M}$.
\end{lem}
\begin{note}[]
	$A_1,A_2,\ldots \in \mathcal{M}$, disjoint then $P^* (\bigcup_{n= 1}^{\infty} A_n)=\sum_{ n=1}^{\infty} P^* (A_n)$
\end{note}
\begin{prf}
\begin{enumerate}[label=\arabic*)]
	\item Prove finite additivity: given $A_1,A_2 \in \mathcal{M}$, disjoint, and $E \in \Omega$. Since $(E \cap A_1) \cup (E \cap A_2)$ are disjoint, $A_1 \cap [E \cap (A_1 \cup A_2)] = A_1 \cap E$. Also $A_2 \subset A^{c}$. 
		\begin{align*}
			P^* (E\cap (A_1 \cup A_2)) &= P^* (A_1 \cap [E \cap (A_1 \cap A_2)]) + P^* (A_1^{c} \cap [E \cap (A_1 \cup A_2)]) \\
						  &=  P^* (A_1 \cap E) + P^* (A_2 \cap E)\\
		\end{align*}
	\item Countable additivity: $A_1, A_2\ldots \in \mathcal{M}$, disjoint
		\begin{align*}
			P^* (E \cap \bigcup_{n= 1}^{\infty} A_n) &= P^* (\bigcup_{n= 1}^{\infty} (E \cap A_n)) \\
			&\leq \sum_{ n=1}^{\infty} P^* (E \cap A_n)
		\end{align*}
		On the other hand:
		\begin{align*}
			P^* \left(E\cap \left( \bigcup_{n= 1}^{\infty} A_n \right) \right) &\geq P^* \left( E \cap \bigcup_{n= 1}^{m} A_n \right) \qquad \text{by monotonicity} \\
										&= \sum_{ n=1}^{m} P^* (E\cap A_n) \text{ by 1) above}  \\
		\end{align*}
		Let $n \to \infty$
		$P^* (E \cap \bigcup_{n= 1}^{\infty} A_n) \geq \sum_{ n=1}^{\infty} P^* (E\cap A_n)$
\end{enumerate}
\end{prf}

\begin{lem}[3]
$\mathcal{M}$ is a $\sigma$-field.
\end{lem}
\begin{prf}
We only need to show that it is closed under countable unions. Given $A_1,A_2,\ldots \in \mathcal{M}$. Let $A=\bigcup_{n= 1}^{\infty} A_n$. We want to show that $A \in \mathcal{M}$.

Assume $A_1, A_2..$ are disjoint. If not write $A = A_1 \cup (A_2\setminus A_1) \cap \ldots$ which is a disjoint union.

Let $B_m:= \bigcup_{n= 1}^{m} A_n$. Note that $B_m \in \mathcal{M}$ since $\mathcal{M}$ is a field. Notice $B_m \subset \bigcup_{n= 1}^{\infty} A_n = A \implies A^{c} \subset  B_m ^{c}$. So $\quad \forall \quad E \subset \Omega$,  
\begin{align*}
	P^* (E) &= P^* (B_m \cap E) + P^* (B_m^{c} \cap E)\\
		&= P^* \left( \bigcup_{n= 1}^{m} (A_n \cap E)  \right) + P^* (B_m ^{c} \cap E) \\
		&\geq \sum_{ n=1}^{m} P^* (A_n \cap E) + P^* (A^{c} \cap E)
\end{align*}
Let $m \to \infty$, 
\begin{align*}
	P^* (E) &\geq \sum_{ n=1}^{\infty} P^* (A_n \cap E) + P^* (A^{c} \cap E) \\
		&= P^* \left( \bigcup_{n= 1}^{\infty} (A_n \cap E) \right) + P^* (A^{c} \cap E)  \\
		&= P^* \left( \bigcup_{n= 1}^{\infty} A_n \cap E \right) + P^* (A^{c} \cap E) \\
		&= P^* (A \cap E) + P^* (A^{c} \cap E)
\end{align*}
Hence $A \in \mathcal{M}$.
\end{prf}

Since $P(\Omega) =1$, by Lemma 5, $P^* (\Omega)=1$. Now we have $P^* (\O) = 0, P ^* (\Omega) =1, \O \subset  A \subset  \Omega \implies 0 \leq P^* (A) \leq 1$, and $P^*$ is countably additive on  $\mathcal{M}$. It follows that $P^*$ is a probability measure on  $\mathcal{M}$ and $P^*=P$ for set on  $\mathcal{F}_0 \subset \mathcal{M}$.
\end{document}

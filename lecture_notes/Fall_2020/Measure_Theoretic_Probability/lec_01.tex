\documentclass[class=article, crop=false]{standalone} 
%Fall 2020
% Some basic packages
\usepackage{standalone}[subpreambles=true]
\usepackage[utf8]{inputenc}
\usepackage[T1]{fontenc}
\usepackage{textcomp}
\usepackage[english]{babel}
\usepackage{url}
\usepackage{graphicx}
\usepackage{float}
\usepackage{enumitem}


\pdfminorversion=7

% Don't indent paragraphs, leave some space between them
\usepackage{parskip}

% Hide page number when page is empty
\usepackage{emptypage}
\usepackage{subcaption}
\usepackage{multicol}
\usepackage[dvipsnames]{xcolor}


% Math stuff
\usepackage{amsmath, amsfonts, mathtools, amsthm, amssymb}
% Fancy script capitals
\usepackage{mathrsfs}
\usepackage{cancel}
% Bold math
\usepackage{bm}
% Some shortcuts
\newcommand{\rr}{\ensuremath{\mathbb{R}}}
\newcommand{\zz}{\ensuremath{\mathbb{Z}}}
\newcommand{\qq}{\ensuremath{\mathbb{Q}}}
\newcommand{\nn}{\ensuremath{\mathbb{N}}}
\newcommand{\ff}{\ensuremath{\mathbb{F}}}
\newcommand{\cc}{\ensuremath{\mathbb{C}}}
\renewcommand\O{\ensuremath{\emptyset}}
\newcommand{\norm}[1]{{\left\lVert{#1}\right\rVert}}
\renewcommand{\vec}[1]{{\mathbf{#1}}}
\newcommand\allbold[1]{{\boldmath\textbf{#1}}}

% Put x \to \infty below \lim
\let\svlim\lim\def\lim{\svlim\limits}

%Make implies and impliedby shorter
\let\implies\Rightarrow
\let\impliedby\Leftarrow
\let\iff\Leftrightarrow
\let\epsilon\varepsilon

% Add \contra symbol to denote contradiction
\usepackage{stmaryrd} % for \lightning
\newcommand\contra{\scalebox{1.5}{$\lightning$}}

% \let\phi\varphi

% Command for short corrections
% Usage: 1+1=\correct{3}{2}

\definecolor{correct}{HTML}{009900}
\newcommand\correct[2]{\ensuremath{\:}{\color{red}{#1}}\ensuremath{\to }{\color{correct}{#2}}\ensuremath{\:}}
\newcommand\green[1]{{\color{correct}{#1}}}

% horizontal rule
\newcommand\hr{
    \noindent\rule[0.5ex]{\linewidth}{0.5pt}
}

% hide parts
\newcommand\hide[1]{}

% si unitx
\usepackage{siunitx}
\sisetup{locale = FR}

% Environments
\makeatother
% For box around Definition, Theorem, \ldots
\usepackage[framemethod=TikZ]{mdframed}
\mdfsetup{skipabove=1em,skipbelow=0em}

%definition
\newenvironment{defn}[1][]{%
\ifstrempty{#1}%
{\mdfsetup{%
frametitle={%
\tikz[baseline=(current bounding box.east),outer sep=0pt]
\node[anchor=east,rectangle,fill=Emerald]
{\strut Definition};}}
}%
{\mdfsetup{%
frametitle={%
\tikz[baseline=(current bounding box.east),outer sep=0pt]
\node[anchor=east,rectangle,fill=Emerald]
{\strut Definition:~#1};}}%
}%
\mdfsetup{innertopmargin=10pt,linecolor=Emerald,%
linewidth=2pt,topline=true,%
frametitleaboveskip=\dimexpr-\ht\strutbox\relax
}
\begin{mdframed}[]\relax%
\label{#1}}{\end{mdframed}}


%theorem
%\newcounter{thm}[section]\setcounter{thm}{0}
%\renewcommand{\thethm}{\arabic{section}.\arabic{thm}}
\newenvironment{thm}[1][]{%
%\refstepcounter{thm}%
\ifstrempty{#1}%
{\mdfsetup{%
frametitle={%
\tikz[baseline=(current bounding box.east),outer sep=0pt]
\node[anchor=east,rectangle,fill=blue!20]
%{\strut Theorem~\thethm};}}
{\strut Theorem};}}
}%
{\mdfsetup{%
frametitle={%
\tikz[baseline=(current bounding box.east),outer sep=0pt]
\node[anchor=east,rectangle,fill=blue!20]
%{\strut Theorem~\thethm:~#1};}}%
{\strut Theorem:~#1};}}%
}%
\mdfsetup{innertopmargin=10pt,linecolor=blue!20,%
linewidth=2pt,topline=true,%
frametitleaboveskip=\dimexpr-\ht\strutbox\relax
}
\begin{mdframed}[]\relax%
\label{#1}}{\end{mdframed}}


%lemma
\newenvironment{lem}[1][]{%
\ifstrempty{#1}%
{\mdfsetup{%
frametitle={%
\tikz[baseline=(current bounding box.east),outer sep=0pt]
\node[anchor=east,rectangle,fill=Dandelion]
{\strut Lemma};}}
}%
{\mdfsetup{%
frametitle={%
\tikz[baseline=(current bounding box.east),outer sep=0pt]
\node[anchor=east,rectangle,fill=Dandelion]
{\strut Lemma:~#1};}}%
}%
\mdfsetup{innertopmargin=10pt,linecolor=Dandelion,%
linewidth=2pt,topline=true,%
frametitleaboveskip=\dimexpr-\ht\strutbox\relax
}
\begin{mdframed}[]\relax%
\label{#1}}{\end{mdframed}}

%corollary
\newenvironment{coro}[1][]{%
\ifstrempty{#1}%
{\mdfsetup{%
frametitle={%
\tikz[baseline=(current bounding box.east),outer sep=0pt]
\node[anchor=east,rectangle,fill=CornflowerBlue]
{\strut Corollary};}}
}%
{\mdfsetup{%
frametitle={%
\tikz[baseline=(current bounding box.east),outer sep=0pt]
\node[anchor=east,rectangle,fill=CornflowerBlue]
{\strut Corollary:~#1};}}%
}%
\mdfsetup{innertopmargin=10pt,linecolor=CornflowerBlue,%
linewidth=2pt,topline=true,%
frametitleaboveskip=\dimexpr-\ht\strutbox\relax
}
\begin{mdframed}[]\relax%
\label{#1}}{\end{mdframed}}

%proof
\newenvironment{prf}[1][]{%
\ifstrempty{#1}%
{\mdfsetup{%
frametitle={%
\tikz[baseline=(current bounding box.east),outer sep=0pt]
\node[anchor=east,rectangle,fill=SpringGreen]
{\strut Proof};}}
}%
{\mdfsetup{%
frametitle={%
\tikz[baseline=(current bounding box.east),outer sep=0pt]
\node[anchor=east,rectangle,fill=SpringGreen]
{\strut Proof:~#1};}}%
}%
\mdfsetup{innertopmargin=10pt,linecolor=SpringGreen,%
linewidth=2pt,topline=true,%
frametitleaboveskip=\dimexpr-\ht\strutbox\relax
}
\begin{mdframed}[]\relax%
\label{#1}}{\qed\end{mdframed}}


\theoremstyle{definition}

\newmdtheoremenv[nobreak=true]{definition}{Definition}
\newmdtheoremenv[nobreak=true]{prop}{Proposition}
\newmdtheoremenv[nobreak=true]{theorem}{Theorem}
\newmdtheoremenv[nobreak=true]{corollary}{Corollary}
\newtheorem*{eg}{Example}
\theoremstyle{remark}
\newtheorem*{case}{Case}
\newtheorem*{notation}{Notation}
\newtheorem*{remark}{Remark}
\newtheorem*{note}{Note}
\newtheorem*{problem}{Problem}
\newtheorem*{observe}{Observe}
\newtheorem*{property}{Property}
\newtheorem*{intuition}{Intuition}


% End example and intermezzo environments with a small diamond (just like proof
% environments end with a small square)
\usepackage{etoolbox}
\AtEndEnvironment{vb}{\null\hfill$\diamond$}%
\AtEndEnvironment{intermezzo}{\null\hfill$\diamond$}%
% \AtEndEnvironment{opmerking}{\null\hfill$\diamond$}%

% Fix some spacing
% http://tex.stackexchange.com/questions/22119/how-can-i-change-the-spacing-before-theorems-with-amsthm
\makeatletter
\def\thm@space@setup{%
  \thm@preskip=\parskip \thm@postskip=0pt
}

% Fix some stuff
% %http://tex.stackexchange.com/questions/76273/multiple-pdfs-with-page-group-included-in-a-single-page-warning
\pdfsuppresswarningpagegroup=1


% My name
\author{Jaden Wang}


\begin{document}
\section{Probability Measures}
Let $\Omega$ be a non-empty set. 
(Think of it as a sample space = set of all possible outcomes of an experiment involving randomness.)
Ex: Flip a coin twice.
\[
	\Omega=\{HH,HT,TH,TT\}
.\]

Let $ A \subset \Omega$. 
Ex: $A=$ an event, $\{HT,TH,TT\}$.
With fair coin $P_r (A) = \frac{3}{4}$

To measure a 2D blob: usual length, area, volume ..

We want to assign probability to subsets of $\Omega$\\
$P:$ subsets of  $\Omega \to [0,1]$.

\begin{defn}[Field]
	Let $\Omega$ be a non-empty set. Let $\mathcal{F}$ a collection of subsets of $\Omega$. $\mathcal{F}$ is called a \allbold{field (or algebra)} if
	\begin{enumerate}
		\item $\Omega \in \mathcal{F}$
		\item $A\in \mathcal{F} \implies A^{c} \in \mathcal{F} $ ("closed under complements")
		\item Given $A_1,A_2,\ldots \in \mathcal{F} \implies \bigcup_{i= 1}^{ n} A_i \in \mathcal{F}$ ("closed under finite unions"). 
	\end{enumerate}
\end{defn}
\begin{remark}
We can also use "closed under intersection" to define because of De Morgan's law.
\end{remark}
\[
	\left( \bigcap_{i= 1}^{ n} A_i \right)^{c} = \bigcup_{i= 1}^{ n} A_i^{c}
.\] 
\begin{defn}[$\sigma$-field]
	If (iii) is replaced by a \emph{countable} union, e.g. $\bigcup_{n= 1}^{\infty} A_n $, then $\mathcal{F}$ is called a \allbold{$\sigma$-field}.  
\end{defn}
\begin{remark}
$\sigma$ field is stronger than field.
\end{remark}
\begin{eg}[$\sigma$-fields]

\end{eg}
$\Omega$.\\

\begin{itemize}
	\item $\mathcal{F}=\{\O,\Omega\}$
	\item $\mathcal{F} =$ the power sets (all possible subsets of $\Omega$). (Billingsley:"power class").
	\item Take any $A\subset \Omega$. Define $\mathcal{F}=\{\O,A,A^{c},\Omega\}$
\end{itemize}

A counterexample of a field that is not a $\sigma$-field.
Let $\Omega = \rr$, $\mathcal{F}=$ the empty set and all finite disjoint unions of things like $(a,b]$ and/or  $(a,\infty)$ for $-\infty\leq a<b<\infty$.\\
(i) $\Omega=(-\infty,\infty) \in \mathcal{F}$
(ii) Complements: Ex: $(a,b]^{c} =(-\infty,a] \cup (b,\infty)$. More generally a set in $\mathcal{F}$ has the form a bunch of disjointed $(]$ or $(\infty)$. \\
(iii)\\ 
\begin{case}[1]
$A_1,A_2$ have no overlaps. Then $A_1 \cup A_2$ is still a finite disjointed union.
\end{case}
Case 2: Some overlap, then $A_1 \cup A_2$ is still the same type of interval.\\
However, it is not a $\sigma$-field. We want to find a countable collection of sets that isn't in here. Let $A_n=(0,1-\frac{1}{n}]$. Then $\bigcup_{n= 1}^{ \infty} A_n =(0,1) \not \in \mathcal{F} $.
\begin{defn}[$\sigma(\mathcal{A})$]
	Let $\mathcal{A}$ be a collection of subsets of $\Omega$. The \allbold{ $\sigma$-field generated by $\mathcal{A}$} is the smallest $\sigma$-field containing all the sets in $\mathcal{A}$. We write it as $\sigma(\mathcal{A})$.
\end{defn}

Note: 
\begin{itemize}
	\item If $\mathcal{F}$ is a $\sigma$-field, $\sigma(\mathcal{F})=\mathcal{F}$.
	\item If $\mathcal{F}$ is a $\sigma$-field and $\mathcal{A}\subset \mathcal{F}$, then $\mathcal{A} \subset \mathcal{F}$.
	\item $\sigma(\mathcal{A}) = \bigcap \mathcal{F}$ the intersection over all $\sigma$-field that contain $\mathcal{A}$.
	\item $\mathcal{A}\subset \mathcal{A}' \implies  \sigma(\mathcal{A}) \subset \sigma(\mathcal{A}')$ 
\end{itemize}

Example:
\begin{itemize}
	\item $A \subset \Omega, \mathcal{A}=\{A\} \implies \sigma(\mathcal{A})=\sigma(A)=\{\O,A,A^{c},\Omega\}$
	\item \allbold{Borel Sets} in $\rr$. Let $\Omega=\rr$, $\mathcal{A}=$ all open finite intervals in $\rr$. $\mathcal{B}(\rr)=\sigma(\mathcal{A})$. Note: include all half open intervals. It also contains single points: $\{a\}=\bigcap_{n= 1}^{ \infty} (a,a+\frac{1}{n}]$. Cantor sets are not in here. 
\end{itemize}
\subsection{Unions and Intersections of $\sigma$-Fields}.
\begin{itemize}
	\item The union of two $\sigma$-Fields is not a necessarily a $\sigma$-Field. Take $A,B \subset \Omega, A \neq B, \sigma(A)=\{\O,A,A^{c},\Omega\}, \sigma(B)={\O,B,B^{c},\Omega}$. So 
			$\sigma(A) \cup \sigma(B)$ 
		\item The intersection of two $\sigma$-Fields is a $\sigma$-Field. Let $\mathcal{F}_1,\mathcal{F}_2$ be two $\sigma$-Fields. $\mathcal{F}_1 \cap \mathcal{F}_2$ 
			(i) $\Omega \in \mathcal{F}_1 \cap \mathcal{F}_2$ since $\O \in \mathcal{F}_1$ and $\O \in  \mathcal{F}_2$.
			(ii) Let $A \in \mathcal{F}_1 \cap \mathcal{F}_2$ 
			$A \in \bigcap_{n= 1}^{ \infty} \mathcal{F}_n \implies A \in \mathcal{F}_n \forall n \implies A^{c} \in \mathcal{F}_n \forall n \implies A^{c} \in \bigcap_{n= 1}^{ \infty} \mathcal{F}_n $
\end{itemize}
\end{document}

\documentclass[class=article,crop=false]{standalone} 
%Fall 2020
% Some basic packages
\usepackage{standalone}[subpreambles=true]
\usepackage[utf8]{inputenc}
\usepackage[T1]{fontenc}
\usepackage{textcomp}
\usepackage[english]{babel}
\usepackage{url}
\usepackage{graphicx}
\usepackage{float}
\usepackage{enumitem}


\pdfminorversion=7

% Don't indent paragraphs, leave some space between them
\usepackage{parskip}

% Hide page number when page is empty
\usepackage{emptypage}
\usepackage{subcaption}
\usepackage{multicol}
\usepackage[dvipsnames]{xcolor}


% Math stuff
\usepackage{amsmath, amsfonts, mathtools, amsthm, amssymb}
% Fancy script capitals
\usepackage{mathrsfs}
\usepackage{cancel}
% Bold math
\usepackage{bm}
% Some shortcuts
\newcommand{\rr}{\ensuremath{\mathbb{R}}}
\newcommand{\zz}{\ensuremath{\mathbb{Z}}}
\newcommand{\qq}{\ensuremath{\mathbb{Q}}}
\newcommand{\nn}{\ensuremath{\mathbb{N}}}
\newcommand{\ff}{\ensuremath{\mathbb{F}}}
\newcommand{\cc}{\ensuremath{\mathbb{C}}}
\renewcommand\O{\ensuremath{\emptyset}}
\newcommand{\norm}[1]{{\left\lVert{#1}\right\rVert}}
\renewcommand{\vec}[1]{{\mathbf{#1}}}
\newcommand\allbold[1]{{\boldmath\textbf{#1}}}

% Put x \to \infty below \lim
\let\svlim\lim\def\lim{\svlim\limits}

%Make implies and impliedby shorter
\let\implies\Rightarrow
\let\impliedby\Leftarrow
\let\iff\Leftrightarrow
\let\epsilon\varepsilon

% Add \contra symbol to denote contradiction
\usepackage{stmaryrd} % for \lightning
\newcommand\contra{\scalebox{1.5}{$\lightning$}}

% \let\phi\varphi

% Command for short corrections
% Usage: 1+1=\correct{3}{2}

\definecolor{correct}{HTML}{009900}
\newcommand\correct[2]{\ensuremath{\:}{\color{red}{#1}}\ensuremath{\to }{\color{correct}{#2}}\ensuremath{\:}}
\newcommand\green[1]{{\color{correct}{#1}}}

% horizontal rule
\newcommand\hr{
    \noindent\rule[0.5ex]{\linewidth}{0.5pt}
}

% hide parts
\newcommand\hide[1]{}

% si unitx
\usepackage{siunitx}
\sisetup{locale = FR}

% Environments
\makeatother
% For box around Definition, Theorem, \ldots
\usepackage[framemethod=TikZ]{mdframed}
\mdfsetup{skipabove=1em,skipbelow=0em}

%definition
\newenvironment{defn}[1][]{%
\ifstrempty{#1}%
{\mdfsetup{%
frametitle={%
\tikz[baseline=(current bounding box.east),outer sep=0pt]
\node[anchor=east,rectangle,fill=Emerald]
{\strut Definition};}}
}%
{\mdfsetup{%
frametitle={%
\tikz[baseline=(current bounding box.east),outer sep=0pt]
\node[anchor=east,rectangle,fill=Emerald]
{\strut Definition:~#1};}}%
}%
\mdfsetup{innertopmargin=10pt,linecolor=Emerald,%
linewidth=2pt,topline=true,%
frametitleaboveskip=\dimexpr-\ht\strutbox\relax
}
\begin{mdframed}[]\relax%
\label{#1}}{\end{mdframed}}


%theorem
%\newcounter{thm}[section]\setcounter{thm}{0}
%\renewcommand{\thethm}{\arabic{section}.\arabic{thm}}
\newenvironment{thm}[1][]{%
%\refstepcounter{thm}%
\ifstrempty{#1}%
{\mdfsetup{%
frametitle={%
\tikz[baseline=(current bounding box.east),outer sep=0pt]
\node[anchor=east,rectangle,fill=blue!20]
%{\strut Theorem~\thethm};}}
{\strut Theorem};}}
}%
{\mdfsetup{%
frametitle={%
\tikz[baseline=(current bounding box.east),outer sep=0pt]
\node[anchor=east,rectangle,fill=blue!20]
%{\strut Theorem~\thethm:~#1};}}%
{\strut Theorem:~#1};}}%
}%
\mdfsetup{innertopmargin=10pt,linecolor=blue!20,%
linewidth=2pt,topline=true,%
frametitleaboveskip=\dimexpr-\ht\strutbox\relax
}
\begin{mdframed}[]\relax%
\label{#1}}{\end{mdframed}}


%lemma
\newenvironment{lem}[1][]{%
\ifstrempty{#1}%
{\mdfsetup{%
frametitle={%
\tikz[baseline=(current bounding box.east),outer sep=0pt]
\node[anchor=east,rectangle,fill=Dandelion]
{\strut Lemma};}}
}%
{\mdfsetup{%
frametitle={%
\tikz[baseline=(current bounding box.east),outer sep=0pt]
\node[anchor=east,rectangle,fill=Dandelion]
{\strut Lemma:~#1};}}%
}%
\mdfsetup{innertopmargin=10pt,linecolor=Dandelion,%
linewidth=2pt,topline=true,%
frametitleaboveskip=\dimexpr-\ht\strutbox\relax
}
\begin{mdframed}[]\relax%
\label{#1}}{\end{mdframed}}

%corollary
\newenvironment{coro}[1][]{%
\ifstrempty{#1}%
{\mdfsetup{%
frametitle={%
\tikz[baseline=(current bounding box.east),outer sep=0pt]
\node[anchor=east,rectangle,fill=CornflowerBlue]
{\strut Corollary};}}
}%
{\mdfsetup{%
frametitle={%
\tikz[baseline=(current bounding box.east),outer sep=0pt]
\node[anchor=east,rectangle,fill=CornflowerBlue]
{\strut Corollary:~#1};}}%
}%
\mdfsetup{innertopmargin=10pt,linecolor=CornflowerBlue,%
linewidth=2pt,topline=true,%
frametitleaboveskip=\dimexpr-\ht\strutbox\relax
}
\begin{mdframed}[]\relax%
\label{#1}}{\end{mdframed}}

%proof
\newenvironment{prf}[1][]{%
\ifstrempty{#1}%
{\mdfsetup{%
frametitle={%
\tikz[baseline=(current bounding box.east),outer sep=0pt]
\node[anchor=east,rectangle,fill=SpringGreen]
{\strut Proof};}}
}%
{\mdfsetup{%
frametitle={%
\tikz[baseline=(current bounding box.east),outer sep=0pt]
\node[anchor=east,rectangle,fill=SpringGreen]
{\strut Proof:~#1};}}%
}%
\mdfsetup{innertopmargin=10pt,linecolor=SpringGreen,%
linewidth=2pt,topline=true,%
frametitleaboveskip=\dimexpr-\ht\strutbox\relax
}
\begin{mdframed}[]\relax%
\label{#1}}{\qed\end{mdframed}}


\theoremstyle{definition}

\newmdtheoremenv[nobreak=true]{definition}{Definition}
\newmdtheoremenv[nobreak=true]{prop}{Proposition}
\newmdtheoremenv[nobreak=true]{theorem}{Theorem}
\newmdtheoremenv[nobreak=true]{corollary}{Corollary}
\newtheorem*{eg}{Example}
\theoremstyle{remark}
\newtheorem*{case}{Case}
\newtheorem*{notation}{Notation}
\newtheorem*{remark}{Remark}
\newtheorem*{note}{Note}
\newtheorem*{problem}{Problem}
\newtheorem*{observe}{Observe}
\newtheorem*{property}{Property}
\newtheorem*{intuition}{Intuition}


% End example and intermezzo environments with a small diamond (just like proof
% environments end with a small square)
\usepackage{etoolbox}
\AtEndEnvironment{vb}{\null\hfill$\diamond$}%
\AtEndEnvironment{intermezzo}{\null\hfill$\diamond$}%
% \AtEndEnvironment{opmerking}{\null\hfill$\diamond$}%

% Fix some spacing
% http://tex.stackexchange.com/questions/22119/how-can-i-change-the-spacing-before-theorems-with-amsthm
\makeatletter
\def\thm@space@setup{%
  \thm@preskip=\parskip \thm@postskip=0pt
}

% Fix some stuff
% %http://tex.stackexchange.com/questions/76273/multiple-pdfs-with-page-group-included-in-a-single-page-warning
\pdfsuppresswarningpagegroup=1


% My name
\author{Jaden Wang}



\begin{document}
\begin{defn}[]
	An infinite (even uncountable) collection of sets is independent if every finite subcollection satisfies the above definition of independence.
\end{defn}

\begin{note}[]
Pairwise independence $ \not \implies$ independence.
\begin{eg}[]
	Roll a fair 6-sided die twice. Let $ A=$ the sum of outcomes is 7.  $ B =$ the first roll is 2. $ C=$ the second roll is 5.  $ P(A)=\frac{6}{36},P(B)=\frac{1}{6},P(C)=\frac{1}{6}, P(A \cap  B) = \frac{1}{36}$. Same for the other two pairs. However, $ P(A \cap B \cap C) = P(B \cap C)= \frac{1}{36} \neq \frac{1}{6^3}$. 
\end{eg}
\end{note}

\begin{defn}[]
Let $ \mathcal{A}_1, \mathcal{A}_2, \ldots, \mathcal{A}_n$ be classes/collections of sets from $ \mathcal{F}$. These classes are independent if $ A_1, A_2,\ldots,A_n$ are independent for all choices of $ A_i \in \mathcal{A}_i$.

Or\\

$ \mathcal{A}_1,\ldots,\mathcal{A}_n$ are independent if $ P(A_1 \cap A_2 \cap \ldots \cap A_n) = P(A_1)P(A_2)\cdots P(A_n)$ for $ A_i \in \mathcal{A}_i$ OR $ A_i = \Omega$.

\end{defn}

Does the independence of classes imply the independence of their sigma fields?

No! They need to be $ \pi$-systems.

Let $\Omega$ be a non-empty set. 

\begin{defn}[]
A $ \pi$-system 
\end{defn}
\begin{defn}[lambda-system]
A $\lambda$-system is a collection of subsets of $ \Omega$ such that 
\begin{enumerate}[label=(\roman*)]
	\item contains $ \Omega$.
	\item closed under complements.
	\item closed under countable disjoint unions.
\end{enumerate}
\end{defn}

\begin{notation}
	
\end{notation}

\begin{eg}[]
	$ \Omega = \{a,b,c,d\} $. Define $ \mathcal{L} = \{\{a,b\} ,\{a,c\} ,\{c,d\} ,\{b,d\} ,\Omega, \O\} $. Note $ \{a,b\} \cup \{a,c\} = \{a,b,c\} \not\in \mathcal{L} \implies \mathcal{L}$ is not a field.
\end{eg}


In the definition of a $\lambda$-system, (ii) can be replaced by\\
(ii') $ A,B \in \mathcal{L}$ with $ A \subseteq B$ then $ B\setminus A \in \mathcal{L} $.

\begin{prf}
	(i) and (ii')  $ \implies$ (ii). Take $ B = \Omega$. Then $ B \setminus A = A^{c} \in \mathcal{L}$.\\

	(ii) and (iii) $ \implies$ $ A \subseteq B \implies B\setminus A = B \cap A^{c} = (A \cup B^{c})^{c} \in \mathcal{L}$ since disjoint union and its complement is in $ \mathcal{L}$.
\end{prf}

\begin{lem}[1]
If $ \mathcal{A}$ is a $\pi$-system and a $\lambda$-system, then $ \mathcal{A}$ is a $\sigma$-field.
\end{lem}

\begin{prf}
	(i) and (ii) for  $\sigma$-field are satisfied from (i) and (ii) from the $\lambda$-system. We want to show (iii) from $\sigma$-field definition.

	Take $ A_1,A_2,\ldots \in \mathcal{A}$. Let $ B_1=A_1, B_2=A_2 \setminus A_1 = A_2 \cap A_1^{c}, B_3 = A_3 \cap A_2^{c} \cap A_1^{c},\ldots$ The $ B_n$ are disjoint. $ \mathcal{A}$ is a $\pi$-system and $\lambda$-system implies that the $ B_n$ are in $ \mathcal{A}$. So
	\[
	\bigcup_{ n} A_n = \bigcup_{ n} B_n \in \mathcal{A}  
	.\] 
\end{prf}

\begin{lem}[2]
	Suppose that $ \mathcal{L}_0$ is a $\lambda$-system. Define, for any $ C \subseteq \Omega$, $ \mathcal{L}_C \coloneq \{D \subseteq \Omega : D \cap  C \in \mathcal{L}_0\} $. If $ C \in \mathcal{L}_0$ then $ \mathcal{L}_C$ is a $\lambda$-system.
\end{lem}

\begin{prf}
\begin{enumerate}[label=(\roman*)]
	\item $ C \in \mathcal{L}_0 \implies \Omega \cap C = C \in \mathcal{L}_0 \implies \Omega \in \mathcal{L}_C$ 
	\item Suppose $ A \in \mathcal{L}_C$, want to show $ A^{c} \in \mathcal{L}_C$. Write $ C = (A \cap C) \cup (A^{c} \cap  C)$. Then $ A^{c} \cap  C = C \setminus (A \cap C) \in \mathcal{L}_0$ since $ \mathcal{L}_0$ is a $\lambda$-system, and $ A \cap C \subseteq C \implies A^{c} \in \mathcal{L}_C$.
	\item Take $ A_1,A_2,\ldots$ disjoint in $ \mathcal{L}_C$. Then $A_1 \cap  C, A_2 \cap C,\ldots$ are in $ \mathcal{L}_0$. Since $ \mathcal{L}_0$ is a $\lambda$-system, we have
		\begin{align*}
			\bigcup_{ n =1}^{\infty} (A_n \cap  C) \in \mathcal{L}_0\\
			\left( \bigcup_{ n =1}^{\infty} A_n \right) \cap C \in \mathcal{L}_0\\
		\end{align*}
\end{enumerate}
\end{prf}

\begin{thm}[Dynkin's pi-lambda Theorem]
	Let $ \mathcal{ P}$ be a $\pi$-system and $ \mathcal{L}$ be a $\lambda$-system. If $ \mathcal{ P} \subseteq \mathcal{L}$ then $ \sigma( \mathcal{ P}) \subseteq \mathcal{L}$.
\end{thm}

\begin{prf}
	Suppose $ \mathcal{ P} \subseteq \mathcal{L}$. Let $ \mathcal{L}_0$ be the smallest $\pi$-system containing $ \mathcal{ P}$. Then $ \mathcal{ P} \subseteq \mathcal{L}_0 \subseteq \mathcal{L}$. If we can show that $ \mathcal{L}_0$ is a $\sigma$-field then $ \sigma( \mathcal{ P}) \subseteq \mathcal{L}_0 \subseteq \mathcal{L}$.

	By Lemma 1, we need to show that $ \mathcal{L}_0$ is a $\pi$-system. Take $ A,B \in \mathcal{L}_0$, we want to show that $ A \cap B \in \mathcal{L}_0$.

	Claim: $ \mathcal{ P} \subseteq \mathcal{L}_B = \{C \subseteq \Omega: C \cap B \in \mathcal{L}_0\} $.
	Then, since $ B \in \mathcal{L}_0$, by Lemma 2, $ \mathcal{L}_B$ is a $\lambda$-system, assuming the claim is true. Since $ \mathcal{ P} \subseteq \mathcal{L}_B$ and $ \mathcal{L}_B$ is a $\lambda$-system, we have $ \mathcal{ P} \subseteq \mathcal{L}_0 \subseteq \mathcal{L}_B$ since $ \mathcal{L}_0$ is the smallest $\lambda$-system containing $ \mathcal{ P}$. So $ A \in \mathcal{L}_B \implies A \cap B \in \mathcal{L}_0$ by the definition of $ \mathcal{L}_B$.
\end{prf}

\begin{thm}[4.2]
	Given $ (\Omega,\mathcal{F},P)$. If $ \mathcal{A}_1,\mathcal{A}_2,\ldots,\mathcal{A}_n$ are independent $\pi$-systems, then $ \sigma(\mathcal{A}_1), \sigma(\mathcal{A}_2), \ldots, \sigma(\mathcal{A}_n)$ are independent.
\end{thm}

\begin{prf}
	For $ i=1,2,\ldots,n$, define $ \mathcal{B}_i = \mathcal{A}_i \cup \{\Omega\} $. Note that $ \mathcal{B}_i$ is still a $\pi$-system and $ \sigma(\mathcal{B}_i) = \sigma(\mathcal{A}_i)$. $ \mathcal{A}_1,\ldots,\mathcal{A}_n$ independent $ \iff P(A_1 \cap \ldots \cap A_n) = P(A_1)\cdots P(A_n)$ for all $ A_i \in \mathcal{A}_i$ or $ A_i=\Omega \iff P(B_1 \cap \ldots \cap B_n) = P(B_1)\cdots P(B_n)$ for any $ B_i \in \mathcal{B}_i$.

	For $ i=2,\ldots , n$, fix sets $ B_i \in \mathcal{B}_i$ and define $ \mathcal{L} = \{A \in \mathcal{F}: P(A \cap B_2 \cap \ldots \cap B_n) = P(A) P(B_2) \cdots P(B_n)\} $.\\

	Claim: $ \mathcal{L}$ is a $\lambda$-system.
	\begin{enumerate}[label=(\roman*)]
		\item (i) $ \Omega \in \mathcal{L}$ since $ P(B_2 \cap \ldots \cap  B_n) = P(B_2)\cdots P(B_n)$
		\item Take $ A \in \mathcal{L}$. Let $ B = B_2 \cap  \ldots \cap  B_n$. Then $ P(B)=P(A \cap B) + P(A^{c} \cap B) \implies P(B_2) \cdots P(B_n)=P(A)P(B_2)\cdots P(B_n)+P(A^{c} \cap B) \implies P(A^{c} \cap B) = (1-P(A)P(B_2)\cdots P(B_n) \implies A^{c} \in \mathcal{L}$.
		\item Take $ A_1,\ldots \in \mathcal{L}$ disjoint. Let $ A=\bigcup_{ m =1}^{\infty} A_m$. Then
			\begin{align*}
				P(A \cap  B_2 \cap  \ldots \cap B_n) &= P(\bigcup_{ m =1}^{\infty} A_m \cap B_2 \cap \ldots \cap B_n  \\
								     &= P\left( \bigcup_{ m =1}^{\infty} A_m \cap (B_2 \cap \ldots \cap B_n) \right) \qquad \text{ by countable additivity}  \\
								     &= \left( \sum_{ m= 1}^{\infty} P(A_m) \right) P(B_2) \cdots P(B_n) \\
								     &= P\left( \bigcup_{ m =1}^{\infty} A_m \right) P(B_2) \cdots P(B_n) \qquad \text{ by countable additivity} 
			\end{align*}
			Hence $ \bigcup_{ m =1}^{\infty} A_m \in \mathcal{L}$.
	\end{enumerate}

	Note that $ B_1 \subseteq \mathcal{L}$, $ \mathcal{B}_i$ is a $\pi$-system, $ \mathcal{L}$ is a $\lambda$-system, by Dynkin we have $ \sigma(\mathcal{A}_1) \sigma(\mathcal{B}_1) \subseteq \mathcal{L}$. By definition of $ \mathcal{L}$, we have $ \sigma(\mathcal{A}_1)$ is independent of $ \mathcal{A}_2,\ldots,\mathcal{A}_n$.
\end{prf}

\end{document}

\documentclass[class=article,crop=false]{standalone} 
%Fall 2020
% Some basic packages
\usepackage{standalone}[subpreambles=true]
\usepackage[utf8]{inputenc}
\usepackage[T1]{fontenc}
\usepackage{textcomp}
\usepackage[english]{babel}
\usepackage{url}
\usepackage{graphicx}
\usepackage{float}
\usepackage{enumitem}


\pdfminorversion=7

% Don't indent paragraphs, leave some space between them
\usepackage{parskip}

% Hide page number when page is empty
\usepackage{emptypage}
\usepackage{subcaption}
\usepackage{multicol}
\usepackage[dvipsnames]{xcolor}


% Math stuff
\usepackage{amsmath, amsfonts, mathtools, amsthm, amssymb}
% Fancy script capitals
\usepackage{mathrsfs}
\usepackage{cancel}
% Bold math
\usepackage{bm}
% Some shortcuts
\newcommand{\rr}{\ensuremath{\mathbb{R}}}
\newcommand{\zz}{\ensuremath{\mathbb{Z}}}
\newcommand{\qq}{\ensuremath{\mathbb{Q}}}
\newcommand{\nn}{\ensuremath{\mathbb{N}}}
\newcommand{\ff}{\ensuremath{\mathbb{F}}}
\newcommand{\cc}{\ensuremath{\mathbb{C}}}
\renewcommand\O{\ensuremath{\emptyset}}
\newcommand{\norm}[1]{{\left\lVert{#1}\right\rVert}}
\renewcommand{\vec}[1]{{\mathbf{#1}}}
\newcommand\allbold[1]{{\boldmath\textbf{#1}}}

% Put x \to \infty below \lim
\let\svlim\lim\def\lim{\svlim\limits}

%Make implies and impliedby shorter
\let\implies\Rightarrow
\let\impliedby\Leftarrow
\let\iff\Leftrightarrow
\let\epsilon\varepsilon

% Add \contra symbol to denote contradiction
\usepackage{stmaryrd} % for \lightning
\newcommand\contra{\scalebox{1.5}{$\lightning$}}

% \let\phi\varphi

% Command for short corrections
% Usage: 1+1=\correct{3}{2}

\definecolor{correct}{HTML}{009900}
\newcommand\correct[2]{\ensuremath{\:}{\color{red}{#1}}\ensuremath{\to }{\color{correct}{#2}}\ensuremath{\:}}
\newcommand\green[1]{{\color{correct}{#1}}}

% horizontal rule
\newcommand\hr{
    \noindent\rule[0.5ex]{\linewidth}{0.5pt}
}

% hide parts
\newcommand\hide[1]{}

% si unitx
\usepackage{siunitx}
\sisetup{locale = FR}

% Environments
\makeatother
% For box around Definition, Theorem, \ldots
\usepackage[framemethod=TikZ]{mdframed}
\mdfsetup{skipabove=1em,skipbelow=0em}

%definition
\newenvironment{defn}[1][]{%
\ifstrempty{#1}%
{\mdfsetup{%
frametitle={%
\tikz[baseline=(current bounding box.east),outer sep=0pt]
\node[anchor=east,rectangle,fill=Emerald]
{\strut Definition};}}
}%
{\mdfsetup{%
frametitle={%
\tikz[baseline=(current bounding box.east),outer sep=0pt]
\node[anchor=east,rectangle,fill=Emerald]
{\strut Definition:~#1};}}%
}%
\mdfsetup{innertopmargin=10pt,linecolor=Emerald,%
linewidth=2pt,topline=true,%
frametitleaboveskip=\dimexpr-\ht\strutbox\relax
}
\begin{mdframed}[]\relax%
\label{#1}}{\end{mdframed}}


%theorem
%\newcounter{thm}[section]\setcounter{thm}{0}
%\renewcommand{\thethm}{\arabic{section}.\arabic{thm}}
\newenvironment{thm}[1][]{%
%\refstepcounter{thm}%
\ifstrempty{#1}%
{\mdfsetup{%
frametitle={%
\tikz[baseline=(current bounding box.east),outer sep=0pt]
\node[anchor=east,rectangle,fill=blue!20]
%{\strut Theorem~\thethm};}}
{\strut Theorem};}}
}%
{\mdfsetup{%
frametitle={%
\tikz[baseline=(current bounding box.east),outer sep=0pt]
\node[anchor=east,rectangle,fill=blue!20]
%{\strut Theorem~\thethm:~#1};}}%
{\strut Theorem:~#1};}}%
}%
\mdfsetup{innertopmargin=10pt,linecolor=blue!20,%
linewidth=2pt,topline=true,%
frametitleaboveskip=\dimexpr-\ht\strutbox\relax
}
\begin{mdframed}[]\relax%
\label{#1}}{\end{mdframed}}


%lemma
\newenvironment{lem}[1][]{%
\ifstrempty{#1}%
{\mdfsetup{%
frametitle={%
\tikz[baseline=(current bounding box.east),outer sep=0pt]
\node[anchor=east,rectangle,fill=Dandelion]
{\strut Lemma};}}
}%
{\mdfsetup{%
frametitle={%
\tikz[baseline=(current bounding box.east),outer sep=0pt]
\node[anchor=east,rectangle,fill=Dandelion]
{\strut Lemma:~#1};}}%
}%
\mdfsetup{innertopmargin=10pt,linecolor=Dandelion,%
linewidth=2pt,topline=true,%
frametitleaboveskip=\dimexpr-\ht\strutbox\relax
}
\begin{mdframed}[]\relax%
\label{#1}}{\end{mdframed}}

%corollary
\newenvironment{coro}[1][]{%
\ifstrempty{#1}%
{\mdfsetup{%
frametitle={%
\tikz[baseline=(current bounding box.east),outer sep=0pt]
\node[anchor=east,rectangle,fill=CornflowerBlue]
{\strut Corollary};}}
}%
{\mdfsetup{%
frametitle={%
\tikz[baseline=(current bounding box.east),outer sep=0pt]
\node[anchor=east,rectangle,fill=CornflowerBlue]
{\strut Corollary:~#1};}}%
}%
\mdfsetup{innertopmargin=10pt,linecolor=CornflowerBlue,%
linewidth=2pt,topline=true,%
frametitleaboveskip=\dimexpr-\ht\strutbox\relax
}
\begin{mdframed}[]\relax%
\label{#1}}{\end{mdframed}}

%proof
\newenvironment{prf}[1][]{%
\ifstrempty{#1}%
{\mdfsetup{%
frametitle={%
\tikz[baseline=(current bounding box.east),outer sep=0pt]
\node[anchor=east,rectangle,fill=SpringGreen]
{\strut Proof};}}
}%
{\mdfsetup{%
frametitle={%
\tikz[baseline=(current bounding box.east),outer sep=0pt]
\node[anchor=east,rectangle,fill=SpringGreen]
{\strut Proof:~#1};}}%
}%
\mdfsetup{innertopmargin=10pt,linecolor=SpringGreen,%
linewidth=2pt,topline=true,%
frametitleaboveskip=\dimexpr-\ht\strutbox\relax
}
\begin{mdframed}[]\relax%
\label{#1}}{\qed\end{mdframed}}


\theoremstyle{definition}

\newmdtheoremenv[nobreak=true]{definition}{Definition}
\newmdtheoremenv[nobreak=true]{prop}{Proposition}
\newmdtheoremenv[nobreak=true]{theorem}{Theorem}
\newmdtheoremenv[nobreak=true]{corollary}{Corollary}
\newtheorem*{eg}{Example}
\theoremstyle{remark}
\newtheorem*{case}{Case}
\newtheorem*{notation}{Notation}
\newtheorem*{remark}{Remark}
\newtheorem*{note}{Note}
\newtheorem*{problem}{Problem}
\newtheorem*{observe}{Observe}
\newtheorem*{property}{Property}
\newtheorem*{intuition}{Intuition}


% End example and intermezzo environments with a small diamond (just like proof
% environments end with a small square)
\usepackage{etoolbox}
\AtEndEnvironment{vb}{\null\hfill$\diamond$}%
\AtEndEnvironment{intermezzo}{\null\hfill$\diamond$}%
% \AtEndEnvironment{opmerking}{\null\hfill$\diamond$}%

% Fix some spacing
% http://tex.stackexchange.com/questions/22119/how-can-i-change-the-spacing-before-theorems-with-amsthm
\makeatletter
\def\thm@space@setup{%
  \thm@preskip=\parskip \thm@postskip=0pt
}

% Fix some stuff
% %http://tex.stackexchange.com/questions/76273/multiple-pdfs-with-page-group-included-in-a-single-page-warning
\pdfsuppresswarningpagegroup=1


% My name
\author{Jaden Wang}



\begin{document}
\begin{thm}[11.4 (Approximation Theorem)]
	Suppose $\mathcal{A} $ is a semiring, and $ \mu$ is a measure on $ \mathcal{F} \coloneqq \sigma(\mathcal{A})$, and $ \mu$ is $ \sigma$-finite on $ \mathcal{A}$. Take $ \epsilon> 0$ and any $ B \in \mathcal{F}$. Then
	\begin{enumerate}[label=(\roman*)]
		\item There exists a disjoint sequence $ A_1, A_2,\ldots \in \mathcal{A}$ (maybe finite with empty sets) such that $ B \subseteq \bigcup_{ n} A_n $ and 
			\[
				\mu\left( \bigcup_n A_n \setminus B  \right) < \epsilon 
			.\] 
		\item If $ \mu(B) < \infty$, there exists a finite disjoint sequence $ A_1, A_2,\ldots,A_n \in \mathcal{A}$ such that 
			\[
				\mu\left( B \Delta \left( \bigcup_{ i= 1}^{ n} A_i \right)  \right) < \epsilon 
			.\] 
	\end{enumerate}
\end{thm}
\begin{note}[]
	Recall $ A \Delta B = (A \setminus B) \cup (B \setminus A)$.
\end{note}

\section*{12: Measures in Euclidean Space}

\begin{eg}[]
	Consider $ \rr$. Let $ \mathcal{A}$ be the collection of half intervals $ (a,b]$. We saw from last time that  $ \mathcal{A}$ is a semiring. Define a measure  $ \lambda$ as $ \lambda( \O) =0 $ and $ \lambda((a,b])=b-a$. Note that $ \lambda$ is defined on the field $ \rr$. Then $ \sigma(\mathcal{A})= \mathcal{B}( \rr)$ since the Borel sets can be generated by these half intervals. By Theorem 11.3, $ \lambda$ can be extended to a measure on $ \sigma(\mathcal{A}) = \mathcal{B}(\rr)$. Since $ \mathcal{A}$ is a $\pi$-system, $ \mathcal{A}, \lambda$ is $\sigma$-finite, Theorem 10.3 tells us that the extension of $ \lambda$ from $ \mathcal{A}$ to $ \sigma(\mathcal{A})$ is unique. So there is no other measure on $ \mathcal{B}( \rr)$ that will assign measure $ (b-a)$ to  $ (a,b]$. But Lebesgue measure assigns  $ (b-a)$ to  $ (a,b]$, then  $ \lambda $ must be Lebesgue measure.
\end{eg}

\begin{eg}[]
In $ \rr^{k}$, the analogy is let
\[
	R = \{(x_1,\ldots,x_k): a_i< x_i \leq b_i \text{ for } i=1,\ldots,k \}  
,\] 

\[
	\lambda( \rr) \coloneqq \prod_{i=1}^k (b_i-a_i)
\] 
and assign $ \lambda( \O) = 0$. Then this extend to all of $ \rr^{k}$ so we can define Lebesgue measure on $ \rr^{k}$.
\end{eg}

\begin{property}[]
~\begin{enumerate}[label=\arabic*)]
	\item \allbold{Translation invariance}: For $ A \in \mathcal{B}( \rr^{k} )$ and any $ \vec{ x} \in \rr^{k} $, define the set $ A + \vec{ x} $ as 
		\[
		A + \vec{ x} = \{ \vec{ a}+ \vec{ x}: \vec{ a} \in A   \}  
		.\] 
		Then
		\begin{enumerate}[label=(\roman*)]
			\item $ A+ \vec{ x} \in \mathcal{B}( \rr^{k)} $.
			\item $ \lambda (A) = \lambda (A + \vec{ x} )$.
		\end{enumerate}
		\begin{note}[]
			$ A \in \mathcal{B}( \rr^{k})$ does not have to be a rectangle.
		\end{note}
		\begin{prf}
		\begin{enumerate}[label=(\roman*)]
			\item Let $ \mathcal{G}=\{A \subseteq R^{k}: A + \vec{ x} \in \mathcal{B}(\rr^{k}) \ \forall \ \vec{ x} \in \rr^{k} \} $. We can show that $ \mathcal{G}$ is a $\sigma$-field. Let $ \mathcal{A}$ be the class of half open rectangles in $ \rr^{k}$. Then $ \sigma(\mathcal{A}) = \mathcal{B}(\rr^{k})$ by the definition of $ \mathcal{B}(\rr^{k})$. We can show that $ \mathcal{A} \subseteq \mathcal{G}$. Thus, \[\mathcal{A} \subseteq \sigma(\mathcal{A}) = \mathcal{B}(\rr^{k}) \subseteq \mathcal{G}.\] Therefore, by the definition of $ \mathcal{G}$, $ A \in \mathcal{B}(\rr^{k}) \subseteq \mathcal{G} \implies A + \vec{ x} \in \mathcal{B}(\rr^{k}) $.
		\end{enumerate}
		\end{prf}
\item For a linear mapping $ T: \rr^{k} \to \rr^{k}$, 
	\begin{enumerate}[label=(\roman*)]
		\item $ A \in \mathcal{B}( \rr^{k}) \implies T(A) \in \mathcal{B}( \rr^{k})$.
		\item $ \lambda(T(A)) = | \det(T)| \cdot \lambda(A) \ \forall \ A \in \mathcal{B}( \rr^{k})$. 
	\end{enumerate}
	\begin{note}[]
		A linear map $ T: \rr^{k} \to \rr^{k}$ can be written as $ T( \vec{ x}) = T \vec{ x}  $ where $ T$ is a  $ k\times k$
	\end{note}
\end{enumerate}
\end{property}
\end{document}

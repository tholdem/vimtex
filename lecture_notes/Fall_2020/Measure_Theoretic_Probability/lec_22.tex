\documentclass[class=article,crop=false]{standalone} 
%Fall 2020
% Some basic packages
\usepackage{standalone}[subpreambles=true]
\usepackage[utf8]{inputenc}
\usepackage[T1]{fontenc}
\usepackage{textcomp}
\usepackage[english]{babel}
\usepackage{url}
\usepackage{graphicx}
\usepackage{float}
\usepackage{enumitem}


\pdfminorversion=7

% Don't indent paragraphs, leave some space between them
\usepackage{parskip}

% Hide page number when page is empty
\usepackage{emptypage}
\usepackage{subcaption}
\usepackage{multicol}
\usepackage[dvipsnames]{xcolor}


% Math stuff
\usepackage{amsmath, amsfonts, mathtools, amsthm, amssymb}
% Fancy script capitals
\usepackage{mathrsfs}
\usepackage{cancel}
% Bold math
\usepackage{bm}
% Some shortcuts
\newcommand{\rr}{\ensuremath{\mathbb{R}}}
\newcommand{\zz}{\ensuremath{\mathbb{Z}}}
\newcommand{\qq}{\ensuremath{\mathbb{Q}}}
\newcommand{\nn}{\ensuremath{\mathbb{N}}}
\newcommand{\ff}{\ensuremath{\mathbb{F}}}
\newcommand{\cc}{\ensuremath{\mathbb{C}}}
\renewcommand\O{\ensuremath{\emptyset}}
\newcommand{\norm}[1]{{\left\lVert{#1}\right\rVert}}
\renewcommand{\vec}[1]{{\mathbf{#1}}}
\newcommand\allbold[1]{{\boldmath\textbf{#1}}}

% Put x \to \infty below \lim
\let\svlim\lim\def\lim{\svlim\limits}

%Make implies and impliedby shorter
\let\implies\Rightarrow
\let\impliedby\Leftarrow
\let\iff\Leftrightarrow
\let\epsilon\varepsilon

% Add \contra symbol to denote contradiction
\usepackage{stmaryrd} % for \lightning
\newcommand\contra{\scalebox{1.5}{$\lightning$}}

% \let\phi\varphi

% Command for short corrections
% Usage: 1+1=\correct{3}{2}

\definecolor{correct}{HTML}{009900}
\newcommand\correct[2]{\ensuremath{\:}{\color{red}{#1}}\ensuremath{\to }{\color{correct}{#2}}\ensuremath{\:}}
\newcommand\green[1]{{\color{correct}{#1}}}

% horizontal rule
\newcommand\hr{
    \noindent\rule[0.5ex]{\linewidth}{0.5pt}
}

% hide parts
\newcommand\hide[1]{}

% si unitx
\usepackage{siunitx}
\sisetup{locale = FR}

% Environments
\makeatother
% For box around Definition, Theorem, \ldots
\usepackage[framemethod=TikZ]{mdframed}
\mdfsetup{skipabove=1em,skipbelow=0em}

%definition
\newenvironment{defn}[1][]{%
\ifstrempty{#1}%
{\mdfsetup{%
frametitle={%
\tikz[baseline=(current bounding box.east),outer sep=0pt]
\node[anchor=east,rectangle,fill=Emerald]
{\strut Definition};}}
}%
{\mdfsetup{%
frametitle={%
\tikz[baseline=(current bounding box.east),outer sep=0pt]
\node[anchor=east,rectangle,fill=Emerald]
{\strut Definition:~#1};}}%
}%
\mdfsetup{innertopmargin=10pt,linecolor=Emerald,%
linewidth=2pt,topline=true,%
frametitleaboveskip=\dimexpr-\ht\strutbox\relax
}
\begin{mdframed}[]\relax%
\label{#1}}{\end{mdframed}}


%theorem
%\newcounter{thm}[section]\setcounter{thm}{0}
%\renewcommand{\thethm}{\arabic{section}.\arabic{thm}}
\newenvironment{thm}[1][]{%
%\refstepcounter{thm}%
\ifstrempty{#1}%
{\mdfsetup{%
frametitle={%
\tikz[baseline=(current bounding box.east),outer sep=0pt]
\node[anchor=east,rectangle,fill=blue!20]
%{\strut Theorem~\thethm};}}
{\strut Theorem};}}
}%
{\mdfsetup{%
frametitle={%
\tikz[baseline=(current bounding box.east),outer sep=0pt]
\node[anchor=east,rectangle,fill=blue!20]
%{\strut Theorem~\thethm:~#1};}}%
{\strut Theorem:~#1};}}%
}%
\mdfsetup{innertopmargin=10pt,linecolor=blue!20,%
linewidth=2pt,topline=true,%
frametitleaboveskip=\dimexpr-\ht\strutbox\relax
}
\begin{mdframed}[]\relax%
\label{#1}}{\end{mdframed}}


%lemma
\newenvironment{lem}[1][]{%
\ifstrempty{#1}%
{\mdfsetup{%
frametitle={%
\tikz[baseline=(current bounding box.east),outer sep=0pt]
\node[anchor=east,rectangle,fill=Dandelion]
{\strut Lemma};}}
}%
{\mdfsetup{%
frametitle={%
\tikz[baseline=(current bounding box.east),outer sep=0pt]
\node[anchor=east,rectangle,fill=Dandelion]
{\strut Lemma:~#1};}}%
}%
\mdfsetup{innertopmargin=10pt,linecolor=Dandelion,%
linewidth=2pt,topline=true,%
frametitleaboveskip=\dimexpr-\ht\strutbox\relax
}
\begin{mdframed}[]\relax%
\label{#1}}{\end{mdframed}}

%corollary
\newenvironment{coro}[1][]{%
\ifstrempty{#1}%
{\mdfsetup{%
frametitle={%
\tikz[baseline=(current bounding box.east),outer sep=0pt]
\node[anchor=east,rectangle,fill=CornflowerBlue]
{\strut Corollary};}}
}%
{\mdfsetup{%
frametitle={%
\tikz[baseline=(current bounding box.east),outer sep=0pt]
\node[anchor=east,rectangle,fill=CornflowerBlue]
{\strut Corollary:~#1};}}%
}%
\mdfsetup{innertopmargin=10pt,linecolor=CornflowerBlue,%
linewidth=2pt,topline=true,%
frametitleaboveskip=\dimexpr-\ht\strutbox\relax
}
\begin{mdframed}[]\relax%
\label{#1}}{\end{mdframed}}

%proof
\newenvironment{prf}[1][]{%
\ifstrempty{#1}%
{\mdfsetup{%
frametitle={%
\tikz[baseline=(current bounding box.east),outer sep=0pt]
\node[anchor=east,rectangle,fill=SpringGreen]
{\strut Proof};}}
}%
{\mdfsetup{%
frametitle={%
\tikz[baseline=(current bounding box.east),outer sep=0pt]
\node[anchor=east,rectangle,fill=SpringGreen]
{\strut Proof:~#1};}}%
}%
\mdfsetup{innertopmargin=10pt,linecolor=SpringGreen,%
linewidth=2pt,topline=true,%
frametitleaboveskip=\dimexpr-\ht\strutbox\relax
}
\begin{mdframed}[]\relax%
\label{#1}}{\qed\end{mdframed}}


\theoremstyle{definition}

\newmdtheoremenv[nobreak=true]{definition}{Definition}
\newmdtheoremenv[nobreak=true]{prop}{Proposition}
\newmdtheoremenv[nobreak=true]{theorem}{Theorem}
\newmdtheoremenv[nobreak=true]{corollary}{Corollary}
\newtheorem*{eg}{Example}
\theoremstyle{remark}
\newtheorem*{case}{Case}
\newtheorem*{notation}{Notation}
\newtheorem*{remark}{Remark}
\newtheorem*{note}{Note}
\newtheorem*{problem}{Problem}
\newtheorem*{observe}{Observe}
\newtheorem*{property}{Property}
\newtheorem*{intuition}{Intuition}


% End example and intermezzo environments with a small diamond (just like proof
% environments end with a small square)
\usepackage{etoolbox}
\AtEndEnvironment{vb}{\null\hfill$\diamond$}%
\AtEndEnvironment{intermezzo}{\null\hfill$\diamond$}%
% \AtEndEnvironment{opmerking}{\null\hfill$\diamond$}%

% Fix some spacing
% http://tex.stackexchange.com/questions/22119/how-can-i-change-the-spacing-before-theorems-with-amsthm
\makeatletter
\def\thm@space@setup{%
  \thm@preskip=\parskip \thm@postskip=0pt
}

% Fix some stuff
% %http://tex.stackexchange.com/questions/76273/multiple-pdfs-with-page-group-included-in-a-single-page-warning
\pdfsuppresswarningpagegroup=1


% My name
\author{Jaden Wang}



\begin{document}
\begin{defn}[density]
A r.v. $ X$ and the corresponding distribution  $ P_X$ has a  \allbold{density} $ f$ wrt Lebesgue measure if 
 \begin{enumerate}[label=(\roman*)]
	 \item $ f: \rr \to [0,\infty]$ is measurable $ \mathcal{B}(\rr) / \mathcal{B}(\rr^{+} \cup \{+ \infty\}) $.
	 \item For all $ A \in \mathcal{B}(\rr)$,
		 \[
			 P_X(A) = P(X \in A) = \int_A f(x) dx
		 .\] 
\end{enumerate}
\end{defn}

\begin{note}[]
~\begin{enumerate}[label=\arabic*)]
	\item can define wrt other measures.
	\item we have the familiar
		\[
			\int_{\rr} f dx = P_X (\rr) = p(X \in \rr) = P(\Omega) = 1
		.\]
	\item A r.v. $ X$ may have multiple densities wrt Lebesgue measure. They can only differ on sets with Lebesgue measure zero.
\end{enumerate}
\end{note}

\begin{claim}
Define $ F(x) = P(X\leq x) = P(X \in (-\infty,x]) = \int_{-\infty}^x f(t) dt$. If $ F$ is a cdf (right-continuous, etc), then this equation implies that  $ F$ is continuous. 
\end{claim}

Consider the random vector $ X = (X_1,\ldots,X_k)$. 
\begin{enumerate}[label=\arabic*)]
	\item This induces a probability measure $ P_X(A) = P((X_1, \ldots, X_k) \in A) \ \forall \ A \in \mathcal{B}(\rr^{k})$. 
	\item has a distribution function
		\[
			F_X(x_1,\ldots,x_k) \coloneqq P(X_1 \leq x_1, \ldots, X_k \leq x_k)
		.\]
	\item may or may not have a density $ f$ satisfying  $ P_X(A) = \int_A f(t) dt$.
		 \begin{align*}
			 F_X(x_1,\ldots,x_k) = \int_{-\infty}^{x_k} \ldots \int_{-\infty}^{x_1} f(t_1,\ldots,t_k) dt_1 \ldots dt_k
		\end{align*}
\end{enumerate}

\begin{property}
~\begin{enumerate}[label=\arabic*)]
	\item non-decreasing on each variable (intersections).
	\item right-continuous in each variable. That is,
		\[
			\lim_{ x_i \searrow a} F_X(x_1,\ldots,x_k) = F(x_1,\ldots,x_{i-1},a,x_{i+1},\ldots,x_k).
		.\] 
		When $ k=2$, this is often referred to as "continuity from the right and from above".
	\item $ \lim_{ x_i \to -\infty} F_X(x_1,\ldots,x_k) =0$.
	\item $ \lim_{ x_i \to \infty} F_X(x_1,\ldots,x_k) = F_X(x_1,\ldots,x_{i-1},x_{i+1},\ldots,x_k)$. This is the marginal CDF. Taking all $ x_i \to \infty$ gives us 1.
	\item $ F_X$ can have an uncountable number of discontinuities.
		 \begin{eg}[]
			 In $ \rr^2$, Extend Lebesgue measure on $ [0,1]$ to  $ \rr^2$ as follows:

			 Let $ A \in \mathcal{B}(\rr^2)$, $ X=(X_1,X_2)$. Define
			 \[
				 P_X(A) = \lambda(A \cap \{(x,y): 0\leq x \leq 1, y=0\} )
			 .\] 
			 It's the length of the overlap of $ A$ with $ [0,1]$. Note that $ P_X$ is a probability measure on  $ \rr^2$ because it is between 0 and 1, $ P_X( \O) =0$, and $ P_X(\rr^2)=\lambda([0,1])=1$. And given $ A_1,\ldots$ disjoint, then
			  \begin{align*}
				  P_X\left( \bigcup_{ n =1}^{\infty} A_n \right) &= \lambda \left( \left( \bigcup_{ n =1}^{\infty} A_n \right) \cap [0,1]  \right)  \\
										 &= \lambda\left( \bigcup_{ n =1}^{\infty} (A_n \cap [0,1]) \right)  \\
										 &= \sum_{ n= 1}^{\infty} \lambda(A_n \cap [0,1]) \text{ disjoint} \\
										 &= \sum_{ n= 1}^{\infty} P_X(A_n) \\
			  \end{align*}
			  Now define $ F(x_1,x_2) = P(X_1\leq x_1, X_2\leq x_2)$. Then
			   \begin{align*}
				   F(1,0) &= P(X_1\leq 1, X_2 \leq 0) \\
					  &= P_X((-\infty,1] \times (-\infty,0]) \\
					  &= \lambda ([0,1]) =1
			  \end{align*}
			  Notice
			  \[
				  F(1,0^-)= P(X_1\leq 1, X_2 < 0) = 0
			  .\] 
			  which is a jumped discontinuity. In fact, $ F(a,0)$ and  $ F(a,0^-)$ for any  $ a \in[0,1]$ will give us a jump discontinuity. And this is uncountably many of $ a$. 
		\end{eg}

	\item In general, since cdfs are right-continuous,  $ F$ is continuous at  $ x \implies F$ is right continuous at  $ x \implies$ the boundary of the half rectangle $ \{y:y_i\leq x_i, i=1,2,\ldots,k\} $ has probability measure zero. 
	\item OTOH, while $ F$ may have an uncountably infinite number of discontinuities, the points of continuities are a dense set in  $ \rr^{k}$.
		\begin{prf}
			Sketch: Take any $ x \in \rr^{k}$ and consider the half rectangles $ \{y_i: y_i\leq x_i + h, i=1,\ldots,k\} $ for various $ h$. The boundaries of these half rectangles are disjoint. Recall from Section 10 if a measure  $ \mu$ on a $\sigma$-field $ \mathcal{F}$ is $\sigma$-finite, then $ \mathcal{F}$ cannot contain an uncountable disjoint collection of sets with positive $ \mu$ measure. $ P_X$ is finite  $ \implies P_X$ is $\sigma$-finite, so only countably many of the $ x+h$ rectangle boundaries can have positive  $ P_X$ measure. That is, only countably many of the  $ x+h$'s can be points of discontinuities for  $ F$. Therefore, we can always select points of continuity of  $ F$ that approach  $ x$ since $ h$ is arbitrary.
		\end{prf}
\end{enumerate}
\end{property}
\end{document}

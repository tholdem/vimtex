\documentclass[class=article,crop=false]{standalone} 
%Fall 2020
% Some basic packages
\usepackage{standalone}[subpreambles=true]
\usepackage[utf8]{inputenc}
\usepackage[T1]{fontenc}
\usepackage{textcomp}
\usepackage[english]{babel}
\usepackage{url}
\usepackage{graphicx}
\usepackage{float}
\usepackage{enumitem}


\pdfminorversion=7

% Don't indent paragraphs, leave some space between them
\usepackage{parskip}

% Hide page number when page is empty
\usepackage{emptypage}
\usepackage{subcaption}
\usepackage{multicol}
\usepackage[dvipsnames]{xcolor}


% Math stuff
\usepackage{amsmath, amsfonts, mathtools, amsthm, amssymb}
% Fancy script capitals
\usepackage{mathrsfs}
\usepackage{cancel}
% Bold math
\usepackage{bm}
% Some shortcuts
\newcommand{\rr}{\ensuremath{\mathbb{R}}}
\newcommand{\zz}{\ensuremath{\mathbb{Z}}}
\newcommand{\qq}{\ensuremath{\mathbb{Q}}}
\newcommand{\nn}{\ensuremath{\mathbb{N}}}
\newcommand{\ff}{\ensuremath{\mathbb{F}}}
\newcommand{\cc}{\ensuremath{\mathbb{C}}}
\renewcommand\O{\ensuremath{\emptyset}}
\newcommand{\norm}[1]{{\left\lVert{#1}\right\rVert}}
\renewcommand{\vec}[1]{{\mathbf{#1}}}
\newcommand\allbold[1]{{\boldmath\textbf{#1}}}

% Put x \to \infty below \lim
\let\svlim\lim\def\lim{\svlim\limits}

%Make implies and impliedby shorter
\let\implies\Rightarrow
\let\impliedby\Leftarrow
\let\iff\Leftrightarrow
\let\epsilon\varepsilon

% Add \contra symbol to denote contradiction
\usepackage{stmaryrd} % for \lightning
\newcommand\contra{\scalebox{1.5}{$\lightning$}}

% \let\phi\varphi

% Command for short corrections
% Usage: 1+1=\correct{3}{2}

\definecolor{correct}{HTML}{009900}
\newcommand\correct[2]{\ensuremath{\:}{\color{red}{#1}}\ensuremath{\to }{\color{correct}{#2}}\ensuremath{\:}}
\newcommand\green[1]{{\color{correct}{#1}}}

% horizontal rule
\newcommand\hr{
    \noindent\rule[0.5ex]{\linewidth}{0.5pt}
}

% hide parts
\newcommand\hide[1]{}

% si unitx
\usepackage{siunitx}
\sisetup{locale = FR}

% Environments
\makeatother
% For box around Definition, Theorem, \ldots
\usepackage[framemethod=TikZ]{mdframed}
\mdfsetup{skipabove=1em,skipbelow=0em}

%definition
\newenvironment{defn}[1][]{%
\ifstrempty{#1}%
{\mdfsetup{%
frametitle={%
\tikz[baseline=(current bounding box.east),outer sep=0pt]
\node[anchor=east,rectangle,fill=Emerald]
{\strut Definition};}}
}%
{\mdfsetup{%
frametitle={%
\tikz[baseline=(current bounding box.east),outer sep=0pt]
\node[anchor=east,rectangle,fill=Emerald]
{\strut Definition:~#1};}}%
}%
\mdfsetup{innertopmargin=10pt,linecolor=Emerald,%
linewidth=2pt,topline=true,%
frametitleaboveskip=\dimexpr-\ht\strutbox\relax
}
\begin{mdframed}[]\relax%
\label{#1}}{\end{mdframed}}


%theorem
%\newcounter{thm}[section]\setcounter{thm}{0}
%\renewcommand{\thethm}{\arabic{section}.\arabic{thm}}
\newenvironment{thm}[1][]{%
%\refstepcounter{thm}%
\ifstrempty{#1}%
{\mdfsetup{%
frametitle={%
\tikz[baseline=(current bounding box.east),outer sep=0pt]
\node[anchor=east,rectangle,fill=blue!20]
%{\strut Theorem~\thethm};}}
{\strut Theorem};}}
}%
{\mdfsetup{%
frametitle={%
\tikz[baseline=(current bounding box.east),outer sep=0pt]
\node[anchor=east,rectangle,fill=blue!20]
%{\strut Theorem~\thethm:~#1};}}%
{\strut Theorem:~#1};}}%
}%
\mdfsetup{innertopmargin=10pt,linecolor=blue!20,%
linewidth=2pt,topline=true,%
frametitleaboveskip=\dimexpr-\ht\strutbox\relax
}
\begin{mdframed}[]\relax%
\label{#1}}{\end{mdframed}}


%lemma
\newenvironment{lem}[1][]{%
\ifstrempty{#1}%
{\mdfsetup{%
frametitle={%
\tikz[baseline=(current bounding box.east),outer sep=0pt]
\node[anchor=east,rectangle,fill=Dandelion]
{\strut Lemma};}}
}%
{\mdfsetup{%
frametitle={%
\tikz[baseline=(current bounding box.east),outer sep=0pt]
\node[anchor=east,rectangle,fill=Dandelion]
{\strut Lemma:~#1};}}%
}%
\mdfsetup{innertopmargin=10pt,linecolor=Dandelion,%
linewidth=2pt,topline=true,%
frametitleaboveskip=\dimexpr-\ht\strutbox\relax
}
\begin{mdframed}[]\relax%
\label{#1}}{\end{mdframed}}

%corollary
\newenvironment{coro}[1][]{%
\ifstrempty{#1}%
{\mdfsetup{%
frametitle={%
\tikz[baseline=(current bounding box.east),outer sep=0pt]
\node[anchor=east,rectangle,fill=CornflowerBlue]
{\strut Corollary};}}
}%
{\mdfsetup{%
frametitle={%
\tikz[baseline=(current bounding box.east),outer sep=0pt]
\node[anchor=east,rectangle,fill=CornflowerBlue]
{\strut Corollary:~#1};}}%
}%
\mdfsetup{innertopmargin=10pt,linecolor=CornflowerBlue,%
linewidth=2pt,topline=true,%
frametitleaboveskip=\dimexpr-\ht\strutbox\relax
}
\begin{mdframed}[]\relax%
\label{#1}}{\end{mdframed}}

%proof
\newenvironment{prf}[1][]{%
\ifstrempty{#1}%
{\mdfsetup{%
frametitle={%
\tikz[baseline=(current bounding box.east),outer sep=0pt]
\node[anchor=east,rectangle,fill=SpringGreen]
{\strut Proof};}}
}%
{\mdfsetup{%
frametitle={%
\tikz[baseline=(current bounding box.east),outer sep=0pt]
\node[anchor=east,rectangle,fill=SpringGreen]
{\strut Proof:~#1};}}%
}%
\mdfsetup{innertopmargin=10pt,linecolor=SpringGreen,%
linewidth=2pt,topline=true,%
frametitleaboveskip=\dimexpr-\ht\strutbox\relax
}
\begin{mdframed}[]\relax%
\label{#1}}{\qed\end{mdframed}}


\theoremstyle{definition}

\newmdtheoremenv[nobreak=true]{definition}{Definition}
\newmdtheoremenv[nobreak=true]{prop}{Proposition}
\newmdtheoremenv[nobreak=true]{theorem}{Theorem}
\newmdtheoremenv[nobreak=true]{corollary}{Corollary}
\newtheorem*{eg}{Example}
\theoremstyle{remark}
\newtheorem*{case}{Case}
\newtheorem*{notation}{Notation}
\newtheorem*{remark}{Remark}
\newtheorem*{note}{Note}
\newtheorem*{problem}{Problem}
\newtheorem*{observe}{Observe}
\newtheorem*{property}{Property}
\newtheorem*{intuition}{Intuition}


% End example and intermezzo environments with a small diamond (just like proof
% environments end with a small square)
\usepackage{etoolbox}
\AtEndEnvironment{vb}{\null\hfill$\diamond$}%
\AtEndEnvironment{intermezzo}{\null\hfill$\diamond$}%
% \AtEndEnvironment{opmerking}{\null\hfill$\diamond$}%

% Fix some spacing
% http://tex.stackexchange.com/questions/22119/how-can-i-change-the-spacing-before-theorems-with-amsthm
\makeatletter
\def\thm@space@setup{%
  \thm@preskip=\parskip \thm@postskip=0pt
}

% Fix some stuff
% %http://tex.stackexchange.com/questions/76273/multiple-pdfs-with-page-group-included-in-a-single-page-warning
\pdfsuppresswarningpagegroup=1


% My name
\author{Jaden Wang}



\begin{document}
\begin{intuition}
Atom = "not splittable".
\end{intuition}
\begin{defn}[atom]
	$ A$ is an  \allbold{atom} if $ P(A) >0$ and  $ B \subseteq A \implies P(B) = 0 \text{ or } P(B)=P(A) $. 
\end{defn}

\begin{claim}[]
\[
	X(\omega) = \sum_{ i= 1}^{ n} x_i I_{A_i}(\omega), A_1,\ldots,A_n \text{ partition } \Omega 
.\] 
\begin{enumerate}[label=\arabic*)]
	\item $ E[X] = \sum_{ i= 1}^{ n} x_i P(A_i)$ is equivalent to the more familiar $ E[X]=\sum_x x P(X=x )$.
		\begin{prf}
		\begin{align*}
			\sum_x P(X=x) &= \sum_x P(\{\omega: X(\omega)=x\} ) \\
				      &= \sum_{ i= 1}^{ n} x_i P(\{\omega: X(\omega)=x_i\} ) \text{ by simple r.v.}  \\
				      &= \sum_{ i= 1}^{ n} x_i P(A_i) \\
		\end{align*}
		\end{prf}
	\item 
~\begin{defn}[independent r.v.]
	R.v.s $ X$ and  $ Y$ are independent if  $ \sigma(X)$ and $ \sigma(Y)$ are independent. That is, given any $ A \in \sigma(X)$ and $ B \in \sigma(Y)$, $ P(A \cap B) = P(A) \cdot  P(B)$.
\end{defn}
\item
~\begin{thm}[]
If $ X$ and  $ Y$ are independent, then
 \[
	 E[XY]=E[X] \cdot E[Y]
.\] 
\end{thm}
\begin{prf}
Use independence to separate the joint indicator variable and thus the double sum.

We have $ X(\omega) = \sum_{ i= 1}^{ n} x_i I_{A_i}(\omega), Y(\omega)= \sum_{ j= 1}^{ m} y_j I_{B_j}(\omega)$. Then
\[
	XY(\omega)=X(\omega) \cdot Y(\omega)=\sum_{ i= 1}^{ n} \sum_{ j= 1}^{ m} x_i y_i I_{A_i \cap  B_j}^{(\omega)}
.\]
Note $ A_i = \{\omega:X(\omega) = x_i\} \in \sigma(X)$. Likewise $ B_j \in \sigma(Y)$. Since $ X,Y$ are independent,  $ \sigma(X), \sigma(Y)$ are independent by definition. So
\[
	P(A_i \cap  B_j) = P(A_i)P(B_j)
.\] 
Therefore,
\begin{align*}
	E[XY] &= \sum_{ i= 1}^{ n} \sum_{ j= 1}^{ m} x_i y_j P(A_i \cap B_j) \\
	      &= \sum_{ i= 1}^{ n} x_i P(A_i) \sum_{ j= 1}^{ m} y_j P(B_j) \\
	      &= E[X] E[Y]\\
\end{align*}
\end{prf}
\item
~\begin{thm}[]
	If $ X=\sum_{ n= 1}^{\infty} X_n \ a.s.$, (\emph{i.e.}: $ P(\{\omega:X(\omega)=\sum_{ n= 1}^{\infty} X_n(\omega)\} ) =1$) and the partial sums of $ \sum_{ n= 1}^{\infty} X_n$ are uniformly bounded, then
	\[
		E[X]=E\left[ \sum_{ n= 1}^{\infty} X_n \right] = \sum_{ n= 1}^{\infty} E[X_n]
	.\] 
\end{thm}
\begin{note}[]
We can exchange sum and expected value in this case.
\end{note}
\begin{prf}
	By linearly of $ E[X]$ for partial sums. Let $ S_n = \sum_{ i= 1}^{ n} X_i, S= \sum_{ n= 1}^{\infty} X_n$. Then $ S_n \xrightarrow{ a.s.} S $. By assumption, $ S_n$ are uniformly bounded. Thus, by previous theorem,
	\begin{align*}
		\lim_{ n \to \infty} E[S_n] &= E[S]\\
		\lim_{ n \to \infty} E\left[ \sum_{ i= 1}^{ n} X_i \right] = \lim_{ n \to \infty} \sum_{ i= 1}^{ n} E[X_i] &= E\left[ \sum_{ n= 1}^{\infty} X_n \right] \text{ by finite sum}  \\
		\sum_{ n= 1}^{\infty} E[X_n]&= E\left[ \sum_{ n= 1}^{\infty} X_n \right]  \\
	\end{align*} 
\end{prf}
\item
~\begin{thm}[]
	Let $ g: \rr \to \rr$, then \[ g(X(\omega))= \sum_{ i= 1}^{ n} g(x_i) I_{A_i}(\omega)\] and \[E[g(X)] = \sum_{ i= 1}^{ n} g(x_i)P(A_i)\]
\end{thm}
\begin{note}[]
$ g$ might not be injective, "non-unique representation". And distinct representations of a r.v. gives us the same expectation. It is because if $ A_i \cap  B_j \neq \O$, then $ x_i = y_j$.
\end{note}

~\begin{prf}
We want to show 
\[
	\sum_{ i= 1}^{ n} x_i P(A_i) = \sum_{ j= 1}^{ m} y_i P(B_j)
\]
and we do not assume that $ x_i, y_j$ are distinct respectively.

Since by assumption
\[
	X= \sum_{ i= 1}^{ n} x_i I_{A_i} = \sum_{ j= 1}^{ m} y_j I_{B_j}
\] 
Given $ \omega \in A_i \cap B_j$, $x_i = y_j$, so we must have $ x_i=y_j$ whenever $ A_i \cap B_j \neq \O$. Notice that since $ A_i, B_j$ form partitions of $ \Omega$,
\begin{align*}
	P(A_i) &= \sum_{ j= 1}^{ m} P(A_i \cap B_j)\\
	P(B_j) &= \sum_{ i= 1}^{ n} P(A_i \cap B_j) \\
\end{align*}
Thus we have
\begin{align*}
	\sum_{ i= 1}^{ n} x_i P(A_i) &=\sum_{ i= 1}^{ n} x_i \sum_{ j= 1}^{ m} P(A_i \cap B_j) \\
	&= \sum_{ i= 1}^{ n} \sum_{ j= 1}^{ m} x_i I_{A_i \cap B_j} \\
	&= \sum_{ j= 1}^{ m} \sum_{ i= 1}^{ n} y_i I_{A_i \cap B_j}  \\
	&= \sum_{ j= 1}^{ m} y_j P(B_j) \\
\end{align*}
\end{prf}
\item
~\begin{coro}[]
	If $ X,Y$ are independent r.v.s, then  $ E[g(X)h(Y)]=E[g(X)] \cdot E[h(Y)]$
\end{coro}
\begin{note}[]
	By (3) and (5).
\end{note}
\item $ \var[X] = E[X^2] - (E[X])^2$.
~\begin{defn}[variance]
\[
	\sigma^2= \var[X] \coloneqq E[(X-\mu)^2]
.\]
\end{defn}

\end{enumerate}
\end{claim}


\section{Inequalities}
\begin{thm}[generalized Markov inequality]
 $ X \geq 0$,  $ g(x)$ real-valued and non-negative,  $ c>0$, then
		 \[
			 P(g(X)\leq c) \leq \frac{E[g(X)]}{c }
		.\]
\end{thm}

\begin{prf}
\begin{align*}
	E[g(X)]&= \sum_x g(x) \cdot P(A_i) \\
	       &= \sum_x g(x) P(X=x) \\
	       &= \sum_{\{x:g(x)\geq c\} } g(x) P(X=x) + \sum_{x: g(x) < c} g(x) P(X=x)\\
	       &\geq   \sum_{\{x:g(x)\geq c\} } g(x) P(X=x) \text{ nonnegative 2nd term} \\
	       &\geq  \sum_{\{x:g(x)\geq c\} } c \cdot  P(X=x)  \\
	       &= c \cdot P(g(X) \geq c) \\
\end{align*}
\end{prf}
\end{document}

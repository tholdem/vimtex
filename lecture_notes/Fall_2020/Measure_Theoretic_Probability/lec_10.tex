\documentclass[class=article,crop=false]{standalone} 
%Fall 2020
% Some basic packages
\usepackage{standalone}[subpreambles=true]
\usepackage[utf8]{inputenc}
\usepackage[T1]{fontenc}
\usepackage{textcomp}
\usepackage[english]{babel}
\usepackage{url}
\usepackage{graphicx}
\usepackage{float}
\usepackage{enumitem}


\pdfminorversion=7

% Don't indent paragraphs, leave some space between them
\usepackage{parskip}

% Hide page number when page is empty
\usepackage{emptypage}
\usepackage{subcaption}
\usepackage{multicol}
\usepackage[dvipsnames]{xcolor}


% Math stuff
\usepackage{amsmath, amsfonts, mathtools, amsthm, amssymb}
% Fancy script capitals
\usepackage{mathrsfs}
\usepackage{cancel}
% Bold math
\usepackage{bm}
% Some shortcuts
\newcommand{\rr}{\ensuremath{\mathbb{R}}}
\newcommand{\zz}{\ensuremath{\mathbb{Z}}}
\newcommand{\qq}{\ensuremath{\mathbb{Q}}}
\newcommand{\nn}{\ensuremath{\mathbb{N}}}
\newcommand{\ff}{\ensuremath{\mathbb{F}}}
\newcommand{\cc}{\ensuremath{\mathbb{C}}}
\renewcommand\O{\ensuremath{\emptyset}}
\newcommand{\norm}[1]{{\left\lVert{#1}\right\rVert}}
\renewcommand{\vec}[1]{{\mathbf{#1}}}
\newcommand\allbold[1]{{\boldmath\textbf{#1}}}

% Put x \to \infty below \lim
\let\svlim\lim\def\lim{\svlim\limits}

%Make implies and impliedby shorter
\let\implies\Rightarrow
\let\impliedby\Leftarrow
\let\iff\Leftrightarrow
\let\epsilon\varepsilon

% Add \contra symbol to denote contradiction
\usepackage{stmaryrd} % for \lightning
\newcommand\contra{\scalebox{1.5}{$\lightning$}}

% \let\phi\varphi

% Command for short corrections
% Usage: 1+1=\correct{3}{2}

\definecolor{correct}{HTML}{009900}
\newcommand\correct[2]{\ensuremath{\:}{\color{red}{#1}}\ensuremath{\to }{\color{correct}{#2}}\ensuremath{\:}}
\newcommand\green[1]{{\color{correct}{#1}}}

% horizontal rule
\newcommand\hr{
    \noindent\rule[0.5ex]{\linewidth}{0.5pt}
}

% hide parts
\newcommand\hide[1]{}

% si unitx
\usepackage{siunitx}
\sisetup{locale = FR}

% Environments
\makeatother
% For box around Definition, Theorem, \ldots
\usepackage[framemethod=TikZ]{mdframed}
\mdfsetup{skipabove=1em,skipbelow=0em}

%definition
\newenvironment{defn}[1][]{%
\ifstrempty{#1}%
{\mdfsetup{%
frametitle={%
\tikz[baseline=(current bounding box.east),outer sep=0pt]
\node[anchor=east,rectangle,fill=Emerald]
{\strut Definition};}}
}%
{\mdfsetup{%
frametitle={%
\tikz[baseline=(current bounding box.east),outer sep=0pt]
\node[anchor=east,rectangle,fill=Emerald]
{\strut Definition:~#1};}}%
}%
\mdfsetup{innertopmargin=10pt,linecolor=Emerald,%
linewidth=2pt,topline=true,%
frametitleaboveskip=\dimexpr-\ht\strutbox\relax
}
\begin{mdframed}[]\relax%
\label{#1}}{\end{mdframed}}


%theorem
%\newcounter{thm}[section]\setcounter{thm}{0}
%\renewcommand{\thethm}{\arabic{section}.\arabic{thm}}
\newenvironment{thm}[1][]{%
%\refstepcounter{thm}%
\ifstrempty{#1}%
{\mdfsetup{%
frametitle={%
\tikz[baseline=(current bounding box.east),outer sep=0pt]
\node[anchor=east,rectangle,fill=blue!20]
%{\strut Theorem~\thethm};}}
{\strut Theorem};}}
}%
{\mdfsetup{%
frametitle={%
\tikz[baseline=(current bounding box.east),outer sep=0pt]
\node[anchor=east,rectangle,fill=blue!20]
%{\strut Theorem~\thethm:~#1};}}%
{\strut Theorem:~#1};}}%
}%
\mdfsetup{innertopmargin=10pt,linecolor=blue!20,%
linewidth=2pt,topline=true,%
frametitleaboveskip=\dimexpr-\ht\strutbox\relax
}
\begin{mdframed}[]\relax%
\label{#1}}{\end{mdframed}}


%lemma
\newenvironment{lem}[1][]{%
\ifstrempty{#1}%
{\mdfsetup{%
frametitle={%
\tikz[baseline=(current bounding box.east),outer sep=0pt]
\node[anchor=east,rectangle,fill=Dandelion]
{\strut Lemma};}}
}%
{\mdfsetup{%
frametitle={%
\tikz[baseline=(current bounding box.east),outer sep=0pt]
\node[anchor=east,rectangle,fill=Dandelion]
{\strut Lemma:~#1};}}%
}%
\mdfsetup{innertopmargin=10pt,linecolor=Dandelion,%
linewidth=2pt,topline=true,%
frametitleaboveskip=\dimexpr-\ht\strutbox\relax
}
\begin{mdframed}[]\relax%
\label{#1}}{\end{mdframed}}

%corollary
\newenvironment{coro}[1][]{%
\ifstrempty{#1}%
{\mdfsetup{%
frametitle={%
\tikz[baseline=(current bounding box.east),outer sep=0pt]
\node[anchor=east,rectangle,fill=CornflowerBlue]
{\strut Corollary};}}
}%
{\mdfsetup{%
frametitle={%
\tikz[baseline=(current bounding box.east),outer sep=0pt]
\node[anchor=east,rectangle,fill=CornflowerBlue]
{\strut Corollary:~#1};}}%
}%
\mdfsetup{innertopmargin=10pt,linecolor=CornflowerBlue,%
linewidth=2pt,topline=true,%
frametitleaboveskip=\dimexpr-\ht\strutbox\relax
}
\begin{mdframed}[]\relax%
\label{#1}}{\end{mdframed}}

%proof
\newenvironment{prf}[1][]{%
\ifstrempty{#1}%
{\mdfsetup{%
frametitle={%
\tikz[baseline=(current bounding box.east),outer sep=0pt]
\node[anchor=east,rectangle,fill=SpringGreen]
{\strut Proof};}}
}%
{\mdfsetup{%
frametitle={%
\tikz[baseline=(current bounding box.east),outer sep=0pt]
\node[anchor=east,rectangle,fill=SpringGreen]
{\strut Proof:~#1};}}%
}%
\mdfsetup{innertopmargin=10pt,linecolor=SpringGreen,%
linewidth=2pt,topline=true,%
frametitleaboveskip=\dimexpr-\ht\strutbox\relax
}
\begin{mdframed}[]\relax%
\label{#1}}{\qed\end{mdframed}}


\theoremstyle{definition}

\newmdtheoremenv[nobreak=true]{definition}{Definition}
\newmdtheoremenv[nobreak=true]{prop}{Proposition}
\newmdtheoremenv[nobreak=true]{theorem}{Theorem}
\newmdtheoremenv[nobreak=true]{corollary}{Corollary}
\newtheorem*{eg}{Example}
\theoremstyle{remark}
\newtheorem*{case}{Case}
\newtheorem*{notation}{Notation}
\newtheorem*{remark}{Remark}
\newtheorem*{note}{Note}
\newtheorem*{problem}{Problem}
\newtheorem*{observe}{Observe}
\newtheorem*{property}{Property}
\newtheorem*{intuition}{Intuition}


% End example and intermezzo environments with a small diamond (just like proof
% environments end with a small square)
\usepackage{etoolbox}
\AtEndEnvironment{vb}{\null\hfill$\diamond$}%
\AtEndEnvironment{intermezzo}{\null\hfill$\diamond$}%
% \AtEndEnvironment{opmerking}{\null\hfill$\diamond$}%

% Fix some spacing
% http://tex.stackexchange.com/questions/22119/how-can-i-change-the-spacing-before-theorems-with-amsthm
\makeatletter
\def\thm@space@setup{%
  \thm@preskip=\parskip \thm@postskip=0pt
}

% Fix some stuff
% %http://tex.stackexchange.com/questions/76273/multiple-pdfs-with-page-group-included-in-a-single-page-warning
\pdfsuppresswarningpagegroup=1


% My name
\author{Jaden Wang}



\begin{document}
\begin{thm}[]
	Let $(X_n) $ be a sequence of r.v. on $ (\Omega,\mathcal{F},P)$. Then
	\[
		\left\{ \omega: \lim_{ n \to \infty} X_n(\omega) \text{ exists}  \right\} \coloneqq \left\{ \lim_{ n \to \infty} X_n \text{ exists}  \right\}  
	\]
	is a tail event. That is, it is in $ \mathcal{F}_T = \bigcap_{ m =1}^{\infty} \sigma(\{X_m,X_{m+1},\ldots\} )$ (let's call each individual term $ \sigma_m$, and denote $ \sigma_{\infty}$ as the whole thing).
\end{thm}

\begin{prf}
	For $ \omega \in \{\lim_{ n \to \infty} X_n \text{ exists} \} $, $ \lim_{ n \to \infty} X_n(\omega)$ exists, meaning that $ (X_n(\omega))$ is Cauchy sequence. That is, for all $ \epsilon>0$, there exists an $ N \in \nn$ such that if $ n>m \geq N$, then  $ |X_n - X_m|< \epsilon$. \emph{i.e.}:
	\begin{align*}
		\{\lim_{ n \to \infty} X_n \text{ exists} \} &= \{ \ \forall \ \epsilon>0, \ \exists \ N \text{ s.t. } \ \forall \ n>m\geq N, |X_n - X_m|< \epsilon\}  \\
		&= \bigcap_{ \epsilon>0} \bigcup_{ N} \bigcap_{ n>m\geq N} \{|X_n - X_m|< \epsilon\} 
       \end{align*}
       since rationals are dense, we can restrict $\epsilon$ to be countable. Since $ X_n$ and $X_m$ are $ \sigma_1$-measurable, $ |X_n-X_m|$ is also $ \sigma_1$-measurable by previous proofs. Then by definition of measurable, $ \{|X_n - X_m|< \epsilon\} \in \sigma_1 $. (Recall $ \{Y< \epsilon\} = \{\omega: Y(\omega) < \epsilon\} = Y^{-1}((-\infty, \epsilon))  $, where $ (- \infty, \epsilon) \in \mathcal{B}( \rr)$.) 

       Then we can repeat this argument on the shifted sequence $ \{X_m,X_{m+1},\ldots\} $, and by induction we can establish that $ \{\lim_{ n \to \infty} X_n \text{ exists} \} \in \sigma_m \ \forall \ m\geq 1$. Hence,
       \[
       \{\lim_{ n \to \infty} X_n \text{ exists} \} \in \sigma_{\infty} = \bigcap_{ m=1}^{ \infty} \sigma_m  
       .\] 
\end{prf}

\begin{eg}[valid r.v. formed from a sequence of r.v.]
~\begin{enumerate}[label=\arabic*)]
	\item $ \sup_n\left\{X_n\right\} $ and $ \inf_n\left\{ X_n \right\} $.
		\begin{prf}
			\[
				\{\omega: \sup_n X_n(\omega) \leq x\} = \bigcap_{ n=1}^{ \infty} \{\omega: X_n(\omega) \leq x\} \in \mathcal{F} \ \forall \ x \in \rr
			\]
			\[
				\{\omega: \inf_n X_n(\omega) \leq x \} = \bigcup_{ n =1}^{\infty} \{ X_n(\omega) \leq x \} \in \mathcal{F} \ \forall \ x \in \rr 
			.\] 
		\end{prf}
	\item $ \limsup_{  n} X_n$ and $ \liminf_{  n} X_n$.
		\begin{prf}
		\begin{align*}
		\{\omega: \limsup_{  n} X_n(\omega) \leq x\} &= \{\omega: \inf_n \sup_{m\geq n} X_m(\omega)\leq x\}\\ 
		&= \bigcup_{ n =1}^{\infty} \{\omega: \sup_{m\geq n} X_m(\omega)\leq x\}\\
		&= \bigcup_{ n =1}^{\infty} \bigcap_{ m=n}^{ \infty} \{\omega: X_m(\omega) \leq x\}   \\
		& \in \mathcal{F} \ \forall \ x \in \rr 
		\end{align*}
		\end{prf}
	\item If $(X_n(\omega)) $ converges $ \ \forall \ \omega \in \Omega$, then $ \lim_{ n \to \infty} X_n$ is a r.v.
		\begin{prf}
		\[
		\lim_{ n \to \infty} X_n = \limsup_{  n} X_n=\liminf_{  n} X_n \in \mathcal{F}
		.\] 
		\end{prf}
	\item If $ (X_n(\omega))$ converges for "almost all" $ \omega \in \Omega$. (It means that $ P(\{\omega: \lim_{ n \to \infty} X_n(\omega) \text{ does not exists} \} ) =0$. Define
\begin{equation*}
	X(\omega) \coloneqq
\begin{cases}
	\lim_{ n \to \infty} X_n(\omega) , & \text{ if }  (X_n(\omega)) \text{ converges}\\
	0, & \text{ otherwise} \\
\end{cases}
\end{equation*}
\end{enumerate}

\end{eg}

\begin{defn}[]
	Given a sequence of r.v. $ (X_n)$ and a r.v. $ X$ on  $ (\Omega,\mathcal{F},P)$. We say $ X_n$ \allbold{converges} to $ X$ w.p. 1 ("almost surely"), $ X_n \xrightarrow{ a.s.} X $, if  \[ P\left( \lim_{ n \to \infty} X_n=X \right) = P(\{\omega: \lim_{ n \to \infty} X_n(\omega) = X(\omega)\} ) =1.\]
\end{defn}

\begin{eg}[]
	$ \Omega = [0,1],\mathcal{F}=\mathcal{B}([0,1]), P=$ Lebesgue measure. Define $ X_n(\omega) = \omega^{n}$, \emph{i.e.} $ X_1(\omega)=\omega, X_2(\omega)=\omega^2$. Then
	\begin{equation*}
		\lim_{ n \to \infty} X_n(\omega) =
	\begin{cases}
		0 & \text{ if } \omega\neq 1\\
		1 & \text{ if } \omega=1 \\
	\end{cases}
	\end{equation*}
	Define
	\begin{equation*}
		X(\omega)=0
	\end{equation*}
	Then
	\[
		P(\{\omega:\lim_{ n \to \infty} X_n(\omega)=X(\omega)\} )=P([0,1))=1-0=1
	.\] 
	So
	\[
	X_n \xrightarrow{ a.s.} X 
	.\] 
\end{eg}

\subsection*{An alternative characterization of almost sure convergence}
~\begin{thm}
\[
	P\left( \lim_{ n \to \infty} X_n =X \right) =1 \iff P(|X_n - X| \geq \epsilon\ i.o.) =0 \ \forall \ \epsilon>0
.\]
\end{thm}
\begin{prf}

	If  $ |X_n(\omega)-X(\omega)| \geq \epsilon \ \forall \ \epsilon>0$ for finitely many $ n$, then  $ X_n(\omega) \to X(\omega)$. So
	\[
		\{\omega:\lim_{ n \to \infty} X_n(\omega)=X(\omega)\} = \{\omega: |X_n(\omega)-X(\omega)|\geq \epsilon \ i.o.\}^{c} \ \forall \ \epsilon>0  
	.\] 

\end{prf}

\begin{defn}[converges in probability]
	$ X_n$ \allbold{converges in probability} to $ X$ if  $ \ \forall \ \epsilon>0, \lim_{ n \to \infty} P(|X_n -X|\geq \epsilon) =0$. And we write $ X_n \xrightarrow{ p}  X$. 
\end{defn}
\end{document}

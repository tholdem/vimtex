\documentclass[12pt]{article}
%Fall 2020
% Some basic packages
\usepackage{standalone}[subpreambles=true]
\usepackage[utf8]{inputenc}
\usepackage[T1]{fontenc}
\usepackage{textcomp}
\usepackage[english]{babel}
\usepackage{url}
\usepackage{graphicx}
\usepackage{float}
\usepackage{enumitem}


\pdfminorversion=7

% Don't indent paragraphs, leave some space between them
\usepackage{parskip}

% Hide page number when page is empty
\usepackage{emptypage}
\usepackage{subcaption}
\usepackage{multicol}
\usepackage[dvipsnames]{xcolor}


% Math stuff
\usepackage{amsmath, amsfonts, mathtools, amsthm, amssymb}
% Fancy script capitals
\usepackage{mathrsfs}
\usepackage{cancel}
% Bold math
\usepackage{bm}
% Some shortcuts
\newcommand{\rr}{\ensuremath{\mathbb{R}}}
\newcommand{\zz}{\ensuremath{\mathbb{Z}}}
\newcommand{\qq}{\ensuremath{\mathbb{Q}}}
\newcommand{\nn}{\ensuremath{\mathbb{N}}}
\newcommand{\ff}{\ensuremath{\mathbb{F}}}
\newcommand{\cc}{\ensuremath{\mathbb{C}}}
\renewcommand\O{\ensuremath{\emptyset}}
\newcommand{\norm}[1]{{\left\lVert{#1}\right\rVert}}
\renewcommand{\vec}[1]{{\mathbf{#1}}}
\newcommand\allbold[1]{{\boldmath\textbf{#1}}}

% Put x \to \infty below \lim
\let\svlim\lim\def\lim{\svlim\limits}

%Make implies and impliedby shorter
\let\implies\Rightarrow
\let\impliedby\Leftarrow
\let\iff\Leftrightarrow
\let\epsilon\varepsilon

% Add \contra symbol to denote contradiction
\usepackage{stmaryrd} % for \lightning
\newcommand\contra{\scalebox{1.5}{$\lightning$}}

% \let\phi\varphi

% Command for short corrections
% Usage: 1+1=\correct{3}{2}

\definecolor{correct}{HTML}{009900}
\newcommand\correct[2]{\ensuremath{\:}{\color{red}{#1}}\ensuremath{\to }{\color{correct}{#2}}\ensuremath{\:}}
\newcommand\green[1]{{\color{correct}{#1}}}

% horizontal rule
\newcommand\hr{
    \noindent\rule[0.5ex]{\linewidth}{0.5pt}
}

% hide parts
\newcommand\hide[1]{}

% si unitx
\usepackage{siunitx}
\sisetup{locale = FR}

% Environments
\makeatother
% For box around Definition, Theorem, \ldots
\usepackage[framemethod=TikZ]{mdframed}
\mdfsetup{skipabove=1em,skipbelow=0em}

%definition
\newenvironment{defn}[1][]{%
\ifstrempty{#1}%
{\mdfsetup{%
frametitle={%
\tikz[baseline=(current bounding box.east),outer sep=0pt]
\node[anchor=east,rectangle,fill=Emerald]
{\strut Definition};}}
}%
{\mdfsetup{%
frametitle={%
\tikz[baseline=(current bounding box.east),outer sep=0pt]
\node[anchor=east,rectangle,fill=Emerald]
{\strut Definition:~#1};}}%
}%
\mdfsetup{innertopmargin=10pt,linecolor=Emerald,%
linewidth=2pt,topline=true,%
frametitleaboveskip=\dimexpr-\ht\strutbox\relax
}
\begin{mdframed}[]\relax%
\label{#1}}{\end{mdframed}}


%theorem
%\newcounter{thm}[section]\setcounter{thm}{0}
%\renewcommand{\thethm}{\arabic{section}.\arabic{thm}}
\newenvironment{thm}[1][]{%
%\refstepcounter{thm}%
\ifstrempty{#1}%
{\mdfsetup{%
frametitle={%
\tikz[baseline=(current bounding box.east),outer sep=0pt]
\node[anchor=east,rectangle,fill=blue!20]
%{\strut Theorem~\thethm};}}
{\strut Theorem};}}
}%
{\mdfsetup{%
frametitle={%
\tikz[baseline=(current bounding box.east),outer sep=0pt]
\node[anchor=east,rectangle,fill=blue!20]
%{\strut Theorem~\thethm:~#1};}}%
{\strut Theorem:~#1};}}%
}%
\mdfsetup{innertopmargin=10pt,linecolor=blue!20,%
linewidth=2pt,topline=true,%
frametitleaboveskip=\dimexpr-\ht\strutbox\relax
}
\begin{mdframed}[]\relax%
\label{#1}}{\end{mdframed}}


%lemma
\newenvironment{lem}[1][]{%
\ifstrempty{#1}%
{\mdfsetup{%
frametitle={%
\tikz[baseline=(current bounding box.east),outer sep=0pt]
\node[anchor=east,rectangle,fill=Dandelion]
{\strut Lemma};}}
}%
{\mdfsetup{%
frametitle={%
\tikz[baseline=(current bounding box.east),outer sep=0pt]
\node[anchor=east,rectangle,fill=Dandelion]
{\strut Lemma:~#1};}}%
}%
\mdfsetup{innertopmargin=10pt,linecolor=Dandelion,%
linewidth=2pt,topline=true,%
frametitleaboveskip=\dimexpr-\ht\strutbox\relax
}
\begin{mdframed}[]\relax%
\label{#1}}{\end{mdframed}}

%corollary
\newenvironment{coro}[1][]{%
\ifstrempty{#1}%
{\mdfsetup{%
frametitle={%
\tikz[baseline=(current bounding box.east),outer sep=0pt]
\node[anchor=east,rectangle,fill=CornflowerBlue]
{\strut Corollary};}}
}%
{\mdfsetup{%
frametitle={%
\tikz[baseline=(current bounding box.east),outer sep=0pt]
\node[anchor=east,rectangle,fill=CornflowerBlue]
{\strut Corollary:~#1};}}%
}%
\mdfsetup{innertopmargin=10pt,linecolor=CornflowerBlue,%
linewidth=2pt,topline=true,%
frametitleaboveskip=\dimexpr-\ht\strutbox\relax
}
\begin{mdframed}[]\relax%
\label{#1}}{\end{mdframed}}

%proof
\newenvironment{prf}[1][]{%
\ifstrempty{#1}%
{\mdfsetup{%
frametitle={%
\tikz[baseline=(current bounding box.east),outer sep=0pt]
\node[anchor=east,rectangle,fill=SpringGreen]
{\strut Proof};}}
}%
{\mdfsetup{%
frametitle={%
\tikz[baseline=(current bounding box.east),outer sep=0pt]
\node[anchor=east,rectangle,fill=SpringGreen]
{\strut Proof:~#1};}}%
}%
\mdfsetup{innertopmargin=10pt,linecolor=SpringGreen,%
linewidth=2pt,topline=true,%
frametitleaboveskip=\dimexpr-\ht\strutbox\relax
}
\begin{mdframed}[]\relax%
\label{#1}}{\qed\end{mdframed}}


\theoremstyle{definition}

\newmdtheoremenv[nobreak=true]{definition}{Definition}
\newmdtheoremenv[nobreak=true]{prop}{Proposition}
\newmdtheoremenv[nobreak=true]{theorem}{Theorem}
\newmdtheoremenv[nobreak=true]{corollary}{Corollary}
\newtheorem*{eg}{Example}
\theoremstyle{remark}
\newtheorem*{case}{Case}
\newtheorem*{notation}{Notation}
\newtheorem*{remark}{Remark}
\newtheorem*{note}{Note}
\newtheorem*{problem}{Problem}
\newtheorem*{observe}{Observe}
\newtheorem*{property}{Property}
\newtheorem*{intuition}{Intuition}


% End example and intermezzo environments with a small diamond (just like proof
% environments end with a small square)
\usepackage{etoolbox}
\AtEndEnvironment{vb}{\null\hfill$\diamond$}%
\AtEndEnvironment{intermezzo}{\null\hfill$\diamond$}%
% \AtEndEnvironment{opmerking}{\null\hfill$\diamond$}%

% Fix some spacing
% http://tex.stackexchange.com/questions/22119/how-can-i-change-the-spacing-before-theorems-with-amsthm
\makeatletter
\def\thm@space@setup{%
  \thm@preskip=\parskip \thm@postskip=0pt
}

% Fix some stuff
% %http://tex.stackexchange.com/questions/76273/multiple-pdfs-with-page-group-included-in-a-single-page-warning
\pdfsuppresswarningpagegroup=1


% My name
\author{Jaden Wang}



\begin{document}
\centerline {\textsf{\textbf{\LARGE{Homework 3}}}}
\centerline {Jaden Wang}
\vspace{.15in}
\begin{problem}[1]
~\\	
$ (\implies)$: Given $ \epsilon > 0$, since almost surely convergence implies convergence in probability,  $ Z_n \xrightarrow{ a.s} Z \implies \lim_{ n \to \infty} P(|Z_n - Z|\geq \epsilon) =0 $. That is, there exists $ n \in \nn$ such that 
	\[
		P(|Z_k - Z|\geq \epsilon, k\geq n)< \epsilon
	.\] 
	Taking the complement yields
\begin{align*}
	P((|Z_k - Z|\geq \epsilon, k \geq n)^{c}) &> 1- \epsilon\\
	P(|Z_k - Z|< \epsilon, n\leq k)&> 1- \epsilon\\
	P(|Z_k - Z|< \epsilon, n\leq k\leq m)&> 1- \epsilon \ \forall \ m \geq n\\
\end{align*}	
The last step follows from that if the statement is true for all $ k \geq n$, then it's true for a subset of such  $ k$ where  $ n \leq k \leq m$ for all $ m \geq n$.

$ (\impliedby)$: Suppose for every $ \epsilon>0$ there exists an $ n$  such that 
\[
	P(|Z_k -Z| < \epsilon, n\leq k\leq m)> 1 - \epsilon
\] 
for all $ m \geq n$. By taking $ n \to \infty$, it will cover the $ n$ required by any arbitrarily small $ \epsilon$, so we can let $ \epsilon \to 0$ and obtain
\[
	P(\lim_{ n \to \infty} Z_n = Z) \geq 1 
.\] 
Since  $ P$ is a probability measure,  $ P(\lim_{ n \to \infty} Z_n = Z) \leq 1$. Hence, 
\[
P(\lim_{ n \to \infty} Z_n = Z) = 1 
\] 
which is the definition of almost sure convergence.
\end{problem}

\begin{problem}[2]
$ ( \subseteq )$: We wish to show that 
\[
	\mathcal{G} = \{(H \cap A) \cup (H^{c} \cap B), A,B \in \mathcal{F}\} 
\]
is a $\sigma$-field containing $ \mathcal{F} \cup \{H\} $.


First, notice that given $ F \in \mathcal{F}$, \[ F = \Omega \cap F= (H \cup H^{c}) \cap F= (H \cap F) \cup (H^{c} \cap F)  \in \mathcal{G}.\] Moreover, since $ \O, \Omega \in \mathcal{F}$, \[H = (H \cap \Omega) \cup (H^{c} \cap \O) \in \mathcal{G}.\] Thus, $ \mathcal{F} \cup \{H\} \subseteq \mathcal{G}$.  

Now let's show that $ \mathcal{G}$ is a $\sigma$-field.
\begin{enumerate}[label=(\roman*)]
	\item Take $ A=B=\Omega \in \mathcal{F}$, we have
		\[
			(H \cap \Omega) \cup (H^{c} \cap \Omega) = H \cup H^{c} = \Omega \in \mathcal{G}
		.\] 
	\item Given $ S \in \mathcal{G}$, we know there exist $ A,B \in \mathcal{F}$ such that $ S=(H \cap A) \cup (H^{c} \cap B)$. Then the complement is
		\begin{align*}
			S^{c} &= ((H \cap A) \cup (H^{c} \cap B) )^{c}\\
			      &= (H \cap A)^{c} \cap (H^{c} \cap B)^{c}\\
			      &= (H^{c} \cup A^{c}) \cap (H \cup B^{c}) \\
			      &= (H^{c} \cap H) \cup (H^{c} \cap B^{c}) \cup (A^{c} \cap H) \cup (A^{c} \cap B^{c}) \\
			      &= (H \cap A^{c}) \cup (H^{c} \cap B^{c}) \cup ((H \cup H^{c}) \cap  (A^{c} \cap B^{c}))   \\
			      &=  (H \cap A^{c}) \cup (H^{c} \cap B^{c}) \cup (H \cap (A^{c} \cap B^{c} )) \cup (H^{c} \cap (A^{c} \cap B^{c}))  \\
			      &= (H \cap (A^{c} \cup (A^{c} \cap B^{c})) \cup (H^{c} \cap (B^{c} \cup (A^{c} \cap B^{c})) \\
			      &= (H \cap A^{c}) \cup (H^{c} \cap B^{c}) \in \mathcal{G}\\
		\end{align*}
	\item Given a sequence $ G_1,G_2,\ldots \in \mathcal{G}$, we can express $ G_n$ as $ (H \cap A_n) \cup (H^{c} \cap B_n)$ for some sequence of $ (A_n),(B_n) \subseteq \mathcal{F}$. Then we have
		\begin{align*}
			\bigcup_{ n =1}^{\infty} G_n &= \bigcup_{ n =1}^{\infty} (H \cap A_n) \cup (H^{c} \cap B_n) \\
						     &= \bigcup_{ n =1}^{\infty} (H \cap A_n) \cup \bigcup_{ n =1}^{\infty} (H^{c} \cap B_n) \\
						     &= \left(H \cap \bigcup_{ n =1}^{\infty} A_n \right) \cup \left(H^{c} \cap \bigcup_{ n =1}^{\infty} B_n \right) \\
		\end{align*}
\end{enumerate}
Hence, we show that $ \mathcal{G}$ is a $\sigma$-field containing $ \mathcal{F} \cup \{H\} $. Since $ \sigma(\mathcal{F} \cap \{H\}) $ is the smallest $\sigma$-field containing $ \mathcal{F} \cup \{H\} $, we prove that $ \sigma(\mathcal{F} \cup  \{H\} ) \subseteq \mathcal{G}$, hence its elements must have such form.

$ ( \supseteq )$: Given $ G \in \mathcal{G}$, let's show that $ G \in \sigma(\mathcal{F} \cup \{H\}) $. By definition of $ \mathcal{G}$, there exists $ A,B \in \mathcal{F}$ such that $ G= (H ^{c} \cap A) \cup (H^{c} \cap B)$.

\begin{align*}
	(H \cap A) \cup (H^{c} \cap B) &= ((H \cup H^{c}) \cap (H \cup B)) \cap ((A \cup H^{c}) \cap (A \cup B))\\
				       &= (H \cup B) \cap (H^{c} \cup A) \cap (A \cup B)  \\
\end{align*}
Since $ H, H^{c}, A, B \in \sigma(\mathcal{F} \cup \{H\} )$, their unions and intersections are also in it. Thus, $ G \in \sigma(\mathcal{F} \cup \{H\} )$ and $ \mathcal{G} \subseteq \sigma(\mathcal{F} \cup \{H\} )$.

By double-containment, we show that $ \sigma(\mathcal{F} \cup \{H\} ) = \mathcal{G}$.
\end{problem}

\begin{problem}[3]

	Recall that $ \mu(A \cup B) = \mu((A \cup B) \setminus (A \cap B)) + \mu(A \cap B)$ by finite additivity. Moreover, $ \mu(A), \mu(B) \leq \mu(A \cup B)$ and $ \mu(A), \mu(B) \geq \mu(A \cap B)$ by monotonicity. Thus, $\mu(A) - \mu(B) \leq \mu(A \cup B) - \mu(A \cap B), \mu(B) -\mu(A) \leq \mu(A \cup B) - \mu(A \cap B) \implies |\mu(A)-\mu(B)| \leq \mu(A \cup B) - \mu(A \cap B)$.
\begin{align*}
	\mu(A \Delta B) &= \mu((A \cap B^{c}) \cup (A^{c} \cap B)) \\
			&= \mu((A \cup A^{c}) \cap (B^{c} \cap A^{c}) \cap (A \cup B) \cap (B^{c} \cup B)) \\
			&= \mu(\Omega \cap (A^{c} \cup B^{c}) \cap (A \cup B) \cap \Omega) \\
			&= \mu((A \cap B)^{c} \cap (A \cup B)) \\
			&= \mu((A \cup B) \setminus (A \cap B)) \\
			&= \mu(A \cup B) - \mu(A \cap B) \\
			&\geq |\mu(A) - \mu(B)|  
\end{align*}
\end{problem}

\begin{problem}[4]
~\begin{enumerate}[label=(\roman*)]
	\item Since $ P$ is a probability measure,  $ 0 \leq P(\{\omega: X(\omega) \in A\} ) \leq 1 \ \forall \ A \subseteq \rr \implies 0\leq P_X(A) \leq 1 \ \forall \ A \subseteq \rr$.
	\item $ P_X( \O) = P(\{\omega: X(\omega) \in \O\} ) = P( \O)=0$. And $ P_X(\rr) = P(\{\omega:X(\omega) \in \rr\} ) = P(\Omega) =1$.
	\item Given a sequence of disjoint sets in $ \rr$, $ (A_n)$,
		\begin{align*}
			P_X\left( \bigcup_{ n =1}^{\infty} A_n \right) &= P\left( \left\{ \omega:X(\omega) \in \bigcup_{ n =1}^{\infty} A_n \right\}   \right)  \\
								       &= P\left( \bigcup_{ n =1}^{\infty} \{\omega: X(\omega) \in A_n\}  \right)   \\
								       &= \sum_{ n= 1}^{\infty} P(\{\omega:X(\omega) \in A_n\} ) \text{ by countable additivity of } P\\
								       &= \sum_{ n= 1}^{\infty} P_X(A_n) 
		\end{align*}
		Thus, $ P_X$ is indeed a probability measure.
\end{enumerate}
\end{problem}

\begin{problem}[5]
	First let's show that $ \mu$ is finitely additive.

	Given disjoint $ A,B \in \sigma(\Omega)$, if both $ A,B$ are finite,  then disjointness gives us
	\[
		\mu(A \cup B) = \sum_{k \in A \cup B} 2^{-k} = \sum_{k \in A} 2^{-k} + \sum_{k \in B} 2^{-k} = \mu(A) + \mu(B)
	.\]
	If at least one of them is infinite, then $ A \cup B$ is infinite. Thus,
	\[
		\mu(A \cup B) = \infty =\mu(A) + \mu(B)
	.\]

	However, we claim that $ \mu$ is not countably additive. Here is an counterexample: let $ A_n = \{n\} $, so they are disjoint and finite. It's easy to see that $ \bigcup_{ n =1}^{\infty} A_n =\Omega$ is infinite, so $ \mu\left( \bigcup_{ n =1}^{\infty} A_n \right) = \infty $. However,
\begin{align*}
	\sum_{n=1}^{\infty} \mu(A_n) &= \sum_{ n= 1}^{\infty} \sum_{k=n}^n 2^{-k}\\
	&= \sum_{n= 1}^{\infty} 2^{-n} \\
	&= 1 < \infty \\
\end{align*} 
Therefore, $\mu$ is not countably additive.
\end{problem}
\end{document}

\documentclass[12pt]{article}
%Fall 2020
% Some basic packages
\usepackage{standalone}[subpreambles=true]
\usepackage[utf8]{inputenc}
\usepackage[T1]{fontenc}
\usepackage{textcomp}
\usepackage[english]{babel}
\usepackage{url}
\usepackage{graphicx}
\usepackage{float}
\usepackage{enumitem}


\pdfminorversion=7

% Don't indent paragraphs, leave some space between them
\usepackage{parskip}

% Hide page number when page is empty
\usepackage{emptypage}
\usepackage{subcaption}
\usepackage{multicol}
\usepackage[dvipsnames]{xcolor}


% Math stuff
\usepackage{amsmath, amsfonts, mathtools, amsthm, amssymb}
% Fancy script capitals
\usepackage{mathrsfs}
\usepackage{cancel}
% Bold math
\usepackage{bm}
% Some shortcuts
\newcommand{\rr}{\ensuremath{\mathbb{R}}}
\newcommand{\zz}{\ensuremath{\mathbb{Z}}}
\newcommand{\qq}{\ensuremath{\mathbb{Q}}}
\newcommand{\nn}{\ensuremath{\mathbb{N}}}
\newcommand{\ff}{\ensuremath{\mathbb{F}}}
\newcommand{\cc}{\ensuremath{\mathbb{C}}}
\renewcommand\O{\ensuremath{\emptyset}}
\newcommand{\norm}[1]{{\left\lVert{#1}\right\rVert}}
\renewcommand{\vec}[1]{{\mathbf{#1}}}
\newcommand\allbold[1]{{\boldmath\textbf{#1}}}

% Put x \to \infty below \lim
\let\svlim\lim\def\lim{\svlim\limits}

%Make implies and impliedby shorter
\let\implies\Rightarrow
\let\impliedby\Leftarrow
\let\iff\Leftrightarrow
\let\epsilon\varepsilon

% Add \contra symbol to denote contradiction
\usepackage{stmaryrd} % for \lightning
\newcommand\contra{\scalebox{1.5}{$\lightning$}}

% \let\phi\varphi

% Command for short corrections
% Usage: 1+1=\correct{3}{2}

\definecolor{correct}{HTML}{009900}
\newcommand\correct[2]{\ensuremath{\:}{\color{red}{#1}}\ensuremath{\to }{\color{correct}{#2}}\ensuremath{\:}}
\newcommand\green[1]{{\color{correct}{#1}}}

% horizontal rule
\newcommand\hr{
    \noindent\rule[0.5ex]{\linewidth}{0.5pt}
}

% hide parts
\newcommand\hide[1]{}

% si unitx
\usepackage{siunitx}
\sisetup{locale = FR}

% Environments
\makeatother
% For box around Definition, Theorem, \ldots
\usepackage[framemethod=TikZ]{mdframed}
\mdfsetup{skipabove=1em,skipbelow=0em}

%definition
\newenvironment{defn}[1][]{%
\ifstrempty{#1}%
{\mdfsetup{%
frametitle={%
\tikz[baseline=(current bounding box.east),outer sep=0pt]
\node[anchor=east,rectangle,fill=Emerald]
{\strut Definition};}}
}%
{\mdfsetup{%
frametitle={%
\tikz[baseline=(current bounding box.east),outer sep=0pt]
\node[anchor=east,rectangle,fill=Emerald]
{\strut Definition:~#1};}}%
}%
\mdfsetup{innertopmargin=10pt,linecolor=Emerald,%
linewidth=2pt,topline=true,%
frametitleaboveskip=\dimexpr-\ht\strutbox\relax
}
\begin{mdframed}[]\relax%
\label{#1}}{\end{mdframed}}


%theorem
%\newcounter{thm}[section]\setcounter{thm}{0}
%\renewcommand{\thethm}{\arabic{section}.\arabic{thm}}
\newenvironment{thm}[1][]{%
%\refstepcounter{thm}%
\ifstrempty{#1}%
{\mdfsetup{%
frametitle={%
\tikz[baseline=(current bounding box.east),outer sep=0pt]
\node[anchor=east,rectangle,fill=blue!20]
%{\strut Theorem~\thethm};}}
{\strut Theorem};}}
}%
{\mdfsetup{%
frametitle={%
\tikz[baseline=(current bounding box.east),outer sep=0pt]
\node[anchor=east,rectangle,fill=blue!20]
%{\strut Theorem~\thethm:~#1};}}%
{\strut Theorem:~#1};}}%
}%
\mdfsetup{innertopmargin=10pt,linecolor=blue!20,%
linewidth=2pt,topline=true,%
frametitleaboveskip=\dimexpr-\ht\strutbox\relax
}
\begin{mdframed}[]\relax%
\label{#1}}{\end{mdframed}}


%lemma
\newenvironment{lem}[1][]{%
\ifstrempty{#1}%
{\mdfsetup{%
frametitle={%
\tikz[baseline=(current bounding box.east),outer sep=0pt]
\node[anchor=east,rectangle,fill=Dandelion]
{\strut Lemma};}}
}%
{\mdfsetup{%
frametitle={%
\tikz[baseline=(current bounding box.east),outer sep=0pt]
\node[anchor=east,rectangle,fill=Dandelion]
{\strut Lemma:~#1};}}%
}%
\mdfsetup{innertopmargin=10pt,linecolor=Dandelion,%
linewidth=2pt,topline=true,%
frametitleaboveskip=\dimexpr-\ht\strutbox\relax
}
\begin{mdframed}[]\relax%
\label{#1}}{\end{mdframed}}

%corollary
\newenvironment{coro}[1][]{%
\ifstrempty{#1}%
{\mdfsetup{%
frametitle={%
\tikz[baseline=(current bounding box.east),outer sep=0pt]
\node[anchor=east,rectangle,fill=CornflowerBlue]
{\strut Corollary};}}
}%
{\mdfsetup{%
frametitle={%
\tikz[baseline=(current bounding box.east),outer sep=0pt]
\node[anchor=east,rectangle,fill=CornflowerBlue]
{\strut Corollary:~#1};}}%
}%
\mdfsetup{innertopmargin=10pt,linecolor=CornflowerBlue,%
linewidth=2pt,topline=true,%
frametitleaboveskip=\dimexpr-\ht\strutbox\relax
}
\begin{mdframed}[]\relax%
\label{#1}}{\end{mdframed}}

%proof
\newenvironment{prf}[1][]{%
\ifstrempty{#1}%
{\mdfsetup{%
frametitle={%
\tikz[baseline=(current bounding box.east),outer sep=0pt]
\node[anchor=east,rectangle,fill=SpringGreen]
{\strut Proof};}}
}%
{\mdfsetup{%
frametitle={%
\tikz[baseline=(current bounding box.east),outer sep=0pt]
\node[anchor=east,rectangle,fill=SpringGreen]
{\strut Proof:~#1};}}%
}%
\mdfsetup{innertopmargin=10pt,linecolor=SpringGreen,%
linewidth=2pt,topline=true,%
frametitleaboveskip=\dimexpr-\ht\strutbox\relax
}
\begin{mdframed}[]\relax%
\label{#1}}{\qed\end{mdframed}}


\theoremstyle{definition}

\newmdtheoremenv[nobreak=true]{definition}{Definition}
\newmdtheoremenv[nobreak=true]{prop}{Proposition}
\newmdtheoremenv[nobreak=true]{theorem}{Theorem}
\newmdtheoremenv[nobreak=true]{corollary}{Corollary}
\newtheorem*{eg}{Example}
\theoremstyle{remark}
\newtheorem*{case}{Case}
\newtheorem*{notation}{Notation}
\newtheorem*{remark}{Remark}
\newtheorem*{note}{Note}
\newtheorem*{problem}{Problem}
\newtheorem*{observe}{Observe}
\newtheorem*{property}{Property}
\newtheorem*{intuition}{Intuition}


% End example and intermezzo environments with a small diamond (just like proof
% environments end with a small square)
\usepackage{etoolbox}
\AtEndEnvironment{vb}{\null\hfill$\diamond$}%
\AtEndEnvironment{intermezzo}{\null\hfill$\diamond$}%
% \AtEndEnvironment{opmerking}{\null\hfill$\diamond$}%

% Fix some spacing
% http://tex.stackexchange.com/questions/22119/how-can-i-change-the-spacing-before-theorems-with-amsthm
\makeatletter
\def\thm@space@setup{%
  \thm@preskip=\parskip \thm@postskip=0pt
}

% Fix some stuff
% %http://tex.stackexchange.com/questions/76273/multiple-pdfs-with-page-group-included-in-a-single-page-warning
\pdfsuppresswarningpagegroup=1


% My name
\author{Jaden Wang}



\begin{document}
\centerline {\textsf{\textbf{\LARGE{Midterm 1 }}}}
\centerline {Jaden Wang}
\vspace{.15in}
\begin{problem}[1]
~\begin{enumerate}[label=\alph*)]
	\item Given $ A,B \in \mathcal{G}$, by definition of $ \mathcal{G}$ there exist $ A_1, B_1 \in \mathcal{F}_1$ and $ A_2, B_2 \in \mathcal{F}_2$ such that $ A=A_1 \cap A_2$, $ B= B_1 \cap B_2$. Then
\begin{align*}
	A \cap B &= (A_1 \cap A_2) \cap (B_1 \cap B_2) \\
		 &= (A_1 \cap B_1) \cap (A_2 \cap B_2)
\end{align*}
Since $\sigma$-fields $ \mathcal{F}_1, \mathcal{F}_2$ are closed under intersections, it follows that $ A_1 \cap B_1 \in \mathcal{F}_1, A_2 \cap B_2 \in \mathcal{F}_2$. Hence $ A \cap B \in \mathcal{G}$, and $ \mathcal{G}$ is a $ \pi$-system by definition.
\item We wish to show double containment. Given $ a \in \mathcal{G}$, there exists $ A_1 \in \mathcal{F}_1, A_2 \in \mathcal{F}_2$ such that $ a \in A_1 \cap A_2$. Since $ \mathcal{F}_1,\mathcal{F}_2$ are $\sigma$-fields, the complements $ A_1^{c} \in \mathcal{F}_1, A_2^{c} \in \mathcal{F}_2 \implies A_1^{c} \cup A_2^{c} \in \mathcal{F}_1 \cup \mathcal{F}_2$. Then  $ (A_1^{c} \cup A_2^{c})^{c} = A_1 \cap A_2 \in \sigma(F_1 \cup F_2) \implies \mathcal{G} \subseteq \sigma(\mathcal{F}_1 \cup \mathcal{F}_2)$ . Since a $\sigma$-field is also a $\lambda$-system, by Dynkin's $ \pi$-$ \lambda$ theorem, $ \sigma(\mathcal{G}) \subseteq \sigma(\mathcal{F}_1 \cup  \mathcal{F}_2)$.

	Now let's show the other direction. Given $ b \in \mathcal{F}_1 \cup \mathcal{F}_2$, there exist $ B_1 \in \mathcal{F}_1, B_2 \in \mathcal{F}_2$ such that $ b \in B_1 \cup B_2 = (B_1^{c} \cap B_2^{c})^{c}$. Clearly $ B_1^{c} \in \mathcal{F}_1, B_2^{c} \in \mathcal{F}_2$, so $ B_1^{c} \cap B_2^{c} \in \mathcal{G} \implies (B_1^{c} \cap B_2^{c})^{c} \in \sigma(\mathcal{G}) \implies b \in \sigma(\mathcal{G}) \implies \mathcal{F}_1 \cup \mathcal{F}_2 \subseteq \sigma(\mathcal{G})$. Again by Dynkin's Theorem, $ \sigma(\mathcal{F}_1 \cup \mathcal{F}_2) \subseteq \sigma(\mathcal{G})$. Therefore, we obtain $ \sigma(\mathcal{G})= \sigma(\mathcal{F}_1 \cup  \mathcal{F}_2)$. 
\end{enumerate}
\end{problem}

\begin{problem}[2]
	We would like to prove by contradiction. Suppose there exists an atom $ A \in \mathcal{F}$ such that $ P(A)>0$ and  $ \ \forall \ B \in \mathcal{F}, B \subseteq A$, we have $ P(B)=0$ or $ P(B)=P(A)$. Let's define
	 \begin{equation*}
	B_n =
	\begin{cases}
		A_n & \text{ if } P(A_n \cap A) = P(A)\\ 	
		A_n^{c} & \text{ if } P(A_n^{c} \cap A) = P(A)\\
	\end{cases}
	\end{equation*}
	Notice that since each $B_n $ only depends on $ A_n$, and $ A_n$ are independent, we know $ B_n$ are independent too. This definition intuitively means that all the $ B_n$ always contain the mass of $ A$. In other words, 
	 \begin{align*}
		 P(A) &\leq P(B_1 \cap B_2 \cap \ldots )\\
		      &= P(B_1)P(B_2)\ldots \text{ by independence} \\
		      &= p_1 \cdot  p_2 \cdot  \ldots 
	\end{align*}
	where we let $p_n = P(B_n)$ for convenience. Recall that $ 1-x \leq e^{-x} \implies x \leq e^{-(1-x)}$, applying to each $ p_n$ and we have
	\begin{align*}
		P(A) &\leq e^{-(1-p_1)} e^{-(1-p_2)} \ldots \\
		\ln P(A) &\leq - \sum_{ n= 1}^{\infty} (1-p_n) \qquad \text{ by monotonicity of }\ln \\ 
		\sum_{ n= 1}^{\infty} (1-p_n) &\leq -\ln P(A) = \ln \frac{1}{P(A)}
	\end{align*}
	Since $ P(A) > 0$,  $ \ln \frac{1}{P(A)}$ is a constant so $ \sum_{ n= 1}^{\infty} (1-p_n) < \infty$.
	Now consider
	\[
		\sum_n \min \{P(A_n), 1-P(A_n)\} = \sum_n \min \{P(B_n), 1-P(B_n)\} \leq \sum_n (1-p_n) < \infty 
	.\] 
	This contradicts with our assumption, therefore we prove that the probability space must be non-atomic.
\end{problem}
\begin{problem}[3]
	Naturally we apply Borel-Cantelli Lemma (i) and obtain $ P\left( \limsup_{  n} (A_n \cap A_{n+1}^{c}) \right) =0$. Also by Theorem 4.1, $ P\left( \liminf_{  n} A_n \right) \leq \liminf_{  n} P(A_n) = 0 \implies P\left( \liminf_{  n} A_n \right) =0$. By homework 2.6,
\begin{align*}
	P\left( \limsup_{  n} A_n \right) &= P\left( \limsup_{  n} (A_n \cap A_{n+1}^{c}) \cup \liminf_{  n} A_n \right)  \\
					  &\leq P\left( \limsup_{  n} (A_n \cap A_{n+1}^{c} \right) + P\left( \liminf_{  n} A_n \right) \text{ by subadditivity}    \\
					  &= 0 + 0 = 0
\end{align*}
Again by Theorem 4.1, $ \limsup_{n} P(A_n) \leq P\left( \limsup_{  n} A_n \right)  =0 \implies \limsup_{  n} P(A_n) = 0 = \liminf_{  n} P(A_n) \implies \lim_{ n \to \infty} P(A_n) = 0$ by definition of the limit of reals.
\end{problem}

\begin{problem}[4]
	We would like to show that $ \{\omega: X(\omega) \leq x\} \in \mathcal{F} \ \forall \ x \in \rr$.
\begin{case}[]
$ x\geq 0$, then
 \begin{equation*}
	 \{\omega: X(\omega) \leq x\} = \{\omega: \lim_{ n \to \infty} X_n(\omega) \leq x  \} \cup \{\omega: \lim_{ n \to \infty} X_n(\omega) \text{ diverges} \}   
\end{equation*}
In class we have already show that the first set is in $ \mathcal{F}$. Notice that
$ \{\omega: \lim_{ n \to \infty} X_n(\omega) \text{ diverges} \}  = \{\omega: \lim_{ n \to \infty} X_n(\omega) \text{ exists} \}^{c} $, which we showed in class that it is in the tail $\sigma$-field. Since the tail $\sigma$-field is the smallest $\sigma$-field containing the tail event, and the tail event is clearly contained in $ \mathcal{F}$, the tail $\sigma$-field must be a subset of $ \mathcal{F}$. It follows that $ \{\omega: \lim_{ n \to \infty} X_n(\omega)\} \in \mathcal{F}$, which implies the complement $ \{\omega: \lim_{ n \to \infty} X_n(\omega) \text{ diverges} \} \in \mathcal{F}$. Therefore, the union of two sets, $ \{\omega:X(\omega) \leq x\} \in \mathcal{F} $.
\end{case}
\begin{case}[]
	$ x<0$, then  $ \{\omega:X(\omega)\leq x\} = \{\omega: \lim_{ n \to \infty} X_n(\omega) \leq x\} \in \mathcal{F} $ as shown in class.
\end{case}
Hence for all $ x \in \rr$, we show that $ \{\omega:X(\omega) \leq x\} $, proving that $ X$ is a random variable.
\end{problem}
\end{document}

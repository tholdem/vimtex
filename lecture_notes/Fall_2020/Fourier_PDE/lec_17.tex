\documentclass[class=article,crop=false]{standalone} 
%Fall 2020
% Some basic packages
\usepackage{standalone}[subpreambles=true]
\usepackage[utf8]{inputenc}
\usepackage[T1]{fontenc}
\usepackage{textcomp}
\usepackage[english]{babel}
\usepackage{url}
\usepackage{graphicx}
\usepackage{float}
\usepackage{enumitem}


\pdfminorversion=7

% Don't indent paragraphs, leave some space between them
\usepackage{parskip}

% Hide page number when page is empty
\usepackage{emptypage}
\usepackage{subcaption}
\usepackage{multicol}
\usepackage[dvipsnames]{xcolor}


% Math stuff
\usepackage{amsmath, amsfonts, mathtools, amsthm, amssymb}
% Fancy script capitals
\usepackage{mathrsfs}
\usepackage{cancel}
% Bold math
\usepackage{bm}
% Some shortcuts
\newcommand{\rr}{\ensuremath{\mathbb{R}}}
\newcommand{\zz}{\ensuremath{\mathbb{Z}}}
\newcommand{\qq}{\ensuremath{\mathbb{Q}}}
\newcommand{\nn}{\ensuremath{\mathbb{N}}}
\newcommand{\ff}{\ensuremath{\mathbb{F}}}
\newcommand{\cc}{\ensuremath{\mathbb{C}}}
\renewcommand\O{\ensuremath{\emptyset}}
\newcommand{\norm}[1]{{\left\lVert{#1}\right\rVert}}
\renewcommand{\vec}[1]{{\mathbf{#1}}}
\newcommand\allbold[1]{{\boldmath\textbf{#1}}}

% Put x \to \infty below \lim
\let\svlim\lim\def\lim{\svlim\limits}

%Make implies and impliedby shorter
\let\implies\Rightarrow
\let\impliedby\Leftarrow
\let\iff\Leftrightarrow
\let\epsilon\varepsilon

% Add \contra symbol to denote contradiction
\usepackage{stmaryrd} % for \lightning
\newcommand\contra{\scalebox{1.5}{$\lightning$}}

% \let\phi\varphi

% Command for short corrections
% Usage: 1+1=\correct{3}{2}

\definecolor{correct}{HTML}{009900}
\newcommand\correct[2]{\ensuremath{\:}{\color{red}{#1}}\ensuremath{\to }{\color{correct}{#2}}\ensuremath{\:}}
\newcommand\green[1]{{\color{correct}{#1}}}

% horizontal rule
\newcommand\hr{
    \noindent\rule[0.5ex]{\linewidth}{0.5pt}
}

% hide parts
\newcommand\hide[1]{}

% si unitx
\usepackage{siunitx}
\sisetup{locale = FR}

% Environments
\makeatother
% For box around Definition, Theorem, \ldots
\usepackage[framemethod=TikZ]{mdframed}
\mdfsetup{skipabove=1em,skipbelow=0em}

%definition
\newenvironment{defn}[1][]{%
\ifstrempty{#1}%
{\mdfsetup{%
frametitle={%
\tikz[baseline=(current bounding box.east),outer sep=0pt]
\node[anchor=east,rectangle,fill=Emerald]
{\strut Definition};}}
}%
{\mdfsetup{%
frametitle={%
\tikz[baseline=(current bounding box.east),outer sep=0pt]
\node[anchor=east,rectangle,fill=Emerald]
{\strut Definition:~#1};}}%
}%
\mdfsetup{innertopmargin=10pt,linecolor=Emerald,%
linewidth=2pt,topline=true,%
frametitleaboveskip=\dimexpr-\ht\strutbox\relax
}
\begin{mdframed}[]\relax%
\label{#1}}{\end{mdframed}}


%theorem
%\newcounter{thm}[section]\setcounter{thm}{0}
%\renewcommand{\thethm}{\arabic{section}.\arabic{thm}}
\newenvironment{thm}[1][]{%
%\refstepcounter{thm}%
\ifstrempty{#1}%
{\mdfsetup{%
frametitle={%
\tikz[baseline=(current bounding box.east),outer sep=0pt]
\node[anchor=east,rectangle,fill=blue!20]
%{\strut Theorem~\thethm};}}
{\strut Theorem};}}
}%
{\mdfsetup{%
frametitle={%
\tikz[baseline=(current bounding box.east),outer sep=0pt]
\node[anchor=east,rectangle,fill=blue!20]
%{\strut Theorem~\thethm:~#1};}}%
{\strut Theorem:~#1};}}%
}%
\mdfsetup{innertopmargin=10pt,linecolor=blue!20,%
linewidth=2pt,topline=true,%
frametitleaboveskip=\dimexpr-\ht\strutbox\relax
}
\begin{mdframed}[]\relax%
\label{#1}}{\end{mdframed}}


%lemma
\newenvironment{lem}[1][]{%
\ifstrempty{#1}%
{\mdfsetup{%
frametitle={%
\tikz[baseline=(current bounding box.east),outer sep=0pt]
\node[anchor=east,rectangle,fill=Dandelion]
{\strut Lemma};}}
}%
{\mdfsetup{%
frametitle={%
\tikz[baseline=(current bounding box.east),outer sep=0pt]
\node[anchor=east,rectangle,fill=Dandelion]
{\strut Lemma:~#1};}}%
}%
\mdfsetup{innertopmargin=10pt,linecolor=Dandelion,%
linewidth=2pt,topline=true,%
frametitleaboveskip=\dimexpr-\ht\strutbox\relax
}
\begin{mdframed}[]\relax%
\label{#1}}{\end{mdframed}}

%corollary
\newenvironment{coro}[1][]{%
\ifstrempty{#1}%
{\mdfsetup{%
frametitle={%
\tikz[baseline=(current bounding box.east),outer sep=0pt]
\node[anchor=east,rectangle,fill=CornflowerBlue]
{\strut Corollary};}}
}%
{\mdfsetup{%
frametitle={%
\tikz[baseline=(current bounding box.east),outer sep=0pt]
\node[anchor=east,rectangle,fill=CornflowerBlue]
{\strut Corollary:~#1};}}%
}%
\mdfsetup{innertopmargin=10pt,linecolor=CornflowerBlue,%
linewidth=2pt,topline=true,%
frametitleaboveskip=\dimexpr-\ht\strutbox\relax
}
\begin{mdframed}[]\relax%
\label{#1}}{\end{mdframed}}

%proof
\newenvironment{prf}[1][]{%
\ifstrempty{#1}%
{\mdfsetup{%
frametitle={%
\tikz[baseline=(current bounding box.east),outer sep=0pt]
\node[anchor=east,rectangle,fill=SpringGreen]
{\strut Proof};}}
}%
{\mdfsetup{%
frametitle={%
\tikz[baseline=(current bounding box.east),outer sep=0pt]
\node[anchor=east,rectangle,fill=SpringGreen]
{\strut Proof:~#1};}}%
}%
\mdfsetup{innertopmargin=10pt,linecolor=SpringGreen,%
linewidth=2pt,topline=true,%
frametitleaboveskip=\dimexpr-\ht\strutbox\relax
}
\begin{mdframed}[]\relax%
\label{#1}}{\qed\end{mdframed}}


\theoremstyle{definition}

\newmdtheoremenv[nobreak=true]{definition}{Definition}
\newmdtheoremenv[nobreak=true]{prop}{Proposition}
\newmdtheoremenv[nobreak=true]{theorem}{Theorem}
\newmdtheoremenv[nobreak=true]{corollary}{Corollary}
\newtheorem*{eg}{Example}
\theoremstyle{remark}
\newtheorem*{case}{Case}
\newtheorem*{notation}{Notation}
\newtheorem*{remark}{Remark}
\newtheorem*{note}{Note}
\newtheorem*{problem}{Problem}
\newtheorem*{observe}{Observe}
\newtheorem*{property}{Property}
\newtheorem*{intuition}{Intuition}


% End example and intermezzo environments with a small diamond (just like proof
% environments end with a small square)
\usepackage{etoolbox}
\AtEndEnvironment{vb}{\null\hfill$\diamond$}%
\AtEndEnvironment{intermezzo}{\null\hfill$\diamond$}%
% \AtEndEnvironment{opmerking}{\null\hfill$\diamond$}%

% Fix some spacing
% http://tex.stackexchange.com/questions/22119/how-can-i-change-the-spacing-before-theorems-with-amsthm
\makeatletter
\def\thm@space@setup{%
  \thm@preskip=\parskip \thm@postskip=0pt
}

% Fix some stuff
% %http://tex.stackexchange.com/questions/76273/multiple-pdfs-with-page-group-included-in-a-single-page-warning
\pdfsuppresswarningpagegroup=1


% My name
\author{Jaden Wang}



\begin{document}
\begin{eg}[]
	Let $ L=10$cm,  $ k=1$ cm$ ^2$/sec (copper) and $ t=0.35$sec and use the first 8 terms.
	 \[
		 u(x,0.35) \approx \frac{400}{\pi} \sum_{ p= 1}^{ 8} \frac{1}{2p-1} e^{-[(2p-1) \frac{\pi}{10}]^2 0.35} \sin \left( \frac{(2p-1) \pi x}{10 } \right) 
	.\] 
	See lecture slide for graph.

	In this case, $ \overline{u}(x)=0$ and for each $ x$,
	 \[
		 u(x,t) \to 0 \text{ as } t \to \infty 
	.\] 
\begin{note}[]
	Recall $ \Phi = -K_0 \frac{\partial u}{\partial x} $. Consider the $ x$-term in  $ u_n(x,t)$
\begin{align*}
	\left[ \sin \left( \frac{ n\pi x}{ L} \right)  \right]' = n \cdot \frac{\pi}{L} \cos \left( \frac{ n\pi x}{ L} \right)  
\end{align*}
Thus the derivative wrt $ x$ is proportional to $ n$ and  $ \Phi$. That is, as $n $ increases, the derivative increases, and the heat flux (loss) increases. So the slowest decaying term is when $ n$ is smallest,  \emph{i.e.} $ n=1$. 
\end{note}

Establishing that the slowest decaying term (dominant term) is at $ n=1$ and for large  $ t$, we can use the  $ n=1$ term (called the "first Fourier mode") as an approximation
	 \[
		 u(x,t) \approx B_1 \sin \left( \frac{ \pi x}{ L} \right) e^{-( \frac{ \pi}{L} )^2 kt} 
	.\] 
	and we can use this for long term temperature prediction.
So for this problem we use the approximation:
\[
	u(x,t) \approx \frac{400}{\pi} \sin \left( \frac{ \pi x}{ L} \right) e^{-( \frac{ \pi}{L} )^2 kt} 
.\] 
to analyze the dynamics of the temperature when $ t $ grows. We expect $ u(x,t) \to 0 $ as $ t \to \infty$. See lecture slides for graph. 
\end{eg}

\begin{eg}[estimating cooling time]
How long will it take for the maximum absolute temperature of the rod to be less than $ \frac{1}{10}$ the initial maximum absolute temperature?

We wish to find $ t$  such that 
\[
	\max_{0<x<L} |u(x,t)| \leq \frac{1}{10} \max_{0<x<L} |f(x)|
.\]

Using the first Fourier mode approximation obtained above, recall there is an upper bound (in fact it's the least upper bound/supremum)
\[
	|u(x,t)| \leq \frac{400}{\pi} e^{-(\frac{\pi}{L})^2 kt}
.\] 
Hence this upper bound is greater or equal to the maximum (in this case they are in fact equal). Thus it suffices to find $ t$  such that 
\[
	\frac{400}{\pi} e^{-(\frac{\pi}{L})^2 kt} \leq \frac{1}{10} \cdot 100 = 10
.\] 
Solving this inequality yields
\[
	t \geq \frac{L^2}{k}\frac{1}{\pi^2} \ln \left( \frac{40}{\pi} \right) 
.\] 
\end{eg}


\newpage
\section{Insulated Rods}

Consider the following BVP:
\begin{equation*}
\begin{cases}
	\text{ PDE: }& \frac{\partial u}{\partial t} = k \frac{\partial^2 u}{\partial { x}^2},\qquad  0<x<L,t>0 \\
	\text{ BC: } & \frac{\partial u}{\partial x}(0,t)=0= \frac{\partial u}{\partial x} (L,t), \qquad  t>0\\
	\text{ IC: } & u(x,0) = f(x), \qquad  0 \leq x \leq L\\ 
\end{cases}
\end{equation*}

Recall Fourier's Law of Heat Conduction regarding the heat flux
\[
\Phi=-K_0 \frac{\partial u}{\partial x} 
.\] 
So here the BCs imply that there is no heat flow at the ends of the rod, \emph{i.e.} the rod is insulated on all sides.

Since there is no external source of heat, we expect this BVP to have a steady state solution.

Hence we can assume the solution has the form
\[
	u(x,t) = \overline{u}(x) + v(x,t)
.\] 

In fact, we can skip this decomposition step because the BVP already gives us a nice vector space due to the homogeneous BCs. And $ \lambda =0$ case will give us the steady-state solution anyway.


\end{document}

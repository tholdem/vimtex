\documentclass[class=article,crop=false]{standalone} 
%Fall 2020
% Some basic packages
\usepackage{standalone}[subpreambles=true]
\usepackage[utf8]{inputenc}
\usepackage[T1]{fontenc}
\usepackage{textcomp}
\usepackage[english]{babel}
\usepackage{url}
\usepackage{graphicx}
\usepackage{float}
\usepackage{enumitem}


\pdfminorversion=7

% Don't indent paragraphs, leave some space between them
\usepackage{parskip}

% Hide page number when page is empty
\usepackage{emptypage}
\usepackage{subcaption}
\usepackage{multicol}
\usepackage[dvipsnames]{xcolor}


% Math stuff
\usepackage{amsmath, amsfonts, mathtools, amsthm, amssymb}
% Fancy script capitals
\usepackage{mathrsfs}
\usepackage{cancel}
% Bold math
\usepackage{bm}
% Some shortcuts
\newcommand{\rr}{\ensuremath{\mathbb{R}}}
\newcommand{\zz}{\ensuremath{\mathbb{Z}}}
\newcommand{\qq}{\ensuremath{\mathbb{Q}}}
\newcommand{\nn}{\ensuremath{\mathbb{N}}}
\newcommand{\ff}{\ensuremath{\mathbb{F}}}
\newcommand{\cc}{\ensuremath{\mathbb{C}}}
\renewcommand\O{\ensuremath{\emptyset}}
\newcommand{\norm}[1]{{\left\lVert{#1}\right\rVert}}
\renewcommand{\vec}[1]{{\mathbf{#1}}}
\newcommand\allbold[1]{{\boldmath\textbf{#1}}}

% Put x \to \infty below \lim
\let\svlim\lim\def\lim{\svlim\limits}

%Make implies and impliedby shorter
\let\implies\Rightarrow
\let\impliedby\Leftarrow
\let\iff\Leftrightarrow
\let\epsilon\varepsilon

% Add \contra symbol to denote contradiction
\usepackage{stmaryrd} % for \lightning
\newcommand\contra{\scalebox{1.5}{$\lightning$}}

% \let\phi\varphi

% Command for short corrections
% Usage: 1+1=\correct{3}{2}

\definecolor{correct}{HTML}{009900}
\newcommand\correct[2]{\ensuremath{\:}{\color{red}{#1}}\ensuremath{\to }{\color{correct}{#2}}\ensuremath{\:}}
\newcommand\green[1]{{\color{correct}{#1}}}

% horizontal rule
\newcommand\hr{
    \noindent\rule[0.5ex]{\linewidth}{0.5pt}
}

% hide parts
\newcommand\hide[1]{}

% si unitx
\usepackage{siunitx}
\sisetup{locale = FR}

% Environments
\makeatother
% For box around Definition, Theorem, \ldots
\usepackage[framemethod=TikZ]{mdframed}
\mdfsetup{skipabove=1em,skipbelow=0em}

%definition
\newenvironment{defn}[1][]{%
\ifstrempty{#1}%
{\mdfsetup{%
frametitle={%
\tikz[baseline=(current bounding box.east),outer sep=0pt]
\node[anchor=east,rectangle,fill=Emerald]
{\strut Definition};}}
}%
{\mdfsetup{%
frametitle={%
\tikz[baseline=(current bounding box.east),outer sep=0pt]
\node[anchor=east,rectangle,fill=Emerald]
{\strut Definition:~#1};}}%
}%
\mdfsetup{innertopmargin=10pt,linecolor=Emerald,%
linewidth=2pt,topline=true,%
frametitleaboveskip=\dimexpr-\ht\strutbox\relax
}
\begin{mdframed}[]\relax%
\label{#1}}{\end{mdframed}}


%theorem
%\newcounter{thm}[section]\setcounter{thm}{0}
%\renewcommand{\thethm}{\arabic{section}.\arabic{thm}}
\newenvironment{thm}[1][]{%
%\refstepcounter{thm}%
\ifstrempty{#1}%
{\mdfsetup{%
frametitle={%
\tikz[baseline=(current bounding box.east),outer sep=0pt]
\node[anchor=east,rectangle,fill=blue!20]
%{\strut Theorem~\thethm};}}
{\strut Theorem};}}
}%
{\mdfsetup{%
frametitle={%
\tikz[baseline=(current bounding box.east),outer sep=0pt]
\node[anchor=east,rectangle,fill=blue!20]
%{\strut Theorem~\thethm:~#1};}}%
{\strut Theorem:~#1};}}%
}%
\mdfsetup{innertopmargin=10pt,linecolor=blue!20,%
linewidth=2pt,topline=true,%
frametitleaboveskip=\dimexpr-\ht\strutbox\relax
}
\begin{mdframed}[]\relax%
\label{#1}}{\end{mdframed}}


%lemma
\newenvironment{lem}[1][]{%
\ifstrempty{#1}%
{\mdfsetup{%
frametitle={%
\tikz[baseline=(current bounding box.east),outer sep=0pt]
\node[anchor=east,rectangle,fill=Dandelion]
{\strut Lemma};}}
}%
{\mdfsetup{%
frametitle={%
\tikz[baseline=(current bounding box.east),outer sep=0pt]
\node[anchor=east,rectangle,fill=Dandelion]
{\strut Lemma:~#1};}}%
}%
\mdfsetup{innertopmargin=10pt,linecolor=Dandelion,%
linewidth=2pt,topline=true,%
frametitleaboveskip=\dimexpr-\ht\strutbox\relax
}
\begin{mdframed}[]\relax%
\label{#1}}{\end{mdframed}}

%corollary
\newenvironment{coro}[1][]{%
\ifstrempty{#1}%
{\mdfsetup{%
frametitle={%
\tikz[baseline=(current bounding box.east),outer sep=0pt]
\node[anchor=east,rectangle,fill=CornflowerBlue]
{\strut Corollary};}}
}%
{\mdfsetup{%
frametitle={%
\tikz[baseline=(current bounding box.east),outer sep=0pt]
\node[anchor=east,rectangle,fill=CornflowerBlue]
{\strut Corollary:~#1};}}%
}%
\mdfsetup{innertopmargin=10pt,linecolor=CornflowerBlue,%
linewidth=2pt,topline=true,%
frametitleaboveskip=\dimexpr-\ht\strutbox\relax
}
\begin{mdframed}[]\relax%
\label{#1}}{\end{mdframed}}

%proof
\newenvironment{prf}[1][]{%
\ifstrempty{#1}%
{\mdfsetup{%
frametitle={%
\tikz[baseline=(current bounding box.east),outer sep=0pt]
\node[anchor=east,rectangle,fill=SpringGreen]
{\strut Proof};}}
}%
{\mdfsetup{%
frametitle={%
\tikz[baseline=(current bounding box.east),outer sep=0pt]
\node[anchor=east,rectangle,fill=SpringGreen]
{\strut Proof:~#1};}}%
}%
\mdfsetup{innertopmargin=10pt,linecolor=SpringGreen,%
linewidth=2pt,topline=true,%
frametitleaboveskip=\dimexpr-\ht\strutbox\relax
}
\begin{mdframed}[]\relax%
\label{#1}}{\qed\end{mdframed}}


\theoremstyle{definition}

\newmdtheoremenv[nobreak=true]{definition}{Definition}
\newmdtheoremenv[nobreak=true]{prop}{Proposition}
\newmdtheoremenv[nobreak=true]{theorem}{Theorem}
\newmdtheoremenv[nobreak=true]{corollary}{Corollary}
\newtheorem*{eg}{Example}
\theoremstyle{remark}
\newtheorem*{case}{Case}
\newtheorem*{notation}{Notation}
\newtheorem*{remark}{Remark}
\newtheorem*{note}{Note}
\newtheorem*{problem}{Problem}
\newtheorem*{observe}{Observe}
\newtheorem*{property}{Property}
\newtheorem*{intuition}{Intuition}


% End example and intermezzo environments with a small diamond (just like proof
% environments end with a small square)
\usepackage{etoolbox}
\AtEndEnvironment{vb}{\null\hfill$\diamond$}%
\AtEndEnvironment{intermezzo}{\null\hfill$\diamond$}%
% \AtEndEnvironment{opmerking}{\null\hfill$\diamond$}%

% Fix some spacing
% http://tex.stackexchange.com/questions/22119/how-can-i-change-the-spacing-before-theorems-with-amsthm
\makeatletter
\def\thm@space@setup{%
  \thm@preskip=\parskip \thm@postskip=0pt
}

% Fix some stuff
% %http://tex.stackexchange.com/questions/76273/multiple-pdfs-with-page-group-included-in-a-single-page-warning
\pdfsuppresswarningpagegroup=1


% My name
\author{Jaden Wang}



\begin{document}
\begin{defn}[Laplace Transform]
	Let $ f(t)$ be given for  $ t\geq 0$ and suppose that  $ f$ satisfies:
	 \begin{enumerate}[label=\arabic*)]
		\item $ f$ is piecewise continuous on the interval  $ 0 \leq t \leq A$ for any positive  $ A$.
		\item $|f(t)|\leq K e^{at}$ when $ t \geq M$. In this inequality  $ K>0,a,M>0 \in \rr$.
	\end{enumerate}
	Then define the \allbold{Laplace Transform}, $ \mathcal{ L} \{f(t)\} = F(s)$ as
	\[
		\mathcal{ L}\{f(t)\} = F(s) = \int_{0}^{\infty} e^{-st} f(t) dt, s>a
	.\] 
	This uses the kernel $ K(s,t) = e^{-st}$ and the parameter $ s$ may be complex but we assume  $ s \in \rr$.
\end{defn}

\begin{claim}[]
Laplace transform is a special case of Fourier transform.
\end{claim}

\begin{prf}
	Suppose $ f$ is such a function that  $ f(t) = 0$ for  $ t<0$ then
	 \[
		 2\pi \wh{ f}(m) = 2\pi \mathcal{ F}[f(t)] - \int_{-\infty}^{\infty} f(t) e^{-imt} dt = \int_{0}^{\infty} f(t) e^{-imt} dt 
	.\] 
	If we let $ m=-is, s \in \rr$, then we have
	\begin{align*}
		2\pi \wh{ f}(m) &= \int_{0}^{\infty} f(t) e^{-i(-is)t} dt = \int_{0}^{\infty} f(t) e^{-st} dt = F(s) \\
		\implies		2\pi \mathcal{ F}[f(t)] &= \mathcal{ L}\{f(t)\}   
	\end{align*} 
\end{prf}

\newpage
\section{Dispersive Waves}

Recall the wave equation
\begin{equation*}
\begin{cases}
	\text{PDE: } \frac{\partial^2 u}{\partial { t}^2} = c^2 \frac{\partial^2 u}{\partial { x}^2}  & 0<x<L, t>0 \\
	\text{BCs: } u(0,t)=0=u(L,t), & t>0 \\
	\text{ICs: } u(x,0) = U(x), \frac{\partial u}{\partial t} (x,0) = V(x).  & 0 \leq x \leq L \\
\end{cases}
\end{equation*}
has the solution
\[
	u(x,t) = \sum_{ n= 1}^{\infty} \sin \left( \frac{ n\pi x}{ L} \right) \left[ A_n \cos \left( \frac{ n\pi c t}{ L} \right) + B_n \sin \left( \frac{ n\pi c t}{ L} \right)  \right] 
.\] 
So it's a product of a wave in $ x$ and a wave in  $ t$. Therefore, suppose the solution to  \emph{wave propagation} problems have the form
\[
	u(x,t) = F(x) G(t) = e^{ikx} \cdot e^{-i\omega t}= e^{i(kx- \omega t)}, k, \omega \in \rr
.\] 

\begin{notation}[]
$ k, \omega$ are newly defined, not related to previous materials.
\end{notation}

\begin{defn}[phase velocity]
~\begin{enumerate}[label=\arabic*)]
	\item The wavelength of the space wave $ e^{ikx} $ is $ L = \frac{2\pi}{k }$ so we see that $ k$ represents the number of wavelengths per  $ 2\pi $ unit of distance (spatial frequency). For the time wave $ e^{-i\omega t}$, the term $ \omega$ is the number of waves per $ 2\pi$ unit time (time frequency).
	\item the \allbold{phase velocity}, $ c_p$, is defined as
		 \[
		c_p = \frac{\omega}{k } \text{ with units distance/time} 
		.\] 
	\item The wave $ e^{ik(x-\frac{\omega}{k }t)}$ represents a "traveling wave" with wave number $ k$ and wave velocity  $ c_p = \frac{\omega}{k}$.
	\item If $ \omega$ is a function of $ k \in \rr$, \emph{i.e.} if $ \omega = \omega(k)$ then $ e^{-i(kx-\omega t)}$ is a \allbold{linear dispersive wave}.
	\item For linear dispersive waves, we have $ c_p = c_p(k)$ and if $ \frac{d}{dk} c_p \neq 0$, \emph{i.e.} if the wave velocity is not constant, then we have a wave propagation problem that is said to be \allbold{dispersive}. 
	\item The term $ \omega(k)$ relates space and time in the expression $ e^{ik(x-\omega \frac{t}{k} }$.
	\item The information about the PDE (but not the BCs or ICs) is encoded in $ \omega(k)$.
\end{enumerate}
\end{defn}

\begin{eg}[]
Consider wave equation with stiffness term and ICs:
\[
	\frac{\partial^2 u}{\partial { t}^2} =c^2 \frac{\partial^2 u}{\partial { x}^2} - \alpha^2 \frac{\partial^4 u}{\partial { x}^4} , u(x,0) = U(x), \frac{\partial }{\partial t} u(x,0) = V(x)
.\] 

Assume $ U(x,t) = e^{ikx-i\omega t}, k \in \rr$ and it yields
\[
	\frac{\partial^2 u}{\partial { t}^2} = -\omega^2 u, c^2 \frac{\partial^2 u}{\partial { x}^2} = c^2(ik)^2 u, \alpha^2 \frac{\partial^4 u}{\partial { x}^4} = \alpha^2 k^{4} u
.\] 
Substituting into the PDE and we obtain:
\[
	w(k) = \pm k \sqrt{c^2 + \alpha^2 k^2} 
.\] 
Thus,
\[
	u_{1,k} = e^{ik(x-\sqrt{^2+\alpha^2 k^2 t}) }, u_{2,k}(x,t) = e^{ik(x+ \sqrt{c^2+ \alpha^2 k^{2}} )}
.\] 
By superposition principle, summing for all $ k \in \rr$ yields
\[
	u(x,t) = \int_{-\infty}^{\infty}  \left[ A(k) e^{ik (x-\sqrt{c^2+ \alpha^2 k^2}) } + B(k) e^{ik(x+ \sqrt{c^2+ \alpha^2 k^2} )} \right] dk
.\] 
Now we can use the ICs to find $ A(k),B(k)$.

\[
	U(x) = u(x,0) = \int_{-\infty}^{\infty} [A(k)+B(k)] e^{ikx} dk = \mathcal{ F}^{-1} [A(k)+B(k)]
.\] 
Thus, taking Fourier transform on both sides gives us
\[
	A(k)+B(k) = \mathcal{ F}[U(x)] = \wh{ U}(k) = \frac{1}{2\pi} \int_{-\infty}^{\infty} U(x) e^{ikx} dx 
.\] 
And similarly,
\[
	-A(k) + B(k) = \frac{1}{2\pi} \int_{-\infty}^{\infty} \frac{V(x)}{ik \sqrt{c^2+ \alpha^2 k^2}  } e^{ikx} dx 
.\] 
These two expressions allow us to solve
\[
	A(k) = \frac{1}{4\pi} \int_{-\infty}^{\infty} \left[ U(x) - \frac{V(x)}{ik \sqrt{c^2 + \alpha^2 k^2}  } \right]  e^{ikx} dx
.\] 
and
\[
	B(k) = \frac{1}{4\pi} \int_{-\infty}^{\infty} \left[ U(x) + \frac{V(x)}{ik \sqrt{c^2 + \alpha^2 k^2}  } \right]  e^{ikx} dx
.\] 

\end{eg}

\subsection{Group Velocity}

We wish to analyze the physical interaction of waves with proximal spatial frequency. Assume two close waves of the form $ \cos[kx-\omega(k) t]$ with wave numbers $ k$ and  $ k+ \Delta k$. That is,
\[
	u(x,t) = A \cos[kx - \omega(k) t] + A \cos[(k+\Delta k)x-\omega(k+ \Delta k )t]
.\] 
Using trig identity and $ \cos$ is even,
\[
	u(x,t) = 2A \cdot \cos\left[ \left( k+ \frac{\Delta k}{2 } \right) x - \frac{\omega(k)+\omega (k+ \Delta k)}{2 } t\right] \cdot  \cos\left[ \frac{\Delta k}{2 } x - \frac{\omega(k+\Delta k)- \omega(k)}{2 } t\right] 
.\] 
The wavelength of the first term is $ 2\pi / (k+\Delta k /2)$, which is shorter than that of the second term, $ 2\pi / \Delta k / 2$. 

The long wave acts as a wave envelop of the short waves and it travels with what is called as \allbold{group velocity} given by
\[
	c_g = \lim_{ \Delta k \to 0} \frac{\omega(k+ \Delta k) - \omega(k)}{\Delta k }= \frac{d}{dk} \omega(k)
.\] 
The short wave moves at almost the speed $ c_p = \frac{\omega(k)}{k }$ which is the phase velocity of an individual dispersive wave. The long wave acts as a wave envelope of the short waves and travels at group velocity. We claim that the  \allbold{wave energy} moves with group velocity $ c_g$.

\subsection{Deep Water Waves}

Gravity and surface tension affects the propagation of water waves. For surface water the equation is
\[
	\omega(k) = \sqrt{gk \tanh(kh)} 
\] 
where $ g=9.8$ m/s and  $ h>0$ is the depth of the water. In deep water where the depth of water $ h$ is large, notice $ \lim_{ h \to \infty} \tanh(kh) =1 \implies \omega(k) = \sqrt{gk \tanh(kh)} \approx \sqrt{gk}  $ thus the phase velocity of a short wave in deep water is approximately
\[
	c_p = \frac{\omega(k)}{k } = \frac{\sqrt{gk \tanh(kh)} }{k } = \sqrt{\frac{g \tanh(kh)}{k }}  \approx \sqrt{\frac{g}{k} }
.\]
Since $ L=\frac{2\pi}{k }$, we finally have
\[
c_p \approx \sqrt{\frac{gL}{2\pi }} 
.\] 
Thus, $ c_p=c_p(k)$ and short waves in deep water are dispersive.

For long wave
\[
	c_g = w'(k) = \frac{d}{dk} \sqrt{gk} = \frac{1}{2} \sqrt{\frac{g}{k}} = \frac{1}{2} \sqrt{\frac{gL}{ 2\pi}}  \implies c_g = \frac{1}{2} c_p 
.\] 
This shows that longer waves will have larger group velocities and arrive at a distance shoreline sooner.

For shallow water, short waves are non-dispersive and the phase velocity mainly depends on gravity and depth of water since
\[
	c_p = \frac{\omega(k)}{k } = \sqrt{\frac{g \tanh(kh)}{k }} = \sqrt{gh \cdot \frac{\tanh(kh)}{kh }} \approx \sqrt{gh}   
.\]
where we use the fact that $ \tanh(kh) / kh \approx 1$ for small $ h$ (by L'Hopital Rule).
\end{document}

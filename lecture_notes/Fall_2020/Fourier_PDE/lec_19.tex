\documentclass[class=article,crop=false]{standalone} 
%Fall 2020
% Some basic packages
\usepackage{standalone}[subpreambles=true]
\usepackage[utf8]{inputenc}
\usepackage[T1]{fontenc}
\usepackage{textcomp}
\usepackage[english]{babel}
\usepackage{url}
\usepackage{graphicx}
\usepackage{float}
\usepackage{enumitem}


\pdfminorversion=7

% Don't indent paragraphs, leave some space between them
\usepackage{parskip}

% Hide page number when page is empty
\usepackage{emptypage}
\usepackage{subcaption}
\usepackage{multicol}
\usepackage[dvipsnames]{xcolor}


% Math stuff
\usepackage{amsmath, amsfonts, mathtools, amsthm, amssymb}
% Fancy script capitals
\usepackage{mathrsfs}
\usepackage{cancel}
% Bold math
\usepackage{bm}
% Some shortcuts
\newcommand{\rr}{\ensuremath{\mathbb{R}}}
\newcommand{\zz}{\ensuremath{\mathbb{Z}}}
\newcommand{\qq}{\ensuremath{\mathbb{Q}}}
\newcommand{\nn}{\ensuremath{\mathbb{N}}}
\newcommand{\ff}{\ensuremath{\mathbb{F}}}
\newcommand{\cc}{\ensuremath{\mathbb{C}}}
\renewcommand\O{\ensuremath{\emptyset}}
\newcommand{\norm}[1]{{\left\lVert{#1}\right\rVert}}
\renewcommand{\vec}[1]{{\mathbf{#1}}}
\newcommand\allbold[1]{{\boldmath\textbf{#1}}}

% Put x \to \infty below \lim
\let\svlim\lim\def\lim{\svlim\limits}

%Make implies and impliedby shorter
\let\implies\Rightarrow
\let\impliedby\Leftarrow
\let\iff\Leftrightarrow
\let\epsilon\varepsilon

% Add \contra symbol to denote contradiction
\usepackage{stmaryrd} % for \lightning
\newcommand\contra{\scalebox{1.5}{$\lightning$}}

% \let\phi\varphi

% Command for short corrections
% Usage: 1+1=\correct{3}{2}

\definecolor{correct}{HTML}{009900}
\newcommand\correct[2]{\ensuremath{\:}{\color{red}{#1}}\ensuremath{\to }{\color{correct}{#2}}\ensuremath{\:}}
\newcommand\green[1]{{\color{correct}{#1}}}

% horizontal rule
\newcommand\hr{
    \noindent\rule[0.5ex]{\linewidth}{0.5pt}
}

% hide parts
\newcommand\hide[1]{}

% si unitx
\usepackage{siunitx}
\sisetup{locale = FR}

% Environments
\makeatother
% For box around Definition, Theorem, \ldots
\usepackage[framemethod=TikZ]{mdframed}
\mdfsetup{skipabove=1em,skipbelow=0em}

%definition
\newenvironment{defn}[1][]{%
\ifstrempty{#1}%
{\mdfsetup{%
frametitle={%
\tikz[baseline=(current bounding box.east),outer sep=0pt]
\node[anchor=east,rectangle,fill=Emerald]
{\strut Definition};}}
}%
{\mdfsetup{%
frametitle={%
\tikz[baseline=(current bounding box.east),outer sep=0pt]
\node[anchor=east,rectangle,fill=Emerald]
{\strut Definition:~#1};}}%
}%
\mdfsetup{innertopmargin=10pt,linecolor=Emerald,%
linewidth=2pt,topline=true,%
frametitleaboveskip=\dimexpr-\ht\strutbox\relax
}
\begin{mdframed}[]\relax%
\label{#1}}{\end{mdframed}}


%theorem
%\newcounter{thm}[section]\setcounter{thm}{0}
%\renewcommand{\thethm}{\arabic{section}.\arabic{thm}}
\newenvironment{thm}[1][]{%
%\refstepcounter{thm}%
\ifstrempty{#1}%
{\mdfsetup{%
frametitle={%
\tikz[baseline=(current bounding box.east),outer sep=0pt]
\node[anchor=east,rectangle,fill=blue!20]
%{\strut Theorem~\thethm};}}
{\strut Theorem};}}
}%
{\mdfsetup{%
frametitle={%
\tikz[baseline=(current bounding box.east),outer sep=0pt]
\node[anchor=east,rectangle,fill=blue!20]
%{\strut Theorem~\thethm:~#1};}}%
{\strut Theorem:~#1};}}%
}%
\mdfsetup{innertopmargin=10pt,linecolor=blue!20,%
linewidth=2pt,topline=true,%
frametitleaboveskip=\dimexpr-\ht\strutbox\relax
}
\begin{mdframed}[]\relax%
\label{#1}}{\end{mdframed}}


%lemma
\newenvironment{lem}[1][]{%
\ifstrempty{#1}%
{\mdfsetup{%
frametitle={%
\tikz[baseline=(current bounding box.east),outer sep=0pt]
\node[anchor=east,rectangle,fill=Dandelion]
{\strut Lemma};}}
}%
{\mdfsetup{%
frametitle={%
\tikz[baseline=(current bounding box.east),outer sep=0pt]
\node[anchor=east,rectangle,fill=Dandelion]
{\strut Lemma:~#1};}}%
}%
\mdfsetup{innertopmargin=10pt,linecolor=Dandelion,%
linewidth=2pt,topline=true,%
frametitleaboveskip=\dimexpr-\ht\strutbox\relax
}
\begin{mdframed}[]\relax%
\label{#1}}{\end{mdframed}}

%corollary
\newenvironment{coro}[1][]{%
\ifstrempty{#1}%
{\mdfsetup{%
frametitle={%
\tikz[baseline=(current bounding box.east),outer sep=0pt]
\node[anchor=east,rectangle,fill=CornflowerBlue]
{\strut Corollary};}}
}%
{\mdfsetup{%
frametitle={%
\tikz[baseline=(current bounding box.east),outer sep=0pt]
\node[anchor=east,rectangle,fill=CornflowerBlue]
{\strut Corollary:~#1};}}%
}%
\mdfsetup{innertopmargin=10pt,linecolor=CornflowerBlue,%
linewidth=2pt,topline=true,%
frametitleaboveskip=\dimexpr-\ht\strutbox\relax
}
\begin{mdframed}[]\relax%
\label{#1}}{\end{mdframed}}

%proof
\newenvironment{prf}[1][]{%
\ifstrempty{#1}%
{\mdfsetup{%
frametitle={%
\tikz[baseline=(current bounding box.east),outer sep=0pt]
\node[anchor=east,rectangle,fill=SpringGreen]
{\strut Proof};}}
}%
{\mdfsetup{%
frametitle={%
\tikz[baseline=(current bounding box.east),outer sep=0pt]
\node[anchor=east,rectangle,fill=SpringGreen]
{\strut Proof:~#1};}}%
}%
\mdfsetup{innertopmargin=10pt,linecolor=SpringGreen,%
linewidth=2pt,topline=true,%
frametitleaboveskip=\dimexpr-\ht\strutbox\relax
}
\begin{mdframed}[]\relax%
\label{#1}}{\qed\end{mdframed}}


\theoremstyle{definition}

\newmdtheoremenv[nobreak=true]{definition}{Definition}
\newmdtheoremenv[nobreak=true]{prop}{Proposition}
\newmdtheoremenv[nobreak=true]{theorem}{Theorem}
\newmdtheoremenv[nobreak=true]{corollary}{Corollary}
\newtheorem*{eg}{Example}
\theoremstyle{remark}
\newtheorem*{case}{Case}
\newtheorem*{notation}{Notation}
\newtheorem*{remark}{Remark}
\newtheorem*{note}{Note}
\newtheorem*{problem}{Problem}
\newtheorem*{observe}{Observe}
\newtheorem*{property}{Property}
\newtheorem*{intuition}{Intuition}


% End example and intermezzo environments with a small diamond (just like proof
% environments end with a small square)
\usepackage{etoolbox}
\AtEndEnvironment{vb}{\null\hfill$\diamond$}%
\AtEndEnvironment{intermezzo}{\null\hfill$\diamond$}%
% \AtEndEnvironment{opmerking}{\null\hfill$\diamond$}%

% Fix some spacing
% http://tex.stackexchange.com/questions/22119/how-can-i-change-the-spacing-before-theorems-with-amsthm
\makeatletter
\def\thm@space@setup{%
  \thm@preskip=\parskip \thm@postskip=0pt
}

% Fix some stuff
% %http://tex.stackexchange.com/questions/76273/multiple-pdfs-with-page-group-included-in-a-single-page-warning
\pdfsuppresswarningpagegroup=1


% My name
\author{Jaden Wang}



\begin{document}
\newpage
\section{Thin Circular Ring}
The wire is circular with circumference $ 2L$ and insulated. The radius is therefore $ r=\frac{L}{\pi}$. If the wire is thin enough then we assume the temperature is constant along the cross sections of the wire and satisfies the following BVP:
\begin{equation*}
\begin{cases}
	\text{PDE: }& \frac{\partial u}{\partial t} = k \frac{\partial^2 u}{\partial { x}^2} , \qquad \qquad \qquad  \qquad \qquad  \qquad  -L<x<L, t>0\\
	\text{ BCs: }& u(-L,t)  = u(L,t), \frac{\partial u}{\partial x} (-L,t) = \frac{\partial u}{\partial x} (L,t), \qquad  t>0 \\
	\text{ IC: }& u(x,0)=f(x), \qquad \qquad \qquad \qquad \qquad \qquad   -L \leq x \leq L \\
\end{cases}
\end{equation*}
The BCs here assume that at the ends, the temperature is continuous and the flux is also continuous.

Due to the circular nature, $ u(x_0,t) = u(x_0+2L,t) \ \forall \ x_0 \in [-L,L]$. Then we can define $ u(x,t) \ \forall \ x \in \rr$.

Do the PDE and BCs form a vector space? See homework, where we check linearity. Yes, so we can try separable of variables $ u(x,t)=F(x) \cdot G(t) \neq 0$. We turn this into a time domain problem and an eigenvalue problem.

\begin{note}[]
As before we have
\[
	\frac{1}{k} \frac{G'(t)}{G(t) } = \frac{F''(x)}{F(x) } =-\lambda \implies G'(t) = -\lambda kG(t) \text{ and } F''(x) = - \lambda F(x) 
.\] 
Then BCs respectively becomes
\begin{align*}
	F(-L)=F(L) \text{ and } F'(-L)=F'(L) 
\end{align*}
\end{note}

\begin{enumerate}[label=\arabic*)]
	\item \emph{time domain problem: } $ G(t) = C e^{-\lambda kt}, C \in \rr$.
	\item \emph{eigenvalue problem: }
~\begin{case}[]
$ \lambda < 0$, again we get the trivial solution.
\end{case}
\begin{case}[]
$ \lambda =0$, then
\[
	F''(x) = 0 \implies F(x) = Ax + B 
.\] 
Then the BC $ F(-L) = F(L) \implies A=0$. Therefore, $ F(x) = B, B \in \rr$. The other BC is trivial and redundant in this case.
\end{case}
\begin{case}[]
$ \lambda>0$, we solve the characteristic equation as before and obtain
\[
	F(x), c_1 \cos(\sqrt{\lambda} x) + c_2 \sin(\sqrt{\lambda} x )
.\] 
Then the BC $ F(-L) = F(L)$ yields
 \[
c_1 \cos(-\sqrt{\lambda} L ) + c_2 \sin( -\sqrt{\lambda}L  ) = c_1\cos(\sqrt{\lambda}L  ) + c_2 \sin(\sqrt{\lambda} L ) \implies 2c_2 \sin(\sqrt{\lambda}L  )=0
.\] 
This implies either $ c_2 =0 $ or $ \sqrt{\lambda}L = n \pi $ for $ n = \pm 1, \pm 2,\ldots$ Applying the other BC $ F'(-L)=F'(L)$, as above we get
 \[
c_1 \sin(\sqrt{\lambda} L )=0
.\] 
This implies either $ c_1 =0$ or $ \sqrt{\lambda} L = n\pi $ for $ n=\pm 1, \pm 2,\ldots$ Recall that we do not want $ c_1=0$ and $ c_2 =0$, \emph{i.e.} the trivial solution, so we require either $ c_1 \neq 0$ or $ c_2 \neq 0$. This means that both the cosine and sine terms survive in the general solution.

Moreover, we get $ \lambda_n= \left( \frac{n\pi}{L } \right)^2 $ for $ n=1,2\ldots$ 

\end{case}
Hence by superposition principle, the general solution is
\[
	u(x,t) = a_0 + \sum_{ n= 1}^{\infty} a_n \cos \left( \frac{ n\pi x}{ L} \right) e^{-( \frac{ n\pi}{L} )^2 kt} + \sum_{ n= 1}^{\infty} b_n \sin \left( \frac{ n\pi x}{ L} \right) e^{-( \frac{ n\pi}{L} )^2 kt} 
.\] 
The coefficients $ a_0, a_n, b_n$ are obtained just as before using projection.
\begin{note}
Suppose $ f(x)$ is odd, then  $ a_0, a_n = 0$, and $ b_n = \frac{2}{L} \int_{0}^{L} f(x) \sin \left( \frac{ n\pi x}{ L} \right)  $, just like in FSS. Similar with even $ f(x)$. 
\end{note}

Now for large but finite time, we can again approximate our temperature prediction using the slowest decaying exponential term, which includes both sine and cosine terms when $ n=1$:
 \[
	 u(x,t) \approx \frac{1}{2L} \int_{-L}^{L} f(x) dx + \left[ a_1 \cos \left( \frac{ \pi x}{ L} \right) + b_1 \sin \left( \frac{ \pi x}{ L} \right)  \right] e^{-( \frac{ \pi}{L} )^2 kt} 
.\] 
And the steady-state solution ($ t \to \infty$) is just
\[
	\overline{u}(x) = a_0
.\] 
\end{enumerate}

How are the FSS, FCS, and FS related?
\begin{defn}[]
	Note that for any function $ f(x)$, we have
	 \[
		 f(x)=\frac{1}{2}[f(x) + f(-x)] + \frac{1}{2} [f(x)- f(-x)]
	.\] 
\begin{enumerate}[label=\arabic*)]
\item Define the \allbold{even part} of $ f(x)$ to be  $ f_e(x) = \frac{1}{2} [f(x)+f(-x)]$, then $ f_e(-x)=f_e(x)$.
\item Define the  \allbold{odd part} of $ f(x)$ to be  $ f_o(x) = \frac{1}{2} [f(x) - f(-x)]$, then $ f_o(-x) = - f_o(x)$.
\item The  $ \text{ F.S.} [ f]( x) $ equals the FCS of $ f_e(x)$ plus the FSS of  $ f_o(x)$. That is
\end{enumerate}
 \[
\text{ F.S.} [ f]( x) = a_0 + \sum_{ n= 1}^{\infty} a_n \cos \left( \frac{ n\pi x}{ L} \right) e^{-( \frac{ n\pi}{L} )^2 kt} + \sum_{ n= 1}^{\infty} b_n \sin \left( \frac{ n\pi x}{ L} \right) e^{-( \frac{ n\pi}{L} )^2 kt} 
.\] 
\end{defn}
\begin{note}[]
	The even and odd parts of $ f(x)$ is NOT the even and odd extension of  $ f(x)$!
\end{note}

This concludes our discussion of the heat equation, for now.
\end{document}

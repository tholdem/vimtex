\documentclass[class=article,crop=false]{standalone} 
%Fall 2020
% Some basic packages
\usepackage{standalone}[subpreambles=true]
\usepackage[utf8]{inputenc}
\usepackage[T1]{fontenc}
\usepackage{textcomp}
\usepackage[english]{babel}
\usepackage{url}
\usepackage{graphicx}
\usepackage{float}
\usepackage{enumitem}


\pdfminorversion=7

% Don't indent paragraphs, leave some space between them
\usepackage{parskip}

% Hide page number when page is empty
\usepackage{emptypage}
\usepackage{subcaption}
\usepackage{multicol}
\usepackage[dvipsnames]{xcolor}


% Math stuff
\usepackage{amsmath, amsfonts, mathtools, amsthm, amssymb}
% Fancy script capitals
\usepackage{mathrsfs}
\usepackage{cancel}
% Bold math
\usepackage{bm}
% Some shortcuts
\newcommand{\rr}{\ensuremath{\mathbb{R}}}
\newcommand{\zz}{\ensuremath{\mathbb{Z}}}
\newcommand{\qq}{\ensuremath{\mathbb{Q}}}
\newcommand{\nn}{\ensuremath{\mathbb{N}}}
\newcommand{\ff}{\ensuremath{\mathbb{F}}}
\newcommand{\cc}{\ensuremath{\mathbb{C}}}
\renewcommand\O{\ensuremath{\emptyset}}
\newcommand{\norm}[1]{{\left\lVert{#1}\right\rVert}}
\renewcommand{\vec}[1]{{\mathbf{#1}}}
\newcommand\allbold[1]{{\boldmath\textbf{#1}}}

% Put x \to \infty below \lim
\let\svlim\lim\def\lim{\svlim\limits}

%Make implies and impliedby shorter
\let\implies\Rightarrow
\let\impliedby\Leftarrow
\let\iff\Leftrightarrow
\let\epsilon\varepsilon

% Add \contra symbol to denote contradiction
\usepackage{stmaryrd} % for \lightning
\newcommand\contra{\scalebox{1.5}{$\lightning$}}

% \let\phi\varphi

% Command for short corrections
% Usage: 1+1=\correct{3}{2}

\definecolor{correct}{HTML}{009900}
\newcommand\correct[2]{\ensuremath{\:}{\color{red}{#1}}\ensuremath{\to }{\color{correct}{#2}}\ensuremath{\:}}
\newcommand\green[1]{{\color{correct}{#1}}}

% horizontal rule
\newcommand\hr{
    \noindent\rule[0.5ex]{\linewidth}{0.5pt}
}

% hide parts
\newcommand\hide[1]{}

% si unitx
\usepackage{siunitx}
\sisetup{locale = FR}

% Environments
\makeatother
% For box around Definition, Theorem, \ldots
\usepackage[framemethod=TikZ]{mdframed}
\mdfsetup{skipabove=1em,skipbelow=0em}

%definition
\newenvironment{defn}[1][]{%
\ifstrempty{#1}%
{\mdfsetup{%
frametitle={%
\tikz[baseline=(current bounding box.east),outer sep=0pt]
\node[anchor=east,rectangle,fill=Emerald]
{\strut Definition};}}
}%
{\mdfsetup{%
frametitle={%
\tikz[baseline=(current bounding box.east),outer sep=0pt]
\node[anchor=east,rectangle,fill=Emerald]
{\strut Definition:~#1};}}%
}%
\mdfsetup{innertopmargin=10pt,linecolor=Emerald,%
linewidth=2pt,topline=true,%
frametitleaboveskip=\dimexpr-\ht\strutbox\relax
}
\begin{mdframed}[]\relax%
\label{#1}}{\end{mdframed}}


%theorem
%\newcounter{thm}[section]\setcounter{thm}{0}
%\renewcommand{\thethm}{\arabic{section}.\arabic{thm}}
\newenvironment{thm}[1][]{%
%\refstepcounter{thm}%
\ifstrempty{#1}%
{\mdfsetup{%
frametitle={%
\tikz[baseline=(current bounding box.east),outer sep=0pt]
\node[anchor=east,rectangle,fill=blue!20]
%{\strut Theorem~\thethm};}}
{\strut Theorem};}}
}%
{\mdfsetup{%
frametitle={%
\tikz[baseline=(current bounding box.east),outer sep=0pt]
\node[anchor=east,rectangle,fill=blue!20]
%{\strut Theorem~\thethm:~#1};}}%
{\strut Theorem:~#1};}}%
}%
\mdfsetup{innertopmargin=10pt,linecolor=blue!20,%
linewidth=2pt,topline=true,%
frametitleaboveskip=\dimexpr-\ht\strutbox\relax
}
\begin{mdframed}[]\relax%
\label{#1}}{\end{mdframed}}


%lemma
\newenvironment{lem}[1][]{%
\ifstrempty{#1}%
{\mdfsetup{%
frametitle={%
\tikz[baseline=(current bounding box.east),outer sep=0pt]
\node[anchor=east,rectangle,fill=Dandelion]
{\strut Lemma};}}
}%
{\mdfsetup{%
frametitle={%
\tikz[baseline=(current bounding box.east),outer sep=0pt]
\node[anchor=east,rectangle,fill=Dandelion]
{\strut Lemma:~#1};}}%
}%
\mdfsetup{innertopmargin=10pt,linecolor=Dandelion,%
linewidth=2pt,topline=true,%
frametitleaboveskip=\dimexpr-\ht\strutbox\relax
}
\begin{mdframed}[]\relax%
\label{#1}}{\end{mdframed}}

%corollary
\newenvironment{coro}[1][]{%
\ifstrempty{#1}%
{\mdfsetup{%
frametitle={%
\tikz[baseline=(current bounding box.east),outer sep=0pt]
\node[anchor=east,rectangle,fill=CornflowerBlue]
{\strut Corollary};}}
}%
{\mdfsetup{%
frametitle={%
\tikz[baseline=(current bounding box.east),outer sep=0pt]
\node[anchor=east,rectangle,fill=CornflowerBlue]
{\strut Corollary:~#1};}}%
}%
\mdfsetup{innertopmargin=10pt,linecolor=CornflowerBlue,%
linewidth=2pt,topline=true,%
frametitleaboveskip=\dimexpr-\ht\strutbox\relax
}
\begin{mdframed}[]\relax%
\label{#1}}{\end{mdframed}}

%proof
\newenvironment{prf}[1][]{%
\ifstrempty{#1}%
{\mdfsetup{%
frametitle={%
\tikz[baseline=(current bounding box.east),outer sep=0pt]
\node[anchor=east,rectangle,fill=SpringGreen]
{\strut Proof};}}
}%
{\mdfsetup{%
frametitle={%
\tikz[baseline=(current bounding box.east),outer sep=0pt]
\node[anchor=east,rectangle,fill=SpringGreen]
{\strut Proof:~#1};}}%
}%
\mdfsetup{innertopmargin=10pt,linecolor=SpringGreen,%
linewidth=2pt,topline=true,%
frametitleaboveskip=\dimexpr-\ht\strutbox\relax
}
\begin{mdframed}[]\relax%
\label{#1}}{\qed\end{mdframed}}


\theoremstyle{definition}

\newmdtheoremenv[nobreak=true]{definition}{Definition}
\newmdtheoremenv[nobreak=true]{prop}{Proposition}
\newmdtheoremenv[nobreak=true]{theorem}{Theorem}
\newmdtheoremenv[nobreak=true]{corollary}{Corollary}
\newtheorem*{eg}{Example}
\theoremstyle{remark}
\newtheorem*{case}{Case}
\newtheorem*{notation}{Notation}
\newtheorem*{remark}{Remark}
\newtheorem*{note}{Note}
\newtheorem*{problem}{Problem}
\newtheorem*{observe}{Observe}
\newtheorem*{property}{Property}
\newtheorem*{intuition}{Intuition}


% End example and intermezzo environments with a small diamond (just like proof
% environments end with a small square)
\usepackage{etoolbox}
\AtEndEnvironment{vb}{\null\hfill$\diamond$}%
\AtEndEnvironment{intermezzo}{\null\hfill$\diamond$}%
% \AtEndEnvironment{opmerking}{\null\hfill$\diamond$}%

% Fix some spacing
% http://tex.stackexchange.com/questions/22119/how-can-i-change-the-spacing-before-theorems-with-amsthm
\makeatletter
\def\thm@space@setup{%
  \thm@preskip=\parskip \thm@postskip=0pt
}

% Fix some stuff
% %http://tex.stackexchange.com/questions/76273/multiple-pdfs-with-page-group-included-in-a-single-page-warning
\pdfsuppresswarningpagegroup=1


% My name
\author{Jaden Wang}



\begin{document}
A soliton in a conduit can be described by the following PDE:
\begin{equation}
	A_t + (A^2)_z - \left( A^2 \left( \frac{A_t}{A } \right)_z  \right)_z =0 	
\end{equation}
where the function $ A(z,t)$ describes the non-dimensionalized cross-sectional area of the wave at location $ z$ and time  $ t$. Furthermore, the PDE satisfies the following boundary conditions:
\begin{align}
	\lim_{ z \to \pm \infty} A(z,t) &= 1 \\
	\lim_{ z \to \pm \infty} A_z(z,t) &= 0 \\
	\lim_{ z \to \pm \infty} A_{zz} (z,t) &= 0 \\
	A(0,t)=a, A'(0,t)&=0
\end{align}

Since we know that the soliton is a traveling wave in the $ +z$ direction, we can convert this PDE into an ODE using change of variables $ \zeta = z-ct$, where $ c$ is an unknown constant that represents traveling speed. Thus we let $ A(z,t) = f(\zeta)$ and by using the Chain rule we obtain the following boundary value problem:
\begin{align}
	-cf'+ (f^2)' - (f^2(-cf^{-1}f')')' &= 0 \\
	\lim_{ \zeta \to \pm \infty} f(\zeta) &= 1 \\
	\lim_{ \zeta \to \pm \infty} f'(\zeta) &= 0 \\
	\lim_{ \zeta \to \pm \infty} f'' (\zeta) &= 0 \\
	f(0)=a, f'(0)&=0 \label{eq:ic}
\end{align}
where the prime notation is short for $ \frac{d}{d \zeta}$.

We aim to reduce this ODE to first order. Notice since all terms on LHS are derivatives with respect to $ \zeta$, we can integrate both sides with respect to $ \zeta$ and obtain
\begin{equation}
\begin{split}
	-cf + f^2 - f^2(-cf^{-1}f')' &= D\\
	-cf + f^2 - cf^2 f^{-2} f' + cf^2 f^{-1} f''&= D \\
	-cf + f^2 - cf' + cf f''&= D \\
\end{split}
\end{equation}
We can find $ D$ by letting $ \zeta \to \infty$ and applying the BCs:
\begin{equation}
\begin{split}
	-c + 1 - 0 + 0 &= D \\
	D &= 1- c \\
\end{split}
\end{equation} 	
Then the ODE becomes
\begin{equation}
	-cf + f^2 - cf' + cf f''= 1-c 
\end{equation}

To obtain first order ODE, we need to multiply the integrating factor $ f^{-3} f'$ on both side and integrate:
\begin{equation}
\begin{split}
	-cf^{-2}f' + f^{-1}f' - cf^{-3}f'^2+ cf^{-2}f'f'' &= (1-c)f^{-3}f' \\
	cf^{-1} + \ln f + \frac{1}{2} c f^{-2}f'^2 &= \frac{1}{2}(c-1) f^{-2} + B 
\end{split}
\end{equation}
Again we take $ \zeta \to \infty$ and apply the BCs to find the constant $ B$:
\begin{equation}
\begin{split}
	c+ 0 + 0 &= \frac{1}{2}(c-1) + B \\
	B &= \frac{1}{2}(c+1) \\
\end{split}
\end{equation}	

And the ODE becomes:
\begin{equation}
\begin{split}
	cf^{-1} + \ln f + \frac{1}{2} c f^{-2}f'^2 &= \frac{1}{2}(c-1) f^{-2} + \frac{1}{2} (c+1) \\
	cf + f^2 \ln f + \frac{1}{2} c f'^2 &= \frac{1}{2} (c-1) + \frac{1}{2} (c+1)f^2 \\
\end{split}
\end{equation}

It remains to find the constant $c $ using the \cref{eq:ic}. At $ \zeta = 0$:
\begin{equation}
\begin{split}
	cf(0) + f(0)^2 \ln f(0) + \frac{1}{2} c f(0)'^2 &= \frac{1}{2} (c-1) + \frac{1}{2} (c+1)f(0)^2 \\
	ca + a^2 \ln a + 0 &= \frac{1}{2} (c-1) + \frac{1}{2}(c+1) a^2 \\
	ac - \frac{1}{2} c -\frac{1}{2} a^2 c &= -\frac{1}{2} + \frac{1}{2}a^2 - a^2 \ln a  \\
	(2a - 1 - a^2) c&= -1 + a^2 - 2a^2 \ln a \\
	c &= \frac{a^2 - 2a^2 \ln a -1}{2a - a^2 -1 } \\
\end{split}
\end{equation}
\end{document}

\documentclass[class=article,crop=false]{standalone} 
%Fall 2020
% Some basic packages
\usepackage{standalone}[subpreambles=true]
\usepackage[utf8]{inputenc}
\usepackage[T1]{fontenc}
\usepackage{textcomp}
\usepackage[english]{babel}
\usepackage{url}
\usepackage{graphicx}
\usepackage{float}
\usepackage{enumitem}


\pdfminorversion=7

% Don't indent paragraphs, leave some space between them
\usepackage{parskip}

% Hide page number when page is empty
\usepackage{emptypage}
\usepackage{subcaption}
\usepackage{multicol}
\usepackage[dvipsnames]{xcolor}


% Math stuff
\usepackage{amsmath, amsfonts, mathtools, amsthm, amssymb}
% Fancy script capitals
\usepackage{mathrsfs}
\usepackage{cancel}
% Bold math
\usepackage{bm}
% Some shortcuts
\newcommand{\rr}{\ensuremath{\mathbb{R}}}
\newcommand{\zz}{\ensuremath{\mathbb{Z}}}
\newcommand{\qq}{\ensuremath{\mathbb{Q}}}
\newcommand{\nn}{\ensuremath{\mathbb{N}}}
\newcommand{\ff}{\ensuremath{\mathbb{F}}}
\newcommand{\cc}{\ensuremath{\mathbb{C}}}
\renewcommand\O{\ensuremath{\emptyset}}
\newcommand{\norm}[1]{{\left\lVert{#1}\right\rVert}}
\renewcommand{\vec}[1]{{\mathbf{#1}}}
\newcommand\allbold[1]{{\boldmath\textbf{#1}}}

% Put x \to \infty below \lim
\let\svlim\lim\def\lim{\svlim\limits}

%Make implies and impliedby shorter
\let\implies\Rightarrow
\let\impliedby\Leftarrow
\let\iff\Leftrightarrow
\let\epsilon\varepsilon

% Add \contra symbol to denote contradiction
\usepackage{stmaryrd} % for \lightning
\newcommand\contra{\scalebox{1.5}{$\lightning$}}

% \let\phi\varphi

% Command for short corrections
% Usage: 1+1=\correct{3}{2}

\definecolor{correct}{HTML}{009900}
\newcommand\correct[2]{\ensuremath{\:}{\color{red}{#1}}\ensuremath{\to }{\color{correct}{#2}}\ensuremath{\:}}
\newcommand\green[1]{{\color{correct}{#1}}}

% horizontal rule
\newcommand\hr{
    \noindent\rule[0.5ex]{\linewidth}{0.5pt}
}

% hide parts
\newcommand\hide[1]{}

% si unitx
\usepackage{siunitx}
\sisetup{locale = FR}

% Environments
\makeatother
% For box around Definition, Theorem, \ldots
\usepackage[framemethod=TikZ]{mdframed}
\mdfsetup{skipabove=1em,skipbelow=0em}

%definition
\newenvironment{defn}[1][]{%
\ifstrempty{#1}%
{\mdfsetup{%
frametitle={%
\tikz[baseline=(current bounding box.east),outer sep=0pt]
\node[anchor=east,rectangle,fill=Emerald]
{\strut Definition};}}
}%
{\mdfsetup{%
frametitle={%
\tikz[baseline=(current bounding box.east),outer sep=0pt]
\node[anchor=east,rectangle,fill=Emerald]
{\strut Definition:~#1};}}%
}%
\mdfsetup{innertopmargin=10pt,linecolor=Emerald,%
linewidth=2pt,topline=true,%
frametitleaboveskip=\dimexpr-\ht\strutbox\relax
}
\begin{mdframed}[]\relax%
\label{#1}}{\end{mdframed}}


%theorem
%\newcounter{thm}[section]\setcounter{thm}{0}
%\renewcommand{\thethm}{\arabic{section}.\arabic{thm}}
\newenvironment{thm}[1][]{%
%\refstepcounter{thm}%
\ifstrempty{#1}%
{\mdfsetup{%
frametitle={%
\tikz[baseline=(current bounding box.east),outer sep=0pt]
\node[anchor=east,rectangle,fill=blue!20]
%{\strut Theorem~\thethm};}}
{\strut Theorem};}}
}%
{\mdfsetup{%
frametitle={%
\tikz[baseline=(current bounding box.east),outer sep=0pt]
\node[anchor=east,rectangle,fill=blue!20]
%{\strut Theorem~\thethm:~#1};}}%
{\strut Theorem:~#1};}}%
}%
\mdfsetup{innertopmargin=10pt,linecolor=blue!20,%
linewidth=2pt,topline=true,%
frametitleaboveskip=\dimexpr-\ht\strutbox\relax
}
\begin{mdframed}[]\relax%
\label{#1}}{\end{mdframed}}


%lemma
\newenvironment{lem}[1][]{%
\ifstrempty{#1}%
{\mdfsetup{%
frametitle={%
\tikz[baseline=(current bounding box.east),outer sep=0pt]
\node[anchor=east,rectangle,fill=Dandelion]
{\strut Lemma};}}
}%
{\mdfsetup{%
frametitle={%
\tikz[baseline=(current bounding box.east),outer sep=0pt]
\node[anchor=east,rectangle,fill=Dandelion]
{\strut Lemma:~#1};}}%
}%
\mdfsetup{innertopmargin=10pt,linecolor=Dandelion,%
linewidth=2pt,topline=true,%
frametitleaboveskip=\dimexpr-\ht\strutbox\relax
}
\begin{mdframed}[]\relax%
\label{#1}}{\end{mdframed}}

%corollary
\newenvironment{coro}[1][]{%
\ifstrempty{#1}%
{\mdfsetup{%
frametitle={%
\tikz[baseline=(current bounding box.east),outer sep=0pt]
\node[anchor=east,rectangle,fill=CornflowerBlue]
{\strut Corollary};}}
}%
{\mdfsetup{%
frametitle={%
\tikz[baseline=(current bounding box.east),outer sep=0pt]
\node[anchor=east,rectangle,fill=CornflowerBlue]
{\strut Corollary:~#1};}}%
}%
\mdfsetup{innertopmargin=10pt,linecolor=CornflowerBlue,%
linewidth=2pt,topline=true,%
frametitleaboveskip=\dimexpr-\ht\strutbox\relax
}
\begin{mdframed}[]\relax%
\label{#1}}{\end{mdframed}}

%proof
\newenvironment{prf}[1][]{%
\ifstrempty{#1}%
{\mdfsetup{%
frametitle={%
\tikz[baseline=(current bounding box.east),outer sep=0pt]
\node[anchor=east,rectangle,fill=SpringGreen]
{\strut Proof};}}
}%
{\mdfsetup{%
frametitle={%
\tikz[baseline=(current bounding box.east),outer sep=0pt]
\node[anchor=east,rectangle,fill=SpringGreen]
{\strut Proof:~#1};}}%
}%
\mdfsetup{innertopmargin=10pt,linecolor=SpringGreen,%
linewidth=2pt,topline=true,%
frametitleaboveskip=\dimexpr-\ht\strutbox\relax
}
\begin{mdframed}[]\relax%
\label{#1}}{\qed\end{mdframed}}


\theoremstyle{definition}

\newmdtheoremenv[nobreak=true]{definition}{Definition}
\newmdtheoremenv[nobreak=true]{prop}{Proposition}
\newmdtheoremenv[nobreak=true]{theorem}{Theorem}
\newmdtheoremenv[nobreak=true]{corollary}{Corollary}
\newtheorem*{eg}{Example}
\theoremstyle{remark}
\newtheorem*{case}{Case}
\newtheorem*{notation}{Notation}
\newtheorem*{remark}{Remark}
\newtheorem*{note}{Note}
\newtheorem*{problem}{Problem}
\newtheorem*{observe}{Observe}
\newtheorem*{property}{Property}
\newtheorem*{intuition}{Intuition}


% End example and intermezzo environments with a small diamond (just like proof
% environments end with a small square)
\usepackage{etoolbox}
\AtEndEnvironment{vb}{\null\hfill$\diamond$}%
\AtEndEnvironment{intermezzo}{\null\hfill$\diamond$}%
% \AtEndEnvironment{opmerking}{\null\hfill$\diamond$}%

% Fix some spacing
% http://tex.stackexchange.com/questions/22119/how-can-i-change-the-spacing-before-theorems-with-amsthm
\makeatletter
\def\thm@space@setup{%
  \thm@preskip=\parskip \thm@postskip=0pt
}

% Fix some stuff
% %http://tex.stackexchange.com/questions/76273/multiple-pdfs-with-page-group-included-in-a-single-page-warning
\pdfsuppresswarningpagegroup=1


% My name
\author{Jaden Wang}



\begin{document}
\subsection{Classification Theorem}

It asserts that every second-order linear PDE with constant coefficients, where the unknown function has \emph{two} independent variables, can be transformed by a change of variables into exactly one of the following forms:
\begin{enumerate}[label=\arabic*)]
	\item generalized wave equation
		\[
			-c^2 u_{x x} + u_{tt} + \alpha u = f(x,t), c>0 \text{ (hyperbolic case)} 
		\] 
		since it's 2nd derivative minus 2nd derivative.
	\item generalized Poisson/Laplace equation $ (t=y)$
		 \[
			 a^2 u_{x x}+ u_{tt} + \alpha u = g(x,t), a>0 \text{ (elliptic case)} 
		.\] 
	\item generalized heat equation
		\[
			-k^2 u_{x x} + u_{t} + \alpha u = h(x,t), k>0 \text{ (parabolic case)} 
		.\]
	\item
		\[
			u_{x x}+ cu=g(x,t), \text{ (degenerate case)} 
		.\] 
\end{enumerate}

\newpage
\section{Fourier Transform}

\subsection{Complex Fourier Series}

Recall by Euler's formula,
\[
\cos \left( \frac{ n\pi x}{ L} \right) =\frac{e^{   i n\pi x / L} + e^{ - i n\pi x / L} }{2 } \text{ and } \sin \left( \frac{ n\pi x}{ L} \right) = \frac{e^{   i n\pi x / L} -e^{ - i n\pi x / L} }{ 2i} 
.\]

Therefore, the F.S. becomes
\begin{align*}
	a_0 + \sum_{ n= 1}^{\infty} a_n \cos \left( \frac{ n\pi x}{ L} \right) + b_n \sin \left( \frac{ n\pi x}{ L} \right) &= a_0 + \sum_{ n= 1}^{\infty} \left( \frac{a_n}{2 }+ \frac{b_n}{2i } \right) e^{   i n\pi x / L} + \left( \frac{a_n}{2 } - \frac{b_n}{2i } \right)  e^{ - i n\pi x / L}  \\
															    &= a_0 + \sum_{ n= 1}^{\infty} \left( \frac{a_n - i b_n}{2 } \right) e^{   i n\pi x / L} + \left( \frac{a_n + ib_n}{2 } \right) e^{ - i n\pi x / L}    \\
															    &= a_0 + \sum_{ n= 1}^{\infty} c_n e^{   i n\pi x / L} + c_{-n} e^{ - i n\pi x / L}  \\
\end{align*}

\begin{notation}
$c_{-n}$ denotes the coefficient of the exponential term with $-n$.
\end{notation}
If we let $ n=-m$, then by projection,
\begin{align*}
	c_{-n}=\frac{1}{2}(a_n+ib_n) &= \frac{1}{2} \left( \frac{1}{L}\int_{-L}^{L} f(x) \cos \left( \frac{ n\pi x}{ L} \right) dx + \frac{i}{L} \int_{-L}^{L} f(x) \sin \left( \frac{ n\pi x}{ L} \right) dx  \right)  \\
				     &= \frac{1}{2} \left( \frac{1}{L} \int_{-L}^{L} f(x) \cos \left( \frac{ -m \pi x}{ L} \right) dx + \frac{i}{L} \int_{-L}^{L} f(x) \sin \left( \frac{ -m\pi x}{ L} \right) dx  \right)  \\
				     &=  \frac{1}{2} \left( \frac{1}{L} \int_{-L}^{L} f(x) \cos \left( \frac{ m \pi x}{ L} \right) dx - \frac{i}{L} \int_{-L}^{L} f(x) \sin \left( \frac{ m\pi x}{ L} \right) dx  \right)   \\
				     &= \frac{1}{2} (a_m-ib_m) = c_m \\
\end{align*}
Therefore, 
\[
\text{ F.S.} [ f]( x) = a_0+ \sum_{ n= 1}^{\infty} c_n e^{   i n\pi x / L} + \sum_{ m= -1}^{-\infty} c_m e^{   i n\pi x / L} =\sum_{ n= -\infty}^{\infty} c_n e^{   i n\pi x / L} 
.\] 
\end{document}

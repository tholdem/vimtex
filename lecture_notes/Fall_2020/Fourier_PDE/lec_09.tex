\documentclass[class=article,crop=false]{standalone} 
%Fall 2020
% Some basic packages
\usepackage{standalone}[subpreambles=true]
\usepackage[utf8]{inputenc}
\usepackage[T1]{fontenc}
\usepackage{textcomp}
\usepackage[english]{babel}
\usepackage{url}
\usepackage{graphicx}
\usepackage{float}
\usepackage{enumitem}


\pdfminorversion=7

% Don't indent paragraphs, leave some space between them
\usepackage{parskip}

% Hide page number when page is empty
\usepackage{emptypage}
\usepackage{subcaption}
\usepackage{multicol}
\usepackage[dvipsnames]{xcolor}


% Math stuff
\usepackage{amsmath, amsfonts, mathtools, amsthm, amssymb}
% Fancy script capitals
\usepackage{mathrsfs}
\usepackage{cancel}
% Bold math
\usepackage{bm}
% Some shortcuts
\newcommand{\rr}{\ensuremath{\mathbb{R}}}
\newcommand{\zz}{\ensuremath{\mathbb{Z}}}
\newcommand{\qq}{\ensuremath{\mathbb{Q}}}
\newcommand{\nn}{\ensuremath{\mathbb{N}}}
\newcommand{\ff}{\ensuremath{\mathbb{F}}}
\newcommand{\cc}{\ensuremath{\mathbb{C}}}
\renewcommand\O{\ensuremath{\emptyset}}
\newcommand{\norm}[1]{{\left\lVert{#1}\right\rVert}}
\renewcommand{\vec}[1]{{\mathbf{#1}}}
\newcommand\allbold[1]{{\boldmath\textbf{#1}}}

% Put x \to \infty below \lim
\let\svlim\lim\def\lim{\svlim\limits}

%Make implies and impliedby shorter
\let\implies\Rightarrow
\let\impliedby\Leftarrow
\let\iff\Leftrightarrow
\let\epsilon\varepsilon

% Add \contra symbol to denote contradiction
\usepackage{stmaryrd} % for \lightning
\newcommand\contra{\scalebox{1.5}{$\lightning$}}

% \let\phi\varphi

% Command for short corrections
% Usage: 1+1=\correct{3}{2}

\definecolor{correct}{HTML}{009900}
\newcommand\correct[2]{\ensuremath{\:}{\color{red}{#1}}\ensuremath{\to }{\color{correct}{#2}}\ensuremath{\:}}
\newcommand\green[1]{{\color{correct}{#1}}}

% horizontal rule
\newcommand\hr{
    \noindent\rule[0.5ex]{\linewidth}{0.5pt}
}

% hide parts
\newcommand\hide[1]{}

% si unitx
\usepackage{siunitx}
\sisetup{locale = FR}

% Environments
\makeatother
% For box around Definition, Theorem, \ldots
\usepackage[framemethod=TikZ]{mdframed}
\mdfsetup{skipabove=1em,skipbelow=0em}

%definition
\newenvironment{defn}[1][]{%
\ifstrempty{#1}%
{\mdfsetup{%
frametitle={%
\tikz[baseline=(current bounding box.east),outer sep=0pt]
\node[anchor=east,rectangle,fill=Emerald]
{\strut Definition};}}
}%
{\mdfsetup{%
frametitle={%
\tikz[baseline=(current bounding box.east),outer sep=0pt]
\node[anchor=east,rectangle,fill=Emerald]
{\strut Definition:~#1};}}%
}%
\mdfsetup{innertopmargin=10pt,linecolor=Emerald,%
linewidth=2pt,topline=true,%
frametitleaboveskip=\dimexpr-\ht\strutbox\relax
}
\begin{mdframed}[]\relax%
\label{#1}}{\end{mdframed}}


%theorem
%\newcounter{thm}[section]\setcounter{thm}{0}
%\renewcommand{\thethm}{\arabic{section}.\arabic{thm}}
\newenvironment{thm}[1][]{%
%\refstepcounter{thm}%
\ifstrempty{#1}%
{\mdfsetup{%
frametitle={%
\tikz[baseline=(current bounding box.east),outer sep=0pt]
\node[anchor=east,rectangle,fill=blue!20]
%{\strut Theorem~\thethm};}}
{\strut Theorem};}}
}%
{\mdfsetup{%
frametitle={%
\tikz[baseline=(current bounding box.east),outer sep=0pt]
\node[anchor=east,rectangle,fill=blue!20]
%{\strut Theorem~\thethm:~#1};}}%
{\strut Theorem:~#1};}}%
}%
\mdfsetup{innertopmargin=10pt,linecolor=blue!20,%
linewidth=2pt,topline=true,%
frametitleaboveskip=\dimexpr-\ht\strutbox\relax
}
\begin{mdframed}[]\relax%
\label{#1}}{\end{mdframed}}


%lemma
\newenvironment{lem}[1][]{%
\ifstrempty{#1}%
{\mdfsetup{%
frametitle={%
\tikz[baseline=(current bounding box.east),outer sep=0pt]
\node[anchor=east,rectangle,fill=Dandelion]
{\strut Lemma};}}
}%
{\mdfsetup{%
frametitle={%
\tikz[baseline=(current bounding box.east),outer sep=0pt]
\node[anchor=east,rectangle,fill=Dandelion]
{\strut Lemma:~#1};}}%
}%
\mdfsetup{innertopmargin=10pt,linecolor=Dandelion,%
linewidth=2pt,topline=true,%
frametitleaboveskip=\dimexpr-\ht\strutbox\relax
}
\begin{mdframed}[]\relax%
\label{#1}}{\end{mdframed}}

%corollary
\newenvironment{coro}[1][]{%
\ifstrempty{#1}%
{\mdfsetup{%
frametitle={%
\tikz[baseline=(current bounding box.east),outer sep=0pt]
\node[anchor=east,rectangle,fill=CornflowerBlue]
{\strut Corollary};}}
}%
{\mdfsetup{%
frametitle={%
\tikz[baseline=(current bounding box.east),outer sep=0pt]
\node[anchor=east,rectangle,fill=CornflowerBlue]
{\strut Corollary:~#1};}}%
}%
\mdfsetup{innertopmargin=10pt,linecolor=CornflowerBlue,%
linewidth=2pt,topline=true,%
frametitleaboveskip=\dimexpr-\ht\strutbox\relax
}
\begin{mdframed}[]\relax%
\label{#1}}{\end{mdframed}}

%proof
\newenvironment{prf}[1][]{%
\ifstrempty{#1}%
{\mdfsetup{%
frametitle={%
\tikz[baseline=(current bounding box.east),outer sep=0pt]
\node[anchor=east,rectangle,fill=SpringGreen]
{\strut Proof};}}
}%
{\mdfsetup{%
frametitle={%
\tikz[baseline=(current bounding box.east),outer sep=0pt]
\node[anchor=east,rectangle,fill=SpringGreen]
{\strut Proof:~#1};}}%
}%
\mdfsetup{innertopmargin=10pt,linecolor=SpringGreen,%
linewidth=2pt,topline=true,%
frametitleaboveskip=\dimexpr-\ht\strutbox\relax
}
\begin{mdframed}[]\relax%
\label{#1}}{\qed\end{mdframed}}


\theoremstyle{definition}

\newmdtheoremenv[nobreak=true]{definition}{Definition}
\newmdtheoremenv[nobreak=true]{prop}{Proposition}
\newmdtheoremenv[nobreak=true]{theorem}{Theorem}
\newmdtheoremenv[nobreak=true]{corollary}{Corollary}
\newtheorem*{eg}{Example}
\theoremstyle{remark}
\newtheorem*{case}{Case}
\newtheorem*{notation}{Notation}
\newtheorem*{remark}{Remark}
\newtheorem*{note}{Note}
\newtheorem*{problem}{Problem}
\newtheorem*{observe}{Observe}
\newtheorem*{property}{Property}
\newtheorem*{intuition}{Intuition}


% End example and intermezzo environments with a small diamond (just like proof
% environments end with a small square)
\usepackage{etoolbox}
\AtEndEnvironment{vb}{\null\hfill$\diamond$}%
\AtEndEnvironment{intermezzo}{\null\hfill$\diamond$}%
% \AtEndEnvironment{opmerking}{\null\hfill$\diamond$}%

% Fix some spacing
% http://tex.stackexchange.com/questions/22119/how-can-i-change-the-spacing-before-theorems-with-amsthm
\makeatletter
\def\thm@space@setup{%
  \thm@preskip=\parskip \thm@postskip=0pt
}

% Fix some stuff
% %http://tex.stackexchange.com/questions/76273/multiple-pdfs-with-page-group-included-in-a-single-page-warning
\pdfsuppresswarningpagegroup=1


% My name
\author{Jaden Wang}



\begin{document}
\newpage
\section{Uniform Convergence}

~\begin{defn}[Pointwise Convergence]
	For every $ \epsilon >0$ and each $ x_0 \in [-L,L]$, there exists a positive, finite integer $ N_{ \epsilon}(x_0)$ such that if $ N \geq N_{ \epsilon}(x_0)$, then
	\[
		|S_N(x_0)-T(x_0)|< \epsilon
	.\] 
	where $ S_N(x_0)$ is the $ N$th partial sum of the Fourier series with  $ x=x_0$.
\end{defn}
\begin{defn}[Uniform Convergence]
For every $ \epsilon > 0$, there exists a $ N_{ \epsilon} \in \nn$ such that 
\[
	|S_N(x)-T(x)|< \epsilon
.\] 
for all $ x \in [-L,L]$.
\end{defn}

\begin{note}[]
~\begin{enumerate}[label=\alph*)]
	\item Pointwise convergence implies $ \lim_{ N \to \infty} |S_N(x)-T(x)| =0$ for all $ x \in [-L,L]$.
	\item Uniform convergence implies $ \lim_{ N \to \infty} \max_{-L\leq x \leq L} |S_N(x) - T(x)| = 0$.
\end{enumerate}
Uniform convergence is stronger and implies pointwise convergence. 
\end{note}

\begin{eg}[]
	Suppose $ f_n(x) = \frac{x+2}{4n}$ for $ n \in \nn$ and $ x \in [-2,2]$. Then $ ( f_n(x))$ converges uniformly to $ h(x) =0 $. Note that  $ (f_n(x))$ is a sequence of constants for each fixed $ x_0 \in [-2,2]$.

	To show that it is pointwise convergence, given $ x_0 \in [-2,2]$, we have
	\[
		\lim_{ n \to \infty} f_n(x_0) = \lim_{ n \to \infty} \frac{x_0+2}{n}=0 \implies \text{ pointwise convergence} 
	.\] 

	For uniform convergence, we observe that for any $ x \in [-2,2]$ the maximum vertical separation of $ f_n(x)$ from $ h(x)$ is  $ \frac{1}{n}$ for each $ n$ (because the maximum difference is achieved at $ x=2$), thus
	 \[
		 \lim_{ n \to \infty} \max_{-2\leq x \leq 2} |f_n(x)-h(x)| = \lim_{ n \to \infty} \frac{1}{n} = 0 \implies \text{ uniform convergence} 
	.\] 
\end{eg}

\begin{eg}[]
\begin{equation*}
	g_n(x)
\begin{cases}
	nx \qquad & 0<x\leq \frac{1}{n}\\
	2-nx \qquad & \frac{1}{n} < x \leq \frac{2}{n}\\
	0 \qquad & \text{ for all other } x \in [-2,2] 
\end{cases}
\end{equation*}
then $ g_n(x)$ converges pointwise but not uniformly.

Pointwise: Clearly if $ x_0 \in [-2,0]$ then $\lim_{ n \to \infty} g_n(x_0)=0$. If $ x_0 \in (0,2]$ then for any $ N > \frac{2}{x_0}$, if $ n \geq N$, then  $ x_0 > \frac{2}{n}$ so $ x_0 \in (\frac{2}{n},2]$ so $ g_n(x_0) =0$ for all $ n\geq N$ so
 \[
	 \lim_{ n \to \infty} g_n(x_0) = 0 
.\] 
Note that $ N> \frac{2}{x_0}$ is obtained by reverse engineering on the scratch paper.

Uniform: the maximum vertical separation of $ g_n(x)$ from $ h(x)$ is a fixed distance of 1 (at $ x=\frac{1}{n}$ ) for any choice of  $ n \geq 1$, thus
 \[
	 \lim_{ n \to \infty} \max_{x \in [-2,2]} |g_n(x) - h(x)| = \lim_{ n \to \infty}  1 \neq 0
.\] 
Hence it doesn't converge uniformly. 

\begin{defn}[Absolute Convergence]
	\item The $ \text{ F.S.} [ f]( x) $ is \allbold{absolutely convergent} if, for every $ \epsilon>0$, there exists an integer $ 0< M_{ \epsilon}< \infty$ such that
		\[
		0\leq \sum_{ n=M_{ \epsilon}+1}^{\infty} |a_n| + \sum_{ n=M_{ \epsilon}+1}^{\infty} |b_n| < \epsilon \qquad \text{ \emph{i.e.} the tail converges absolutely} 
		.\] 
\begin{note}[]
~\begin{enumerate}[label=\arabic*)]
	\item if $ \text{ F.S.} [ f]( x) $ is absolutely convergent then
	\begin{align*}
		0 &\leq |a_0| + \sum_{ n= 1}^{\infty} \left| a_n \cos \left( \frac{ n\pi x}{ L} \right)  \right| + \sum_{ n= 1}^{\infty} \left| b_n \sin \left( \frac{ n\pi x}{ L} \right)  \right| \\
		  &\leq |a_0| + \sum_{ n= 1}^{\infty} |a_n| + \sum_{ n= 1}^{\infty} |b_n|\\
		  &< \infty
	\end{align*}
	\item if $ \text{ F.S.} [ f]( x) $ is absolutely convergent then it is uniformly convergent.
	\item if $ \text{ F.S.} [ f]( x) $ is uniformly convergent then it is pointwise convergent.
	\item there exist series of functions which are uniformly convergent but not absolutely convergent.
\end{enumerate}
\end{note}
\end{defn}

\begin{thm}[Weierstrass M-test]
	If $ (f_n(x))$ is a sequence of functions defined on a set $ E$ and  $ (M_n)$ is a sequence of non-negative numbers such that $ |f_n(x)|<M_n$ for all $ x \in E$ and $ n\geq 0$. Then $ \sum_{ n= 0}^{\infty} f_n(x)$ converges uniformly if $ \sum_{ n= 0}^{\infty} M_n$ converges.
\end{thm}
\end{eg}

\begin{defn}[Gibbs Phenomenon]
\begin{enumerate}[label=\alph*)]
	\item "Gibbs phenomenon" is a persistent overestimation or underestimation of the value of any piece wise smooth function with a jump discontinuity.
	\item It occurs in truncated Fourier series of functions with jump discontinuities and does NOT go away as the number of terms is increased.
	\item As the number of terms used in increased, the location of the overshoot moves closer and closer to jump discontinuity without ever reaching it.
	\item As the number of terms increases, the size of the overshoot approaches a limiting value, proportional to the magnitude of the jump discontinuity with a constant of proportionality that is universal.
\end{enumerate}
\end{defn}

\begin{eg}[]
\begin{equation*}
	f(x)=
\begin{cases}
	0, \qquad & -\pi<x<0\\
	\pi, \qquad &0\leq x \leq \pi
\end{cases}
\end{equation*}
\[
	\text{ F.S.} [ f]( x) = \frac{\pi}{2} + 2 \sum_{ n= 0}^{\infty} \frac{\sin[(2n+1)x]}{2n+1}
.\] 

The truncated form of the Fourier series has the form
\[
	\tilde{ f_M}( x) = \frac{\pi}{2} + 2\left( \sin(x) + \frac{\sin(3x)}{3} + \ldots + \frac{\sin[(2(M-2)+1)x]}{2(M-2)+1}\right)  
.\]

\end{eg}
\begin{intuition}
	Gibbs phenomenon is the result of the fact that points in the middle of the interval are converging faster than points at the endpoints/discontinuities, due to pointwise convergence.
\end{intuition}
\end{document}

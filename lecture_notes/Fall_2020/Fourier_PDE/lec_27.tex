\documentclass[class=article,crop=false]{standalone} 
%Fall 2020
% Some basic packages
\usepackage{standalone}[subpreambles=true]
\usepackage[utf8]{inputenc}
\usepackage[T1]{fontenc}
\usepackage{textcomp}
\usepackage[english]{babel}
\usepackage{url}
\usepackage{graphicx}
\usepackage{float}
\usepackage{enumitem}


\pdfminorversion=7

% Don't indent paragraphs, leave some space between them
\usepackage{parskip}

% Hide page number when page is empty
\usepackage{emptypage}
\usepackage{subcaption}
\usepackage{multicol}
\usepackage[dvipsnames]{xcolor}


% Math stuff
\usepackage{amsmath, amsfonts, mathtools, amsthm, amssymb}
% Fancy script capitals
\usepackage{mathrsfs}
\usepackage{cancel}
% Bold math
\usepackage{bm}
% Some shortcuts
\newcommand{\rr}{\ensuremath{\mathbb{R}}}
\newcommand{\zz}{\ensuremath{\mathbb{Z}}}
\newcommand{\qq}{\ensuremath{\mathbb{Q}}}
\newcommand{\nn}{\ensuremath{\mathbb{N}}}
\newcommand{\ff}{\ensuremath{\mathbb{F}}}
\newcommand{\cc}{\ensuremath{\mathbb{C}}}
\renewcommand\O{\ensuremath{\emptyset}}
\newcommand{\norm}[1]{{\left\lVert{#1}\right\rVert}}
\renewcommand{\vec}[1]{{\mathbf{#1}}}
\newcommand\allbold[1]{{\boldmath\textbf{#1}}}

% Put x \to \infty below \lim
\let\svlim\lim\def\lim{\svlim\limits}

%Make implies and impliedby shorter
\let\implies\Rightarrow
\let\impliedby\Leftarrow
\let\iff\Leftrightarrow
\let\epsilon\varepsilon

% Add \contra symbol to denote contradiction
\usepackage{stmaryrd} % for \lightning
\newcommand\contra{\scalebox{1.5}{$\lightning$}}

% \let\phi\varphi

% Command for short corrections
% Usage: 1+1=\correct{3}{2}

\definecolor{correct}{HTML}{009900}
\newcommand\correct[2]{\ensuremath{\:}{\color{red}{#1}}\ensuremath{\to }{\color{correct}{#2}}\ensuremath{\:}}
\newcommand\green[1]{{\color{correct}{#1}}}

% horizontal rule
\newcommand\hr{
    \noindent\rule[0.5ex]{\linewidth}{0.5pt}
}

% hide parts
\newcommand\hide[1]{}

% si unitx
\usepackage{siunitx}
\sisetup{locale = FR}

% Environments
\makeatother
% For box around Definition, Theorem, \ldots
\usepackage[framemethod=TikZ]{mdframed}
\mdfsetup{skipabove=1em,skipbelow=0em}

%definition
\newenvironment{defn}[1][]{%
\ifstrempty{#1}%
{\mdfsetup{%
frametitle={%
\tikz[baseline=(current bounding box.east),outer sep=0pt]
\node[anchor=east,rectangle,fill=Emerald]
{\strut Definition};}}
}%
{\mdfsetup{%
frametitle={%
\tikz[baseline=(current bounding box.east),outer sep=0pt]
\node[anchor=east,rectangle,fill=Emerald]
{\strut Definition:~#1};}}%
}%
\mdfsetup{innertopmargin=10pt,linecolor=Emerald,%
linewidth=2pt,topline=true,%
frametitleaboveskip=\dimexpr-\ht\strutbox\relax
}
\begin{mdframed}[]\relax%
\label{#1}}{\end{mdframed}}


%theorem
%\newcounter{thm}[section]\setcounter{thm}{0}
%\renewcommand{\thethm}{\arabic{section}.\arabic{thm}}
\newenvironment{thm}[1][]{%
%\refstepcounter{thm}%
\ifstrempty{#1}%
{\mdfsetup{%
frametitle={%
\tikz[baseline=(current bounding box.east),outer sep=0pt]
\node[anchor=east,rectangle,fill=blue!20]
%{\strut Theorem~\thethm};}}
{\strut Theorem};}}
}%
{\mdfsetup{%
frametitle={%
\tikz[baseline=(current bounding box.east),outer sep=0pt]
\node[anchor=east,rectangle,fill=blue!20]
%{\strut Theorem~\thethm:~#1};}}%
{\strut Theorem:~#1};}}%
}%
\mdfsetup{innertopmargin=10pt,linecolor=blue!20,%
linewidth=2pt,topline=true,%
frametitleaboveskip=\dimexpr-\ht\strutbox\relax
}
\begin{mdframed}[]\relax%
\label{#1}}{\end{mdframed}}


%lemma
\newenvironment{lem}[1][]{%
\ifstrempty{#1}%
{\mdfsetup{%
frametitle={%
\tikz[baseline=(current bounding box.east),outer sep=0pt]
\node[anchor=east,rectangle,fill=Dandelion]
{\strut Lemma};}}
}%
{\mdfsetup{%
frametitle={%
\tikz[baseline=(current bounding box.east),outer sep=0pt]
\node[anchor=east,rectangle,fill=Dandelion]
{\strut Lemma:~#1};}}%
}%
\mdfsetup{innertopmargin=10pt,linecolor=Dandelion,%
linewidth=2pt,topline=true,%
frametitleaboveskip=\dimexpr-\ht\strutbox\relax
}
\begin{mdframed}[]\relax%
\label{#1}}{\end{mdframed}}

%corollary
\newenvironment{coro}[1][]{%
\ifstrempty{#1}%
{\mdfsetup{%
frametitle={%
\tikz[baseline=(current bounding box.east),outer sep=0pt]
\node[anchor=east,rectangle,fill=CornflowerBlue]
{\strut Corollary};}}
}%
{\mdfsetup{%
frametitle={%
\tikz[baseline=(current bounding box.east),outer sep=0pt]
\node[anchor=east,rectangle,fill=CornflowerBlue]
{\strut Corollary:~#1};}}%
}%
\mdfsetup{innertopmargin=10pt,linecolor=CornflowerBlue,%
linewidth=2pt,topline=true,%
frametitleaboveskip=\dimexpr-\ht\strutbox\relax
}
\begin{mdframed}[]\relax%
\label{#1}}{\end{mdframed}}

%proof
\newenvironment{prf}[1][]{%
\ifstrempty{#1}%
{\mdfsetup{%
frametitle={%
\tikz[baseline=(current bounding box.east),outer sep=0pt]
\node[anchor=east,rectangle,fill=SpringGreen]
{\strut Proof};}}
}%
{\mdfsetup{%
frametitle={%
\tikz[baseline=(current bounding box.east),outer sep=0pt]
\node[anchor=east,rectangle,fill=SpringGreen]
{\strut Proof:~#1};}}%
}%
\mdfsetup{innertopmargin=10pt,linecolor=SpringGreen,%
linewidth=2pt,topline=true,%
frametitleaboveskip=\dimexpr-\ht\strutbox\relax
}
\begin{mdframed}[]\relax%
\label{#1}}{\qed\end{mdframed}}


\theoremstyle{definition}

\newmdtheoremenv[nobreak=true]{definition}{Definition}
\newmdtheoremenv[nobreak=true]{prop}{Proposition}
\newmdtheoremenv[nobreak=true]{theorem}{Theorem}
\newmdtheoremenv[nobreak=true]{corollary}{Corollary}
\newtheorem*{eg}{Example}
\theoremstyle{remark}
\newtheorem*{case}{Case}
\newtheorem*{notation}{Notation}
\newtheorem*{remark}{Remark}
\newtheorem*{note}{Note}
\newtheorem*{problem}{Problem}
\newtheorem*{observe}{Observe}
\newtheorem*{property}{Property}
\newtheorem*{intuition}{Intuition}


% End example and intermezzo environments with a small diamond (just like proof
% environments end with a small square)
\usepackage{etoolbox}
\AtEndEnvironment{vb}{\null\hfill$\diamond$}%
\AtEndEnvironment{intermezzo}{\null\hfill$\diamond$}%
% \AtEndEnvironment{opmerking}{\null\hfill$\diamond$}%

% Fix some spacing
% http://tex.stackexchange.com/questions/22119/how-can-i-change-the-spacing-before-theorems-with-amsthm
\makeatletter
\def\thm@space@setup{%
  \thm@preskip=\parskip \thm@postskip=0pt
}

% Fix some stuff
% %http://tex.stackexchange.com/questions/76273/multiple-pdfs-with-page-group-included-in-a-single-page-warning
\pdfsuppresswarningpagegroup=1


% My name
\author{Jaden Wang}



\begin{document}

\begin{equation*}
\begin{cases}
	G''(y) = \lambda G(y)\\
	G(0)=0, G(H)=0\\
\end{cases}
\end{equation*}
and  $F''(x)=-\lambda F(x)$. We do the same process again to solve $ u_2(x,t)$. Note that since only $ \lambda <0$ works to solve to F-equation, we require $ |\lambda|=\left( \frac{n\pi}{H } \right)^2 \implies \lambda = -\left(\frac{n\pi}{H } \right)^2$. The solution is

\[
	u_2(x,y)=\sum_{ n= 1}^{\infty} c_n \sinh\left(\frac{n\pi x}{H }  \right)
	\sin \left( \frac{ n\pi y}{ H} \right) + d_n \sinh\left(\frac{n\pi [x-L]}{H } \right) \sin \left( \frac{ n\pi y}{ H} \right) 
.\] 
with coefficients
\[
	c_n= \frac{\frac{2}{H} \int_0^H g_2(y) \sin \left( \frac{ n\pi y}{ H} \right) dy }{\sinh\left(\frac{n\pi L}{H } \right) } \text{ and } d_n =  \frac{\frac{2}{H} \int_0^H g_1(y) \sin \left( \frac{ n\pi y}{ H} \right) dy }{\sinh\left(\frac{-n\pi L}{H } \right) }
.\] 
Then the \allbold{formal solution} is 
\begin{align*}
	u(x,y) &= u_1(x,y)+u_2(x,y) \\
	&= \sum_{ n= 1}^{\infty} a_n \sinh\left(\frac{n\pi y}{L }  \right) \sin \left( \frac{ n\pi x}{ L} \right) + b_n \sinh\left(\frac{n\pi [y-H]}{L } \right) \sin \left( \frac{ n\pi x}{ L} \right) \\
	&\quad + \sum_{ n= 1}^{\infty} c_n \sinh\left(\frac{n\pi x}{H }  \right) \sin \left( \frac{ n\pi y}{ H} \right) + d_n \sinh\left(\frac{n\pi [x-L]}{H } \right) \sin \left( \frac{ n\pi y}{ H} \right) \\
\end{align*}
with the coefficients:
\begin{align*}
	a_n= \frac{\frac{2}{L} \int_0^L f_2(x) \sin \left( \frac{ n\pi x}{ L} \right) dx }{\sinh\left(\frac{n\pi H}{L } \right) } &\text{ and } b_n =  \frac{\frac{2}{L} \int_0^L f_1(x) \sin \left( \frac{ n\pi x}{ L} \right) dx }{\sinh\left(\frac{-n\pi H}{L } \right) }\\
	c_n= \frac{\frac{2}{H} \int_0^H g_2(y) \sin \left( \frac{ n\pi y}{ H} \right) dy }{\sinh\left(\frac{n\pi L}{H } \right) } &\text{ and } d_n =  \frac{\frac{2}{H} \int_0^H g_1(y) \sin \left( \frac{ n\pi y}{ H} \right) dy }{\sinh\left(\frac{-n\pi L}{H } \right) }
\end{align*}
\begin{note}[]
	At the corners we have $ u(0,0)=u(L,0)=u(0,H)=u(L,H)=0$.
\end{note}
\begin{intuition}
Recall the formal solution $ u(x,y)$ from above. We can show if  $ 0<a<b \implies \frac{\sinh(na)}{\sinh(nb) }< e^{n(a-b)}$ and $ e^{n(a-b)}<1$. Since $ x<L$ and  $ y<H$, the coefficients are finite integral of quotient of two hyperbolic terms (less than 1) and thus are bounded. We can show convergence along this line of logic.
\end{intuition}

\newpage
\section{Laplace in Circular Geometry}

We now wish to determine the formula for Laplace's equation in terms of the polar coordinates. Let's denote $ u(x,y)=v(r, \theta)$. Suppose $ x=r \cos(\theta ), y=r \sin(\theta )$. Then
\[
	\Delta u = \frac{1}{r} \frac{\partial }{\partial r} \left(r \frac{\partial v}{\partial r} \right) + \frac{1}{r^2} \frac{\partial^2 v}{\partial { \theta}^2}, r>0
.\] 
Using the Chain rule,
\[
\frac{\partial v}{\partial r} = \frac{\partial u}{\partial r} = \frac{\partial u}{\partial x} \cdot \frac{\partial x}{\partial r} + \frac{\partial u}{\partial y} \cdot \frac{\partial y}{\partial r} = u_x \cos(\theta ) + u_y \sin(\theta )
.\] 
and
\[
	\frac{\partial v}{\partial \theta} = \frac{\partial u}{\partial \theta} = u_x(-r \sin\theta) + u_y(r \cos\theta)  
.\] 
Now for the 2nd partials, we use the rules we discovered above:
\begin{align*}
	\frac{\partial^2 v}{\partial { r}^2} &= \frac{\partial }{\partial r} [u_x \cos\theta + u_y \sin\theta] \\
					     &= \frac{\partial }{\partial r} [u_x \cos\theta] + \frac{\partial }{\partial r} [u_y \sin\theta] \\
					     &= [u_{x x}\cos^2\theta + u_{xy} \cos \theta \sin\theta ]+ [u_{yx} \cos\theta \sin\theta + u_{yy}\sin^2\theta] \\
					     &= u_{x x} \cos^2\theta + 2u_{xy}\cos\theta \sin\theta + u_{yy} \sin^2\theta 
\end{align*}
Again doing this for $ \theta$,
\[
	\frac{\partial^2 v}{\partial { \theta}^2} = \frac{\partial }{\partial t} [u_x (-r \sin\theta)] + \frac{\partial }{\partial t} [u_y (r\cos\theta)]
.\] 
Thus by product rule, we eventually obtain
\[
	\frac{\partial^2 v}{\partial { \theta}^2} = u_{x x} r^2 \sin^2\theta - 2u_{yx}r^2 \cos\theta\sin\theta + u_{yy} r^2 \cos^2\theta - r(u_x \cos\theta+ u_y \sin\theta)
.\] 
dividing both sides by $ r^2$ yields:
\[	\frac{1}{r^2} \frac{\partial^2 v}{\partial { \theta}^2} = u_{x x} \sin^2\theta - 2u_{yx} \cos\theta\sin\theta + u_{yy} \cos^2\theta - \frac{1}{r} (u_x \cos\theta+ u_y \sin\theta)
.\]
Therefore, if we add these two second partials together:
\[
v_{rr} + \frac{1}{r^2} \cdot v_{\theta \theta} = u_{x x} + u_{yy} - \frac{1}{r}v_r
.\] 
This gives us
\[
\Delta u = u_{x x}+ u_{yy} = v_{rr} + \frac{1}{r}v_r + \frac{1}{r^2} v_{\theta \theta}
.\] 

\end{document}

\documentclass[class=article,crop=false]{standalone} 
%Fall 2020
% Some basic packages
\usepackage{standalone}[subpreambles=true]
\usepackage[utf8]{inputenc}
\usepackage[T1]{fontenc}
\usepackage{textcomp}
\usepackage[english]{babel}
\usepackage{url}
\usepackage{graphicx}
\usepackage{float}
\usepackage{enumitem}


\pdfminorversion=7

% Don't indent paragraphs, leave some space between them
\usepackage{parskip}

% Hide page number when page is empty
\usepackage{emptypage}
\usepackage{subcaption}
\usepackage{multicol}
\usepackage[dvipsnames]{xcolor}


% Math stuff
\usepackage{amsmath, amsfonts, mathtools, amsthm, amssymb}
% Fancy script capitals
\usepackage{mathrsfs}
\usepackage{cancel}
% Bold math
\usepackage{bm}
% Some shortcuts
\newcommand{\rr}{\ensuremath{\mathbb{R}}}
\newcommand{\zz}{\ensuremath{\mathbb{Z}}}
\newcommand{\qq}{\ensuremath{\mathbb{Q}}}
\newcommand{\nn}{\ensuremath{\mathbb{N}}}
\newcommand{\ff}{\ensuremath{\mathbb{F}}}
\newcommand{\cc}{\ensuremath{\mathbb{C}}}
\renewcommand\O{\ensuremath{\emptyset}}
\newcommand{\norm}[1]{{\left\lVert{#1}\right\rVert}}
\renewcommand{\vec}[1]{{\mathbf{#1}}}
\newcommand\allbold[1]{{\boldmath\textbf{#1}}}

% Put x \to \infty below \lim
\let\svlim\lim\def\lim{\svlim\limits}

%Make implies and impliedby shorter
\let\implies\Rightarrow
\let\impliedby\Leftarrow
\let\iff\Leftrightarrow
\let\epsilon\varepsilon

% Add \contra symbol to denote contradiction
\usepackage{stmaryrd} % for \lightning
\newcommand\contra{\scalebox{1.5}{$\lightning$}}

% \let\phi\varphi

% Command for short corrections
% Usage: 1+1=\correct{3}{2}

\definecolor{correct}{HTML}{009900}
\newcommand\correct[2]{\ensuremath{\:}{\color{red}{#1}}\ensuremath{\to }{\color{correct}{#2}}\ensuremath{\:}}
\newcommand\green[1]{{\color{correct}{#1}}}

% horizontal rule
\newcommand\hr{
    \noindent\rule[0.5ex]{\linewidth}{0.5pt}
}

% hide parts
\newcommand\hide[1]{}

% si unitx
\usepackage{siunitx}
\sisetup{locale = FR}

% Environments
\makeatother
% For box around Definition, Theorem, \ldots
\usepackage[framemethod=TikZ]{mdframed}
\mdfsetup{skipabove=1em,skipbelow=0em}

%definition
\newenvironment{defn}[1][]{%
\ifstrempty{#1}%
{\mdfsetup{%
frametitle={%
\tikz[baseline=(current bounding box.east),outer sep=0pt]
\node[anchor=east,rectangle,fill=Emerald]
{\strut Definition};}}
}%
{\mdfsetup{%
frametitle={%
\tikz[baseline=(current bounding box.east),outer sep=0pt]
\node[anchor=east,rectangle,fill=Emerald]
{\strut Definition:~#1};}}%
}%
\mdfsetup{innertopmargin=10pt,linecolor=Emerald,%
linewidth=2pt,topline=true,%
frametitleaboveskip=\dimexpr-\ht\strutbox\relax
}
\begin{mdframed}[]\relax%
\label{#1}}{\end{mdframed}}


%theorem
%\newcounter{thm}[section]\setcounter{thm}{0}
%\renewcommand{\thethm}{\arabic{section}.\arabic{thm}}
\newenvironment{thm}[1][]{%
%\refstepcounter{thm}%
\ifstrempty{#1}%
{\mdfsetup{%
frametitle={%
\tikz[baseline=(current bounding box.east),outer sep=0pt]
\node[anchor=east,rectangle,fill=blue!20]
%{\strut Theorem~\thethm};}}
{\strut Theorem};}}
}%
{\mdfsetup{%
frametitle={%
\tikz[baseline=(current bounding box.east),outer sep=0pt]
\node[anchor=east,rectangle,fill=blue!20]
%{\strut Theorem~\thethm:~#1};}}%
{\strut Theorem:~#1};}}%
}%
\mdfsetup{innertopmargin=10pt,linecolor=blue!20,%
linewidth=2pt,topline=true,%
frametitleaboveskip=\dimexpr-\ht\strutbox\relax
}
\begin{mdframed}[]\relax%
\label{#1}}{\end{mdframed}}


%lemma
\newenvironment{lem}[1][]{%
\ifstrempty{#1}%
{\mdfsetup{%
frametitle={%
\tikz[baseline=(current bounding box.east),outer sep=0pt]
\node[anchor=east,rectangle,fill=Dandelion]
{\strut Lemma};}}
}%
{\mdfsetup{%
frametitle={%
\tikz[baseline=(current bounding box.east),outer sep=0pt]
\node[anchor=east,rectangle,fill=Dandelion]
{\strut Lemma:~#1};}}%
}%
\mdfsetup{innertopmargin=10pt,linecolor=Dandelion,%
linewidth=2pt,topline=true,%
frametitleaboveskip=\dimexpr-\ht\strutbox\relax
}
\begin{mdframed}[]\relax%
\label{#1}}{\end{mdframed}}

%corollary
\newenvironment{coro}[1][]{%
\ifstrempty{#1}%
{\mdfsetup{%
frametitle={%
\tikz[baseline=(current bounding box.east),outer sep=0pt]
\node[anchor=east,rectangle,fill=CornflowerBlue]
{\strut Corollary};}}
}%
{\mdfsetup{%
frametitle={%
\tikz[baseline=(current bounding box.east),outer sep=0pt]
\node[anchor=east,rectangle,fill=CornflowerBlue]
{\strut Corollary:~#1};}}%
}%
\mdfsetup{innertopmargin=10pt,linecolor=CornflowerBlue,%
linewidth=2pt,topline=true,%
frametitleaboveskip=\dimexpr-\ht\strutbox\relax
}
\begin{mdframed}[]\relax%
\label{#1}}{\end{mdframed}}

%proof
\newenvironment{prf}[1][]{%
\ifstrempty{#1}%
{\mdfsetup{%
frametitle={%
\tikz[baseline=(current bounding box.east),outer sep=0pt]
\node[anchor=east,rectangle,fill=SpringGreen]
{\strut Proof};}}
}%
{\mdfsetup{%
frametitle={%
\tikz[baseline=(current bounding box.east),outer sep=0pt]
\node[anchor=east,rectangle,fill=SpringGreen]
{\strut Proof:~#1};}}%
}%
\mdfsetup{innertopmargin=10pt,linecolor=SpringGreen,%
linewidth=2pt,topline=true,%
frametitleaboveskip=\dimexpr-\ht\strutbox\relax
}
\begin{mdframed}[]\relax%
\label{#1}}{\qed\end{mdframed}}


\theoremstyle{definition}

\newmdtheoremenv[nobreak=true]{definition}{Definition}
\newmdtheoremenv[nobreak=true]{prop}{Proposition}
\newmdtheoremenv[nobreak=true]{theorem}{Theorem}
\newmdtheoremenv[nobreak=true]{corollary}{Corollary}
\newtheorem*{eg}{Example}
\theoremstyle{remark}
\newtheorem*{case}{Case}
\newtheorem*{notation}{Notation}
\newtheorem*{remark}{Remark}
\newtheorem*{note}{Note}
\newtheorem*{problem}{Problem}
\newtheorem*{observe}{Observe}
\newtheorem*{property}{Property}
\newtheorem*{intuition}{Intuition}


% End example and intermezzo environments with a small diamond (just like proof
% environments end with a small square)
\usepackage{etoolbox}
\AtEndEnvironment{vb}{\null\hfill$\diamond$}%
\AtEndEnvironment{intermezzo}{\null\hfill$\diamond$}%
% \AtEndEnvironment{opmerking}{\null\hfill$\diamond$}%

% Fix some spacing
% http://tex.stackexchange.com/questions/22119/how-can-i-change-the-spacing-before-theorems-with-amsthm
\makeatletter
\def\thm@space@setup{%
  \thm@preskip=\parskip \thm@postskip=0pt
}

% Fix some stuff
% %http://tex.stackexchange.com/questions/76273/multiple-pdfs-with-page-group-included-in-a-single-page-warning
\pdfsuppresswarningpagegroup=1


% My name
\author{Jaden Wang}



\begin{document}
\begin{intuition}
	If the rope is tight enough, \emph{i.e.} $ T_0$ is large, then there will be barely any oscillation. It's some sort of restorative force.
\end{intuition} 

Recall the slope of the string may be represented as $ \frac{\partial u}{\partial x} $ or $ \tan \theta$. 
\[
\tan \theta = \frac{\partial u}{\partial x} 
.\]
Thus
\[
\frac{\partial }{\partial x} \tan \theta = \frac{\partial }{\partial x} \cdot  \frac{\partial u}{\partial x} = \frac{\partial^2 u}{\partial { x}^2} 
.\] 
Then the PDE becomes
\[
\frac{\partial^2 u}{\partial { t}^2} = \frac{T_0}{\rho_0} \cdot  \frac{\partial^2 u}{\partial { x}^2} -g
.\] 
Assuming $ g=0$, let  $ c^2 = \frac{T_0}{\rho_0} = \frac{T_0}{\delta \cdot A} > 0$, then we obtain the 1D wave equation
\begin{thm}[1D wave equation]
\[
\frac{\partial^2 u}{\partial { t}^2} = c^2 \frac{\partial^2 u}{\partial { x}^2} 
.\]
The more general version is
\[
	\frac{\partial^2 u}{\partial { t}^2} = c^2 \frac{\partial^2 u}{\partial { x}^2} - g + f(x,t)
.\]

\end{thm}

The LHS is the vertical acceleration of string, and the RHS is the restoring force due to tension (- gravitational acceleration + acceleration from known external forces). 

\subsection{initial conditions}
Second order time derivative needs two ICs!

Initial position: $ u(x,0)= U(x)$.

Initial velocity:  $ \frac{\partial }{\partial t} u(x,0) = V(x)$.

The boundary conditions are fixed: $ u(x,0)=u(x,L)=0$. 

So we have the following:

\begin{equation*}
\begin{cases}
	\text{PDE: } \frac{\partial^2 u}{\partial { t}^2} = c^2 \frac{\partial^2 u}{\partial { x}^2}  & 0<x<L, t>0 \\
	\text{BCs: } u(0,t) = 0 = u(L,t) & t>0\\
	\text{ICs: } u(x,0)=U(x), \frac{\partial u}{\partial t} (x,0) =V(x) & 0\leq x \leq L\\
\end{cases}
\end{equation*}

In general, we assume that the solution has the form $ u(x,t) = \overline{u}(x) + w(x,t)$, where $ \overline{u}(x)$ is the steady state solution and $ w(x,t)$ is in the vector space of functions that satisfy the PDE and BCs.

\begin{eg}[]
Suppose
\[
\frac{\partial^2 u}{\partial { t}^2} = \frac{T_0}{\rho_0} \cdot \frac{\partial^2 u}{\partial { x}^2} - g
.\] 
with $ u(0,t)=0=u(L,t)$. Find the steady state position of the string.

For steady state, $ \frac{\partial u}{\partial t} =0$, so $ \frac{\partial^2 u}{\partial { t}^2}=0 $, and
\[
	\frac{T_0}{\rho_0} \overline{u}''(x) - g=0 \implies \overline{u}(x) = \frac{\rho_0 g}{2 T_0 }x^2 + A x + B
.\] 
Then BC $ \overline{u}(0)=0$ implies $ B =0$. And the BC $ \overline{u}(L)=0$ implies
\[
	0=\overline{u}(L) = \frac{\rho_0 g}{2 T_0 } L^2 + A \cdot L \implies A = \frac{\rho_0 g}{2T_0 }L
.\] 
therefore the steady state solution becomes
\[
	\overline{u}(x) = \frac{\rho_0 g}{2 T_0 } \cdot x \cdot (x-L)
.\] 
Therefore, the graph is a parabola with minimum occurring when $ x=\frac{L}{2}$, then the minimum $ \overline{u}_{\min} = \overline{u}\left( \frac{L}{2} \right)  = \frac{\rho_0 g}{2 T_0 }L^2$. This is the sag in the center of the string due to gravity. 
\end{eg}

\begin{intuition}
	Let's take a look at the units of $ c = \sqrt{\frac{T_0}{\rho_0}} $. 
	\[
		c = \sqrt{ \text{ mass} \cdot \text{ length} / \text{time}^2 \cdot \text{ length} / \text{ mass}    } = \text{ speed} 
	.\] 
So $ c$ should be some sort of speed. The only likely candidate seems to be the speed of propagation. 
If that's true, then we can also calculate propagation time,
\[
\tau = \frac{L}{c} + \frac{L}{c} = \frac{2L}{c }
.\] 

Then frequency (cycles divided by time) follows:
\[
f = \text{ constant} \cdot  \frac{c}{L} = \text{ constant}\frac{1}{2 } \cdot \frac{1}{L} \cdot \sqrt{\frac{T_0}{\delta \cdot A}}   
.\] 

How can we increase the frequency?

\begin{enumerate}[label=\arabic*)]
	\item Decrease length $ L$.
	\item Increase tension  $ T_0$.
	\item Decrease density $ \delta$.
	\item Increase area $ A$.
\end{enumerate}
They all match our intuition!

\begin{note}[]
Gravity doesn't change frequency according to this model!
\end{note}
\end{intuition}

\subsection{Variations on the Wave Equation (optional)} 
Transverse Motion: stiffness.

Damped Motion: eventually return to stationary.

Transverse Vibrations: 2D membrane (drum).

Longitudinal Vibration: does not assume totally vertical motion.

\end{document}

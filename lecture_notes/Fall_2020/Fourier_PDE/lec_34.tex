\documentclass[class=article,crop=false]{standalone} 
%Fall 2020
% Some basic packages
\usepackage{standalone}[subpreambles=true]
\usepackage[utf8]{inputenc}
\usepackage[T1]{fontenc}
\usepackage{textcomp}
\usepackage[english]{babel}
\usepackage{url}
\usepackage{graphicx}
\usepackage{float}
\usepackage{enumitem}


\pdfminorversion=7

% Don't indent paragraphs, leave some space between them
\usepackage{parskip}

% Hide page number when page is empty
\usepackage{emptypage}
\usepackage{subcaption}
\usepackage{multicol}
\usepackage[dvipsnames]{xcolor}


% Math stuff
\usepackage{amsmath, amsfonts, mathtools, amsthm, amssymb}
% Fancy script capitals
\usepackage{mathrsfs}
\usepackage{cancel}
% Bold math
\usepackage{bm}
% Some shortcuts
\newcommand{\rr}{\ensuremath{\mathbb{R}}}
\newcommand{\zz}{\ensuremath{\mathbb{Z}}}
\newcommand{\qq}{\ensuremath{\mathbb{Q}}}
\newcommand{\nn}{\ensuremath{\mathbb{N}}}
\newcommand{\ff}{\ensuremath{\mathbb{F}}}
\newcommand{\cc}{\ensuremath{\mathbb{C}}}
\renewcommand\O{\ensuremath{\emptyset}}
\newcommand{\norm}[1]{{\left\lVert{#1}\right\rVert}}
\renewcommand{\vec}[1]{{\mathbf{#1}}}
\newcommand\allbold[1]{{\boldmath\textbf{#1}}}

% Put x \to \infty below \lim
\let\svlim\lim\def\lim{\svlim\limits}

%Make implies and impliedby shorter
\let\implies\Rightarrow
\let\impliedby\Leftarrow
\let\iff\Leftrightarrow
\let\epsilon\varepsilon

% Add \contra symbol to denote contradiction
\usepackage{stmaryrd} % for \lightning
\newcommand\contra{\scalebox{1.5}{$\lightning$}}

% \let\phi\varphi

% Command for short corrections
% Usage: 1+1=\correct{3}{2}

\definecolor{correct}{HTML}{009900}
\newcommand\correct[2]{\ensuremath{\:}{\color{red}{#1}}\ensuremath{\to }{\color{correct}{#2}}\ensuremath{\:}}
\newcommand\green[1]{{\color{correct}{#1}}}

% horizontal rule
\newcommand\hr{
    \noindent\rule[0.5ex]{\linewidth}{0.5pt}
}

% hide parts
\newcommand\hide[1]{}

% si unitx
\usepackage{siunitx}
\sisetup{locale = FR}

% Environments
\makeatother
% For box around Definition, Theorem, \ldots
\usepackage[framemethod=TikZ]{mdframed}
\mdfsetup{skipabove=1em,skipbelow=0em}

%definition
\newenvironment{defn}[1][]{%
\ifstrempty{#1}%
{\mdfsetup{%
frametitle={%
\tikz[baseline=(current bounding box.east),outer sep=0pt]
\node[anchor=east,rectangle,fill=Emerald]
{\strut Definition};}}
}%
{\mdfsetup{%
frametitle={%
\tikz[baseline=(current bounding box.east),outer sep=0pt]
\node[anchor=east,rectangle,fill=Emerald]
{\strut Definition:~#1};}}%
}%
\mdfsetup{innertopmargin=10pt,linecolor=Emerald,%
linewidth=2pt,topline=true,%
frametitleaboveskip=\dimexpr-\ht\strutbox\relax
}
\begin{mdframed}[]\relax%
\label{#1}}{\end{mdframed}}


%theorem
%\newcounter{thm}[section]\setcounter{thm}{0}
%\renewcommand{\thethm}{\arabic{section}.\arabic{thm}}
\newenvironment{thm}[1][]{%
%\refstepcounter{thm}%
\ifstrempty{#1}%
{\mdfsetup{%
frametitle={%
\tikz[baseline=(current bounding box.east),outer sep=0pt]
\node[anchor=east,rectangle,fill=blue!20]
%{\strut Theorem~\thethm};}}
{\strut Theorem};}}
}%
{\mdfsetup{%
frametitle={%
\tikz[baseline=(current bounding box.east),outer sep=0pt]
\node[anchor=east,rectangle,fill=blue!20]
%{\strut Theorem~\thethm:~#1};}}%
{\strut Theorem:~#1};}}%
}%
\mdfsetup{innertopmargin=10pt,linecolor=blue!20,%
linewidth=2pt,topline=true,%
frametitleaboveskip=\dimexpr-\ht\strutbox\relax
}
\begin{mdframed}[]\relax%
\label{#1}}{\end{mdframed}}


%lemma
\newenvironment{lem}[1][]{%
\ifstrempty{#1}%
{\mdfsetup{%
frametitle={%
\tikz[baseline=(current bounding box.east),outer sep=0pt]
\node[anchor=east,rectangle,fill=Dandelion]
{\strut Lemma};}}
}%
{\mdfsetup{%
frametitle={%
\tikz[baseline=(current bounding box.east),outer sep=0pt]
\node[anchor=east,rectangle,fill=Dandelion]
{\strut Lemma:~#1};}}%
}%
\mdfsetup{innertopmargin=10pt,linecolor=Dandelion,%
linewidth=2pt,topline=true,%
frametitleaboveskip=\dimexpr-\ht\strutbox\relax
}
\begin{mdframed}[]\relax%
\label{#1}}{\end{mdframed}}

%corollary
\newenvironment{coro}[1][]{%
\ifstrempty{#1}%
{\mdfsetup{%
frametitle={%
\tikz[baseline=(current bounding box.east),outer sep=0pt]
\node[anchor=east,rectangle,fill=CornflowerBlue]
{\strut Corollary};}}
}%
{\mdfsetup{%
frametitle={%
\tikz[baseline=(current bounding box.east),outer sep=0pt]
\node[anchor=east,rectangle,fill=CornflowerBlue]
{\strut Corollary:~#1};}}%
}%
\mdfsetup{innertopmargin=10pt,linecolor=CornflowerBlue,%
linewidth=2pt,topline=true,%
frametitleaboveskip=\dimexpr-\ht\strutbox\relax
}
\begin{mdframed}[]\relax%
\label{#1}}{\end{mdframed}}

%proof
\newenvironment{prf}[1][]{%
\ifstrempty{#1}%
{\mdfsetup{%
frametitle={%
\tikz[baseline=(current bounding box.east),outer sep=0pt]
\node[anchor=east,rectangle,fill=SpringGreen]
{\strut Proof};}}
}%
{\mdfsetup{%
frametitle={%
\tikz[baseline=(current bounding box.east),outer sep=0pt]
\node[anchor=east,rectangle,fill=SpringGreen]
{\strut Proof:~#1};}}%
}%
\mdfsetup{innertopmargin=10pt,linecolor=SpringGreen,%
linewidth=2pt,topline=true,%
frametitleaboveskip=\dimexpr-\ht\strutbox\relax
}
\begin{mdframed}[]\relax%
\label{#1}}{\qed\end{mdframed}}


\theoremstyle{definition}

\newmdtheoremenv[nobreak=true]{definition}{Definition}
\newmdtheoremenv[nobreak=true]{prop}{Proposition}
\newmdtheoremenv[nobreak=true]{theorem}{Theorem}
\newmdtheoremenv[nobreak=true]{corollary}{Corollary}
\newtheorem*{eg}{Example}
\theoremstyle{remark}
\newtheorem*{case}{Case}
\newtheorem*{notation}{Notation}
\newtheorem*{remark}{Remark}
\newtheorem*{note}{Note}
\newtheorem*{problem}{Problem}
\newtheorem*{observe}{Observe}
\newtheorem*{property}{Property}
\newtheorem*{intuition}{Intuition}


% End example and intermezzo environments with a small diamond (just like proof
% environments end with a small square)
\usepackage{etoolbox}
\AtEndEnvironment{vb}{\null\hfill$\diamond$}%
\AtEndEnvironment{intermezzo}{\null\hfill$\diamond$}%
% \AtEndEnvironment{opmerking}{\null\hfill$\diamond$}%

% Fix some spacing
% http://tex.stackexchange.com/questions/22119/how-can-i-change-the-spacing-before-theorems-with-amsthm
\makeatletter
\def\thm@space@setup{%
  \thm@preskip=\parskip \thm@postskip=0pt
}

% Fix some stuff
% %http://tex.stackexchange.com/questions/76273/multiple-pdfs-with-page-group-included-in-a-single-page-warning
\pdfsuppresswarningpagegroup=1


% My name
\author{Jaden Wang}



\begin{document}


\subsection{Asymptotic Behavior}

Recall $ I = \int_{-\infty}^{\infty} e^{-x^2} \ dx = \sqrt{\pi}  $. Then for all $ x$ and  $ t>0$,
 \begin{align*}
	 |u(x,t)|&\leq \int_{-\infty}^{\infty} | \hat{ U}(m)| \cdot e^{-m^2kt} \cdot  |e^{imx}|\ dm \\
		 &\leq \frac{M}{2\pi} \int_{-\infty}^{\infty} e^{-m^2kt} \ d m \\ 
		 &= \frac{M}{2\pi} \int_{-\infty}^{\infty} e^{-(m\sqrt{kt})^2 } \ d m \\
		 &= \frac{M}{2\pi} \cdot \frac{\sqrt{\pi} }{\sqrt{kt}  } 
\end{align*}
Therefore, $ u(x,t) \to 0$ as $ t \to \infty$.

\subsection{Large time estimate}

Note that since $-\infty<m< \infty $, there is no smallest Fourier mode, instead we expand $ \hat{ U}(m)$ as a Maclaurin Series (assuming it can be done) then
\[
	\hat{ U}(m) = \hat{ U}(0) + \hat{ U}'(0)m + \frac{\hat{ U}''(0)}{2! } m^2 + \ldots
.\]
By completing the square, we can show that
\[
	e^{-m^2kt} e^{imx} = e^{-kt(m-ix / 2kt)^2} e^{-x^2 /4kt}
.\] 
So we can approximate
\begin{align*}
	u(x,t) &\approx \int_{-\infty}^{\infty} \hat{ U}(0) e^{imx}e^{-m^2kt}\ d m \\
	       &= \hat{ U}(0) e^{-x^2 / 4kt} \int_{-\infty}^{\infty} e^{-kt(m-ix /2kt)^2} \ d m \\
	       &= \hat{ U}(0) e^{-x^2 / 4kt} \cdot \frac{\sqrt{\pi} }{\sqrt{kt}  }
\end{align*}

So for large but finite $ t$, we have
 \[
	 u(x,t) \approx \hat{ U}(0) e^{-x^2 / 4kt} \cdot \frac{\sqrt{\pi} }{\sqrt{kt}  }
.\]
Now we can motivate the fundamental source solution:
\[
	\sqrt{kt} \cdot u(x,t) \approx \hat{ U}(0) \sqrt{\pi} \cdot e^{-x^2 /4kt}  
.\]
So for each fixed $ t$,  $ \sqrt{kt} \cdot u(x,t)$ has a bell shape and will have a bell shape for almost initial data as $ kt \to \infty$. Using the Maclaurin Series approximation of $ u(x,t)$, we can show  $ u(x,t) \to 0$ as $ |x|\to \infty$.

\begin{defn}[Dirac delta function]
	Define the \allbold{Dirac delta function} to be the concentrated pulse "function" $ \delta(x)$ with the property that
	\[
		\int_{-\infty}^{\infty} \delta(x-c) f(x) dx = f(c) = \int_{-\infty}^{\infty} \delta(c-x) f(x) dx, c \in \rr  
	.\] 
	That is, formally define
	\begin{equation*}
		\delta(x-c)=
	\begin{cases}
		0,& \text{ if } x\neq c\\
		\infty,& \text{ if } x=c\\ 
	\end{cases}
	\end{equation*}
\end{defn}
\begin{note}[]
	If we let $ f(x) =1$ and  $ c=0$, then  $ \int_{-\infty}^{\infty} \delta(x) dx =1 $. The integrals above cannot be defined as limits of Riemann sums because $ \delta(x)$ is no ordinary function. The integral statement above is true by definition.
\end{note}

\begin{thm}[fundamental solution of the heat equation]
Given the problem
\begin{equation*}
\begin{cases}
	\text{PDE: } \frac{\partial u}{\partial t} =k\frac{\partial^2 u}{\partial { x}^2}  & -\infty<x<\infty, t>0 \\
	\text{ICs: } u(x,0) = \delta(x) & \\
\end{cases}
\end{equation*}
Then we can show that the fundamental solution is
\begin{align*}
	u(x,t) &= \int_{-\infty}^{\infty} \hat{ \delta}(m) e^{imx} e^{-m^2 kt}  dm \\
	       &= \int_{-\infty}^{\infty} \left[ \frac{1}{2\pi} \int_{-\infty}^{\infty} \delta( \overline{x}) e^{-im \overline{x}} d \overline{x}  \right]  e^{imx} e^{-m^2 kt} d m \\
	       &= \int_{-\infty}^{\infty} \frac{1}{2\pi} e^{-im \cdot  0} e^{imx} e^{-m^2 kt} d m \\
	       &= \frac{1}{\sqrt{4\pi kt} } e^{-x^2 /4kt} \text{ by above approximation} \\
\end{align*} 
\end{thm}
\begin{remark}
This solution represents the evolution of the temperature due to an initial heat source at $ x=0, t=0$ for an infinite rod and the temperature has a Gaussian distribution.
\end{remark}
\section{Finite Rod}
Consider the finite interval problem:
\begin{equation*}
\begin{cases}
	\text{PDE: } \frac{\partial u}{\partial t} = k \frac{\partial^2 u}{\partial { x}^2}  & 0<x<L, t>0 \\
	\text{BCs: } u(0,t)=0=u(L,t) & t>0 \\
	\text{ICs: } u(x,0) =f(x) & 0 \leq x \leq L \\
\end{cases}
\end{equation*}
where $ f(x)$ is continuous and  $ f(0)=f(L)=0$. Note that the DE is not required to hold when  $ t=0$ (so  $ f(x)$ is not required to be twice differentiable). However,  $ u(x,t)$ is required to be continuous for  $ (x,t) \in [0,L]\times [0,\infty)$. That is,
\[
	\lim_{ (x,t) \to (x_0,0^+} u(x,t) = u(x_0,0) = f(x_0) , x_0 \in \rr 
.\] 
\begin{thm}[heat equation by method of images]
	Let $ \tilde{f}_{odd}(x), -\infty< x < \infty$, be the periodic extension of the odd extension of $ f(x)$. Then the unique solution of problem above, which is continuous for all  $(x,t) \in (-\infty, \infty) \times [0,\infty)$, is
	\begin{equation*}
		u(x,t)=
	\begin{cases}
		\frac{1}{\sqrt{4\pi k t} } \int_{-\infty}^{\infty} e^{-(x-\overline{x})^2 /4kt} \tilde{f}_{odd}(\overline{x}) d\overline{x}, & \text{ if } t>0\\
		f(x), & \text{ if } t=0\\ 
	\end{cases}
	\end{equation*}
\end{thm}

\begin{claim}[]
The fundamental source solution and the F.S. solution to the heat equation are the same.
\end{claim}
\begin{prf}
	Since $ u(x,t) \in \mathcal{ C}^{\infty}$ odd periodic function of $ x$,  $ t>0$. Thus, fundamental source solution is equal to its F.S.S on $ [0,L], t>0$. That is,
	\[
		u(x,t) = \sum_{ n= 1}^{\infty} B_n(t) \sin \left( \frac{ n\pi x}{ L} \right) , B_n(t) = \frac{2}{L} \int_{0}^{L} u(x,t) \sin \left( \frac{ n\pi x}{ L} \right) dx 
	.\] 
Using integration by parts, we move derivative of $ f$ wrt $ x$ to derivative of  $ g$ wrt  $ x$. 
	\begin{align*}
	\text{ In general,} 	\int_{ a}^{ b} \frac{df}{dx} g(x) dx &= g(x) f(x) \bigg |_{a}^b - \int_{ a}^{ b} f(x) \cdot \frac{dg}{dx} dx \\
	\text{ In our case, } 	\frac{\partial u}{\partial t} &= \sum_{ n= 1}^{\infty} \frac{\partial }{\partial t} B_n(t) \sin \left( \frac{ n\pi x}{ L} \right)  \\
							      &= \sum_{ n= 1}^{\infty} \frac{2}{L} \left[\int_{0}^{L} u_t(x,t) \sin \left( \frac{ n\pi x}{ L} \right) dx \right] \sin \left( \frac{ n\pi x}{ L} \right)   \\
							      &= \sum_{ n= 1}^{\infty} \frac{2}{L} \left[ \int_{0}^{L} k u_{xx}(x,t) \sin \left( \frac{ n\pi x}{ L} \right) dx \right] \sin \left( \frac{ n\pi x}{ L} \right)   \\
							      &= \sum_{ n= 1}^{\infty} \frac{2}{L}\left[ \int_{0}^{L} k u(x,t) \frac{d^2 }{d x^2} \left( \sin \left( \frac{ n\pi x}{ L} \right) \right) dx \right] \sin \left( \frac{ n\pi x}{ L} \right)   	
	\end{align*}
	After differentiation we can show that $ B_n(t)$ satisfies the following ODE:
	\[
		B_n(t) = -k\left( \frac{n \pi}{L } \right)^2 B_n'(t) \implies B_n(t) = c_n e^{-(n\pi /L})^2 kt 
	.\] 
	We can finally show that $ c_n = b_n$ and thus complete the proof.
\end{prf}

\begin{thm}[transform method]
	Suppose $ f(x) \in \mathcal{ C}^1$ and suppose $ |f(x)|+ |f'(x)| \leq K |x|^{-2}$, then
	\[
		\frac{\wh{ df}}{dx }(m) = im \cdot \wh{ f}(m) \iff \mathcal{ F} [f'(x)] = im \cdot \mathcal{ F}[f(x)].
	\]
\end{thm}
\begin{prf}
	Use integration by parts on $ \mathcal{ F}[f'(x)] =\frac{1}{2\pi} \int_{-\infty}^{\infty} f'(x) e^{-imx} dx$ and the fact that $ \lim_{ x \to \pm \infty} f(x) = 0$ to prove it. 
\end{prf}

\begin{eg}[]
	Take the Fourier transform of both sides of the heat equation wrt $ x$, fix  $ t$, yields:
	 \[
		 \frac{1}{2\pi} \int_{-\infty}^{\infty} u_t e^{-imx} dx = \frac{1}{2\pi} \int_{-\infty}^{\infty} k u_{x x} e^{-imx} dx \implies \frac{\partial }{\partial t} \wh{ u}(m,t) = k \wh{ u_{xx}}(m,t)  
	.\] 
	Apply the property above and we have
\[
	\frac{\partial }{\partial t} \hat{ u}(m,t) = k (im)^2 \hat{ u}(m,t) \implies \frac{\partial }{\partial t} \hat{ u}(m,t) = -km^2 \hat{ u}(m,t)
.\] 
Thus the PDE has been essentially transformed into an ODE (because $ x$ doesn't affect the frequency domain). It can be shown that the solution is  $ \hat{ u}(m,t) = \hat{ f}(m)e^{-km^2t}$ which implies that $ u(x,t) = \mathcal{ F}^{-1}[\hat{ u}(m,t)]$ is equal to the Fourier solution.
\end{eg}
\end{document}

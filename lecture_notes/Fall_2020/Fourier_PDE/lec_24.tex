\documentclass[class=article,crop=false]{standalone} 
%Fall 2020
% Some basic packages
\usepackage{standalone}[subpreambles=true]
\usepackage[utf8]{inputenc}
\usepackage[T1]{fontenc}
\usepackage{textcomp}
\usepackage[english]{babel}
\usepackage{url}
\usepackage{graphicx}
\usepackage{float}
\usepackage{enumitem}


\pdfminorversion=7

% Don't indent paragraphs, leave some space between them
\usepackage{parskip}

% Hide page number when page is empty
\usepackage{emptypage}
\usepackage{subcaption}
\usepackage{multicol}
\usepackage[dvipsnames]{xcolor}


% Math stuff
\usepackage{amsmath, amsfonts, mathtools, amsthm, amssymb}
% Fancy script capitals
\usepackage{mathrsfs}
\usepackage{cancel}
% Bold math
\usepackage{bm}
% Some shortcuts
\newcommand{\rr}{\ensuremath{\mathbb{R}}}
\newcommand{\zz}{\ensuremath{\mathbb{Z}}}
\newcommand{\qq}{\ensuremath{\mathbb{Q}}}
\newcommand{\nn}{\ensuremath{\mathbb{N}}}
\newcommand{\ff}{\ensuremath{\mathbb{F}}}
\newcommand{\cc}{\ensuremath{\mathbb{C}}}
\renewcommand\O{\ensuremath{\emptyset}}
\newcommand{\norm}[1]{{\left\lVert{#1}\right\rVert}}
\renewcommand{\vec}[1]{{\mathbf{#1}}}
\newcommand\allbold[1]{{\boldmath\textbf{#1}}}

% Put x \to \infty below \lim
\let\svlim\lim\def\lim{\svlim\limits}

%Make implies and impliedby shorter
\let\implies\Rightarrow
\let\impliedby\Leftarrow
\let\iff\Leftrightarrow
\let\epsilon\varepsilon

% Add \contra symbol to denote contradiction
\usepackage{stmaryrd} % for \lightning
\newcommand\contra{\scalebox{1.5}{$\lightning$}}

% \let\phi\varphi

% Command for short corrections
% Usage: 1+1=\correct{3}{2}

\definecolor{correct}{HTML}{009900}
\newcommand\correct[2]{\ensuremath{\:}{\color{red}{#1}}\ensuremath{\to }{\color{correct}{#2}}\ensuremath{\:}}
\newcommand\green[1]{{\color{correct}{#1}}}

% horizontal rule
\newcommand\hr{
    \noindent\rule[0.5ex]{\linewidth}{0.5pt}
}

% hide parts
\newcommand\hide[1]{}

% si unitx
\usepackage{siunitx}
\sisetup{locale = FR}

% Environments
\makeatother
% For box around Definition, Theorem, \ldots
\usepackage[framemethod=TikZ]{mdframed}
\mdfsetup{skipabove=1em,skipbelow=0em}

%definition
\newenvironment{defn}[1][]{%
\ifstrempty{#1}%
{\mdfsetup{%
frametitle={%
\tikz[baseline=(current bounding box.east),outer sep=0pt]
\node[anchor=east,rectangle,fill=Emerald]
{\strut Definition};}}
}%
{\mdfsetup{%
frametitle={%
\tikz[baseline=(current bounding box.east),outer sep=0pt]
\node[anchor=east,rectangle,fill=Emerald]
{\strut Definition:~#1};}}%
}%
\mdfsetup{innertopmargin=10pt,linecolor=Emerald,%
linewidth=2pt,topline=true,%
frametitleaboveskip=\dimexpr-\ht\strutbox\relax
}
\begin{mdframed}[]\relax%
\label{#1}}{\end{mdframed}}


%theorem
%\newcounter{thm}[section]\setcounter{thm}{0}
%\renewcommand{\thethm}{\arabic{section}.\arabic{thm}}
\newenvironment{thm}[1][]{%
%\refstepcounter{thm}%
\ifstrempty{#1}%
{\mdfsetup{%
frametitle={%
\tikz[baseline=(current bounding box.east),outer sep=0pt]
\node[anchor=east,rectangle,fill=blue!20]
%{\strut Theorem~\thethm};}}
{\strut Theorem};}}
}%
{\mdfsetup{%
frametitle={%
\tikz[baseline=(current bounding box.east),outer sep=0pt]
\node[anchor=east,rectangle,fill=blue!20]
%{\strut Theorem~\thethm:~#1};}}%
{\strut Theorem:~#1};}}%
}%
\mdfsetup{innertopmargin=10pt,linecolor=blue!20,%
linewidth=2pt,topline=true,%
frametitleaboveskip=\dimexpr-\ht\strutbox\relax
}
\begin{mdframed}[]\relax%
\label{#1}}{\end{mdframed}}


%lemma
\newenvironment{lem}[1][]{%
\ifstrempty{#1}%
{\mdfsetup{%
frametitle={%
\tikz[baseline=(current bounding box.east),outer sep=0pt]
\node[anchor=east,rectangle,fill=Dandelion]
{\strut Lemma};}}
}%
{\mdfsetup{%
frametitle={%
\tikz[baseline=(current bounding box.east),outer sep=0pt]
\node[anchor=east,rectangle,fill=Dandelion]
{\strut Lemma:~#1};}}%
}%
\mdfsetup{innertopmargin=10pt,linecolor=Dandelion,%
linewidth=2pt,topline=true,%
frametitleaboveskip=\dimexpr-\ht\strutbox\relax
}
\begin{mdframed}[]\relax%
\label{#1}}{\end{mdframed}}

%corollary
\newenvironment{coro}[1][]{%
\ifstrempty{#1}%
{\mdfsetup{%
frametitle={%
\tikz[baseline=(current bounding box.east),outer sep=0pt]
\node[anchor=east,rectangle,fill=CornflowerBlue]
{\strut Corollary};}}
}%
{\mdfsetup{%
frametitle={%
\tikz[baseline=(current bounding box.east),outer sep=0pt]
\node[anchor=east,rectangle,fill=CornflowerBlue]
{\strut Corollary:~#1};}}%
}%
\mdfsetup{innertopmargin=10pt,linecolor=CornflowerBlue,%
linewidth=2pt,topline=true,%
frametitleaboveskip=\dimexpr-\ht\strutbox\relax
}
\begin{mdframed}[]\relax%
\label{#1}}{\end{mdframed}}

%proof
\newenvironment{prf}[1][]{%
\ifstrempty{#1}%
{\mdfsetup{%
frametitle={%
\tikz[baseline=(current bounding box.east),outer sep=0pt]
\node[anchor=east,rectangle,fill=SpringGreen]
{\strut Proof};}}
}%
{\mdfsetup{%
frametitle={%
\tikz[baseline=(current bounding box.east),outer sep=0pt]
\node[anchor=east,rectangle,fill=SpringGreen]
{\strut Proof:~#1};}}%
}%
\mdfsetup{innertopmargin=10pt,linecolor=SpringGreen,%
linewidth=2pt,topline=true,%
frametitleaboveskip=\dimexpr-\ht\strutbox\relax
}
\begin{mdframed}[]\relax%
\label{#1}}{\qed\end{mdframed}}


\theoremstyle{definition}

\newmdtheoremenv[nobreak=true]{definition}{Definition}
\newmdtheoremenv[nobreak=true]{prop}{Proposition}
\newmdtheoremenv[nobreak=true]{theorem}{Theorem}
\newmdtheoremenv[nobreak=true]{corollary}{Corollary}
\newtheorem*{eg}{Example}
\theoremstyle{remark}
\newtheorem*{case}{Case}
\newtheorem*{notation}{Notation}
\newtheorem*{remark}{Remark}
\newtheorem*{note}{Note}
\newtheorem*{problem}{Problem}
\newtheorem*{observe}{Observe}
\newtheorem*{property}{Property}
\newtheorem*{intuition}{Intuition}


% End example and intermezzo environments with a small diamond (just like proof
% environments end with a small square)
\usepackage{etoolbox}
\AtEndEnvironment{vb}{\null\hfill$\diamond$}%
\AtEndEnvironment{intermezzo}{\null\hfill$\diamond$}%
% \AtEndEnvironment{opmerking}{\null\hfill$\diamond$}%

% Fix some spacing
% http://tex.stackexchange.com/questions/22119/how-can-i-change-the-spacing-before-theorems-with-amsthm
\makeatletter
\def\thm@space@setup{%
  \thm@preskip=\parskip \thm@postskip=0pt
}

% Fix some stuff
% %http://tex.stackexchange.com/questions/76273/multiple-pdfs-with-page-group-included-in-a-single-page-warning
\pdfsuppresswarningpagegroup=1


% My name
\author{Jaden Wang}



\begin{document}
How large is the signal propagation speed $ c =\sqrt{\frac{T_0}{\delta A}} $?

The time required to complete one cycle, $ T_1 = \frac{2L}{c}$. So
\[
\omega_1 = \frac{1}{T_1} = \frac{c}{2L} \implies c = 2L \omega_1 = 106.9 \text{ m/sec} 
.\]
Then for $ n$th harmonics,  $c = n \cdot \omega_1 = n \cdot 106.9 $ m/sec.

\newpage
\section{d'Alembert's solution to the wave equation}

We want to use trigonometric identity to transform the solution.

Recall
\[
	\sin(a )\cos(b ) = \frac{1}{2}[\sin(a+b )+ \sin(a-b )] \text{ and } \sin(a )\sin(b ) = \frac{1}{2} [\cos(a-b) -\cos(a+b ) ] 
.\] 
Applying this to the solution and we obtain
\begin{align*}
	& \qquad A_n \sin \left( \frac{ n\pi x}{ L} \right) \cos \left( \frac{ n\pi c t}{ L} \right) + B_n \sin \left( \frac{ n\pi x}{ L} \right) \sin \left( \frac{ n\pi c t}{ L} \right) \\
	&= \frac{A_n}{2} \left( \sin\left( \frac{n\pi}{L }(x+ct) \right) + \sin\left( \frac{n\pi}{L} (x-ct) \right)  \right) + \frac{B_n}{2} \left( \cos\left( \frac{n\pi}{L }(x-ct) \right) - \cos\left( \frac{n\pi}{L }(x+ct) \right)   \right)  \\
																						  &= f(x+ct) + g(x-ct) 
\end{align*}

Let's introduce the natural variables
\[
y=x+ct \text{ and } z=x-ct \text{ for } c>0 \implies x=\frac{y+z}{2 }, t=\frac{y-z}{2c }
.\] 
Note that any point $ (x,t) \in \rr\times [0,\infty)$ can be written as a point $ (y,z) \in \rr^2$.
\begin{intuition}
	$ y,z$ represent two lines in the  $ x,t$ plane and their intersection gives us  $ (x,t)$.
\end{intuition}
We want to solve the wave equation in terms of these new variables so we need to write the wave equation in terms of $ (y,z)$. Note this is not $ u(y,z)$, as we can see from the counterexample below.
\begin{eg}[]
	Let $ u(x,t) = x \cdot t$. Then
	\begin{align*}
		u(x,t) &= \left( \frac{y+z}{2 } \right) \left( \frac{y-z}{2c } \right)   \\
		&= \frac{y^2 - z^2}{4c } \\
		&\neq u(y,z) = yz  \\
	\end{align*}
\end{eg}

Rewrite $ u(x,t)= v(y,z)$, then by Chain Rule,
 \[
	 \frac{\partial u}{\partial x} = \frac{\partial v}{\partial x} (y,z) = \frac{\partial v}{\partial y} \cdot \frac{\partial y}{\partial x} + \frac{\partial v}{\partial z} \cdot \frac{\partial z}{\partial x} = v_y + v_z
.\]
In general, for any function $ f(y,z)$, we have  $ \frac{\partial f}{\partial x} = f_y+ f_z$.

Again by Chain Rule,
\begin{align*}
	\frac{\partial^2 u}{\partial { x}^2} &= \frac{\partial }{\partial x} \frac{\partial u}{\partial x}  \\
					     &= \frac{\partial }{\partial x} (v_y + v_z)  \\
					     &= [v_{yy}+ v_{yz}] + [v_{zy}+ v_{zz}]\\
					     &=  \frac{\partial^2 v}{\partial { u}^2} + 2 \frac{\partial^2 v}{\partial { y} \partial z} + \frac{\partial^2 v}{\partial { z}^2}  
\end{align*}
the combining is due to Clairaut's Theorem. Similarly,
\[
	\frac{\partial u}{\partial t} = c(v_y - v_z)
.\] 
and
\[
	\frac{\partial^2 u}{\partial { t}^2} = c^2 \left( \frac{\partial^2 v}{\partial { y}^2} - 2 \frac{\partial^2 v}{\partial { y} \partial z} + \frac{\partial^2 v}{\partial { z}^2}  \right) 
.\] 
Now plug them into the wave equation and we can rewrite it after cancellation as
\[
4 c^2 \frac{\partial^2 v}{\partial { y} \partial z} =  0
.\]
We keep $ c^2$ there because it contains useful information about the problem. This is a much simpler equation to solve! Note $ \frac{\partial }{\partial y} v_z = 0$ implies that $ v_z$ constant with respect to $ y$. That is,
 \[
	 \frac{\partial }{\partial y} \left( \frac{\partial v}{\partial z}  \right) =0 \implies \frac{\partial v}{\partial z} = f'(z)
.\]
Integrating both sides wrt $ z$, we have
 \[
	 v(y,z) = f(z) + C_1(y)
.\] 
Note that the constant is only wrt $ z$, so it can be a function of  $ y$.

Now doing the same thing wrt  $ y$, we obtain
 \[
	 v(y,z) = g(y) + C_2(z)
.\] 
By term matching, we obtain
\[
	v(y,z) = f(z) + g(y)
.\] 
Plugging back $ x,t$, we obtain
 \[
	 v(y,z) = u(x,t) = f(x-ct)+ g(x+ct)
.\] 
We can use the BCs and ICs to determine $ f(z), g(y)$. This is the  \allbold{d'Alembert's solution to the traveling wave equation}.

\begin{defn}[traveling waves]
\begin{enumerate}[label=\arabic*)]
	\item The function $ f(x-ct)$ is a waveform that moves to the right with velocity $ c>0$ and is called a \allbold{traveling wave}.
	\item Likewise, the function $ g(x+ct)$ is a traveling wave moving to the left with velocity  $ -c$.
\end{enumerate}
\end{defn}

\begin{note}[]
~\begin{enumerate}[label=\arabic*)]
	\item This solution is the general solution of the 1D wave equation as long as the 2nd derivatives are continuous. So we don't need to deal with convergence issues.
	\item For the Fourier Series solution of the wave equation, it can be shown that in order for the series for $ \partial^2_x u(x,t)$ to converge for $ t \geq 0$, we need  $ U''(x)$ to have a convergent series at  $ t=0$. Then if the initial data, that is  $ U(x)$ and  $ V(t)$, are smooth enough then the Fourier series solution is the solution to the BVP.
\end{enumerate}
\end{note}
\end{document}

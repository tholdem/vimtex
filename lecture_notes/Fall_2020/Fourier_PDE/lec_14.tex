\documentclass[class=article,crop=false]{standalone} 
%Fall 2020
% Some basic packages
\usepackage{standalone}[subpreambles=true]
\usepackage[utf8]{inputenc}
\usepackage[T1]{fontenc}
\usepackage{textcomp}
\usepackage[english]{babel}
\usepackage{url}
\usepackage{graphicx}
\usepackage{float}
\usepackage{enumitem}


\pdfminorversion=7

% Don't indent paragraphs, leave some space between them
\usepackage{parskip}

% Hide page number when page is empty
\usepackage{emptypage}
\usepackage{subcaption}
\usepackage{multicol}
\usepackage[dvipsnames]{xcolor}


% Math stuff
\usepackage{amsmath, amsfonts, mathtools, amsthm, amssymb}
% Fancy script capitals
\usepackage{mathrsfs}
\usepackage{cancel}
% Bold math
\usepackage{bm}
% Some shortcuts
\newcommand{\rr}{\ensuremath{\mathbb{R}}}
\newcommand{\zz}{\ensuremath{\mathbb{Z}}}
\newcommand{\qq}{\ensuremath{\mathbb{Q}}}
\newcommand{\nn}{\ensuremath{\mathbb{N}}}
\newcommand{\ff}{\ensuremath{\mathbb{F}}}
\newcommand{\cc}{\ensuremath{\mathbb{C}}}
\renewcommand\O{\ensuremath{\emptyset}}
\newcommand{\norm}[1]{{\left\lVert{#1}\right\rVert}}
\renewcommand{\vec}[1]{{\mathbf{#1}}}
\newcommand\allbold[1]{{\boldmath\textbf{#1}}}

% Put x \to \infty below \lim
\let\svlim\lim\def\lim{\svlim\limits}

%Make implies and impliedby shorter
\let\implies\Rightarrow
\let\impliedby\Leftarrow
\let\iff\Leftrightarrow
\let\epsilon\varepsilon

% Add \contra symbol to denote contradiction
\usepackage{stmaryrd} % for \lightning
\newcommand\contra{\scalebox{1.5}{$\lightning$}}

% \let\phi\varphi

% Command for short corrections
% Usage: 1+1=\correct{3}{2}

\definecolor{correct}{HTML}{009900}
\newcommand\correct[2]{\ensuremath{\:}{\color{red}{#1}}\ensuremath{\to }{\color{correct}{#2}}\ensuremath{\:}}
\newcommand\green[1]{{\color{correct}{#1}}}

% horizontal rule
\newcommand\hr{
    \noindent\rule[0.5ex]{\linewidth}{0.5pt}
}

% hide parts
\newcommand\hide[1]{}

% si unitx
\usepackage{siunitx}
\sisetup{locale = FR}

% Environments
\makeatother
% For box around Definition, Theorem, \ldots
\usepackage[framemethod=TikZ]{mdframed}
\mdfsetup{skipabove=1em,skipbelow=0em}

%definition
\newenvironment{defn}[1][]{%
\ifstrempty{#1}%
{\mdfsetup{%
frametitle={%
\tikz[baseline=(current bounding box.east),outer sep=0pt]
\node[anchor=east,rectangle,fill=Emerald]
{\strut Definition};}}
}%
{\mdfsetup{%
frametitle={%
\tikz[baseline=(current bounding box.east),outer sep=0pt]
\node[anchor=east,rectangle,fill=Emerald]
{\strut Definition:~#1};}}%
}%
\mdfsetup{innertopmargin=10pt,linecolor=Emerald,%
linewidth=2pt,topline=true,%
frametitleaboveskip=\dimexpr-\ht\strutbox\relax
}
\begin{mdframed}[]\relax%
\label{#1}}{\end{mdframed}}


%theorem
%\newcounter{thm}[section]\setcounter{thm}{0}
%\renewcommand{\thethm}{\arabic{section}.\arabic{thm}}
\newenvironment{thm}[1][]{%
%\refstepcounter{thm}%
\ifstrempty{#1}%
{\mdfsetup{%
frametitle={%
\tikz[baseline=(current bounding box.east),outer sep=0pt]
\node[anchor=east,rectangle,fill=blue!20]
%{\strut Theorem~\thethm};}}
{\strut Theorem};}}
}%
{\mdfsetup{%
frametitle={%
\tikz[baseline=(current bounding box.east),outer sep=0pt]
\node[anchor=east,rectangle,fill=blue!20]
%{\strut Theorem~\thethm:~#1};}}%
{\strut Theorem:~#1};}}%
}%
\mdfsetup{innertopmargin=10pt,linecolor=blue!20,%
linewidth=2pt,topline=true,%
frametitleaboveskip=\dimexpr-\ht\strutbox\relax
}
\begin{mdframed}[]\relax%
\label{#1}}{\end{mdframed}}


%lemma
\newenvironment{lem}[1][]{%
\ifstrempty{#1}%
{\mdfsetup{%
frametitle={%
\tikz[baseline=(current bounding box.east),outer sep=0pt]
\node[anchor=east,rectangle,fill=Dandelion]
{\strut Lemma};}}
}%
{\mdfsetup{%
frametitle={%
\tikz[baseline=(current bounding box.east),outer sep=0pt]
\node[anchor=east,rectangle,fill=Dandelion]
{\strut Lemma:~#1};}}%
}%
\mdfsetup{innertopmargin=10pt,linecolor=Dandelion,%
linewidth=2pt,topline=true,%
frametitleaboveskip=\dimexpr-\ht\strutbox\relax
}
\begin{mdframed}[]\relax%
\label{#1}}{\end{mdframed}}

%corollary
\newenvironment{coro}[1][]{%
\ifstrempty{#1}%
{\mdfsetup{%
frametitle={%
\tikz[baseline=(current bounding box.east),outer sep=0pt]
\node[anchor=east,rectangle,fill=CornflowerBlue]
{\strut Corollary};}}
}%
{\mdfsetup{%
frametitle={%
\tikz[baseline=(current bounding box.east),outer sep=0pt]
\node[anchor=east,rectangle,fill=CornflowerBlue]
{\strut Corollary:~#1};}}%
}%
\mdfsetup{innertopmargin=10pt,linecolor=CornflowerBlue,%
linewidth=2pt,topline=true,%
frametitleaboveskip=\dimexpr-\ht\strutbox\relax
}
\begin{mdframed}[]\relax%
\label{#1}}{\end{mdframed}}

%proof
\newenvironment{prf}[1][]{%
\ifstrempty{#1}%
{\mdfsetup{%
frametitle={%
\tikz[baseline=(current bounding box.east),outer sep=0pt]
\node[anchor=east,rectangle,fill=SpringGreen]
{\strut Proof};}}
}%
{\mdfsetup{%
frametitle={%
\tikz[baseline=(current bounding box.east),outer sep=0pt]
\node[anchor=east,rectangle,fill=SpringGreen]
{\strut Proof:~#1};}}%
}%
\mdfsetup{innertopmargin=10pt,linecolor=SpringGreen,%
linewidth=2pt,topline=true,%
frametitleaboveskip=\dimexpr-\ht\strutbox\relax
}
\begin{mdframed}[]\relax%
\label{#1}}{\qed\end{mdframed}}


\theoremstyle{definition}

\newmdtheoremenv[nobreak=true]{definition}{Definition}
\newmdtheoremenv[nobreak=true]{prop}{Proposition}
\newmdtheoremenv[nobreak=true]{theorem}{Theorem}
\newmdtheoremenv[nobreak=true]{corollary}{Corollary}
\newtheorem*{eg}{Example}
\theoremstyle{remark}
\newtheorem*{case}{Case}
\newtheorem*{notation}{Notation}
\newtheorem*{remark}{Remark}
\newtheorem*{note}{Note}
\newtheorem*{problem}{Problem}
\newtheorem*{observe}{Observe}
\newtheorem*{property}{Property}
\newtheorem*{intuition}{Intuition}


% End example and intermezzo environments with a small diamond (just like proof
% environments end with a small square)
\usepackage{etoolbox}
\AtEndEnvironment{vb}{\null\hfill$\diamond$}%
\AtEndEnvironment{intermezzo}{\null\hfill$\diamond$}%
% \AtEndEnvironment{opmerking}{\null\hfill$\diamond$}%

% Fix some spacing
% http://tex.stackexchange.com/questions/22119/how-can-i-change-the-spacing-before-theorems-with-amsthm
\makeatletter
\def\thm@space@setup{%
  \thm@preskip=\parskip \thm@postskip=0pt
}

% Fix some stuff
% %http://tex.stackexchange.com/questions/76273/multiple-pdfs-with-page-group-included-in-a-single-page-warning
\pdfsuppresswarningpagegroup=1


% My name
\author{Jaden Wang}



\begin{document}
\begin{intuition}
	The PDE and BCs allow us to form a vector space of solutions to the homogeneous equation.
\end{intuition}

The \allbold{time domain problem} is 
\[
	G'(t)=-\lambda k G(t) \implies \int \frac{1}{G(t)} G'(t) dt = \int -\lambda k\ dt \implies \ln|G(t)|=-\lambda kt + C_1
.\]
which finally yields the general solution
\[
	G(t) = Ce^{-\lambda k t}, C \in \rr
.\]

Physically we expect that $ \lambda>0$ (because since boundary condition is 0 we expect temperature to decay as time goes on). Now the boundary conditions give:
\[
	u(0,t)=0 \implies F(0) G(t) = 0 \implies F(0) = 0
.\] 
Because otherwise , if $ G(t)=0 \implies u(x,t)=F(x)G(t)=0$ which is a trivial solution that violates our separation assumption. Similarly $ u(L,t)=0 \implies F(L)=0$. Thus the \allbold{boundary value problem} (or \emph{eigenvalue problem)} can be formulated as
\begin{equation*}
\begin{cases}
	\frac{d^2 F}{d { x }^2} =-\lambda F(x) \\
	F(0) = 0 = F(L)\\
\end{cases}
\end{equation*}
To solve this ODE, let $ F(x)=e^{rx}$, so
\[
	F''(x)=-\lambda F(x) \implies r^2e^{rx}=-\lambda e^{rx} \implies r^2=-\lambda
.\] 
and the last equation is the characteristic equation. We claim that $ \lambda \leq 0$ gives the trivial solution $ F(x)=0$. 
\begin{prf}
\begin{case}[1]
	If $ \lambda = 0$, then $ r=0$ with repeated roots. Let  $ F(x) = C_1 e^{0} + C_2 x e^{0} \implies F(x) = C_1+C_2x$. Since $ 0=F(0)=C_1$, $ 0=F(L) =C_2 L \implies C_2=0$. Hence $ F(x)=0$ which is the trivial solution.
\end{case}
\begin{case}[2]
	Similarly when $ \lambda<0$ also gives us the trivial solution. (See homework).
\end{case}
\end{prf}

The only interesting case is $ \lambda >0$. Let's solve the BVP:
$ r^2 = -\lambda$ then we have purely imaginary roots $ r_{1,2}=\pm i \sqrt{\lambda} $ and by Euler's Formula, we have
\[
	e^{r_1 x} = e^{i \sqrt{\lambda} x }= \cos( \sqrt{\lambda} x)+ i \sin(\sqrt{\lambda} x )
.\] 
and
\[
e^{r_2 x} =e^{-i \sqrt{\lambda}x }= \cos(\sqrt{\lambda}x  ) - i\sin(\sqrt{\lambda}x  )
.\]
Then the general solution can be any linear combination of these functions. We can now convert them to the wave form:
\[
\frac{e^{i\sqrt{\lambda} x}+e^{-i\sqrt{\lambda} x}}{2}= \cos(\sqrt{\lambda} x )
.\] 
and
\[
\frac{e^{i\sqrt{\lambda} x}- e^{-i\sqrt{\lambda} x}}{2i} = \sin(\sqrt{\lambda}x  )
.\] 
So a general solution of the ODE is
\[
	F(x) = C_1 \cos(\sqrt{ \lambda} x ) + C_2 \sin(\sqrt{\lambda} x )
.\] 
The boundary conditions give
\[
	F(0) = 0 \implies C_1 = 0 \text{ and } F(L)=0 = C_2 \sin(\sqrt{\lambda}L  ) 
.\] 
Since $ C_2 \neq 0$ (or it would be trivial solution), this implies
\[
	\sin(\sqrt{\lambda} L ) = 0 \implies \sqrt{\lambda}L = n \pi \implies \sqrt{\lambda} =\frac{n \pi}{L} \implies \lambda = \left( \frac{n\pi}{L} \right)^2  = \lambda_n, \text{ for } n=\pm 1,\pm 2,\ldots   
.\]
Note $ n$ should not be 0 since it would give $\lambda =0 $. Since sine is an odd function, for $ n = 1,2,3,\ldots $ we can write $ F_n(x) = C_2 \sin(\frac{n\pi x}{L} )$. Thus we have the \emph{product solution}
\[
	u_n(x,t)=F_n(x)G_n(t)=B_n \sin \left( \frac{ n\pi x}{ L} \right) e^{-\left( \frac{n\pi}{L} \right)^2 kt} \text{ for } n=1,2,3,\ldots 
.\] 
and some constants $ B_n$. 

Now by superposition principle, the solution of the homogeneous PDE (if it converges) is the linear combination of the product solutions,
\[
	u(x,t)=\sum_{ n= 1}^{\infty} B_n \sin \left( \frac{ n\pi x}{ L} \right) e^{-\left( \frac{n\pi}{L} \right)^2 kt} \text{ for some constants } B_n 
.\] 
and now using the initial condition to determine $ B_n$. Note that we have an orthogonal basis of sines. This allows us to use the projection formula to find $ B_n$. When $ t=0$, we get a  \allbold{Fourier Sine Series} (FSS), thus by projection formula (f(x) inner product with a sine basis vector):
\[
	f(x)=u(x,0)=\sum_{ n= 1}^{\infty} B_n \sin \left( \frac{ n\pi x}{ L} \right) \implies B_n = \frac{2}{L} \int_{0}^{L} f(x) \sin \left( \frac{ n\pi x}{ L} \right) dx
.\]
Note that the domain is only from 0 to $ L$, and the odd extension of $f(x)$ makes the whole integrand odd, allowing us to simply multiply by 2 using symmetry. That is, the solution of the heat equation assuming convergence is
\[
	u(x,t)=\sum_{ n= 1}^{\infty} \left( \frac{2}{L} \int_{0}^{L} f(x) \sin \left( \frac{ n\pi x}{ L} \right) dx  \right) \sin \left( \frac{ n\pi x}{ L} \right) e^{-\left( \frac{n\pi}{L} \right)^2 kt }
.\]
Note that the exponentially decaying term guarantees the convergence. 
\end{document}

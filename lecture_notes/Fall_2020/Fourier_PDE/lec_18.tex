\documentclass[class=article,crop=false]{standalone} 
%Fall 2020
% Some basic packages
\usepackage{standalone}[subpreambles=true]
\usepackage[utf8]{inputenc}
\usepackage[T1]{fontenc}
\usepackage{textcomp}
\usepackage[english]{babel}
\usepackage{url}
\usepackage{graphicx}
\usepackage{float}
\usepackage{enumitem}


\pdfminorversion=7

% Don't indent paragraphs, leave some space between them
\usepackage{parskip}

% Hide page number when page is empty
\usepackage{emptypage}
\usepackage{subcaption}
\usepackage{multicol}
\usepackage[dvipsnames]{xcolor}


% Math stuff
\usepackage{amsmath, amsfonts, mathtools, amsthm, amssymb}
% Fancy script capitals
\usepackage{mathrsfs}
\usepackage{cancel}
% Bold math
\usepackage{bm}
% Some shortcuts
\newcommand{\rr}{\ensuremath{\mathbb{R}}}
\newcommand{\zz}{\ensuremath{\mathbb{Z}}}
\newcommand{\qq}{\ensuremath{\mathbb{Q}}}
\newcommand{\nn}{\ensuremath{\mathbb{N}}}
\newcommand{\ff}{\ensuremath{\mathbb{F}}}
\newcommand{\cc}{\ensuremath{\mathbb{C}}}
\renewcommand\O{\ensuremath{\emptyset}}
\newcommand{\norm}[1]{{\left\lVert{#1}\right\rVert}}
\renewcommand{\vec}[1]{{\mathbf{#1}}}
\newcommand\allbold[1]{{\boldmath\textbf{#1}}}

% Put x \to \infty below \lim
\let\svlim\lim\def\lim{\svlim\limits}

%Make implies and impliedby shorter
\let\implies\Rightarrow
\let\impliedby\Leftarrow
\let\iff\Leftrightarrow
\let\epsilon\varepsilon

% Add \contra symbol to denote contradiction
\usepackage{stmaryrd} % for \lightning
\newcommand\contra{\scalebox{1.5}{$\lightning$}}

% \let\phi\varphi

% Command for short corrections
% Usage: 1+1=\correct{3}{2}

\definecolor{correct}{HTML}{009900}
\newcommand\correct[2]{\ensuremath{\:}{\color{red}{#1}}\ensuremath{\to }{\color{correct}{#2}}\ensuremath{\:}}
\newcommand\green[1]{{\color{correct}{#1}}}

% horizontal rule
\newcommand\hr{
    \noindent\rule[0.5ex]{\linewidth}{0.5pt}
}

% hide parts
\newcommand\hide[1]{}

% si unitx
\usepackage{siunitx}
\sisetup{locale = FR}

% Environments
\makeatother
% For box around Definition, Theorem, \ldots
\usepackage[framemethod=TikZ]{mdframed}
\mdfsetup{skipabove=1em,skipbelow=0em}

%definition
\newenvironment{defn}[1][]{%
\ifstrempty{#1}%
{\mdfsetup{%
frametitle={%
\tikz[baseline=(current bounding box.east),outer sep=0pt]
\node[anchor=east,rectangle,fill=Emerald]
{\strut Definition};}}
}%
{\mdfsetup{%
frametitle={%
\tikz[baseline=(current bounding box.east),outer sep=0pt]
\node[anchor=east,rectangle,fill=Emerald]
{\strut Definition:~#1};}}%
}%
\mdfsetup{innertopmargin=10pt,linecolor=Emerald,%
linewidth=2pt,topline=true,%
frametitleaboveskip=\dimexpr-\ht\strutbox\relax
}
\begin{mdframed}[]\relax%
\label{#1}}{\end{mdframed}}


%theorem
%\newcounter{thm}[section]\setcounter{thm}{0}
%\renewcommand{\thethm}{\arabic{section}.\arabic{thm}}
\newenvironment{thm}[1][]{%
%\refstepcounter{thm}%
\ifstrempty{#1}%
{\mdfsetup{%
frametitle={%
\tikz[baseline=(current bounding box.east),outer sep=0pt]
\node[anchor=east,rectangle,fill=blue!20]
%{\strut Theorem~\thethm};}}
{\strut Theorem};}}
}%
{\mdfsetup{%
frametitle={%
\tikz[baseline=(current bounding box.east),outer sep=0pt]
\node[anchor=east,rectangle,fill=blue!20]
%{\strut Theorem~\thethm:~#1};}}%
{\strut Theorem:~#1};}}%
}%
\mdfsetup{innertopmargin=10pt,linecolor=blue!20,%
linewidth=2pt,topline=true,%
frametitleaboveskip=\dimexpr-\ht\strutbox\relax
}
\begin{mdframed}[]\relax%
\label{#1}}{\end{mdframed}}


%lemma
\newenvironment{lem}[1][]{%
\ifstrempty{#1}%
{\mdfsetup{%
frametitle={%
\tikz[baseline=(current bounding box.east),outer sep=0pt]
\node[anchor=east,rectangle,fill=Dandelion]
{\strut Lemma};}}
}%
{\mdfsetup{%
frametitle={%
\tikz[baseline=(current bounding box.east),outer sep=0pt]
\node[anchor=east,rectangle,fill=Dandelion]
{\strut Lemma:~#1};}}%
}%
\mdfsetup{innertopmargin=10pt,linecolor=Dandelion,%
linewidth=2pt,topline=true,%
frametitleaboveskip=\dimexpr-\ht\strutbox\relax
}
\begin{mdframed}[]\relax%
\label{#1}}{\end{mdframed}}

%corollary
\newenvironment{coro}[1][]{%
\ifstrempty{#1}%
{\mdfsetup{%
frametitle={%
\tikz[baseline=(current bounding box.east),outer sep=0pt]
\node[anchor=east,rectangle,fill=CornflowerBlue]
{\strut Corollary};}}
}%
{\mdfsetup{%
frametitle={%
\tikz[baseline=(current bounding box.east),outer sep=0pt]
\node[anchor=east,rectangle,fill=CornflowerBlue]
{\strut Corollary:~#1};}}%
}%
\mdfsetup{innertopmargin=10pt,linecolor=CornflowerBlue,%
linewidth=2pt,topline=true,%
frametitleaboveskip=\dimexpr-\ht\strutbox\relax
}
\begin{mdframed}[]\relax%
\label{#1}}{\end{mdframed}}

%proof
\newenvironment{prf}[1][]{%
\ifstrempty{#1}%
{\mdfsetup{%
frametitle={%
\tikz[baseline=(current bounding box.east),outer sep=0pt]
\node[anchor=east,rectangle,fill=SpringGreen]
{\strut Proof};}}
}%
{\mdfsetup{%
frametitle={%
\tikz[baseline=(current bounding box.east),outer sep=0pt]
\node[anchor=east,rectangle,fill=SpringGreen]
{\strut Proof:~#1};}}%
}%
\mdfsetup{innertopmargin=10pt,linecolor=SpringGreen,%
linewidth=2pt,topline=true,%
frametitleaboveskip=\dimexpr-\ht\strutbox\relax
}
\begin{mdframed}[]\relax%
\label{#1}}{\qed\end{mdframed}}


\theoremstyle{definition}

\newmdtheoremenv[nobreak=true]{definition}{Definition}
\newmdtheoremenv[nobreak=true]{prop}{Proposition}
\newmdtheoremenv[nobreak=true]{theorem}{Theorem}
\newmdtheoremenv[nobreak=true]{corollary}{Corollary}
\newtheorem*{eg}{Example}
\theoremstyle{remark}
\newtheorem*{case}{Case}
\newtheorem*{notation}{Notation}
\newtheorem*{remark}{Remark}
\newtheorem*{note}{Note}
\newtheorem*{problem}{Problem}
\newtheorem*{observe}{Observe}
\newtheorem*{property}{Property}
\newtheorem*{intuition}{Intuition}


% End example and intermezzo environments with a small diamond (just like proof
% environments end with a small square)
\usepackage{etoolbox}
\AtEndEnvironment{vb}{\null\hfill$\diamond$}%
\AtEndEnvironment{intermezzo}{\null\hfill$\diamond$}%
% \AtEndEnvironment{opmerking}{\null\hfill$\diamond$}%

% Fix some spacing
% http://tex.stackexchange.com/questions/22119/how-can-i-change-the-spacing-before-theorems-with-amsthm
\makeatletter
\def\thm@space@setup{%
  \thm@preskip=\parskip \thm@postskip=0pt
}

% Fix some stuff
% %http://tex.stackexchange.com/questions/76273/multiple-pdfs-with-page-group-included-in-a-single-page-warning
\pdfsuppresswarningpagegroup=1


% My name
\author{Jaden Wang}



\begin{document}

\begin{note}[]
	The BCs in this problem is the \allbold{Von Neumann condition}, which gives rise to FCS. The BCs from a previous problem with no derivatives is the \allbold{Dirichlet condition}, which gives rise to FSS.  
\end{note}
\begin{note}[]
	In the case when PDE and BCs already form a vector space, we don't need to solve for steady-state and transient solutions separately because the eigenvalue problem at $ \lambda = 0$ case gives the steady-state solution. Let's directly use $ u(x,t)=F(x)G(t) \neq 0$ and apply separation of variables.
\end{note}
Then the time domain problem is
	\[
		G'(t)=-\lambda kG(t)
	.\] 
	And the solution is again $ G(t) = Ce^{-\lambda kt}, C \in \rr$. The boundary value problem is:
\begin{equation*}
\begin{cases}
	\frac{d^2 F}{d {x }^2} = -\lambda F(x)\\
	F'(0)=0=F'(L)\\
\end{cases}
\end{equation*}
This is equivalent to the eigenvalue problem:
\begin{case}[]
$ \lambda <0$ we get trivial solution.
\end{case}
\begin{case}[]
$ \lambda =0$, then
\[
	F''(x) =0 \implies F(x) = Ax+B
.\] 
and the BCs yields $ A=0$ thus  $ F(x)=B, B \in \rr$.
\begin{note}[]
We didn't get trivial solution here because it is the Von Neumann condition, as opposed to the Dirichlet condition from before.
\end{note}
\end{case}
\begin{case}[]
$ \lambda >0$, this is the same as before, we have
\[
	F(x) = c_1 \cos(\sqrt{\lambda }x  ) + c_2 \sin(\sqrt{\lambda } x )
.\] 
And apply BCs:
\[
	0=F'(0) \implies c_2 = 0 \implies F(x) = c_1 \cos(\sqrt{\lambda} x )
.\] 
\[
	0 = F'(L) \implies \sin(\sqrt{\lambda} L ) = 0 \implies \sqrt{\lambda} = \frac{n\pi}{L } \text{ for }  n=\pm 1, \pm 2,\ldots  
.\]
which implies $ \lambda_n =\left( \frac{n\pi}{L } \right)^2 $ for $ n=1,2,\ldots$.

Now we have $ u_n(x,t) = a_n \cos \left( \frac{ n\pi x}{ L} \right) e^{-( \frac{ n\pi}{L} )^2 kt} $ for $ n=1,2,\ldots$. The superposition principle asserts that the solution of the homogeneous PDE (if it converges) is the linear combination of all $ u_n(x,t)$. That is,
\[
	u(x,t) = A_0 + \sum_{ n= 1}^{\infty} A_n \cos \left( \frac{ n\pi x}{ L} \right) e^{-( \frac{ n\pi}{L} )^2 kt} 
\] 
for some constants $ A_n$. Now we apply the IC to find these constants. Since we have orthogonal cosine basis, it is in fact a Fourier Consine Series (FCS): 
 \[
	 f(x) = u(x,0)=A_0 +\sum_{ n= 1}^{\infty} A_n \cos \left( \frac{ n\pi x}{ L} \right) 
 .\] 
 we use the projection formula for the case $ t=0$ to find the coefficients of this basis:
 \[
	 A_0 = \frac{1}{L} \int_{0}^{L} f(x) dx 
 .\] 
 and
 \[
	 A_n = \frac{2}{L}\int_{0}^{L} f(x) \cos \left( \frac{ n\pi x}{ L} \right) dx 
 .\]
\end{case}

\begin{thm}[convergence]
For $ t>0$, if there exists a constant  $ 0<M< \infty$ such that $ |A_n|\leq M$, for all $ n$, then
 \[
A_0 + \sum_{ n= 1}^{\infty} A_n \cos \left( \frac{ n\pi x}{ L} \right) e^{-( \frac{ n\pi}{L} )^2 kt} 
\] 
converges absolutely for each $ x \in [0,L]$
\end{thm}
Therefore, our final solution is
\[
	u(x,t) = \frac{1}{L}\int_{0}^{L} f(x) dx + \sum_{ n= 1}^{\infty} \left( \frac{2}{L} \int_{0}^{L} f(x) \cos \left( \frac{ n\pi x}{ L} \right) dx   \right)   \cos \left( \frac{ n\pi x}{ L} \right) e^{-( \frac{ n\pi}{L} )^2 kt} 
.\] 
and note that when $ t \to \infty$, we obtain the steady state solution.

For large but finite time, we can use the slowest decaying term to approximate the solution
\[
	u(x,t) = A_0 + A_1 \cos \left( \frac{ \pi x}{ L} \right) e^{-( \frac{ \pi}{L} )^2 kt} 
.\]

\subsection{Fourier Cosine Series}
What does it represent?
\begin{defn}[even extension]
\begin{enumerate}[label=\arabic*)]
	\item Define the \allbold{even extension} of $ f(x)$ to be
		 \begin{equation*}
			 f_{even}(x)=
		\begin{cases}
			f(x), & \text{ if }0<x<L\\
			f(-x), & \text{ if } -L<x<0 
		\end{cases}
		\end{equation*}
		Then $ f_{even}(-x)=f_{even}(x)$ for any $ x \in (-L,L)$.
	\item If $ f(x)$ is piecewise smooth then  $ f(x)$ has a Fourier series representation and if  \[
			f(x) = A_0 + \sum_{ n= 1}^{\infty} A_n \cos \left( \frac{ n\pi x}{ L} \right) , \text{ for } 0<x<L 
	.\] 
	Then note that the RHS is continuous, even, and $ 2L$-periodic. Thus the Fourier cosine series of  $ f(x)$ represents the periodoc extension of the (adjusted) even extension of  $ f(x)$ that is
	 \[
	A_0 + \sum_{ n= 1}^{\infty} A_n \cos \left( \frac{ n\pi x}{ L} \right) = \tilde{ \overline{f}}_{even}( x) 
	.\] 
\item in general, FCS$[f](x) = \text{ F.S.} [ f_{even}]( x)  $.
\end{enumerate}
\end{defn}
\begin{eg}[]
	See lecture notes for figures of FCS.
\end{eg}



\end{document}


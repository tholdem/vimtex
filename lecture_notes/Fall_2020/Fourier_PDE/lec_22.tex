\documentclass[class=article,crop=false]{standalone} 
%Fall 2020
% Some basic packages
\usepackage{standalone}[subpreambles=true]
\usepackage[utf8]{inputenc}
\usepackage[T1]{fontenc}
\usepackage{textcomp}
\usepackage[english]{babel}
\usepackage{url}
\usepackage{graphicx}
\usepackage{float}
\usepackage{enumitem}


\pdfminorversion=7

% Don't indent paragraphs, leave some space between them
\usepackage{parskip}

% Hide page number when page is empty
\usepackage{emptypage}
\usepackage{subcaption}
\usepackage{multicol}
\usepackage[dvipsnames]{xcolor}


% Math stuff
\usepackage{amsmath, amsfonts, mathtools, amsthm, amssymb}
% Fancy script capitals
\usepackage{mathrsfs}
\usepackage{cancel}
% Bold math
\usepackage{bm}
% Some shortcuts
\newcommand{\rr}{\ensuremath{\mathbb{R}}}
\newcommand{\zz}{\ensuremath{\mathbb{Z}}}
\newcommand{\qq}{\ensuremath{\mathbb{Q}}}
\newcommand{\nn}{\ensuremath{\mathbb{N}}}
\newcommand{\ff}{\ensuremath{\mathbb{F}}}
\newcommand{\cc}{\ensuremath{\mathbb{C}}}
\renewcommand\O{\ensuremath{\emptyset}}
\newcommand{\norm}[1]{{\left\lVert{#1}\right\rVert}}
\renewcommand{\vec}[1]{{\mathbf{#1}}}
\newcommand\allbold[1]{{\boldmath\textbf{#1}}}

% Put x \to \infty below \lim
\let\svlim\lim\def\lim{\svlim\limits}

%Make implies and impliedby shorter
\let\implies\Rightarrow
\let\impliedby\Leftarrow
\let\iff\Leftrightarrow
\let\epsilon\varepsilon

% Add \contra symbol to denote contradiction
\usepackage{stmaryrd} % for \lightning
\newcommand\contra{\scalebox{1.5}{$\lightning$}}

% \let\phi\varphi

% Command for short corrections
% Usage: 1+1=\correct{3}{2}

\definecolor{correct}{HTML}{009900}
\newcommand\correct[2]{\ensuremath{\:}{\color{red}{#1}}\ensuremath{\to }{\color{correct}{#2}}\ensuremath{\:}}
\newcommand\green[1]{{\color{correct}{#1}}}

% horizontal rule
\newcommand\hr{
    \noindent\rule[0.5ex]{\linewidth}{0.5pt}
}

% hide parts
\newcommand\hide[1]{}

% si unitx
\usepackage{siunitx}
\sisetup{locale = FR}

% Environments
\makeatother
% For box around Definition, Theorem, \ldots
\usepackage[framemethod=TikZ]{mdframed}
\mdfsetup{skipabove=1em,skipbelow=0em}

%definition
\newenvironment{defn}[1][]{%
\ifstrempty{#1}%
{\mdfsetup{%
frametitle={%
\tikz[baseline=(current bounding box.east),outer sep=0pt]
\node[anchor=east,rectangle,fill=Emerald]
{\strut Definition};}}
}%
{\mdfsetup{%
frametitle={%
\tikz[baseline=(current bounding box.east),outer sep=0pt]
\node[anchor=east,rectangle,fill=Emerald]
{\strut Definition:~#1};}}%
}%
\mdfsetup{innertopmargin=10pt,linecolor=Emerald,%
linewidth=2pt,topline=true,%
frametitleaboveskip=\dimexpr-\ht\strutbox\relax
}
\begin{mdframed}[]\relax%
\label{#1}}{\end{mdframed}}


%theorem
%\newcounter{thm}[section]\setcounter{thm}{0}
%\renewcommand{\thethm}{\arabic{section}.\arabic{thm}}
\newenvironment{thm}[1][]{%
%\refstepcounter{thm}%
\ifstrempty{#1}%
{\mdfsetup{%
frametitle={%
\tikz[baseline=(current bounding box.east),outer sep=0pt]
\node[anchor=east,rectangle,fill=blue!20]
%{\strut Theorem~\thethm};}}
{\strut Theorem};}}
}%
{\mdfsetup{%
frametitle={%
\tikz[baseline=(current bounding box.east),outer sep=0pt]
\node[anchor=east,rectangle,fill=blue!20]
%{\strut Theorem~\thethm:~#1};}}%
{\strut Theorem:~#1};}}%
}%
\mdfsetup{innertopmargin=10pt,linecolor=blue!20,%
linewidth=2pt,topline=true,%
frametitleaboveskip=\dimexpr-\ht\strutbox\relax
}
\begin{mdframed}[]\relax%
\label{#1}}{\end{mdframed}}


%lemma
\newenvironment{lem}[1][]{%
\ifstrempty{#1}%
{\mdfsetup{%
frametitle={%
\tikz[baseline=(current bounding box.east),outer sep=0pt]
\node[anchor=east,rectangle,fill=Dandelion]
{\strut Lemma};}}
}%
{\mdfsetup{%
frametitle={%
\tikz[baseline=(current bounding box.east),outer sep=0pt]
\node[anchor=east,rectangle,fill=Dandelion]
{\strut Lemma:~#1};}}%
}%
\mdfsetup{innertopmargin=10pt,linecolor=Dandelion,%
linewidth=2pt,topline=true,%
frametitleaboveskip=\dimexpr-\ht\strutbox\relax
}
\begin{mdframed}[]\relax%
\label{#1}}{\end{mdframed}}

%corollary
\newenvironment{coro}[1][]{%
\ifstrempty{#1}%
{\mdfsetup{%
frametitle={%
\tikz[baseline=(current bounding box.east),outer sep=0pt]
\node[anchor=east,rectangle,fill=CornflowerBlue]
{\strut Corollary};}}
}%
{\mdfsetup{%
frametitle={%
\tikz[baseline=(current bounding box.east),outer sep=0pt]
\node[anchor=east,rectangle,fill=CornflowerBlue]
{\strut Corollary:~#1};}}%
}%
\mdfsetup{innertopmargin=10pt,linecolor=CornflowerBlue,%
linewidth=2pt,topline=true,%
frametitleaboveskip=\dimexpr-\ht\strutbox\relax
}
\begin{mdframed}[]\relax%
\label{#1}}{\end{mdframed}}

%proof
\newenvironment{prf}[1][]{%
\ifstrempty{#1}%
{\mdfsetup{%
frametitle={%
\tikz[baseline=(current bounding box.east),outer sep=0pt]
\node[anchor=east,rectangle,fill=SpringGreen]
{\strut Proof};}}
}%
{\mdfsetup{%
frametitle={%
\tikz[baseline=(current bounding box.east),outer sep=0pt]
\node[anchor=east,rectangle,fill=SpringGreen]
{\strut Proof:~#1};}}%
}%
\mdfsetup{innertopmargin=10pt,linecolor=SpringGreen,%
linewidth=2pt,topline=true,%
frametitleaboveskip=\dimexpr-\ht\strutbox\relax
}
\begin{mdframed}[]\relax%
\label{#1}}{\qed\end{mdframed}}


\theoremstyle{definition}

\newmdtheoremenv[nobreak=true]{definition}{Definition}
\newmdtheoremenv[nobreak=true]{prop}{Proposition}
\newmdtheoremenv[nobreak=true]{theorem}{Theorem}
\newmdtheoremenv[nobreak=true]{corollary}{Corollary}
\newtheorem*{eg}{Example}
\theoremstyle{remark}
\newtheorem*{case}{Case}
\newtheorem*{notation}{Notation}
\newtheorem*{remark}{Remark}
\newtheorem*{note}{Note}
\newtheorem*{problem}{Problem}
\newtheorem*{observe}{Observe}
\newtheorem*{property}{Property}
\newtheorem*{intuition}{Intuition}


% End example and intermezzo environments with a small diamond (just like proof
% environments end with a small square)
\usepackage{etoolbox}
\AtEndEnvironment{vb}{\null\hfill$\diamond$}%
\AtEndEnvironment{intermezzo}{\null\hfill$\diamond$}%
% \AtEndEnvironment{opmerking}{\null\hfill$\diamond$}%

% Fix some spacing
% http://tex.stackexchange.com/questions/22119/how-can-i-change-the-spacing-before-theorems-with-amsthm
\makeatletter
\def\thm@space@setup{%
  \thm@preskip=\parskip \thm@postskip=0pt
}

% Fix some stuff
% %http://tex.stackexchange.com/questions/76273/multiple-pdfs-with-page-group-included-in-a-single-page-warning
\pdfsuppresswarningpagegroup=1


% My name
\author{Jaden Wang}



\begin{document}
\newpage
\section{Solve A Wave Equation}
\begin{equation*}
\begin{cases}
	\text{PDE: } \frac{\partial^2 u}{\partial { t}^2} = c^2 \frac{\partial^2 u}{\partial { x}^2}  & 0<x<L, t>0 \\
	\text{BCs: } u(0,t)=0=u(L,t) & t>0 \\
	\text{ICs: } u(x,0)=U(x), \frac{\partial u}{\partial t} (x,0) = V(x) & 0 \leq x \leq L \\
\end{cases}
\end{equation*}

We need to verify solutions that satisfy PDE and BCs form a vector space in order to use superposition principle. That is, we want to show that
\[
	W=\{u(x,t) \text{ is defined on } (x,t) \in (0,L) \times (0,\infty) \text{ that satisfies PDE and BCs} \} 
\] 
forms a vector space.

\begin{prf}
	Notice that $V= \{f:[0,L] \to \rr\} $ is a vector space. We wish to show that $ W$ is a subspace of  $ V$. 

	Given $ u_1, u_2 \in W$, then consider $ u_3(x,t) = a u_1(x,t) + b u_2(x,t)$ and note that partial derivatives are linear, we have
\begin{align*}
	\frac{\partial^2 u_3}{\partial { t}^2} &= \frac{\partial^2 }{\partial { t}^2} (a u_1 + b u_2)\\
	a \frac{\partial^2 u_1}{\partial { t}^2} + b \frac{\partial^2 u_2}{\partial { x}^2} &= a c^2 \frac{\partial^2 u_1}{\partial { x}^2} + b c^2 \frac{\partial^2 u_2}{\partial { x}^2}\\
											    &= c^2 (a \frac{\partial^2 u_1}{\partial { x}^2} + b \frac{\partial^2 u_2}{\partial { x}^2} ) 
\end{align*}
So the PDE condition is satisfied. Then $ u_3(0,t) = a u_1(0,t) + b u_2(0,t) = 0+0=0$ and $ u_3(L,t) = a u_1(L,t) + u_2(L,t) = 0+0=0$.

So the BCs condition is satisfied. Finally, since piecewise smooth functions are closed under addition and scalar multiplication, we have shown that $ u_3(x,t) \in W \subseteq V$. Hence $ W \leq V$. 
\end{prf}

Now we assume separation of variables, $ u(x,t) = F(x)G(t) \neq 0$, then the second partial derivatives are 
 \[
	 F(x)G''(t)= c^2F''(x)G(t) \implies \frac{1}{c^2} \frac{G''(t)}{G(t) }= \frac{F''(x)}{F(x) } = -\lambda
.\] 
Notice that the eigenvalue problem is the same as the heat equation. Then the BCs become
\[
	u(0,t)=0 \implies F(0)G(t)=0 \implies F(0)=0
.\] 
and
\[
	u(L,t) = 0 \implies F(L)G(t) = 0 \implies F(L) = 0
.\]
Thus the time domain problem and eigenvalue problem become
\[
	G''(t)=-\lambda c^2 G(t)
.\] 
and
\begin{equation*}
\begin{cases}
	&F''(x) = -\lambda F(x),	\\
	& F(0)=0=F(L).\\
\end{cases}
\end{equation*}

\subsection{Eigenvalue Problem}

\begin{case}[]
$ \lambda < 0$, then the general solution is
\[
	F(x) = c_1 e^{-\sqrt{\lambda}x } + c_2 e^{\sqrt{\lambda}x }
.\] 
Applying the BCs,
\[
	F(0) = 0 \implies c_2= -c_1 \text{ and } F(L) = 0 \implies c_1 = c_2 =0 
.\]
because $ e^{-\sqrt{\lambda} L } - e^{\sqrt{\lambda} L} \neq \iff \lambda \neq 0$. Hence this case yields the trivial solution. 
\end{case}
\begin{case}[]
$ \lambda = 0$ also yields trivial solution as before.
\end{case}
\begin{case}[]
$ \lambda >0$, by the exact same procedure as before, we obtain $ \sqrt{\lambda}  =\frac{n \pi }{L } $ and 
\[
	F_n(x) = c_2 \sin \left( \frac{ n\pi x}{ L} \right) \text{ for } n =1,2,\ldots 
.\] 
\end{case}
\subsection{Time-Domain Problem}
Similarly as above, we can obtain the general solution of $ G(t)$
\[
	G_n(t) = d_1 \cos(\sqrt{\lambda_n}ct  )+ d_2 \sin(\sqrt{\lambda_n}ct  ) \text{ for }n=1,2,\ldots 
.\] 
\subsection{General Solution}
Therefore,
\[
	u_n(x,t) = F_n(x)G_n(t) = \tilde{ A}_n \sin \left( \frac{ n\pi x}{ L} \right) \cos \left( \frac{ n\pi c t}{ L} \right) + \tilde{ B}_n \sin \left( \frac{ n\pi x}{ L} \right) \sin \left( \frac{ n\pi c t}{ L} \right) \text{ for } n=1,2,\ldots  
.\]
Now by superposition principle, the general solution is
\[
	u(x,t) = \sum_{ n= 1}^{\infty} a_n \cdot u_n(x,t) = \sum_{ n= 1}^{\infty} A_n \sin \left( \frac{ n\pi x}{ L} \right) \cos \left( \frac{ n\pi c t}{ L} \right) + B_n \sin \left( \frac{ n\pi x}{ L} \right) \sin \left( \frac{ n\pi c t}{ L} \right) 
.\]
Now applying the ICs to find the coefficients:
\[
	U(x) = u(x,0) = \sum_{ n= 1}^{\infty} A_n \sin \left( \frac{ n\pi x}{ L} \right) 
.\] 
This is just FSS. Applying the projection formula,
\[
	A_n = \frac{2}{L} \int_{0}^{L}  U(x) \sin \left( \frac{ n\pi x}{ L} \right) dx 
.\] 
The other IC requires us to do term-by-term differentiation:
\[
	\frac{\partial u}{\partial t} = \sum_{ n= 1}^{\infty} A_n\left(-\frac{n \pi c}{L } \right) \sin \left( \frac{ n\pi x}{ L} \right) \sin \left( \frac{ n\pi c t}{ L} \right) + B_n \left( \frac{n \pi c}{L } \right) \sin \left( \frac{ n\pi x}{ L} \right) \cos \left( \frac{ n\pi c t}{ L} \right) 
.\]
Therefore, the other IC also yields a FSS,
\[
	V(x) = u_t(x,0) = \sum_{ n= 1}^{\infty} B_n\left( \frac{n \pi c}{ L} \right) \sin \left( \frac{ n\pi x}{ L} \right)  
.\]
By projection,
\[
	B_n = \frac{2}{n \pi c} \int_{0}^{L} V(x) \sin \left( \frac{ n\pi x}{ L} \right) dx 
.\] 
\subsection{Convergence}
Note that the wave equation doesn't have an exponentially-decaying term as the heat equation. We have a spatial wave and a temporal wave. Thus, the convergence depends on how $ A_n, B_n$ behave as $ n \to \infty $, which in turn depends on the initial position and velocity.

While this is a drawback of the FS solution of the wave equation, it motivates d'Alembert's solution using traveling waves:
\[
	u(x,t) = f(x-ct) + g(x + c t).
.\] 

\end{document}

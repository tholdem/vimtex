\documentclass[class=article,crop=false]{standalone} 
%Fall 2020
% Some basic packages
\usepackage{standalone}[subpreambles=true]
\usepackage[utf8]{inputenc}
\usepackage[T1]{fontenc}
\usepackage{textcomp}
\usepackage[english]{babel}
\usepackage{url}
\usepackage{graphicx}
\usepackage{float}
\usepackage{enumitem}


\pdfminorversion=7

% Don't indent paragraphs, leave some space between them
\usepackage{parskip}

% Hide page number when page is empty
\usepackage{emptypage}
\usepackage{subcaption}
\usepackage{multicol}
\usepackage[dvipsnames]{xcolor}


% Math stuff
\usepackage{amsmath, amsfonts, mathtools, amsthm, amssymb}
% Fancy script capitals
\usepackage{mathrsfs}
\usepackage{cancel}
% Bold math
\usepackage{bm}
% Some shortcuts
\newcommand{\rr}{\ensuremath{\mathbb{R}}}
\newcommand{\zz}{\ensuremath{\mathbb{Z}}}
\newcommand{\qq}{\ensuremath{\mathbb{Q}}}
\newcommand{\nn}{\ensuremath{\mathbb{N}}}
\newcommand{\ff}{\ensuremath{\mathbb{F}}}
\newcommand{\cc}{\ensuremath{\mathbb{C}}}
\renewcommand\O{\ensuremath{\emptyset}}
\newcommand{\norm}[1]{{\left\lVert{#1}\right\rVert}}
\renewcommand{\vec}[1]{{\mathbf{#1}}}
\newcommand\allbold[1]{{\boldmath\textbf{#1}}}

% Put x \to \infty below \lim
\let\svlim\lim\def\lim{\svlim\limits}

%Make implies and impliedby shorter
\let\implies\Rightarrow
\let\impliedby\Leftarrow
\let\iff\Leftrightarrow
\let\epsilon\varepsilon

% Add \contra symbol to denote contradiction
\usepackage{stmaryrd} % for \lightning
\newcommand\contra{\scalebox{1.5}{$\lightning$}}

% \let\phi\varphi

% Command for short corrections
% Usage: 1+1=\correct{3}{2}

\definecolor{correct}{HTML}{009900}
\newcommand\correct[2]{\ensuremath{\:}{\color{red}{#1}}\ensuremath{\to }{\color{correct}{#2}}\ensuremath{\:}}
\newcommand\green[1]{{\color{correct}{#1}}}

% horizontal rule
\newcommand\hr{
    \noindent\rule[0.5ex]{\linewidth}{0.5pt}
}

% hide parts
\newcommand\hide[1]{}

% si unitx
\usepackage{siunitx}
\sisetup{locale = FR}

% Environments
\makeatother
% For box around Definition, Theorem, \ldots
\usepackage[framemethod=TikZ]{mdframed}
\mdfsetup{skipabove=1em,skipbelow=0em}

%definition
\newenvironment{defn}[1][]{%
\ifstrempty{#1}%
{\mdfsetup{%
frametitle={%
\tikz[baseline=(current bounding box.east),outer sep=0pt]
\node[anchor=east,rectangle,fill=Emerald]
{\strut Definition};}}
}%
{\mdfsetup{%
frametitle={%
\tikz[baseline=(current bounding box.east),outer sep=0pt]
\node[anchor=east,rectangle,fill=Emerald]
{\strut Definition:~#1};}}%
}%
\mdfsetup{innertopmargin=10pt,linecolor=Emerald,%
linewidth=2pt,topline=true,%
frametitleaboveskip=\dimexpr-\ht\strutbox\relax
}
\begin{mdframed}[]\relax%
\label{#1}}{\end{mdframed}}


%theorem
%\newcounter{thm}[section]\setcounter{thm}{0}
%\renewcommand{\thethm}{\arabic{section}.\arabic{thm}}
\newenvironment{thm}[1][]{%
%\refstepcounter{thm}%
\ifstrempty{#1}%
{\mdfsetup{%
frametitle={%
\tikz[baseline=(current bounding box.east),outer sep=0pt]
\node[anchor=east,rectangle,fill=blue!20]
%{\strut Theorem~\thethm};}}
{\strut Theorem};}}
}%
{\mdfsetup{%
frametitle={%
\tikz[baseline=(current bounding box.east),outer sep=0pt]
\node[anchor=east,rectangle,fill=blue!20]
%{\strut Theorem~\thethm:~#1};}}%
{\strut Theorem:~#1};}}%
}%
\mdfsetup{innertopmargin=10pt,linecolor=blue!20,%
linewidth=2pt,topline=true,%
frametitleaboveskip=\dimexpr-\ht\strutbox\relax
}
\begin{mdframed}[]\relax%
\label{#1}}{\end{mdframed}}


%lemma
\newenvironment{lem}[1][]{%
\ifstrempty{#1}%
{\mdfsetup{%
frametitle={%
\tikz[baseline=(current bounding box.east),outer sep=0pt]
\node[anchor=east,rectangle,fill=Dandelion]
{\strut Lemma};}}
}%
{\mdfsetup{%
frametitle={%
\tikz[baseline=(current bounding box.east),outer sep=0pt]
\node[anchor=east,rectangle,fill=Dandelion]
{\strut Lemma:~#1};}}%
}%
\mdfsetup{innertopmargin=10pt,linecolor=Dandelion,%
linewidth=2pt,topline=true,%
frametitleaboveskip=\dimexpr-\ht\strutbox\relax
}
\begin{mdframed}[]\relax%
\label{#1}}{\end{mdframed}}

%corollary
\newenvironment{coro}[1][]{%
\ifstrempty{#1}%
{\mdfsetup{%
frametitle={%
\tikz[baseline=(current bounding box.east),outer sep=0pt]
\node[anchor=east,rectangle,fill=CornflowerBlue]
{\strut Corollary};}}
}%
{\mdfsetup{%
frametitle={%
\tikz[baseline=(current bounding box.east),outer sep=0pt]
\node[anchor=east,rectangle,fill=CornflowerBlue]
{\strut Corollary:~#1};}}%
}%
\mdfsetup{innertopmargin=10pt,linecolor=CornflowerBlue,%
linewidth=2pt,topline=true,%
frametitleaboveskip=\dimexpr-\ht\strutbox\relax
}
\begin{mdframed}[]\relax%
\label{#1}}{\end{mdframed}}

%proof
\newenvironment{prf}[1][]{%
\ifstrempty{#1}%
{\mdfsetup{%
frametitle={%
\tikz[baseline=(current bounding box.east),outer sep=0pt]
\node[anchor=east,rectangle,fill=SpringGreen]
{\strut Proof};}}
}%
{\mdfsetup{%
frametitle={%
\tikz[baseline=(current bounding box.east),outer sep=0pt]
\node[anchor=east,rectangle,fill=SpringGreen]
{\strut Proof:~#1};}}%
}%
\mdfsetup{innertopmargin=10pt,linecolor=SpringGreen,%
linewidth=2pt,topline=true,%
frametitleaboveskip=\dimexpr-\ht\strutbox\relax
}
\begin{mdframed}[]\relax%
\label{#1}}{\qed\end{mdframed}}


\theoremstyle{definition}

\newmdtheoremenv[nobreak=true]{definition}{Definition}
\newmdtheoremenv[nobreak=true]{prop}{Proposition}
\newmdtheoremenv[nobreak=true]{theorem}{Theorem}
\newmdtheoremenv[nobreak=true]{corollary}{Corollary}
\newtheorem*{eg}{Example}
\theoremstyle{remark}
\newtheorem*{case}{Case}
\newtheorem*{notation}{Notation}
\newtheorem*{remark}{Remark}
\newtheorem*{note}{Note}
\newtheorem*{problem}{Problem}
\newtheorem*{observe}{Observe}
\newtheorem*{property}{Property}
\newtheorem*{intuition}{Intuition}


% End example and intermezzo environments with a small diamond (just like proof
% environments end with a small square)
\usepackage{etoolbox}
\AtEndEnvironment{vb}{\null\hfill$\diamond$}%
\AtEndEnvironment{intermezzo}{\null\hfill$\diamond$}%
% \AtEndEnvironment{opmerking}{\null\hfill$\diamond$}%

% Fix some spacing
% http://tex.stackexchange.com/questions/22119/how-can-i-change-the-spacing-before-theorems-with-amsthm
\makeatletter
\def\thm@space@setup{%
  \thm@preskip=\parskip \thm@postskip=0pt
}

% Fix some stuff
% %http://tex.stackexchange.com/questions/76273/multiple-pdfs-with-page-group-included-in-a-single-page-warning
\pdfsuppresswarningpagegroup=1


% My name
\author{Jaden Wang}



\begin{document}

~\begin{thm}[Heat Flow Rules]
We now assume the "heat flow rules":
\begin{enumerate}[label=\arabic*)]
	\item Constant temperature in a region implies that there is no heat flow.
	\item heat energy flows from hotter regions to colder regions.
	\item the greater the temperature difference, the greater  is the flow of heat energy.
	\item flow of heat energy will vary for differential materials
\end{enumerate}
\end{thm}

\begin{thm}[Fourier's Law of Heat Conduction]
\[
\Phi = -K_0 \frac{\partial u}{\partial x} 
.\] 
where $ K_0$ is known as the \allbold{thermal conductivity constant}. This equation can be read as "the heat flux is proportional to the temperature difference (per unit length). 
\end{thm}

\begin{thm}[the Heat Equation]
If we combine the equations:
\[
	e(x,t) =  c(x) \rho(x) u(x,t) \text{ and } \Phi = -K_0 \frac{\partial u}{\partial x}  
.\] 
with the partial differential equation
\[
\frac{\partial e}{\partial t} = - \frac{\partial \Phi}{\partial x}  + Q
.\] 
this implies:
\[
	\frac{\partial }{\partial t} [c(x)\rho(x) u(x,t)] = -\frac{\partial }{\partial x} \left(-K_0 \frac{\partial u}{\partial x} \right) + Q \implies c(x) \rho(x) \frac{\partial u}{\partial t} = K_0 \frac{\partial^2 u}{\partial {x}^2} +Q 
.\] 
and if we assume $ c(x)$ and  $ \rho(x)$ are also constants and that $ Q(x,t)=0$ then
 \begin{align*}
	 c\rho \frac{\partial u}{\partial t} &= K_0 \frac{\partial^2 u}{\partial {x}^2}\\
	 \frac{\partial u}{\partial t} &= k \frac{\partial^2 u}{\partial x^2}  \\
	 u_t &= k u_{x x} 
\end{align*}
where $ k=\frac{K_0}{c\rho}>0$ is the \allbold{thermal diffusivity constant}. The solution we seek is $ u(x,t)$, the temperature at any time and point along the rod. 
\end{thm}

\subsection{Initial and Boundary Conditions}

\begin{defn}
\begin{enumerate}[label=\arabic*)]
	\item Since the heat equation $ u_t = k u_{x x}$ has one time derivative, we need additional information, usually in the form of an initial condition (IC) at $ t=0$:
		 \[
			 u(x,0)=f(x)
		.\] 
		where $ f(x)$ specifies the initial (spatial) temperature distribution of the rod at time  $ t=0$.
	\item The two spatial derivatives in the term  $ u_{x x}$ requires two additional boudnary conditions (BC). In theory boundary condition information an be given for any two points $ x_1$ and $ x_2$ in the interval $ [0,L]$, however, conditions at  $ x=0$ and  $ x=L$ are usually given. Some examples include:
		 \begin{itemize}
			\item prescribed temperature
			\item insulated boundary
			\item Newton's law of cooling
		\end{itemize}
\end{enumerate}
\end{defn}

\begin{eg}[Prescribed Temperature]
	Let $ u_B(t)$ be the temperature of a fluid bath which one end of the rod is in contact with. In this case, we can prescribe the temperature of one end of the rod with a boundary condition of the form
	 \[
		 u(0,t)=u_B(t) \text{ or } u(L,t) = u_B(t) 
	.\] 
\end{eg}

\begin{eg}[Insulated Boundary]
	If we know the heat flux behavior at a single point, say $ x=0$, then the heat flux will be a function of  $ t$ only (since the location is fixed) and when combined with Fourier's Law of Heat Conduction, yields a boundary condition of the form
	 \[
		 -K_0 \frac{\partial u}{\partial x} (0,t) = \Phi(t) \implies \frac{\partial u}{\partial x} (0,t) = -\frac{1}{K_0}\Phi(t) \text{ where } \Phi(t) \text{ is given}  
	.\] 
	The simplest case of this is when one end of the rod is \allbold{perfectly insulated} (no heat flow at the boundaryso $ \Phi(t) = 0$ which yields the boundary condition:
	\[
		\frac{\partial u}{\partial x} (0,t) = \frac{1}{-K_0} \cdot  0 \implies \frac{\partial u}{\partial x}(0,t)  = 0
	.\]
\end{eg}

\begin{thm}[Newton's Law of Cooling]
The heat flow leaving the rod is proportional to the temperature difference between the rod and the external temperature. In this case we can define the heat flux for the boundary condition as
\[
	\Phi(t) = -H [u(0,t) - u_B(t)]
.\] 
where $ H>0$ is the  \allbold{heat transfer coefficient}. Note that if for example $ u(0,t) > u_B(t)$ then  $ \Phi(t)<0$ and heat flows out of the rod to the left as expected. We can use Fourier's Law of Heat Conduction to specify a boundary condition:
\[
	-K_0 \frac{\partial u}{\partial x} (0,t) = \Phi(t) \implies -K_0 \frac{\partial u}{\partial x} (0,t) = -H[u(0,t) - u_B(t)]
.\] 
That is the BC is
\[
	\frac{\partial u}{\partial x} (0,t) = \frac{H}{K_0} [u(0,t) - u_B(t)]
.\] 
Note that for $ x=L$, since the heat exits through the right we have  $ H$ becoming  $ -H$, so
 \[
	 u_x(L,t) = -\frac{H}{K_0}[u(L,t)-u_B(t)]
.\] 
\end{thm}


\subsection{Thermal Equilibrium}
~\begin{defn}[]
	The \allbold{steady state/equilibrium solution}  of the heat equation is a solution that does not depends on time, \emph{i.e.} $u(x,t) = u(x)$.
\end{defn}
\begin{note}[]
~\begin{enumerate}[label=\arabic*)]
	\item No matter what the initial temperature distribution of the rod is, some systems will undergo a process that brings it into "thermal equilibrium". That is, there exists a finite time $ T>0$  such that $ u_t=0$ for  $ t>T$.
	\item We expect that thermal equilibrium will be achieved in time,  \emph{i.e.} 
		\[
			\lim_{ t \to \infty} u(x,t)= \overline{u}(x)
		.\] 
		where $ \overline{u}(t)$ is the steady state temperature or steady state solution.
	\item We claim $ u(x,t) = \overline{u}(x) + v(x,t)$ where $ v(x,t) \to 0$ as $ t \to \infty$. Think of it as an interaction between short-term and long-term behaviors. We aim to solve these two solutions separately.
	\item Consider blinking holiday lights, the lightbulbs are either being heated up (ON) or cooling down (OFF). Thus there is no steady state temperature for this system. Not all systems have equilibrium solutions.
\end{enumerate}
\end{note}

\begin{eg}[1]
	Suppose that $ u(x,t) = u(x)$ then this implies  $ \frac{\partial u}{\partial t} =0$ and so the heat equation becomes 
	\[
		0=k \frac{\partial^2 u}{\partial {x}^2} \implies \frac{d^2 u}{d {x}^2} = 0 \implies \frac{d u}{d x} =C_1 \implies u(x) = C_1 x + C_2 
	.\]
	and suppose the boundary conditions are steady, suppose $ u(0) = T_1$ and $ u(L) = T_2$ then the steady state solution is the line
	\[
		\overline{u}(x) = T1 + \frac{T_2-T_1}{L} x
	.\] 
\end{eg}


\end{document}

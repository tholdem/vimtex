\documentclass[class=article,crop=false]{standalone} 
%Fall 2020
% Some basic packages
\usepackage{standalone}[subpreambles=true]
\usepackage[utf8]{inputenc}
\usepackage[T1]{fontenc}
\usepackage{textcomp}
\usepackage[english]{babel}
\usepackage{url}
\usepackage{graphicx}
\usepackage{float}
\usepackage{enumitem}


\pdfminorversion=7

% Don't indent paragraphs, leave some space between them
\usepackage{parskip}

% Hide page number when page is empty
\usepackage{emptypage}
\usepackage{subcaption}
\usepackage{multicol}
\usepackage[dvipsnames]{xcolor}


% Math stuff
\usepackage{amsmath, amsfonts, mathtools, amsthm, amssymb}
% Fancy script capitals
\usepackage{mathrsfs}
\usepackage{cancel}
% Bold math
\usepackage{bm}
% Some shortcuts
\newcommand{\rr}{\ensuremath{\mathbb{R}}}
\newcommand{\zz}{\ensuremath{\mathbb{Z}}}
\newcommand{\qq}{\ensuremath{\mathbb{Q}}}
\newcommand{\nn}{\ensuremath{\mathbb{N}}}
\newcommand{\ff}{\ensuremath{\mathbb{F}}}
\newcommand{\cc}{\ensuremath{\mathbb{C}}}
\renewcommand\O{\ensuremath{\emptyset}}
\newcommand{\norm}[1]{{\left\lVert{#1}\right\rVert}}
\renewcommand{\vec}[1]{{\mathbf{#1}}}
\newcommand\allbold[1]{{\boldmath\textbf{#1}}}

% Put x \to \infty below \lim
\let\svlim\lim\def\lim{\svlim\limits}

%Make implies and impliedby shorter
\let\implies\Rightarrow
\let\impliedby\Leftarrow
\let\iff\Leftrightarrow
\let\epsilon\varepsilon

% Add \contra symbol to denote contradiction
\usepackage{stmaryrd} % for \lightning
\newcommand\contra{\scalebox{1.5}{$\lightning$}}

% \let\phi\varphi

% Command for short corrections
% Usage: 1+1=\correct{3}{2}

\definecolor{correct}{HTML}{009900}
\newcommand\correct[2]{\ensuremath{\:}{\color{red}{#1}}\ensuremath{\to }{\color{correct}{#2}}\ensuremath{\:}}
\newcommand\green[1]{{\color{correct}{#1}}}

% horizontal rule
\newcommand\hr{
    \noindent\rule[0.5ex]{\linewidth}{0.5pt}
}

% hide parts
\newcommand\hide[1]{}

% si unitx
\usepackage{siunitx}
\sisetup{locale = FR}

% Environments
\makeatother
% For box around Definition, Theorem, \ldots
\usepackage[framemethod=TikZ]{mdframed}
\mdfsetup{skipabove=1em,skipbelow=0em}

%definition
\newenvironment{defn}[1][]{%
\ifstrempty{#1}%
{\mdfsetup{%
frametitle={%
\tikz[baseline=(current bounding box.east),outer sep=0pt]
\node[anchor=east,rectangle,fill=Emerald]
{\strut Definition};}}
}%
{\mdfsetup{%
frametitle={%
\tikz[baseline=(current bounding box.east),outer sep=0pt]
\node[anchor=east,rectangle,fill=Emerald]
{\strut Definition:~#1};}}%
}%
\mdfsetup{innertopmargin=10pt,linecolor=Emerald,%
linewidth=2pt,topline=true,%
frametitleaboveskip=\dimexpr-\ht\strutbox\relax
}
\begin{mdframed}[]\relax%
\label{#1}}{\end{mdframed}}


%theorem
%\newcounter{thm}[section]\setcounter{thm}{0}
%\renewcommand{\thethm}{\arabic{section}.\arabic{thm}}
\newenvironment{thm}[1][]{%
%\refstepcounter{thm}%
\ifstrempty{#1}%
{\mdfsetup{%
frametitle={%
\tikz[baseline=(current bounding box.east),outer sep=0pt]
\node[anchor=east,rectangle,fill=blue!20]
%{\strut Theorem~\thethm};}}
{\strut Theorem};}}
}%
{\mdfsetup{%
frametitle={%
\tikz[baseline=(current bounding box.east),outer sep=0pt]
\node[anchor=east,rectangle,fill=blue!20]
%{\strut Theorem~\thethm:~#1};}}%
{\strut Theorem:~#1};}}%
}%
\mdfsetup{innertopmargin=10pt,linecolor=blue!20,%
linewidth=2pt,topline=true,%
frametitleaboveskip=\dimexpr-\ht\strutbox\relax
}
\begin{mdframed}[]\relax%
\label{#1}}{\end{mdframed}}


%lemma
\newenvironment{lem}[1][]{%
\ifstrempty{#1}%
{\mdfsetup{%
frametitle={%
\tikz[baseline=(current bounding box.east),outer sep=0pt]
\node[anchor=east,rectangle,fill=Dandelion]
{\strut Lemma};}}
}%
{\mdfsetup{%
frametitle={%
\tikz[baseline=(current bounding box.east),outer sep=0pt]
\node[anchor=east,rectangle,fill=Dandelion]
{\strut Lemma:~#1};}}%
}%
\mdfsetup{innertopmargin=10pt,linecolor=Dandelion,%
linewidth=2pt,topline=true,%
frametitleaboveskip=\dimexpr-\ht\strutbox\relax
}
\begin{mdframed}[]\relax%
\label{#1}}{\end{mdframed}}

%corollary
\newenvironment{coro}[1][]{%
\ifstrempty{#1}%
{\mdfsetup{%
frametitle={%
\tikz[baseline=(current bounding box.east),outer sep=0pt]
\node[anchor=east,rectangle,fill=CornflowerBlue]
{\strut Corollary};}}
}%
{\mdfsetup{%
frametitle={%
\tikz[baseline=(current bounding box.east),outer sep=0pt]
\node[anchor=east,rectangle,fill=CornflowerBlue]
{\strut Corollary:~#1};}}%
}%
\mdfsetup{innertopmargin=10pt,linecolor=CornflowerBlue,%
linewidth=2pt,topline=true,%
frametitleaboveskip=\dimexpr-\ht\strutbox\relax
}
\begin{mdframed}[]\relax%
\label{#1}}{\end{mdframed}}

%proof
\newenvironment{prf}[1][]{%
\ifstrempty{#1}%
{\mdfsetup{%
frametitle={%
\tikz[baseline=(current bounding box.east),outer sep=0pt]
\node[anchor=east,rectangle,fill=SpringGreen]
{\strut Proof};}}
}%
{\mdfsetup{%
frametitle={%
\tikz[baseline=(current bounding box.east),outer sep=0pt]
\node[anchor=east,rectangle,fill=SpringGreen]
{\strut Proof:~#1};}}%
}%
\mdfsetup{innertopmargin=10pt,linecolor=SpringGreen,%
linewidth=2pt,topline=true,%
frametitleaboveskip=\dimexpr-\ht\strutbox\relax
}
\begin{mdframed}[]\relax%
\label{#1}}{\qed\end{mdframed}}


\theoremstyle{definition}

\newmdtheoremenv[nobreak=true]{definition}{Definition}
\newmdtheoremenv[nobreak=true]{prop}{Proposition}
\newmdtheoremenv[nobreak=true]{theorem}{Theorem}
\newmdtheoremenv[nobreak=true]{corollary}{Corollary}
\newtheorem*{eg}{Example}
\theoremstyle{remark}
\newtheorem*{case}{Case}
\newtheorem*{notation}{Notation}
\newtheorem*{remark}{Remark}
\newtheorem*{note}{Note}
\newtheorem*{problem}{Problem}
\newtheorem*{observe}{Observe}
\newtheorem*{property}{Property}
\newtheorem*{intuition}{Intuition}


% End example and intermezzo environments with a small diamond (just like proof
% environments end with a small square)
\usepackage{etoolbox}
\AtEndEnvironment{vb}{\null\hfill$\diamond$}%
\AtEndEnvironment{intermezzo}{\null\hfill$\diamond$}%
% \AtEndEnvironment{opmerking}{\null\hfill$\diamond$}%

% Fix some spacing
% http://tex.stackexchange.com/questions/22119/how-can-i-change-the-spacing-before-theorems-with-amsthm
\makeatletter
\def\thm@space@setup{%
  \thm@preskip=\parskip \thm@postskip=0pt
}

% Fix some stuff
% %http://tex.stackexchange.com/questions/76273/multiple-pdfs-with-page-group-included-in-a-single-page-warning
\pdfsuppresswarningpagegroup=1


% My name
\author{Jaden Wang}



\begin{document}
\begin{note}[]
	Periodic extension requires the value at the end points be equal. Otherwise, the function would have two different outputs at the end points, which makes it not well-defined. A simple fix is to restrict the domain (remove one end point). 
\end{note}

\begin{defn}[generalized Fourier series]
	Given $y=F(x)$, where  $-L\leq x \leq L$ for some positive real number  $L>0$, define the inner product
	 \[
		 \langle f(x),g(x) \rangle = \int_{ -L}^{ L}  f(x) g(x) dx  \text{ with norm } \norm{f}_2  
	.\] 
	We can show that the countably infinite set
	\[
		\left\{ 1,\cos\left( \frac{\pi x}{L} \right), \sin \left( \frac{\pi x}{L} \right), \ldots   \right\} 
	.\] 
	is a set of orthogonal functions with respect to the inner product given above. If we assume that 
	\[
		F(x) = a_0 + \sum_{ n=1}^{\infty} a_n + \cos\left( \frac{n\pi x}{L} \right) + \sin\left( \frac{n \pi x}{L} \right) 
	.\] 
	then, by the projection formula for finding coordinates, the corresponding Fourier coefficients are
	\begin{align*}
		a_0 &= \frac{\langle F(x),1 \rangle}{\norm{1}^2 } =\frac{1}{2L} \int_{ -L}^{ L} F(x) dx\\
		a_n &= \frac{\langle F(x), \cos\left( \frac{n\pi x}{L} \right)  \rangle}{\norm{ \cos\left( \frac{n \pi x}{L} \right)}^2 } = \frac{1}{L} \int_{ -L}^{ L} F(x) \cos\left( \frac{n \pi x}{L} \right) dx\\
		b_n &=  \frac{\langle F(x), \sin\left( \frac{n\pi x}{L} \right)  \rangle}{\norm{ \sin\left( \frac{n \pi x}{L} \right)}^2 } = \frac{1}{L} \int_{ -L}^{ L} F(x) \sin\left( \frac{n \pi x}{L} \right) dx
	\end{align*}
\end{defn}

\begin{note}[]
	The generalization was achieved using a change of variables: let $z = \frac{\pi x}{L}$. The $L^2$-inner product on $[-L,L]$ and on $-\pi,\pi$ only differ by  $\frac{L}{\pi}$.
\end{note}


\begin{itemize}
	\item restrict domain
	\item redefine the value at end point so that $F(L)=F(-L)$.
\end{itemize}

\begin{eg}[]
	$F(x) = x, -L\leq x\leq L$. Redefine:
	 \[
		 F'(-L)=F'(L) = \frac{F(-L)+F(L)}{2} = \frac{-L+L}{2} = 0
	.\] 
\end{eg}

\subsection{Convergence}

~\begin{defn}[]
	Suppose $f(x)$ and  $g(x)$ are defined on  $[-L,L]$, we say  $f(x)$ and  $g(x)$ are  \allbold{equivalent} or \allbold{equal almost everywhere} (denoted as $f(x) \sim g(x)$). If $f(x)=g(x)$ for all  $x \in (-L,L)$ except possibly at finite set of points $\{x_1,x_2,\ldots\} $ (in fact measure-zero sets) at which $f(x_i) \neq g(x_i)$ where $|f(x_i)|<\infty$ and $|g(x_i)|<\infty$ for $i\leq i \leq k$. 
\end{defn}

\begin{note}[]
	If $f(x) \sim g(x)$ then  F.S. $[f](x) =$ F.S.$[g](x)$ but $f(x) \neq g(x)$. Hence the need for restriction.
\end{note}

~\begin{defn}[dense]
	Let $A$ be a non-empty set and suppose  $B$ is a subset of  $A$. We say set  $B$ is  \allbold{dense} in $A$ if any point of  $a$ of  $A$ can be written as a limit of points from  $B$, that is, if for any  $a \in A$ there exists a sequence points $(b_n)$ from  $B$ s.t.  $\lim_{ n \to \infty} b_n=a$. 
\end{defn}

\begin{defn}[trigonometric polynomial]
	A \allbold{trigonometric polynomial} is a finite sum of the form
	\[
		S_N = a_0 + \sum_{ n= 1}^{ N} a_n \cos\left( \frac{n\pi x}{L} \right)  + \sum_{ n= 1}^{ N} b_n \sin\left( \frac{n \pi x}{L} \right) 
	.\] 
\end{defn}
\begin{note}[]
 The Nth partial sum of a Fourier Series is a trigonometric polynomial.
\end{note}

\begin{prop}[]
The set of trigonometric polynomials is dense in the set of continuous functions.
\end{prop}

\begin{intuition}
	The analogy is that rational numbers are dense in real numbers.
\end{intuition}

\subsection{Periodicity}
It "smooths" the bad end points or removable discontinuity. see lecture notes.\\



\end{document}

\documentclass[class=article,crop=false]{standalone} 
%Fall 2020
% Some basic packages
\usepackage{standalone}[subpreambles=true]
\usepackage[utf8]{inputenc}
\usepackage[T1]{fontenc}
\usepackage{textcomp}
\usepackage[english]{babel}
\usepackage{url}
\usepackage{graphicx}
\usepackage{float}
\usepackage{enumitem}


\pdfminorversion=7

% Don't indent paragraphs, leave some space between them
\usepackage{parskip}

% Hide page number when page is empty
\usepackage{emptypage}
\usepackage{subcaption}
\usepackage{multicol}
\usepackage[dvipsnames]{xcolor}


% Math stuff
\usepackage{amsmath, amsfonts, mathtools, amsthm, amssymb}
% Fancy script capitals
\usepackage{mathrsfs}
\usepackage{cancel}
% Bold math
\usepackage{bm}
% Some shortcuts
\newcommand{\rr}{\ensuremath{\mathbb{R}}}
\newcommand{\zz}{\ensuremath{\mathbb{Z}}}
\newcommand{\qq}{\ensuremath{\mathbb{Q}}}
\newcommand{\nn}{\ensuremath{\mathbb{N}}}
\newcommand{\ff}{\ensuremath{\mathbb{F}}}
\newcommand{\cc}{\ensuremath{\mathbb{C}}}
\renewcommand\O{\ensuremath{\emptyset}}
\newcommand{\norm}[1]{{\left\lVert{#1}\right\rVert}}
\renewcommand{\vec}[1]{{\mathbf{#1}}}
\newcommand\allbold[1]{{\boldmath\textbf{#1}}}

% Put x \to \infty below \lim
\let\svlim\lim\def\lim{\svlim\limits}

%Make implies and impliedby shorter
\let\implies\Rightarrow
\let\impliedby\Leftarrow
\let\iff\Leftrightarrow
\let\epsilon\varepsilon

% Add \contra symbol to denote contradiction
\usepackage{stmaryrd} % for \lightning
\newcommand\contra{\scalebox{1.5}{$\lightning$}}

% \let\phi\varphi

% Command for short corrections
% Usage: 1+1=\correct{3}{2}

\definecolor{correct}{HTML}{009900}
\newcommand\correct[2]{\ensuremath{\:}{\color{red}{#1}}\ensuremath{\to }{\color{correct}{#2}}\ensuremath{\:}}
\newcommand\green[1]{{\color{correct}{#1}}}

% horizontal rule
\newcommand\hr{
    \noindent\rule[0.5ex]{\linewidth}{0.5pt}
}

% hide parts
\newcommand\hide[1]{}

% si unitx
\usepackage{siunitx}
\sisetup{locale = FR}

% Environments
\makeatother
% For box around Definition, Theorem, \ldots
\usepackage[framemethod=TikZ]{mdframed}
\mdfsetup{skipabove=1em,skipbelow=0em}

%definition
\newenvironment{defn}[1][]{%
\ifstrempty{#1}%
{\mdfsetup{%
frametitle={%
\tikz[baseline=(current bounding box.east),outer sep=0pt]
\node[anchor=east,rectangle,fill=Emerald]
{\strut Definition};}}
}%
{\mdfsetup{%
frametitle={%
\tikz[baseline=(current bounding box.east),outer sep=0pt]
\node[anchor=east,rectangle,fill=Emerald]
{\strut Definition:~#1};}}%
}%
\mdfsetup{innertopmargin=10pt,linecolor=Emerald,%
linewidth=2pt,topline=true,%
frametitleaboveskip=\dimexpr-\ht\strutbox\relax
}
\begin{mdframed}[]\relax%
\label{#1}}{\end{mdframed}}


%theorem
%\newcounter{thm}[section]\setcounter{thm}{0}
%\renewcommand{\thethm}{\arabic{section}.\arabic{thm}}
\newenvironment{thm}[1][]{%
%\refstepcounter{thm}%
\ifstrempty{#1}%
{\mdfsetup{%
frametitle={%
\tikz[baseline=(current bounding box.east),outer sep=0pt]
\node[anchor=east,rectangle,fill=blue!20]
%{\strut Theorem~\thethm};}}
{\strut Theorem};}}
}%
{\mdfsetup{%
frametitle={%
\tikz[baseline=(current bounding box.east),outer sep=0pt]
\node[anchor=east,rectangle,fill=blue!20]
%{\strut Theorem~\thethm:~#1};}}%
{\strut Theorem:~#1};}}%
}%
\mdfsetup{innertopmargin=10pt,linecolor=blue!20,%
linewidth=2pt,topline=true,%
frametitleaboveskip=\dimexpr-\ht\strutbox\relax
}
\begin{mdframed}[]\relax%
\label{#1}}{\end{mdframed}}


%lemma
\newenvironment{lem}[1][]{%
\ifstrempty{#1}%
{\mdfsetup{%
frametitle={%
\tikz[baseline=(current bounding box.east),outer sep=0pt]
\node[anchor=east,rectangle,fill=Dandelion]
{\strut Lemma};}}
}%
{\mdfsetup{%
frametitle={%
\tikz[baseline=(current bounding box.east),outer sep=0pt]
\node[anchor=east,rectangle,fill=Dandelion]
{\strut Lemma:~#1};}}%
}%
\mdfsetup{innertopmargin=10pt,linecolor=Dandelion,%
linewidth=2pt,topline=true,%
frametitleaboveskip=\dimexpr-\ht\strutbox\relax
}
\begin{mdframed}[]\relax%
\label{#1}}{\end{mdframed}}

%corollary
\newenvironment{coro}[1][]{%
\ifstrempty{#1}%
{\mdfsetup{%
frametitle={%
\tikz[baseline=(current bounding box.east),outer sep=0pt]
\node[anchor=east,rectangle,fill=CornflowerBlue]
{\strut Corollary};}}
}%
{\mdfsetup{%
frametitle={%
\tikz[baseline=(current bounding box.east),outer sep=0pt]
\node[anchor=east,rectangle,fill=CornflowerBlue]
{\strut Corollary:~#1};}}%
}%
\mdfsetup{innertopmargin=10pt,linecolor=CornflowerBlue,%
linewidth=2pt,topline=true,%
frametitleaboveskip=\dimexpr-\ht\strutbox\relax
}
\begin{mdframed}[]\relax%
\label{#1}}{\end{mdframed}}

%proof
\newenvironment{prf}[1][]{%
\ifstrempty{#1}%
{\mdfsetup{%
frametitle={%
\tikz[baseline=(current bounding box.east),outer sep=0pt]
\node[anchor=east,rectangle,fill=SpringGreen]
{\strut Proof};}}
}%
{\mdfsetup{%
frametitle={%
\tikz[baseline=(current bounding box.east),outer sep=0pt]
\node[anchor=east,rectangle,fill=SpringGreen]
{\strut Proof:~#1};}}%
}%
\mdfsetup{innertopmargin=10pt,linecolor=SpringGreen,%
linewidth=2pt,topline=true,%
frametitleaboveskip=\dimexpr-\ht\strutbox\relax
}
\begin{mdframed}[]\relax%
\label{#1}}{\qed\end{mdframed}}


\theoremstyle{definition}

\newmdtheoremenv[nobreak=true]{definition}{Definition}
\newmdtheoremenv[nobreak=true]{prop}{Proposition}
\newmdtheoremenv[nobreak=true]{theorem}{Theorem}
\newmdtheoremenv[nobreak=true]{corollary}{Corollary}
\newtheorem*{eg}{Example}
\theoremstyle{remark}
\newtheorem*{case}{Case}
\newtheorem*{notation}{Notation}
\newtheorem*{remark}{Remark}
\newtheorem*{note}{Note}
\newtheorem*{problem}{Problem}
\newtheorem*{observe}{Observe}
\newtheorem*{property}{Property}
\newtheorem*{intuition}{Intuition}


% End example and intermezzo environments with a small diamond (just like proof
% environments end with a small square)
\usepackage{etoolbox}
\AtEndEnvironment{vb}{\null\hfill$\diamond$}%
\AtEndEnvironment{intermezzo}{\null\hfill$\diamond$}%
% \AtEndEnvironment{opmerking}{\null\hfill$\diamond$}%

% Fix some spacing
% http://tex.stackexchange.com/questions/22119/how-can-i-change-the-spacing-before-theorems-with-amsthm
\makeatletter
\def\thm@space@setup{%
  \thm@preskip=\parskip \thm@postskip=0pt
}

% Fix some stuff
% %http://tex.stackexchange.com/questions/76273/multiple-pdfs-with-page-group-included-in-a-single-page-warning
\pdfsuppresswarningpagegroup=1


% My name
\author{Jaden Wang}



\begin{document}
\newpage
\section{Convergence Part 1}

~\begin{defn}[]
\begin{enumerate}[label=\arabic*)]
	\item $f(x)$ is \allbold{continuous} at  $x=a$ if  $\lim_{ x \to a} f(x) = f(a)$.
	\item $f(x)$ is \allbold{piecewise continuous} on the interval $[a,b]$ if  $f(x)$ is defined and continuous at all but a finite number of points  $x_i \in [a,b], 1\leq i \leq N$, where $f(x)$ has either a jump discontinuity or a removable discontinuity and  $f(a^+) = \lim_{ x \to a^+} f(x) $ and $f(b^-) = \lim_{ x \to b^-} f(x) $ exist.
	\item A differentiable function $f(x)$ is \allbold{piecewise smooth} on $[a,b]$ if both  $f(x)$ and  $f'(x)$ are piecewise continuous on  $[a,b]$, \emph{i.e.} $f \in \text{ piecewise } \mathcal{C}^1$ .
	\item The function  $f(x)$ is \allbold{continuous piecewise smooth} on  $[a,b]$ if  $f(x)$ is piecewise smooth on $[a,b]$ and has no discontinuities in  $(a,b)$.
\end{enumerate}
\end{defn}
\begin{note}[]
~\begin{enumerate}[label=\alph*)]
	\item $f'(x)$ will not be continuous at the discontinuities of  $f(x)$ (if any) and may have its own additional discontinuities.
	\item If  $f'(x)$ is piecewise continuous on  $(a,b)$, then  $f(x)$ is also piecewise continuous on  $(a,b)$. Why? (Jaden: $f'$ is piecewise continuous on $(a,b) \implies f'$ exists on $(a,b) \implies f$ is piecewise differentiable on $(a,b) \implies f$ is piecewise continuous on $(a,b)$. ) 
	\item "finite number of points" can be replaced with "a collection of points that is measure-zero" for full generality.
\end{enumerate}
\end{note}
\begin{eg}[]
	$f(x) = \frac{1}{x}$ where $x\neq 0$ and $x \in [-1,1]$ is not piecewise continuous because the left and right limits do not exist at $x=0$.
\end{eg}

\begin{eg}[]
	Consider $x \in [-1,1]$, if 
	\begin{equation*}
		f(x)=|x|^{\frac{1}{4}} =
	\begin{cases}
		x^{\frac{1}{4}} &\text{ if } x \in [0,1] \\
		(-x)^{\frac{1}{4}} & \text{ if } x \in [-1,0) \\
	\end{cases}	
	\end{equation*}
	\begin{equation*}
		f'(x)=
	\begin{cases}
		\frac{1}{4}x^{-\frac{3}{4}} &\text{ if }x \in (0,1)\\ 
		-\frac{1}{4}(-x)^{-\frac{3}{4}} &\text{ if } x \in (-1,0)\\ 
	\end{cases}
	\end{equation*}
	$f(x)$ is piecewise continuous on  $[-1,1]$ but the derivative is not piecewise continuous.
\end{eg}

\begin{eg}[]
	$f(x) = |x|$ is piecewise smooth.
\end{eg}


To prove convergence, we want:
\begin{enumerate}[label=\arabic*)]
	\item $|a_0| < \infty$
	\item $a_n \to 0, b_n \to 0$ as $n \to \infty$. Otherwise the series will diverge.
		\begin{note}[]
			As $n \to \infty$, the periodicity goes to zero. So $ \int_{-L}^{L}  f(x) \sin \left( \frac{ n\pi x}{ L } \right) $ is the cumulative rapidly oscillating positive and negative area bounded by $|f(x)|$. Intuitively they cancel each other out as oscillation increases to infinity.
		\end{note}
	\end{enumerate}

\subsection*{Restrictions}
\begin{enumerate}[label=\arabic*]
	\item If $f(-L) = f(L)$, then we require that  $\tilde{ f } (x)$, the periodic extension of $f(x)$, be piecewise smooth on  $[-L,L]$.
	\item If  $f(-L) \neq f(L)$, then redefine  $f_{new}(-L) = f_{new}(L) = \frac{f(-L)+f(L)}{2} $. We denote the adjusted function $\overline{f}(x)$. Now we require that the periodic extension of the adjusted function, $\tilde{\bar{f}} (x)$, be piecewise smooth on $[-L,L]$.
\end{enumerate}
\begin{notation}
	We will use $\tilde{ f } $ instead of $\tilde{\bar{f}} $ for brevity.
\end{notation}
\begin{lem}[1]
	The set of real functions defined on $[-L,L]$ that satisfy Restriction 1 and 2 form a vector space (satisfies linearity).
\end{lem}

\begin{lem}[2]
	If $f(x)$ satisfies Restriction 1 and 2 then there exists a number  $0<M<\infty$ such that 
	\[
		|\tilde{ f } (x)|< M \quad \forall x \in \rr
	.\] 
\end{lem}
\begin{prf}
Any closed interval on $\rr$ is compact. Since $f$ is piecewise continuous, the image of a compact set is also compact. And compact set in $\rr$ means it's closed and bounded by Heine-Borel Theorem.
\end{prf}

\begin{lem}[3]
	If $f(x)$ satisfies Restriction 1 and 2 and if  $M$ is the positive constant from Lemma 2 then the Fourier coefficients satisfy:
	 \[
	|a_0|<M, |a_n|<2M, \text{ and } |b_n|<2M
	.\] 
\end{lem}
\begin{prf}
	Given $a \in \rr$, $-|a|\leq a \leq |a|$. Thus given  $f(x)$ defined on  $[-L,L]$,
	 \begin{align*}
		 -|f(x)| \leq f(x) \leq |f(x)| &\implies  - \int_{-L}^{L} |f(x)| dx \leq \int_{-L}^{L} f(x)  dx \leq \int_{-L}^{L} |f(x) | dx\\
					       & \implies \left|\int_{-L}^{L} f(x)dx \right| \leq \int_{-L}^{L} f(x) dx 
	\end{align*}

	\begin{align*}
		|a_n| &= \left| \frac{1}{L} \int_{-L}^{L} f(x) \cos \left( \frac{ n\pi x}{ L } \right) dx  \right|  \\
		      &\leq \frac{1}{L} \int_{-L}^{L} |f(x)| \left| \cos \left( \frac{ n\pi x}{ L } \right)  \right| dx   \\
		      &\leq \frac{1}{L} \int_{-L}^{L} |f(x)|dx \\
		      &\leq \frac{1}{L} \int_{-L}^{L} M dx = \frac{M}{L}2L = 2M  
	\end{align*}
	That is, $|a_n|<2M$. Repeat for $b_n<2M$ and $a_0<M$.
\end{prf}

\begin{lem}[Riemann-Lebesgue (special case)]
	If $f(x)$ satisfies Restriction 1 and 2 then  $a_n \to 0$ and $b_n \to 0$ as $n \to \infty $.
\end{lem}

\begin{prf}
	Suppose $\tilde{\bar{f}} (x)$, denote any discontinuities as $\{x_1,x_2,\ldots\,x_N\} $ with $x_0=-L$ and $x_{N+1} = L$. Then we can write
	 \[
		 a_n = \frac{1}{L} \int_{-L}^{L} \tilde{ f } (x) \cos \left( \frac{ n\pi x}{ L } \right) dx = \sum_{ p=1}^{ N+1} \frac{1}{L} \int_{ x_{p-1}}^{ x_{p}} \tilde{ f } (x) \cos \left( \frac{ n\pi x}{ L } \right) dx
	 .\]
		 Now let 
		 \[
			 \alpha_{n,p} = \frac{1}{L} \int_{ x_{p-1}}^{ x_{p}} \tilde{ f } (x) \cos \left( \frac{ n\pi x}{ L } \right) dx  
		 .\] 
		 where $\tilde{ f } (x) = u$ and $\cos \left( \frac{ n\pi x}{ L } \right) dx = dv$. Then using integration by parts,
		 \begin{align*}
			 \alpha_{n,p} &= \frac{1}{L} \left( uv \big|_{x_{p-1}}^{x_p} - \int_{ x_{p-1}}^{ x_p} v du  \right)  \\
				  |a_{n,p}|    &= \frac{1}{n\pi} \left| \tilde{ f } (x) \sin \left( \frac{ n\pi x_p}{ L} \right) -\tilde{ f } (x)\sin \left( \frac{ n\pi x_{p-1}}{ L} \right) -\int_{ x_{p-1}}^{ x_p} \tilde{f'}(x)\sin \left( \frac{ n\pi x}{ L} \right) dx   \right|  \\
				      &\leq \frac{1}{n\pi} \bigg[ \left| \tilde{ f } (x) \sin \left( \frac{ n\pi x_p}{ L} \right)  \right| +\left| \tilde{ f } (x) \sin \left( \frac{ n\pi x_{p-1}}{ L} \right)  \right| \\
				      & \quad + \int_{x_{p-1}}^{x_p} \left| \tilde{f'}(x) \sin \left( \frac{ n\pi x}{ L} \right)   \right| dx  \bigg]  \text{$\qquad $ using triangle inequality } \\
			 &\leq \frac{1}{n\pi} \left| \tilde{ f } (x) \right| + \left|\tilde{ f } (x)\right| + \int_{ x_{p-1}}^{ x_p} |\tilde{ f } (x)| dx  \\
			 &\leq \frac{1}{n\pi} \cdot C
		\end{align*}
		 where $C$ represents the constant that bounds  $\tilde{f}(x) $ and $\tilde{f'}(x) $ on $[-L,L]$. As $n \to \infty$,
		 \[
|\alpha_{n,p}| \to 0 \implies a_n=\sum_{ p=1}^{ N+1} \alpha_{n,p} \to 0 
.\] 
\end{prf}


\end{document}

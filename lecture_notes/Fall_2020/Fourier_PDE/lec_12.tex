\documentclass[class=article,crop=false]{standalone} 
%Fall 2020
% Some basic packages
\usepackage{standalone}[subpreambles=true]
\usepackage[utf8]{inputenc}
\usepackage[T1]{fontenc}
\usepackage{textcomp}
\usepackage[english]{babel}
\usepackage{url}
\usepackage{graphicx}
\usepackage{float}
\usepackage{enumitem}


\pdfminorversion=7

% Don't indent paragraphs, leave some space between them
\usepackage{parskip}

% Hide page number when page is empty
\usepackage{emptypage}
\usepackage{subcaption}
\usepackage{multicol}
\usepackage[dvipsnames]{xcolor}


% Math stuff
\usepackage{amsmath, amsfonts, mathtools, amsthm, amssymb}
% Fancy script capitals
\usepackage{mathrsfs}
\usepackage{cancel}
% Bold math
\usepackage{bm}
% Some shortcuts
\newcommand{\rr}{\ensuremath{\mathbb{R}}}
\newcommand{\zz}{\ensuremath{\mathbb{Z}}}
\newcommand{\qq}{\ensuremath{\mathbb{Q}}}
\newcommand{\nn}{\ensuremath{\mathbb{N}}}
\newcommand{\ff}{\ensuremath{\mathbb{F}}}
\newcommand{\cc}{\ensuremath{\mathbb{C}}}
\renewcommand\O{\ensuremath{\emptyset}}
\newcommand{\norm}[1]{{\left\lVert{#1}\right\rVert}}
\renewcommand{\vec}[1]{{\mathbf{#1}}}
\newcommand\allbold[1]{{\boldmath\textbf{#1}}}

% Put x \to \infty below \lim
\let\svlim\lim\def\lim{\svlim\limits}

%Make implies and impliedby shorter
\let\implies\Rightarrow
\let\impliedby\Leftarrow
\let\iff\Leftrightarrow
\let\epsilon\varepsilon

% Add \contra symbol to denote contradiction
\usepackage{stmaryrd} % for \lightning
\newcommand\contra{\scalebox{1.5}{$\lightning$}}

% \let\phi\varphi

% Command for short corrections
% Usage: 1+1=\correct{3}{2}

\definecolor{correct}{HTML}{009900}
\newcommand\correct[2]{\ensuremath{\:}{\color{red}{#1}}\ensuremath{\to }{\color{correct}{#2}}\ensuremath{\:}}
\newcommand\green[1]{{\color{correct}{#1}}}

% horizontal rule
\newcommand\hr{
    \noindent\rule[0.5ex]{\linewidth}{0.5pt}
}

% hide parts
\newcommand\hide[1]{}

% si unitx
\usepackage{siunitx}
\sisetup{locale = FR}

% Environments
\makeatother
% For box around Definition, Theorem, \ldots
\usepackage[framemethod=TikZ]{mdframed}
\mdfsetup{skipabove=1em,skipbelow=0em}

%definition
\newenvironment{defn}[1][]{%
\ifstrempty{#1}%
{\mdfsetup{%
frametitle={%
\tikz[baseline=(current bounding box.east),outer sep=0pt]
\node[anchor=east,rectangle,fill=Emerald]
{\strut Definition};}}
}%
{\mdfsetup{%
frametitle={%
\tikz[baseline=(current bounding box.east),outer sep=0pt]
\node[anchor=east,rectangle,fill=Emerald]
{\strut Definition:~#1};}}%
}%
\mdfsetup{innertopmargin=10pt,linecolor=Emerald,%
linewidth=2pt,topline=true,%
frametitleaboveskip=\dimexpr-\ht\strutbox\relax
}
\begin{mdframed}[]\relax%
\label{#1}}{\end{mdframed}}


%theorem
%\newcounter{thm}[section]\setcounter{thm}{0}
%\renewcommand{\thethm}{\arabic{section}.\arabic{thm}}
\newenvironment{thm}[1][]{%
%\refstepcounter{thm}%
\ifstrempty{#1}%
{\mdfsetup{%
frametitle={%
\tikz[baseline=(current bounding box.east),outer sep=0pt]
\node[anchor=east,rectangle,fill=blue!20]
%{\strut Theorem~\thethm};}}
{\strut Theorem};}}
}%
{\mdfsetup{%
frametitle={%
\tikz[baseline=(current bounding box.east),outer sep=0pt]
\node[anchor=east,rectangle,fill=blue!20]
%{\strut Theorem~\thethm:~#1};}}%
{\strut Theorem:~#1};}}%
}%
\mdfsetup{innertopmargin=10pt,linecolor=blue!20,%
linewidth=2pt,topline=true,%
frametitleaboveskip=\dimexpr-\ht\strutbox\relax
}
\begin{mdframed}[]\relax%
\label{#1}}{\end{mdframed}}


%lemma
\newenvironment{lem}[1][]{%
\ifstrempty{#1}%
{\mdfsetup{%
frametitle={%
\tikz[baseline=(current bounding box.east),outer sep=0pt]
\node[anchor=east,rectangle,fill=Dandelion]
{\strut Lemma};}}
}%
{\mdfsetup{%
frametitle={%
\tikz[baseline=(current bounding box.east),outer sep=0pt]
\node[anchor=east,rectangle,fill=Dandelion]
{\strut Lemma:~#1};}}%
}%
\mdfsetup{innertopmargin=10pt,linecolor=Dandelion,%
linewidth=2pt,topline=true,%
frametitleaboveskip=\dimexpr-\ht\strutbox\relax
}
\begin{mdframed}[]\relax%
\label{#1}}{\end{mdframed}}

%corollary
\newenvironment{coro}[1][]{%
\ifstrempty{#1}%
{\mdfsetup{%
frametitle={%
\tikz[baseline=(current bounding box.east),outer sep=0pt]
\node[anchor=east,rectangle,fill=CornflowerBlue]
{\strut Corollary};}}
}%
{\mdfsetup{%
frametitle={%
\tikz[baseline=(current bounding box.east),outer sep=0pt]
\node[anchor=east,rectangle,fill=CornflowerBlue]
{\strut Corollary:~#1};}}%
}%
\mdfsetup{innertopmargin=10pt,linecolor=CornflowerBlue,%
linewidth=2pt,topline=true,%
frametitleaboveskip=\dimexpr-\ht\strutbox\relax
}
\begin{mdframed}[]\relax%
\label{#1}}{\end{mdframed}}

%proof
\newenvironment{prf}[1][]{%
\ifstrempty{#1}%
{\mdfsetup{%
frametitle={%
\tikz[baseline=(current bounding box.east),outer sep=0pt]
\node[anchor=east,rectangle,fill=SpringGreen]
{\strut Proof};}}
}%
{\mdfsetup{%
frametitle={%
\tikz[baseline=(current bounding box.east),outer sep=0pt]
\node[anchor=east,rectangle,fill=SpringGreen]
{\strut Proof:~#1};}}%
}%
\mdfsetup{innertopmargin=10pt,linecolor=SpringGreen,%
linewidth=2pt,topline=true,%
frametitleaboveskip=\dimexpr-\ht\strutbox\relax
}
\begin{mdframed}[]\relax%
\label{#1}}{\qed\end{mdframed}}


\theoremstyle{definition}

\newmdtheoremenv[nobreak=true]{definition}{Definition}
\newmdtheoremenv[nobreak=true]{prop}{Proposition}
\newmdtheoremenv[nobreak=true]{theorem}{Theorem}
\newmdtheoremenv[nobreak=true]{corollary}{Corollary}
\newtheorem*{eg}{Example}
\theoremstyle{remark}
\newtheorem*{case}{Case}
\newtheorem*{notation}{Notation}
\newtheorem*{remark}{Remark}
\newtheorem*{note}{Note}
\newtheorem*{problem}{Problem}
\newtheorem*{observe}{Observe}
\newtheorem*{property}{Property}
\newtheorem*{intuition}{Intuition}


% End example and intermezzo environments with a small diamond (just like proof
% environments end with a small square)
\usepackage{etoolbox}
\AtEndEnvironment{vb}{\null\hfill$\diamond$}%
\AtEndEnvironment{intermezzo}{\null\hfill$\diamond$}%
% \AtEndEnvironment{opmerking}{\null\hfill$\diamond$}%

% Fix some spacing
% http://tex.stackexchange.com/questions/22119/how-can-i-change-the-spacing-before-theorems-with-amsthm
\makeatletter
\def\thm@space@setup{%
  \thm@preskip=\parskip \thm@postskip=0pt
}

% Fix some stuff
% %http://tex.stackexchange.com/questions/76273/multiple-pdfs-with-page-group-included-in-a-single-page-warning
\pdfsuppresswarningpagegroup=1


% My name
\author{Jaden Wang}



\begin{document}
\begin{eg}[continued]
According to Fourier's Law of Heat Conduction, the steady state heat flux is
\[
	\overline{\Phi}(x) = -k \frac{d }{d x} \overline{u}(x) = - \left( \frac{T_2-T_1}{L} \right)  
.\] 
\end{eg}

\begin{eg}[]
	Given a heat source $ Q(x,t)$, consider the equation
	 \[
		 k \frac{\partial^2 u}{\partial x^2} + Q(x,t) = 0 \implies \frac{\partial^2 u}{\partial {x}^2}= -\frac{1}{k} Q(x,t) \implies u \text{ may be a function of } t  
	.\] 
	So if $ \frac{\partial Q}{\partial t} \neq 0 $ we will have no steady state solution.
\end{eg}
\begin{eg}[]
	Suppose $ Q(x,t)=M$ and consider  $ \frac{\partial^2 u}{\partial {x}^2} +M=0 $ with $ u(0)=T_1$ and $ u(L)=T_2$ then 
	\[
		u''(x)=-M \implies u(x)=-\frac{Mx^2}{2}+C_1x+C_2
	.\] 
and the boundary conditions imply the equilibrium solution is
\[
	\overline{u}(x) = T_1 + \left( \frac{T_2-T_1}{L}+\frac{ML}{2} \right) x - \frac{Mx^2}{2}
.\] 
\end{eg}


\subsection{Insulated Boundaries}

Consider the PDE with the BCs and IC:
\[
	\frac{\partial u}{\partial t} =k \frac{\partial^2 u}{\partial {x}^2}, u(x,0)=f(x), \frac{\partial u}{\partial x} (0,t)=0, \frac{\partial u}{\partial x} (L,t)=0 
.\] 
where IC is when $ t=0$, and BCs are  $ x=0$ and  $ x=L$, which have zero values and means these are insulated boundaries. \\

With regards to the steady state solution, if we assume $ u(x,t)=u(x)$ then  $ u(x)=C_1 x + C_2$ and using the BCs we have
\[
	u'(x)=C_1 \implies C_1=u'(0)=\frac{\partial u}{\partial x} (0,t) =0 \implies u(x)=C_2
.\]

and we expect that $ \overline{u}(x) = \lim_{ t \to \infty} u(x,t) = \lim_{ t \to \infty} u(x)= C_2$. Rewriting the original heat equation and integrating both sides yields:
\[
	c\rho\frac{\partial u}{\partial t} =K_0 \frac{\partial^2 u}{\partial {x}^2} \implies \int_{0}^{L} c\rho \frac{\partial u}{\partial x} dx= \int_{0}^{L}  K_0 \frac{\partial^2 u}{\partial {x}^2} dx = K_0\left[ \frac{\partial u}{\partial x} (L,t) -\frac{\partial u}{\partial x} (0,t) \right] =0 
.\]
and multiplying by the constant area $ A$ and interchanging the derivative and integral yields
 \[
	 \int_{0}^{L} c\rho \frac{\partial u}{\partial x} A dx = 0 \implies \frac{d }{d t}  \left[  \int_{0}^{L}  c\rho u(x,t) A dx \right] =0 
.\] 
This implies that the total thermal energy is constant wrt time.

Using IC,
\[
	\int_{0}^{L}  c\rho u(x,0) A dx= \int_{0}^{L}  c\rho f(x) A dx
.\] 
And the equilibrium thermal energy is
\[
	\lim_{ t \to \infty} \int_{0}^{L}  c\rho u(x,t) A dx = \int_{0}^{L}  c\rho \left[ \lim_{ t \to \infty} u(x,t) \right] A dx = \int_{0}^{L}  c\rho C_2 A dx = C_2 c \rho AL
.\] 
setting the initial and equilibrium thermal energy equal to each other and solving yields
\[
	c\rho A \int_{0}^{L}  f(x) dx = c\rho A C_2 L \implies C_2 = \frac{1}{L} \int_{0}^{L} f(x) dx 
.\] 
So the equilibrium solution to the heat equation with insulated boundaries is the \emph{average value of the initial temperature} f(x) over the interval $ [0,L]$ . 

\subsection{Heat Equation in 3D}

When can we switch integration and differentiation for partial differential equations?
\begin{thm}[]
Suppose
\begin{enumerate}[label=\arabic*)]
	\item $ u(x,t)$ is defined for  $ a\leq x \leq b, c\leq t\leq d$.
	\item $u(x,t)$ is Riemann integrable for every  $ t \in [c,d]$
	\item $\partial_t u(x,t)$ is continuous for $ (x,t) \in [a,b] \times [c,d]$
\end{enumerate}
then $ \partial_t u(x,t)$ is Riemann integrable for every  $ t \in [c,d]$, and 
\[
	\frac{d }{d t} \int_{ a}^{ b} u(x,t) dx = \int_{ a}^{ b} \frac{\partial u}{\partial x} dx   
.\] 
\end{thm}
\begin{note}[]
We can replace the closed intervals above with $ \rr$.
\end{note}
\subsection{Boundary Heat Flux}
Let's generalize our result from 1D to 3D:

\begin{thm}[Fourier's law of heat conduction 3D]
\[
\vv{\phi}(x)(\allbold{x},t) = -K_0 \nabla u(\allbold{x},t)
\]
where $\nabla u$ is the spatial gradient of $u$ and $K_0$ is the thermal conductivity constant.
\end{thm}

Consider a region $R$ with closed boundary $\partial R$ and outward unit normal vector $\allbold{n}$. Let $ \vv{\phi}(x)$ be the heat flux vector which specifies the direction of heat flow at the point \allbold{x}= $ (x,y,z)$. Then the magnitude is the flux, and direction is the normal to the surface area. So the heat energy flowing across boundaries per unit time is:
 \[
	 -\iint_{\partial R} \vv{\phi}(x) \cdot \allbold{n} \ dS
.\] 
if the dot product is positive then heat is flowing out of the object and the total energy would decrease.

Then the heat flow process is:
\[
	\frac{d }{d t} \iiint_R c( \allbold{x}) \rho( \allbold{x} ) u( \allbold{x},t) dV = -\iint_{\partial R} \vv{\phi}( \allbold{x} ) \cdot \allbold{n}(\allbold{x})  dS + \iiint_R Q( \allbold{x},t ) dV
.\] 
Recall the \allbold{Divergence Theorem}, we have
\[
	\iint_{\partial R} \vv{\phi}( \allbold{x}) \cdot \allbold{n}( \allbold{x}) dS = \iiint_R\nabla \cdot \vv{\phi}( \allbold{x}) dV 
.\] 
where $ \nabla = \langle \partial_x, \partial_y, \partial_z \rangle$. Now we bring the derivative inside the integral:
\[
	\iiint_R c( \allbold{x} ) \rho( \allbold{x}) \frac{\partial }{\partial t} u( \allbold{x},t) dV = -\iiint_R \nabla \cdot \vv{\phi}( \allbold{x})  + \iiint_R Q( \allbold{x},t ) dV
.\] 
and combining all the triple integrals on the left hand side yields
\[
	c( \allbold{x}) \rho( \allbold{x}) \frac{\partial }{\partial t} u( \allbold{x},t) + \nabla \cdot \vv{ \phi} ( \allbold{x}) - Q( \allbold{x},t) =0     
.\]
where the last equality follows from that the integral equation is true for any region $ R$ and by continuity. 
\end{document}

\documentclass[class=article,crop=false]{standalone} 
%Fall 2020
% Some basic packages
\usepackage{standalone}[subpreambles=true]
\usepackage[utf8]{inputenc}
\usepackage[T1]{fontenc}
\usepackage{textcomp}
\usepackage[english]{babel}
\usepackage{url}
\usepackage{graphicx}
\usepackage{float}
\usepackage{enumitem}


\pdfminorversion=7

% Don't indent paragraphs, leave some space between them
\usepackage{parskip}

% Hide page number when page is empty
\usepackage{emptypage}
\usepackage{subcaption}
\usepackage{multicol}
\usepackage[dvipsnames]{xcolor}


% Math stuff
\usepackage{amsmath, amsfonts, mathtools, amsthm, amssymb}
% Fancy script capitals
\usepackage{mathrsfs}
\usepackage{cancel}
% Bold math
\usepackage{bm}
% Some shortcuts
\newcommand{\rr}{\ensuremath{\mathbb{R}}}
\newcommand{\zz}{\ensuremath{\mathbb{Z}}}
\newcommand{\qq}{\ensuremath{\mathbb{Q}}}
\newcommand{\nn}{\ensuremath{\mathbb{N}}}
\newcommand{\ff}{\ensuremath{\mathbb{F}}}
\newcommand{\cc}{\ensuremath{\mathbb{C}}}
\renewcommand\O{\ensuremath{\emptyset}}
\newcommand{\norm}[1]{{\left\lVert{#1}\right\rVert}}
\renewcommand{\vec}[1]{{\mathbf{#1}}}
\newcommand\allbold[1]{{\boldmath\textbf{#1}}}

% Put x \to \infty below \lim
\let\svlim\lim\def\lim{\svlim\limits}

%Make implies and impliedby shorter
\let\implies\Rightarrow
\let\impliedby\Leftarrow
\let\iff\Leftrightarrow
\let\epsilon\varepsilon

% Add \contra symbol to denote contradiction
\usepackage{stmaryrd} % for \lightning
\newcommand\contra{\scalebox{1.5}{$\lightning$}}

% \let\phi\varphi

% Command for short corrections
% Usage: 1+1=\correct{3}{2}

\definecolor{correct}{HTML}{009900}
\newcommand\correct[2]{\ensuremath{\:}{\color{red}{#1}}\ensuremath{\to }{\color{correct}{#2}}\ensuremath{\:}}
\newcommand\green[1]{{\color{correct}{#1}}}

% horizontal rule
\newcommand\hr{
    \noindent\rule[0.5ex]{\linewidth}{0.5pt}
}

% hide parts
\newcommand\hide[1]{}

% si unitx
\usepackage{siunitx}
\sisetup{locale = FR}

% Environments
\makeatother
% For box around Definition, Theorem, \ldots
\usepackage[framemethod=TikZ]{mdframed}
\mdfsetup{skipabove=1em,skipbelow=0em}

%definition
\newenvironment{defn}[1][]{%
\ifstrempty{#1}%
{\mdfsetup{%
frametitle={%
\tikz[baseline=(current bounding box.east),outer sep=0pt]
\node[anchor=east,rectangle,fill=Emerald]
{\strut Definition};}}
}%
{\mdfsetup{%
frametitle={%
\tikz[baseline=(current bounding box.east),outer sep=0pt]
\node[anchor=east,rectangle,fill=Emerald]
{\strut Definition:~#1};}}%
}%
\mdfsetup{innertopmargin=10pt,linecolor=Emerald,%
linewidth=2pt,topline=true,%
frametitleaboveskip=\dimexpr-\ht\strutbox\relax
}
\begin{mdframed}[]\relax%
\label{#1}}{\end{mdframed}}


%theorem
%\newcounter{thm}[section]\setcounter{thm}{0}
%\renewcommand{\thethm}{\arabic{section}.\arabic{thm}}
\newenvironment{thm}[1][]{%
%\refstepcounter{thm}%
\ifstrempty{#1}%
{\mdfsetup{%
frametitle={%
\tikz[baseline=(current bounding box.east),outer sep=0pt]
\node[anchor=east,rectangle,fill=blue!20]
%{\strut Theorem~\thethm};}}
{\strut Theorem};}}
}%
{\mdfsetup{%
frametitle={%
\tikz[baseline=(current bounding box.east),outer sep=0pt]
\node[anchor=east,rectangle,fill=blue!20]
%{\strut Theorem~\thethm:~#1};}}%
{\strut Theorem:~#1};}}%
}%
\mdfsetup{innertopmargin=10pt,linecolor=blue!20,%
linewidth=2pt,topline=true,%
frametitleaboveskip=\dimexpr-\ht\strutbox\relax
}
\begin{mdframed}[]\relax%
\label{#1}}{\end{mdframed}}


%lemma
\newenvironment{lem}[1][]{%
\ifstrempty{#1}%
{\mdfsetup{%
frametitle={%
\tikz[baseline=(current bounding box.east),outer sep=0pt]
\node[anchor=east,rectangle,fill=Dandelion]
{\strut Lemma};}}
}%
{\mdfsetup{%
frametitle={%
\tikz[baseline=(current bounding box.east),outer sep=0pt]
\node[anchor=east,rectangle,fill=Dandelion]
{\strut Lemma:~#1};}}%
}%
\mdfsetup{innertopmargin=10pt,linecolor=Dandelion,%
linewidth=2pt,topline=true,%
frametitleaboveskip=\dimexpr-\ht\strutbox\relax
}
\begin{mdframed}[]\relax%
\label{#1}}{\end{mdframed}}

%corollary
\newenvironment{coro}[1][]{%
\ifstrempty{#1}%
{\mdfsetup{%
frametitle={%
\tikz[baseline=(current bounding box.east),outer sep=0pt]
\node[anchor=east,rectangle,fill=CornflowerBlue]
{\strut Corollary};}}
}%
{\mdfsetup{%
frametitle={%
\tikz[baseline=(current bounding box.east),outer sep=0pt]
\node[anchor=east,rectangle,fill=CornflowerBlue]
{\strut Corollary:~#1};}}%
}%
\mdfsetup{innertopmargin=10pt,linecolor=CornflowerBlue,%
linewidth=2pt,topline=true,%
frametitleaboveskip=\dimexpr-\ht\strutbox\relax
}
\begin{mdframed}[]\relax%
\label{#1}}{\end{mdframed}}

%proof
\newenvironment{prf}[1][]{%
\ifstrempty{#1}%
{\mdfsetup{%
frametitle={%
\tikz[baseline=(current bounding box.east),outer sep=0pt]
\node[anchor=east,rectangle,fill=SpringGreen]
{\strut Proof};}}
}%
{\mdfsetup{%
frametitle={%
\tikz[baseline=(current bounding box.east),outer sep=0pt]
\node[anchor=east,rectangle,fill=SpringGreen]
{\strut Proof:~#1};}}%
}%
\mdfsetup{innertopmargin=10pt,linecolor=SpringGreen,%
linewidth=2pt,topline=true,%
frametitleaboveskip=\dimexpr-\ht\strutbox\relax
}
\begin{mdframed}[]\relax%
\label{#1}}{\qed\end{mdframed}}


\theoremstyle{definition}

\newmdtheoremenv[nobreak=true]{definition}{Definition}
\newmdtheoremenv[nobreak=true]{prop}{Proposition}
\newmdtheoremenv[nobreak=true]{theorem}{Theorem}
\newmdtheoremenv[nobreak=true]{corollary}{Corollary}
\newtheorem*{eg}{Example}
\theoremstyle{remark}
\newtheorem*{case}{Case}
\newtheorem*{notation}{Notation}
\newtheorem*{remark}{Remark}
\newtheorem*{note}{Note}
\newtheorem*{problem}{Problem}
\newtheorem*{observe}{Observe}
\newtheorem*{property}{Property}
\newtheorem*{intuition}{Intuition}


% End example and intermezzo environments with a small diamond (just like proof
% environments end with a small square)
\usepackage{etoolbox}
\AtEndEnvironment{vb}{\null\hfill$\diamond$}%
\AtEndEnvironment{intermezzo}{\null\hfill$\diamond$}%
% \AtEndEnvironment{opmerking}{\null\hfill$\diamond$}%

% Fix some spacing
% http://tex.stackexchange.com/questions/22119/how-can-i-change-the-spacing-before-theorems-with-amsthm
\makeatletter
\def\thm@space@setup{%
  \thm@preskip=\parskip \thm@postskip=0pt
}

% Fix some stuff
% %http://tex.stackexchange.com/questions/76273/multiple-pdfs-with-page-group-included-in-a-single-page-warning
\pdfsuppresswarningpagegroup=1


% My name
\author{Jaden Wang}



\begin{document}
\subsection{Hyperbolic functions}
\begin{itemize}
	\item $\cosh( )(x )+\sinh( )(x )=e^{x} \text{ and } \cosh( )(x )-\sinh( )(x )=e^{-x} $ 
	\item $\cosh( )(x )^2-\sinh( )(x )^2=1$
	\item \ldots
\end{itemize}

\subsection{Series}
There are three ways to diverge: 1. goes to $\infty$ 2. goes to $-\infty$ 3. oscillates.\\

\subsubsection{Convergence}
\begin{itemize}
	\item divergent test: if $\lim_{ n \to \infty} a_n \neq 0$, then the series $\sum_{ n=0}^{\infty} a_n$ diverges. 
	\item Geometric Series: Note that if $ |r|<1 $, then  $\sum_{ n=0}^{\infty} a r^{n=\frac{1}{1-r}}$ and diverges otherwise.
	\item Direct Comparison Test: suppose $0 \leq a_n \leq \b_n \forall n \geq 0$, then $0 \leq \sum_{ n=0}^{\infty} a_n \leq \sum_{ n=0}^{\infty} b_n $ and if $\sum_{ n=0}^{\infty} b_n$ converges then $\sum_{ n=0}^{\infty} a_n$ converges. Likewise for divergence.
\end{itemize}
Geometric series:
\begin{prf}
\begin{align*}
	s_N &= a + ar+ ar^2 + \ldots + ar^{N-1} \\
	s_N r &= ar + ar^2+\ldots+ ar^{N} \\
	s_N - s_N r &= a - ar^{N} = a(1-r^{N)}\\
	s_N &= \frac{a(1-r^{N}}{1-r} \\
	\lim_{ N \to \infty} s_N &= \lim_{ N \to \infty} \frac{a(1-r^{N)}}{1-r} = \frac{a}{1-r} \text{ iff |r|<1} 
\end{align*}
\end{prf}

\begin{eg}[Absolutely convergent test]
\begin{prf}
Using the comparison test, we have
$0 \leq a_n + |a_n| \leq 2|a_n| \implies 0 \leq \sum_{ n=0}^{\infty} (a_n+|a_n|) \leq \sum_{ n=0}^{\infty} 2|a_n|$. Since $\sum_{ n=0}^{\infty} |a_n|$ converges, there exists some finite number $L$ such that $\sum_{ n=0}^{\infty} |a_n|=L$, which implies $\sum_{ n=0}^{\infty} 2|a_n| = 2L$ so $\sum_{ n=0}^{\infty} 2|a_n|$ converges. Thus, $\sum_{ n=0}^{\infty} (a_n+|a_n|)$ converges by comparison test. Finally,
\[
	\sum_{ n=0}^{\infty} a_n = \sum_{ n=0}^{\infty} (a_n+|a_n|) - \sum_{ n=0}^{\infty} |a_n|
.\] 
Since both terms on the RHS are finite, their difference is finite and therefore the original series converges.
\end{prf}
\end{eg}

\begin{note}[]
	$\left| \sum_{ n=0}^{\infty} a_n \right| \leq \sum_{ n=0}^{\infty} |a_n| $ doesn't guarantee convergence because it can be oscillating divergence.
\end{note}

\subsubsection{Rearrangement}
\begin{itemize}
	\item The real numbers possesses a property known as the \emph{communitive property of addition} which states that the order in which we form a finite sum doesn't matter.
	\item Given a series $\sum_{ n=1}^{\infty} a_n$ with partial sums $(s_N)$, if we formulate the sum in a different order then this results in a different series and is known as a \emph{rearrangement} of the original series.
\end{itemize}

\begin{thm}[]
Let $ \sum a_n$ be a series of real numbers which converges but not absolutely. Suppose $-\infty \leq \alpha\leq \beta \leq \infty$. Then there exists rearrangements of the original series, say, $\sum \hat{\alpha}_n$ and $\sum \hat{\beta}_n$, such that
\[
\sum_{ n=1}^{\infty} \hat{\alpha}_n \text{ and } \sum_{ n=1}^{\infty} \hat{\beta}_n \text{ where } -\infty\leq \alpha \leq \beta \leq \infty  
.\] 
\end{thm}

\subsubsection{Ratio test}
Suppose $\lim_{ n \to \infty} \left| \frac{a_{n+1}}{a_n} =R \right| $. If $R<1$, then the series  $\sum_{ n=0}^{\infty} a_n$ converges absolutely, if $R>1$ then the series diverges and if  $R=1$ the it's inconclusive.\\

Use ratio test to determine the convergence of Taylor series, which is "radius of convergence".


\end{document}

\documentclass[class=article,crop=false]{standalone} 
%Fall 2020
% Some basic packages
\usepackage{standalone}[subpreambles=true]
\usepackage[utf8]{inputenc}
\usepackage[T1]{fontenc}
\usepackage{textcomp}
\usepackage[english]{babel}
\usepackage{url}
\usepackage{graphicx}
\usepackage{float}
\usepackage{enumitem}


\pdfminorversion=7

% Don't indent paragraphs, leave some space between them
\usepackage{parskip}

% Hide page number when page is empty
\usepackage{emptypage}
\usepackage{subcaption}
\usepackage{multicol}
\usepackage[dvipsnames]{xcolor}


% Math stuff
\usepackage{amsmath, amsfonts, mathtools, amsthm, amssymb}
% Fancy script capitals
\usepackage{mathrsfs}
\usepackage{cancel}
% Bold math
\usepackage{bm}
% Some shortcuts
\newcommand{\rr}{\ensuremath{\mathbb{R}}}
\newcommand{\zz}{\ensuremath{\mathbb{Z}}}
\newcommand{\qq}{\ensuremath{\mathbb{Q}}}
\newcommand{\nn}{\ensuremath{\mathbb{N}}}
\newcommand{\ff}{\ensuremath{\mathbb{F}}}
\newcommand{\cc}{\ensuremath{\mathbb{C}}}
\renewcommand\O{\ensuremath{\emptyset}}
\newcommand{\norm}[1]{{\left\lVert{#1}\right\rVert}}
\renewcommand{\vec}[1]{{\mathbf{#1}}}
\newcommand\allbold[1]{{\boldmath\textbf{#1}}}

% Put x \to \infty below \lim
\let\svlim\lim\def\lim{\svlim\limits}

%Make implies and impliedby shorter
\let\implies\Rightarrow
\let\impliedby\Leftarrow
\let\iff\Leftrightarrow
\let\epsilon\varepsilon

% Add \contra symbol to denote contradiction
\usepackage{stmaryrd} % for \lightning
\newcommand\contra{\scalebox{1.5}{$\lightning$}}

% \let\phi\varphi

% Command for short corrections
% Usage: 1+1=\correct{3}{2}

\definecolor{correct}{HTML}{009900}
\newcommand\correct[2]{\ensuremath{\:}{\color{red}{#1}}\ensuremath{\to }{\color{correct}{#2}}\ensuremath{\:}}
\newcommand\green[1]{{\color{correct}{#1}}}

% horizontal rule
\newcommand\hr{
    \noindent\rule[0.5ex]{\linewidth}{0.5pt}
}

% hide parts
\newcommand\hide[1]{}

% si unitx
\usepackage{siunitx}
\sisetup{locale = FR}

% Environments
\makeatother
% For box around Definition, Theorem, \ldots
\usepackage[framemethod=TikZ]{mdframed}
\mdfsetup{skipabove=1em,skipbelow=0em}

%definition
\newenvironment{defn}[1][]{%
\ifstrempty{#1}%
{\mdfsetup{%
frametitle={%
\tikz[baseline=(current bounding box.east),outer sep=0pt]
\node[anchor=east,rectangle,fill=Emerald]
{\strut Definition};}}
}%
{\mdfsetup{%
frametitle={%
\tikz[baseline=(current bounding box.east),outer sep=0pt]
\node[anchor=east,rectangle,fill=Emerald]
{\strut Definition:~#1};}}%
}%
\mdfsetup{innertopmargin=10pt,linecolor=Emerald,%
linewidth=2pt,topline=true,%
frametitleaboveskip=\dimexpr-\ht\strutbox\relax
}
\begin{mdframed}[]\relax%
\label{#1}}{\end{mdframed}}


%theorem
%\newcounter{thm}[section]\setcounter{thm}{0}
%\renewcommand{\thethm}{\arabic{section}.\arabic{thm}}
\newenvironment{thm}[1][]{%
%\refstepcounter{thm}%
\ifstrempty{#1}%
{\mdfsetup{%
frametitle={%
\tikz[baseline=(current bounding box.east),outer sep=0pt]
\node[anchor=east,rectangle,fill=blue!20]
%{\strut Theorem~\thethm};}}
{\strut Theorem};}}
}%
{\mdfsetup{%
frametitle={%
\tikz[baseline=(current bounding box.east),outer sep=0pt]
\node[anchor=east,rectangle,fill=blue!20]
%{\strut Theorem~\thethm:~#1};}}%
{\strut Theorem:~#1};}}%
}%
\mdfsetup{innertopmargin=10pt,linecolor=blue!20,%
linewidth=2pt,topline=true,%
frametitleaboveskip=\dimexpr-\ht\strutbox\relax
}
\begin{mdframed}[]\relax%
\label{#1}}{\end{mdframed}}


%lemma
\newenvironment{lem}[1][]{%
\ifstrempty{#1}%
{\mdfsetup{%
frametitle={%
\tikz[baseline=(current bounding box.east),outer sep=0pt]
\node[anchor=east,rectangle,fill=Dandelion]
{\strut Lemma};}}
}%
{\mdfsetup{%
frametitle={%
\tikz[baseline=(current bounding box.east),outer sep=0pt]
\node[anchor=east,rectangle,fill=Dandelion]
{\strut Lemma:~#1};}}%
}%
\mdfsetup{innertopmargin=10pt,linecolor=Dandelion,%
linewidth=2pt,topline=true,%
frametitleaboveskip=\dimexpr-\ht\strutbox\relax
}
\begin{mdframed}[]\relax%
\label{#1}}{\end{mdframed}}

%corollary
\newenvironment{coro}[1][]{%
\ifstrempty{#1}%
{\mdfsetup{%
frametitle={%
\tikz[baseline=(current bounding box.east),outer sep=0pt]
\node[anchor=east,rectangle,fill=CornflowerBlue]
{\strut Corollary};}}
}%
{\mdfsetup{%
frametitle={%
\tikz[baseline=(current bounding box.east),outer sep=0pt]
\node[anchor=east,rectangle,fill=CornflowerBlue]
{\strut Corollary:~#1};}}%
}%
\mdfsetup{innertopmargin=10pt,linecolor=CornflowerBlue,%
linewidth=2pt,topline=true,%
frametitleaboveskip=\dimexpr-\ht\strutbox\relax
}
\begin{mdframed}[]\relax%
\label{#1}}{\end{mdframed}}

%proof
\newenvironment{prf}[1][]{%
\ifstrempty{#1}%
{\mdfsetup{%
frametitle={%
\tikz[baseline=(current bounding box.east),outer sep=0pt]
\node[anchor=east,rectangle,fill=SpringGreen]
{\strut Proof};}}
}%
{\mdfsetup{%
frametitle={%
\tikz[baseline=(current bounding box.east),outer sep=0pt]
\node[anchor=east,rectangle,fill=SpringGreen]
{\strut Proof:~#1};}}%
}%
\mdfsetup{innertopmargin=10pt,linecolor=SpringGreen,%
linewidth=2pt,topline=true,%
frametitleaboveskip=\dimexpr-\ht\strutbox\relax
}
\begin{mdframed}[]\relax%
\label{#1}}{\qed\end{mdframed}}


\theoremstyle{definition}

\newmdtheoremenv[nobreak=true]{definition}{Definition}
\newmdtheoremenv[nobreak=true]{prop}{Proposition}
\newmdtheoremenv[nobreak=true]{theorem}{Theorem}
\newmdtheoremenv[nobreak=true]{corollary}{Corollary}
\newtheorem*{eg}{Example}
\theoremstyle{remark}
\newtheorem*{case}{Case}
\newtheorem*{notation}{Notation}
\newtheorem*{remark}{Remark}
\newtheorem*{note}{Note}
\newtheorem*{problem}{Problem}
\newtheorem*{observe}{Observe}
\newtheorem*{property}{Property}
\newtheorem*{intuition}{Intuition}


% End example and intermezzo environments with a small diamond (just like proof
% environments end with a small square)
\usepackage{etoolbox}
\AtEndEnvironment{vb}{\null\hfill$\diamond$}%
\AtEndEnvironment{intermezzo}{\null\hfill$\diamond$}%
% \AtEndEnvironment{opmerking}{\null\hfill$\diamond$}%

% Fix some spacing
% http://tex.stackexchange.com/questions/22119/how-can-i-change-the-spacing-before-theorems-with-amsthm
\makeatletter
\def\thm@space@setup{%
  \thm@preskip=\parskip \thm@postskip=0pt
}

% Fix some stuff
% %http://tex.stackexchange.com/questions/76273/multiple-pdfs-with-page-group-included-in-a-single-page-warning
\pdfsuppresswarningpagegroup=1


% My name
\author{Jaden Wang}



\begin{document}

\section{Review}
\subsection{2nd order ODE}
\begin{eg}[]
Solve \[
\frac{d^2 y}{d {x }^2} = \lambda y
.\] 
The general solution is:
\[
	y(x) = c_1 e^{\sqrt{\lambda} x} + c_2 e^{-\sqrt{\lambda} x}
.\]
Or we can write it as:
\[
	y(x)= c_1 \cosh(\sqrt{\lambda} x  ) + c_2 \sinh(\sqrt{\lambda} x  )
.\] 
\begin{note}[]
Hyperbolic functions have easy derivatives and nice for initial conditions.
\end{note}
\end{eg}

\begin{defn}[linear independence]
	The functions $y_1(x),\ldots,y_n(x)$ are \allbold{linearly independent} if 
	\[
		c_1 y_1(x) + c_2 y_2(x) + \ldots + c_n y_n(x) = 0 \implies c1=c2=\ldots=c_n=0
	.\] 
\end{defn}
\subsection{The complex plane}
The \emph{modulus} of a complex number $a+bi$ is
 \[
|z| = \sqrt{z \cdot \overline{z} } = \sqrt{a^2+b^2}  
.\] 
\subsection{Euler's formula}

~\begin{prf}
\begin{align*}
	e^{x} &= \sum_{ n=0}^{\infty} \frac{x^{n}}{n!} \\
	e^{i\theta} &= 1 + i\theta + \frac{(i\theta)^2}{2!} + \ldots\\
	&= 1 + i\theta + \frac{-\theta^2}{2!} + \frac{-i\theta^3}{3!} + \frac{\theta^{4}}{4!} + \ldots \\
	&= (1-\frac{\theta^2}{2!}+\frac{\theta^{4}}{4!}-\ldots) + i(\theta - \frac{\theta^3}{3!} + \ldots)\\
	&= \cos(\theta ) + i \sin(\theta )
\end{align*}
\end{prf}
\begin{note}[]
$\left| e^{i\theta} \right| = 1$ so it's on the unit circle. Moreover,
\[
	z=a+ib=\rho \cos(\theta ) + i\rho \sin(\theta ) = \rho e^{i\theta}
.\] 
where $\rho=\sqrt{a^2+b^2} $ and $\tan(\theta) = \frac{b}{a}$.
\end{note}

\newpage

\section{Fourier Series and Orthogonal Vectors (ch.1 + 2)}
\begin{defn}[L2 inner product]
	Let $f(x)$ and  $g(x)$ be continuous functions defined on  $[a,b]$, we defined the  \allbold{$L^2$-inner product} on $[a,b]$ to be 
	 \[
		 \langle f,g \rangle = \int_{ a}^{ b} f(x)g(x)dx
	.\] 
	with the corresponding $L^2$ norm,
	\[
		\norm{f}_2 = \sqrt{\langle f,f \rangle} = \left( \int_{ a}^{ b} f^2(x) dx  \right)^{\frac{1}{2}}  
	.\] 
\end{defn}
\begin{defn}[Fourier basis]
Suppose $-\pi\leq z \leq \pi$, the \allbold{Fourier basis} is defined as 
\[
	\{1,\cos(z), \sin(z ),\cos(2z ),\sin(2z ),\ldots\}
.\] 
This is an infinite, mutually orthogonal basis of the vector space of continuous functions on $[-\pi,\pi]$.
\end{defn}

\begin{defn}[projection]
Suppose $\{\vec{e}_1, \vec{e}_2 \ldots \} $ are an orthogonal basis, then $v_i = \frac{\langle \vec{v},\vec{e}_i   \rangle}{\norm{\vec{e}_i}^2 }$.
\end{defn}

\begin{defn}[Fourier series]
	Suppose $f(z)$ is defined on  $[-\pi,\pi]$ and is in "the proper space of functions" (see Ch.5 notes) then $f(z)$ has a Fourier series is of the form
	 \[
		 f(z)=a_0 + \sum_{ n=1}^{\infty} a_n \cos(nz ) + \sum_{ n=1}^{\infty} b_n \sin(nz )
	.\]
	where 
	\[
		b_n = \frac{\langle f(z),\sin(nz ) \rangle}{\norm{\sin(nz )}_2^2 } = \frac{1}{\pi} \int_{ -\pi}^{ \pi} f(z)\sin(nz )dz \text{ for } n=1,2,\ldots 
	.\]
	\[
		a_n = \frac{\langle f(z),\cos(nz ) \rangle}{\norm{\cos(nz )}_2^2 } = \frac{1}{\pi} \int_{ -\pi}^{ \pi} f(z)\cos(nz ) dz \text{ for } n=1,2,\ldots 
	.\]
	\[
		a_0 = \frac{\langle f(z),1 \rangle}{\norm{1}_2^2 }= \frac{1}{2\pi} \int_{ -\pi}^{ \pi} f(z) dz 
	.\] 
\end{defn}

\end{document}

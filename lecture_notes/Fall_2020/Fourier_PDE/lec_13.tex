\documentclass[class=article,crop=false]{standalone} 
%Fall 2020
% Some basic packages
\usepackage{standalone}[subpreambles=true]
\usepackage[utf8]{inputenc}
\usepackage[T1]{fontenc}
\usepackage{textcomp}
\usepackage[english]{babel}
\usepackage{url}
\usepackage{graphicx}
\usepackage{float}
\usepackage{enumitem}


\pdfminorversion=7

% Don't indent paragraphs, leave some space between them
\usepackage{parskip}

% Hide page number when page is empty
\usepackage{emptypage}
\usepackage{subcaption}
\usepackage{multicol}
\usepackage[dvipsnames]{xcolor}


% Math stuff
\usepackage{amsmath, amsfonts, mathtools, amsthm, amssymb}
% Fancy script capitals
\usepackage{mathrsfs}
\usepackage{cancel}
% Bold math
\usepackage{bm}
% Some shortcuts
\newcommand{\rr}{\ensuremath{\mathbb{R}}}
\newcommand{\zz}{\ensuremath{\mathbb{Z}}}
\newcommand{\qq}{\ensuremath{\mathbb{Q}}}
\newcommand{\nn}{\ensuremath{\mathbb{N}}}
\newcommand{\ff}{\ensuremath{\mathbb{F}}}
\newcommand{\cc}{\ensuremath{\mathbb{C}}}
\renewcommand\O{\ensuremath{\emptyset}}
\newcommand{\norm}[1]{{\left\lVert{#1}\right\rVert}}
\renewcommand{\vec}[1]{{\mathbf{#1}}}
\newcommand\allbold[1]{{\boldmath\textbf{#1}}}

% Put x \to \infty below \lim
\let\svlim\lim\def\lim{\svlim\limits}

%Make implies and impliedby shorter
\let\implies\Rightarrow
\let\impliedby\Leftarrow
\let\iff\Leftrightarrow
\let\epsilon\varepsilon

% Add \contra symbol to denote contradiction
\usepackage{stmaryrd} % for \lightning
\newcommand\contra{\scalebox{1.5}{$\lightning$}}

% \let\phi\varphi

% Command for short corrections
% Usage: 1+1=\correct{3}{2}

\definecolor{correct}{HTML}{009900}
\newcommand\correct[2]{\ensuremath{\:}{\color{red}{#1}}\ensuremath{\to }{\color{correct}{#2}}\ensuremath{\:}}
\newcommand\green[1]{{\color{correct}{#1}}}

% horizontal rule
\newcommand\hr{
    \noindent\rule[0.5ex]{\linewidth}{0.5pt}
}

% hide parts
\newcommand\hide[1]{}

% si unitx
\usepackage{siunitx}
\sisetup{locale = FR}

% Environments
\makeatother
% For box around Definition, Theorem, \ldots
\usepackage[framemethod=TikZ]{mdframed}
\mdfsetup{skipabove=1em,skipbelow=0em}

%definition
\newenvironment{defn}[1][]{%
\ifstrempty{#1}%
{\mdfsetup{%
frametitle={%
\tikz[baseline=(current bounding box.east),outer sep=0pt]
\node[anchor=east,rectangle,fill=Emerald]
{\strut Definition};}}
}%
{\mdfsetup{%
frametitle={%
\tikz[baseline=(current bounding box.east),outer sep=0pt]
\node[anchor=east,rectangle,fill=Emerald]
{\strut Definition:~#1};}}%
}%
\mdfsetup{innertopmargin=10pt,linecolor=Emerald,%
linewidth=2pt,topline=true,%
frametitleaboveskip=\dimexpr-\ht\strutbox\relax
}
\begin{mdframed}[]\relax%
\label{#1}}{\end{mdframed}}


%theorem
%\newcounter{thm}[section]\setcounter{thm}{0}
%\renewcommand{\thethm}{\arabic{section}.\arabic{thm}}
\newenvironment{thm}[1][]{%
%\refstepcounter{thm}%
\ifstrempty{#1}%
{\mdfsetup{%
frametitle={%
\tikz[baseline=(current bounding box.east),outer sep=0pt]
\node[anchor=east,rectangle,fill=blue!20]
%{\strut Theorem~\thethm};}}
{\strut Theorem};}}
}%
{\mdfsetup{%
frametitle={%
\tikz[baseline=(current bounding box.east),outer sep=0pt]
\node[anchor=east,rectangle,fill=blue!20]
%{\strut Theorem~\thethm:~#1};}}%
{\strut Theorem:~#1};}}%
}%
\mdfsetup{innertopmargin=10pt,linecolor=blue!20,%
linewidth=2pt,topline=true,%
frametitleaboveskip=\dimexpr-\ht\strutbox\relax
}
\begin{mdframed}[]\relax%
\label{#1}}{\end{mdframed}}


%lemma
\newenvironment{lem}[1][]{%
\ifstrempty{#1}%
{\mdfsetup{%
frametitle={%
\tikz[baseline=(current bounding box.east),outer sep=0pt]
\node[anchor=east,rectangle,fill=Dandelion]
{\strut Lemma};}}
}%
{\mdfsetup{%
frametitle={%
\tikz[baseline=(current bounding box.east),outer sep=0pt]
\node[anchor=east,rectangle,fill=Dandelion]
{\strut Lemma:~#1};}}%
}%
\mdfsetup{innertopmargin=10pt,linecolor=Dandelion,%
linewidth=2pt,topline=true,%
frametitleaboveskip=\dimexpr-\ht\strutbox\relax
}
\begin{mdframed}[]\relax%
\label{#1}}{\end{mdframed}}

%corollary
\newenvironment{coro}[1][]{%
\ifstrempty{#1}%
{\mdfsetup{%
frametitle={%
\tikz[baseline=(current bounding box.east),outer sep=0pt]
\node[anchor=east,rectangle,fill=CornflowerBlue]
{\strut Corollary};}}
}%
{\mdfsetup{%
frametitle={%
\tikz[baseline=(current bounding box.east),outer sep=0pt]
\node[anchor=east,rectangle,fill=CornflowerBlue]
{\strut Corollary:~#1};}}%
}%
\mdfsetup{innertopmargin=10pt,linecolor=CornflowerBlue,%
linewidth=2pt,topline=true,%
frametitleaboveskip=\dimexpr-\ht\strutbox\relax
}
\begin{mdframed}[]\relax%
\label{#1}}{\end{mdframed}}

%proof
\newenvironment{prf}[1][]{%
\ifstrempty{#1}%
{\mdfsetup{%
frametitle={%
\tikz[baseline=(current bounding box.east),outer sep=0pt]
\node[anchor=east,rectangle,fill=SpringGreen]
{\strut Proof};}}
}%
{\mdfsetup{%
frametitle={%
\tikz[baseline=(current bounding box.east),outer sep=0pt]
\node[anchor=east,rectangle,fill=SpringGreen]
{\strut Proof:~#1};}}%
}%
\mdfsetup{innertopmargin=10pt,linecolor=SpringGreen,%
linewidth=2pt,topline=true,%
frametitleaboveskip=\dimexpr-\ht\strutbox\relax
}
\begin{mdframed}[]\relax%
\label{#1}}{\qed\end{mdframed}}


\theoremstyle{definition}

\newmdtheoremenv[nobreak=true]{definition}{Definition}
\newmdtheoremenv[nobreak=true]{prop}{Proposition}
\newmdtheoremenv[nobreak=true]{theorem}{Theorem}
\newmdtheoremenv[nobreak=true]{corollary}{Corollary}
\newtheorem*{eg}{Example}
\theoremstyle{remark}
\newtheorem*{case}{Case}
\newtheorem*{notation}{Notation}
\newtheorem*{remark}{Remark}
\newtheorem*{note}{Note}
\newtheorem*{problem}{Problem}
\newtheorem*{observe}{Observe}
\newtheorem*{property}{Property}
\newtheorem*{intuition}{Intuition}


% End example and intermezzo environments with a small diamond (just like proof
% environments end with a small square)
\usepackage{etoolbox}
\AtEndEnvironment{vb}{\null\hfill$\diamond$}%
\AtEndEnvironment{intermezzo}{\null\hfill$\diamond$}%
% \AtEndEnvironment{opmerking}{\null\hfill$\diamond$}%

% Fix some spacing
% http://tex.stackexchange.com/questions/22119/how-can-i-change-the-spacing-before-theorems-with-amsthm
\makeatletter
\def\thm@space@setup{%
  \thm@preskip=\parskip \thm@postskip=0pt
}

% Fix some stuff
% %http://tex.stackexchange.com/questions/76273/multiple-pdfs-with-page-group-included-in-a-single-page-warning
\pdfsuppresswarningpagegroup=1


% My name
\author{Jaden Wang}



\begin{document}

\begin{intuition}
	We transform the double integral to triple integral using divergence theorem so we can combine them under the same domain of $ \allbold{ x} $. Now we want to replace flux with temperature function.
\end{intuition}

\begin{defn}[Laplacian]
For 3D, the \allbold{Laplacian} is defined as 
	\[
\nabla ^2 u = \Delta u = u_{xx}+u_{yy}+u_{zz}
.\] 
\end{defn}


Recall Fourier's Law says heat flows from hot to cold in the direction where the temperature differences are the greatest and $ \nabla u$ represents the direction of greatest temperature increases, so
\[
	\vv{ \phi}=-K_0 \cdot  \nabla u \implies \nabla \cdot \vv{ \phi}( \allbold{x} ) = \nabla \cdot (-K_0 \nabla u) = -K_0 \cdot \Delta u
.\] 

Then
\[
	c( \allbold{x} )\rho ( \allbold{x}) \frac{\partial }{\partial t} u( \allbold{x},t ) -K_0 \Delta u - Q( \allbold{x},t )=0
.\] 
Thus \allbold{the heat equation with internal source of energy} is
\[
	c( \allbold{x} )\rho ( \allbold{x}) \frac{\partial }{\partial t} u( \allbold{x},t ) =K_0 \Delta u + Q( \allbold{x},t )
.\] 
Assuming $ Q=0$ and the thermal coefficients are constant, we get
\begin{thm}[3D heat equation]
 \[
\frac{\partial u}{\partial t} = k \Delta u 
.\] 
where $k=\frac{K_0}{c \rho}= $ "thermal diffusivity", with initial condition $ u( \allbold{x},0 )= f ( \allbold{x} )$ and boundary condition $ u( \allbold{x} ,t)= T( \allbold{x},t )$ for $ \allbold{x} \in \partial R $.
\end{thm}
\subsection{Steady State}

~\begin{thm}[Laplace's Equation]
Consider the heat equation with internal source of energy defined above, then if $ u_t=0$ this gives  \allbold{Poisson's Equation}, $ \Delta u = -\frac{Q}{K_0}$, and if $ Q=0$ this yields  \allbold{Laplace's Equation}:
\[
\Delta u = \frac{\partial^2 u}{\partial { x}^2} + \frac{\partial^2 u}{\partial { y}^2} + \frac{\partial^2 u}{\partial { z}^2} =0
.\] 
\end{thm}

\begin{thm}[Laplace's Equation in Cylindrical Coordinates]
Let $  x= r \cos(\theta ), y=r \sin(\theta ), z=z$ then using the Chain Rule, 
\[
	\Delta u = \frac{1}{r} \frac{\partial }{\partial r} \left( r \frac{\partial u}{\partial r}  \right) + \frac{1}{r^2} \frac{\partial^2 u}{\partial { \theta}^2}  + \frac{\partial^2 u}{\partial { z}^2} 
.\] 
\end{thm}

\begin{thm}[Spherical]
\[
	\Delta u = \frac{1}{\rho^2} \frac{\partial }{\partial \rho} \left( \rho^2 \frac{\partial u}{\partial \rho}  \right)+ \frac{1}{\rho^2 \sin(\phi )} \frac{\partial }{\partial \phi} \left( \sin(\phi )\frac{\partial u}{\partial \phi}  \right) + \frac{1}{\rho^2 \sin^2(\phi )} \frac{\partial^2 u}{\partial { \phi}^2}  
.\] 
\end{thm}

\newpage
\section{Solving the Heat Equation}
~\begin{defn}[the Heat Operator]
\begin{enumerate}[label=\arabic*)]
	\item Define the \allbold{heat operator} as 
		\[
			L(u) = \frac{\partial u}{\partial t} -k \frac{\partial^2 u}{\partial { x}^2} 
		.\] 
		for any $ u(x,t)$ in the appropriate function space (once differentiable in $ t$ and twice differentiable in  $ x$).
		Then $ L(u)$ is a linear operator.
	\item The set of functions that satisfy the boundary conditions $ u(0,t)=0=u(L,t)$ form a vector space. That is, if  $ u_i$ satisfy these boundary condition for $ i=1,2$ and if $ u_3(x,t)=c_1 u_1(x,t)+ c_2 u_2(x,t)$ then $ u_3(0,t)=0=u_3(L,t) $ for any $ c_1,c_2 \in \rr$.
\end{enumerate}
\begin{note}[]
	The set of function that satisfy the initial condition $ u(x,0)=f(x) \neq 0$ does NOT form a vector space.
\end{note}
\end{defn}

\subsection{Separation of Variables}
Consider the following boundary value problem, there are three pieces of the full story:
\begin{equation*}
\begin{cases}
	\text{ PDE:} \qquad  \frac{\partial u}{\partial t} =k\frac{\partial^2 u}{\partial { x}^2} , & 0<x<L,t>0\\
	\text{ BC:} \qquad  u(0,t)=0=u(L,t), & t>0\\
	\text{ IC:} \qquad  u(x,0) = f(x), & 0\leq x \leq L 
\end{cases}
\end{equation*}

\begin{intuition}
	We will take the non-zero part of the boundary conditions into the steady-state ODE, so that the PDE forms a vector space and becomes easier to solve.
\end{intuition}
Typically we would assume $ u(x,t)= \overline{u}(x) + v(x,t)$ but in this case $ \overline{u}(x)=0$, so we apply separation of variables directly to $ u(x,t)$. Assume (separable functions wrt $ t$ and  $ x$): $ u(x,t)=F(x) \cdot G(t) \neq 0$ then
\[
	\frac{\partial u}{\partial t} =k\frac{\partial^2 u}{\partial { x}^2} \implies F(x) \frac{d G}{d t} = k \frac{d^2 F}{d { x }^2} G(t) \implies \frac{1}{k} \cdot \frac{G'(t)}{G(t)} = \frac{F''(x)}{F(x)}
.\] 
For this to satisfy, the ratio must be a constant.
\begin{prf}
We aim to show that the derivative of LHS wrt $ t$ is zero for all $ t$. Note that
\[
	\frac{d }{d t} \left( \frac{1}{k} \cdot  \frac{G'(t)}{G(t)} \right) = \frac{d }{d t} \left( \frac{F''(x)}{F(x)} \right) =0
.\]
and likewise for the RHS
\[
	\frac{d }{d x} \left( \frac{F''(x)}{F(x)} \right) =0 
.\] 
Together 0 derivative everywhere implies a constant function:
\[
	\frac{1}{k} \cdot  \frac{G'(t)}{G(t)}=\frac{F''(x)}{F(x)}=-\lambda
.\] 
where $ \lambda$ is some constant. 
\end{prf}
Now consider the differential equations
\[
	G'(t)=-\lambda k G(t) \text{ and } F''(x) = -\lambda F(x) 
.\] 
\end{document}

\documentclass[class=article,crop=false]{standalone} 
%Fall 2020
% Some basic packages
\usepackage{standalone}[subpreambles=true]
\usepackage[utf8]{inputenc}
\usepackage[T1]{fontenc}
\usepackage{textcomp}
\usepackage[english]{babel}
\usepackage{url}
\usepackage{graphicx}
\usepackage{float}
\usepackage{enumitem}


\pdfminorversion=7

% Don't indent paragraphs, leave some space between them
\usepackage{parskip}

% Hide page number when page is empty
\usepackage{emptypage}
\usepackage{subcaption}
\usepackage{multicol}
\usepackage[dvipsnames]{xcolor}


% Math stuff
\usepackage{amsmath, amsfonts, mathtools, amsthm, amssymb}
% Fancy script capitals
\usepackage{mathrsfs}
\usepackage{cancel}
% Bold math
\usepackage{bm}
% Some shortcuts
\newcommand{\rr}{\ensuremath{\mathbb{R}}}
\newcommand{\zz}{\ensuremath{\mathbb{Z}}}
\newcommand{\qq}{\ensuremath{\mathbb{Q}}}
\newcommand{\nn}{\ensuremath{\mathbb{N}}}
\newcommand{\ff}{\ensuremath{\mathbb{F}}}
\newcommand{\cc}{\ensuremath{\mathbb{C}}}
\renewcommand\O{\ensuremath{\emptyset}}
\newcommand{\norm}[1]{{\left\lVert{#1}\right\rVert}}
\renewcommand{\vec}[1]{{\mathbf{#1}}}
\newcommand\allbold[1]{{\boldmath\textbf{#1}}}

% Put x \to \infty below \lim
\let\svlim\lim\def\lim{\svlim\limits}

%Make implies and impliedby shorter
\let\implies\Rightarrow
\let\impliedby\Leftarrow
\let\iff\Leftrightarrow
\let\epsilon\varepsilon

% Add \contra symbol to denote contradiction
\usepackage{stmaryrd} % for \lightning
\newcommand\contra{\scalebox{1.5}{$\lightning$}}

% \let\phi\varphi

% Command for short corrections
% Usage: 1+1=\correct{3}{2}

\definecolor{correct}{HTML}{009900}
\newcommand\correct[2]{\ensuremath{\:}{\color{red}{#1}}\ensuremath{\to }{\color{correct}{#2}}\ensuremath{\:}}
\newcommand\green[1]{{\color{correct}{#1}}}

% horizontal rule
\newcommand\hr{
    \noindent\rule[0.5ex]{\linewidth}{0.5pt}
}

% hide parts
\newcommand\hide[1]{}

% si unitx
\usepackage{siunitx}
\sisetup{locale = FR}

% Environments
\makeatother
% For box around Definition, Theorem, \ldots
\usepackage[framemethod=TikZ]{mdframed}
\mdfsetup{skipabove=1em,skipbelow=0em}

%definition
\newenvironment{defn}[1][]{%
\ifstrempty{#1}%
{\mdfsetup{%
frametitle={%
\tikz[baseline=(current bounding box.east),outer sep=0pt]
\node[anchor=east,rectangle,fill=Emerald]
{\strut Definition};}}
}%
{\mdfsetup{%
frametitle={%
\tikz[baseline=(current bounding box.east),outer sep=0pt]
\node[anchor=east,rectangle,fill=Emerald]
{\strut Definition:~#1};}}%
}%
\mdfsetup{innertopmargin=10pt,linecolor=Emerald,%
linewidth=2pt,topline=true,%
frametitleaboveskip=\dimexpr-\ht\strutbox\relax
}
\begin{mdframed}[]\relax%
\label{#1}}{\end{mdframed}}


%theorem
%\newcounter{thm}[section]\setcounter{thm}{0}
%\renewcommand{\thethm}{\arabic{section}.\arabic{thm}}
\newenvironment{thm}[1][]{%
%\refstepcounter{thm}%
\ifstrempty{#1}%
{\mdfsetup{%
frametitle={%
\tikz[baseline=(current bounding box.east),outer sep=0pt]
\node[anchor=east,rectangle,fill=blue!20]
%{\strut Theorem~\thethm};}}
{\strut Theorem};}}
}%
{\mdfsetup{%
frametitle={%
\tikz[baseline=(current bounding box.east),outer sep=0pt]
\node[anchor=east,rectangle,fill=blue!20]
%{\strut Theorem~\thethm:~#1};}}%
{\strut Theorem:~#1};}}%
}%
\mdfsetup{innertopmargin=10pt,linecolor=blue!20,%
linewidth=2pt,topline=true,%
frametitleaboveskip=\dimexpr-\ht\strutbox\relax
}
\begin{mdframed}[]\relax%
\label{#1}}{\end{mdframed}}


%lemma
\newenvironment{lem}[1][]{%
\ifstrempty{#1}%
{\mdfsetup{%
frametitle={%
\tikz[baseline=(current bounding box.east),outer sep=0pt]
\node[anchor=east,rectangle,fill=Dandelion]
{\strut Lemma};}}
}%
{\mdfsetup{%
frametitle={%
\tikz[baseline=(current bounding box.east),outer sep=0pt]
\node[anchor=east,rectangle,fill=Dandelion]
{\strut Lemma:~#1};}}%
}%
\mdfsetup{innertopmargin=10pt,linecolor=Dandelion,%
linewidth=2pt,topline=true,%
frametitleaboveskip=\dimexpr-\ht\strutbox\relax
}
\begin{mdframed}[]\relax%
\label{#1}}{\end{mdframed}}

%corollary
\newenvironment{coro}[1][]{%
\ifstrempty{#1}%
{\mdfsetup{%
frametitle={%
\tikz[baseline=(current bounding box.east),outer sep=0pt]
\node[anchor=east,rectangle,fill=CornflowerBlue]
{\strut Corollary};}}
}%
{\mdfsetup{%
frametitle={%
\tikz[baseline=(current bounding box.east),outer sep=0pt]
\node[anchor=east,rectangle,fill=CornflowerBlue]
{\strut Corollary:~#1};}}%
}%
\mdfsetup{innertopmargin=10pt,linecolor=CornflowerBlue,%
linewidth=2pt,topline=true,%
frametitleaboveskip=\dimexpr-\ht\strutbox\relax
}
\begin{mdframed}[]\relax%
\label{#1}}{\end{mdframed}}

%proof
\newenvironment{prf}[1][]{%
\ifstrempty{#1}%
{\mdfsetup{%
frametitle={%
\tikz[baseline=(current bounding box.east),outer sep=0pt]
\node[anchor=east,rectangle,fill=SpringGreen]
{\strut Proof};}}
}%
{\mdfsetup{%
frametitle={%
\tikz[baseline=(current bounding box.east),outer sep=0pt]
\node[anchor=east,rectangle,fill=SpringGreen]
{\strut Proof:~#1};}}%
}%
\mdfsetup{innertopmargin=10pt,linecolor=SpringGreen,%
linewidth=2pt,topline=true,%
frametitleaboveskip=\dimexpr-\ht\strutbox\relax
}
\begin{mdframed}[]\relax%
\label{#1}}{\qed\end{mdframed}}


\theoremstyle{definition}

\newmdtheoremenv[nobreak=true]{definition}{Definition}
\newmdtheoremenv[nobreak=true]{prop}{Proposition}
\newmdtheoremenv[nobreak=true]{theorem}{Theorem}
\newmdtheoremenv[nobreak=true]{corollary}{Corollary}
\newtheorem*{eg}{Example}
\theoremstyle{remark}
\newtheorem*{case}{Case}
\newtheorem*{notation}{Notation}
\newtheorem*{remark}{Remark}
\newtheorem*{note}{Note}
\newtheorem*{problem}{Problem}
\newtheorem*{observe}{Observe}
\newtheorem*{property}{Property}
\newtheorem*{intuition}{Intuition}


% End example and intermezzo environments with a small diamond (just like proof
% environments end with a small square)
\usepackage{etoolbox}
\AtEndEnvironment{vb}{\null\hfill$\diamond$}%
\AtEndEnvironment{intermezzo}{\null\hfill$\diamond$}%
% \AtEndEnvironment{opmerking}{\null\hfill$\diamond$}%

% Fix some spacing
% http://tex.stackexchange.com/questions/22119/how-can-i-change-the-spacing-before-theorems-with-amsthm
\makeatletter
\def\thm@space@setup{%
  \thm@preskip=\parskip \thm@postskip=0pt
}

% Fix some stuff
% %http://tex.stackexchange.com/questions/76273/multiple-pdfs-with-page-group-included-in-a-single-page-warning
\pdfsuppresswarningpagegroup=1


% My name
\author{Jaden Wang}



\begin{document}
\newpage
\section{Motion of a guitar string}

For a plucked guitar string at $ t=0$, the initial position, $ U(x)$, looks like either a sharp isosceles triangle or a hill-like smooth curve.

In either case  $ U(x)$ is continuous and  $ U'(x)$ is piecewise continuous or smooth. Note that if the initial velocity is  $ V(x)=0$ then  $ B_n=0$. If the initial position is given by
\begin{equation*}
	U(x)=
\begin{cases}
	\frac{ax}{h}, &0<x\leq h\\
	\frac{a(L-x)}{L-h }, & h<x<L\\
\end{cases}
\end{equation*}
Then using integration by parts, we get
\[
	A_n = \frac{2a \sin(n\pi h /L)}{(n\pi)^2 \frac{h}{L} \frac{L-h}{L }}
.\] 
We can guess this converges because of $ n^2$ term on the denominator.

So the position of the string is
\[
	u(x,t) = \sum_{ n= 1}^{\infty} \frac{2a}{\pi^2} \frac{L}{h} \frac{L}{L-h} \frac{\sin(n \pi h /L)  )}{n^2 } \cos \left( \frac{n\pi ct}{L } \right) \sin \left( \frac{ n\pi x}{ L} \right) 
.\]

I claim that this converges absolutely:
\[
	0\leq \sum_{ n= 1}^{\infty} | u(x,t)| \leq \sum_{ n= 1}^{\infty} \frac{M}{n^2}
\]
for some constant $ M>0$, so it converges absolutely. Hence it converges uniformly to a continuous function. But note
 \[
	 \frac{\partial^2 u}{\partial { x}^2} = \sum_{ n= 1}^{\infty} \frac{2a}{\pi^2} \frac{L}{h} \frac{L}{L-h} \frac{\sin(n \pi h /L)  )}{n^2 }  \cdot - \left( \frac{n \pi x}{L } \right) ^2 \cdot \cos \left( \frac{ n\pi c t}{ L} \right) \sin \left( \frac{ n\pi x}{ L} \right) 
.\] 
Notice $ n^2$ that helped achieving convergence got canceled! By divergence test (take n to infinity but the term doesn't go to zero), this doesn't converge. This is okay because in practice we always use a finite sum, so divergence is not a big concern.

Thus, for each fixed $ N>0$, the partial sum of the formal solution,  $ S_N(x)$, will satisfy the PDE and BC and will approximate the initial condition as accurately as we wish.

\section{Supplement: The Sound of Music}

\begin{eg}[space and time waves]
	For the general solution, if we fix a $ t$, then we get a space wave, which represents the physical position of the string at a specific time. If we fix a  $ x$, we get a time wave. which represents the evolution of each point of the string through time. 
\end{eg}

\subsection{Standing Waves}

The vertical displacement is a linear combination of the simple product solutions.

Each is called the "normal nodes of vibration" for $ n=1,2,\ldots$. And each mode has amplitude $ \sqrt{A_n^2+ B_n^2} $.

The nth normal node is called the "nth harmonic".

For each fixed $ t$, each node looks like a simple oscillation called a  \emph{standing wave}.

Note that the period of each mode (in time) is $ \tau_n = 2\pi \cdot  \frac{L}{n \pi c} = \frac{2L}{ nc}$.

So one cycle is completed every $ \tau_n$ thus the frequency is $ f= \frac{1}{\tau_n} = \frac{nc}{2L } = \frac{n}{\tau_1}$ for $ n=1,2,\ldots$

The sound produced consists of the superposition of these infinite frequencies.

\subsection{Music Theory}
See slides. This is common sense. $ E_2 $ is 2 octaves above $ E_0$. 

When we pluck guitar, at $ n=1$ we have the  \allbold{first harmonic} or the \emph{fundaemental mode:} $ A_1 \sin \left( \frac{ \pi x}{ L} \right) \cos \left( \frac{ \pi c t}{ L} \right)  $.

Assume $ \omega_1  = 82.41 $Hz. We cannot mute fundamental harmonic because it has no inner stationary points. 

At $ n=2$, the shape of the standing wave is determined by  $ \sin \left( \frac{ 2\pi x}{ L} \right) $, which has 1 inner stationary point (that never moves). Then the frequency is twice the fundamental harmonic. The amplitude is a quarter of the fundamental harmonic.

Similarly, $ n=3$, it has 2 inner stationary points. So we can mute the third harmonic at $ h=\frac{L}{3}$ or $ \frac{2L}{3}$.

\subsection{Fun Facts}
\begin{itemize}
	\item Each octave increase corresponds to a doubling of the frequency.
	\item For every octave the frequency doubles per 12 notes then the frequency per note increases by a factor of $ 1.06$.
\end{itemize}
\end{document}

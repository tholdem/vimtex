\documentclass[class=article,crop=false]{standalone} 
%Fall 2020
% Some basic packages
\usepackage{standalone}[subpreambles=true]
\usepackage[utf8]{inputenc}
\usepackage[T1]{fontenc}
\usepackage{textcomp}
\usepackage[english]{babel}
\usepackage{url}
\usepackage{graphicx}
\usepackage{float}
\usepackage{enumitem}


\pdfminorversion=7

% Don't indent paragraphs, leave some space between them
\usepackage{parskip}

% Hide page number when page is empty
\usepackage{emptypage}
\usepackage{subcaption}
\usepackage{multicol}
\usepackage[dvipsnames]{xcolor}


% Math stuff
\usepackage{amsmath, amsfonts, mathtools, amsthm, amssymb}
% Fancy script capitals
\usepackage{mathrsfs}
\usepackage{cancel}
% Bold math
\usepackage{bm}
% Some shortcuts
\newcommand{\rr}{\ensuremath{\mathbb{R}}}
\newcommand{\zz}{\ensuremath{\mathbb{Z}}}
\newcommand{\qq}{\ensuremath{\mathbb{Q}}}
\newcommand{\nn}{\ensuremath{\mathbb{N}}}
\newcommand{\ff}{\ensuremath{\mathbb{F}}}
\newcommand{\cc}{\ensuremath{\mathbb{C}}}
\renewcommand\O{\ensuremath{\emptyset}}
\newcommand{\norm}[1]{{\left\lVert{#1}\right\rVert}}
\renewcommand{\vec}[1]{{\mathbf{#1}}}
\newcommand\allbold[1]{{\boldmath\textbf{#1}}}

% Put x \to \infty below \lim
\let\svlim\lim\def\lim{\svlim\limits}

%Make implies and impliedby shorter
\let\implies\Rightarrow
\let\impliedby\Leftarrow
\let\iff\Leftrightarrow
\let\epsilon\varepsilon

% Add \contra symbol to denote contradiction
\usepackage{stmaryrd} % for \lightning
\newcommand\contra{\scalebox{1.5}{$\lightning$}}

% \let\phi\varphi

% Command for short corrections
% Usage: 1+1=\correct{3}{2}

\definecolor{correct}{HTML}{009900}
\newcommand\correct[2]{\ensuremath{\:}{\color{red}{#1}}\ensuremath{\to }{\color{correct}{#2}}\ensuremath{\:}}
\newcommand\green[1]{{\color{correct}{#1}}}

% horizontal rule
\newcommand\hr{
    \noindent\rule[0.5ex]{\linewidth}{0.5pt}
}

% hide parts
\newcommand\hide[1]{}

% si unitx
\usepackage{siunitx}
\sisetup{locale = FR}

% Environments
\makeatother
% For box around Definition, Theorem, \ldots
\usepackage[framemethod=TikZ]{mdframed}
\mdfsetup{skipabove=1em,skipbelow=0em}

%definition
\newenvironment{defn}[1][]{%
\ifstrempty{#1}%
{\mdfsetup{%
frametitle={%
\tikz[baseline=(current bounding box.east),outer sep=0pt]
\node[anchor=east,rectangle,fill=Emerald]
{\strut Definition};}}
}%
{\mdfsetup{%
frametitle={%
\tikz[baseline=(current bounding box.east),outer sep=0pt]
\node[anchor=east,rectangle,fill=Emerald]
{\strut Definition:~#1};}}%
}%
\mdfsetup{innertopmargin=10pt,linecolor=Emerald,%
linewidth=2pt,topline=true,%
frametitleaboveskip=\dimexpr-\ht\strutbox\relax
}
\begin{mdframed}[]\relax%
\label{#1}}{\end{mdframed}}


%theorem
%\newcounter{thm}[section]\setcounter{thm}{0}
%\renewcommand{\thethm}{\arabic{section}.\arabic{thm}}
\newenvironment{thm}[1][]{%
%\refstepcounter{thm}%
\ifstrempty{#1}%
{\mdfsetup{%
frametitle={%
\tikz[baseline=(current bounding box.east),outer sep=0pt]
\node[anchor=east,rectangle,fill=blue!20]
%{\strut Theorem~\thethm};}}
{\strut Theorem};}}
}%
{\mdfsetup{%
frametitle={%
\tikz[baseline=(current bounding box.east),outer sep=0pt]
\node[anchor=east,rectangle,fill=blue!20]
%{\strut Theorem~\thethm:~#1};}}%
{\strut Theorem:~#1};}}%
}%
\mdfsetup{innertopmargin=10pt,linecolor=blue!20,%
linewidth=2pt,topline=true,%
frametitleaboveskip=\dimexpr-\ht\strutbox\relax
}
\begin{mdframed}[]\relax%
\label{#1}}{\end{mdframed}}


%lemma
\newenvironment{lem}[1][]{%
\ifstrempty{#1}%
{\mdfsetup{%
frametitle={%
\tikz[baseline=(current bounding box.east),outer sep=0pt]
\node[anchor=east,rectangle,fill=Dandelion]
{\strut Lemma};}}
}%
{\mdfsetup{%
frametitle={%
\tikz[baseline=(current bounding box.east),outer sep=0pt]
\node[anchor=east,rectangle,fill=Dandelion]
{\strut Lemma:~#1};}}%
}%
\mdfsetup{innertopmargin=10pt,linecolor=Dandelion,%
linewidth=2pt,topline=true,%
frametitleaboveskip=\dimexpr-\ht\strutbox\relax
}
\begin{mdframed}[]\relax%
\label{#1}}{\end{mdframed}}

%corollary
\newenvironment{coro}[1][]{%
\ifstrempty{#1}%
{\mdfsetup{%
frametitle={%
\tikz[baseline=(current bounding box.east),outer sep=0pt]
\node[anchor=east,rectangle,fill=CornflowerBlue]
{\strut Corollary};}}
}%
{\mdfsetup{%
frametitle={%
\tikz[baseline=(current bounding box.east),outer sep=0pt]
\node[anchor=east,rectangle,fill=CornflowerBlue]
{\strut Corollary:~#1};}}%
}%
\mdfsetup{innertopmargin=10pt,linecolor=CornflowerBlue,%
linewidth=2pt,topline=true,%
frametitleaboveskip=\dimexpr-\ht\strutbox\relax
}
\begin{mdframed}[]\relax%
\label{#1}}{\end{mdframed}}

%proof
\newenvironment{prf}[1][]{%
\ifstrempty{#1}%
{\mdfsetup{%
frametitle={%
\tikz[baseline=(current bounding box.east),outer sep=0pt]
\node[anchor=east,rectangle,fill=SpringGreen]
{\strut Proof};}}
}%
{\mdfsetup{%
frametitle={%
\tikz[baseline=(current bounding box.east),outer sep=0pt]
\node[anchor=east,rectangle,fill=SpringGreen]
{\strut Proof:~#1};}}%
}%
\mdfsetup{innertopmargin=10pt,linecolor=SpringGreen,%
linewidth=2pt,topline=true,%
frametitleaboveskip=\dimexpr-\ht\strutbox\relax
}
\begin{mdframed}[]\relax%
\label{#1}}{\qed\end{mdframed}}


\theoremstyle{definition}

\newmdtheoremenv[nobreak=true]{definition}{Definition}
\newmdtheoremenv[nobreak=true]{prop}{Proposition}
\newmdtheoremenv[nobreak=true]{theorem}{Theorem}
\newmdtheoremenv[nobreak=true]{corollary}{Corollary}
\newtheorem*{eg}{Example}
\theoremstyle{remark}
\newtheorem*{case}{Case}
\newtheorem*{notation}{Notation}
\newtheorem*{remark}{Remark}
\newtheorem*{note}{Note}
\newtheorem*{problem}{Problem}
\newtheorem*{observe}{Observe}
\newtheorem*{property}{Property}
\newtheorem*{intuition}{Intuition}


% End example and intermezzo environments with a small diamond (just like proof
% environments end with a small square)
\usepackage{etoolbox}
\AtEndEnvironment{vb}{\null\hfill$\diamond$}%
\AtEndEnvironment{intermezzo}{\null\hfill$\diamond$}%
% \AtEndEnvironment{opmerking}{\null\hfill$\diamond$}%

% Fix some spacing
% http://tex.stackexchange.com/questions/22119/how-can-i-change-the-spacing-before-theorems-with-amsthm
\makeatletter
\def\thm@space@setup{%
  \thm@preskip=\parskip \thm@postskip=0pt
}

% Fix some stuff
% %http://tex.stackexchange.com/questions/76273/multiple-pdfs-with-page-group-included-in-a-single-page-warning
\pdfsuppresswarningpagegroup=1


% My name
\author{Jaden Wang}



\begin{document}
For the convergence of the partials, take the time partial of the solution and use the heat equation to get the 2nd space partial:
\begin{align*}
	\frac{\partial u}{\partial t} (x,t) &= \sum_{ n= 1}^{\infty} B_n \cdot -k \left( \frac{n \pi }{L} \right)^2 \sin \left( \frac{ n\pi x}{ L} \right) \cdot  e^{-( \frac{ n\pi}{L} )^2 kt}   \\
	\frac{\partial^2 u}{\partial { x}^2} &= \sum_{ n= 1}^{\infty} B_n \cdot -\left( \frac{n \pi}{L} \right)^2 \sin \left( \frac{ n\pi x}{ L} \right) e^{-( \frac{ n\pi}{L} )^2 kt}   \\
\end{align*}
They have the general form
\[
	\sum_{ n= 1}^{\infty} \tilde{B}_n n^2 \sin \left( \frac{ n\pi x}{ L} \right) e^{-C t( n )^2 } 
.\]
For some constant $ C>0$ and $ \tilde{ B}_n$. Since  $ t>0$
 \[
n\geq 1 \implies n^2 \geq n \implies Ctn^2 \geq Ct n \implies e^{Ctn^2} \geq e^{Ctn} \implies e^{-CTn^2} \leq e^{-Ctn}
.\] 
So by triangle inequality we have
\[
\sum_{ n= 1}^{\infty} \left| \tilde{ B}_n n^2 \sin \left( \frac{ n\pi x}{ L} \right) e^{-Ctn^2}  \right| \leq \sum_{ n= 1}^{\infty} |\tilde{ B}_n | \cdot  n^2 \cdot 1 \cdot  e^{-Ctn}
.\] 
And if $ \tilde{ B}_n $ is bounded by some $ M>0$, then we have
 \[
\sum_{ n= 1}^{\infty} \left| \tilde{ B}_n n^2 \sin \left( \frac{ n\pi x}{ L} \right) e^{-Ctn^2}  \right| \leq \sum_{ n= 1}^{\infty} Mn^2 e^{-Ctn}
.\] 
So it suffices to show that the RHS converges to show the convergence of the partials. This has been done in the homework using the ratio test. Hence the partials converge! 

\subsection{Interpret Solution}

\begin{thm}[convergence of a series solution of the heat equation]
For $ t>0$, if there exists a constant  $ M>0$  such that $ |B_n|\leq M \ \forall \ n$, then
\[
\sum_{ n= 1}^{\infty} B_n \sin \left( \frac{ n\pi x}{ L} \right) e^{-( \frac{ n\pi}{L} )^2 kt} 
\]
converges absolutely for each $ x \in [0,L]$.
\end{thm}
\begin{prf}
Note that given any $ n$,
 \[
\left| B_n \sin \left( \frac{ n\pi x}{ L} \right) e^{-( \frac{ n\pi}{L} )^2 kt}  \right| \leq |B_n| \cdot  1 \cdot e^{-( \frac{ n\pi}{L} )^2 kt} \leq M e^{-( \frac{ n\pi}{L} )^2 kt} 
.\]
so given any $ N>0$, we have
 \[
	 0< \sum_{ n= 1}^{ N} \left| B_n \sin \left( \frac{ n\pi x}{ L} \right) e^{-( \frac{ n\pi}{L} )^2 kt}  \right| \leq \sum_{ n= 1}^{ N} M e^{-( \frac{ n\pi}{L} )^2 kt} 
.\] 
and taking the limit $ N \to \infty$ yields
\[
0 < \sum_{ n= 1}^{\infty} \left| B_n \sin \left( \frac{ n\pi x}{ L} \right) e^{-( \frac{ n\pi}{L} )^2 kt}  \right| \leq \sum_{ n= 1}^{\infty} M e^{-( \frac{ n\pi}{L} )^2 kt} 
.\] 
by Order Limit Theorem.

Now again $ n \geq 1 \implies n^2 \geq n$ and since $ \left( \frac{\pi}{L} \right)^2 kt >0 $,
\[
e^{-( \frac{ n\pi}{L} )^2 kt} \leq e^{-( \frac{ \pi}{L} )^2 ktn} 
.\] 
and since this holds for any $ n$, 
 \[
	 0< \sum_{ n= 1}^{\infty} M e^{-( \frac{ n\pi}{L} )^2 kt} \leq \sum_{ n= 1}^{\infty} M e^{-( \frac{ \pi}{L} ) ktn} =  \sum_{ n= 1}^{\infty} M \left[{e^{-( \frac{ \pi}{L} ) kt}}\right]^{n} < \infty 
.\] 
The last step comes from convergence of Geometric series:
Note $ e^{(\frac{\pi}{L})kt} > e^{0} = 1$, so the inverse is $ <1$.
Then
\begin{align*}
	\sum_{ n= 1}^{\infty} M e^{-( \frac{ \pi}{L} )^2 ktn} &= \sum_{ n= 1}^{\infty} M e^{-( \frac{ \pi}{L} ) kt} \left[e^{-( \frac{ \pi}{L} ) kt}\right]^{n-1}  \\
	&= \sum_{ n= 1}^{\infty} a \cdot  r^{n-1} \\
	&= \frac{a}{1-r} \\
	&= \frac{M e^{-( \frac{ \pi}{L} ) kt} }{1- e^{-( \frac{ \pi}{L} ) kt} } < \infty
\end{align*}

Therefore, by direct comparison test, the Fourier sine series converges absolutely on $ [0,L]$.
\end{prf}

\begin{note}[]
	For the heat equation if we start with reasonable data then the solution is almost guaranteed to converge. The assumption of $ |B_n| < M$ needs to hold. And since $ B_n$ is a definite integral, and its boundedness only depends on $ f(x)$. As long as  $ f(x)$ is "nice",  \emph{i.e.} piecewise continuous with no crazy spikes, then it converges.
\end{note}

\begin{eg}[]
\begin{equation*}
\begin{cases}
	\text{ PDE: } \frac{\partial u}{\partial t} = k\frac{\partial^2 u}{\partial { x}^2} , & 0<x<L,t>0 \\
	\text{ BC: } u(0,t)=0=u(L,t), & t>0\\
	\text{ IC:} u(x,0) = 100, & 0\leq x\leq L  \\
\end{cases}
\end{equation*}

\begin{align*}
	\frac{2}{L} \int_{0}^{L} 100 \sin \left( \frac{ n\pi x}{ L} \right) dx &= \frac{200}{L} \cdot  - \cos \left( \frac{ n\pi x}{ L} \right) \frac{L}{n \pi} \bigg |_{x=0}^{x=L} \\
	&= -\frac{200}{n\pi} \left[ \cos(n\pi  ) -1 \right]  \\
	&= -\frac{200}{n\pi} \left[ (-1)^{n} -1 \right] \\
	&= \frac{400}{n \pi} \text{ if n is odd} 
\end{align*}
So all even terms vanish, then $ B_n$ is a decreasing sequence. So  $ |B_n|\leq M = \frac{400}{n\pi}$ for all $ n\geq 1$.  Thus for  $ t>0$,  $ u(x,t)$ is absolutely convergent for each  $ x$. The series solution has the form:
 \[
	 u(x,t) = \frac{400}{\pi} \sum_{ p= 1}^{\infty} \frac{1}{2p-1} e^{-( \frac{ (2p-1)\pi}{L} )^2 kt} \sin \left( \frac{ (2p-1)\pi x}{ L} \right) 
.\] 
\end{eg}
\end{document}

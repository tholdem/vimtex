\documentclass[class=article,crop=false]{standalone} 
%Fall 2020
% Some basic packages
\usepackage{standalone}[subpreambles=true]
\usepackage[utf8]{inputenc}
\usepackage[T1]{fontenc}
\usepackage{textcomp}
\usepackage[english]{babel}
\usepackage{url}
\usepackage{graphicx}
\usepackage{float}
\usepackage{enumitem}


\pdfminorversion=7

% Don't indent paragraphs, leave some space between them
\usepackage{parskip}

% Hide page number when page is empty
\usepackage{emptypage}
\usepackage{subcaption}
\usepackage{multicol}
\usepackage[dvipsnames]{xcolor}


% Math stuff
\usepackage{amsmath, amsfonts, mathtools, amsthm, amssymb}
% Fancy script capitals
\usepackage{mathrsfs}
\usepackage{cancel}
% Bold math
\usepackage{bm}
% Some shortcuts
\newcommand{\rr}{\ensuremath{\mathbb{R}}}
\newcommand{\zz}{\ensuremath{\mathbb{Z}}}
\newcommand{\qq}{\ensuremath{\mathbb{Q}}}
\newcommand{\nn}{\ensuremath{\mathbb{N}}}
\newcommand{\ff}{\ensuremath{\mathbb{F}}}
\newcommand{\cc}{\ensuremath{\mathbb{C}}}
\renewcommand\O{\ensuremath{\emptyset}}
\newcommand{\norm}[1]{{\left\lVert{#1}\right\rVert}}
\renewcommand{\vec}[1]{{\mathbf{#1}}}
\newcommand\allbold[1]{{\boldmath\textbf{#1}}}

% Put x \to \infty below \lim
\let\svlim\lim\def\lim{\svlim\limits}

%Make implies and impliedby shorter
\let\implies\Rightarrow
\let\impliedby\Leftarrow
\let\iff\Leftrightarrow
\let\epsilon\varepsilon

% Add \contra symbol to denote contradiction
\usepackage{stmaryrd} % for \lightning
\newcommand\contra{\scalebox{1.5}{$\lightning$}}

% \let\phi\varphi

% Command for short corrections
% Usage: 1+1=\correct{3}{2}

\definecolor{correct}{HTML}{009900}
\newcommand\correct[2]{\ensuremath{\:}{\color{red}{#1}}\ensuremath{\to }{\color{correct}{#2}}\ensuremath{\:}}
\newcommand\green[1]{{\color{correct}{#1}}}

% horizontal rule
\newcommand\hr{
    \noindent\rule[0.5ex]{\linewidth}{0.5pt}
}

% hide parts
\newcommand\hide[1]{}

% si unitx
\usepackage{siunitx}
\sisetup{locale = FR}

% Environments
\makeatother
% For box around Definition, Theorem, \ldots
\usepackage[framemethod=TikZ]{mdframed}
\mdfsetup{skipabove=1em,skipbelow=0em}

%definition
\newenvironment{defn}[1][]{%
\ifstrempty{#1}%
{\mdfsetup{%
frametitle={%
\tikz[baseline=(current bounding box.east),outer sep=0pt]
\node[anchor=east,rectangle,fill=Emerald]
{\strut Definition};}}
}%
{\mdfsetup{%
frametitle={%
\tikz[baseline=(current bounding box.east),outer sep=0pt]
\node[anchor=east,rectangle,fill=Emerald]
{\strut Definition:~#1};}}%
}%
\mdfsetup{innertopmargin=10pt,linecolor=Emerald,%
linewidth=2pt,topline=true,%
frametitleaboveskip=\dimexpr-\ht\strutbox\relax
}
\begin{mdframed}[]\relax%
\label{#1}}{\end{mdframed}}


%theorem
%\newcounter{thm}[section]\setcounter{thm}{0}
%\renewcommand{\thethm}{\arabic{section}.\arabic{thm}}
\newenvironment{thm}[1][]{%
%\refstepcounter{thm}%
\ifstrempty{#1}%
{\mdfsetup{%
frametitle={%
\tikz[baseline=(current bounding box.east),outer sep=0pt]
\node[anchor=east,rectangle,fill=blue!20]
%{\strut Theorem~\thethm};}}
{\strut Theorem};}}
}%
{\mdfsetup{%
frametitle={%
\tikz[baseline=(current bounding box.east),outer sep=0pt]
\node[anchor=east,rectangle,fill=blue!20]
%{\strut Theorem~\thethm:~#1};}}%
{\strut Theorem:~#1};}}%
}%
\mdfsetup{innertopmargin=10pt,linecolor=blue!20,%
linewidth=2pt,topline=true,%
frametitleaboveskip=\dimexpr-\ht\strutbox\relax
}
\begin{mdframed}[]\relax%
\label{#1}}{\end{mdframed}}


%lemma
\newenvironment{lem}[1][]{%
\ifstrempty{#1}%
{\mdfsetup{%
frametitle={%
\tikz[baseline=(current bounding box.east),outer sep=0pt]
\node[anchor=east,rectangle,fill=Dandelion]
{\strut Lemma};}}
}%
{\mdfsetup{%
frametitle={%
\tikz[baseline=(current bounding box.east),outer sep=0pt]
\node[anchor=east,rectangle,fill=Dandelion]
{\strut Lemma:~#1};}}%
}%
\mdfsetup{innertopmargin=10pt,linecolor=Dandelion,%
linewidth=2pt,topline=true,%
frametitleaboveskip=\dimexpr-\ht\strutbox\relax
}
\begin{mdframed}[]\relax%
\label{#1}}{\end{mdframed}}

%corollary
\newenvironment{coro}[1][]{%
\ifstrempty{#1}%
{\mdfsetup{%
frametitle={%
\tikz[baseline=(current bounding box.east),outer sep=0pt]
\node[anchor=east,rectangle,fill=CornflowerBlue]
{\strut Corollary};}}
}%
{\mdfsetup{%
frametitle={%
\tikz[baseline=(current bounding box.east),outer sep=0pt]
\node[anchor=east,rectangle,fill=CornflowerBlue]
{\strut Corollary:~#1};}}%
}%
\mdfsetup{innertopmargin=10pt,linecolor=CornflowerBlue,%
linewidth=2pt,topline=true,%
frametitleaboveskip=\dimexpr-\ht\strutbox\relax
}
\begin{mdframed}[]\relax%
\label{#1}}{\end{mdframed}}

%proof
\newenvironment{prf}[1][]{%
\ifstrempty{#1}%
{\mdfsetup{%
frametitle={%
\tikz[baseline=(current bounding box.east),outer sep=0pt]
\node[anchor=east,rectangle,fill=SpringGreen]
{\strut Proof};}}
}%
{\mdfsetup{%
frametitle={%
\tikz[baseline=(current bounding box.east),outer sep=0pt]
\node[anchor=east,rectangle,fill=SpringGreen]
{\strut Proof:~#1};}}%
}%
\mdfsetup{innertopmargin=10pt,linecolor=SpringGreen,%
linewidth=2pt,topline=true,%
frametitleaboveskip=\dimexpr-\ht\strutbox\relax
}
\begin{mdframed}[]\relax%
\label{#1}}{\qed\end{mdframed}}


\theoremstyle{definition}

\newmdtheoremenv[nobreak=true]{definition}{Definition}
\newmdtheoremenv[nobreak=true]{prop}{Proposition}
\newmdtheoremenv[nobreak=true]{theorem}{Theorem}
\newmdtheoremenv[nobreak=true]{corollary}{Corollary}
\newtheorem*{eg}{Example}
\theoremstyle{remark}
\newtheorem*{case}{Case}
\newtheorem*{notation}{Notation}
\newtheorem*{remark}{Remark}
\newtheorem*{note}{Note}
\newtheorem*{problem}{Problem}
\newtheorem*{observe}{Observe}
\newtheorem*{property}{Property}
\newtheorem*{intuition}{Intuition}


% End example and intermezzo environments with a small diamond (just like proof
% environments end with a small square)
\usepackage{etoolbox}
\AtEndEnvironment{vb}{\null\hfill$\diamond$}%
\AtEndEnvironment{intermezzo}{\null\hfill$\diamond$}%
% \AtEndEnvironment{opmerking}{\null\hfill$\diamond$}%

% Fix some spacing
% http://tex.stackexchange.com/questions/22119/how-can-i-change-the-spacing-before-theorems-with-amsthm
\makeatletter
\def\thm@space@setup{%
  \thm@preskip=\parskip \thm@postskip=0pt
}

% Fix some stuff
% %http://tex.stackexchange.com/questions/76273/multiple-pdfs-with-page-group-included-in-a-single-page-warning
\pdfsuppresswarningpagegroup=1


% My name
\author{Jaden Wang}



\begin{document}

\begin{note}[]
Gibbs only occur if FS is truncated. Gibbs has about $ 9\%$ over/undershoot.
\end{note}

\begin{intuition}
	A sequence of continuous function cannot converge uniformly to a discontinuous function.
\end{intuition}

\begin{itemize}
	\item if the adjusted periodic extension $ \tilde{  f}( x) $ is piecewise smooth on every finite interval but has a jump discontinuity then the Fourier Series of $ f(x)$
		 \begin{enumerate}[label=\alph*)]
			\item converges pointwise by Dirichlet's Theorem for pointwise convergence.
			\item will converge at different rates of convergence at each point.
			\item is not uniformly convergent and therefore not absolutely convergent.
			\item exhibits Gibbs Phenomenon in every open interval around a jump discontinuity and does not converge to a continuous function (but does converge).
		\end{enumerate}
	\item A series that converges uniformly will not exhibit Gibbs phenomenon.
	\item if $ \tilde{f}(x)$ is continuous everywhere then we expect absolute convergence.
\end{itemize}

\subsection{Integration and Differentiation of Fourier Series}

~\begin{thm}[term-by-term integration]
	Let $ \sum_{ n= 0}^{\infty} f_n(x)$ be defined on $ [a,b]$. If each  $ f_n(x)$ is continuous on $ [a,b]$ and if the series  $ \sum_{ n= 0}^{\infty} f_n(x)$ converges uniformly to $ f(x)$ on  $ [a,b]$ then
	 \begin{enumerate}[label=(\roman*)]
		 \item $ f(x) = \sum_{ n= 0}^{\infty} f_n(x)$ is continuous on $ [a,b]$ 
		 \item
			 \[
				 \int_{ a}^{ b} \sum_{ n= 0}^{\infty} f_n(x) dx = \sum_{ n= 0}^{\infty} \int_{ a}^{ b} f_n(x) dx  
			 .\] 
	\end{enumerate}
\end{thm}
\begin{thm}[term-by-term differentiation]
	Let $ f_n$ be differentiable functions defined on $ [a,b]$ and suppose the series  $ \sum_{ n= 0}^{\infty} f_n'(x)$ converges uniformly to a limit $ g(x)$ on $ [a,b]$. If there exists a point $ x_0 \in [a,b]$ where $ \sum_{ n= 0}^{\infty} f_n(x_0)$ converges, then $ \sum_{ n= 1}^{\infty} f_n$ converges uniformly to a differentiable function $ f(x)$ satisfying  $ f'(x)=g(x)$ on  $ [a,b]$. That is,
	\[
		f(x) = \sum_{ n= 1}^{\infty} f_n(x)
	.\] 
	and
	 \[
		 f'(x) = \frac{d}{dx} \left[ \sum_{ n= 0}^{\infty} f_n(x) \right] = \sum_{ n= 0}^{\infty} f_n'(x)
	.\] 
\end{thm}

\begin{thm}[univerform convergence theorem]
	Let $ f(x)$ be a continuous piecewise smooth function  $ [-L,L]$ such that  $ f(-L)=f(L)$. Then  $ \text{ F.S.} [ f]( x) $ converges uniformly to $ f(x)$ on  $ [-L,L]$. That is,
	 \[
		 \lim_{ n \to \infty} \max_{-L\leq x \leq L} |S_N(x) -f(x)| =0
	.\] 
\end{thm}

Pinsky book has much more analysis.

\newpage
\section{Derivation of the Heat Equation 1}
\subsection{Insulated Road}
We model the transfer of thermal energy in a one dimensional rod with ends at $ x=0$ and  $ x=L$ and where the lateral surface of the rod is insulated perfectly. 

~\begin{defn}[thermal energy density]
~\begin{enumerate}[label=\arabic*)]
	\item The \allbold{thermal energy density} $ e(x,t)$ is the amount of thermal energy per unit volume.
	\item Consider a thin slice of the rod with cross sectional area  $ A$ between  $ x$ and  $ x+ \Delta x$. The heat energy changes in time due only to heat flowing across the edges ($ x$ and  $ x+\Delta x$). If $ \Delta x$ is small then $ e(x,t)$ may be approximated as constant throughout the slice so:
		 \[
			 \text{ heat energy in slice }[x,x+\Delta x] = e(x,t) \cdot A \cdot \Delta x
		.\] 
		Integrating it yields:
		\[
			\text{ Total heat energy in the rod} = \int_{ 0}^{ L} e(x,t) A dx  
		.\] 
\end{enumerate}

\end{defn}
\begin{defn}[heat flux]
	The \allbold{heat flux}, $ \Phi(x,t)$, is the amount of thermal energy flowing to the right per unit time per unit surface area. If $ \Phi(x,t)<0$ then energy flows to the left. 
\end{defn}

The heat energy flow per unit time across the boundaries of slice $ [x,x+\Delta x]$ with cross sectional surface area $ A$ is:
 \[
	 \Phi(x,t) \cdot A \text{ (heat gain) } + (-\Phi(x+\Delta x,t) \cdot A \text{ (heat loss) }  = -[\Phi(x+\Delta x,t) - \Phi(x,t)] \cdot A
.\] 

\begin{defn}[]
	In the model we allow for \allbold{internal sources of energy}. Let $ Q(x,t)$ be the heat energy generated per unit volume  per unit time within the rod then
	\[
		\text{ heat energy per unit time } = Q(x,t) \cdot  A \cdot \Delta x 
	.\] 
\end{defn}

\begin{thm}[Heat Flow Process]
The fundamental heat flow process in the rod is conceptually described as:
\\

rate of change of heat energy wrt time = heat energy flowing across boundaries per unit time + heat energy generated inside the rod per unit time.
\\

Now consider any finite sement of the rod (from $a$ to $b$), then the conservation of heat energy principle given above implies:
\[
	\frac{d}{dt} \int_{ a}^{ b} e(x,t) A dx = -[\Phi(b,t)-\Phi(a,t)] \cdot A + \int_{ a}^{ b} Q(x,t) A dx  
.\] 
which after canceling $ A>0$ can be rewritten as (by fundamental theorem of calculus):
 \[
	 \int_{ a}^{ b} \frac{\partial}{\partial t} e(x,t) dx = -\int_{ a}^{ b} \frac{\partial }{\partial x} \Phi(x,t) dx + \int_{ a}^{ b} Q(x,t) dx     
.\] 
which yields the "Integral Conservation Law"
\[
	\int_{ a}^{ b} \left[ \frac{\partial }{\partial t} e(x,t) + \frac{\partial }{\partial x} \Phi(x,t) - Q(x,t) \right]  dx = 0
.\] 
which holds for any $ a$ and  $ b$ within the rod, and since the integrand is assumed to be continuous, this implies (proof by contradiction):
 \[
	 \frac{\partial }{\partial t} e(x,t) + \frac{\partial }{\partial x} -Q(x,t) = 0 \implies \frac{\partial }{\partial t} e(x,t) = -\frac{\partial }{\partial x} \Phi(x,t) + Q(x,t)
.\] 
\end{thm}

If $ \partial_x \Phi > 0 $ then $ \Phi$ is an increasing function in $ x$ so the heat flowing to the right at  $ x=b$ is greater than the heat flowing to the right at $ x=a$ thus the heat energy decreases between  $ x=a$ and  $ x=b$ (hence the minus sign).


~\begin{defn}[heat capacity]
\begin{enumerate}[label=\arabic*)]
	\item Let $ u(x,t)$ be the temperature of the rod at point  $ x$ and at time  $ t$. Note that it may take different amounts of thermal energy to raise two different materials from one temperature to another.
	\item Define the heat capacity, $ c(x,u)$, to be the heat energy required to raise its temperature one unit. We will either assume  $ c=c(x)$ or  $ c$ is a constant.
	\item An alternate description of thermal energy is that it is the amount of energy needed to raise the rod's temperature from 0 to the actual temperature $ u(x,t)$. Thus if  $ \rho(x)$ is the mass density of the road then
\[
\text{ heat energy } = c(x) \cdot u(x,t) \cdot  \rho(x) \cdot A \cdot \Delta x 
.\] 
	now equating the expression for heat energy derived earlier with this expression yields
		\[
			e(x,t) \cdot  A \Delta x = c(x) u(x,t)\rho(x) \cdot  A \Delta x\implies e(x,t) = c(x) \rho(x) u(x,t)
		.\] 
\end{enumerate}
\end{defn}
\end{document}

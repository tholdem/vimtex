\documentclass[class=article,crop=false]{standalone} 
%Fall 2020
% Some basic packages
\usepackage{standalone}[subpreambles=true]
\usepackage[utf8]{inputenc}
\usepackage[T1]{fontenc}
\usepackage{textcomp}
\usepackage[english]{babel}
\usepackage{url}
\usepackage{graphicx}
\usepackage{float}
\usepackage{enumitem}


\pdfminorversion=7

% Don't indent paragraphs, leave some space between them
\usepackage{parskip}

% Hide page number when page is empty
\usepackage{emptypage}
\usepackage{subcaption}
\usepackage{multicol}
\usepackage[dvipsnames]{xcolor}


% Math stuff
\usepackage{amsmath, amsfonts, mathtools, amsthm, amssymb}
% Fancy script capitals
\usepackage{mathrsfs}
\usepackage{cancel}
% Bold math
\usepackage{bm}
% Some shortcuts
\newcommand{\rr}{\ensuremath{\mathbb{R}}}
\newcommand{\zz}{\ensuremath{\mathbb{Z}}}
\newcommand{\qq}{\ensuremath{\mathbb{Q}}}
\newcommand{\nn}{\ensuremath{\mathbb{N}}}
\newcommand{\ff}{\ensuremath{\mathbb{F}}}
\newcommand{\cc}{\ensuremath{\mathbb{C}}}
\renewcommand\O{\ensuremath{\emptyset}}
\newcommand{\norm}[1]{{\left\lVert{#1}\right\rVert}}
\renewcommand{\vec}[1]{{\mathbf{#1}}}
\newcommand\allbold[1]{{\boldmath\textbf{#1}}}

% Put x \to \infty below \lim
\let\svlim\lim\def\lim{\svlim\limits}

%Make implies and impliedby shorter
\let\implies\Rightarrow
\let\impliedby\Leftarrow
\let\iff\Leftrightarrow
\let\epsilon\varepsilon

% Add \contra symbol to denote contradiction
\usepackage{stmaryrd} % for \lightning
\newcommand\contra{\scalebox{1.5}{$\lightning$}}

% \let\phi\varphi

% Command for short corrections
% Usage: 1+1=\correct{3}{2}

\definecolor{correct}{HTML}{009900}
\newcommand\correct[2]{\ensuremath{\:}{\color{red}{#1}}\ensuremath{\to }{\color{correct}{#2}}\ensuremath{\:}}
\newcommand\green[1]{{\color{correct}{#1}}}

% horizontal rule
\newcommand\hr{
    \noindent\rule[0.5ex]{\linewidth}{0.5pt}
}

% hide parts
\newcommand\hide[1]{}

% si unitx
\usepackage{siunitx}
\sisetup{locale = FR}

% Environments
\makeatother
% For box around Definition, Theorem, \ldots
\usepackage[framemethod=TikZ]{mdframed}
\mdfsetup{skipabove=1em,skipbelow=0em}

%definition
\newenvironment{defn}[1][]{%
\ifstrempty{#1}%
{\mdfsetup{%
frametitle={%
\tikz[baseline=(current bounding box.east),outer sep=0pt]
\node[anchor=east,rectangle,fill=Emerald]
{\strut Definition};}}
}%
{\mdfsetup{%
frametitle={%
\tikz[baseline=(current bounding box.east),outer sep=0pt]
\node[anchor=east,rectangle,fill=Emerald]
{\strut Definition:~#1};}}%
}%
\mdfsetup{innertopmargin=10pt,linecolor=Emerald,%
linewidth=2pt,topline=true,%
frametitleaboveskip=\dimexpr-\ht\strutbox\relax
}
\begin{mdframed}[]\relax%
\label{#1}}{\end{mdframed}}


%theorem
%\newcounter{thm}[section]\setcounter{thm}{0}
%\renewcommand{\thethm}{\arabic{section}.\arabic{thm}}
\newenvironment{thm}[1][]{%
%\refstepcounter{thm}%
\ifstrempty{#1}%
{\mdfsetup{%
frametitle={%
\tikz[baseline=(current bounding box.east),outer sep=0pt]
\node[anchor=east,rectangle,fill=blue!20]
%{\strut Theorem~\thethm};}}
{\strut Theorem};}}
}%
{\mdfsetup{%
frametitle={%
\tikz[baseline=(current bounding box.east),outer sep=0pt]
\node[anchor=east,rectangle,fill=blue!20]
%{\strut Theorem~\thethm:~#1};}}%
{\strut Theorem:~#1};}}%
}%
\mdfsetup{innertopmargin=10pt,linecolor=blue!20,%
linewidth=2pt,topline=true,%
frametitleaboveskip=\dimexpr-\ht\strutbox\relax
}
\begin{mdframed}[]\relax%
\label{#1}}{\end{mdframed}}


%lemma
\newenvironment{lem}[1][]{%
\ifstrempty{#1}%
{\mdfsetup{%
frametitle={%
\tikz[baseline=(current bounding box.east),outer sep=0pt]
\node[anchor=east,rectangle,fill=Dandelion]
{\strut Lemma};}}
}%
{\mdfsetup{%
frametitle={%
\tikz[baseline=(current bounding box.east),outer sep=0pt]
\node[anchor=east,rectangle,fill=Dandelion]
{\strut Lemma:~#1};}}%
}%
\mdfsetup{innertopmargin=10pt,linecolor=Dandelion,%
linewidth=2pt,topline=true,%
frametitleaboveskip=\dimexpr-\ht\strutbox\relax
}
\begin{mdframed}[]\relax%
\label{#1}}{\end{mdframed}}

%corollary
\newenvironment{coro}[1][]{%
\ifstrempty{#1}%
{\mdfsetup{%
frametitle={%
\tikz[baseline=(current bounding box.east),outer sep=0pt]
\node[anchor=east,rectangle,fill=CornflowerBlue]
{\strut Corollary};}}
}%
{\mdfsetup{%
frametitle={%
\tikz[baseline=(current bounding box.east),outer sep=0pt]
\node[anchor=east,rectangle,fill=CornflowerBlue]
{\strut Corollary:~#1};}}%
}%
\mdfsetup{innertopmargin=10pt,linecolor=CornflowerBlue,%
linewidth=2pt,topline=true,%
frametitleaboveskip=\dimexpr-\ht\strutbox\relax
}
\begin{mdframed}[]\relax%
\label{#1}}{\end{mdframed}}

%proof
\newenvironment{prf}[1][]{%
\ifstrempty{#1}%
{\mdfsetup{%
frametitle={%
\tikz[baseline=(current bounding box.east),outer sep=0pt]
\node[anchor=east,rectangle,fill=SpringGreen]
{\strut Proof};}}
}%
{\mdfsetup{%
frametitle={%
\tikz[baseline=(current bounding box.east),outer sep=0pt]
\node[anchor=east,rectangle,fill=SpringGreen]
{\strut Proof:~#1};}}%
}%
\mdfsetup{innertopmargin=10pt,linecolor=SpringGreen,%
linewidth=2pt,topline=true,%
frametitleaboveskip=\dimexpr-\ht\strutbox\relax
}
\begin{mdframed}[]\relax%
\label{#1}}{\qed\end{mdframed}}


\theoremstyle{definition}

\newmdtheoremenv[nobreak=true]{definition}{Definition}
\newmdtheoremenv[nobreak=true]{prop}{Proposition}
\newmdtheoremenv[nobreak=true]{theorem}{Theorem}
\newmdtheoremenv[nobreak=true]{corollary}{Corollary}
\newtheorem*{eg}{Example}
\theoremstyle{remark}
\newtheorem*{case}{Case}
\newtheorem*{notation}{Notation}
\newtheorem*{remark}{Remark}
\newtheorem*{note}{Note}
\newtheorem*{problem}{Problem}
\newtheorem*{observe}{Observe}
\newtheorem*{property}{Property}
\newtheorem*{intuition}{Intuition}


% End example and intermezzo environments with a small diamond (just like proof
% environments end with a small square)
\usepackage{etoolbox}
\AtEndEnvironment{vb}{\null\hfill$\diamond$}%
\AtEndEnvironment{intermezzo}{\null\hfill$\diamond$}%
% \AtEndEnvironment{opmerking}{\null\hfill$\diamond$}%

% Fix some spacing
% http://tex.stackexchange.com/questions/22119/how-can-i-change-the-spacing-before-theorems-with-amsthm
\makeatletter
\def\thm@space@setup{%
  \thm@preskip=\parskip \thm@postskip=0pt
}

% Fix some stuff
% %http://tex.stackexchange.com/questions/76273/multiple-pdfs-with-page-group-included-in-a-single-page-warning
\pdfsuppresswarningpagegroup=1


% My name
\author{Jaden Wang}



\begin{document}
From previous lecture, we obtain the \allbold{Lorentz series.}

The coefficients are
\begin{align*}
	c_n = \frac{1}{2}(a_n - ib_n) &= \frac{1}{2L} \int_{-L}^{L} f(x) \left[ \cos \left( \frac{ n\pi x}{ L} \right) -i\sin \left( \frac{ n\pi x}{ L} \right)  \right] dx  \\
				      &= \frac{1}{2L} \int_{-L}^{L} f(x) \cdot e^{-i n \pi x / L} dx \\
\end{align*}

Thus the complex form of the Fourier series of $ f(x)$ is
 \[
\text{ F.S.} [ f]( x) = \sum_{ n=-\infty}^{\infty} c_n e^{in\pi x /L}
\]
where
\[
c_n =  \frac{1}{2L} \int_{-L}^{L} f(x) \cdot e^{-i n \pi x / L} dx, n=0, \pm 1, \pm 2, \ldots 
.\]
Notice that the positive exponential term is used in the series and the negative exponential term is used to find the coefficients.

\subsection{Orthogonality}
The inner product of two complex-valued functions $ f(x) $ and  $ g(x)$, piecewise continuous on  $ [-L,L]$ is defined as
 \[
	 \langle f(x),g(x) \rangle = \int_{-L}^{L} f(x) \overline{g(x)} dx = \int_{-L}^{L} [f_1(x)+ if_2(x)]\overline{[g_1(x)+ig_2(x)]}  
.\] 
with norm defined by
 \[
	 \norm{f} = \sqrt{\langle f(x),f(x) \rangle}  = \int_{-L}^{L} |f(x)|^2 dx \in \rr 
.\] 
Note that 
\begin{align*}
	\langle e^{i m \pi x / L} , e^{i n \pi x /L}\rangle &= \int_{-L}^{L} e^{i m \pi x /L} \cdot \overline{e^{i n x /L}}dx  \\
							    &= \int_{-L}^{L} \left[ \cos \left( \frac{ (m-n) \pi x}{ L} \right) + i \sin \left( \frac{ (m-n)\pi x}{ L} \right)  \right]  dx \\
							    &= \begin{cases}
								    0& \text{ if } m\neq n\\
								    2L & \text{ if } m=n \
							    \end{cases} \\
\end{align*}

\begin{eg}[]
	Compute the complex F.S. of $ f(x)=e^{ax}, x \in [-L,L]$, where $ a \in \rr$.

	\begin{align*}
		c_n &= \frac{1}{2L} \int_{-L}^{L} e^{ax} \cdot e^{-i n \pi x /L} dx\\
		    &= \frac{1}{2L[a-(in\pi /L]} e^{[a-(in\pi /L]x} \bigg |_{-L}^L \\
		    &= \frac{1}{2[aL-in\pi]} \left[ e^{aL} e^{-in\pi}-e^{-aL} e^{-in\pi} \right]  \\
		    &= \frac{1}{2[aL-in\pi]} \left[ e^{aL} (\cos(n\pi )-i\sin(n\pi )) -e^{aL}(\cos(n\pi )+i \sin(n\pi )) \right]  \\
		    &= \frac{aL+in\pi}{[(aL)^2+(n\pi)^2] } \cdot (-1)^{n} \cdot \frac{e^{aL}-e^{-aL}}{2 }\\
		    &= \frac{(-1)^{n} (aL+in\pi)}{(aL)^2 + (n\pi)^2 } \sinh(aL)\\
	\end{align*}
	Therefore, the complex F.S. is
	\[
		\sum_{ n= -\infty}^{\infty} c_n e^{in\pi x /L} = \sum_{ n= -\infty}^{\infty}  \frac{(-1)^{n} (aL+in\pi)}{(aL)^2 + (n\pi)^2 } \sinh(aL) e^{in\pi x /L}
	.\] 
	Note that we can use $ c_n= \frac{1}{2} (a_n - ib_n)$ to find the real F.S. coefficients $ a_n$ and $ b_n$ which is much easier than finding them directly!
\end{eg}

Fun Facts:

\begin{itemize}
		\item The coefficients $ c_n$ are usually complex even if $ f(x)$ is real.
		\item If  $ f(x)$ is real then  $ c_{-n} = \overline{c_n}$.
		\item If $ f(x)$ is an even function then  $ c_{-n}=c_n$ and if $ f(x)$ is an odd function then  $ c_{-n}=-c_n$.
		\item Note that
			\[
				c_0 = \frac{1}{2L} \int_{-L}^{L} f(x) dx = \text{ average value of }f(x) \text{ on } [-L,L]
			.\]

		\item If $ f(x)$ is  piecewise smooth then the complex F.S. of $ f(x)$ converges to the periodic extension of the adjusted version of  $ f(x)$.
		\item Parseval's Identity states that
			 \[
				 \frac{1}{2L} \int_{-L}^{L} |f(x)|^2 dx = \sum_{ n= -\infty}^{\infty}  |c_n|^2
			.\] 
\end{itemize}


\subsection{Integral Transform}

The Fourier transform is a \emph{continuous analog} of the F.S. In theory, a Fourier integral would lead to more manageable and understandable solutions in closed form.

\begin{defn}[]
	Given any "reasonable" function $ K(x,z)$, we can define the  \allbold{integral transform}, $ T[f](z)$ of a function  $ f(x)$,  $ a\leq x\leq b$, by
	 \[
		 T[f](z) = \int_{a}^{b} K(x,z) f(x) dx 
	.\] 
	where the function $ f(x)$ is transformed into a new function  $ T[f](z)$. Such transforms are linear. The function  $ K(x,z)$ is known as the  \allbold{kernel} of the transform. 
\end{defn}

\begin{remark}
The Fourier transform is helpful in solving PDEs, primarily because it converts differentiation into algebraic multiplication:
\[
	T[f'](z) = iz T[f](z)
.\] 
\end{remark}

\subsection{Fourier transform}
Given $ f(x), x \in \rr$, we wish to represent $ f(x)$ as Fourier integral:
 \begin{enumerate}[label=\arabic*)]
	 \item Suppose $ \int_{-\infty}^{\infty} |f(x)|dx =M < \infty$ and $ f(x)$ is piecewise smooth on every finite interval.
	 \item Let  $ f(x) = \sum_{ n=-\infty}^{\infty} c_n e^{in\pi x /L} $ and let $m_n =\frac{n\pi}{L }$ then this is a partition of $ (-\infty,\infty)$ for $ n \in \zz$.
	 \item Note that $ \Delta m_n=m_{n+1}-m_n = \frac{(n+1)\pi}{L } - \frac{n\pi}{L } = \frac{\pi}{L}$. Thus $ \frac{L}{\pi} \Delta m_n =1$.
	 \item Using this fact we can write the complex F.S. as a Riemann Sum:
		 \[
			 f(x) = \sum_{ n=-\infty}^{\infty} c_n e^{-in\pi x /L} \cdot \frac{L}{\pi} \Delta m_n = \sum_{ n= -\infty}^{\infty}  \left( \frac{L}{\pi} c_n \right) e^{i n\pi x /L} \Delta m_n = \sum_{ n=-\infty}^{\infty} \hat{ f}(m_n)e^{im_n x} \Delta m_n 
		 .\] 
		 where we let $ \hat{ f}(m_n) = L c_n / \pi$. 
	 \item Taking the limit $ L \to \infty$ on both sides $ \Delta m_n \to 0$ yields:
		 \[
			 f(x) = \lim_{ L \to \infty} \sum_{ n= -\infty}^{\infty} \hat{ f}(m_n) e^{i  m_n x} \Delta m_n = \int_{-\infty}^{\infty} \hat{ f}(m) e^{imx} dm
		 .\] 
		 which is the Fourier integral representation of $ f(x)$.
	 \item Note that $ \hat{ f}(m)$ is the \emph{Fourier transform of $ f(x)$} and $ f(x)$ is the  \emph{inverse Fourier transform of $ \hat{ f}(m)$}.  
\end{enumerate}
\end{document}

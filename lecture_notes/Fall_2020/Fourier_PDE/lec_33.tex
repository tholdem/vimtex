\documentclass[class=article,crop=false]{standalone} 
%Fall 2020
% Some basic packages
\usepackage{standalone}[subpreambles=true]
\usepackage[utf8]{inputenc}
\usepackage[T1]{fontenc}
\usepackage{textcomp}
\usepackage[english]{babel}
\usepackage{url}
\usepackage{graphicx}
\usepackage{float}
\usepackage{enumitem}


\pdfminorversion=7

% Don't indent paragraphs, leave some space between them
\usepackage{parskip}

% Hide page number when page is empty
\usepackage{emptypage}
\usepackage{subcaption}
\usepackage{multicol}
\usepackage[dvipsnames]{xcolor}


% Math stuff
\usepackage{amsmath, amsfonts, mathtools, amsthm, amssymb}
% Fancy script capitals
\usepackage{mathrsfs}
\usepackage{cancel}
% Bold math
\usepackage{bm}
% Some shortcuts
\newcommand{\rr}{\ensuremath{\mathbb{R}}}
\newcommand{\zz}{\ensuremath{\mathbb{Z}}}
\newcommand{\qq}{\ensuremath{\mathbb{Q}}}
\newcommand{\nn}{\ensuremath{\mathbb{N}}}
\newcommand{\ff}{\ensuremath{\mathbb{F}}}
\newcommand{\cc}{\ensuremath{\mathbb{C}}}
\renewcommand\O{\ensuremath{\emptyset}}
\newcommand{\norm}[1]{{\left\lVert{#1}\right\rVert}}
\renewcommand{\vec}[1]{{\mathbf{#1}}}
\newcommand\allbold[1]{{\boldmath\textbf{#1}}}

% Put x \to \infty below \lim
\let\svlim\lim\def\lim{\svlim\limits}

%Make implies and impliedby shorter
\let\implies\Rightarrow
\let\impliedby\Leftarrow
\let\iff\Leftrightarrow
\let\epsilon\varepsilon

% Add \contra symbol to denote contradiction
\usepackage{stmaryrd} % for \lightning
\newcommand\contra{\scalebox{1.5}{$\lightning$}}

% \let\phi\varphi

% Command for short corrections
% Usage: 1+1=\correct{3}{2}

\definecolor{correct}{HTML}{009900}
\newcommand\correct[2]{\ensuremath{\:}{\color{red}{#1}}\ensuremath{\to }{\color{correct}{#2}}\ensuremath{\:}}
\newcommand\green[1]{{\color{correct}{#1}}}

% horizontal rule
\newcommand\hr{
    \noindent\rule[0.5ex]{\linewidth}{0.5pt}
}

% hide parts
\newcommand\hide[1]{}

% si unitx
\usepackage{siunitx}
\sisetup{locale = FR}

% Environments
\makeatother
% For box around Definition, Theorem, \ldots
\usepackage[framemethod=TikZ]{mdframed}
\mdfsetup{skipabove=1em,skipbelow=0em}

%definition
\newenvironment{defn}[1][]{%
\ifstrempty{#1}%
{\mdfsetup{%
frametitle={%
\tikz[baseline=(current bounding box.east),outer sep=0pt]
\node[anchor=east,rectangle,fill=Emerald]
{\strut Definition};}}
}%
{\mdfsetup{%
frametitle={%
\tikz[baseline=(current bounding box.east),outer sep=0pt]
\node[anchor=east,rectangle,fill=Emerald]
{\strut Definition:~#1};}}%
}%
\mdfsetup{innertopmargin=10pt,linecolor=Emerald,%
linewidth=2pt,topline=true,%
frametitleaboveskip=\dimexpr-\ht\strutbox\relax
}
\begin{mdframed}[]\relax%
\label{#1}}{\end{mdframed}}


%theorem
%\newcounter{thm}[section]\setcounter{thm}{0}
%\renewcommand{\thethm}{\arabic{section}.\arabic{thm}}
\newenvironment{thm}[1][]{%
%\refstepcounter{thm}%
\ifstrempty{#1}%
{\mdfsetup{%
frametitle={%
\tikz[baseline=(current bounding box.east),outer sep=0pt]
\node[anchor=east,rectangle,fill=blue!20]
%{\strut Theorem~\thethm};}}
{\strut Theorem};}}
}%
{\mdfsetup{%
frametitle={%
\tikz[baseline=(current bounding box.east),outer sep=0pt]
\node[anchor=east,rectangle,fill=blue!20]
%{\strut Theorem~\thethm:~#1};}}%
{\strut Theorem:~#1};}}%
}%
\mdfsetup{innertopmargin=10pt,linecolor=blue!20,%
linewidth=2pt,topline=true,%
frametitleaboveskip=\dimexpr-\ht\strutbox\relax
}
\begin{mdframed}[]\relax%
\label{#1}}{\end{mdframed}}


%lemma
\newenvironment{lem}[1][]{%
\ifstrempty{#1}%
{\mdfsetup{%
frametitle={%
\tikz[baseline=(current bounding box.east),outer sep=0pt]
\node[anchor=east,rectangle,fill=Dandelion]
{\strut Lemma};}}
}%
{\mdfsetup{%
frametitle={%
\tikz[baseline=(current bounding box.east),outer sep=0pt]
\node[anchor=east,rectangle,fill=Dandelion]
{\strut Lemma:~#1};}}%
}%
\mdfsetup{innertopmargin=10pt,linecolor=Dandelion,%
linewidth=2pt,topline=true,%
frametitleaboveskip=\dimexpr-\ht\strutbox\relax
}
\begin{mdframed}[]\relax%
\label{#1}}{\end{mdframed}}

%corollary
\newenvironment{coro}[1][]{%
\ifstrempty{#1}%
{\mdfsetup{%
frametitle={%
\tikz[baseline=(current bounding box.east),outer sep=0pt]
\node[anchor=east,rectangle,fill=CornflowerBlue]
{\strut Corollary};}}
}%
{\mdfsetup{%
frametitle={%
\tikz[baseline=(current bounding box.east),outer sep=0pt]
\node[anchor=east,rectangle,fill=CornflowerBlue]
{\strut Corollary:~#1};}}%
}%
\mdfsetup{innertopmargin=10pt,linecolor=CornflowerBlue,%
linewidth=2pt,topline=true,%
frametitleaboveskip=\dimexpr-\ht\strutbox\relax
}
\begin{mdframed}[]\relax%
\label{#1}}{\end{mdframed}}

%proof
\newenvironment{prf}[1][]{%
\ifstrempty{#1}%
{\mdfsetup{%
frametitle={%
\tikz[baseline=(current bounding box.east),outer sep=0pt]
\node[anchor=east,rectangle,fill=SpringGreen]
{\strut Proof};}}
}%
{\mdfsetup{%
frametitle={%
\tikz[baseline=(current bounding box.east),outer sep=0pt]
\node[anchor=east,rectangle,fill=SpringGreen]
{\strut Proof:~#1};}}%
}%
\mdfsetup{innertopmargin=10pt,linecolor=SpringGreen,%
linewidth=2pt,topline=true,%
frametitleaboveskip=\dimexpr-\ht\strutbox\relax
}
\begin{mdframed}[]\relax%
\label{#1}}{\qed\end{mdframed}}


\theoremstyle{definition}

\newmdtheoremenv[nobreak=true]{definition}{Definition}
\newmdtheoremenv[nobreak=true]{prop}{Proposition}
\newmdtheoremenv[nobreak=true]{theorem}{Theorem}
\newmdtheoremenv[nobreak=true]{corollary}{Corollary}
\newtheorem*{eg}{Example}
\theoremstyle{remark}
\newtheorem*{case}{Case}
\newtheorem*{notation}{Notation}
\newtheorem*{remark}{Remark}
\newtheorem*{note}{Note}
\newtheorem*{problem}{Problem}
\newtheorem*{observe}{Observe}
\newtheorem*{property}{Property}
\newtheorem*{intuition}{Intuition}


% End example and intermezzo environments with a small diamond (just like proof
% environments end with a small square)
\usepackage{etoolbox}
\AtEndEnvironment{vb}{\null\hfill$\diamond$}%
\AtEndEnvironment{intermezzo}{\null\hfill$\diamond$}%
% \AtEndEnvironment{opmerking}{\null\hfill$\diamond$}%

% Fix some spacing
% http://tex.stackexchange.com/questions/22119/how-can-i-change-the-spacing-before-theorems-with-amsthm
\makeatletter
\def\thm@space@setup{%
  \thm@preskip=\parskip \thm@postskip=0pt
}

% Fix some stuff
% %http://tex.stackexchange.com/questions/76273/multiple-pdfs-with-page-group-included-in-a-single-page-warning
\pdfsuppresswarningpagegroup=1


% My name
\author{Jaden Wang}



\begin{document}
\newpage
\section{Heat Equation on $ \rr$}

\begin{equation*}
\begin{cases}
	\text{PDE: } \frac{\partial u}{\partial t} = k \frac{\partial^2 u}{\partial { x}^2} & x \in \rr, t>0 \\
	\text{"BCs": } u(x,t)< \infty  & \ \forall \ x \in \rr, t>0 \\
	\text{ICs: } u(x,0)= U(x) & \text{ where } \int_{-\infty}^{\infty} |U(x)|dx = M < \infty   \\
\end{cases}
\end{equation*}

Again PDE and BCs form a vector space so we can use separation of variables. Then we get

\begin{equation*}
\begin{cases}
	F''(x)= \lambda F(x)\\
	F(x) < \infty \ \forall \ x \in \rr, t>0\\
\end{cases}
\text{ and } G'(t) = \lambda k G(t)
\end{equation*}
\begin{note}[]
This $ \lambda$ has opposite sign as the one we used before.
\end{note}
If $ \lambda >0$, the solutions are not bounded. If $ \lambda = 0$, we have $ F_0(x) = B$. If  $ \lambda < 0$, we get $ \lambda = -m^2$ for nonzero $ m \in \rr$. And
\[
F_n=A_m e^{imx} + B_m e^{-imx} \text{ for } m \in \rr 
.\] 
For the time domain problem,
\[
	G'(t) = \lambda k G(t) \implies G_m(t) = G(0) e^{-m^2 kt} \text{ for } m \in \rr 
.\] 
Thus, summing over all product solutions with real numbers $ m \in \rr$ yields
\[
	u(x,t) = \int_{-\infty}^{\infty} a_m e^{imx} e^{-m^2kt} d m + \int_{-\infty}^{\infty} b_m e^{-imx} e^{-m^2kt} d m  
.\] 
\begin{note}[]
The complex exponential is oscillating whereas the real exponential is a time-decaying term.
\end{note}
Notice that the second integral is redundant, since we can just use a change of variable $ m=-p$ to get the same form. So
 \[
	 u(x,t) = \int_{-\infty}^{\infty} a_m e^{imx} e^{-m^2kt} d m  
.\]
To find $ a_m$, we use the IC:
 \[
	 U(x) = u(x,0) = \int_{-\infty}^{\infty} a_m e^{imx}d m \text{ and } U(x) = \int_{-\infty}^{\infty} \hat{ U}(m) e^{imx} d m \implies a_m = \hat{ U}(m)   
\] 
by form matching according to our theory. So we can use the Fourier transform as the coefficient. So the solution becomes
\[
	u(x,t) = \int_{-\infty}^{\infty} \hat{ U}(m) e^{imx}e^{-m^2kt} d m
\]
where
\[
	\hat{ U}(m) = \frac{1}{2\pi} \int_{-\infty}^{\infty} U(x) e^{-imx} dx 
.\] 
Note that $ \hat{ U}(0) = \frac{1}{2\pi} \int_{-\infty}^{\infty} U(x) dx$ which is the area under the IC curve divided by $ 2\pi$.  Intuitively this is like area divided by something like a length.

$ \hat{ U}(x)$ is bounded because by boundedness of the integral of $ U(x)$
\[
	|\hat{ U}(m)| \leq \frac{1}{2\pi} \int_{-\infty}^{\infty} |U(x)| \cdot |e^{-imx}| dx = \frac{1}{2\pi} \int_{-\infty}^{\infty} |U(x)|dx \leq \frac{M}{2\pi}  
.\] 

\begin{remark}
This is part of the \emph{Riemann-Lebesgue Lemma} for Fourier transform.  
\end{remark}

\begin{lem}[Riemann-Lebesgue for Fourier transform]
	Suppose $ U(x)$ is defined on  $ -\infty<x<\infty$ and let
	\[
		\hat{ U}(m) = \mathcal{F}[U(x)] = \frac{1}{2\pi} \int_{-\infty}^{\infty} U(x) e^{-imx} dx, \ \forall \ m \in \rr 
	.\] 
	If $ \int_{-\infty}^{\infty} |U(x)| dx = M < \infty$, then
	\begin{enumerate}[label=\arabic*)]
		\item $ \hat{ U}(m)$ is bounded and $ \hat{ U}(m) \leq \frac{M}{2\pi}$.
		\item $ \hat{ U}(m)$ is continuous for all real $ m$.
		\item  $ \hat{ U}(m) \to 0$ as $ m \to \infty$.
	\end{enumerate}
\end{lem}
\begin{note}[]
	This is a special case. Intuitively if the IC is "well-behaved". then the Fourier transform is also well-behaved. Then we can truncate $ \hat{ U}(m)$ without much loss of information.
\end{note}

\begin{prf}
~\begin{enumerate}[label=\arabic*)]
	\item See previous.
	\item Given real number  $ m_n$, define $ U_n(x) = U(x) e^{-im_n x}$ then $ \ \forall \ x \in r$, we have
		\[
			\lim_{ n \to \infty} U_n(x) = \lim_{ n \to \infty} U(x) e^{-im_n x} = U(x) e^{-imx}, |U_n(x)|\leq |U(x)| \ \forall \ n
		.\] 
		Thus by \emph{Lebesgue Dominated Convergence Theorem}, this result yields
		\[
		 \lim_{ n \to \infty} \int_{-\infty}^{\infty} U_n(x) dx = \int_{-\infty}^{\infty}  U(x) e^{-imx} dx \implies \lim_{ m_n \to m} \hat{ U}(m_n) = \hat{ U}(m) 
		.\] 
		Thus $ \hat{ U}(m)$ is continuous.
	\item We will only prove 3 for the simple case that $ U(x)$ is the indicator function over the set  $ [-L,L]$:
		 \begin{equation*}
			 U(x)=
		\begin{cases}
			1, &\text{ if } x \in [-L,L] \\
			0, & \text{ otherwise}\\ 
		\end{cases}
		\end{equation*}

		Then
		\begin{align*}
			\left| \int_{-\infty}^{\infty} U(x) e^{-imx}  \right| &= \left| \int_{-L}^{L} e^{-imx}  \right|  \\
			&= \left| \frac{e^{-imL} - e^{imL}}{ -im} \right|  \\
			&\leq \frac{|i^{e^{-imL}}| + |i e^{imL}|}{|m|}  \\
			&= \frac{2}{|m|} 
		\end{align*}
		By Squeeze Theorem, taking $ m \to \pm \infty$, this bound gives us
		\[
			\lim_{ m \to \pm \infty} \int_{-\infty}^{\infty} U(x) e^{-imx} dx = 0 \implies \hat{ U}(m) \to 0 \text{ as } m  \to \pm \infty  
		.\] 
		Note that this result can be extended to any function $ U(x)$ such that  $ \int_{-\infty}^{\infty} |U(x)|dx < \infty$. Such result is the \emph{Riemann-Lebesgue Theorem}.  
\end{enumerate}
\end{prf}
\end{document}

\documentclass[class=article,crop=false]{standalone} 
%Fall 2020
% Some basic packages
\usepackage{standalone}[subpreambles=true]
\usepackage[utf8]{inputenc}
\usepackage[T1]{fontenc}
\usepackage{textcomp}
\usepackage[english]{babel}
\usepackage{url}
\usepackage{graphicx}
\usepackage{float}
\usepackage{enumitem}


\pdfminorversion=7

% Don't indent paragraphs, leave some space between them
\usepackage{parskip}

% Hide page number when page is empty
\usepackage{emptypage}
\usepackage{subcaption}
\usepackage{multicol}
\usepackage[dvipsnames]{xcolor}


% Math stuff
\usepackage{amsmath, amsfonts, mathtools, amsthm, amssymb}
% Fancy script capitals
\usepackage{mathrsfs}
\usepackage{cancel}
% Bold math
\usepackage{bm}
% Some shortcuts
\newcommand{\rr}{\ensuremath{\mathbb{R}}}
\newcommand{\zz}{\ensuremath{\mathbb{Z}}}
\newcommand{\qq}{\ensuremath{\mathbb{Q}}}
\newcommand{\nn}{\ensuremath{\mathbb{N}}}
\newcommand{\ff}{\ensuremath{\mathbb{F}}}
\newcommand{\cc}{\ensuremath{\mathbb{C}}}
\renewcommand\O{\ensuremath{\emptyset}}
\newcommand{\norm}[1]{{\left\lVert{#1}\right\rVert}}
\renewcommand{\vec}[1]{{\mathbf{#1}}}
\newcommand\allbold[1]{{\boldmath\textbf{#1}}}

% Put x \to \infty below \lim
\let\svlim\lim\def\lim{\svlim\limits}

%Make implies and impliedby shorter
\let\implies\Rightarrow
\let\impliedby\Leftarrow
\let\iff\Leftrightarrow
\let\epsilon\varepsilon

% Add \contra symbol to denote contradiction
\usepackage{stmaryrd} % for \lightning
\newcommand\contra{\scalebox{1.5}{$\lightning$}}

% \let\phi\varphi

% Command for short corrections
% Usage: 1+1=\correct{3}{2}

\definecolor{correct}{HTML}{009900}
\newcommand\correct[2]{\ensuremath{\:}{\color{red}{#1}}\ensuremath{\to }{\color{correct}{#2}}\ensuremath{\:}}
\newcommand\green[1]{{\color{correct}{#1}}}

% horizontal rule
\newcommand\hr{
    \noindent\rule[0.5ex]{\linewidth}{0.5pt}
}

% hide parts
\newcommand\hide[1]{}

% si unitx
\usepackage{siunitx}
\sisetup{locale = FR}

% Environments
\makeatother
% For box around Definition, Theorem, \ldots
\usepackage[framemethod=TikZ]{mdframed}
\mdfsetup{skipabove=1em,skipbelow=0em}

%definition
\newenvironment{defn}[1][]{%
\ifstrempty{#1}%
{\mdfsetup{%
frametitle={%
\tikz[baseline=(current bounding box.east),outer sep=0pt]
\node[anchor=east,rectangle,fill=Emerald]
{\strut Definition};}}
}%
{\mdfsetup{%
frametitle={%
\tikz[baseline=(current bounding box.east),outer sep=0pt]
\node[anchor=east,rectangle,fill=Emerald]
{\strut Definition:~#1};}}%
}%
\mdfsetup{innertopmargin=10pt,linecolor=Emerald,%
linewidth=2pt,topline=true,%
frametitleaboveskip=\dimexpr-\ht\strutbox\relax
}
\begin{mdframed}[]\relax%
\label{#1}}{\end{mdframed}}


%theorem
%\newcounter{thm}[section]\setcounter{thm}{0}
%\renewcommand{\thethm}{\arabic{section}.\arabic{thm}}
\newenvironment{thm}[1][]{%
%\refstepcounter{thm}%
\ifstrempty{#1}%
{\mdfsetup{%
frametitle={%
\tikz[baseline=(current bounding box.east),outer sep=0pt]
\node[anchor=east,rectangle,fill=blue!20]
%{\strut Theorem~\thethm};}}
{\strut Theorem};}}
}%
{\mdfsetup{%
frametitle={%
\tikz[baseline=(current bounding box.east),outer sep=0pt]
\node[anchor=east,rectangle,fill=blue!20]
%{\strut Theorem~\thethm:~#1};}}%
{\strut Theorem:~#1};}}%
}%
\mdfsetup{innertopmargin=10pt,linecolor=blue!20,%
linewidth=2pt,topline=true,%
frametitleaboveskip=\dimexpr-\ht\strutbox\relax
}
\begin{mdframed}[]\relax%
\label{#1}}{\end{mdframed}}


%lemma
\newenvironment{lem}[1][]{%
\ifstrempty{#1}%
{\mdfsetup{%
frametitle={%
\tikz[baseline=(current bounding box.east),outer sep=0pt]
\node[anchor=east,rectangle,fill=Dandelion]
{\strut Lemma};}}
}%
{\mdfsetup{%
frametitle={%
\tikz[baseline=(current bounding box.east),outer sep=0pt]
\node[anchor=east,rectangle,fill=Dandelion]
{\strut Lemma:~#1};}}%
}%
\mdfsetup{innertopmargin=10pt,linecolor=Dandelion,%
linewidth=2pt,topline=true,%
frametitleaboveskip=\dimexpr-\ht\strutbox\relax
}
\begin{mdframed}[]\relax%
\label{#1}}{\end{mdframed}}

%corollary
\newenvironment{coro}[1][]{%
\ifstrempty{#1}%
{\mdfsetup{%
frametitle={%
\tikz[baseline=(current bounding box.east),outer sep=0pt]
\node[anchor=east,rectangle,fill=CornflowerBlue]
{\strut Corollary};}}
}%
{\mdfsetup{%
frametitle={%
\tikz[baseline=(current bounding box.east),outer sep=0pt]
\node[anchor=east,rectangle,fill=CornflowerBlue]
{\strut Corollary:~#1};}}%
}%
\mdfsetup{innertopmargin=10pt,linecolor=CornflowerBlue,%
linewidth=2pt,topline=true,%
frametitleaboveskip=\dimexpr-\ht\strutbox\relax
}
\begin{mdframed}[]\relax%
\label{#1}}{\end{mdframed}}

%proof
\newenvironment{prf}[1][]{%
\ifstrempty{#1}%
{\mdfsetup{%
frametitle={%
\tikz[baseline=(current bounding box.east),outer sep=0pt]
\node[anchor=east,rectangle,fill=SpringGreen]
{\strut Proof};}}
}%
{\mdfsetup{%
frametitle={%
\tikz[baseline=(current bounding box.east),outer sep=0pt]
\node[anchor=east,rectangle,fill=SpringGreen]
{\strut Proof:~#1};}}%
}%
\mdfsetup{innertopmargin=10pt,linecolor=SpringGreen,%
linewidth=2pt,topline=true,%
frametitleaboveskip=\dimexpr-\ht\strutbox\relax
}
\begin{mdframed}[]\relax%
\label{#1}}{\qed\end{mdframed}}


\theoremstyle{definition}

\newmdtheoremenv[nobreak=true]{definition}{Definition}
\newmdtheoremenv[nobreak=true]{prop}{Proposition}
\newmdtheoremenv[nobreak=true]{theorem}{Theorem}
\newmdtheoremenv[nobreak=true]{corollary}{Corollary}
\newtheorem*{eg}{Example}
\theoremstyle{remark}
\newtheorem*{case}{Case}
\newtheorem*{notation}{Notation}
\newtheorem*{remark}{Remark}
\newtheorem*{note}{Note}
\newtheorem*{problem}{Problem}
\newtheorem*{observe}{Observe}
\newtheorem*{property}{Property}
\newtheorem*{intuition}{Intuition}


% End example and intermezzo environments with a small diamond (just like proof
% environments end with a small square)
\usepackage{etoolbox}
\AtEndEnvironment{vb}{\null\hfill$\diamond$}%
\AtEndEnvironment{intermezzo}{\null\hfill$\diamond$}%
% \AtEndEnvironment{opmerking}{\null\hfill$\diamond$}%

% Fix some spacing
% http://tex.stackexchange.com/questions/22119/how-can-i-change-the-spacing-before-theorems-with-amsthm
\makeatletter
\def\thm@space@setup{%
  \thm@preskip=\parskip \thm@postskip=0pt
}

% Fix some stuff
% %http://tex.stackexchange.com/questions/76273/multiple-pdfs-with-page-group-included-in-a-single-page-warning
\pdfsuppresswarningpagegroup=1


% My name
\author{Jaden Wang}



\begin{document}
\begin{remark}
~\\
Wave Equation: we have the same eigenvalue problem as heat equation. The time domain problem has a negative sign which leads to oscillating terms in time.

Laplace Equation: Same eigenvalue problem. The time domain problem has positive sign so we obtain hyperbolic functions in order to easily solve the coefficients.
\end{remark}

\subsection{different PDE domain}
\begin{enumerate}[label=\arabic*)]
	\item Inside the disc of radius $ R$. Then $ 0<r<R, \theta_0<\theta\leq \theta_0 + 2\pi$. Physical boundary is $ r=R$.
	\item Outside the disc of radius  $ R$. Then  $ R<r<\infty, \theta_0< \theta\leq \theta_0 + 2\pi$. Physical boundary: $ r=R$.
	\item Annulus: $ R_i<r<R_o, \theta_0<\theta\leq \theta_0 + 2\pi$. Physical boundaries: $ r=R_i, r=R_o$.
	\item Pie shaped sector: $ 0<r<R, \theta_1 < \theta \leq \theta_2$. Physical boundary: $ r=R, \theta = \theta_1, \theta=\theta_2$.
\end{enumerate}

\begin{eg}[circular disc]
	Suppose $ \Delta u = \frac{1}{r} \frac{\partial }{\partial r} \left( r \frac{\partial v}{\partial r}  \right) + \frac{1}{r^2} \frac{\partial^2 v}{\partial { \theta}^2} $ on the domain $ D=\{(r,\theta)|0\leq r \leq R, -\pi<\theta\leq \pi\} $.

	\begin{note}[]
		$ r=0,r=\infty$ are \allbold{singular points} of the coordinate system for $ \Delta u=0$ but \emph{not} of the physical system.  For physical reasons it is reasonable to assume boundedness at the origin: $ |v(0, \theta)|< \infty$.
	\end{note}
	By periodicity we can assume continuity on the derivatives at the boundaries $ \theta=\pm \pi$.

	\begin{claim}[]
	The set of functions that satisfy the \emph{boundedness condition} or the \emph{periodicity conditions} form a vector space.  
	\end{claim}
\subsection{separation of variables}
Consider
\begin{equation*}
\begin{cases}
	\text{PDE: } \Delta u=0 & 0<r<R, \theta \in (-\pi,\pi) \\
	\text{BCs: }  v(R,\theta) = f(\theta) &  \\
\end{cases}
\end{equation*}
Assume $ v(r,\theta) = F(\theta)G(r) \neq 0$. Then we obtain
\begin{equation*}
\begin{cases}
	F''(\theta)&= -\lambda F(\theta) \\
	F(-\pi)&=F(\pi)\\
	F'(-\pi) &= F'(\pi) \\
\end{cases}
\qquad r^2G''(r)+rG'(r)-\lambda G(r)=0
\end{equation*}
\subsection{F-equation}
The same eigenvalue problem. Only $ \lambda>0$ is nontrivial, so
\[
	F_n(\theta)=A_n \sin(\sqrt{\lambda_n} \theta  )+ B_n\cos(\sqrt{\lambda_n}\theta  ) 
\] 
where $ \lambda_n=n^2,n=1,2,\ldots$, since $ L=\pi$.
\subsection{G-equation}
Plug in $ \lambda=n^2$, we have
\[
	r^2G''(r)+rG'(r)-n^2G(r)=0
.\] 
Let $ G(r)=r^{p}$. Then we get
\[
	r^2 p(p-1)r^{p-2} + r p r^{p-1}-n^2 r^2 = 0
.\] 
After cancellation, we obtain
\[
p=\pm n,n=1,2,\ldots
.\]
If $ n\neq 0$, then  $ r_1=r^{n}, r_2=r^{-n}$. So by superposition, we get 
\[
	G(r)=c_1 r^{n} + c_2 r^{-n}
.\] 
The boundedness condition yields,
\[
	|G(0)|<\infty \implies c_2 =0 \implies G_n(r)=c_1 r^{n}, n=1,2,\ldots
\]
Because $ r=0$ makes the second term undefined.

If $ n=0$, then
 \[
	 r^2G''(r)+rG'(r)=0 \implies rG''(r)+G'(r)=0 \implies \frac{d}{dr} (r G'(r))=0 \implies rG'(r)=C_1
.\] 
Finally note that 
\[
	rG'(r)=C_1 \implies G'(r)=\frac{C_1}{r} \implies G(r)=C_1 \ln(r)+ C_2
.\] 
Boundedness again forces $ C_1 =0 \implies G_0(r)=C_2$.

\begin{note}[]
If the domain is Annulus we would keep both terms. Also if the BCs aren't so nice we might have to use periodicity condition.
\end{note}

Now the general solution is
\begin{align*}
	v(r,\theta) &= a_0 + \sum_{ n= 1}^{\infty} a_n r^{n} \cos(n \theta ) + b_n r^{n} \sin(n \theta )\\
		    &= a_0 + \sum_{ n= 1}^{\infty} A_n \left( \frac{r}{R} \right)^{n} \cos(n\theta  ) + B_n \left( \frac{r}{R}\right) ^{n} \sin(n \theta )   \\
\end{align*}
where $ A_n = a_n R^{n}, B_n = b_n R^{n}$. 
Now using the BCs:
\[
	f(\theta) = v(R,\theta)= a_0 + \sum_{ n= 1}^{\infty} A_n \cdot 1 \cdot \cos(n \theta)+ B_n \cdot 1 \cdot \sin(n \theta ) 
.\] 
which is a F.S.! So,
\[
	a_0 = \frac{1}{2\pi} \int_{-\pi}^{\pi} f(\theta) d\theta  , \quad  a_n =\frac{1}{\pi} \int_{-\pi}^{\pi} f(\theta) \cos(n \theta ) d\theta, \quad  b_n=\frac{1}{\pi} \int_{-\pi}^{\pi} f(\theta) \sin(n\theta ) d\theta   
.\] 
\end{eg}

\end{document}

\documentclass[class=article,crop=false]{standalone} 
%Fall 2020
% Some basic packages
\usepackage{standalone}[subpreambles=true]
\usepackage[utf8]{inputenc}
\usepackage[T1]{fontenc}
\usepackage{textcomp}
\usepackage[english]{babel}
\usepackage{url}
\usepackage{graphicx}
\usepackage{float}
\usepackage{enumitem}


\pdfminorversion=7

% Don't indent paragraphs, leave some space between them
\usepackage{parskip}

% Hide page number when page is empty
\usepackage{emptypage}
\usepackage{subcaption}
\usepackage{multicol}
\usepackage[dvipsnames]{xcolor}


% Math stuff
\usepackage{amsmath, amsfonts, mathtools, amsthm, amssymb}
% Fancy script capitals
\usepackage{mathrsfs}
\usepackage{cancel}
% Bold math
\usepackage{bm}
% Some shortcuts
\newcommand{\rr}{\ensuremath{\mathbb{R}}}
\newcommand{\zz}{\ensuremath{\mathbb{Z}}}
\newcommand{\qq}{\ensuremath{\mathbb{Q}}}
\newcommand{\nn}{\ensuremath{\mathbb{N}}}
\newcommand{\ff}{\ensuremath{\mathbb{F}}}
\newcommand{\cc}{\ensuremath{\mathbb{C}}}
\renewcommand\O{\ensuremath{\emptyset}}
\newcommand{\norm}[1]{{\left\lVert{#1}\right\rVert}}
\renewcommand{\vec}[1]{{\mathbf{#1}}}
\newcommand\allbold[1]{{\boldmath\textbf{#1}}}

% Put x \to \infty below \lim
\let\svlim\lim\def\lim{\svlim\limits}

%Make implies and impliedby shorter
\let\implies\Rightarrow
\let\impliedby\Leftarrow
\let\iff\Leftrightarrow
\let\epsilon\varepsilon

% Add \contra symbol to denote contradiction
\usepackage{stmaryrd} % for \lightning
\newcommand\contra{\scalebox{1.5}{$\lightning$}}

% \let\phi\varphi

% Command for short corrections
% Usage: 1+1=\correct{3}{2}

\definecolor{correct}{HTML}{009900}
\newcommand\correct[2]{\ensuremath{\:}{\color{red}{#1}}\ensuremath{\to }{\color{correct}{#2}}\ensuremath{\:}}
\newcommand\green[1]{{\color{correct}{#1}}}

% horizontal rule
\newcommand\hr{
    \noindent\rule[0.5ex]{\linewidth}{0.5pt}
}

% hide parts
\newcommand\hide[1]{}

% si unitx
\usepackage{siunitx}
\sisetup{locale = FR}

% Environments
\makeatother
% For box around Definition, Theorem, \ldots
\usepackage[framemethod=TikZ]{mdframed}
\mdfsetup{skipabove=1em,skipbelow=0em}

%definition
\newenvironment{defn}[1][]{%
\ifstrempty{#1}%
{\mdfsetup{%
frametitle={%
\tikz[baseline=(current bounding box.east),outer sep=0pt]
\node[anchor=east,rectangle,fill=Emerald]
{\strut Definition};}}
}%
{\mdfsetup{%
frametitle={%
\tikz[baseline=(current bounding box.east),outer sep=0pt]
\node[anchor=east,rectangle,fill=Emerald]
{\strut Definition:~#1};}}%
}%
\mdfsetup{innertopmargin=10pt,linecolor=Emerald,%
linewidth=2pt,topline=true,%
frametitleaboveskip=\dimexpr-\ht\strutbox\relax
}
\begin{mdframed}[]\relax%
\label{#1}}{\end{mdframed}}


%theorem
%\newcounter{thm}[section]\setcounter{thm}{0}
%\renewcommand{\thethm}{\arabic{section}.\arabic{thm}}
\newenvironment{thm}[1][]{%
%\refstepcounter{thm}%
\ifstrempty{#1}%
{\mdfsetup{%
frametitle={%
\tikz[baseline=(current bounding box.east),outer sep=0pt]
\node[anchor=east,rectangle,fill=blue!20]
%{\strut Theorem~\thethm};}}
{\strut Theorem};}}
}%
{\mdfsetup{%
frametitle={%
\tikz[baseline=(current bounding box.east),outer sep=0pt]
\node[anchor=east,rectangle,fill=blue!20]
%{\strut Theorem~\thethm:~#1};}}%
{\strut Theorem:~#1};}}%
}%
\mdfsetup{innertopmargin=10pt,linecolor=blue!20,%
linewidth=2pt,topline=true,%
frametitleaboveskip=\dimexpr-\ht\strutbox\relax
}
\begin{mdframed}[]\relax%
\label{#1}}{\end{mdframed}}


%lemma
\newenvironment{lem}[1][]{%
\ifstrempty{#1}%
{\mdfsetup{%
frametitle={%
\tikz[baseline=(current bounding box.east),outer sep=0pt]
\node[anchor=east,rectangle,fill=Dandelion]
{\strut Lemma};}}
}%
{\mdfsetup{%
frametitle={%
\tikz[baseline=(current bounding box.east),outer sep=0pt]
\node[anchor=east,rectangle,fill=Dandelion]
{\strut Lemma:~#1};}}%
}%
\mdfsetup{innertopmargin=10pt,linecolor=Dandelion,%
linewidth=2pt,topline=true,%
frametitleaboveskip=\dimexpr-\ht\strutbox\relax
}
\begin{mdframed}[]\relax%
\label{#1}}{\end{mdframed}}

%corollary
\newenvironment{coro}[1][]{%
\ifstrempty{#1}%
{\mdfsetup{%
frametitle={%
\tikz[baseline=(current bounding box.east),outer sep=0pt]
\node[anchor=east,rectangle,fill=CornflowerBlue]
{\strut Corollary};}}
}%
{\mdfsetup{%
frametitle={%
\tikz[baseline=(current bounding box.east),outer sep=0pt]
\node[anchor=east,rectangle,fill=CornflowerBlue]
{\strut Corollary:~#1};}}%
}%
\mdfsetup{innertopmargin=10pt,linecolor=CornflowerBlue,%
linewidth=2pt,topline=true,%
frametitleaboveskip=\dimexpr-\ht\strutbox\relax
}
\begin{mdframed}[]\relax%
\label{#1}}{\end{mdframed}}

%proof
\newenvironment{prf}[1][]{%
\ifstrempty{#1}%
{\mdfsetup{%
frametitle={%
\tikz[baseline=(current bounding box.east),outer sep=0pt]
\node[anchor=east,rectangle,fill=SpringGreen]
{\strut Proof};}}
}%
{\mdfsetup{%
frametitle={%
\tikz[baseline=(current bounding box.east),outer sep=0pt]
\node[anchor=east,rectangle,fill=SpringGreen]
{\strut Proof:~#1};}}%
}%
\mdfsetup{innertopmargin=10pt,linecolor=SpringGreen,%
linewidth=2pt,topline=true,%
frametitleaboveskip=\dimexpr-\ht\strutbox\relax
}
\begin{mdframed}[]\relax%
\label{#1}}{\qed\end{mdframed}}


\theoremstyle{definition}

\newmdtheoremenv[nobreak=true]{definition}{Definition}
\newmdtheoremenv[nobreak=true]{prop}{Proposition}
\newmdtheoremenv[nobreak=true]{theorem}{Theorem}
\newmdtheoremenv[nobreak=true]{corollary}{Corollary}
\newtheorem*{eg}{Example}
\theoremstyle{remark}
\newtheorem*{case}{Case}
\newtheorem*{notation}{Notation}
\newtheorem*{remark}{Remark}
\newtheorem*{note}{Note}
\newtheorem*{problem}{Problem}
\newtheorem*{observe}{Observe}
\newtheorem*{property}{Property}
\newtheorem*{intuition}{Intuition}


% End example and intermezzo environments with a small diamond (just like proof
% environments end with a small square)
\usepackage{etoolbox}
\AtEndEnvironment{vb}{\null\hfill$\diamond$}%
\AtEndEnvironment{intermezzo}{\null\hfill$\diamond$}%
% \AtEndEnvironment{opmerking}{\null\hfill$\diamond$}%

% Fix some spacing
% http://tex.stackexchange.com/questions/22119/how-can-i-change-the-spacing-before-theorems-with-amsthm
\makeatletter
\def\thm@space@setup{%
  \thm@preskip=\parskip \thm@postskip=0pt
}

% Fix some stuff
% %http://tex.stackexchange.com/questions/76273/multiple-pdfs-with-page-group-included-in-a-single-page-warning
\pdfsuppresswarningpagegroup=1


% My name
\author{Jaden Wang}



\begin{document}
\newpage
\section{Laplace's Equation and Solution}
We wish to solve the 2D equilibrium heat equation
\[
	\frac{\partial u}{\partial t} = k\left( \frac{\partial^2 u}{\partial { x}^2} + \frac{\partial^2 u}{\partial { y}^2}  \right) = 0
.\] 
Recall the Laplace Operator from prior lecture:
\[
	L(u) = \nabla^2 u = \Delta u  = \frac{\partial^2 u}{\partial { x}^2} + \frac{\partial^2 u}{\partial { y}^2} 
.\] 

\begin{eg}[Thought Experiment in 1D]
~\begin{enumerate}[label=\arabic*)]
	\item Suppose $ f''(x)=0 \implies f(x)=kx+c$, whose rate of change is constant.
	\item for all possible $ k$, the extrema only occur at the boundaries.
	\item For any  $ x_0 \in (a,b)$, and for any $ \epsilon>0$ such that $ [x_0- \epsilon,x_0+ \epsilon] \subseteq (a,b)$, we have
		\[
			f(x_0) = \frac{f(x_0- \epsilon) + f(x_0+ \epsilon)}{ 2} = \frac{k(x_0 - \epsilon)+c-k(x_0 + \epsilon) -c}{2 }= f(x_0)
		.\]
		This is a characterization of equilibrium.
\end{enumerate}
\end{eg}

\subsection{Rectangular Domain}
\begin{equation*}
\begin{cases}
	\text{PDE: }& \Delta u =0, \qquad (x,y) \in (0,L) \times (0,H) \\
	\text{BCs: }& u(x,0) =f_1(x)  \\
		    &u(0,H)=f_2(x)\\
		    &u(0,y)=g_1(y)\\
		    &u(L,y) = g_2(y)\\
\end{cases}
\end{equation*}
Note that we don't have ICs because this is a steady state problem. Since Laplace Operator is linear, we can use the same trick by decomposing it into two problems. Let $ u(x,y)=u_1(x,y)+u_2(x,y)$.
\begin{equation*}
\begin{cases}
	\text{PDE: }& \Delta u_1 =0, (x,y) \in (0,L) \times (0,H) \\
	\text{BCs: }& u_1(x,0) =f_1(x)  \\
		    &u_1(0,H)=f_2(x)\\
		    &u_1(0,y)=0\\
		    &u_1(L,y) = 0\\
\end{cases}
\begin{cases}
	\text{PDE: }& \Delta u =0, (x,y) \in (0,L) \times (0,H) \\
	\text{BCs: }& u_2(x,0) =0  \\
		    &u_2(0,H)=0\\
		    &u_2(0,y)=g_1(y)\\
		    &u_2(L,y) = g_2(y)\\
\end{cases}
\end{equation*}
Note the homogeneous BCs and PDE form a vector space. Then we can use the nonhomogeneous BCs to get the coefficients (like ICs).

\subsubsection{separation of variables}
Let $ u_1(x,y)=F(x)G(y) \neq 0$. Then $ \Delta u_1=0$ implies
\[
	F''(x)G(y)=F(x)G''(y) \implies -\frac{G''(y)}{G(y) }=\frac{F''(x)}{F(x) } = -\lambda
.\] 
\subsubsection{F-equation}
We can solve $ F(x)$ under Dirichlet conditions exactly like before. Note that $ F$ is nontrivial only if  $ \lambda >0$. Then we obtain
\[
	F_n(x) = C_n \sin \left( \frac{ n\pi x}{ L} \right) , n=1,2,\ldots
.\]
\subsubsection{G-equation}
\[
	G''(y)=\lambda_n G(y) \implies r^2e^{ry}=\lambda_n e^{ry} \implies r^2 = \left( \frac{n \pi}{L} \right)^2 \implies r=\pm \frac{n\pi}{L }\implies e^{ry}=e^{\pm n\pi y / L} 
.\]
So $ \{e^{n\pi y/L}, e^{-n \pi y/L}\} $ is a basis solutions for the G-equation. For convenience, we wish to have basis functions that vanish at the endpoints $ y=0$ and  $ y=H$ because it makes fining the coefficients really easy.

Recall that $ \cosh$ is even and $ \sinh $ is odd. Then
\[
	\frac{e^{n\pi y /L}+ e^{-n\pi y /L}}{2 }= \cosh\left( \frac{n \pi y}{L } \right) \text{ and } \frac{e^{n \pi y /L}- e^{-n \pi y /L}}{2 }= \sinh\left( \frac{n\pi y}{L } \right)  
.\]

However, using $ \{\sinh\left( \frac{n\pi y}{L } \right), \cosh\left( \frac{n \pi y}{L } \right)  \} $ is not the basis we seek because $ \cosh$ doesn't go to zero. Let's use the identity:
\[
	\sinh\left( \frac{n \pi [y-H]}{L } \right)= \sinh\left( \frac{n\pi y}{L } - \frac{n\pi H}{L } \right) = \sinh\left( \frac{n\pi y}{L } \right)  \cosh\left( \frac{n\pi H}{L } \right) - \cosh\left( \frac{n\pi y}{L } \right) \sinh\left( \frac{n\pi H}{L } \right)  
.\] 
Therefore, $ \left\{ \sinh\left( \frac{n\pi y}{L } \right) , \sinh\left( \frac{n\pi [y-H]}{L } \right)  \right\}  $ is also a basis of solutions for the G-equation. Note that this is not an orthogonal basis. Thus, the general solution of $ G_n$ is:
\[
	G_n(y) = A_n \sinh\left( \frac{n\pi y}{L } \right) + B_n \sinh\left( \frac{n\pi [y-H]}{L } \right) , n=1,2,\ldots
.\] 

Therefore, the general solution of $ u(x,y)$ is
 \[
	 u_1(x,y) = \sum_{ n= 1}^{\infty} a_n \sinh \left( \frac{ n\pi y}{ L} \right) \sin \left( \frac{ n\pi x}{ L} \right) + b_n \sinh \left( \frac{ n\pi [y-H]}{ L} \right) \sin \left( \frac{ n\pi x}{ L} \right) 
.\] 
When $ y=0$, only the second term remains, so the nonhomogeneous BC yields
 \[
	 f_1(x) = \sum_{ n= 1}^{\infty} b_n \sinh \left( \frac{- n\pi H}{ L} \right) \sin \left( \frac{ n\pi x}{ L} \right) = \sum_{ n= 1}^{\infty} \tilde{ b}_n \sin \left( \frac{ n\pi x}{ L} \right)  
.\]
So by projection formula,
\[
	 b_n = \frac{ \frac{2}{L} \int_{0}^{L} f_1(x) \sin \left( \frac{ n\pi x}{ L} \right)  }{ \sinh \left( \frac{ -n\pi H}{ L} \right) } 
.\] 
And similarly, 
 \[
	 f_2(x) = \sum_{ n= 1}^{\infty} a_n \sinh \left( \frac{ n\pi H}{ L} \right) \sin \left( \frac{ n\pi x}{ L} \right)  = \sum_{ n= 1}^{\infty} \tilde{ a}_n \sin \left( \frac{ n\pi x}{ L} \right)  
.\] 
And
\[
a_n = \frac{\frac{2}{L} \int_{0}^{L} f_2(x) \sin \left( \frac{ n\pi x}{ L} \right)  }{\sinh \left( \frac{ n\pi H}{ L} \right)  } 
.\] 
\end{document}

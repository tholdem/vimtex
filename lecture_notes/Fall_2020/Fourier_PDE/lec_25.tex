\documentclass[class=article,crop=false]{standalone} 
%Fall 2020
% Some basic packages
\usepackage{standalone}[subpreambles=true]
\usepackage[utf8]{inputenc}
\usepackage[T1]{fontenc}
\usepackage{textcomp}
\usepackage[english]{babel}
\usepackage{url}
\usepackage{graphicx}
\usepackage{float}
\usepackage{enumitem}


\pdfminorversion=7

% Don't indent paragraphs, leave some space between them
\usepackage{parskip}

% Hide page number when page is empty
\usepackage{emptypage}
\usepackage{subcaption}
\usepackage{multicol}
\usepackage[dvipsnames]{xcolor}


% Math stuff
\usepackage{amsmath, amsfonts, mathtools, amsthm, amssymb}
% Fancy script capitals
\usepackage{mathrsfs}
\usepackage{cancel}
% Bold math
\usepackage{bm}
% Some shortcuts
\newcommand{\rr}{\ensuremath{\mathbb{R}}}
\newcommand{\zz}{\ensuremath{\mathbb{Z}}}
\newcommand{\qq}{\ensuremath{\mathbb{Q}}}
\newcommand{\nn}{\ensuremath{\mathbb{N}}}
\newcommand{\ff}{\ensuremath{\mathbb{F}}}
\newcommand{\cc}{\ensuremath{\mathbb{C}}}
\renewcommand\O{\ensuremath{\emptyset}}
\newcommand{\norm}[1]{{\left\lVert{#1}\right\rVert}}
\renewcommand{\vec}[1]{{\mathbf{#1}}}
\newcommand\allbold[1]{{\boldmath\textbf{#1}}}

% Put x \to \infty below \lim
\let\svlim\lim\def\lim{\svlim\limits}

%Make implies and impliedby shorter
\let\implies\Rightarrow
\let\impliedby\Leftarrow
\let\iff\Leftrightarrow
\let\epsilon\varepsilon

% Add \contra symbol to denote contradiction
\usepackage{stmaryrd} % for \lightning
\newcommand\contra{\scalebox{1.5}{$\lightning$}}

% \let\phi\varphi

% Command for short corrections
% Usage: 1+1=\correct{3}{2}

\definecolor{correct}{HTML}{009900}
\newcommand\correct[2]{\ensuremath{\:}{\color{red}{#1}}\ensuremath{\to }{\color{correct}{#2}}\ensuremath{\:}}
\newcommand\green[1]{{\color{correct}{#1}}}

% horizontal rule
\newcommand\hr{
    \noindent\rule[0.5ex]{\linewidth}{0.5pt}
}

% hide parts
\newcommand\hide[1]{}

% si unitx
\usepackage{siunitx}
\sisetup{locale = FR}

% Environments
\makeatother
% For box around Definition, Theorem, \ldots
\usepackage[framemethod=TikZ]{mdframed}
\mdfsetup{skipabove=1em,skipbelow=0em}

%definition
\newenvironment{defn}[1][]{%
\ifstrempty{#1}%
{\mdfsetup{%
frametitle={%
\tikz[baseline=(current bounding box.east),outer sep=0pt]
\node[anchor=east,rectangle,fill=Emerald]
{\strut Definition};}}
}%
{\mdfsetup{%
frametitle={%
\tikz[baseline=(current bounding box.east),outer sep=0pt]
\node[anchor=east,rectangle,fill=Emerald]
{\strut Definition:~#1};}}%
}%
\mdfsetup{innertopmargin=10pt,linecolor=Emerald,%
linewidth=2pt,topline=true,%
frametitleaboveskip=\dimexpr-\ht\strutbox\relax
}
\begin{mdframed}[]\relax%
\label{#1}}{\end{mdframed}}


%theorem
%\newcounter{thm}[section]\setcounter{thm}{0}
%\renewcommand{\thethm}{\arabic{section}.\arabic{thm}}
\newenvironment{thm}[1][]{%
%\refstepcounter{thm}%
\ifstrempty{#1}%
{\mdfsetup{%
frametitle={%
\tikz[baseline=(current bounding box.east),outer sep=0pt]
\node[anchor=east,rectangle,fill=blue!20]
%{\strut Theorem~\thethm};}}
{\strut Theorem};}}
}%
{\mdfsetup{%
frametitle={%
\tikz[baseline=(current bounding box.east),outer sep=0pt]
\node[anchor=east,rectangle,fill=blue!20]
%{\strut Theorem~\thethm:~#1};}}%
{\strut Theorem:~#1};}}%
}%
\mdfsetup{innertopmargin=10pt,linecolor=blue!20,%
linewidth=2pt,topline=true,%
frametitleaboveskip=\dimexpr-\ht\strutbox\relax
}
\begin{mdframed}[]\relax%
\label{#1}}{\end{mdframed}}


%lemma
\newenvironment{lem}[1][]{%
\ifstrempty{#1}%
{\mdfsetup{%
frametitle={%
\tikz[baseline=(current bounding box.east),outer sep=0pt]
\node[anchor=east,rectangle,fill=Dandelion]
{\strut Lemma};}}
}%
{\mdfsetup{%
frametitle={%
\tikz[baseline=(current bounding box.east),outer sep=0pt]
\node[anchor=east,rectangle,fill=Dandelion]
{\strut Lemma:~#1};}}%
}%
\mdfsetup{innertopmargin=10pt,linecolor=Dandelion,%
linewidth=2pt,topline=true,%
frametitleaboveskip=\dimexpr-\ht\strutbox\relax
}
\begin{mdframed}[]\relax%
\label{#1}}{\end{mdframed}}

%corollary
\newenvironment{coro}[1][]{%
\ifstrempty{#1}%
{\mdfsetup{%
frametitle={%
\tikz[baseline=(current bounding box.east),outer sep=0pt]
\node[anchor=east,rectangle,fill=CornflowerBlue]
{\strut Corollary};}}
}%
{\mdfsetup{%
frametitle={%
\tikz[baseline=(current bounding box.east),outer sep=0pt]
\node[anchor=east,rectangle,fill=CornflowerBlue]
{\strut Corollary:~#1};}}%
}%
\mdfsetup{innertopmargin=10pt,linecolor=CornflowerBlue,%
linewidth=2pt,topline=true,%
frametitleaboveskip=\dimexpr-\ht\strutbox\relax
}
\begin{mdframed}[]\relax%
\label{#1}}{\end{mdframed}}

%proof
\newenvironment{prf}[1][]{%
\ifstrempty{#1}%
{\mdfsetup{%
frametitle={%
\tikz[baseline=(current bounding box.east),outer sep=0pt]
\node[anchor=east,rectangle,fill=SpringGreen]
{\strut Proof};}}
}%
{\mdfsetup{%
frametitle={%
\tikz[baseline=(current bounding box.east),outer sep=0pt]
\node[anchor=east,rectangle,fill=SpringGreen]
{\strut Proof:~#1};}}%
}%
\mdfsetup{innertopmargin=10pt,linecolor=SpringGreen,%
linewidth=2pt,topline=true,%
frametitleaboveskip=\dimexpr-\ht\strutbox\relax
}
\begin{mdframed}[]\relax%
\label{#1}}{\qed\end{mdframed}}


\theoremstyle{definition}

\newmdtheoremenv[nobreak=true]{definition}{Definition}
\newmdtheoremenv[nobreak=true]{prop}{Proposition}
\newmdtheoremenv[nobreak=true]{theorem}{Theorem}
\newmdtheoremenv[nobreak=true]{corollary}{Corollary}
\newtheorem*{eg}{Example}
\theoremstyle{remark}
\newtheorem*{case}{Case}
\newtheorem*{notation}{Notation}
\newtheorem*{remark}{Remark}
\newtheorem*{note}{Note}
\newtheorem*{problem}{Problem}
\newtheorem*{observe}{Observe}
\newtheorem*{property}{Property}
\newtheorem*{intuition}{Intuition}


% End example and intermezzo environments with a small diamond (just like proof
% environments end with a small square)
\usepackage{etoolbox}
\AtEndEnvironment{vb}{\null\hfill$\diamond$}%
\AtEndEnvironment{intermezzo}{\null\hfill$\diamond$}%
% \AtEndEnvironment{opmerking}{\null\hfill$\diamond$}%

% Fix some spacing
% http://tex.stackexchange.com/questions/22119/how-can-i-change-the-spacing-before-theorems-with-amsthm
\makeatletter
\def\thm@space@setup{%
  \thm@preskip=\parskip \thm@postskip=0pt
}

% Fix some stuff
% %http://tex.stackexchange.com/questions/76273/multiple-pdfs-with-page-group-included-in-a-single-page-warning
\pdfsuppresswarningpagegroup=1


% My name
\author{Jaden Wang}



\begin{document}
\section{D'Alembert's Solution continued}
\begin{equation*}
\begin{cases}
	\text{PDE: } \frac{\partial^2 u}{\partial { t}^2} =c^2 \frac{\partial^2 u}{\partial { x}^2}  & x \in \rr, t>0 \\
	\text{ICs: } u(x,0)=U(x), \frac{\partial u}{\partial t} (x,0)=V(t) & x \in \rr \\
\end{cases}
\end{equation*}
\begin{note}[]
Since $ x \in \rr$, we no longer have boundaries or BCs.
\end{note}
The solution has the form:
\[
	u(x,t)=f(x-ct)+g(x+ct)
.\] 
where $ f,g$ are traveling waves and need to be twice differentiable. We find  $ f(z)$ and  $ g(y)$ using the ICs.

We know that the set of functions that satisfy the wave equation form a vector space. So we can divide and conquer by separating the PDE into two problems:
 \begin{equation*}
\qquad 
\begin{cases}
\partial_t^2 u_1 = c^2 \partial_x^2 u_1 \qquad \\
u_1(x,0)=U(x) \qquad \qquad \qquad \qquad \\
\partial_x u_1 (x,0) = 0 \qquad \qquad  \\
\end{cases}
\begin{cases}
\partial_t^2 u_2 = c^2 \partial_x^2 u_2\\
u_2(x,0)=0\\
\partial_x u_2 (x,0) = V(x)\\
\end{cases}
\end{equation*}
\begin{intuition}
	By having 0 on the RHS, we can express $ f$ in terms of  $ g$, and solve an ODE involving one function instead.
\end{intuition}
If there exists two solutions $ u_1,u_2$, then their sum satisfies the PDE and ICs.

Let's assume $ u_1(x,t)=f_1(x-ct)+g_1(x+ct)$. Using the Chain Rule,
\[
	0=\frac{\partial u_1}{\partial t} (x,0) = f_1'(x) \cdot (-c) + g_1'(x) \cdot (c) \implies f_1'(x)=g_1'(x)
.\] 
Therefore, $ f_1(x)=g_1(x)+k$ for some constant $ k$. Using the other initial condition,
 \[
	 U(x)=u_1(x,0)=f_1(x)+g_1(x)=[g_1(x)+k]+g_1(x) \implies g_1(x)=\frac{U(x)}{2}-\frac{k}{2}
.\] 
Then
\[
	f_1(x) = g_1(x)+k = \frac{U(x)}{2}+ \frac{k}{2}
.\] 
Therefore, the solution is
\[
	u_1(x,t)=f_1(x-ct)+g_1(x+ct) = \frac{U(x-ct)}{ 2} + \frac{U(x+ct)}{2 }
.\]
\begin{intuition}
	If started at rest, the initial position of the string breaks in two, half moving to the left and half moving to the right at equal speeds $ c$, each with half the amplitude of the original. The solution is the simple sum of these traveling waves.
\end{intuition}
Similarly, we assume $ u_2 = f_2(x-ct)+g_2(x+ct)$, and
\[
	0=u_2(x,0)=f_2(x)+g_2(x)\implies f_2(x)=-g_2(x) \implies f_2'(x)=-g_2'(x)
.\]
Using the second initial condition:
\[
	V(x)= \frac{\partial u}{\partial t} (x,0) = f_2'(x) \cdot (-c) + g_2'(x) \cdot c= -g_2'(x) \cdot (-c) + g_2'(x) \cdot c = 2c g_2'(x) \implies g_2'(x)=\frac{V(x)}{2c}
.\] 
That is, $ g_2'(x)$ is an antiderivative of $ \frac{V(x)}{2 }$. Recall by Fundamental Theorem of Calculus,
\[
	\frac{d}{dx} \int_a^{g(x)} f(t) dt = f(g(x)) \cdot g'(x)
.\] 

So by the FTC, we can define
\[
	g_2(x)=\int_a^x \frac{V(t)}{2c} dt \text{ and } f_2(x) = \int_x^a \frac{V(t)}{ 2c} dt 
.\] 
Therefore,
\[
	u_2(x.t) = f_2(x-ct) + g_2(x+ct) = \frac{1}{2c} \int_{x-ct}^{x+ct} \frac{V(t)}{2c}dt
.\]

Hence, the \allbold{general solution} is
\[
	u(x,t) = \frac{1}{2}U(x-ct) + \frac{1}{2}U(x+ct) + \frac{1}{2c} \int_{x-ct}^{x+ct} V(s) ds
.\] 
We can check it satisfies the ICs. When checking the second IC, we would need to apply the FTC using chain rule:
\begin{align*}
	\frac{1}{2c} \frac{d}{dt} \int_{x-ct}^{x+ct} V(s)ds &= \frac{1}{2c} \frac{d}{dt} \left(- \int_{a}^{x-ct} V(s)ds + \int_a^{x+ct} V(s) ds \right) \\
							    &= \frac{1}{2c} (-V(x-ct) \cdot (-c) + V(x+ct) \cdot c) \\
							    &= \frac{1}{2} (V(x-ct)+V(x+ct)) \\
\end{align*}
At $ t=0$, we obtain  $ \frac{\partial u}{\partial t} (x,0) =V(x)$.

\begin{eg}[]
Solve the PDE:
\begin{equation*}
\begin{cases}
	\text{PDE: } \frac{\partial^2 u}{\partial { t}^2} =c^2 \frac{\partial^2 u}{\partial { x}^2}  & x \in \rr, t>0 \\
	\text{ICs: } u(x,0)=0 , \frac{\partial u}{\partial t} (x,0)= \frac{2}{1+x^2} & x \in \rr \\
\end{cases}
\end{equation*}

Apply the d'Alembert's formula we have
\begin{align*}
	u(x,t)&=0+\frac{1}{2c} \int_{x-ct}^{x+ct} \frac{2}{1+x^2} ds\\
	      &= \frac{1}{c} [\arctan(x+ct) - \arctan(x-ct)] 
\end{align*}
Taking the limit as $ t \to \infty$, for each fixed $ x$,
 \[
	 \lim_{ t \to \infty} \frac{1}{c} [\arctan(x+ct) - \arctan(x-ct)] = \frac{1}{c}\left\{ \frac{\pi}{2}-\left(- \frac{\pi}{2} \right)  \right\} = \frac{\pi}{c} 
.\]

See lecture slides for figures. The top of the wave just flattens out.
\end{eg}
\end{document}

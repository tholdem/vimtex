\documentclass[class=article,crop=false]{standalone} 
%Fall 2020
% Some basic packages
\usepackage{standalone}[subpreambles=true]
\usepackage[utf8]{inputenc}
\usepackage[T1]{fontenc}
\usepackage{textcomp}
\usepackage[english]{babel}
\usepackage{url}
\usepackage{graphicx}
\usepackage{float}
\usepackage{enumitem}


\pdfminorversion=7

% Don't indent paragraphs, leave some space between them
\usepackage{parskip}

% Hide page number when page is empty
\usepackage{emptypage}
\usepackage{subcaption}
\usepackage{multicol}
\usepackage[dvipsnames]{xcolor}


% Math stuff
\usepackage{amsmath, amsfonts, mathtools, amsthm, amssymb}
% Fancy script capitals
\usepackage{mathrsfs}
\usepackage{cancel}
% Bold math
\usepackage{bm}
% Some shortcuts
\newcommand{\rr}{\ensuremath{\mathbb{R}}}
\newcommand{\zz}{\ensuremath{\mathbb{Z}}}
\newcommand{\qq}{\ensuremath{\mathbb{Q}}}
\newcommand{\nn}{\ensuremath{\mathbb{N}}}
\newcommand{\ff}{\ensuremath{\mathbb{F}}}
\newcommand{\cc}{\ensuremath{\mathbb{C}}}
\renewcommand\O{\ensuremath{\emptyset}}
\newcommand{\norm}[1]{{\left\lVert{#1}\right\rVert}}
\renewcommand{\vec}[1]{{\mathbf{#1}}}
\newcommand\allbold[1]{{\boldmath\textbf{#1}}}

% Put x \to \infty below \lim
\let\svlim\lim\def\lim{\svlim\limits}

%Make implies and impliedby shorter
\let\implies\Rightarrow
\let\impliedby\Leftarrow
\let\iff\Leftrightarrow
\let\epsilon\varepsilon

% Add \contra symbol to denote contradiction
\usepackage{stmaryrd} % for \lightning
\newcommand\contra{\scalebox{1.5}{$\lightning$}}

% \let\phi\varphi

% Command for short corrections
% Usage: 1+1=\correct{3}{2}

\definecolor{correct}{HTML}{009900}
\newcommand\correct[2]{\ensuremath{\:}{\color{red}{#1}}\ensuremath{\to }{\color{correct}{#2}}\ensuremath{\:}}
\newcommand\green[1]{{\color{correct}{#1}}}

% horizontal rule
\newcommand\hr{
    \noindent\rule[0.5ex]{\linewidth}{0.5pt}
}

% hide parts
\newcommand\hide[1]{}

% si unitx
\usepackage{siunitx}
\sisetup{locale = FR}

% Environments
\makeatother
% For box around Definition, Theorem, \ldots
\usepackage[framemethod=TikZ]{mdframed}
\mdfsetup{skipabove=1em,skipbelow=0em}

%definition
\newenvironment{defn}[1][]{%
\ifstrempty{#1}%
{\mdfsetup{%
frametitle={%
\tikz[baseline=(current bounding box.east),outer sep=0pt]
\node[anchor=east,rectangle,fill=Emerald]
{\strut Definition};}}
}%
{\mdfsetup{%
frametitle={%
\tikz[baseline=(current bounding box.east),outer sep=0pt]
\node[anchor=east,rectangle,fill=Emerald]
{\strut Definition:~#1};}}%
}%
\mdfsetup{innertopmargin=10pt,linecolor=Emerald,%
linewidth=2pt,topline=true,%
frametitleaboveskip=\dimexpr-\ht\strutbox\relax
}
\begin{mdframed}[]\relax%
\label{#1}}{\end{mdframed}}


%theorem
%\newcounter{thm}[section]\setcounter{thm}{0}
%\renewcommand{\thethm}{\arabic{section}.\arabic{thm}}
\newenvironment{thm}[1][]{%
%\refstepcounter{thm}%
\ifstrempty{#1}%
{\mdfsetup{%
frametitle={%
\tikz[baseline=(current bounding box.east),outer sep=0pt]
\node[anchor=east,rectangle,fill=blue!20]
%{\strut Theorem~\thethm};}}
{\strut Theorem};}}
}%
{\mdfsetup{%
frametitle={%
\tikz[baseline=(current bounding box.east),outer sep=0pt]
\node[anchor=east,rectangle,fill=blue!20]
%{\strut Theorem~\thethm:~#1};}}%
{\strut Theorem:~#1};}}%
}%
\mdfsetup{innertopmargin=10pt,linecolor=blue!20,%
linewidth=2pt,topline=true,%
frametitleaboveskip=\dimexpr-\ht\strutbox\relax
}
\begin{mdframed}[]\relax%
\label{#1}}{\end{mdframed}}


%lemma
\newenvironment{lem}[1][]{%
\ifstrempty{#1}%
{\mdfsetup{%
frametitle={%
\tikz[baseline=(current bounding box.east),outer sep=0pt]
\node[anchor=east,rectangle,fill=Dandelion]
{\strut Lemma};}}
}%
{\mdfsetup{%
frametitle={%
\tikz[baseline=(current bounding box.east),outer sep=0pt]
\node[anchor=east,rectangle,fill=Dandelion]
{\strut Lemma:~#1};}}%
}%
\mdfsetup{innertopmargin=10pt,linecolor=Dandelion,%
linewidth=2pt,topline=true,%
frametitleaboveskip=\dimexpr-\ht\strutbox\relax
}
\begin{mdframed}[]\relax%
\label{#1}}{\end{mdframed}}

%corollary
\newenvironment{coro}[1][]{%
\ifstrempty{#1}%
{\mdfsetup{%
frametitle={%
\tikz[baseline=(current bounding box.east),outer sep=0pt]
\node[anchor=east,rectangle,fill=CornflowerBlue]
{\strut Corollary};}}
}%
{\mdfsetup{%
frametitle={%
\tikz[baseline=(current bounding box.east),outer sep=0pt]
\node[anchor=east,rectangle,fill=CornflowerBlue]
{\strut Corollary:~#1};}}%
}%
\mdfsetup{innertopmargin=10pt,linecolor=CornflowerBlue,%
linewidth=2pt,topline=true,%
frametitleaboveskip=\dimexpr-\ht\strutbox\relax
}
\begin{mdframed}[]\relax%
\label{#1}}{\end{mdframed}}

%proof
\newenvironment{prf}[1][]{%
\ifstrempty{#1}%
{\mdfsetup{%
frametitle={%
\tikz[baseline=(current bounding box.east),outer sep=0pt]
\node[anchor=east,rectangle,fill=SpringGreen]
{\strut Proof};}}
}%
{\mdfsetup{%
frametitle={%
\tikz[baseline=(current bounding box.east),outer sep=0pt]
\node[anchor=east,rectangle,fill=SpringGreen]
{\strut Proof:~#1};}}%
}%
\mdfsetup{innertopmargin=10pt,linecolor=SpringGreen,%
linewidth=2pt,topline=true,%
frametitleaboveskip=\dimexpr-\ht\strutbox\relax
}
\begin{mdframed}[]\relax%
\label{#1}}{\qed\end{mdframed}}


\theoremstyle{definition}

\newmdtheoremenv[nobreak=true]{definition}{Definition}
\newmdtheoremenv[nobreak=true]{prop}{Proposition}
\newmdtheoremenv[nobreak=true]{theorem}{Theorem}
\newmdtheoremenv[nobreak=true]{corollary}{Corollary}
\newtheorem*{eg}{Example}
\theoremstyle{remark}
\newtheorem*{case}{Case}
\newtheorem*{notation}{Notation}
\newtheorem*{remark}{Remark}
\newtheorem*{note}{Note}
\newtheorem*{problem}{Problem}
\newtheorem*{observe}{Observe}
\newtheorem*{property}{Property}
\newtheorem*{intuition}{Intuition}


% End example and intermezzo environments with a small diamond (just like proof
% environments end with a small square)
\usepackage{etoolbox}
\AtEndEnvironment{vb}{\null\hfill$\diamond$}%
\AtEndEnvironment{intermezzo}{\null\hfill$\diamond$}%
% \AtEndEnvironment{opmerking}{\null\hfill$\diamond$}%

% Fix some spacing
% http://tex.stackexchange.com/questions/22119/how-can-i-change-the-spacing-before-theorems-with-amsthm
\makeatletter
\def\thm@space@setup{%
  \thm@preskip=\parskip \thm@postskip=0pt
}

% Fix some stuff
% %http://tex.stackexchange.com/questions/76273/multiple-pdfs-with-page-group-included-in-a-single-page-warning
\pdfsuppresswarningpagegroup=1


% My name
\author{Jaden Wang}



\begin{document}
\newpage
\section{Summary of Heat Equations}

BCs whose solutions form a vector space:
~\begin{enumerate}[label=\arabic*)]
	\item Dirichlet BCs: temperature fixed at ends, homogeneous function BCs. Solution is FSS.
	\item Neumann BCs: perfectly insulated, homogeneous derivative BCs. Solution is FCS.
	\item Cauchy BCs: thin circular wire, equal function and derivative BCs. Solution is FS.
	\item Variations: mixture of Dirichlet and Neumann BCs. Depending on the mixture, we get different answers. See homework and practice exam.
\end{enumerate}
For those that don't form a vector space, we move the nonhomogeneous part to the steady state BCs.
\begin{note}[]
The initial condition changes after removing the steady state component.
\end{note}

\newpage
\section{Motion of Stretched String}
\begin{motivation}
	We consider a \emph{horizontally stretched string} with ends that are tied down (something like a guitar). The string moves in time and we wish to track the position of each point on the string during vibration. The motion of a point on the string is NOT entirely vertical, but we are going to assume the motion is entirely vertical. See lecture slides for illustrations.
\end{motivation}

\subsection{Assumptions}
\begin{enumerate}[label=\arabic*)]
	\item With \allbold{no motion}, the string has
		\begin{itemize}
			\item $ \delta(s)$: density
			\item $ A(s)$: cross-sectional area
			\item $ u(s)$: vertical displacement at arc length $ s$.
		\end{itemize}
	\item The \emph{linear mass density} of the string is $ \rho_0 = ( \delta \cdot A)$.
	\item \emph{Boundary Conditions}:  The ends of the string with length $ L$ are fixed:
		$ u(0,t)=u(L,t)=0$.
	\item Possible external forces: gravity, violin bow, guitar pick, etc.
	\item  \allbold{Trivial equilibrium} is if there are no external forces and no motion, then we assume the string lies along a straight line, so $ ds=dx$. Then we assume there is a constant  \allbold{tensile force} or \emph{tension} along the string.
	\item For small vibrations, $ u=u(x,t)$ measures the vertical displacement from the trivial equilibrium at time  $ t$. That is, the shape of the string at time  $ t=t_0$ is given by $ u(x,t_0)$.
\end{enumerate}

\subsection{Additional Assumptions}
~\begin{enumerate}[label=\arabic*)]
	\item We assume mass is constant.
	\item Assume the string is perfectly flexible and has no stiffness.
	\item The forces exerted by the string on the ends act purely in the \emph{tangential direction} and there are no transverse forces and no torque (twisting).
\end{enumerate}

\subsection{Derivation}
Let $ T(x,t) \geq 0$ represent the magnitude of the tangential force due to the tension. Then the horizontal tension balances each other out because there is no horizontal motion. That is,
 \[
	 T(x,t) \cos(x, \theta ) = T(x+ \Delta x, t) \cos(x+ \Delta x, \theta )
.\] 
Therefore, we conclude that $ T$ is constant.  $ T(x,t)\cos(x,t ) = T_0$.

\subsection{Vertical Forces}
According to Newton's second law, $ \vec{ F}= m \vec{ a}  $, the vertical net force equals the tensile forces plus the vertical components of any external forces. Therefore,
\[
	\rho_0 \Delta x \cdot  \frac{\partial^2 u}{\partial { t}^2} = T(x+ \Delta x,t) \sin(\theta (x+ \Delta x, t) ) - T(x,t) \sin(\theta(x,t) ) - \rho_0 \Delta x \cdot  g
.\] 
This is force equals to opposing vertical tensile forces minus the gravity.
Note that $ T(x,t)=\frac{T_0}{\cos(x,t )} \implies T(x,t) \sin(\theta(x,t) ) = T_0 \tan(\theta(x,t)) $, so
\[
	\rho_0 \cdot \frac{\partial^2 u}{\partial { t}^2} = T_0 \cdot  \left[ \frac{\tan(\theta(x+ \Delta x, t))- \tan(\theta(x,t))}{\Delta x } \right] -\rho_0 \cdot  g
.\] 
Taking the limit $ \Delta x \to 0$,
\[
	\rho_0 \cdot  \frac{\partial^2 u}{\partial { t}^2} = T_0 \cdot  \frac{\partial }{\partial x} \tan(\theta(x,t)) - \rho_0 \cdot g
.\] 
\end{document}

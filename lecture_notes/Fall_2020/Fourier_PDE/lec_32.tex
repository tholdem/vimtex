\documentclass[class=article,crop=false]{standalone} 
%Fall 2020
% Some basic packages
\usepackage{standalone}[subpreambles=true]
\usepackage[utf8]{inputenc}
\usepackage[T1]{fontenc}
\usepackage{textcomp}
\usepackage[english]{babel}
\usepackage{url}
\usepackage{graphicx}
\usepackage{float}
\usepackage{enumitem}


\pdfminorversion=7

% Don't indent paragraphs, leave some space between them
\usepackage{parskip}

% Hide page number when page is empty
\usepackage{emptypage}
\usepackage{subcaption}
\usepackage{multicol}
\usepackage[dvipsnames]{xcolor}


% Math stuff
\usepackage{amsmath, amsfonts, mathtools, amsthm, amssymb}
% Fancy script capitals
\usepackage{mathrsfs}
\usepackage{cancel}
% Bold math
\usepackage{bm}
% Some shortcuts
\newcommand{\rr}{\ensuremath{\mathbb{R}}}
\newcommand{\zz}{\ensuremath{\mathbb{Z}}}
\newcommand{\qq}{\ensuremath{\mathbb{Q}}}
\newcommand{\nn}{\ensuremath{\mathbb{N}}}
\newcommand{\ff}{\ensuremath{\mathbb{F}}}
\newcommand{\cc}{\ensuremath{\mathbb{C}}}
\renewcommand\O{\ensuremath{\emptyset}}
\newcommand{\norm}[1]{{\left\lVert{#1}\right\rVert}}
\renewcommand{\vec}[1]{{\mathbf{#1}}}
\newcommand\allbold[1]{{\boldmath\textbf{#1}}}

% Put x \to \infty below \lim
\let\svlim\lim\def\lim{\svlim\limits}

%Make implies and impliedby shorter
\let\implies\Rightarrow
\let\impliedby\Leftarrow
\let\iff\Leftrightarrow
\let\epsilon\varepsilon

% Add \contra symbol to denote contradiction
\usepackage{stmaryrd} % for \lightning
\newcommand\contra{\scalebox{1.5}{$\lightning$}}

% \let\phi\varphi

% Command for short corrections
% Usage: 1+1=\correct{3}{2}

\definecolor{correct}{HTML}{009900}
\newcommand\correct[2]{\ensuremath{\:}{\color{red}{#1}}\ensuremath{\to }{\color{correct}{#2}}\ensuremath{\:}}
\newcommand\green[1]{{\color{correct}{#1}}}

% horizontal rule
\newcommand\hr{
    \noindent\rule[0.5ex]{\linewidth}{0.5pt}
}

% hide parts
\newcommand\hide[1]{}

% si unitx
\usepackage{siunitx}
\sisetup{locale = FR}

% Environments
\makeatother
% For box around Definition, Theorem, \ldots
\usepackage[framemethod=TikZ]{mdframed}
\mdfsetup{skipabove=1em,skipbelow=0em}

%definition
\newenvironment{defn}[1][]{%
\ifstrempty{#1}%
{\mdfsetup{%
frametitle={%
\tikz[baseline=(current bounding box.east),outer sep=0pt]
\node[anchor=east,rectangle,fill=Emerald]
{\strut Definition};}}
}%
{\mdfsetup{%
frametitle={%
\tikz[baseline=(current bounding box.east),outer sep=0pt]
\node[anchor=east,rectangle,fill=Emerald]
{\strut Definition:~#1};}}%
}%
\mdfsetup{innertopmargin=10pt,linecolor=Emerald,%
linewidth=2pt,topline=true,%
frametitleaboveskip=\dimexpr-\ht\strutbox\relax
}
\begin{mdframed}[]\relax%
\label{#1}}{\end{mdframed}}


%theorem
%\newcounter{thm}[section]\setcounter{thm}{0}
%\renewcommand{\thethm}{\arabic{section}.\arabic{thm}}
\newenvironment{thm}[1][]{%
%\refstepcounter{thm}%
\ifstrempty{#1}%
{\mdfsetup{%
frametitle={%
\tikz[baseline=(current bounding box.east),outer sep=0pt]
\node[anchor=east,rectangle,fill=blue!20]
%{\strut Theorem~\thethm};}}
{\strut Theorem};}}
}%
{\mdfsetup{%
frametitle={%
\tikz[baseline=(current bounding box.east),outer sep=0pt]
\node[anchor=east,rectangle,fill=blue!20]
%{\strut Theorem~\thethm:~#1};}}%
{\strut Theorem:~#1};}}%
}%
\mdfsetup{innertopmargin=10pt,linecolor=blue!20,%
linewidth=2pt,topline=true,%
frametitleaboveskip=\dimexpr-\ht\strutbox\relax
}
\begin{mdframed}[]\relax%
\label{#1}}{\end{mdframed}}


%lemma
\newenvironment{lem}[1][]{%
\ifstrempty{#1}%
{\mdfsetup{%
frametitle={%
\tikz[baseline=(current bounding box.east),outer sep=0pt]
\node[anchor=east,rectangle,fill=Dandelion]
{\strut Lemma};}}
}%
{\mdfsetup{%
frametitle={%
\tikz[baseline=(current bounding box.east),outer sep=0pt]
\node[anchor=east,rectangle,fill=Dandelion]
{\strut Lemma:~#1};}}%
}%
\mdfsetup{innertopmargin=10pt,linecolor=Dandelion,%
linewidth=2pt,topline=true,%
frametitleaboveskip=\dimexpr-\ht\strutbox\relax
}
\begin{mdframed}[]\relax%
\label{#1}}{\end{mdframed}}

%corollary
\newenvironment{coro}[1][]{%
\ifstrempty{#1}%
{\mdfsetup{%
frametitle={%
\tikz[baseline=(current bounding box.east),outer sep=0pt]
\node[anchor=east,rectangle,fill=CornflowerBlue]
{\strut Corollary};}}
}%
{\mdfsetup{%
frametitle={%
\tikz[baseline=(current bounding box.east),outer sep=0pt]
\node[anchor=east,rectangle,fill=CornflowerBlue]
{\strut Corollary:~#1};}}%
}%
\mdfsetup{innertopmargin=10pt,linecolor=CornflowerBlue,%
linewidth=2pt,topline=true,%
frametitleaboveskip=\dimexpr-\ht\strutbox\relax
}
\begin{mdframed}[]\relax%
\label{#1}}{\end{mdframed}}

%proof
\newenvironment{prf}[1][]{%
\ifstrempty{#1}%
{\mdfsetup{%
frametitle={%
\tikz[baseline=(current bounding box.east),outer sep=0pt]
\node[anchor=east,rectangle,fill=SpringGreen]
{\strut Proof};}}
}%
{\mdfsetup{%
frametitle={%
\tikz[baseline=(current bounding box.east),outer sep=0pt]
\node[anchor=east,rectangle,fill=SpringGreen]
{\strut Proof:~#1};}}%
}%
\mdfsetup{innertopmargin=10pt,linecolor=SpringGreen,%
linewidth=2pt,topline=true,%
frametitleaboveskip=\dimexpr-\ht\strutbox\relax
}
\begin{mdframed}[]\relax%
\label{#1}}{\qed\end{mdframed}}


\theoremstyle{definition}

\newmdtheoremenv[nobreak=true]{definition}{Definition}
\newmdtheoremenv[nobreak=true]{prop}{Proposition}
\newmdtheoremenv[nobreak=true]{theorem}{Theorem}
\newmdtheoremenv[nobreak=true]{corollary}{Corollary}
\newtheorem*{eg}{Example}
\theoremstyle{remark}
\newtheorem*{case}{Case}
\newtheorem*{notation}{Notation}
\newtheorem*{remark}{Remark}
\newtheorem*{note}{Note}
\newtheorem*{problem}{Problem}
\newtheorem*{observe}{Observe}
\newtheorem*{property}{Property}
\newtheorem*{intuition}{Intuition}


% End example and intermezzo environments with a small diamond (just like proof
% environments end with a small square)
\usepackage{etoolbox}
\AtEndEnvironment{vb}{\null\hfill$\diamond$}%
\AtEndEnvironment{intermezzo}{\null\hfill$\diamond$}%
% \AtEndEnvironment{opmerking}{\null\hfill$\diamond$}%

% Fix some spacing
% http://tex.stackexchange.com/questions/22119/how-can-i-change-the-spacing-before-theorems-with-amsthm
\makeatletter
\def\thm@space@setup{%
  \thm@preskip=\parskip \thm@postskip=0pt
}

% Fix some stuff
% %http://tex.stackexchange.com/questions/76273/multiple-pdfs-with-page-group-included-in-a-single-page-warning
\pdfsuppresswarningpagegroup=1


% My name
\author{Jaden Wang}



\begin{document}
\begin{intuition}
	The Fourier transform is basically the "Fourier coefficient" of the basis function for the integral.
\end{intuition}

\subsection{Notation}
~\begin{enumerate}[label=\arabic*)]
	\item 
		\begin{align*}
			\hat{ f}(m) = \lim_{ L \to \infty} \hat{ f}(m_n) = \lim_{ L \to \infty} \frac{L}{\pi} c_n &= \lim_{ L \to \infty} \frac{L}{\pi} \left( \frac{1}{2L} \int_{-L}^{L} f(x) e^{ - i n\pi x / L}   \right) \ dx \\
														  &= \frac{1}{2\pi} \int_{-\infty}^{\infty} f(x) e^{   i m x }  \ dx  \\
		\end{align*}
		\begin{intuition}
			This is sort of the projection formula, since $ 2\pi$ is the circumference of the unit circle.
		\end{intuition}
	\item 
		~\begin{defn} 
			Define the \allbold{Fourier transform} of $ f(x)$ to be  $ \hat{ f}(m) = \mathcal{F}[f](m)$ where
		\[
			\hat{ f}(m) = \mathcal{F}[f](m) = \frac{1}{2\pi } \int_{-\infty}^{\infty} f(x) e^{ -  i m x } \ \forall \ m \in \rr  
		.\] 
	where the kernel is $ K(x,m)=e^{-imx}$. 
\end{defn}
	\item 
		~\begin{defn}
		Define the \allbold{inverse Fourier transform} of $ \hat{ f}(m)$: $ f(x) = \mathcal{F}^{-1} [\hat{ f}](x)$,
		\[
			f(x) = \mathcal{F}^{-1} [\hat{ f}](x) = \int_{-\infty}^{\infty} \hat{ f}(m) e^{imx} dm \ \forall \ x \in \rr  
		.\] 
		where the kernel is $ \hat{ K}(m,x) = e^{imx}$.
	        \end{defn}
	\item Given $ \hat{ f}(m)$ then $ f(x) =\mathcal{F} ^{-1}[\hat{ f}]$ and given $ f(x)$ then  $ \hat{ f}(m) = \mathcal{F}[f]$.
		\begin{intuition}
			The Fourier transform represents a function $ f(x)$ in a new "coordinate system" using different eigenfunction basis. 
		\end{intuition}
\end{enumerate}

Fun Facts:
~\begin{enumerate}[label=\arabic*)]
	\item If $ \int_{-\infty}^{\infty} |f(x)| dx = M<\infty $ then
		\[
			|\hat{ f}(m)|\leq \frac{1}{2\pi} \int_{-\infty}^{\infty} |f(x)| \cdot |e^{-imx}| dx = \frac{1}{2\pi} \int_{-\infty}^{\infty} |f(x)|dx = \frac{M}{2\pi}   
		.\]
	\item Note that
		\[
			\hat{ f}(0)= \frac{1}{2\pi} \int_{-\infty}^{\infty} f(x) dx  = \frac{1}{2\pi} \cdot [ \text{ area under the curve }f(x) ]
		.\] 
	\item If $ f(x)$ is real then  $ \hat{ f}(m)= \overline{\hat{ f}(m)}$.
	\item If $ f(x)$ is even then  $ \hat{ f}(m)$ is even, likewise for odd.
	\item The data $ f(x)$ is transformed to a representation  $ \hat{ f}(m)$ in the frequency domain.
\end{enumerate}

\begin{eg}[]
Suppose
\begin{equation*}
	f(x)=
\begin{cases}
	A, & -L<x<L\\
	\frac{A}{2}, & x=L \text{ or } x=-L\\
	0, & \text{ else} 
\end{cases}
\end{equation*}
Find the Fourier transform of $ f(x)$.

 \begin{align*}
	 \hat{ f}(m) &= \frac{1}{2\pi} \int_{-\infty}^{\infty} f(x)e^{-imx} dx  \\
	 &= \frac{1}{2\pi} \int_{-L}^{L} A e^{-imx} dx  \\
	 &= \frac{A}{2\pi i m} (e^{imL} - e^{-imL}) \\
	 &= \frac{A}{\pi m } \left( \frac{e^{imL}-e^{-imL}}{2i } \right) \\
	 &= \frac{A}{\pi m } \sin(mL ) \\
	 &= \frac{AL}{ \pi} \cdot \frac{\sin(mL )}{mL } \\
\end{align*}
Define the "sinc function" as $ \sinc = \frac{\sin z}{z }$, then
\[
	\hat{ f}(m) = \frac{AL}{\pi} \sinc(mL) = \mathcal{F}[f](m)
.\] 
Moreover, by applying inverse transform on both sides,
\[
	f(x) = \int_{-\infty}^{\infty} \frac{AL}{\pi } \sinc (mL) e^{imx} dm = \mathcal{F}^{-1}[\hat{ f}](x)
.\] 
\end{eg}

\begin{notation}
	Haberman textbook uses the negative exponential term for the transform, but they are equivalent by a change of variable. 
\end{notation}
\end{document}

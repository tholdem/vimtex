\documentclass[class=article,crop=false]{standalone} 
%Fall 2020
% Some basic packages
\usepackage{standalone}[subpreambles=true]
\usepackage[utf8]{inputenc}
\usepackage[T1]{fontenc}
\usepackage{textcomp}
\usepackage[english]{babel}
\usepackage{url}
\usepackage{graphicx}
\usepackage{float}
\usepackage{enumitem}


\pdfminorversion=7

% Don't indent paragraphs, leave some space between them
\usepackage{parskip}

% Hide page number when page is empty
\usepackage{emptypage}
\usepackage{subcaption}
\usepackage{multicol}
\usepackage[dvipsnames]{xcolor}


% Math stuff
\usepackage{amsmath, amsfonts, mathtools, amsthm, amssymb}
% Fancy script capitals
\usepackage{mathrsfs}
\usepackage{cancel}
% Bold math
\usepackage{bm}
% Some shortcuts
\newcommand{\rr}{\ensuremath{\mathbb{R}}}
\newcommand{\zz}{\ensuremath{\mathbb{Z}}}
\newcommand{\qq}{\ensuremath{\mathbb{Q}}}
\newcommand{\nn}{\ensuremath{\mathbb{N}}}
\newcommand{\ff}{\ensuremath{\mathbb{F}}}
\newcommand{\cc}{\ensuremath{\mathbb{C}}}
\renewcommand\O{\ensuremath{\emptyset}}
\newcommand{\norm}[1]{{\left\lVert{#1}\right\rVert}}
\renewcommand{\vec}[1]{{\mathbf{#1}}}
\newcommand\allbold[1]{{\boldmath\textbf{#1}}}

% Put x \to \infty below \lim
\let\svlim\lim\def\lim{\svlim\limits}

%Make implies and impliedby shorter
\let\implies\Rightarrow
\let\impliedby\Leftarrow
\let\iff\Leftrightarrow
\let\epsilon\varepsilon

% Add \contra symbol to denote contradiction
\usepackage{stmaryrd} % for \lightning
\newcommand\contra{\scalebox{1.5}{$\lightning$}}

% \let\phi\varphi

% Command for short corrections
% Usage: 1+1=\correct{3}{2}

\definecolor{correct}{HTML}{009900}
\newcommand\correct[2]{\ensuremath{\:}{\color{red}{#1}}\ensuremath{\to }{\color{correct}{#2}}\ensuremath{\:}}
\newcommand\green[1]{{\color{correct}{#1}}}

% horizontal rule
\newcommand\hr{
    \noindent\rule[0.5ex]{\linewidth}{0.5pt}
}

% hide parts
\newcommand\hide[1]{}

% si unitx
\usepackage{siunitx}
\sisetup{locale = FR}

% Environments
\makeatother
% For box around Definition, Theorem, \ldots
\usepackage[framemethod=TikZ]{mdframed}
\mdfsetup{skipabove=1em,skipbelow=0em}

%definition
\newenvironment{defn}[1][]{%
\ifstrempty{#1}%
{\mdfsetup{%
frametitle={%
\tikz[baseline=(current bounding box.east),outer sep=0pt]
\node[anchor=east,rectangle,fill=Emerald]
{\strut Definition};}}
}%
{\mdfsetup{%
frametitle={%
\tikz[baseline=(current bounding box.east),outer sep=0pt]
\node[anchor=east,rectangle,fill=Emerald]
{\strut Definition:~#1};}}%
}%
\mdfsetup{innertopmargin=10pt,linecolor=Emerald,%
linewidth=2pt,topline=true,%
frametitleaboveskip=\dimexpr-\ht\strutbox\relax
}
\begin{mdframed}[]\relax%
\label{#1}}{\end{mdframed}}


%theorem
%\newcounter{thm}[section]\setcounter{thm}{0}
%\renewcommand{\thethm}{\arabic{section}.\arabic{thm}}
\newenvironment{thm}[1][]{%
%\refstepcounter{thm}%
\ifstrempty{#1}%
{\mdfsetup{%
frametitle={%
\tikz[baseline=(current bounding box.east),outer sep=0pt]
\node[anchor=east,rectangle,fill=blue!20]
%{\strut Theorem~\thethm};}}
{\strut Theorem};}}
}%
{\mdfsetup{%
frametitle={%
\tikz[baseline=(current bounding box.east),outer sep=0pt]
\node[anchor=east,rectangle,fill=blue!20]
%{\strut Theorem~\thethm:~#1};}}%
{\strut Theorem:~#1};}}%
}%
\mdfsetup{innertopmargin=10pt,linecolor=blue!20,%
linewidth=2pt,topline=true,%
frametitleaboveskip=\dimexpr-\ht\strutbox\relax
}
\begin{mdframed}[]\relax%
\label{#1}}{\end{mdframed}}


%lemma
\newenvironment{lem}[1][]{%
\ifstrempty{#1}%
{\mdfsetup{%
frametitle={%
\tikz[baseline=(current bounding box.east),outer sep=0pt]
\node[anchor=east,rectangle,fill=Dandelion]
{\strut Lemma};}}
}%
{\mdfsetup{%
frametitle={%
\tikz[baseline=(current bounding box.east),outer sep=0pt]
\node[anchor=east,rectangle,fill=Dandelion]
{\strut Lemma:~#1};}}%
}%
\mdfsetup{innertopmargin=10pt,linecolor=Dandelion,%
linewidth=2pt,topline=true,%
frametitleaboveskip=\dimexpr-\ht\strutbox\relax
}
\begin{mdframed}[]\relax%
\label{#1}}{\end{mdframed}}

%corollary
\newenvironment{coro}[1][]{%
\ifstrempty{#1}%
{\mdfsetup{%
frametitle={%
\tikz[baseline=(current bounding box.east),outer sep=0pt]
\node[anchor=east,rectangle,fill=CornflowerBlue]
{\strut Corollary};}}
}%
{\mdfsetup{%
frametitle={%
\tikz[baseline=(current bounding box.east),outer sep=0pt]
\node[anchor=east,rectangle,fill=CornflowerBlue]
{\strut Corollary:~#1};}}%
}%
\mdfsetup{innertopmargin=10pt,linecolor=CornflowerBlue,%
linewidth=2pt,topline=true,%
frametitleaboveskip=\dimexpr-\ht\strutbox\relax
}
\begin{mdframed}[]\relax%
\label{#1}}{\end{mdframed}}

%proof
\newenvironment{prf}[1][]{%
\ifstrempty{#1}%
{\mdfsetup{%
frametitle={%
\tikz[baseline=(current bounding box.east),outer sep=0pt]
\node[anchor=east,rectangle,fill=SpringGreen]
{\strut Proof};}}
}%
{\mdfsetup{%
frametitle={%
\tikz[baseline=(current bounding box.east),outer sep=0pt]
\node[anchor=east,rectangle,fill=SpringGreen]
{\strut Proof:~#1};}}%
}%
\mdfsetup{innertopmargin=10pt,linecolor=SpringGreen,%
linewidth=2pt,topline=true,%
frametitleaboveskip=\dimexpr-\ht\strutbox\relax
}
\begin{mdframed}[]\relax%
\label{#1}}{\qed\end{mdframed}}


\theoremstyle{definition}

\newmdtheoremenv[nobreak=true]{definition}{Definition}
\newmdtheoremenv[nobreak=true]{prop}{Proposition}
\newmdtheoremenv[nobreak=true]{theorem}{Theorem}
\newmdtheoremenv[nobreak=true]{corollary}{Corollary}
\newtheorem*{eg}{Example}
\theoremstyle{remark}
\newtheorem*{case}{Case}
\newtheorem*{notation}{Notation}
\newtheorem*{remark}{Remark}
\newtheorem*{note}{Note}
\newtheorem*{problem}{Problem}
\newtheorem*{observe}{Observe}
\newtheorem*{property}{Property}
\newtheorem*{intuition}{Intuition}


% End example and intermezzo environments with a small diamond (just like proof
% environments end with a small square)
\usepackage{etoolbox}
\AtEndEnvironment{vb}{\null\hfill$\diamond$}%
\AtEndEnvironment{intermezzo}{\null\hfill$\diamond$}%
% \AtEndEnvironment{opmerking}{\null\hfill$\diamond$}%

% Fix some spacing
% http://tex.stackexchange.com/questions/22119/how-can-i-change-the-spacing-before-theorems-with-amsthm
\makeatletter
\def\thm@space@setup{%
  \thm@preskip=\parskip \thm@postskip=0pt
}

% Fix some stuff
% %http://tex.stackexchange.com/questions/76273/multiple-pdfs-with-page-group-included-in-a-single-page-warning
\pdfsuppresswarningpagegroup=1


% My name
\author{Jaden Wang}



\begin{document}
\section{Convergence Part 2}
~\begin{defn}[Dirichlet Kernel]
	We define \allbold{Dirichlet Kernel} to be: 
\[
	D_N\left( \frac{\pi u}{L} \right) = \frac{1}{2}+\sum_{ n=1}^{ N} \cos \left( \frac{ n\pi u}{ L} \right)  
.\] 
\end{defn}
\begin{property}
~\begin{enumerate}[label=\arabic*)]
	\item we have $\frac{1}{L}\int_{-L-\delta}^{L-\delta} D_N \left( \frac{\pi u}{L} \right) du =1 $ for any $\delta \in \rr$.
	\item 
		\[
			D_N\left( \frac{\pi u}{L} \right) = \frac{\sin[(N+\frac{1}{2}) \frac{\pi}{L}u]}{2 \sin\left( \frac{\pi}{2L} u \right) }
		.\] 
\end{enumerate}
\end{property}
\begin{prf}
\begin{enumerate}[label=\arabic*)]
	\item Prove by direct integration.
	\item use $ 2 \sin(\alpha )\cos(\beta )=\sin(\beta+ \alpha )-\sin(\beta -\alpha )$ to show
		\[
			\sin\left( \frac{u}{2} \right) +\sum_{ n= 1}^{ N} 2\sin(\frac{u}{2} )\cos(nu )=\sin[(N+\frac{1}{2})u]
		.\]
		The sum will telescope away.
\end{enumerate}
\end{prf}
We will prove pointwise convergence first.

Recall that the adjusted function $\tilde{\overline{f}} $ just average the discontinuities.
\newpage

\begin{thm}[Dirichlet]
	Suppose $f(x)$ is a piecewise smooth function on $[-L,L]$ and let $\tilde{ f } (x)$ denote the periodic extension of the adjusted function. For any fixed integer $N>0$ and at each point  $x$, we can define the  $N$th partial sum of the Fourier Series representing  $f(x)$ as
	\[
		S_N(x)=a_0 +\sum_{ n= 1}^{ N} a_n\cos \left( \frac{ n\pi x}{ L} \right) + \sum_{ n= 1}^{ N} b_n \sin \left( \frac{ n\pi x}{ L} \right) 
	.\]
	then 
\[
	\text{ F.S.} [ f]( x) = \tilde{ f } (x) \quad \forall \  x \in [-L,L]
,\]
and
	\begin{enumerate}[label=\arabic*)]
		 \item If $\tilde{f} (x)$ is continuous at any $x_0$ then
			 \[
				 \lim_{ N \to \infty} S_N(x_0) = \tilde{f} (x_0)
			 .\] 
		 \item If $ \tilde{ f } (x)$ is discontinuous at any real $ x_0$ then
			 \[
				 \lim_{ N \to \infty} S_N(x_0) = \frac{\tilde{ f } (x_0^-)+\tilde{ f } (x_0^+)}{2}
			 .\] 
	\end{enumerate}
	That is, $ \text{ F.S.} [ f]( x) =\tilde{\bar{f}} (x) \quad \forall \ x$.
\end{thm}

\begin{prf}[Pointwise Convergence]
	Given $x_0 \in \rr$, we will use the Riemann-Lebesgue Lemma, so we need to first write $S_N(x_0)$ in integral form. Note that:
	\[
		a_n \cos \left( \frac{ n\pi x}{ L} \right) = \frac{1}{L} \int_{-L}^{L} f(t) \cos \left( \frac{ n\pi t}{ L} \right) \cos \left( \frac{ n\pi x}{ L} \right) dt 
	.\] 

and in general, for each $n \geq 1$, we can write the $n$th element in the integral form:
 \begin{align*}
	 a_n \cos \left( \frac{ n\pi x}{ L} \right) + b_n \sin \left( \frac{ n\pi x}{ L} \right) &= \frac{1}{L} \int_{-L}^{L} f(t) \cos \left( \frac{ n\pi t}{ L} \right) \cos \left( \frac{ n\pi x}{ L} \right) \\
	 &\quad  + \frac{1}{L} \int_{-L}^{L} f(t) \sin \left( \frac{ n\pi t}{ L} \right) \sin \left( \frac{ n\pi x}{ L} \right) dt   \\
	&= \frac{1}{L}\int_{-L}^{L} f(t) \left[ \cos \left( \frac{ n\pi t}{ L} \right) \cos \left( \frac{ n\pi x}{ L} \right) + \sin \left( \frac{ n\pi t}{ L} \right) \sin \left( \frac{ n\pi x}{ L} \right)  \right]dt  \\ 
	&=  \frac{1}{L}\int_{-L}^{L} f(t) \left[ \cos \left( \frac{ n\pi (t-x)}{ L} \right)  \right] dt  
\end{align*}

Thus, we have
\begin{align*}
	S_N(x_0) &= a_0 + \sum_{ n=1}^{ N} a_n \cos \left( \frac{ n\pi x_0}{ L} \right) + \sum_{ n=1}^{ N} b_n \sin \left( \frac{ n\pi x_0}{ L} \right)  \\
		 &= \frac{1}{2L} \int_{-L}^{L} f(t) dt + \sum_{ n=1}^{ N} \frac{1}{L} f(t) \left[ \cos \left( \frac{ n\pi (t-x_0)}{ L} \right)  \right]  dt \\
		 &= \frac{1}{L} \int_{-L}^{L} f(t) \left[ \frac{1}{2} + \sum_{ n=1}^{ N} \cos \left( \frac{ n\pi (t-x_0)}{ L} \right)  \right] dt  
\end{align*}


we now apply the Dirichlet Kernel result to $S_N(x_0) - \tilde{f}(x_0) $. 
\begin{align*}
	S_N(x_0) = \frac{1}{L} \int_{-L}^{L} f(t) D_N\left(\frac{\pi (t-x_0)}{L}\right) dt 
\end{align*}
Now, using substitution $u = t-x_0$, $t=u+x_0$, we have
\[
	\frac{1}{L} \int_{-L-x_0}^{L-x_0} D_N\left( \frac{\pi u}{L} \right) du = 1 \implies \frac{1}{L} \int_{-L}^{L} D_N\left( \frac{\pi (t-x_0)}{L} \right) dt = 1  
.\] 
so multiplying both sides of the equation by $\tilde{ f } (x_0)$ yields
\[
	1=\frac{1}{L}\int_{-L}^{L} D_N\left( \frac{\pi u}{L} \right) dt \implies \tilde{ f } (x_0) = \frac{1}{L} \int_{-L}^{L} \tilde{ f } (x_0) D_N\left( \frac{\pi (t-x_0)}{L} \right) dt  
.\] 
then applying Property 2 of Dirichlet Kernel yields 
\begin{align*}
	S_N(x_0)-\tilde{ f } (x_0) &= \frac{1}{L} \int_{-L}^{L} (f(t) - \tilde{ f } (x_0) D_N\left( \frac{\pi(t-x_0)}{L} \right) dt  \\
				   &= \frac{1}{L} \int_{-L}^{L} (f(t) -\tilde{ f } (x_0)) \frac{\sin\left( (N+\frac{1}{2}) \frac{\pi}{L} (t-x_0) \right) }{2\sin(\frac{\pi}{2L}(t-x_0) )} \\
\end{align*}

Finally we will apply the Riemann-Lebesgue Lemma. Note that if we let $ u=t-x_0$ and $M=(N+\frac{1}{2})\frac{\pi}{L}$, we have
\[
	\frac{1}{L}\int_{-L-x_0}^{L-x_0} \frac{f(u+x_0)-\tilde{ f } (x_0)}{2\sin\left( \frac{\pi}{2L}u \right) } \sin\left( Mu \right)   
.\] 
Denote the quotient as $Q(u)$. Then the lemma implies that
 \[
	 \int_{-L-x_0}^{L-x_0} Q(u) \sin(Mu) du \to 0 \text{ as } M \to \infty   
.\] 
provided that $Q(u)$ is piecewise smooth. We proceed to show this. First, we show  $Q(u)$ is piecewise continuous. 

Since the quotient of two continuous functions is continuous where defined we claim that  $Q(u)$, being the quotient of two piecewise continuous functions is piecewise continuous on its domain. Consider the denominator of  $Q(u)$ and its roots. WLOG (due to periodicity), assume that  $x_0 \in [-L,L]$. 
\begin{case}[1]
	If $x_0 \in (-L,L)$, then since $u \in [-L-x_0,L-x_0]$, we can show that $\sin\left( \frac{\pi}{2L} u \right) = 0 \iff u=0$, so we need to examine the limit $\lim_{ u \to 0} Q(u) $.
	Note that since $u+x_0 \in [-L,L]$ we have $f(u+x_0) = \tilde{ f } (u+x_0)$ so
	\[
		Q(u) = \frac{f(u+x_0)-\tilde{ f}(x_0)}{2\sin\left( \frac{\pi}{2L} u \right) } = \frac{\tilde{ f } (u+x_0) -\tilde{ f } (x_0)}{2\sin\left( \frac{\pi}{2L} u \right) }
	.\] 
	Using L'Hopital's Rule,
	\[
		\lim_{ u \to 0} Q(u) = \lim_{ u \to 0} \frac{\tilde{ f' }(u+x_0)}{\frac{\pi}{L}\cos \left( \frac{ \pi}{ 2L} u \right) } = \tilde{ f' } (x_0) \frac{L}{\pi} < \infty
	.\]
	Since $ x_0 \in (-L,L)$, $ f'(x_0)$ is well-defined. This implies there is a removable discontinuity at $ u=0$ so  $ Q(u)$ is piecewise continuous if  $ |x_0|<L$.
\end{case}
\begin{case}[2]
	Suppose $|x_0| =L$, and WLOG assume $x_0=-L$ so $u  \in [0,2L]$. Notice
	\[
		\sin\left( \frac{\pi}{2L}u \right) =0 \implies u=0 \text{ or } u=2L
	.\] 
	So we only need to check these two cases. As before, since $ u+x_0 \in[-L,L]$ we can interchange $ f$ and $ \tilde{ f }$.
	By the continuity and differentiability of $\tilde{ f } (x)$ (at 0), we can show $\lim_{ u \to 0^+} Q(u) = \lim_{ u \to 0} Q(u) = \tilde{ f }' (x_0)\frac{L}{\pi} < \infty$ so there is a removable discontinuity at $u=0$. 

	At  $u=2L$, by periodicity, we have  $\tilde{ f } (x_0+2L) = \tilde{f} (x_0)$ so again using L'Hopital's Rule:
	\[
		\lim_{ u \to 2L^-} Q(u) = \lim_{ u \to 2L^-} \frac{\tilde{f'}(u+x_0) }{\frac{\pi}{L}\cos \left( \frac{ \pi}{ 2L} u \right) } = \tilde{ f' }(2L+x_0) \frac{-L}{\pi} = \tilde{f'}(x_0) \frac{-L}{\pi}   < \infty  
	.\]
	(Jaden: note $ f'(x_0)$ for both cases is only defined if $ \tilde{  f}( x) $ is differentiable for all $x \in [-L,L]$. Additional argument is required to prove it for all piecewise smooth functions. Here we seemed to skip the proof for the cases when $x_0$ happens to be jumped discontinuities.)
	Now we have established that $Q(u)$ is piecewise continuous. To show  $Q'(u)$ is also piecewise continuous we leave it to the readers. So  $Q(u)$ is piecewise smooth.
	By the lemma,  $S_N(x_0)-\tilde{ f } (x_0) \to 0$ as $N \to \infty$. That is
	\[
		\lim_{ N \to \infty} S_N(x_0)=\tilde{ f } (x_0) \iff \text{ F.S.} [ f](x_0) = \tilde{ f } (x_0)
	.\] 
	Thus we have established the pointwise convergence.
\end{case}
\end{prf}
\end{document}

\documentclass[12pt]{article}
%Fall 2020
% Some basic packages
\usepackage{standalone}[subpreambles=true]
\usepackage[utf8]{inputenc}
\usepackage[T1]{fontenc}
\usepackage{textcomp}
\usepackage[english]{babel}
\usepackage{url}
\usepackage{graphicx}
\usepackage{float}
\usepackage{enumitem}


\pdfminorversion=7

% Don't indent paragraphs, leave some space between them
\usepackage{parskip}

% Hide page number when page is empty
\usepackage{emptypage}
\usepackage{subcaption}
\usepackage{multicol}
\usepackage[dvipsnames]{xcolor}


% Math stuff
\usepackage{amsmath, amsfonts, mathtools, amsthm, amssymb}
% Fancy script capitals
\usepackage{mathrsfs}
\usepackage{cancel}
% Bold math
\usepackage{bm}
% Some shortcuts
\newcommand{\rr}{\ensuremath{\mathbb{R}}}
\newcommand{\zz}{\ensuremath{\mathbb{Z}}}
\newcommand{\qq}{\ensuremath{\mathbb{Q}}}
\newcommand{\nn}{\ensuremath{\mathbb{N}}}
\newcommand{\ff}{\ensuremath{\mathbb{F}}}
\newcommand{\cc}{\ensuremath{\mathbb{C}}}
\renewcommand\O{\ensuremath{\emptyset}}
\newcommand{\norm}[1]{{\left\lVert{#1}\right\rVert}}
\renewcommand{\vec}[1]{{\mathbf{#1}}}
\newcommand\allbold[1]{{\boldmath\textbf{#1}}}

% Put x \to \infty below \lim
\let\svlim\lim\def\lim{\svlim\limits}

%Make implies and impliedby shorter
\let\implies\Rightarrow
\let\impliedby\Leftarrow
\let\iff\Leftrightarrow
\let\epsilon\varepsilon

% Add \contra symbol to denote contradiction
\usepackage{stmaryrd} % for \lightning
\newcommand\contra{\scalebox{1.5}{$\lightning$}}

% \let\phi\varphi

% Command for short corrections
% Usage: 1+1=\correct{3}{2}

\definecolor{correct}{HTML}{009900}
\newcommand\correct[2]{\ensuremath{\:}{\color{red}{#1}}\ensuremath{\to }{\color{correct}{#2}}\ensuremath{\:}}
\newcommand\green[1]{{\color{correct}{#1}}}

% horizontal rule
\newcommand\hr{
    \noindent\rule[0.5ex]{\linewidth}{0.5pt}
}

% hide parts
\newcommand\hide[1]{}

% si unitx
\usepackage{siunitx}
\sisetup{locale = FR}

% Environments
\makeatother
% For box around Definition, Theorem, \ldots
\usepackage[framemethod=TikZ]{mdframed}
\mdfsetup{skipabove=1em,skipbelow=0em}

%definition
\newenvironment{defn}[1][]{%
\ifstrempty{#1}%
{\mdfsetup{%
frametitle={%
\tikz[baseline=(current bounding box.east),outer sep=0pt]
\node[anchor=east,rectangle,fill=Emerald]
{\strut Definition};}}
}%
{\mdfsetup{%
frametitle={%
\tikz[baseline=(current bounding box.east),outer sep=0pt]
\node[anchor=east,rectangle,fill=Emerald]
{\strut Definition:~#1};}}%
}%
\mdfsetup{innertopmargin=10pt,linecolor=Emerald,%
linewidth=2pt,topline=true,%
frametitleaboveskip=\dimexpr-\ht\strutbox\relax
}
\begin{mdframed}[]\relax%
\label{#1}}{\end{mdframed}}


%theorem
%\newcounter{thm}[section]\setcounter{thm}{0}
%\renewcommand{\thethm}{\arabic{section}.\arabic{thm}}
\newenvironment{thm}[1][]{%
%\refstepcounter{thm}%
\ifstrempty{#1}%
{\mdfsetup{%
frametitle={%
\tikz[baseline=(current bounding box.east),outer sep=0pt]
\node[anchor=east,rectangle,fill=blue!20]
%{\strut Theorem~\thethm};}}
{\strut Theorem};}}
}%
{\mdfsetup{%
frametitle={%
\tikz[baseline=(current bounding box.east),outer sep=0pt]
\node[anchor=east,rectangle,fill=blue!20]
%{\strut Theorem~\thethm:~#1};}}%
{\strut Theorem:~#1};}}%
}%
\mdfsetup{innertopmargin=10pt,linecolor=blue!20,%
linewidth=2pt,topline=true,%
frametitleaboveskip=\dimexpr-\ht\strutbox\relax
}
\begin{mdframed}[]\relax%
\label{#1}}{\end{mdframed}}


%lemma
\newenvironment{lem}[1][]{%
\ifstrempty{#1}%
{\mdfsetup{%
frametitle={%
\tikz[baseline=(current bounding box.east),outer sep=0pt]
\node[anchor=east,rectangle,fill=Dandelion]
{\strut Lemma};}}
}%
{\mdfsetup{%
frametitle={%
\tikz[baseline=(current bounding box.east),outer sep=0pt]
\node[anchor=east,rectangle,fill=Dandelion]
{\strut Lemma:~#1};}}%
}%
\mdfsetup{innertopmargin=10pt,linecolor=Dandelion,%
linewidth=2pt,topline=true,%
frametitleaboveskip=\dimexpr-\ht\strutbox\relax
}
\begin{mdframed}[]\relax%
\label{#1}}{\end{mdframed}}

%corollary
\newenvironment{coro}[1][]{%
\ifstrempty{#1}%
{\mdfsetup{%
frametitle={%
\tikz[baseline=(current bounding box.east),outer sep=0pt]
\node[anchor=east,rectangle,fill=CornflowerBlue]
{\strut Corollary};}}
}%
{\mdfsetup{%
frametitle={%
\tikz[baseline=(current bounding box.east),outer sep=0pt]
\node[anchor=east,rectangle,fill=CornflowerBlue]
{\strut Corollary:~#1};}}%
}%
\mdfsetup{innertopmargin=10pt,linecolor=CornflowerBlue,%
linewidth=2pt,topline=true,%
frametitleaboveskip=\dimexpr-\ht\strutbox\relax
}
\begin{mdframed}[]\relax%
\label{#1}}{\end{mdframed}}

%proof
\newenvironment{prf}[1][]{%
\ifstrempty{#1}%
{\mdfsetup{%
frametitle={%
\tikz[baseline=(current bounding box.east),outer sep=0pt]
\node[anchor=east,rectangle,fill=SpringGreen]
{\strut Proof};}}
}%
{\mdfsetup{%
frametitle={%
\tikz[baseline=(current bounding box.east),outer sep=0pt]
\node[anchor=east,rectangle,fill=SpringGreen]
{\strut Proof:~#1};}}%
}%
\mdfsetup{innertopmargin=10pt,linecolor=SpringGreen,%
linewidth=2pt,topline=true,%
frametitleaboveskip=\dimexpr-\ht\strutbox\relax
}
\begin{mdframed}[]\relax%
\label{#1}}{\qed\end{mdframed}}


\theoremstyle{definition}

\newmdtheoremenv[nobreak=true]{definition}{Definition}
\newmdtheoremenv[nobreak=true]{prop}{Proposition}
\newmdtheoremenv[nobreak=true]{theorem}{Theorem}
\newmdtheoremenv[nobreak=true]{corollary}{Corollary}
\newtheorem*{eg}{Example}
\theoremstyle{remark}
\newtheorem*{case}{Case}
\newtheorem*{notation}{Notation}
\newtheorem*{remark}{Remark}
\newtheorem*{note}{Note}
\newtheorem*{problem}{Problem}
\newtheorem*{observe}{Observe}
\newtheorem*{property}{Property}
\newtheorem*{intuition}{Intuition}


% End example and intermezzo environments with a small diamond (just like proof
% environments end with a small square)
\usepackage{etoolbox}
\AtEndEnvironment{vb}{\null\hfill$\diamond$}%
\AtEndEnvironment{intermezzo}{\null\hfill$\diamond$}%
% \AtEndEnvironment{opmerking}{\null\hfill$\diamond$}%

% Fix some spacing
% http://tex.stackexchange.com/questions/22119/how-can-i-change-the-spacing-before-theorems-with-amsthm
\makeatletter
\def\thm@space@setup{%
  \thm@preskip=\parskip \thm@postskip=0pt
}

% Fix some stuff
% %http://tex.stackexchange.com/questions/76273/multiple-pdfs-with-page-group-included-in-a-single-page-warning
\pdfsuppresswarningpagegroup=1


% My name
\author{Jaden Wang}



\begin{document}
\centerline {\textsf{\textbf{\LARGE{Homework 5}}}}
\centerline {Jaden Wang}
\vspace{.15in}

\begin{problem}[9.1]
Let's start tracing the orbit from 1: $ \{1,5,2\} $.\\
3: $ \{3\} $.\\
4: $ \{4,6\} $ \\
The orbits are: $ \{1,5,2\}, \{3\},\{4,6\}. $
\end{problem}
\begin{problem}[9.2]
1: $ \{1,5,8,7\} $.\\
2: $ \{2,6,3\} $.\\
4: $ \{4\} $.\\
The orbits are: $ \{1,5,8,7\}, \{2,6,3\},\{4\} $.
\end{problem}

\begin{problem}[9.7]
\begin{align*}
	\text{(1 4 5)(7 8)(2 5 7)}&=\text{(1 4 5 8 7 2)}\\
				  &= \begin{pmatrix} 1&2&3&4&5&6&7&8\\4&1&3&5&8&6&2&7 \end{pmatrix}   \\
\end{align*}
\end{problem}
\begin{problem}[9.8]
\begin{align*}
	\text{ (1 2)(4 7 8)(2 1)(7 2 8 1 5)} &= \text{(1 5 8)(2 4 7)} \\
					     &= \begin{pmatrix} 1&2&3&4&5&6&7&8\\5&4&3&7&8&6&2&1 \end{pmatrix}  \\
\end{align*}
\end{problem}
\begin{problem}[9.10]
By tracing the orbits we can rewrite the permutation as a product of disjointed cycles:\\

(1 8)(3 6 4)(5 7) = (1 8)(3 4)(3 6)(5 7).
\end{problem}

\begin{problem}[9.11]
	(1 3 4)(2 6)(5 8 7) = (1 4)(1 3)(2 6)(5 7)(5 8)
\end{problem}
\begin{problem}[9.23]
~\begin{enumerate}[label=\alph*)]
	\item False. A cycle requires at least one orbit with more than one elements. So the identity permutation is not a cycle.
	\item True. By definition.
	\item False. 9.15 eliminated a case when a permutation is both a product of even number and product of odd number of transpositions, so we don't have to consider such case in 9.18. 
	\item True. By Corollary 9.12, since a nontrivial subgroup of $ S_9$ contains permutations of at least two elements, they must be a product of transpositions. Hence some transpositions must be in the subgroup.
	\item False. $ |A_5|=5!/2=60$.
	\item False. $ S_1 = \{id\} $ is trivially cyclic with $ id$ as the generator.
	\item True.  $ |A_3|=3!/2=3$. We know that all groups with order 3 are isomorphic to $ \zz_3$ and it is commutative. So $ A_3$ must be isomorphic to that group and be commutative too.
	\item True. By fixing 8 we are only permuting 7 elements. This is isomorphic to $ S_7$.
	\item True. Same as above.
	\item False. The identity is an even permutation so this set of only odd permutations cannot be a subgroup.
\end{enumerate}
\end{problem}
\begin{problem}[9.24]
	$ A_3$ are:
\begin{align*}
	\rho^0 &= \text{ (1 2)(1 2)}  \\
	\rho^1 &= \text{(1 2 3)} = \text{(1 3)(1 2)} \\
\rho^2 &= \text{(1 3 2)} = \text{(1 2)(1 3)}  \\
\end{align*}
which are all products of even number of transpositions.
\begin{table}[H]
	\centering
	\begin{tabular}{c||c|c|c}
	& $ \rho^0 $ & $ \rho^1$ & $ \rho^2$\\
	\hline
	\hline
		$ \rho^0 $& $ \rho^0$ & $ \rho^1$ & $ \rho^2$\\
		\hline
		$ \rho^1 $ & $ \rho^1$ & $ \rho^2$& $ \rho^0$\\
		\hline
		$ \rho^2 $ & $ \rho^2$& $ \rho^0$ & $ \rho^1$\\

	\end{tabular}
\end{table}
\end{problem}
\end{document}

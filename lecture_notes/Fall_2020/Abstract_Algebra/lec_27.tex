\documentclass[class=article,crop=false]{standalone} 
%Fall 2020
% Some basic packages
\usepackage{standalone}[subpreambles=true]
\usepackage[utf8]{inputenc}
\usepackage[T1]{fontenc}
\usepackage{textcomp}
\usepackage[english]{babel}
\usepackage{url}
\usepackage{graphicx}
\usepackage{float}
\usepackage{enumitem}


\pdfminorversion=7

% Don't indent paragraphs, leave some space between them
\usepackage{parskip}

% Hide page number when page is empty
\usepackage{emptypage}
\usepackage{subcaption}
\usepackage{multicol}
\usepackage[dvipsnames]{xcolor}


% Math stuff
\usepackage{amsmath, amsfonts, mathtools, amsthm, amssymb}
% Fancy script capitals
\usepackage{mathrsfs}
\usepackage{cancel}
% Bold math
\usepackage{bm}
% Some shortcuts
\newcommand{\rr}{\ensuremath{\mathbb{R}}}
\newcommand{\zz}{\ensuremath{\mathbb{Z}}}
\newcommand{\qq}{\ensuremath{\mathbb{Q}}}
\newcommand{\nn}{\ensuremath{\mathbb{N}}}
\newcommand{\ff}{\ensuremath{\mathbb{F}}}
\newcommand{\cc}{\ensuremath{\mathbb{C}}}
\renewcommand\O{\ensuremath{\emptyset}}
\newcommand{\norm}[1]{{\left\lVert{#1}\right\rVert}}
\renewcommand{\vec}[1]{{\mathbf{#1}}}
\newcommand\allbold[1]{{\boldmath\textbf{#1}}}

% Put x \to \infty below \lim
\let\svlim\lim\def\lim{\svlim\limits}

%Make implies and impliedby shorter
\let\implies\Rightarrow
\let\impliedby\Leftarrow
\let\iff\Leftrightarrow
\let\epsilon\varepsilon

% Add \contra symbol to denote contradiction
\usepackage{stmaryrd} % for \lightning
\newcommand\contra{\scalebox{1.5}{$\lightning$}}

% \let\phi\varphi

% Command for short corrections
% Usage: 1+1=\correct{3}{2}

\definecolor{correct}{HTML}{009900}
\newcommand\correct[2]{\ensuremath{\:}{\color{red}{#1}}\ensuremath{\to }{\color{correct}{#2}}\ensuremath{\:}}
\newcommand\green[1]{{\color{correct}{#1}}}

% horizontal rule
\newcommand\hr{
    \noindent\rule[0.5ex]{\linewidth}{0.5pt}
}

% hide parts
\newcommand\hide[1]{}

% si unitx
\usepackage{siunitx}
\sisetup{locale = FR}

% Environments
\makeatother
% For box around Definition, Theorem, \ldots
\usepackage[framemethod=TikZ]{mdframed}
\mdfsetup{skipabove=1em,skipbelow=0em}

%definition
\newenvironment{defn}[1][]{%
\ifstrempty{#1}%
{\mdfsetup{%
frametitle={%
\tikz[baseline=(current bounding box.east),outer sep=0pt]
\node[anchor=east,rectangle,fill=Emerald]
{\strut Definition};}}
}%
{\mdfsetup{%
frametitle={%
\tikz[baseline=(current bounding box.east),outer sep=0pt]
\node[anchor=east,rectangle,fill=Emerald]
{\strut Definition:~#1};}}%
}%
\mdfsetup{innertopmargin=10pt,linecolor=Emerald,%
linewidth=2pt,topline=true,%
frametitleaboveskip=\dimexpr-\ht\strutbox\relax
}
\begin{mdframed}[]\relax%
\label{#1}}{\end{mdframed}}


%theorem
%\newcounter{thm}[section]\setcounter{thm}{0}
%\renewcommand{\thethm}{\arabic{section}.\arabic{thm}}
\newenvironment{thm}[1][]{%
%\refstepcounter{thm}%
\ifstrempty{#1}%
{\mdfsetup{%
frametitle={%
\tikz[baseline=(current bounding box.east),outer sep=0pt]
\node[anchor=east,rectangle,fill=blue!20]
%{\strut Theorem~\thethm};}}
{\strut Theorem};}}
}%
{\mdfsetup{%
frametitle={%
\tikz[baseline=(current bounding box.east),outer sep=0pt]
\node[anchor=east,rectangle,fill=blue!20]
%{\strut Theorem~\thethm:~#1};}}%
{\strut Theorem:~#1};}}%
}%
\mdfsetup{innertopmargin=10pt,linecolor=blue!20,%
linewidth=2pt,topline=true,%
frametitleaboveskip=\dimexpr-\ht\strutbox\relax
}
\begin{mdframed}[]\relax%
\label{#1}}{\end{mdframed}}


%lemma
\newenvironment{lem}[1][]{%
\ifstrempty{#1}%
{\mdfsetup{%
frametitle={%
\tikz[baseline=(current bounding box.east),outer sep=0pt]
\node[anchor=east,rectangle,fill=Dandelion]
{\strut Lemma};}}
}%
{\mdfsetup{%
frametitle={%
\tikz[baseline=(current bounding box.east),outer sep=0pt]
\node[anchor=east,rectangle,fill=Dandelion]
{\strut Lemma:~#1};}}%
}%
\mdfsetup{innertopmargin=10pt,linecolor=Dandelion,%
linewidth=2pt,topline=true,%
frametitleaboveskip=\dimexpr-\ht\strutbox\relax
}
\begin{mdframed}[]\relax%
\label{#1}}{\end{mdframed}}

%corollary
\newenvironment{coro}[1][]{%
\ifstrempty{#1}%
{\mdfsetup{%
frametitle={%
\tikz[baseline=(current bounding box.east),outer sep=0pt]
\node[anchor=east,rectangle,fill=CornflowerBlue]
{\strut Corollary};}}
}%
{\mdfsetup{%
frametitle={%
\tikz[baseline=(current bounding box.east),outer sep=0pt]
\node[anchor=east,rectangle,fill=CornflowerBlue]
{\strut Corollary:~#1};}}%
}%
\mdfsetup{innertopmargin=10pt,linecolor=CornflowerBlue,%
linewidth=2pt,topline=true,%
frametitleaboveskip=\dimexpr-\ht\strutbox\relax
}
\begin{mdframed}[]\relax%
\label{#1}}{\end{mdframed}}

%proof
\newenvironment{prf}[1][]{%
\ifstrempty{#1}%
{\mdfsetup{%
frametitle={%
\tikz[baseline=(current bounding box.east),outer sep=0pt]
\node[anchor=east,rectangle,fill=SpringGreen]
{\strut Proof};}}
}%
{\mdfsetup{%
frametitle={%
\tikz[baseline=(current bounding box.east),outer sep=0pt]
\node[anchor=east,rectangle,fill=SpringGreen]
{\strut Proof:~#1};}}%
}%
\mdfsetup{innertopmargin=10pt,linecolor=SpringGreen,%
linewidth=2pt,topline=true,%
frametitleaboveskip=\dimexpr-\ht\strutbox\relax
}
\begin{mdframed}[]\relax%
\label{#1}}{\qed\end{mdframed}}


\theoremstyle{definition}

\newmdtheoremenv[nobreak=true]{definition}{Definition}
\newmdtheoremenv[nobreak=true]{prop}{Proposition}
\newmdtheoremenv[nobreak=true]{theorem}{Theorem}
\newmdtheoremenv[nobreak=true]{corollary}{Corollary}
\newtheorem*{eg}{Example}
\theoremstyle{remark}
\newtheorem*{case}{Case}
\newtheorem*{notation}{Notation}
\newtheorem*{remark}{Remark}
\newtheorem*{note}{Note}
\newtheorem*{problem}{Problem}
\newtheorem*{observe}{Observe}
\newtheorem*{property}{Property}
\newtheorem*{intuition}{Intuition}


% End example and intermezzo environments with a small diamond (just like proof
% environments end with a small square)
\usepackage{etoolbox}
\AtEndEnvironment{vb}{\null\hfill$\diamond$}%
\AtEndEnvironment{intermezzo}{\null\hfill$\diamond$}%
% \AtEndEnvironment{opmerking}{\null\hfill$\diamond$}%

% Fix some spacing
% http://tex.stackexchange.com/questions/22119/how-can-i-change-the-spacing-before-theorems-with-amsthm
\makeatletter
\def\thm@space@setup{%
  \thm@preskip=\parskip \thm@postskip=0pt
}

% Fix some stuff
% %http://tex.stackexchange.com/questions/76273/multiple-pdfs-with-page-group-included-in-a-single-page-warning
\pdfsuppresswarningpagegroup=1


% My name
\author{Jaden Wang}



\begin{document}
\begin{thm}[]
Suppose $ S \subseteq  R$. $ S \leq R$ is a subring if
 \begin{enumerate}[label=(\roman*)]
	\item $ 0_R \in S$ or check $ S \neq \O$ if we use it in junction with the negation axiom.
	\item Closed under $ +$.
	\item Closed under  $ -$ or negation: if  $ a,b \in S$ then $ a-b \in S$. Or if $ a \in S$ then $ -a \in S$.
	\item Closed under $ \times $: if $ a,b \in S$ then $ a \times b \in S$.
\end{enumerate}
\end{thm}
\begin{note}[]
	The first three proves that $ S$ is a subgroup.
\end{note}

\begin{eg}[integral domain]
	$ \zz$ is an integral domain (but not a field). Having inverses doesn't mean an element is not a zero divisor.

In a group, recall the cancellation laws: $ ab=ac \implies b=c, ba=ca\implies b=c$.

In $ \zz$, if $ 3x=3y$, then  $ x=y$. If  $ ab=ac$, then  $ a=0$ or  $ b=c$.
\end{eg}

\begin{prf}
Suppose $ ab=ac$. Then
 \begin{align*}
	ab-ac&= 0 \\
	a(b-c)&= 0 \\
	a=0 \text{ or } & b-c=0 \text{ since no zero divisors}\\  
	a=0 \text{ or } &b=c 
\end{align*}
\end{prf}
\begin{thm}[]
If $ R$ is an integral domain and  $ ab=ac$, then  $ a=0$ or  $ b=c$. The other direction follows from commutativity.
\end{thm}
\begin{eg}[bad]
	$ \zz_{12}$ is not an integral domain. It is commutative, has identity 1, but has zero divisors like $ 3\times 4=0$. In general, any composite (non-prime) order of $ \zz_n$ is not an integral domain.

	Consider $ x^2-5x+6=0$ in $ \zz_{12}$. 2,3,6,11 are all solutions. 
	\[
		(6-2)(6-3)=4\times 3=0
	.\] 
So what are the zero divisors of $ \zz_{12}$?

It's 2,3,4,6,8,9,10. It happens that the non-zero divisors 1,5,7,11 are units. Their inverses are themselves. They form a group isomorphic to $ V_4$. 

\begin{prop}[]
In $ \zz_n$, any nonzero element is either a zero divisor or a unit.
\end{prop}
\begin{note}[]
In $ \zz$, this is not true. Units are $ \{\pm 1\} $ but there is no zero divisors.
\end{note}

\begin{prf}
If $ a$ is a zero divisor then  $ ab=0$ for  $ a\neq 0, b\neq 0$.
If  $ a$ is a unit, then there exists  $ c \in R: ac=ca=1$.
\begin{align*}
	ab&=0\\
	(ca)b&= c0 \\
	1b&= 0 \\
	b &= 0 \\
\end{align*}
which is a contradiction.
\end{prf}
\end{eg}
\begin{thm}[]
	Any field is an integral domain.
\end{thm}

\begin{prf}
	\begin{enumerate}[label=(\roman*)]
		\item $ F$ is commutative by definition of field.
		\item $ F$ has identity by definition of division ring.
		\item  $ F$ has no zero divisors. Suppose  $ ab=0, a\neq 0, b\neq 0$. Then  $ a$ is a unit and cannot be a zero divisor.  Division ring forces all nonzero elements units.
	\end{enumerate}
\end{prf}

Is every integral domain a field? No. $ \zz$.

\begin{thm}[]
	Every finite integral domain is a field.
\end{thm}
\begin{coro}[]
	$ \zz_p$ is a field.
\end{coro}
\begin{intuition}[]
Why is $ 3$ a unit in  $ \zz_7$?

The function $ f_3: \zz_7 \to \zz_7, x\mapsto 3x \mod 7$ is injective. Thus $ f_3$ is surjective. Hence $ f_3(y)=1$ for some $ y$.
\end{intuition}
\begin{prf}
We use the pigeonhole principle. If we have $ f:A \to B$, $ |A|=|B|=5$. Then  $ f$ injective  $ \iff$ f surjective.

Let $ R = \{a_1=1,a_2,\ldots,a_n\} $. Let $ a \in R\setminus \{0\} $. We need to show $ a$ has a multiplicative inverse.

Consider the sequence  $ \{aa_1,aa_2,\ldots,aa_n\} $. Suppose there is a repeat $ a a_i=a a_j$, by cancellation law (since $ a\neq 0$), $ a_i = a_j$. So no such repeat exists. Therefore, multiplication by $ a$ is injective, thus it's surjective, thus  $ ax=1$ holds for some  $ x$. 
\end{prf}
\begin{eg}[zero divisors in $ \zz_{12}$]
Immediately we can say 2,3,4,6 are zero divisors because they are factors of 12.

All multiples of these are zero divisors too. 8,9,10 are such multiples.

Therefore, as long as $ \gcd ( a,n) \neq 1 $, $ a$ is a zero divisor.

\begin{note}[]
In $ \zz$, the gcd of $ a$ and $ b$ is of the form  $ ra+sb$ where  $ r,s \in \zz$.
\end{note}

So 5 is coprime to 12 means $ \gcd ( 5,12)=1 \implies 5r+12s=1$ in integers. Let $ r=5,s=-2$ so it works.
\end{eg}
\end{document}

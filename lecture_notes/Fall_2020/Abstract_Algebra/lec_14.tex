\documentclass[class=article,crop=false]{standalone} 
%Fall 2020
% Some basic packages
\usepackage{standalone}[subpreambles=true]
\usepackage[utf8]{inputenc}
\usepackage[T1]{fontenc}
\usepackage{textcomp}
\usepackage[english]{babel}
\usepackage{url}
\usepackage{graphicx}
\usepackage{float}
\usepackage{enumitem}


\pdfminorversion=7

% Don't indent paragraphs, leave some space between them
\usepackage{parskip}

% Hide page number when page is empty
\usepackage{emptypage}
\usepackage{subcaption}
\usepackage{multicol}
\usepackage[dvipsnames]{xcolor}


% Math stuff
\usepackage{amsmath, amsfonts, mathtools, amsthm, amssymb}
% Fancy script capitals
\usepackage{mathrsfs}
\usepackage{cancel}
% Bold math
\usepackage{bm}
% Some shortcuts
\newcommand{\rr}{\ensuremath{\mathbb{R}}}
\newcommand{\zz}{\ensuremath{\mathbb{Z}}}
\newcommand{\qq}{\ensuremath{\mathbb{Q}}}
\newcommand{\nn}{\ensuremath{\mathbb{N}}}
\newcommand{\ff}{\ensuremath{\mathbb{F}}}
\newcommand{\cc}{\ensuremath{\mathbb{C}}}
\renewcommand\O{\ensuremath{\emptyset}}
\newcommand{\norm}[1]{{\left\lVert{#1}\right\rVert}}
\renewcommand{\vec}[1]{{\mathbf{#1}}}
\newcommand\allbold[1]{{\boldmath\textbf{#1}}}

% Put x \to \infty below \lim
\let\svlim\lim\def\lim{\svlim\limits}

%Make implies and impliedby shorter
\let\implies\Rightarrow
\let\impliedby\Leftarrow
\let\iff\Leftrightarrow
\let\epsilon\varepsilon

% Add \contra symbol to denote contradiction
\usepackage{stmaryrd} % for \lightning
\newcommand\contra{\scalebox{1.5}{$\lightning$}}

% \let\phi\varphi

% Command for short corrections
% Usage: 1+1=\correct{3}{2}

\definecolor{correct}{HTML}{009900}
\newcommand\correct[2]{\ensuremath{\:}{\color{red}{#1}}\ensuremath{\to }{\color{correct}{#2}}\ensuremath{\:}}
\newcommand\green[1]{{\color{correct}{#1}}}

% horizontal rule
\newcommand\hr{
    \noindent\rule[0.5ex]{\linewidth}{0.5pt}
}

% hide parts
\newcommand\hide[1]{}

% si unitx
\usepackage{siunitx}
\sisetup{locale = FR}

% Environments
\makeatother
% For box around Definition, Theorem, \ldots
\usepackage[framemethod=TikZ]{mdframed}
\mdfsetup{skipabove=1em,skipbelow=0em}

%definition
\newenvironment{defn}[1][]{%
\ifstrempty{#1}%
{\mdfsetup{%
frametitle={%
\tikz[baseline=(current bounding box.east),outer sep=0pt]
\node[anchor=east,rectangle,fill=Emerald]
{\strut Definition};}}
}%
{\mdfsetup{%
frametitle={%
\tikz[baseline=(current bounding box.east),outer sep=0pt]
\node[anchor=east,rectangle,fill=Emerald]
{\strut Definition:~#1};}}%
}%
\mdfsetup{innertopmargin=10pt,linecolor=Emerald,%
linewidth=2pt,topline=true,%
frametitleaboveskip=\dimexpr-\ht\strutbox\relax
}
\begin{mdframed}[]\relax%
\label{#1}}{\end{mdframed}}


%theorem
%\newcounter{thm}[section]\setcounter{thm}{0}
%\renewcommand{\thethm}{\arabic{section}.\arabic{thm}}
\newenvironment{thm}[1][]{%
%\refstepcounter{thm}%
\ifstrempty{#1}%
{\mdfsetup{%
frametitle={%
\tikz[baseline=(current bounding box.east),outer sep=0pt]
\node[anchor=east,rectangle,fill=blue!20]
%{\strut Theorem~\thethm};}}
{\strut Theorem};}}
}%
{\mdfsetup{%
frametitle={%
\tikz[baseline=(current bounding box.east),outer sep=0pt]
\node[anchor=east,rectangle,fill=blue!20]
%{\strut Theorem~\thethm:~#1};}}%
{\strut Theorem:~#1};}}%
}%
\mdfsetup{innertopmargin=10pt,linecolor=blue!20,%
linewidth=2pt,topline=true,%
frametitleaboveskip=\dimexpr-\ht\strutbox\relax
}
\begin{mdframed}[]\relax%
\label{#1}}{\end{mdframed}}


%lemma
\newenvironment{lem}[1][]{%
\ifstrempty{#1}%
{\mdfsetup{%
frametitle={%
\tikz[baseline=(current bounding box.east),outer sep=0pt]
\node[anchor=east,rectangle,fill=Dandelion]
{\strut Lemma};}}
}%
{\mdfsetup{%
frametitle={%
\tikz[baseline=(current bounding box.east),outer sep=0pt]
\node[anchor=east,rectangle,fill=Dandelion]
{\strut Lemma:~#1};}}%
}%
\mdfsetup{innertopmargin=10pt,linecolor=Dandelion,%
linewidth=2pt,topline=true,%
frametitleaboveskip=\dimexpr-\ht\strutbox\relax
}
\begin{mdframed}[]\relax%
\label{#1}}{\end{mdframed}}

%corollary
\newenvironment{coro}[1][]{%
\ifstrempty{#1}%
{\mdfsetup{%
frametitle={%
\tikz[baseline=(current bounding box.east),outer sep=0pt]
\node[anchor=east,rectangle,fill=CornflowerBlue]
{\strut Corollary};}}
}%
{\mdfsetup{%
frametitle={%
\tikz[baseline=(current bounding box.east),outer sep=0pt]
\node[anchor=east,rectangle,fill=CornflowerBlue]
{\strut Corollary:~#1};}}%
}%
\mdfsetup{innertopmargin=10pt,linecolor=CornflowerBlue,%
linewidth=2pt,topline=true,%
frametitleaboveskip=\dimexpr-\ht\strutbox\relax
}
\begin{mdframed}[]\relax%
\label{#1}}{\end{mdframed}}

%proof
\newenvironment{prf}[1][]{%
\ifstrempty{#1}%
{\mdfsetup{%
frametitle={%
\tikz[baseline=(current bounding box.east),outer sep=0pt]
\node[anchor=east,rectangle,fill=SpringGreen]
{\strut Proof};}}
}%
{\mdfsetup{%
frametitle={%
\tikz[baseline=(current bounding box.east),outer sep=0pt]
\node[anchor=east,rectangle,fill=SpringGreen]
{\strut Proof:~#1};}}%
}%
\mdfsetup{innertopmargin=10pt,linecolor=SpringGreen,%
linewidth=2pt,topline=true,%
frametitleaboveskip=\dimexpr-\ht\strutbox\relax
}
\begin{mdframed}[]\relax%
\label{#1}}{\qed\end{mdframed}}


\theoremstyle{definition}

\newmdtheoremenv[nobreak=true]{definition}{Definition}
\newmdtheoremenv[nobreak=true]{prop}{Proposition}
\newmdtheoremenv[nobreak=true]{theorem}{Theorem}
\newmdtheoremenv[nobreak=true]{corollary}{Corollary}
\newtheorem*{eg}{Example}
\theoremstyle{remark}
\newtheorem*{case}{Case}
\newtheorem*{notation}{Notation}
\newtheorem*{remark}{Remark}
\newtheorem*{note}{Note}
\newtheorem*{problem}{Problem}
\newtheorem*{observe}{Observe}
\newtheorem*{property}{Property}
\newtheorem*{intuition}{Intuition}


% End example and intermezzo environments with a small diamond (just like proof
% environments end with a small square)
\usepackage{etoolbox}
\AtEndEnvironment{vb}{\null\hfill$\diamond$}%
\AtEndEnvironment{intermezzo}{\null\hfill$\diamond$}%
% \AtEndEnvironment{opmerking}{\null\hfill$\diamond$}%

% Fix some spacing
% http://tex.stackexchange.com/questions/22119/how-can-i-change-the-spacing-before-theorems-with-amsthm
\makeatletter
\def\thm@space@setup{%
  \thm@preskip=\parskip \thm@postskip=0pt
}

% Fix some stuff
% %http://tex.stackexchange.com/questions/76273/multiple-pdfs-with-page-group-included-in-a-single-page-warning
\pdfsuppresswarningpagegroup=1


% My name
\author{Jaden Wang}



\begin{document}

\begin{defn}[order]
Let $ G$ be a group and let $ a \in G$. The \allbold{order} of $ a$ is the number of elements in  $ \langle a \rangle$. Alternatively, the \allbold{order} of $ a$ is the smallest  $ n>0$ such that  $ a^{n}=e$ or $ \infty$ if no such $ n$ exists. We denote the order of  $ a$ by  $ |a|$.
\end{defn}

\section{Cosets and Lagrange's Theorem}

See iPad screenshots for what cosets look like. We cut $ G$ into pieces and each is the same size as  $ H$. This forms a partition of $ G$. None of them are empty, intersection is empty, and union is the whole group. Only one can be a subgroup since identity can only exist in one of them.

\begin{defn}[]
	$ G$ is a group,  $ H \leq G$. Define a relation,  $ \sim_L$, on $ G$,  such that $ a \sim_L b \implies a^{-1}b \in H$ (where the inverse has to be on the left).
\end{defn}
WLOG everything below are similar for right cosets.
\begin{thm}[]
$ \sim_L$ is an equivalence relationship.
\end{thm}
\begin{prf}
\begin{enumerate}[label=(\roman*)]
	\item Reflexive: need $ a \sim_L a$ for $ a \in G$. $ a^{-1}a = e \in H$ since $ H$ is a subgroup. 
	\item Symmetric: if $ a \sim_L b$, then $ b \sim_L a$. Since $ a^{-1}b$ and $ b ^{-1} a$ are inverses. Since $ H$ is closed under inverses, hence $ b ^{-1}a \in H$.
	\item transitive: If $ a \sim_L b$ and $ b \sim_L c$, then $ a \sim_L c$. If $ a^{-1} b \in H$ and $ b ^{-1}c \in H$, then $ a^{-1} b b ^{-1} c = a^{-1} c \in H$ since $ H$ is closed under operation.
\end{enumerate}
\end{prf}
\begin{note}[]
We used all necessary conditions of a subgroup for the above proof.
\end{note}


What is $ [a]$, the equivalence class of  $ a$?
 \[
	 [a] = \{g \in G: a \sim_L g\} 
.\]
$ a \sim_L g$ means $ a^{-1} g \in H \iff a^{-1} g = h \in H \iff g=ah, h \in H$.
Then
\[
	[a] = \{ah:h \in H\} \coloneqq aH
.\] 
\begin{defn}[cosets]
Group $ G$ and subgroup  $ H\leq G$. The \allbold{left cosets of $ H$ in  $ G$} is $ aH = \{ah: h \in H\} $. The \allbold{right cosets of $ H$ in  $ G$} is $ Ha=\{ha: h \in H\} $  .
\end{defn}
\begin{property}
\begin{itemize}
	\item Left cosets of $ H$ partition  $ G$. (Any two left cosets are equal or disjoint).
	\item $ xH = yH$ does  \allbold{NOT} mean that $ x=y$!!!!
	\item Consider  $ xH= \{xh: h \in H\} $. Then $xH $ contains  $ x$ because  $ e \in H$. So $ xH$ is the left coset containing  $ x$. No other coset of $ H$ does because they form a partition of $ G$. 
	\item $ eH \{eh: h \in H\} =H $. So $ H $ is one of the left cosets.
	\item there is a bijection between $ H$ and  $ xH$. Can always multiply by  $ x^{-1}$ to undo it. This means that any two left cosets have the same size, and any two right cosets have the same size.
	\item there is a bijection between left cosets of $ H$ and right cosets of  $ H$. We can just take inverses of each element in the left coset:
		 \begin{align*}
			xh&= \{xh : h \in H\}  \\
			&= \{ h^{-1}x^{-1}: h \in H\}  \\
			&= \{ h x^{-1}: h \in H\}  \\
			&= H x^{-1} 
		\end{align*}
		This is a bijection. So there are same number of left cosets as right cosets.
\end{itemize}
\end{property}

Question: What is the condition for $ x$ and  $ y$ to be in the same cosets. Answer:  $ x \sim y$.

\begin{thm}[]
$ xH = yH \iff x^{-1}y \in H$ and $ Hx = Hy \iff xy^{-1} \in H$.
\end{thm}
\begin{note}[]
The important thing is for the inverse to be on the left side for left coset.
\end{note}
\begin{thm}[Lagrange's Theorem]
	Let $ G$ be a finite group and let  $ H \leq G$. Then  $ |G|$ is equal to  $ |H|$ times the number of cosets of  $ H$ in  $ G$.
\end{thm}

See iPad for picture. Each set has the same size and together they form a partition.
\begin{thm}[]
$ |H|$ divides  $ |G|$.
\end{thm}

$ H \leq V_4$:  $ H$ cannot have size 3 because 3 cannot divide 4.

\begin{note}[]
The converse of Lagrange's Theorem is false.
\end{note}
\begin{eg}[]
$ A_4$ is a group of order $ 4!/2= 12$. But  $ A_4$ has no subgroup of order 6, although Lagrange's Theorem allows it. This is the smallest example for this.
\end{eg}
\begin{eg}[]
$ D_6$ has order 12 and does have a subgroup of order 6 which are the rotations.
\end{eg}

Question: Is $ D_6$ isomorphic to $ A_4$? No. Because they have a structural difference described above.

\begin{defn}[index]
	Let $ G$ be a group and  $ H \leq G$. The  \allbold{index of $ H$ in  $ G$} is the number of cosets of $ H$ in  $ G$. We denote this by $|G:H|$ or $ \{G:H\} $ or $ (G:H)$.  
\end{defn}

So Lagrange's Theorem implies $ |G|=|H| \times \{G:H\} $.

\begin{eg}[infinite group]
$ G= \zz$ and  $ H=3 \zz$. If the operation is addition-like, we write $ a+H$ for  $ aH$.  If $ G$  is abelian, left and right cosets are the same. What are the cosets?

$ x+H = y+H \iff y-x \in H$.
\end{eg}
\end{document}

\documentclass[12pt]{article}
%Fall 2020
% Some basic packages
\usepackage{standalone}[subpreambles=true]
\usepackage[utf8]{inputenc}
\usepackage[T1]{fontenc}
\usepackage{textcomp}
\usepackage[english]{babel}
\usepackage{url}
\usepackage{graphicx}
\usepackage{float}
\usepackage{enumitem}


\pdfminorversion=7

% Don't indent paragraphs, leave some space between them
\usepackage{parskip}

% Hide page number when page is empty
\usepackage{emptypage}
\usepackage{subcaption}
\usepackage{multicol}
\usepackage[dvipsnames]{xcolor}


% Math stuff
\usepackage{amsmath, amsfonts, mathtools, amsthm, amssymb}
% Fancy script capitals
\usepackage{mathrsfs}
\usepackage{cancel}
% Bold math
\usepackage{bm}
% Some shortcuts
\newcommand{\rr}{\ensuremath{\mathbb{R}}}
\newcommand{\zz}{\ensuremath{\mathbb{Z}}}
\newcommand{\qq}{\ensuremath{\mathbb{Q}}}
\newcommand{\nn}{\ensuremath{\mathbb{N}}}
\newcommand{\ff}{\ensuremath{\mathbb{F}}}
\newcommand{\cc}{\ensuremath{\mathbb{C}}}
\renewcommand\O{\ensuremath{\emptyset}}
\newcommand{\norm}[1]{{\left\lVert{#1}\right\rVert}}
\renewcommand{\vec}[1]{{\mathbf{#1}}}
\newcommand\allbold[1]{{\boldmath\textbf{#1}}}

% Put x \to \infty below \lim
\let\svlim\lim\def\lim{\svlim\limits}

%Make implies and impliedby shorter
\let\implies\Rightarrow
\let\impliedby\Leftarrow
\let\iff\Leftrightarrow
\let\epsilon\varepsilon

% Add \contra symbol to denote contradiction
\usepackage{stmaryrd} % for \lightning
\newcommand\contra{\scalebox{1.5}{$\lightning$}}

% \let\phi\varphi

% Command for short corrections
% Usage: 1+1=\correct{3}{2}

\definecolor{correct}{HTML}{009900}
\newcommand\correct[2]{\ensuremath{\:}{\color{red}{#1}}\ensuremath{\to }{\color{correct}{#2}}\ensuremath{\:}}
\newcommand\green[1]{{\color{correct}{#1}}}

% horizontal rule
\newcommand\hr{
    \noindent\rule[0.5ex]{\linewidth}{0.5pt}
}

% hide parts
\newcommand\hide[1]{}

% si unitx
\usepackage{siunitx}
\sisetup{locale = FR}

% Environments
\makeatother
% For box around Definition, Theorem, \ldots
\usepackage[framemethod=TikZ]{mdframed}
\mdfsetup{skipabove=1em,skipbelow=0em}

%definition
\newenvironment{defn}[1][]{%
\ifstrempty{#1}%
{\mdfsetup{%
frametitle={%
\tikz[baseline=(current bounding box.east),outer sep=0pt]
\node[anchor=east,rectangle,fill=Emerald]
{\strut Definition};}}
}%
{\mdfsetup{%
frametitle={%
\tikz[baseline=(current bounding box.east),outer sep=0pt]
\node[anchor=east,rectangle,fill=Emerald]
{\strut Definition:~#1};}}%
}%
\mdfsetup{innertopmargin=10pt,linecolor=Emerald,%
linewidth=2pt,topline=true,%
frametitleaboveskip=\dimexpr-\ht\strutbox\relax
}
\begin{mdframed}[]\relax%
\label{#1}}{\end{mdframed}}


%theorem
%\newcounter{thm}[section]\setcounter{thm}{0}
%\renewcommand{\thethm}{\arabic{section}.\arabic{thm}}
\newenvironment{thm}[1][]{%
%\refstepcounter{thm}%
\ifstrempty{#1}%
{\mdfsetup{%
frametitle={%
\tikz[baseline=(current bounding box.east),outer sep=0pt]
\node[anchor=east,rectangle,fill=blue!20]
%{\strut Theorem~\thethm};}}
{\strut Theorem};}}
}%
{\mdfsetup{%
frametitle={%
\tikz[baseline=(current bounding box.east),outer sep=0pt]
\node[anchor=east,rectangle,fill=blue!20]
%{\strut Theorem~\thethm:~#1};}}%
{\strut Theorem:~#1};}}%
}%
\mdfsetup{innertopmargin=10pt,linecolor=blue!20,%
linewidth=2pt,topline=true,%
frametitleaboveskip=\dimexpr-\ht\strutbox\relax
}
\begin{mdframed}[]\relax%
\label{#1}}{\end{mdframed}}


%lemma
\newenvironment{lem}[1][]{%
\ifstrempty{#1}%
{\mdfsetup{%
frametitle={%
\tikz[baseline=(current bounding box.east),outer sep=0pt]
\node[anchor=east,rectangle,fill=Dandelion]
{\strut Lemma};}}
}%
{\mdfsetup{%
frametitle={%
\tikz[baseline=(current bounding box.east),outer sep=0pt]
\node[anchor=east,rectangle,fill=Dandelion]
{\strut Lemma:~#1};}}%
}%
\mdfsetup{innertopmargin=10pt,linecolor=Dandelion,%
linewidth=2pt,topline=true,%
frametitleaboveskip=\dimexpr-\ht\strutbox\relax
}
\begin{mdframed}[]\relax%
\label{#1}}{\end{mdframed}}

%corollary
\newenvironment{coro}[1][]{%
\ifstrempty{#1}%
{\mdfsetup{%
frametitle={%
\tikz[baseline=(current bounding box.east),outer sep=0pt]
\node[anchor=east,rectangle,fill=CornflowerBlue]
{\strut Corollary};}}
}%
{\mdfsetup{%
frametitle={%
\tikz[baseline=(current bounding box.east),outer sep=0pt]
\node[anchor=east,rectangle,fill=CornflowerBlue]
{\strut Corollary:~#1};}}%
}%
\mdfsetup{innertopmargin=10pt,linecolor=CornflowerBlue,%
linewidth=2pt,topline=true,%
frametitleaboveskip=\dimexpr-\ht\strutbox\relax
}
\begin{mdframed}[]\relax%
\label{#1}}{\end{mdframed}}

%proof
\newenvironment{prf}[1][]{%
\ifstrempty{#1}%
{\mdfsetup{%
frametitle={%
\tikz[baseline=(current bounding box.east),outer sep=0pt]
\node[anchor=east,rectangle,fill=SpringGreen]
{\strut Proof};}}
}%
{\mdfsetup{%
frametitle={%
\tikz[baseline=(current bounding box.east),outer sep=0pt]
\node[anchor=east,rectangle,fill=SpringGreen]
{\strut Proof:~#1};}}%
}%
\mdfsetup{innertopmargin=10pt,linecolor=SpringGreen,%
linewidth=2pt,topline=true,%
frametitleaboveskip=\dimexpr-\ht\strutbox\relax
}
\begin{mdframed}[]\relax%
\label{#1}}{\qed\end{mdframed}}


\theoremstyle{definition}

\newmdtheoremenv[nobreak=true]{definition}{Definition}
\newmdtheoremenv[nobreak=true]{prop}{Proposition}
\newmdtheoremenv[nobreak=true]{theorem}{Theorem}
\newmdtheoremenv[nobreak=true]{corollary}{Corollary}
\newtheorem*{eg}{Example}
\theoremstyle{remark}
\newtheorem*{case}{Case}
\newtheorem*{notation}{Notation}
\newtheorem*{remark}{Remark}
\newtheorem*{note}{Note}
\newtheorem*{problem}{Problem}
\newtheorem*{observe}{Observe}
\newtheorem*{property}{Property}
\newtheorem*{intuition}{Intuition}


% End example and intermezzo environments with a small diamond (just like proof
% environments end with a small square)
\usepackage{etoolbox}
\AtEndEnvironment{vb}{\null\hfill$\diamond$}%
\AtEndEnvironment{intermezzo}{\null\hfill$\diamond$}%
% \AtEndEnvironment{opmerking}{\null\hfill$\diamond$}%

% Fix some spacing
% http://tex.stackexchange.com/questions/22119/how-can-i-change-the-spacing-before-theorems-with-amsthm
\makeatletter
\def\thm@space@setup{%
  \thm@preskip=\parskip \thm@postskip=0pt
}

% Fix some stuff
% %http://tex.stackexchange.com/questions/76273/multiple-pdfs-with-page-group-included-in-a-single-page-warning
\pdfsuppresswarningpagegroup=1


% My name
\author{Jaden Wang}



\begin{document}
\centerline {\textsf{\textbf{\LARGE{Homework 11}}}}
\centerline {Jaden Wang}
\vspace{.15in}

\begin{problem}[20.2]
	Since 11 is prime, $ \zz_{11}$ is a field, and $ \phi(11)=10$. Therefore, we are trying to find a generator from 1 to 10 that generates the group $ U(\zz_{11})$ under $ \times _{11}$. 7 happens to work:
	\begin{table}[H]
		\centering
		\begin{tabular}{c|c|c|c|c|c|c|c|c|c|c}
			$ \times_{11} $ &7&5&2&3&10&4&6&9&8&1
		\end{tabular}
	\end{table}
	Thus, $ \langle 7 \rangle = U(\zz_{11})$. 
\end{problem}

\begin{problem}[20.4]
	By FlT, since 23 is prime,  $ 3^{23-1}=3^{22} \equiv 1 \bmod 23$.
\begin{align*}
	3^{47} &= 3^{44} \cdot 3^{3} \\
	&\equiv 1 \cdot 27 \bmod 23 \\
	&\equiv 4 \bmod 23 \\
\end{align*}
\end{problem}

\begin{problem}[20.5]
Since $ 7$ is a prime, by FlT  $ 37^{6}=1 \bmod 7 $.
\begin{align*}
	37^{49}&= 37^{6\times 8} \cdot 37 \\
	&\equiv 1 \cdot 37 \bmod 7 \\
	&\equiv 2 \bmod 7 \\
\end{align*}
\end{problem}

\begin{problem}[20.8]
	Notice that since $ \zz_{p^2}$ only has factor $ p$ which is prime, only multiples of $ p$ are not coprime with  $ p^2$ in $ \zz_{p^2}$. There are $ (p-1)$ such multiples in  $ \zz_{p^2}$. These multiples are the only zero divisors of $ \zz_{p^2}$. Thus by theorem, the number of units are the group order of nonzero elements $ (p^2-1)$ subtracting the number of zero divisors $ p-1$:
	\[
		\phi(p^2)=(p^2-1)-(p-1) = (p+1)(p-1)-(p-1) = p(p-1)
	.\] 
\end{problem}

\begin{problem}[20.10]
Since $ \gcd ( 7,24) =1$, we can apply Euler and obtain $ 7^{23}=1 \bmod 24$. Also notice $ 7^{2} \bmod 24 = 1$, so 7 to the odd power mod 24 is $ 7$. Therefore,
\[
7^{1000}=7^{43\times 23} \cdot 7^{11} \equiv 7 \bmod 24
.\] 
\end{problem}

\begin{problem}[20.13]
$ d=\gcd ( 36,24)=12 $. Clearly $ d$ doesn't divide 15, so there is no solution by theorem.
\end{problem}

\begin{problem}[20.14]
$ d= \gcd ( 45,24)=3 $. And $ 3 /15$. Now let's divide everything by 3:  $ a'=\frac{45}{3}=15, m'=\frac{24}{3}, b'=\frac{15}{3}=5$. Thus we have
\begin{align*}
	a'x &\equiv b' \bmod m'\\
	15 x &\equiv 5 \bmod 8\\
	8x+7x &\equiv 5 \bmod 8\\
	7x &\equiv 5 \bmod 8
\end{align*}
The units in $ \zz_8$ are 1,3,5,7. Notice $ 7 \times _8 7 \equiv 49 \bmod 8 \equiv 1 \bmod 8 $. So 7 is its own inverse in $ \zz_8$. Multiplying 7 on both sides:
\begin{align*}
	7 \times _8 7x &\equiv 7 \times _8 5\\
	x &\equiv 3\\
\end{align*}
\end{problem}

\begin{problem}[20.23]
	~\begin{enumerate}[label=\alph*)]
	\item False. If $ a=0 \in \zz$, then $ a^{p-1} \equiv 0 \bmod p$.
	\item True.
	\item True. Since $ \zz_n$ has order $ n$, the number of units must be less or equal to  $ n$.
	\item False. If  $ n=1$, then  $ \phi(n)$ is defined as $ 1 \neq 1-1=0$. 
	\item True. By theorem.
	\item True. Given units $ a,b \in \zz_n$, then $ b ^{-1}, a^{-1} \in \zz_n$, and the inverse of $ ab$ is  $ b ^{-1} a^{-1} \in \zz_n$.
	\item False. If $ a, \in \zz_n$ are nonunits, then $ \gcd ( a,n) \neq 1 $ and $ \gcd ( b,n) \neq 1 $. It follows that $ \gcd ( ab, n) \neq 1 $, which makes it not a unit.
	\item False. By the same gcd argument as above.
	\item False. Let $ a=0, b =1$,  $ 0x \equiv b \bmod p$ has no solution. 
	\item True. By theorem.
\end{enumerate}
\end{problem}

\begin{problem}[20.24]
\begin{table}[H]
	\centering
	\begin{tabular}{c||c|c|c|c}
		$ \times _{12}$ &1&5&7&11\\
		\hline
		\hline
		1&1&5&7&11\\
		\hline
		5&5&1&11&7\\
		\hline
		7&7&11&1&5\\
		\hline
		11 &11&7&5&1\\
	\end{tabular}
\end{table}
This is $ V_4$ because all elements are their own inverses.
\end{problem}

\begin{problem}[21.1]
	We guess that $ F= \{p+qi : p,q \in \qq\} $ is the field of fraction of $ D$. Recall that we have shown in HW9 18.12 that structures similar to this is a field. Moreover, given $ d = a+bi \in D$, $ a,b \in \zz \subseteq \qq$, so $ d \in F \implies D \subseteq F$. It remains to show that $ F$ is "not too big". That is, every element in  $ F$ can be expressed as a fraction of two elements in  $ D$.

	Given  $ \frac{r}{s}+\frac{t}{u} i \in F, r,s,t,u \in \zz, s,u \neq 0$, we have
	\[
	\frac{r}{s}+\frac{t}{u} i = \frac{ru+ sti}{su }= \frac{ru+sti}{su+0i }
	.\] 
	Since $ ru+sti, su+0i \in D, s,u\neq 0 \implies su+0i \neq 0$ since $ D$ has no zero divisors, this is indeed a well-defined fraction representation, as required.
\end{problem}

\begin{problem}[21.2]
	We guess that $ F= \{p+q\sqrt{2}  : p,q \in \qq\} $ is the field of fraction of $ D$. Recall that we have shown in HW9 18.12 that this is a field. Moreover, given $ d = a+b\sqrt{2}  \in D$, $ a,b \in \zz \subseteq \qq$, so $ d \in F \implies D \subseteq F$. It remains to show that $ F$ is "not too big". That is, every element in  $ F$ can be expressed as a fraction of two elements in  $ D$.

	Given  $ \frac{r}{s}+\frac{t}{u} \sqrt{2}  \in F, r,s,t,u \in \zz, s,u \neq 0$, we have
	\[
	\frac{r}{s}+\frac{t}{u} i = \frac{ru+ st \sqrt{2} }{su }= \frac{ru+st \sqrt{2} }{su+0 \sqrt{2}  }
	.\] 
	Since $ ru+st \sqrt{2} , su+0 \sqrt{2}  \in D, s,u\neq 0 \implies su+0 \sqrt{2}  \neq 0$ since $ D$ has no zero divisors, this is indeed a well-defined fraction representation, as required.
\
\end{problem}

\begin{problem}[21.4]
~\begin{enumerate}[label=\alph*)]
	\item True. 
	\item False. $ \qq$ is and field of fraction is unique up to isomorphism. 
	\item True. By theorem.
	\item False. $ \rr$ is and it's unique up to isomorphism.
	\item True. By theorem and uniqueness up to isomorphism.
	\item True. The first time was for cancellation law to prove transitivity. The second time was for proving the 2nd element is non zero in addition and multiplication operations.
	\item False. $ 0 \in D$ but $ 0$ cannot be a unit.
	\item True. By definition of a field and  $ D$ is contained in  $ F$.
	\item  Since $ D' \subseteq D \subseteq F$, so $ F$ is a field containing  $ D'$. Since  $ F'$ is the smallest field containing  $ D'$, it follows that $ F' \leq F$. 
	\item True. Since it's unique up to isomorphism.
\end{enumerate}
\end{problem}

\begin{problem}[21.5]
	Since $ \qq$ is a field, it is also a domain by theorem. Let $ D'=\zz$, and we know $ \zz leq \qq$ is a subdomain. Moreover, we know $ F'= \qq$, which is also the field of fractions of $ \qq$ itself as required.
\end{problem}
\end{document}

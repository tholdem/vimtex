\documentclass[12pt]{article}
%Fall 2020
% Some basic packages
\usepackage{standalone}[subpreambles=true]
\usepackage[utf8]{inputenc}
\usepackage[T1]{fontenc}
\usepackage{textcomp}
\usepackage[english]{babel}
\usepackage{url}
\usepackage{graphicx}
\usepackage{float}
\usepackage{enumitem}


\pdfminorversion=7

% Don't indent paragraphs, leave some space between them
\usepackage{parskip}

% Hide page number when page is empty
\usepackage{emptypage}
\usepackage{subcaption}
\usepackage{multicol}
\usepackage[dvipsnames]{xcolor}


% Math stuff
\usepackage{amsmath, amsfonts, mathtools, amsthm, amssymb}
% Fancy script capitals
\usepackage{mathrsfs}
\usepackage{cancel}
% Bold math
\usepackage{bm}
% Some shortcuts
\newcommand{\rr}{\ensuremath{\mathbb{R}}}
\newcommand{\zz}{\ensuremath{\mathbb{Z}}}
\newcommand{\qq}{\ensuremath{\mathbb{Q}}}
\newcommand{\nn}{\ensuremath{\mathbb{N}}}
\newcommand{\ff}{\ensuremath{\mathbb{F}}}
\newcommand{\cc}{\ensuremath{\mathbb{C}}}
\renewcommand\O{\ensuremath{\emptyset}}
\newcommand{\norm}[1]{{\left\lVert{#1}\right\rVert}}
\renewcommand{\vec}[1]{{\mathbf{#1}}}
\newcommand\allbold[1]{{\boldmath\textbf{#1}}}

% Put x \to \infty below \lim
\let\svlim\lim\def\lim{\svlim\limits}

%Make implies and impliedby shorter
\let\implies\Rightarrow
\let\impliedby\Leftarrow
\let\iff\Leftrightarrow
\let\epsilon\varepsilon

% Add \contra symbol to denote contradiction
\usepackage{stmaryrd} % for \lightning
\newcommand\contra{\scalebox{1.5}{$\lightning$}}

% \let\phi\varphi

% Command for short corrections
% Usage: 1+1=\correct{3}{2}

\definecolor{correct}{HTML}{009900}
\newcommand\correct[2]{\ensuremath{\:}{\color{red}{#1}}\ensuremath{\to }{\color{correct}{#2}}\ensuremath{\:}}
\newcommand\green[1]{{\color{correct}{#1}}}

% horizontal rule
\newcommand\hr{
    \noindent\rule[0.5ex]{\linewidth}{0.5pt}
}

% hide parts
\newcommand\hide[1]{}

% si unitx
\usepackage{siunitx}
\sisetup{locale = FR}

% Environments
\makeatother
% For box around Definition, Theorem, \ldots
\usepackage[framemethod=TikZ]{mdframed}
\mdfsetup{skipabove=1em,skipbelow=0em}

%definition
\newenvironment{defn}[1][]{%
\ifstrempty{#1}%
{\mdfsetup{%
frametitle={%
\tikz[baseline=(current bounding box.east),outer sep=0pt]
\node[anchor=east,rectangle,fill=Emerald]
{\strut Definition};}}
}%
{\mdfsetup{%
frametitle={%
\tikz[baseline=(current bounding box.east),outer sep=0pt]
\node[anchor=east,rectangle,fill=Emerald]
{\strut Definition:~#1};}}%
}%
\mdfsetup{innertopmargin=10pt,linecolor=Emerald,%
linewidth=2pt,topline=true,%
frametitleaboveskip=\dimexpr-\ht\strutbox\relax
}
\begin{mdframed}[]\relax%
\label{#1}}{\end{mdframed}}


%theorem
%\newcounter{thm}[section]\setcounter{thm}{0}
%\renewcommand{\thethm}{\arabic{section}.\arabic{thm}}
\newenvironment{thm}[1][]{%
%\refstepcounter{thm}%
\ifstrempty{#1}%
{\mdfsetup{%
frametitle={%
\tikz[baseline=(current bounding box.east),outer sep=0pt]
\node[anchor=east,rectangle,fill=blue!20]
%{\strut Theorem~\thethm};}}
{\strut Theorem};}}
}%
{\mdfsetup{%
frametitle={%
\tikz[baseline=(current bounding box.east),outer sep=0pt]
\node[anchor=east,rectangle,fill=blue!20]
%{\strut Theorem~\thethm:~#1};}}%
{\strut Theorem:~#1};}}%
}%
\mdfsetup{innertopmargin=10pt,linecolor=blue!20,%
linewidth=2pt,topline=true,%
frametitleaboveskip=\dimexpr-\ht\strutbox\relax
}
\begin{mdframed}[]\relax%
\label{#1}}{\end{mdframed}}


%lemma
\newenvironment{lem}[1][]{%
\ifstrempty{#1}%
{\mdfsetup{%
frametitle={%
\tikz[baseline=(current bounding box.east),outer sep=0pt]
\node[anchor=east,rectangle,fill=Dandelion]
{\strut Lemma};}}
}%
{\mdfsetup{%
frametitle={%
\tikz[baseline=(current bounding box.east),outer sep=0pt]
\node[anchor=east,rectangle,fill=Dandelion]
{\strut Lemma:~#1};}}%
}%
\mdfsetup{innertopmargin=10pt,linecolor=Dandelion,%
linewidth=2pt,topline=true,%
frametitleaboveskip=\dimexpr-\ht\strutbox\relax
}
\begin{mdframed}[]\relax%
\label{#1}}{\end{mdframed}}

%corollary
\newenvironment{coro}[1][]{%
\ifstrempty{#1}%
{\mdfsetup{%
frametitle={%
\tikz[baseline=(current bounding box.east),outer sep=0pt]
\node[anchor=east,rectangle,fill=CornflowerBlue]
{\strut Corollary};}}
}%
{\mdfsetup{%
frametitle={%
\tikz[baseline=(current bounding box.east),outer sep=0pt]
\node[anchor=east,rectangle,fill=CornflowerBlue]
{\strut Corollary:~#1};}}%
}%
\mdfsetup{innertopmargin=10pt,linecolor=CornflowerBlue,%
linewidth=2pt,topline=true,%
frametitleaboveskip=\dimexpr-\ht\strutbox\relax
}
\begin{mdframed}[]\relax%
\label{#1}}{\end{mdframed}}

%proof
\newenvironment{prf}[1][]{%
\ifstrempty{#1}%
{\mdfsetup{%
frametitle={%
\tikz[baseline=(current bounding box.east),outer sep=0pt]
\node[anchor=east,rectangle,fill=SpringGreen]
{\strut Proof};}}
}%
{\mdfsetup{%
frametitle={%
\tikz[baseline=(current bounding box.east),outer sep=0pt]
\node[anchor=east,rectangle,fill=SpringGreen]
{\strut Proof:~#1};}}%
}%
\mdfsetup{innertopmargin=10pt,linecolor=SpringGreen,%
linewidth=2pt,topline=true,%
frametitleaboveskip=\dimexpr-\ht\strutbox\relax
}
\begin{mdframed}[]\relax%
\label{#1}}{\qed\end{mdframed}}


\theoremstyle{definition}

\newmdtheoremenv[nobreak=true]{definition}{Definition}
\newmdtheoremenv[nobreak=true]{prop}{Proposition}
\newmdtheoremenv[nobreak=true]{theorem}{Theorem}
\newmdtheoremenv[nobreak=true]{corollary}{Corollary}
\newtheorem*{eg}{Example}
\theoremstyle{remark}
\newtheorem*{case}{Case}
\newtheorem*{notation}{Notation}
\newtheorem*{remark}{Remark}
\newtheorem*{note}{Note}
\newtheorem*{problem}{Problem}
\newtheorem*{observe}{Observe}
\newtheorem*{property}{Property}
\newtheorem*{intuition}{Intuition}


% End example and intermezzo environments with a small diamond (just like proof
% environments end with a small square)
\usepackage{etoolbox}
\AtEndEnvironment{vb}{\null\hfill$\diamond$}%
\AtEndEnvironment{intermezzo}{\null\hfill$\diamond$}%
% \AtEndEnvironment{opmerking}{\null\hfill$\diamond$}%

% Fix some spacing
% http://tex.stackexchange.com/questions/22119/how-can-i-change-the-spacing-before-theorems-with-amsthm
\makeatletter
\def\thm@space@setup{%
  \thm@preskip=\parskip \thm@postskip=0pt
}

% Fix some stuff
% %http://tex.stackexchange.com/questions/76273/multiple-pdfs-with-page-group-included-in-a-single-page-warning
\pdfsuppresswarningpagegroup=1


% My name
\author{Jaden Wang}



\begin{document}
\centerline {\textsf{\textbf{\LARGE{Homework 8}}}}
\centerline {Jaden Wang}
\vspace{.15in}
\begin{problem}[14.5]
	It suffices to find the index of the subgroup. Let $ G=\zz_2 \times \zz_4$, $ N = \langle (1,1) \rangle$. Then the index is just $ |G|/|N| $. That is,
	\[
	|G / N| = \frac{8}{ \lcm \left( \frac{2}{\gcd ( 1,2) }, \frac{4}{\gcd ( 1,4) } \right) } = \frac{8}{4} = 2
	.\] 
\end{problem}
\begin{problem}[14.6]
	Let $ G = \zz_{12} \times \zz_{18}$, $ N = \langle (4,3) \rangle$. Then
	\[
	|G/ N| = \frac{12 \times 18}{ \lcm \left( \frac{12}{\gcd ( 4,12), \frac{18}{ \gcd ( 3,18) } } \right)  } = \frac{12 \times 18}{ \lcm \left( 3,6 \right) } = 36 
	.\] 
\end{problem}

\begin{problem}[14.11]
	Let $ a = (2,1)$,  $ G=\zz_3 \times \zz_6$, $ N = \langle (1,1) \rangle$. First the elements of $ N$ are
	 \[
		 N = \{(0,0),(1,1),(2,2),(0,3),(1,4)\} 
	.\] 
	By theorem, the order of the coset $ a+N$ is the smallest positive integer $ n$ such that $ na \in N$ or infinity. Let's check:
	\begin{align*}
		2(2,1) &= (1,2) \not\in N\\
		3(2,1) &= (0,3) \in N
	\end{align*}
	Hence by theorem, $ |G /N|=3$.
\end{problem}
\begin{problem}[14.12]
	Let $ a = (3,1)$,  $ G=\zz_4 \times \zz_4$, $ N = \langle (1,1) \rangle$. The elements of $ N$ are
	 \[
		 N = \{(0,0),(1,1),(2,2),(3,3)\} 
	.\]
	Again we check:
	\begin{align*}
		2(3,1)=(2,2) \in N
	\end{align*}
	Hence by theorem, $ |G /N| = 2$.
\end{problem}

\begin{problem}[4.23]
\begin{enumerate}[label=\alph*)]
	\item True. By theorem 14.4.
	\item True. By a theorem from class.
	\item True. $\iota_g(x)= gxg^{-1} = g g^{-1} x = ex=x$ by commutativity.
	\item True. Because if $ G$ is finite, its normal subgroups are also finite, so the index is finite.
	\item True. 
	\item False.  $ \zz$ is torsion-free, but $ \zz / 3 \zz = \zz_3$, where each element of $ \zz_3$ has finite order and is thus a torsion group.
	\item True. By theorem from class.
	\item False. Counterexamples are $ GL_n(\rr) / SL_n(\rr) \simeq R^* $ and $ D_4 / \{\rho_0,\rho_2\} \simeq V_4 $. Both are abelian.
	\item Since $ \zz_n \coloneqq \zz / n \zz$ has order $ n$.
	\item False.  $ \rr / n \rr \simeq \rr$, which has infinite order. It is also not isomorphic to $ \zz$ and thus not cyclic.
\end{enumerate}
\end{problem}

\begin{problem}[15.1]
	Let $ G= \zz_2 \times \zz_4, N = \langle (0,1) \rangle$. By Theorem 15.8, $ N$ is a normal subgroup of $ G$. Then
	\[
	|G /N| = |G| / |N| = \frac{2 \times 4}{\frac{4}{\gcd ( 1,4) } } = \frac{8}{4}= 2
	.\] 
	Since there is only one group of order 2, $ G / N \simeq \zz_2$.
\end{problem}

\begin{problem}[15.2]
	Let $G = \zz_2 \times \zz_4$, $ N = \langle (0,2) \rangle$. Since $ G$ is abelian and  $ N\leq G$, $ N$ is a normal subgroup of $ G$. Then
	\[
	|G/ N| = |G| / |N| = \frac{8}{\frac{4}{\gcd ( 2,4) } } = \frac{8}{2} = 4
	.\] 
	It remains to determine if this is $ \zz_4$ or $ V_4$. Let's show that it is not cyclic. Notice for $ n=0,1$ and  $ m=0,1,2,3$, we can represent an arbitrary coset in the quotient group as $ (n,m)+ N$. Then let's find the order of  $ (n,m)+N$ (ignoring the trivial case when $ (n,m) \in N$ ):
	\begin{align*}
		2(n,m) &= (n+_{ 2} n, m+_{ 4} m )\\
		       &= (2n \mod 2, 2m \mod 4) \\
		       &=  
		       \begin{cases}
			       (0,0) \text{ if } m \text{ is even}\\
			       (0,2) \text{ if } m \text{ is odd}  
		       \end{cases}
		       \in N
	\end{align*}
	Thus, $ |(n,m)+N|$ is at most 2. But to be a generator of  $ G / N$ it requires order 4. Therefore, there is no generator in the quotient group, proving it is not cyclic. Then it must be  $ V_4$.
\end{problem}
\begin{problem}[15.3]
	Let $ G= \zz_2 \times \zz_4, N= \langle (1,2) \rangle$. Then
	 \[
		 N = \{(0,0), (1,2)\} 
	.\] 
	$ G$ is abelian and $ N\leq G$, therefore $ N \trianglelefteq G$.
	\[
	|G /N|  = |G| / |N| = \frac{8}{ \lcm \left( \frac{2}{\gcd ( 1,2) },\frac{4}{\gcd ( 2,4) } \right) } = \frac{8}{2} = 4
	.\] 
	Let $ g= (1,1) \not\in N, \in G$. Notice
	\begin{align*}
		2(1,1)&=(0,2) \not\in N\\
		3(1,1)&= (1,3) \not\in N \\
		4(1,1)&= (0,0) \in N 
	\end{align*}
	Hence $ |g+ N| =4$ and  $ g$ is a generator of  $ G /N$. Thus  $ G/ N$ is cyclic and isomorphic to $ \zz_4$.
\end{problem}

\begin{problem}[15.6]
	Let $ G=H \times K= \zz \times \zz$ and $ N=\langle (0,1) \rangle$. Notice $ N=\{(0,k):k \in K\} $. By theorem 15.8, $ N \trianglelefteq G$, and $ G / N \simeq H= \zz$.
\end{problem}
\begin{problem}[15.13]
	By table, we see that $ Z(D_4) = \{e, (1\ 3)(2\ 4) \} $ or $ \{\rho_0, \rho_2\} $.

	For $ C(D_4)$, we want to find the smallest normal subgroup with abelian quotient. $ D_4$ has the following normal subgroups:
	\begin{align*}
		D_4 \text{ of order }8: &D_4 /D_4 \simeq \{e\}, \text{ abelian}\\
		\{e\} \text{ of order 1}: & D_4 /\{e\} \simeq D_4 , \text{ nonabelian}\\
		Z(D_4) \text{ of order 2}: & D_4 /Z(D_4) \simeq V_4, \text{ abelian}\\
		\{ \text{ rotations } \} \text{ of order 4}: & D_4 / \{ \text{ rotations } \} \simeq\zz_2, \text{ abelian}    
	\end{align*}
	Clearly, the smallest normal subgroup with abelian quotient is $ Z(D_4)$. So $ C(G) = Z(D_4)$.
\end{problem}
\begin{problem}[15.19]
~\begin{enumerate}[label=\alph*)]
	\item True. By theorem 15.9, if $ a$ is a generator of  $ G$, then  $ aN$ is a generator of  $ G /N$.
	\item False.  $ S_n /A_n \simeq \zz_2$, but $ S_n$ is non-cyclic.
	\item False. Let $ x = \frac{1}{2}$, then $ 2x =1\in \zz$, so $ |x+\zz|=2$.
	\item True. Given $ n \in \zz^+$, choose $ x=\frac{1}{n}$ which is well-defined. Then $ nx = n\frac{1}{n}=1 \in \zz$, so $ |x+\zz|=n$.
	\item False. To find all elements in $ \rr /\zz$ of order 4, let $ x$ be a representative of any element, then $| x+\zz | =4 \iff 4x = a \in \zz$, where $ \gcd(a,4)=1$. Thus $ x$ needs to satisfy  $ x=\frac{a}{4}$ where $ \gcd(a,4)=1$ to be a representative of an element of order 4. To see how many such elements exist, let $ y+\zz$ be a different element where $ y=\frac{b}{4}, \gcd(b,4)=1$. Since  $ x+\zz \neq y+\zz \iff x-y \not\in \zz$, we have
		\begin{align*}
			x-y &\not\in \zz\\
			a-b &\not\in 4\zz\\
			a-b \mod 4 &\neq 0\\
			a \mod 4 +_{ 4} (-b \mod 4) &\neq 0\\
			a \mod 4 &\neq b \mod 4\\
			p &\neq q \in \zz_4
		\end{align*}
		Where $ p=a \mod 4, q=b \mod 4$. Since $ p,q \in \zz_4$, and $ \gcd(a,4)=\gcd(b,4)=1 \implies \gcd(p,4)=\gcd(q,4)=1$,  we see that there are only 2 distinct elements we can choose from $ \zz_4$, namely 1 and 3, to form distinct cosets of order 4. So it is not infinite.
	\item True. $C(G) = \{e\} \implies aba^{-1}b ^{-1} = e \implies ab(ba)^{-1}=e \implies ab=ba \ \forall \ a,b \in G$.
	\item False. $ \{e, (1\ 3)(2\ 4)\} \not \supseteq D_4 $.
	\item False. Let $ G$ be an abelian simple group. Then it only has two subgroups,  $ \{e\} $ and itself. $ G / \{e\}= G $ is abelian, and $ G / G =\{e\} $ is abelian.  Clearly $ \{e\} $ is the smallest normal subgroup that has abelian quotient, thus $ C(G) = \{e\} $.
	\item True. Let $ G$ be an nonabelian simple group, then by argument above,  $ G / \{e\} = G $ is nonabelian, so the only candidate left is $ G$. Thus,  $ C(G)=G$.
	\item False.  $ |A_5| = 5!$ is not prime order, yet $ A_5$ is finite and simple. 
\end{enumerate}
\end{problem}
\begin{problem}[15.22]
Let $ g: \rr \to \rr$ such that $ g \not\in K$. For $ |g+K|=2 $ we need  $ 2g \in K$. This implies that $ 2g $ is continuous, so it follows that  $ g$ is continuous, contradicting our assumption that  $ g \not\in K$. Hence, no such element exists.
\end{problem}

\begin{problem}[15.23]
	Let $ g: \rr \to \rr$ such that $ g \not\in K^* $. For $ |g+K|=2$ we need  $ g^2 \in K^* $. Consider
	 \begin{equation*}
		 g(x)=
	\begin{cases}
		1, & x \geq 0\\
		-1, & x<0\\
	\end{cases}
	\not\in K^* 
	\end{equation*}
However,
\begin{equation*}
	g^2(x)=
\begin{cases}
	1, & x\geq 0\\
	1, & x<0
\end{cases}
=1 \ \forall \ x \in \rr \implies  g^2 \in K^* 
\end{equation*}
\end{problem}
\begin{problem}[15.30]
~\begin{enumerate}[label=\alph*)]
	\item By definition of abelian group, every element commutes, hence $ Z(G)=G$.
	\item Since  $ Z(G) \trianglelefteq G$, and $ G$ is simple, the center is either  $ \{e\} $ or $ G$. Since  $ G$ is nonabelian, it must be that  $ Z(G)=\{e\} $.
\end{enumerate}
\end{problem}
\begin{problem}[15.31]
~\begin{enumerate}[label=\alph*)]
	\item If $ G$ is abelian, then we know  $ G /\{e\} = G $ is abelian, and clearly $ \{e\} $ is the smallest normal subgroup with abelian quotient. $ C(G)=\{e\} $.
	\item If $ G$ is nonabelian, since  $ G$ is simple, the only candidate left is  $ G$, where $ G /G = \{e\} $ is abelian. So  $C(G)=G $.
\end{enumerate}
\end{problem}
\end{document}

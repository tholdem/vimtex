\documentclass[class=article,crop=false]{standalone} 
%Fall 2020
% Some basic packages
\usepackage{standalone}[subpreambles=true]
\usepackage[utf8]{inputenc}
\usepackage[T1]{fontenc}
\usepackage{textcomp}
\usepackage[english]{babel}
\usepackage{url}
\usepackage{graphicx}
\usepackage{float}
\usepackage{enumitem}


\pdfminorversion=7

% Don't indent paragraphs, leave some space between them
\usepackage{parskip}

% Hide page number when page is empty
\usepackage{emptypage}
\usepackage{subcaption}
\usepackage{multicol}
\usepackage[dvipsnames]{xcolor}


% Math stuff
\usepackage{amsmath, amsfonts, mathtools, amsthm, amssymb}
% Fancy script capitals
\usepackage{mathrsfs}
\usepackage{cancel}
% Bold math
\usepackage{bm}
% Some shortcuts
\newcommand{\rr}{\ensuremath{\mathbb{R}}}
\newcommand{\zz}{\ensuremath{\mathbb{Z}}}
\newcommand{\qq}{\ensuremath{\mathbb{Q}}}
\newcommand{\nn}{\ensuremath{\mathbb{N}}}
\newcommand{\ff}{\ensuremath{\mathbb{F}}}
\newcommand{\cc}{\ensuremath{\mathbb{C}}}
\renewcommand\O{\ensuremath{\emptyset}}
\newcommand{\norm}[1]{{\left\lVert{#1}\right\rVert}}
\renewcommand{\vec}[1]{{\mathbf{#1}}}
\newcommand\allbold[1]{{\boldmath\textbf{#1}}}

% Put x \to \infty below \lim
\let\svlim\lim\def\lim{\svlim\limits}

%Make implies and impliedby shorter
\let\implies\Rightarrow
\let\impliedby\Leftarrow
\let\iff\Leftrightarrow
\let\epsilon\varepsilon

% Add \contra symbol to denote contradiction
\usepackage{stmaryrd} % for \lightning
\newcommand\contra{\scalebox{1.5}{$\lightning$}}

% \let\phi\varphi

% Command for short corrections
% Usage: 1+1=\correct{3}{2}

\definecolor{correct}{HTML}{009900}
\newcommand\correct[2]{\ensuremath{\:}{\color{red}{#1}}\ensuremath{\to }{\color{correct}{#2}}\ensuremath{\:}}
\newcommand\green[1]{{\color{correct}{#1}}}

% horizontal rule
\newcommand\hr{
    \noindent\rule[0.5ex]{\linewidth}{0.5pt}
}

% hide parts
\newcommand\hide[1]{}

% si unitx
\usepackage{siunitx}
\sisetup{locale = FR}

% Environments
\makeatother
% For box around Definition, Theorem, \ldots
\usepackage[framemethod=TikZ]{mdframed}
\mdfsetup{skipabove=1em,skipbelow=0em}

%definition
\newenvironment{defn}[1][]{%
\ifstrempty{#1}%
{\mdfsetup{%
frametitle={%
\tikz[baseline=(current bounding box.east),outer sep=0pt]
\node[anchor=east,rectangle,fill=Emerald]
{\strut Definition};}}
}%
{\mdfsetup{%
frametitle={%
\tikz[baseline=(current bounding box.east),outer sep=0pt]
\node[anchor=east,rectangle,fill=Emerald]
{\strut Definition:~#1};}}%
}%
\mdfsetup{innertopmargin=10pt,linecolor=Emerald,%
linewidth=2pt,topline=true,%
frametitleaboveskip=\dimexpr-\ht\strutbox\relax
}
\begin{mdframed}[]\relax%
\label{#1}}{\end{mdframed}}


%theorem
%\newcounter{thm}[section]\setcounter{thm}{0}
%\renewcommand{\thethm}{\arabic{section}.\arabic{thm}}
\newenvironment{thm}[1][]{%
%\refstepcounter{thm}%
\ifstrempty{#1}%
{\mdfsetup{%
frametitle={%
\tikz[baseline=(current bounding box.east),outer sep=0pt]
\node[anchor=east,rectangle,fill=blue!20]
%{\strut Theorem~\thethm};}}
{\strut Theorem};}}
}%
{\mdfsetup{%
frametitle={%
\tikz[baseline=(current bounding box.east),outer sep=0pt]
\node[anchor=east,rectangle,fill=blue!20]
%{\strut Theorem~\thethm:~#1};}}%
{\strut Theorem:~#1};}}%
}%
\mdfsetup{innertopmargin=10pt,linecolor=blue!20,%
linewidth=2pt,topline=true,%
frametitleaboveskip=\dimexpr-\ht\strutbox\relax
}
\begin{mdframed}[]\relax%
\label{#1}}{\end{mdframed}}


%lemma
\newenvironment{lem}[1][]{%
\ifstrempty{#1}%
{\mdfsetup{%
frametitle={%
\tikz[baseline=(current bounding box.east),outer sep=0pt]
\node[anchor=east,rectangle,fill=Dandelion]
{\strut Lemma};}}
}%
{\mdfsetup{%
frametitle={%
\tikz[baseline=(current bounding box.east),outer sep=0pt]
\node[anchor=east,rectangle,fill=Dandelion]
{\strut Lemma:~#1};}}%
}%
\mdfsetup{innertopmargin=10pt,linecolor=Dandelion,%
linewidth=2pt,topline=true,%
frametitleaboveskip=\dimexpr-\ht\strutbox\relax
}
\begin{mdframed}[]\relax%
\label{#1}}{\end{mdframed}}

%corollary
\newenvironment{coro}[1][]{%
\ifstrempty{#1}%
{\mdfsetup{%
frametitle={%
\tikz[baseline=(current bounding box.east),outer sep=0pt]
\node[anchor=east,rectangle,fill=CornflowerBlue]
{\strut Corollary};}}
}%
{\mdfsetup{%
frametitle={%
\tikz[baseline=(current bounding box.east),outer sep=0pt]
\node[anchor=east,rectangle,fill=CornflowerBlue]
{\strut Corollary:~#1};}}%
}%
\mdfsetup{innertopmargin=10pt,linecolor=CornflowerBlue,%
linewidth=2pt,topline=true,%
frametitleaboveskip=\dimexpr-\ht\strutbox\relax
}
\begin{mdframed}[]\relax%
\label{#1}}{\end{mdframed}}

%proof
\newenvironment{prf}[1][]{%
\ifstrempty{#1}%
{\mdfsetup{%
frametitle={%
\tikz[baseline=(current bounding box.east),outer sep=0pt]
\node[anchor=east,rectangle,fill=SpringGreen]
{\strut Proof};}}
}%
{\mdfsetup{%
frametitle={%
\tikz[baseline=(current bounding box.east),outer sep=0pt]
\node[anchor=east,rectangle,fill=SpringGreen]
{\strut Proof:~#1};}}%
}%
\mdfsetup{innertopmargin=10pt,linecolor=SpringGreen,%
linewidth=2pt,topline=true,%
frametitleaboveskip=\dimexpr-\ht\strutbox\relax
}
\begin{mdframed}[]\relax%
\label{#1}}{\qed\end{mdframed}}


\theoremstyle{definition}

\newmdtheoremenv[nobreak=true]{definition}{Definition}
\newmdtheoremenv[nobreak=true]{prop}{Proposition}
\newmdtheoremenv[nobreak=true]{theorem}{Theorem}
\newmdtheoremenv[nobreak=true]{corollary}{Corollary}
\newtheorem*{eg}{Example}
\theoremstyle{remark}
\newtheorem*{case}{Case}
\newtheorem*{notation}{Notation}
\newtheorem*{remark}{Remark}
\newtheorem*{note}{Note}
\newtheorem*{problem}{Problem}
\newtheorem*{observe}{Observe}
\newtheorem*{property}{Property}
\newtheorem*{intuition}{Intuition}


% End example and intermezzo environments with a small diamond (just like proof
% environments end with a small square)
\usepackage{etoolbox}
\AtEndEnvironment{vb}{\null\hfill$\diamond$}%
\AtEndEnvironment{intermezzo}{\null\hfill$\diamond$}%
% \AtEndEnvironment{opmerking}{\null\hfill$\diamond$}%

% Fix some spacing
% http://tex.stackexchange.com/questions/22119/how-can-i-change-the-spacing-before-theorems-with-amsthm
\makeatletter
\def\thm@space@setup{%
  \thm@preskip=\parskip \thm@postskip=0pt
}

% Fix some stuff
% %http://tex.stackexchange.com/questions/76273/multiple-pdfs-with-page-group-included-in-a-single-page-warning
\pdfsuppresswarningpagegroup=1


% My name
\author{Jaden Wang}



\begin{document}
\section*{26: Homomorphism and Quotient Rings}

Given a ring $ R$ and  $ S \leq R$ a subring, we wish to make  $ R /S = \{r+S: r \in R\} $ (additive cosets) into a ring. 

If we only consider the additive groups of $ R,S$. We already know that $ R /S$ is an abelian group because  $ R$ is an abelian group, so  $ S$ must be a normal subgroup. 

Now let's consider the multiplication:
\[
	(r_1+S)\times (r_2+S) = r_1 r_2 +S
.\] 
Let's try
\begin{align*}
	(r_1+s_1+ S) \times (r_2+s_2+S) &= (r_1+s_1)(r_2+s_2) + S \\
	&= r_1 r_2 + s_1 r_2 + r_1 s_2 + s_1 s_2 +S 
\end{align*}
We need $ (r_1 r_2 + s_1 r_2 + r_1 s_2 + s_1 +s_2)-r_1 r_2 = s_1 r_2 + r_1 s_2 + s_1 + s_2 \in S$.

This must work when $ r_1 = r_2 =0$, $ s_1 s_2 \in S \ \forall \ s_1, s_2 \in S$.

This must work when $ s_1 =0$, $ r_1 s_2 \in S \ \forall \ r_1 \in R, s_2 \in S$.

This must work when $ s_2=0$, $ s_1 r_2 \in S \ \forall \ s_1 \in S, r_2 \in R$.

By the above observation/requirements, we need to require $ S$ to be nonempty, closed under  $ +,-$, and if $ r \in R$ and $ s \in S$, then $ rs, sr \in S$.

\begin{defn}[ideal]
	Let $ R$ be a ring and let  $ I\leq R$. We say  $ I$ is a (two-sided)  \allbold{ideal} of $ R$ if
	 \begin{enumerate}[label=(\roman*)]
		\item $ 0 \in I$ or $ I \neq \O$.
		\item If $ i_1,i_2 \in I$ then $ i_1+i_2 \in I$ and $ i_1 - i_2 \in I$ (or negation).
		\item If $ i \in I$ and $ r \in R$ then $ ri \in I$ (left ideal) and $ ir \in I$ (right ideal).
	\end{enumerate}
	If $ I$ is an ideal of  $ R$, we write  $ I \trianglelefteq R$.
\end{defn}
\begin{note}[]
	The first two conditions are already given by subring. The third condition implies closure under multiplication.
\end{note}
\begin{eg}[]
$ \zz \leq \qq$ but $ \zz$ is not an ideal of $ \qq$. Take $ r=3 \in \zz$ and $ \frac{1}{7} \in \qq$, but $ 3 \times \frac{1}{7} \not\in \zz$.
\end{eg}

\begin{eg}[]
$ 2 \zz \trianglelefteq \zz$. We know this is a subring. If $ i \in 2 \zz$ and $ r \in \zz$, then $ ri= ir \in 2\zz$. "Multiplying an even integer by any integer gives an even integer."
\end{eg}

\begin{eg}[]
	$ R=M_2 (\rr)$ (matrix ring). Let $ s= \{ \begin{pmatrix} a&0\\b&0\\ \end{pmatrix} : a,b \in \rr \} $. It's a subring. 
	\[
		\begin{pmatrix} a&b\\c&d\\ \end{pmatrix} \begin{pmatrix} e&0\\f&0\\ \end{pmatrix} = \begin{pmatrix} ae+bf&0\\ce+df&0\\ \end{pmatrix} 
	.\] 
	But it wouldn't work on the right.
	\[
		\begin{pmatrix} 1&0\\1&0\\ \end{pmatrix} \begin{pmatrix} 1&1\\1&1\\ \end{pmatrix}  = \begin{pmatrix} 1&1\\1&1\\ \end{pmatrix} \not\in S
	.\] 
So this is a left ideal but not a ring ideal. If we swap the columns or transpose, we can get a right ideal.
\end{eg}


Let $ R$ be a ring and  $ I \trianglelefteq R$. Define $ R /I = \{r+I : r \in R\} $. Define
\begin{align*}
	(r+I)+(s+I) &= r+s+I\\
	(r+I) \times (s+I) &= rs + I \text{ well-defined} 
\end{align*}
To show that this is a ring, we need to show associativity of $ \times $ and distributive laws.

Distributivity:
\begin{align*}
	((r+I) + (s+I))(t+I) &= ((r+s)+I) \times (t+I) \\
			     &= (r+s) t + I \\
			     &= rt + st + I \\
			     &= (rt + I) +(st + I) \\
			     &= ((r+I)(t+I)) + ((s+I)(t+I))
\end{align*}
The rest are similar. Thus $ R /S$ is a ring!

\begin{eg}[]
Let $ n \in \nn$. Then $ n \zz \trianglelefteq \zz$. Then $ \zz /n\zz$ is a ring. This is $ \zz_n$ by definition.
\end{eg}
\begin{eg}[]
$ \zz_7 \coloneqq \zz / 7\zz$.
\begin{align*}
	(5+7 \zz)\times (4+7\zz) &= 20 + 7\zz \\
	&= 6+ 7\zz \text{ because } 20-6 \in 7\zz  
\end{align*}
So $ \overline{5} \times \overline{4} = \overline{6}$. 
\end{eg}

\begin{thm}[]
	If $ R$ is commutative, then  so is  $ R /I$.  (The converse is false.)

	If $ R$ has identity $ 1_R$, then so does  $ R /I$ which is  $ 1_R + I$.
\end{thm}

Let $ \phi: R \to S$ be a ring homomorphism. It preserves addition and multiplication. The kernel
\[
	\ker \phi = \{r \in R: \phi(r) = 0_S\} 
.\] 
\begin{thm}[]
$ \ker \phi \trianglelefteq R$.
\end{thm}

\begin{prf}
We know $ \ker \phi \leq R$ subgroup by applying group theory and ignoring multiplication. We also need to know if $ k \in \ker \phi, r \in R$, then $ rk \in \ker \phi$ and $ kr \in \ker \phi$.

\begin{align*}
	\phi(rk) = \phi(r) \phi(k) = \phi(r) \cdot  0 &= 0\\
	\phi(kr) &=0 
\end{align*}
Thus, $ rk \in \ker \phi, kr \in \ker \phi$.
\end{prf}

\begin{eg}[]

Let $ I \trianglelefteq R$. Consider the map $ \pi: R \to R /I, r \mapsto r+I$. Then $ \pi$ is a ring homomorphism.
\begin{align*}
	\pi(r+s) &= (r+s)+I\\
	\pi(r)+\pi(s) &= (r+I)+ (s+I) \\
	\pi(r\times s)&= rs+ I \\
	\pi(r) \times \pi(s) &= (r+I) (s+I) 
\end{align*}
$ \im \pi= R /I$. The zero element of $ R /I$ is  $ 0_R + I$.
 \begin{align*}
	 r \in \ker \phi &\iff \pi(r) = 0+I\\
			 & \iff r+I = 0+I\\
			 & \iff r-0 \in I \iff r \in I
\end{align*}
Summary: kernels of ring homomorphisms are ideals. Ideals are kernels of ring homomorphisms.
\end{eg}

\begin{thm}[]
Let $ \phi: R \to S$ be a homomorphism of rings. $ \phi$ is injective if and only if $ \ker \phi = \{0_R\} $.
\end{thm}
\end{document}

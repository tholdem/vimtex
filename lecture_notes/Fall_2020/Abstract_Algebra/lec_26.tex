\documentclass[class=article,crop=false]{standalone} 
%Fall 2020
% Some basic packages
\usepackage{standalone}[subpreambles=true]
\usepackage[utf8]{inputenc}
\usepackage[T1]{fontenc}
\usepackage{textcomp}
\usepackage[english]{babel}
\usepackage{url}
\usepackage{graphicx}
\usepackage{float}
\usepackage{enumitem}


\pdfminorversion=7

% Don't indent paragraphs, leave some space between them
\usepackage{parskip}

% Hide page number when page is empty
\usepackage{emptypage}
\usepackage{subcaption}
\usepackage{multicol}
\usepackage[dvipsnames]{xcolor}


% Math stuff
\usepackage{amsmath, amsfonts, mathtools, amsthm, amssymb}
% Fancy script capitals
\usepackage{mathrsfs}
\usepackage{cancel}
% Bold math
\usepackage{bm}
% Some shortcuts
\newcommand{\rr}{\ensuremath{\mathbb{R}}}
\newcommand{\zz}{\ensuremath{\mathbb{Z}}}
\newcommand{\qq}{\ensuremath{\mathbb{Q}}}
\newcommand{\nn}{\ensuremath{\mathbb{N}}}
\newcommand{\ff}{\ensuremath{\mathbb{F}}}
\newcommand{\cc}{\ensuremath{\mathbb{C}}}
\renewcommand\O{\ensuremath{\emptyset}}
\newcommand{\norm}[1]{{\left\lVert{#1}\right\rVert}}
\renewcommand{\vec}[1]{{\mathbf{#1}}}
\newcommand\allbold[1]{{\boldmath\textbf{#1}}}

% Put x \to \infty below \lim
\let\svlim\lim\def\lim{\svlim\limits}

%Make implies and impliedby shorter
\let\implies\Rightarrow
\let\impliedby\Leftarrow
\let\iff\Leftrightarrow
\let\epsilon\varepsilon

% Add \contra symbol to denote contradiction
\usepackage{stmaryrd} % for \lightning
\newcommand\contra{\scalebox{1.5}{$\lightning$}}

% \let\phi\varphi

% Command for short corrections
% Usage: 1+1=\correct{3}{2}

\definecolor{correct}{HTML}{009900}
\newcommand\correct[2]{\ensuremath{\:}{\color{red}{#1}}\ensuremath{\to }{\color{correct}{#2}}\ensuremath{\:}}
\newcommand\green[1]{{\color{correct}{#1}}}

% horizontal rule
\newcommand\hr{
    \noindent\rule[0.5ex]{\linewidth}{0.5pt}
}

% hide parts
\newcommand\hide[1]{}

% si unitx
\usepackage{siunitx}
\sisetup{locale = FR}

% Environments
\makeatother
% For box around Definition, Theorem, \ldots
\usepackage[framemethod=TikZ]{mdframed}
\mdfsetup{skipabove=1em,skipbelow=0em}

%definition
\newenvironment{defn}[1][]{%
\ifstrempty{#1}%
{\mdfsetup{%
frametitle={%
\tikz[baseline=(current bounding box.east),outer sep=0pt]
\node[anchor=east,rectangle,fill=Emerald]
{\strut Definition};}}
}%
{\mdfsetup{%
frametitle={%
\tikz[baseline=(current bounding box.east),outer sep=0pt]
\node[anchor=east,rectangle,fill=Emerald]
{\strut Definition:~#1};}}%
}%
\mdfsetup{innertopmargin=10pt,linecolor=Emerald,%
linewidth=2pt,topline=true,%
frametitleaboveskip=\dimexpr-\ht\strutbox\relax
}
\begin{mdframed}[]\relax%
\label{#1}}{\end{mdframed}}


%theorem
%\newcounter{thm}[section]\setcounter{thm}{0}
%\renewcommand{\thethm}{\arabic{section}.\arabic{thm}}
\newenvironment{thm}[1][]{%
%\refstepcounter{thm}%
\ifstrempty{#1}%
{\mdfsetup{%
frametitle={%
\tikz[baseline=(current bounding box.east),outer sep=0pt]
\node[anchor=east,rectangle,fill=blue!20]
%{\strut Theorem~\thethm};}}
{\strut Theorem};}}
}%
{\mdfsetup{%
frametitle={%
\tikz[baseline=(current bounding box.east),outer sep=0pt]
\node[anchor=east,rectangle,fill=blue!20]
%{\strut Theorem~\thethm:~#1};}}%
{\strut Theorem:~#1};}}%
}%
\mdfsetup{innertopmargin=10pt,linecolor=blue!20,%
linewidth=2pt,topline=true,%
frametitleaboveskip=\dimexpr-\ht\strutbox\relax
}
\begin{mdframed}[]\relax%
\label{#1}}{\end{mdframed}}


%lemma
\newenvironment{lem}[1][]{%
\ifstrempty{#1}%
{\mdfsetup{%
frametitle={%
\tikz[baseline=(current bounding box.east),outer sep=0pt]
\node[anchor=east,rectangle,fill=Dandelion]
{\strut Lemma};}}
}%
{\mdfsetup{%
frametitle={%
\tikz[baseline=(current bounding box.east),outer sep=0pt]
\node[anchor=east,rectangle,fill=Dandelion]
{\strut Lemma:~#1};}}%
}%
\mdfsetup{innertopmargin=10pt,linecolor=Dandelion,%
linewidth=2pt,topline=true,%
frametitleaboveskip=\dimexpr-\ht\strutbox\relax
}
\begin{mdframed}[]\relax%
\label{#1}}{\end{mdframed}}

%corollary
\newenvironment{coro}[1][]{%
\ifstrempty{#1}%
{\mdfsetup{%
frametitle={%
\tikz[baseline=(current bounding box.east),outer sep=0pt]
\node[anchor=east,rectangle,fill=CornflowerBlue]
{\strut Corollary};}}
}%
{\mdfsetup{%
frametitle={%
\tikz[baseline=(current bounding box.east),outer sep=0pt]
\node[anchor=east,rectangle,fill=CornflowerBlue]
{\strut Corollary:~#1};}}%
}%
\mdfsetup{innertopmargin=10pt,linecolor=CornflowerBlue,%
linewidth=2pt,topline=true,%
frametitleaboveskip=\dimexpr-\ht\strutbox\relax
}
\begin{mdframed}[]\relax%
\label{#1}}{\end{mdframed}}

%proof
\newenvironment{prf}[1][]{%
\ifstrempty{#1}%
{\mdfsetup{%
frametitle={%
\tikz[baseline=(current bounding box.east),outer sep=0pt]
\node[anchor=east,rectangle,fill=SpringGreen]
{\strut Proof};}}
}%
{\mdfsetup{%
frametitle={%
\tikz[baseline=(current bounding box.east),outer sep=0pt]
\node[anchor=east,rectangle,fill=SpringGreen]
{\strut Proof:~#1};}}%
}%
\mdfsetup{innertopmargin=10pt,linecolor=SpringGreen,%
linewidth=2pt,topline=true,%
frametitleaboveskip=\dimexpr-\ht\strutbox\relax
}
\begin{mdframed}[]\relax%
\label{#1}}{\qed\end{mdframed}}


\theoremstyle{definition}

\newmdtheoremenv[nobreak=true]{definition}{Definition}
\newmdtheoremenv[nobreak=true]{prop}{Proposition}
\newmdtheoremenv[nobreak=true]{theorem}{Theorem}
\newmdtheoremenv[nobreak=true]{corollary}{Corollary}
\newtheorem*{eg}{Example}
\theoremstyle{remark}
\newtheorem*{case}{Case}
\newtheorem*{notation}{Notation}
\newtheorem*{remark}{Remark}
\newtheorem*{note}{Note}
\newtheorem*{problem}{Problem}
\newtheorem*{observe}{Observe}
\newtheorem*{property}{Property}
\newtheorem*{intuition}{Intuition}


% End example and intermezzo environments with a small diamond (just like proof
% environments end with a small square)
\usepackage{etoolbox}
\AtEndEnvironment{vb}{\null\hfill$\diamond$}%
\AtEndEnvironment{intermezzo}{\null\hfill$\diamond$}%
% \AtEndEnvironment{opmerking}{\null\hfill$\diamond$}%

% Fix some spacing
% http://tex.stackexchange.com/questions/22119/how-can-i-change-the-spacing-before-theorems-with-amsthm
\makeatletter
\def\thm@space@setup{%
  \thm@preskip=\parskip \thm@postskip=0pt
}

% Fix some stuff
% %http://tex.stackexchange.com/questions/76273/multiple-pdfs-with-page-group-included-in-a-single-page-warning
\pdfsuppresswarningpagegroup=1


% My name
\author{Jaden Wang}



\begin{document}
\begin{eg}[evaluation homomorphism]
	Let $ R= \{f: \rr \to \rr\}$ with pointwise addition and multiplication. Let $a \in \rr$, define $ \phi_a: R \to \rr$ and $ \phi_a(f(x))=f(a)$. This is a ring homomorphism.
	\[
		\phi_a(f(x)+g(x))= (f+g)(a) = f(a)+g(a)=\phi_a(f(x))+\phi_a(g(x))
	\] 
By pointwise addition. Similarly,
\[
	\phi_a(f(x)g(x))= fg(a)= f(a)g(a)= \phi_a(f(x)) \phi_a(g(x))
.\] 

Suppose $ a=2$, so  $ \phi_2 : f(x) \mapsto f(2)$.
\end{eg}

\begin{defn}[]
$ \phi: R \to S$ is a homomorphism of rings. Then
\[
	\ker \phi = \{ r \in R: \phi(r)=0\} 
.\] 
and
\[
	\im \phi= \{s \in S: \phi(r)=s \text{ for some }r \} 
.\] 
\end{defn}

\begin{note}[]
Direct product of rings follows intuitively from that of groups. The projection map is again a homomorphism.
\end{note}

\begin{defn}[unit]
Let $ R$ be a ring with identity. A \allbold{unit} in $ R$ is an element with a multiplicative inverse. 
\end{defn}
\begin{eg}[]
In $ \zz_{12}$, 7 is a unit because $ 7\times 7=1 \mod 12$. The units are $ \{1,5,7,11\} $, coprimes of 12.

In $ \zz_7$, 3 is a unit. $ 3 \times 5= 1 \mod 7$. 

In $ \zz$, the units are $ \{1,-1\} $. Warning: the answer to "what are the units" is usually not $ \pm 1$. This is true for  $ \zz$.

In $ \qq$, the units are all NONZERO elements.
\end{eg}

\begin{note}[]
$ 0$ is NEVER a unit. Because  by Theorem 18.8, $ u0=0u=0 \neq 1$ by definition of multiplicative identity.
\end{note}

\begin{thm}[]
	The units, $ U(R)$ of  $ R$, form a group under multiplication.
\end{thm}
\begin{prf}
\begin{enumerate}[label=(\roman*)]
	\item closure: If $ u$ and  $ v$ are units, so is  $ uv$. The inverse of  $ uv$ is $ v^{-1}u^{-1}$.
	\item associativity: definition of $ \times _R$.
	\item identity: $ I_R$ is a unit. It is its own inverse.
	\item inverses: If  $ u$ is a unit, so is  $ u^{-1}$.
\end{enumerate}
\end{prf}
\begin{defn}[division ring]
A \allbold{division ring} is one in which every nonzero element is a unit. 
\end{defn}
\begin{defn}[fields]
	A \allbold{field} is a commutative division ring. 
\end{defn}
\begin{eg}[]
$ \qq, \rr, \cc, \zz_p$.
\end{eg}
\begin{eg}[division ring not field]
	$ \mathbb{H}$ real quaternions. They are like complex numbers but worse. Complex numbers are a vector space of dimension 2 over the reals. Quaternions are dimensional 4, $ \{a+bi+cj+dk: a,b,c,d \in \rr\} $ with basis $ \{1,i,j,k\} $, where $ i^2=j^2=k^2=-1$. This is not commutative, similar to cross-product. The inverse comes from conjugation.
\end{eg}

What is the additive order of $ 1_R$? We know distributivity laws might be involved.
 \begin{eg}[]
	 Let $ F$ be a field, and let  $ 1_F$ be the multiplicative identity. Could  $ 1_F$ have additive order 6?

	 No. Suppose  $ 1_F+1_F+1_F+1_F+1_F+1_F = 0_F \implies (1_F + 1_F)\times (1_F+1_F+1_F) = 0_F$. In a field, if $ xy=0$, then  $ x=0$ or  $ y=0$.
	  \begin{prf}
	 Suppose $ x \neq 0$, we will show that  $ y=0$.  This implies $ x$ has a multiplicative inverse,  $ x^{-1}$. Then
	 \begin{align*}
		 xy&=0\\
		 x^{-1}(xy) &= x^{-1} 0 =0\\
		 (x^{-1} x)y &= 0 \\
		 1_R y &= 0 \\
		 y&= 0 \\
	 \end{align*}
	 \end{prf}
	 Therefore, $ 1_F+1F=0$ or  $ 1_F+1_F+1_F=0$. So we found a smaller number for the order! 
\end{eg}

\begin{defn}[characteristic of a field]
	Let $ n$ be the additive order of  $ 1_F$. If  $ n$ finite,  the characteristic of $ F$ is  $ n$. If  $ n$ is infinite, the characteristic of  $ F$ is 0.
\end{defn}

\begin{thm}[]
	The characteristic of a field is either 0 or a prime.
\end{thm}
\begin{eg}[]
	Extreme example: $ \zz_2 $ is field. 
\end{eg}
\begin{eg}[zero divisors]
	In $ M_2(\rr)$. $ \begin{pmatrix} 1&0\\0&0 \end{pmatrix} \begin{pmatrix} (0&0\\0&1 \end{pmatrix}  = \begin{pmatrix} 0&0\\0&0 \end{pmatrix} $ 
	In $ \zz_{10}, 4\times 5=0$.
\end{eg}
\begin{defn}[zero divisors]
Let $ R$ be a ring and  $ x,y \in R$. If $ xy=0$ but  $ x\neq 0$ and  $ y\neq 0$, we call  $ x,y$  \allbold{zero divisors.} 
\end{defn}

\begin{defn}[integral domain]
An \allbold{integral domain} is a commutative ring with identity that has no zero divisors. 
\end{defn}
\begin{eg}[]
$ \zz$.
\end{eg}

\begin{eg}[unrelated]
	$ \mathbb{H}$. Then  $ \{1,-1,i,-i,j,-j,k,-k\} $ under $ \times $ form a group. Then the order of its elements are:
\begin{align*}
	1:&1\\
	-1:&2\\
	i:&4\\
	-i:&4\\
	j:&4\\
	-j:&4\\
	k:&4\\
	-k:&4\\
\end{align*}
But for $ D_4$, the reflections have order 2, and rotations have order $ 1,4,2,4$. Element order is a structural property. $ Q_8$ is not abelian, but every subgroup is normal.

So the complete list of groups of order 8 is: $ \zz_2 \times \zz_2 \times \zz_2, \zz_4 \times \zz_2, \zz_8, D_4, Q_8 $.

\begin{eg}[complete list of groups with order 1 to 15]
Note that prime orders only have $ \zz_{p}$. Orders of $ p^2$ only has $ \zz_{p^2}$ and $ \zz_p \times \zz_p$.
\begin{align*}
	1:& \{e\} \\
	2:&\zz_2\\
	3:&\zz_3\\
	4:& \zz_4, V_4\\
	5:& \zz_5\\
	6:& \zz_6, S_3 \simeq D_3\\
	7:& \zz_7\\
	8:& \text{ described above }\\
	9:& \zz_9, \zz_3 \times \zz_3\\
	10:& \zz_{10}, D_5\\
	11:& \zz_{11}\\
	12:& \zz_{12}, \zz_2 \times \zz_6, D_6, A_4, T\\
	13:& \zz_{13}\\
	14:& \zz_{14}, D_7\\
	15:& \zz_{15}\\
\end{align*}
\end{eg}

\end{eg}
\end{document}

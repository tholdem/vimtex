\documentclass[12pt]{article} 
%Fall 2020
% Some basic packages
\usepackage{standalone}[subpreambles=true]
\usepackage[utf8]{inputenc}
\usepackage[T1]{fontenc}
\usepackage{textcomp}
\usepackage[english]{babel}
\usepackage{url}
\usepackage{graphicx}
\usepackage{float}
\usepackage{enumitem}


\pdfminorversion=7

% Don't indent paragraphs, leave some space between them
\usepackage{parskip}

% Hide page number when page is empty
\usepackage{emptypage}
\usepackage{subcaption}
\usepackage{multicol}
\usepackage[dvipsnames]{xcolor}


% Math stuff
\usepackage{amsmath, amsfonts, mathtools, amsthm, amssymb}
% Fancy script capitals
\usepackage{mathrsfs}
\usepackage{cancel}
% Bold math
\usepackage{bm}
% Some shortcuts
\newcommand{\rr}{\ensuremath{\mathbb{R}}}
\newcommand{\zz}{\ensuremath{\mathbb{Z}}}
\newcommand{\qq}{\ensuremath{\mathbb{Q}}}
\newcommand{\nn}{\ensuremath{\mathbb{N}}}
\newcommand{\ff}{\ensuremath{\mathbb{F}}}
\newcommand{\cc}{\ensuremath{\mathbb{C}}}
\renewcommand\O{\ensuremath{\emptyset}}
\newcommand{\norm}[1]{{\left\lVert{#1}\right\rVert}}
\renewcommand{\vec}[1]{{\mathbf{#1}}}
\newcommand\allbold[1]{{\boldmath\textbf{#1}}}

% Put x \to \infty below \lim
\let\svlim\lim\def\lim{\svlim\limits}

%Make implies and impliedby shorter
\let\implies\Rightarrow
\let\impliedby\Leftarrow
\let\iff\Leftrightarrow
\let\epsilon\varepsilon

% Add \contra symbol to denote contradiction
\usepackage{stmaryrd} % for \lightning
\newcommand\contra{\scalebox{1.5}{$\lightning$}}

% \let\phi\varphi

% Command for short corrections
% Usage: 1+1=\correct{3}{2}

\definecolor{correct}{HTML}{009900}
\newcommand\correct[2]{\ensuremath{\:}{\color{red}{#1}}\ensuremath{\to }{\color{correct}{#2}}\ensuremath{\:}}
\newcommand\green[1]{{\color{correct}{#1}}}

% horizontal rule
\newcommand\hr{
    \noindent\rule[0.5ex]{\linewidth}{0.5pt}
}

% hide parts
\newcommand\hide[1]{}

% si unitx
\usepackage{siunitx}
\sisetup{locale = FR}

% Environments
\makeatother
% For box around Definition, Theorem, \ldots
\usepackage[framemethod=TikZ]{mdframed}
\mdfsetup{skipabove=1em,skipbelow=0em}

%definition
\newenvironment{defn}[1][]{%
\ifstrempty{#1}%
{\mdfsetup{%
frametitle={%
\tikz[baseline=(current bounding box.east),outer sep=0pt]
\node[anchor=east,rectangle,fill=Emerald]
{\strut Definition};}}
}%
{\mdfsetup{%
frametitle={%
\tikz[baseline=(current bounding box.east),outer sep=0pt]
\node[anchor=east,rectangle,fill=Emerald]
{\strut Definition:~#1};}}%
}%
\mdfsetup{innertopmargin=10pt,linecolor=Emerald,%
linewidth=2pt,topline=true,%
frametitleaboveskip=\dimexpr-\ht\strutbox\relax
}
\begin{mdframed}[]\relax%
\label{#1}}{\end{mdframed}}


%theorem
%\newcounter{thm}[section]\setcounter{thm}{0}
%\renewcommand{\thethm}{\arabic{section}.\arabic{thm}}
\newenvironment{thm}[1][]{%
%\refstepcounter{thm}%
\ifstrempty{#1}%
{\mdfsetup{%
frametitle={%
\tikz[baseline=(current bounding box.east),outer sep=0pt]
\node[anchor=east,rectangle,fill=blue!20]
%{\strut Theorem~\thethm};}}
{\strut Theorem};}}
}%
{\mdfsetup{%
frametitle={%
\tikz[baseline=(current bounding box.east),outer sep=0pt]
\node[anchor=east,rectangle,fill=blue!20]
%{\strut Theorem~\thethm:~#1};}}%
{\strut Theorem:~#1};}}%
}%
\mdfsetup{innertopmargin=10pt,linecolor=blue!20,%
linewidth=2pt,topline=true,%
frametitleaboveskip=\dimexpr-\ht\strutbox\relax
}
\begin{mdframed}[]\relax%
\label{#1}}{\end{mdframed}}


%lemma
\newenvironment{lem}[1][]{%
\ifstrempty{#1}%
{\mdfsetup{%
frametitle={%
\tikz[baseline=(current bounding box.east),outer sep=0pt]
\node[anchor=east,rectangle,fill=Dandelion]
{\strut Lemma};}}
}%
{\mdfsetup{%
frametitle={%
\tikz[baseline=(current bounding box.east),outer sep=0pt]
\node[anchor=east,rectangle,fill=Dandelion]
{\strut Lemma:~#1};}}%
}%
\mdfsetup{innertopmargin=10pt,linecolor=Dandelion,%
linewidth=2pt,topline=true,%
frametitleaboveskip=\dimexpr-\ht\strutbox\relax
}
\begin{mdframed}[]\relax%
\label{#1}}{\end{mdframed}}

%corollary
\newenvironment{coro}[1][]{%
\ifstrempty{#1}%
{\mdfsetup{%
frametitle={%
\tikz[baseline=(current bounding box.east),outer sep=0pt]
\node[anchor=east,rectangle,fill=CornflowerBlue]
{\strut Corollary};}}
}%
{\mdfsetup{%
frametitle={%
\tikz[baseline=(current bounding box.east),outer sep=0pt]
\node[anchor=east,rectangle,fill=CornflowerBlue]
{\strut Corollary:~#1};}}%
}%
\mdfsetup{innertopmargin=10pt,linecolor=CornflowerBlue,%
linewidth=2pt,topline=true,%
frametitleaboveskip=\dimexpr-\ht\strutbox\relax
}
\begin{mdframed}[]\relax%
\label{#1}}{\end{mdframed}}

%proof
\newenvironment{prf}[1][]{%
\ifstrempty{#1}%
{\mdfsetup{%
frametitle={%
\tikz[baseline=(current bounding box.east),outer sep=0pt]
\node[anchor=east,rectangle,fill=SpringGreen]
{\strut Proof};}}
}%
{\mdfsetup{%
frametitle={%
\tikz[baseline=(current bounding box.east),outer sep=0pt]
\node[anchor=east,rectangle,fill=SpringGreen]
{\strut Proof:~#1};}}%
}%
\mdfsetup{innertopmargin=10pt,linecolor=SpringGreen,%
linewidth=2pt,topline=true,%
frametitleaboveskip=\dimexpr-\ht\strutbox\relax
}
\begin{mdframed}[]\relax%
\label{#1}}{\qed\end{mdframed}}


\theoremstyle{definition}

\newmdtheoremenv[nobreak=true]{definition}{Definition}
\newmdtheoremenv[nobreak=true]{prop}{Proposition}
\newmdtheoremenv[nobreak=true]{theorem}{Theorem}
\newmdtheoremenv[nobreak=true]{corollary}{Corollary}
\newtheorem*{eg}{Example}
\theoremstyle{remark}
\newtheorem*{case}{Case}
\newtheorem*{notation}{Notation}
\newtheorem*{remark}{Remark}
\newtheorem*{note}{Note}
\newtheorem*{problem}{Problem}
\newtheorem*{observe}{Observe}
\newtheorem*{property}{Property}
\newtheorem*{intuition}{Intuition}


% End example and intermezzo environments with a small diamond (just like proof
% environments end with a small square)
\usepackage{etoolbox}
\AtEndEnvironment{vb}{\null\hfill$\diamond$}%
\AtEndEnvironment{intermezzo}{\null\hfill$\diamond$}%
% \AtEndEnvironment{opmerking}{\null\hfill$\diamond$}%

% Fix some spacing
% http://tex.stackexchange.com/questions/22119/how-can-i-change-the-spacing-before-theorems-with-amsthm
\makeatletter
\def\thm@space@setup{%
  \thm@preskip=\parskip \thm@postskip=0pt
}

% Fix some stuff
% %http://tex.stackexchange.com/questions/76273/multiple-pdfs-with-page-group-included-in-a-single-page-warning
\pdfsuppresswarningpagegroup=1


% My name
\author{Jaden Wang}



\begin{document}
\begin{problem}[1.25]
	\[
	\frac{1}{2} +_1 \frac{7}{8} = \frac{4}{8} +_1 \frac{7}{8} =\frac{3}{8}
	.\] 
\end{problem}
\begin{problem}[1.26]
	\[
		\frac{3\pi}{4} +_{2\pi} \frac{3\pi}{2} = \frac{3\pi}{4} +_{2\pi} \frac{6\pi}{4} = \frac{\pi}{4}
	.\] 
\end{problem}

\begin{problem}[1.31]
Given the small set, we would like to find the solution exhaustively:\\
If $x=0$,  $0+_{7} 0 = 0 \neq 3$.\\
If $x = 1$, $1 +_{7} 1 = 2 \neq 3$.\\
If $x = 2$, $2 +_{7} 2 = 4 \neq 3$.\\
If $x = 3$, $3 +_{7} 3 = 6 \neq 3$.\\
If $x = 4$, $4 +_{7} 4 = 1 \neq 3$.\\
If $x = 5$, $5 +_{7} 5 = 3$.\\
If $x = 6$, $6 +_{7} 6 = 5 \neq 3$.\\
We have covered all cases of $x \in \zz_7$, and have found that $x=5$ is the only solution.
\end{problem}
\begin{problem}[1.32]
Again using exhaustive search:\\
If $x=0$,  $0+_{7} 0 +_{7} 0  = 0 \neq 5$.\\
If $x=1$,  $1+_{7} 1 +_{7} 1  = 3 \neq 5$.\\
If $x=2$,  $2+_{7} 2 +_{7} 2  = 6 \neq 5$.\\
If $x=3$,  $3+_{7} 3 +_{7} 3  = 2 \neq 5$.\\
If $x=4$,  $4+_{7} 4 +_{7} 4  = 5$.\\
If $x=5$,  $5+_{7} 5 +_{7} 5  = 1 \neq 5$.\\
If $x=6$,  $6+_{7} 6 +_{7} 6  = 4 \neq 5$.\\
Hence, $x=4$ is the only solution.
\end{problem}
\begin{problem}[1.35]
Due to isomorphism, we know that $\zeta \times \zeta = \zeta^2$ is isomorphic to $5+_{8}5 = 2$. Repeating this process yields:
\begin{align*}
	5+_{8} 5 = 2 \qquad \zeta^2 & \leftrightarrow 2\\
	2+_{8} +5 = 7 \qquad \zeta^3 & \leftrightarrow 7\\
	7 +_{8} 5 = 4 \qquad \zeta^{4} & \leftrightarrow 4\\
	4 +_{8} 5 = 1 \qquad \zeta^{5} & \leftrightarrow 1\\
	1+_{8} 5 = 6 \qquad \zeta^{6} & \leftrightarrow 6\\
	6+_{8} 5 = 3 \qquad \zeta^{7} & \leftrightarrow 3\\
	3 +_{8} 5 = 0 \qquad \zeta^{0} &\leftrightarrow 0\\
	0+_{8} 5 = 5 \qquad  \zeta^{1} & \leftrightarrow 5\\	
\end{align*}
\end{problem}

\begin{problem}[1.37]
Because $\zeta \leftrightarrow 4$ implies that $\zeta \times \zeta \leftrightarrow 4 +_{8} 4 = 0$, and $\zeta \times  \zeta \times \zeta \leftrightarrow 4+_{8} 4+_{8} 4 = 4$. And since isomorphism requires an one-to-one mapping between $U_6$ and $\rr_6$, yet both $\zeta$ and $\zeta^3$ map to $4$, the mapping cannot be one-to-one and therefore isomorphism doesn't exist.
\end{problem}
\begin{problem}[2.1]
Table 2.26 tells us that $b*d=e$,  $c*c=b$, and 
 \begin{align*}
	 [(a*c)*e]*a &= [c*e] * a\\
	 &= a*a \\
	 &= a \\
\end{align*}
\end{problem}

\begin{problem}[2.5]
For $*$ to be commutative, we require the table to be symmetric about the diagonal. Note that answers are bolded.
\begin{table}[htpb]
	\centering
	\begin{tabular}{c||c|c|c|c}
		*&a&b&c&d\\
		\hline\hline
		a&a&b&c&\allbold{d}\\ 
	        \hline
		b&b&d&\allbold{a}&c\\
		\hline
		c&c&a&d&b\\
		\hline
		d&d&\allbold{c}&\allbold{b}&a  
	\end{tabular}
\end{table}
\end{problem}
\begin{problem}[2.6]
The missing entries represents $d*a, d*b, d*c,$ and $d*d$. Since we have the complete information of how the operation is defined among pairs consisting of the other three elements, we can replace  $d$ with the other elements. Notice that there is only one such pair that engendered  $d$:  $c*b=d$. Now let's consider each entry and apply associativity:\\
\[
	d*a=(c*b)*a=c*(b*a)=c*b=d
.\] 
\[
	d*b=(c*b)*b=c*(b*b)=c*a=c
.\] 
\[
	d*c=(c*b)*c=c*(b*c)=c*c=c
.\] 
\[
	d*d= (c*b)*d=c*(b*d)=c*d=d
.\] 
Hence, the missing entires are
\begin{table}[htpb]
	\centering
	\begin{tabular}{c||c|c|c|c}
		\hline
		d&d&c&c&d
	\end{tabular}
\end{table}

\end{problem}
\begin{problem}[2.7]
~\begin{enumerate}[label=\arabic*)]
	\item Commutativity:
		Let $a=1$,$b=2$, it is easy to see that $a*b=1-2\neq 2-1=b*a$. Hence $*$ is not commutative.
	\item Associativity:
		Let  $a=1,b=2,c=3$. Since  $(a*b)*c=(a-b)-c=(1-2)-3=-4$ and  $a*(b*c)=a-(b-c)=1-(2-3)=2$, clearly  $(a*b)*c\neq a*(b*c)$ so $*$ is not associative.
\end{enumerate}
\end{problem}
\begin{problem}[2.8]
~\begin{enumerate}[label=\arabic*)]
	\item Commutativity: Given $a,b \in \qq$, we have $a*b=ab+1=ba+1=b*a$ since scalar multiplication is commutative. Hence $*$ is commutative.
	\item  Associativity: Let $a=1,b=\frac{1}{2},c=\frac{1}{3}$, $(a*b)*c=\left( 1 \times \frac{1}{2}+ 1 \right) \times \frac{1}{3} + 1 = \frac{3}{2}$ and $a*(b*c)=1\times \left( \frac{1}{2} \times \frac{1}{3}+ 1 \right) + 1 = \frac{13}{6}$. Hence $(a*b)*c \neq a*(b*c)$ and $*$ is not associative.
\end{enumerate}
\end{problem}

\begin{problem}[2.24]
~\begin{enumerate}[label=\alph*)]
	\item True. Consider $S_0=\{a\}$. Since $*$ is a binary operation on any arbitrary set, it is defined on  $S_0$ as well. It follows that  $a*a \in S_0$. Since there is only one element in $S_0$, this forces  $a*a=a$. 
	\item True. Since  $*$ is a commutative binary operation on $S$, given  $a,b,c \in S$, we know that $b*c \in S$, and thus $a*(b*c)=(b*c)*a$ by commutativity.
	\item False. Consider $S$ as the set of all  $3 \times 3$ permutation matrices and $*$ is the matrix multiplication, which we know is associative but not commutative. Let  $a=\begin{pmatrix} 0 & 1 & 0\\ 1 & 0 & 0\\ 0 & 0 & 1\\ \end{pmatrix} $, $b=\begin{pmatrix} 0&0&1\\0&1&0\\1&0&0 \end{pmatrix} $, and $c=\begin{pmatrix} 1&0&0\\0&0&1\\0&1&0 \end{pmatrix} $. LHS yields:
		\[
			a*(b*c) = \begin{pmatrix} 0 & 1 & 0\\ 1 & 0 & 0\\ 0 & 0 & 1\\ \end{pmatrix} \begin{pmatrix} 0&1&0\\0&0&1\\1&0&0 \end{pmatrix} = \begin{pmatrix} 0&0&1\\0&1&0\\1&0&0 \end{pmatrix}  
		.\]
	However, the RHS yields:
	\[
		(b*c)*a=  \begin{pmatrix} 0&1&0\\0&0&1\\1&0&0 \end{pmatrix} \begin{pmatrix} 0 & 1 & 0\\ 1 & 0 & 0\\ 0 & 0 & 1\\ \end{pmatrix} = \begin{pmatrix} 1&0&0\\0&0&1\\0&1&0 \end{pmatrix} 
	.\]
	LHS and RHS do not equal so the statement is false.
	\item False. The more abstract math gets the less important numbers are\ldots
	\item False. It should be for all $a,b \in S$.
	\item True. It is easy to see that given $S_0=\{a\} $, $a*a=a*a=a$, and  $(a*a)*a=a*a=a=a*(a*a)$.
	\item True. Because at least one includes exactly one.
	\item True. Because at most one includes exactly one.
	\item True. Because binary operation is a function and can only have one output.
	\item False. As above.
\end{enumerate}
\end{problem}
\end{document}


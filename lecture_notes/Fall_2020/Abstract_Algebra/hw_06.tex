\documentclass[12pt]{article}
%Fall 2020
% Some basic packages
\usepackage{standalone}[subpreambles=true]
\usepackage[utf8]{inputenc}
\usepackage[T1]{fontenc}
\usepackage{textcomp}
\usepackage[english]{babel}
\usepackage{url}
\usepackage{graphicx}
\usepackage{float}
\usepackage{enumitem}


\pdfminorversion=7

% Don't indent paragraphs, leave some space between them
\usepackage{parskip}

% Hide page number when page is empty
\usepackage{emptypage}
\usepackage{subcaption}
\usepackage{multicol}
\usepackage[dvipsnames]{xcolor}


% Math stuff
\usepackage{amsmath, amsfonts, mathtools, amsthm, amssymb}
% Fancy script capitals
\usepackage{mathrsfs}
\usepackage{cancel}
% Bold math
\usepackage{bm}
% Some shortcuts
\newcommand{\rr}{\ensuremath{\mathbb{R}}}
\newcommand{\zz}{\ensuremath{\mathbb{Z}}}
\newcommand{\qq}{\ensuremath{\mathbb{Q}}}
\newcommand{\nn}{\ensuremath{\mathbb{N}}}
\newcommand{\ff}{\ensuremath{\mathbb{F}}}
\newcommand{\cc}{\ensuremath{\mathbb{C}}}
\renewcommand\O{\ensuremath{\emptyset}}
\newcommand{\norm}[1]{{\left\lVert{#1}\right\rVert}}
\renewcommand{\vec}[1]{{\mathbf{#1}}}
\newcommand\allbold[1]{{\boldmath\textbf{#1}}}

% Put x \to \infty below \lim
\let\svlim\lim\def\lim{\svlim\limits}

%Make implies and impliedby shorter
\let\implies\Rightarrow
\let\impliedby\Leftarrow
\let\iff\Leftrightarrow
\let\epsilon\varepsilon

% Add \contra symbol to denote contradiction
\usepackage{stmaryrd} % for \lightning
\newcommand\contra{\scalebox{1.5}{$\lightning$}}

% \let\phi\varphi

% Command for short corrections
% Usage: 1+1=\correct{3}{2}

\definecolor{correct}{HTML}{009900}
\newcommand\correct[2]{\ensuremath{\:}{\color{red}{#1}}\ensuremath{\to }{\color{correct}{#2}}\ensuremath{\:}}
\newcommand\green[1]{{\color{correct}{#1}}}

% horizontal rule
\newcommand\hr{
    \noindent\rule[0.5ex]{\linewidth}{0.5pt}
}

% hide parts
\newcommand\hide[1]{}

% si unitx
\usepackage{siunitx}
\sisetup{locale = FR}

% Environments
\makeatother
% For box around Definition, Theorem, \ldots
\usepackage[framemethod=TikZ]{mdframed}
\mdfsetup{skipabove=1em,skipbelow=0em}

%definition
\newenvironment{defn}[1][]{%
\ifstrempty{#1}%
{\mdfsetup{%
frametitle={%
\tikz[baseline=(current bounding box.east),outer sep=0pt]
\node[anchor=east,rectangle,fill=Emerald]
{\strut Definition};}}
}%
{\mdfsetup{%
frametitle={%
\tikz[baseline=(current bounding box.east),outer sep=0pt]
\node[anchor=east,rectangle,fill=Emerald]
{\strut Definition:~#1};}}%
}%
\mdfsetup{innertopmargin=10pt,linecolor=Emerald,%
linewidth=2pt,topline=true,%
frametitleaboveskip=\dimexpr-\ht\strutbox\relax
}
\begin{mdframed}[]\relax%
\label{#1}}{\end{mdframed}}


%theorem
%\newcounter{thm}[section]\setcounter{thm}{0}
%\renewcommand{\thethm}{\arabic{section}.\arabic{thm}}
\newenvironment{thm}[1][]{%
%\refstepcounter{thm}%
\ifstrempty{#1}%
{\mdfsetup{%
frametitle={%
\tikz[baseline=(current bounding box.east),outer sep=0pt]
\node[anchor=east,rectangle,fill=blue!20]
%{\strut Theorem~\thethm};}}
{\strut Theorem};}}
}%
{\mdfsetup{%
frametitle={%
\tikz[baseline=(current bounding box.east),outer sep=0pt]
\node[anchor=east,rectangle,fill=blue!20]
%{\strut Theorem~\thethm:~#1};}}%
{\strut Theorem:~#1};}}%
}%
\mdfsetup{innertopmargin=10pt,linecolor=blue!20,%
linewidth=2pt,topline=true,%
frametitleaboveskip=\dimexpr-\ht\strutbox\relax
}
\begin{mdframed}[]\relax%
\label{#1}}{\end{mdframed}}


%lemma
\newenvironment{lem}[1][]{%
\ifstrempty{#1}%
{\mdfsetup{%
frametitle={%
\tikz[baseline=(current bounding box.east),outer sep=0pt]
\node[anchor=east,rectangle,fill=Dandelion]
{\strut Lemma};}}
}%
{\mdfsetup{%
frametitle={%
\tikz[baseline=(current bounding box.east),outer sep=0pt]
\node[anchor=east,rectangle,fill=Dandelion]
{\strut Lemma:~#1};}}%
}%
\mdfsetup{innertopmargin=10pt,linecolor=Dandelion,%
linewidth=2pt,topline=true,%
frametitleaboveskip=\dimexpr-\ht\strutbox\relax
}
\begin{mdframed}[]\relax%
\label{#1}}{\end{mdframed}}

%corollary
\newenvironment{coro}[1][]{%
\ifstrempty{#1}%
{\mdfsetup{%
frametitle={%
\tikz[baseline=(current bounding box.east),outer sep=0pt]
\node[anchor=east,rectangle,fill=CornflowerBlue]
{\strut Corollary};}}
}%
{\mdfsetup{%
frametitle={%
\tikz[baseline=(current bounding box.east),outer sep=0pt]
\node[anchor=east,rectangle,fill=CornflowerBlue]
{\strut Corollary:~#1};}}%
}%
\mdfsetup{innertopmargin=10pt,linecolor=CornflowerBlue,%
linewidth=2pt,topline=true,%
frametitleaboveskip=\dimexpr-\ht\strutbox\relax
}
\begin{mdframed}[]\relax%
\label{#1}}{\end{mdframed}}

%proof
\newenvironment{prf}[1][]{%
\ifstrempty{#1}%
{\mdfsetup{%
frametitle={%
\tikz[baseline=(current bounding box.east),outer sep=0pt]
\node[anchor=east,rectangle,fill=SpringGreen]
{\strut Proof};}}
}%
{\mdfsetup{%
frametitle={%
\tikz[baseline=(current bounding box.east),outer sep=0pt]
\node[anchor=east,rectangle,fill=SpringGreen]
{\strut Proof:~#1};}}%
}%
\mdfsetup{innertopmargin=10pt,linecolor=SpringGreen,%
linewidth=2pt,topline=true,%
frametitleaboveskip=\dimexpr-\ht\strutbox\relax
}
\begin{mdframed}[]\relax%
\label{#1}}{\qed\end{mdframed}}


\theoremstyle{definition}

\newmdtheoremenv[nobreak=true]{definition}{Definition}
\newmdtheoremenv[nobreak=true]{prop}{Proposition}
\newmdtheoremenv[nobreak=true]{theorem}{Theorem}
\newmdtheoremenv[nobreak=true]{corollary}{Corollary}
\newtheorem*{eg}{Example}
\theoremstyle{remark}
\newtheorem*{case}{Case}
\newtheorem*{notation}{Notation}
\newtheorem*{remark}{Remark}
\newtheorem*{note}{Note}
\newtheorem*{problem}{Problem}
\newtheorem*{observe}{Observe}
\newtheorem*{property}{Property}
\newtheorem*{intuition}{Intuition}


% End example and intermezzo environments with a small diamond (just like proof
% environments end with a small square)
\usepackage{etoolbox}
\AtEndEnvironment{vb}{\null\hfill$\diamond$}%
\AtEndEnvironment{intermezzo}{\null\hfill$\diamond$}%
% \AtEndEnvironment{opmerking}{\null\hfill$\diamond$}%

% Fix some spacing
% http://tex.stackexchange.com/questions/22119/how-can-i-change-the-spacing-before-theorems-with-amsthm
\makeatletter
\def\thm@space@setup{%
  \thm@preskip=\parskip \thm@postskip=0pt
}

% Fix some stuff
% %http://tex.stackexchange.com/questions/76273/multiple-pdfs-with-page-group-included-in-a-single-page-warning
\pdfsuppresswarningpagegroup=1


% My name
\author{Jaden Wang}



\begin{document}
\centerline {\textsf{\textbf{\LARGE{Homework 6}}}}
\centerline {Jaden Wang}
\vspace{.15in}

\begin{problem}[10.1]
$ 4 \zz = \{\ldots,-8,-4,0,4,8,\ldots\} $.
The left cosets of $ 4 \zz$ in $ \zz$ are:
\begin{enumerate}[label=\arabic*)]
	\item itself.
	\item Notice $ 1$ is missing. Let $ a=1$, then  $ 1+4 \zz = \{\ldots,-7,-3,1,5,9,\ldots\} $.
	\item 2 is missing. Let $ a=2$, then  $ 2+ 4 \zz =\{\ldots-6,-2,2,6,10,\ldots\} $.
	\item 3 is missing. Let $ a=3$, then  $ 3+4 \zz = \{\ldots,-5,-1,3,7,11,\ldots\} $
\end{enumerate}
This exhausts all elements in $ \zz$. And since $ \zz$ is abelian, it follows that the left and right cosets are the same. That's all the cosets.
\end{problem}

\begin{problem}[10.2]
~\begin{enumerate}[label=\arabic*)]
	\item $ 4 \zz$ itself.
	\item $ 2+ 4 \zz = \{\ldots,-6,-2,2,6,10,\ldots\}$.
\end{enumerate}
Clearly this exhausts all elements of $ 2 \zz &= \{\ldots,-4,-2,0,2,4,\ldots\} $. 
\end{problem}

\begin{problem}[10.6]
By Lagrange, it should have $ \frac{8}{2}=4$ left cosets.
\begin{enumerate}[label=\arabic*)]
	\item itself $H= \{\rho_0, \mu_2\} $ 
	\item notice $ \rho_1$ is missing. $ \rho_1 H = \{\rho_1, \delta_2\} $ 
	\item notice $ \rho_2$ is missing. $ \rho_2 H = \{\rho_2,\mu_1\} $
	\item notice $ \rho_3$ is missing. $ \rho_3 H = \{\rho_3, \delta_1\} $
\end{enumerate}
\end{problem}

\begin{problem}[10.7]
It should have 4 right cosets. We can apply the inverse of the elements we multiplied above on the left on the right.
\begin{enumerate}[label=\arabic*)]
	\item itself $ H = \{\rho_0, \mu_2\} $ 
	\item $ H \rho_3^{-1} = H \rho_1 = \{\rho_1, \delta_1\} $ 
	\item $ H \rho_2^{-1} = H \rho_2 = \{\rho_2,\mu_1\} $ 
	\item $H \rho_1^{-1} =H \rho_3= \{\rho_3, \delta_2\} $
\end{enumerate}
No they are not the same, because $ D_4$ is not abelian.
\end{problem}

\begin{problem}[10.12]
$ |\langle 3 \rangle| = \frac{n}{ \gcd ( 3,24) }=24 \div 3 = 8$.\\
$\{ \zz_{24}: |\langle 3 \rangle|\} = | \zz_{24}| \div |\langle 3 \rangle| = 24 \div 8 = 3$.
\end{problem}

\begin{problem}[10.15]
\[
	\sigma = \text{ (1 2 5 4)(2 3) = (1 2 3 5 4) }
.\] 
	Since $ \sigma$ is a 5-cycle, its order $ |\langle \sigma \rangle| = 5$. \\
$ \{S_5 : |\langle \sigma \rangle|\} =\frac{|S_5|}{|\langle \sigma \rangle| } = 5!/5 = 24$ by Lagrange.
\end{problem}

\begin{problem}[10.19]
~\begin{enumerate}[label=\alph*)]
	\item True. Itself.
	\item True. Lagrange.
	\item True. Cyclic implies abelian by Corollary 10.11.
	\item False. $ \{0\} $ is a subgroup and thus a left coset of $ \zz$, yet it is finite.
	\item True. $ eH = H$.
	\item False. $ 3 \zz$ in $ \zz$ is infinite.
	\item True. $ |A_n|=n!/2, |S_n| = n!$.
	\item True!
	\item False. $ |A_4|=12$, clearly $ 6$ divides 12 but we know that  $ A_4$ has no subgroup of order 6.
	\item True. $ |\langle a \rangle| = \frac{n}{ \gcd ( n,a) }$.
\end{enumerate}
\end{problem}

\begin{problem}[11.1]
\begin{align*}
	1(0,0)=(0,0) \implies |\langle (0,0) \rangle|&=1\\
	|\langle (0,1) \rangle|=\frac{4}{ \gcd ( 1,4) } &= 4\\
	|\langle (1,0) \rangle| = \frac{2}{ \gcd ( 1,2) } &= 2\\
	|\langle (1,1) \rangle|= \lcm \left( \frac{2}{ \gcd ( 1,2) },  \frac{4}{ \gcd ( 1,4) } \right)  &=4\\
	|\langle (0,2) \rangle| =\frac{4}{ \gcd ( 2,4) } &=2\\
	|\langle (1,2) \rangle| = \lcm \left( \frac{2}{ \gcd ( 1,2) }, \frac{4}{ \gcd ( 2,4) } \right)   &= 2\\
	|\langle (0,3) \rangle|=\frac{4}{ \gcd ( 3,4) } &=4\\
	|\langle (1,3) \rangle|= \lcm \left( \frac{2}{ \gcd ( 1,2) }, \frac{4}{ \gcd ( 3,4) } \right)   &=4 
\end{align*}
No, by Theorem 11.5 since $ 2$ and  $ 4$ are not relatively prime.

\end{problem}
\begin{problem}[11.2]
\begin{align*}
	1(0,0)=(0,0) \implies |\langle (0,0) \rangle|&=1\\
	|\langle (0,1) \rangle|=\frac{4}{ \gcd ( 1,4) } &= 4\\
	|\langle (1,0) \rangle| = \frac{3}{ \gcd ( 1,3) } &= 3\\
	|\langle (1,1) \rangle|= \lcm \left( \frac{3}{ \gcd ( 1,3) },  \frac{4}{ \gcd ( 1,4) } \right)  &=12\\
	|\langle (0,2) \rangle| =\frac{4}{ \gcd ( 2,4) } &=2\\
	|\langle (1,2) \rangle| = \lcm \left( \frac{3}{ \gcd ( 1,3) }, \frac{4}{ \gcd ( 2,4) } \right)   &= 6\\
	|\langle (0,3) \rangle|=\frac{4}{ \gcd ( 3,4) } &=4\\
	|\langle (1,3) \rangle|= \lcm \left( \frac{3}{ \gcd ( 1,3) }, \frac{4}{ \gcd ( 3,4) } \right)   &=12\\
	|\langle (2,2) \rangle| = \lcm \left( \frac{3}{ \gcd ( 2,3) }, \frac{4}{ \gcd ( 2,4) } \right)  &= 6 \\
	|\langle (2,3) \rangle| = \lcm \left( \frac{3}{ \gcd ( 2,3) }, \frac{4}{ \gcd ( 3,4) } \right) &= 12  
\end{align*}
Yes, because 3 and 4 are relatively prime, so by Theorem 11.5 it is cyclic.
\end{problem}
\begin{problem}[11.14]
~\begin{enumerate}[label=\alph*)]
	\item $ |\langle 18 \rangle|= \frac{24}{ \gcd ( 18,24) } = \frac{24}{6} =4$.
	\item $ \lcm ( 3,4)=12 $.
	\item $ \lcm \left( \frac{12}{ \gcd ( 4,12) }, \frac{8}{ \gcd ( 8,2) } \right) = \lcm ( 3,4) =12   $.
	\item $ V_4 \simeq \zz_2 \times \zz_2$.
	\item Choose 1 element from each group, we can have $ 2 \times 1 \times 4 = 8$ elements.
\end{enumerate}
\end{problem}
 \begin{problem}[11.20]
 \begin{align*}
	 \zz_4 \times \zz_{18} \times \zz_{15} &\simeq \zz_{2^2} \times \zz_{3^2} \times \zz_2 \times  \zz_3 \times \zz_5\\
	 \zz_3 \times \zz_{36} \times \zz_{10} &\simeq \zz_3 \times \zz_{3^2} \times \zz_{2^2} \times \zz_2 \times \zz_5
 \end{align*}
 Notice that the RHSs are just rearrangement of each other, hence by the fundamental theorem of FGG, they are isomorphic to each other.
 \end{problem}

\begin{problem}[11.23]
Since $ 32= 2^{5}$, let's consider the partitions of 5:
\begin{align*}
	5 &: \zz_{32}\\
	4+1&: \zz_{16} \times \zz_2\\
	3+2&: \zz_{8} \times \zz_4\\
	3+1+1&: \zz_8 \times \zz_2 \times \zz_2\\
	2+2+1&: \zz_4 \times \zz_4 \times \zz_2\\
	2+1+1+1 &: \zz_4 \times \zz_2 \times \zz_2 \times \zz_2\\
	1+1+1+1+1 &: \zz_2 \times \zz_2 \times \zz_2 \times \zz_2 \times \zz_2
\end{align*}
That's all the abelian groups of order 32, up to isomorphism.
\end{problem}

\begin{problem}[11.24]
	Since $ 720=2^{4} 3^2 5$, let's first consider the partition of $ 2^{4}=16$:
	\begin{align*}
		4&: \zz_{16}\\
		3+1&: \zz_8 \times \zz_2\\
		2+2&: \zz_4 \times \zz_4\\
		2+1+1&: \zz_4 \times \zz_2 \times \zz_2\\
		1+1+1+1&: \zz_2 \times \zz_2 \times \zz_2 \times \zz_2
	\end{align*}
\end{problem}
Then let's consider the partition of $ 3^2=9$:
\begin{align*}
	2&: \zz_9\\
	1+1&: \zz_3 \times \zz_3 
\end{align*}
Finally for$ 5^{1} = 5$, the partition of 1 is just 1, hence it only yields $ \zz_5$. Now choosing one from each group, we list all $ 5\times 2\times 1 =10 $ possible abelian groups of order 720 below:
\begin{align*}
	&\zz_{16} \times \zz_9 \times \zz_5\\
	& \zz_8 \times \zz_2 \times \zz_9 \times \zz_5\\
	& \zz_4 \times \zz_4 \times \zz_9 \times \zz_5\\
	& \zz_4 \times \zz_2 \times \zz_2 \times \zz_9 \times \zz_5\\
	& \zz_2 \times \zz_2 \times \zz_2 \times \zz_2 \times \zz_9 \times \zz_5\\
	& \zz_{16} \times \zz_3 \times \zz_3 \times \zz_5\\
	& \zz_8 \times \zz_2 \times \zz_3 \times \zz_3 \times \zz_5\\
	& \zz_4 \times \zz_4 \times \zz_3 \times \zz_3 \times \zz_5\\
	& \zz_4 \times \zz_2 \times \zz_2 \times \zz_3 \times \zz_3 \times \zz_5\\
	& \zz_2 \times \zz_2 \times \zz_2 \times \zz_2 \times \zz_3 \times \zz_3 \times \zz_5\\
\end{align*}
\begin{problem}[11.46]
	WLOG, let's consider the direct product of two abelian groups $ (G,*)$ and  $ (H,*)$,  $ G\times H$. Given two elements from this product, $ (g_1, h_1)$ and $ (g_2,h_2)$, we want to show that their product under componentwise $ *$ commutes. Since  $ G, H$ are abelian,  $ *$ certainly commutes. Then
	 \begin{align*}
		 (g_1, h_1) * (g_2, h_2) &= (g_1*g_2, h_1*h_2) \\
					 &= (g_2* g_1, h_2*h_1) \\
					 &= (g_2,h_2) * (g_1,h_1) \\
	\end{align*}
	This shows that componentwise $ *$ commutes, and it follows that  $ G\times H$ is abelian.
\end{problem}
\end{document}

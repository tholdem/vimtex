\documentclass[class=article,crop=false]{standalone} 
%Fall 2020
% Some basic packages
\usepackage{standalone}[subpreambles=true]
\usepackage[utf8]{inputenc}
\usepackage[T1]{fontenc}
\usepackage{textcomp}
\usepackage[english]{babel}
\usepackage{url}
\usepackage{graphicx}
\usepackage{float}
\usepackage{enumitem}


\pdfminorversion=7

% Don't indent paragraphs, leave some space between them
\usepackage{parskip}

% Hide page number when page is empty
\usepackage{emptypage}
\usepackage{subcaption}
\usepackage{multicol}
\usepackage[dvipsnames]{xcolor}


% Math stuff
\usepackage{amsmath, amsfonts, mathtools, amsthm, amssymb}
% Fancy script capitals
\usepackage{mathrsfs}
\usepackage{cancel}
% Bold math
\usepackage{bm}
% Some shortcuts
\newcommand{\rr}{\ensuremath{\mathbb{R}}}
\newcommand{\zz}{\ensuremath{\mathbb{Z}}}
\newcommand{\qq}{\ensuremath{\mathbb{Q}}}
\newcommand{\nn}{\ensuremath{\mathbb{N}}}
\newcommand{\ff}{\ensuremath{\mathbb{F}}}
\newcommand{\cc}{\ensuremath{\mathbb{C}}}
\renewcommand\O{\ensuremath{\emptyset}}
\newcommand{\norm}[1]{{\left\lVert{#1}\right\rVert}}
\renewcommand{\vec}[1]{{\mathbf{#1}}}
\newcommand\allbold[1]{{\boldmath\textbf{#1}}}

% Put x \to \infty below \lim
\let\svlim\lim\def\lim{\svlim\limits}

%Make implies and impliedby shorter
\let\implies\Rightarrow
\let\impliedby\Leftarrow
\let\iff\Leftrightarrow
\let\epsilon\varepsilon

% Add \contra symbol to denote contradiction
\usepackage{stmaryrd} % for \lightning
\newcommand\contra{\scalebox{1.5}{$\lightning$}}

% \let\phi\varphi

% Command for short corrections
% Usage: 1+1=\correct{3}{2}

\definecolor{correct}{HTML}{009900}
\newcommand\correct[2]{\ensuremath{\:}{\color{red}{#1}}\ensuremath{\to }{\color{correct}{#2}}\ensuremath{\:}}
\newcommand\green[1]{{\color{correct}{#1}}}

% horizontal rule
\newcommand\hr{
    \noindent\rule[0.5ex]{\linewidth}{0.5pt}
}

% hide parts
\newcommand\hide[1]{}

% si unitx
\usepackage{siunitx}
\sisetup{locale = FR}

% Environments
\makeatother
% For box around Definition, Theorem, \ldots
\usepackage[framemethod=TikZ]{mdframed}
\mdfsetup{skipabove=1em,skipbelow=0em}

%definition
\newenvironment{defn}[1][]{%
\ifstrempty{#1}%
{\mdfsetup{%
frametitle={%
\tikz[baseline=(current bounding box.east),outer sep=0pt]
\node[anchor=east,rectangle,fill=Emerald]
{\strut Definition};}}
}%
{\mdfsetup{%
frametitle={%
\tikz[baseline=(current bounding box.east),outer sep=0pt]
\node[anchor=east,rectangle,fill=Emerald]
{\strut Definition:~#1};}}%
}%
\mdfsetup{innertopmargin=10pt,linecolor=Emerald,%
linewidth=2pt,topline=true,%
frametitleaboveskip=\dimexpr-\ht\strutbox\relax
}
\begin{mdframed}[]\relax%
\label{#1}}{\end{mdframed}}


%theorem
%\newcounter{thm}[section]\setcounter{thm}{0}
%\renewcommand{\thethm}{\arabic{section}.\arabic{thm}}
\newenvironment{thm}[1][]{%
%\refstepcounter{thm}%
\ifstrempty{#1}%
{\mdfsetup{%
frametitle={%
\tikz[baseline=(current bounding box.east),outer sep=0pt]
\node[anchor=east,rectangle,fill=blue!20]
%{\strut Theorem~\thethm};}}
{\strut Theorem};}}
}%
{\mdfsetup{%
frametitle={%
\tikz[baseline=(current bounding box.east),outer sep=0pt]
\node[anchor=east,rectangle,fill=blue!20]
%{\strut Theorem~\thethm:~#1};}}%
{\strut Theorem:~#1};}}%
}%
\mdfsetup{innertopmargin=10pt,linecolor=blue!20,%
linewidth=2pt,topline=true,%
frametitleaboveskip=\dimexpr-\ht\strutbox\relax
}
\begin{mdframed}[]\relax%
\label{#1}}{\end{mdframed}}


%lemma
\newenvironment{lem}[1][]{%
\ifstrempty{#1}%
{\mdfsetup{%
frametitle={%
\tikz[baseline=(current bounding box.east),outer sep=0pt]
\node[anchor=east,rectangle,fill=Dandelion]
{\strut Lemma};}}
}%
{\mdfsetup{%
frametitle={%
\tikz[baseline=(current bounding box.east),outer sep=0pt]
\node[anchor=east,rectangle,fill=Dandelion]
{\strut Lemma:~#1};}}%
}%
\mdfsetup{innertopmargin=10pt,linecolor=Dandelion,%
linewidth=2pt,topline=true,%
frametitleaboveskip=\dimexpr-\ht\strutbox\relax
}
\begin{mdframed}[]\relax%
\label{#1}}{\end{mdframed}}

%corollary
\newenvironment{coro}[1][]{%
\ifstrempty{#1}%
{\mdfsetup{%
frametitle={%
\tikz[baseline=(current bounding box.east),outer sep=0pt]
\node[anchor=east,rectangle,fill=CornflowerBlue]
{\strut Corollary};}}
}%
{\mdfsetup{%
frametitle={%
\tikz[baseline=(current bounding box.east),outer sep=0pt]
\node[anchor=east,rectangle,fill=CornflowerBlue]
{\strut Corollary:~#1};}}%
}%
\mdfsetup{innertopmargin=10pt,linecolor=CornflowerBlue,%
linewidth=2pt,topline=true,%
frametitleaboveskip=\dimexpr-\ht\strutbox\relax
}
\begin{mdframed}[]\relax%
\label{#1}}{\end{mdframed}}

%proof
\newenvironment{prf}[1][]{%
\ifstrempty{#1}%
{\mdfsetup{%
frametitle={%
\tikz[baseline=(current bounding box.east),outer sep=0pt]
\node[anchor=east,rectangle,fill=SpringGreen]
{\strut Proof};}}
}%
{\mdfsetup{%
frametitle={%
\tikz[baseline=(current bounding box.east),outer sep=0pt]
\node[anchor=east,rectangle,fill=SpringGreen]
{\strut Proof:~#1};}}%
}%
\mdfsetup{innertopmargin=10pt,linecolor=SpringGreen,%
linewidth=2pt,topline=true,%
frametitleaboveskip=\dimexpr-\ht\strutbox\relax
}
\begin{mdframed}[]\relax%
\label{#1}}{\qed\end{mdframed}}


\theoremstyle{definition}

\newmdtheoremenv[nobreak=true]{definition}{Definition}
\newmdtheoremenv[nobreak=true]{prop}{Proposition}
\newmdtheoremenv[nobreak=true]{theorem}{Theorem}
\newmdtheoremenv[nobreak=true]{corollary}{Corollary}
\newtheorem*{eg}{Example}
\theoremstyle{remark}
\newtheorem*{case}{Case}
\newtheorem*{notation}{Notation}
\newtheorem*{remark}{Remark}
\newtheorem*{note}{Note}
\newtheorem*{problem}{Problem}
\newtheorem*{observe}{Observe}
\newtheorem*{property}{Property}
\newtheorem*{intuition}{Intuition}


% End example and intermezzo environments with a small diamond (just like proof
% environments end with a small square)
\usepackage{etoolbox}
\AtEndEnvironment{vb}{\null\hfill$\diamond$}%
\AtEndEnvironment{intermezzo}{\null\hfill$\diamond$}%
% \AtEndEnvironment{opmerking}{\null\hfill$\diamond$}%

% Fix some spacing
% http://tex.stackexchange.com/questions/22119/how-can-i-change-the-spacing-before-theorems-with-amsthm
\makeatletter
\def\thm@space@setup{%
  \thm@preskip=\parskip \thm@postskip=0pt
}

% Fix some stuff
% %http://tex.stackexchange.com/questions/76273/multiple-pdfs-with-page-group-included-in-a-single-page-warning
\pdfsuppresswarningpagegroup=1


% My name
\author{Jaden Wang}



\begin{document}

We can show one binary structure is a group if it is isomorphic to another group. Because the required properties for groups are all structural properties.

\section{Subgroups}
\begin{defn}[subgroup]
	Let $(G,*)$ be a group. Let  $H \subseteq G$. We call $H$ a subgroup of  $G$ if  $H$ is a group under the same operation.
\end{defn}

\begin{note}[]
Subspace in linear algebra is an example of subgroup.
\end{note}
\begin{eg}[]
	~\begin{itemize}
		\item $\zz$ is a subgroup of $\qq$ under addition.
	\item $U_{28}$ is a subgroup of $U$, where  $U$ is the unit circle under $\times $.
	\item $\zz_2$ is the integers mod 2 under $+_2$ is NOT a subgroup of $\zz$ because they don't have the same operation.
	\item $\{1,2,3,\ldots\} $ under addition is NOT a subgroup of the $\zz$ because there is no identity.
	\item $\{0,1,2,\ldots\}$ under $+ $ is NOT a subgroup of $\zz$ because $1$ doesn't have an inverse.
	\end{itemize}
\end{eg}

\begin{thm}[]
	Let $(G,*)$ be a group and let  $H \subseteq G$. Then $H\leq G$ if 
\begin{itemize}
	\item $e \in H$
	\item if $x \in H$ then $x^{-1} \in H$ 
	\item if $x,y \in H$, then $x*y \in H$
\end{itemize}
\end{thm}

\begin{note}[]
	Associativity is implied because it's the same operation. The first condition ensures that $H \neq \O$.
\end{note}

\begin{coro}[]
	Any group $G$ has the following as subgroups:
	 \begin{itemize}
		\item $G$ itself
		\item $\{e\} $
	\end{itemize}
\end{coro}

\begin{eg}[]
Find the subgroups of $V_4$. See iPad.

\begin{itemize}
	\item 4 elements: $\{3,a,b,c\} $ 
	\item 1 element: $\{e\} $ 
	\item 2 elements: $\{e,a\},\{e,b\} ,\{e,c\}  $
	\item 3 elements: nope
\end{itemize}
There are a total of 5 subgroups.
\begin{note}[]
In $V_4$, the smallest subgroup containing  $a$ is $\{e,a\} $. Likewise for other non-identity elements. The smallest subgroup for  $e$ is  $\{e\} $.
\end{note}
\end{eg}

\begin{eg}[]
Find the subgroups of $\zz_4$. See iPad.
\begin{itemize}
	\item $4$ elements:  $\{0,1,2,3\} $ 
	\item 1 elements: $0$ 
\item  $2$ elements: only $\{0,2\} $ works
\item 3 elements: nope
\end{itemize}
There are only 3 subgroups.
\begin{note}[]
In $\zz_4$ the smallest subgroup containing 2 is $\{0,2\} $, the smallest for 0 is $\{0\} $, the smallest for 1 or 3 is  $\{0,1,2,3\} $. This is a good thing.
\end{note}
\end{eg}

A group with this property is called \allbold{cyclic}.

\begin{defn}[generator]
	The elements 1 (or 3) for $\zz_4$ is called a \allbold{generator} for $\zz_4$. 
\end{defn}

\begin{defn}[cyclic group]
	A group is \allbold{cyclic} if it has a generator.
\end{defn}

\begin{defn}[generated subgroup]
The \allbold{subgroup generated by} $x \in G$ is the smallest subgroup of $G$ that contains $x$. We denote the subgroup generated by  $x$ by  $\langle x \rangle$. 
\end{defn}

\begin{eg}[]
In $V_4$, the following hold:
 \begin{itemize}
	\item $\langle a \rangle = \{e,a\} $ 
	\item $\langle b \rangle = \{e,b\} $ 
	\item $\langle c \rangle = \{e,c\} $ 
	\item $\langle e \rangle=\{e\} $
\end{itemize}
None of these is the whole group. This means $V_4$ is not cyclic, and has no generator.
\end{eg}

\begin{eg}[]
In $\zz_4$, 
\begin{itemize}
	\item $\langle 0 \rangle=\{0\} $ 
	\item $\langle 1 \rangle = \{0,1,2,3\} $ 
	\item $\langle 2 \rangle= \{0,2\} $ 
	\item $\langle 3 \rangle= \{0,1,2,3\} $
\end{itemize}
$\zz_4$ is cyclic, it is generated by 1 or 3.
\end{eg}
\end{document}

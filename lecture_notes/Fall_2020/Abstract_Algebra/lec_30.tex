\documentclass[class=article,crop=false]{standalone} 
%Fall 2020
% Some basic packages
\usepackage{standalone}[subpreambles=true]
\usepackage[utf8]{inputenc}
\usepackage[T1]{fontenc}
\usepackage{textcomp}
\usepackage[english]{babel}
\usepackage{url}
\usepackage{graphicx}
\usepackage{float}
\usepackage{enumitem}


\pdfminorversion=7

% Don't indent paragraphs, leave some space between them
\usepackage{parskip}

% Hide page number when page is empty
\usepackage{emptypage}
\usepackage{subcaption}
\usepackage{multicol}
\usepackage[dvipsnames]{xcolor}


% Math stuff
\usepackage{amsmath, amsfonts, mathtools, amsthm, amssymb}
% Fancy script capitals
\usepackage{mathrsfs}
\usepackage{cancel}
% Bold math
\usepackage{bm}
% Some shortcuts
\newcommand{\rr}{\ensuremath{\mathbb{R}}}
\newcommand{\zz}{\ensuremath{\mathbb{Z}}}
\newcommand{\qq}{\ensuremath{\mathbb{Q}}}
\newcommand{\nn}{\ensuremath{\mathbb{N}}}
\newcommand{\ff}{\ensuremath{\mathbb{F}}}
\newcommand{\cc}{\ensuremath{\mathbb{C}}}
\renewcommand\O{\ensuremath{\emptyset}}
\newcommand{\norm}[1]{{\left\lVert{#1}\right\rVert}}
\renewcommand{\vec}[1]{{\mathbf{#1}}}
\newcommand\allbold[1]{{\boldmath\textbf{#1}}}

% Put x \to \infty below \lim
\let\svlim\lim\def\lim{\svlim\limits}

%Make implies and impliedby shorter
\let\implies\Rightarrow
\let\impliedby\Leftarrow
\let\iff\Leftrightarrow
\let\epsilon\varepsilon

% Add \contra symbol to denote contradiction
\usepackage{stmaryrd} % for \lightning
\newcommand\contra{\scalebox{1.5}{$\lightning$}}

% \let\phi\varphi

% Command for short corrections
% Usage: 1+1=\correct{3}{2}

\definecolor{correct}{HTML}{009900}
\newcommand\correct[2]{\ensuremath{\:}{\color{red}{#1}}\ensuremath{\to }{\color{correct}{#2}}\ensuremath{\:}}
\newcommand\green[1]{{\color{correct}{#1}}}

% horizontal rule
\newcommand\hr{
    \noindent\rule[0.5ex]{\linewidth}{0.5pt}
}

% hide parts
\newcommand\hide[1]{}

% si unitx
\usepackage{siunitx}
\sisetup{locale = FR}

% Environments
\makeatother
% For box around Definition, Theorem, \ldots
\usepackage[framemethod=TikZ]{mdframed}
\mdfsetup{skipabove=1em,skipbelow=0em}

%definition
\newenvironment{defn}[1][]{%
\ifstrempty{#1}%
{\mdfsetup{%
frametitle={%
\tikz[baseline=(current bounding box.east),outer sep=0pt]
\node[anchor=east,rectangle,fill=Emerald]
{\strut Definition};}}
}%
{\mdfsetup{%
frametitle={%
\tikz[baseline=(current bounding box.east),outer sep=0pt]
\node[anchor=east,rectangle,fill=Emerald]
{\strut Definition:~#1};}}%
}%
\mdfsetup{innertopmargin=10pt,linecolor=Emerald,%
linewidth=2pt,topline=true,%
frametitleaboveskip=\dimexpr-\ht\strutbox\relax
}
\begin{mdframed}[]\relax%
\label{#1}}{\end{mdframed}}


%theorem
%\newcounter{thm}[section]\setcounter{thm}{0}
%\renewcommand{\thethm}{\arabic{section}.\arabic{thm}}
\newenvironment{thm}[1][]{%
%\refstepcounter{thm}%
\ifstrempty{#1}%
{\mdfsetup{%
frametitle={%
\tikz[baseline=(current bounding box.east),outer sep=0pt]
\node[anchor=east,rectangle,fill=blue!20]
%{\strut Theorem~\thethm};}}
{\strut Theorem};}}
}%
{\mdfsetup{%
frametitle={%
\tikz[baseline=(current bounding box.east),outer sep=0pt]
\node[anchor=east,rectangle,fill=blue!20]
%{\strut Theorem~\thethm:~#1};}}%
{\strut Theorem:~#1};}}%
}%
\mdfsetup{innertopmargin=10pt,linecolor=blue!20,%
linewidth=2pt,topline=true,%
frametitleaboveskip=\dimexpr-\ht\strutbox\relax
}
\begin{mdframed}[]\relax%
\label{#1}}{\end{mdframed}}


%lemma
\newenvironment{lem}[1][]{%
\ifstrempty{#1}%
{\mdfsetup{%
frametitle={%
\tikz[baseline=(current bounding box.east),outer sep=0pt]
\node[anchor=east,rectangle,fill=Dandelion]
{\strut Lemma};}}
}%
{\mdfsetup{%
frametitle={%
\tikz[baseline=(current bounding box.east),outer sep=0pt]
\node[anchor=east,rectangle,fill=Dandelion]
{\strut Lemma:~#1};}}%
}%
\mdfsetup{innertopmargin=10pt,linecolor=Dandelion,%
linewidth=2pt,topline=true,%
frametitleaboveskip=\dimexpr-\ht\strutbox\relax
}
\begin{mdframed}[]\relax%
\label{#1}}{\end{mdframed}}

%corollary
\newenvironment{coro}[1][]{%
\ifstrempty{#1}%
{\mdfsetup{%
frametitle={%
\tikz[baseline=(current bounding box.east),outer sep=0pt]
\node[anchor=east,rectangle,fill=CornflowerBlue]
{\strut Corollary};}}
}%
{\mdfsetup{%
frametitle={%
\tikz[baseline=(current bounding box.east),outer sep=0pt]
\node[anchor=east,rectangle,fill=CornflowerBlue]
{\strut Corollary:~#1};}}%
}%
\mdfsetup{innertopmargin=10pt,linecolor=CornflowerBlue,%
linewidth=2pt,topline=true,%
frametitleaboveskip=\dimexpr-\ht\strutbox\relax
}
\begin{mdframed}[]\relax%
\label{#1}}{\end{mdframed}}

%proof
\newenvironment{prf}[1][]{%
\ifstrempty{#1}%
{\mdfsetup{%
frametitle={%
\tikz[baseline=(current bounding box.east),outer sep=0pt]
\node[anchor=east,rectangle,fill=SpringGreen]
{\strut Proof};}}
}%
{\mdfsetup{%
frametitle={%
\tikz[baseline=(current bounding box.east),outer sep=0pt]
\node[anchor=east,rectangle,fill=SpringGreen]
{\strut Proof:~#1};}}%
}%
\mdfsetup{innertopmargin=10pt,linecolor=SpringGreen,%
linewidth=2pt,topline=true,%
frametitleaboveskip=\dimexpr-\ht\strutbox\relax
}
\begin{mdframed}[]\relax%
\label{#1}}{\qed\end{mdframed}}


\theoremstyle{definition}

\newmdtheoremenv[nobreak=true]{definition}{Definition}
\newmdtheoremenv[nobreak=true]{prop}{Proposition}
\newmdtheoremenv[nobreak=true]{theorem}{Theorem}
\newmdtheoremenv[nobreak=true]{corollary}{Corollary}
\newtheorem*{eg}{Example}
\theoremstyle{remark}
\newtheorem*{case}{Case}
\newtheorem*{notation}{Notation}
\newtheorem*{remark}{Remark}
\newtheorem*{note}{Note}
\newtheorem*{problem}{Problem}
\newtheorem*{observe}{Observe}
\newtheorem*{property}{Property}
\newtheorem*{intuition}{Intuition}


% End example and intermezzo environments with a small diamond (just like proof
% environments end with a small square)
\usepackage{etoolbox}
\AtEndEnvironment{vb}{\null\hfill$\diamond$}%
\AtEndEnvironment{intermezzo}{\null\hfill$\diamond$}%
% \AtEndEnvironment{opmerking}{\null\hfill$\diamond$}%

% Fix some spacing
% http://tex.stackexchange.com/questions/22119/how-can-i-change-the-spacing-before-theorems-with-amsthm
\makeatletter
\def\thm@space@setup{%
  \thm@preskip=\parskip \thm@postskip=0pt
}

% Fix some stuff
% %http://tex.stackexchange.com/questions/76273/multiple-pdfs-with-page-group-included-in-a-single-page-warning
\pdfsuppresswarningpagegroup=1


% My name
\author{Jaden Wang}



\begin{document}
\begin{remark}
	$ \mathbb{H}$ is a division ring but not an integral domain, because it is not commutative.
\end{remark}

\begin{remark}
Idempotent elements in integral domain: 0 and 1.

In $ \zz_6$, we have 0, 1, . They pair up because $ e ^2 = e \implies (1-e)^2=1-e$.

If $ S \leq R$ and  $ S$ has identity  $ e$, then  $ e$ is an idempotent in  $ R$.

 \begin{eg}[]
	 $ R=M_3 (\rr)$. $ S= \begin{pmatrix} a&b&0\\c&d&0\\0&0&0 \end{pmatrix} : a,b,c,d \in \rr $. Then $ S\leq R$.  $ S \neq \O$ and $ S$ is closed under  $ +,-,\times $. $ S$ has the identity
	  \[
		  \begin{pmatrix} 1&0&0\\0&1&0\\0&0&0 \end{pmatrix} ^2 =  \begin{pmatrix} 1&0&0\\0&1&0\\0&0&0 \end{pmatrix} 
	 .\]
	 So the identity of the subring doesn't have to be the identity of the parent ring, but it has to be idempotent.
\end{eg}
\end{remark}

Recall we are trying to construct a field $ F$ containing an integral domain  $ D$ so that  $ F$ is as small as possible.

 \begin{intuition}
	 Consider $ \frac{a}{b}, a,b \in \zz, b\neq 0$. We denote $ (a,b)$ as an element in $ S=\{(a,b) \in \zz \times \zz: b\neq 0\} $. So we want to express that two fractions are the same by using $ ad=bc$. We would use an equivalence relation to show that.
\end{intuition}

\begin{lem}[]
	Define $ (a,b) \sim (c,d) \iff ad=bc$. Then $ \sim$ is an equivalence relation on  $ S$.
\end{lem}
\begin{prf}
\begin{enumerate}[label=(\roman*)]
	\item Reflective: $ ab=ba$ is true by commutativity.
	\item Symmetric: $ ad=bc \implies ^3=da$ is true by commutativity.
	\item Transitive: If $ (a,b) \sim (c,d)$ and  $ (c,d) \sim (e,f)$, then  $ (a,b) \sim (e,f)$.
		Assume  $ ad=bc, cf=de$. Then
		 \begin{align*}
			adcf &= bcde \\
			acfd &= bced \\
			acf&= bce \text{ by cancellation and } d\neq 0 \\
		\end{align*}
		\begin{case}[]
		$ c\neq 0$. Then  $ afc=bec \implies af=be$. Done.
		\end{case}
		\begin{case}[]
		$ c=0$. Then  $ ad=bc=0 \implies a=0$ since $ d\neq 0$. Also  $ cf=de \implies de=0 \implies e=0$ since $ d\neq 0$. So  $ af=be =0$.
		\end{case}
\end{enumerate}
\end{prf}
\begin{notation}
Define (the set of) the field $ F$ as  $ F= S / \sim$, which is the set of the  $ \sim$-equivalence classes of  $ S$.
\end{notation}
 \begin{eg}[]
	 Suppose $ D=\zz$, then denote the equivalence class of $ (1,2)$ as $ [(1,2)] = \{b=2a: a,b \in \zz, b\neq 0\} $.
\end{eg}
\begin{eg}[]
In $ \qq$, $ \frac{1}{2}+\frac{1}{3}=\frac{5}{6}$. What if we use different names will it still be well-defined? Yes.
\end{eg}

How can we add them?
\[
	[(a,b)]+[(c,d)]=[(ad+bc, ad)]
.\] 
For multiplication,
\[
	[(a,b)] \times [(c,d)] = [(ac,db)]
.\]
We need to check they are well-defined.
\begin{prf}
	Suppose $ (a',b') \sim (a,b), (c',d') \sim (c,d)$. We need to show that  $ (a'd'+b'c',a'd') \sim (ad+bc,ad)$. Equivalence gives us  $ a'b=b'a, c'd=d'c$.
	 \begin{align*}
		 (a'd'+b'c')ad&= a'd'(ad+bc) \\
		 a'd'ad+b'c'ad&= a'd'ad + bca'd' \\
		 b'c'ad &= bc'a'd = bca'd' \\
	\end{align*}
\end{prf}

Likewise for addition. So both are well-defined. 

Addition is closed in $ S$ because  $ b,d \neq 0 $ and  $ D$ is a domain without zero divisors. Likewise for multiplication and subtraction.

We still need to show addition and multiplication are associative (omitted).

What is the additive identity?  $ [(a,b)] = [(0,1)]$.  $ b$ can be anything nonzero since $ (0,1) \sim (0,b)$. This works. 

Additive inverse of $ [(a,b)]$ is  $[(-a,b)] $. We again can check it's true since  $ [(0,b^2)] \sim [(0,1)]$. 

What is the multiplicative identity? $ [(a,b)] = [(1,1)]$. 

Multiplicative inverse of  $ [(a,b)] $ is  $ [(b,a)]$ since if $ a = 0$, then $ [(a,b)]= [(0,b)] = [(0,1)]$ which is the zero so we can ignore. And  $ [(ab,ab)] = [(1,1)]$.

Commutative? Yes.

Division ring? If $ [(a,b)]\neq 0_F$ then  $ a\neq 0$ and  $ [(b,a)]$ is the inverse.

So  $ F$ is a field. We still need to show that  $ F$ contains the  $ D$ or has a subring that looks like  $ D$.
\end{document}

\documentclass[12pt]{article}
%Fall 2020
% Some basic packages
\usepackage{standalone}[subpreambles=true]
\usepackage[utf8]{inputenc}
\usepackage[T1]{fontenc}
\usepackage{textcomp}
\usepackage[english]{babel}
\usepackage{url}
\usepackage{graphicx}
\usepackage{float}
\usepackage{enumitem}


\pdfminorversion=7

% Don't indent paragraphs, leave some space between them
\usepackage{parskip}

% Hide page number when page is empty
\usepackage{emptypage}
\usepackage{subcaption}
\usepackage{multicol}
\usepackage[dvipsnames]{xcolor}


% Math stuff
\usepackage{amsmath, amsfonts, mathtools, amsthm, amssymb}
% Fancy script capitals
\usepackage{mathrsfs}
\usepackage{cancel}
% Bold math
\usepackage{bm}
% Some shortcuts
\newcommand{\rr}{\ensuremath{\mathbb{R}}}
\newcommand{\zz}{\ensuremath{\mathbb{Z}}}
\newcommand{\qq}{\ensuremath{\mathbb{Q}}}
\newcommand{\nn}{\ensuremath{\mathbb{N}}}
\newcommand{\ff}{\ensuremath{\mathbb{F}}}
\newcommand{\cc}{\ensuremath{\mathbb{C}}}
\renewcommand\O{\ensuremath{\emptyset}}
\newcommand{\norm}[1]{{\left\lVert{#1}\right\rVert}}
\renewcommand{\vec}[1]{{\mathbf{#1}}}
\newcommand\allbold[1]{{\boldmath\textbf{#1}}}

% Put x \to \infty below \lim
\let\svlim\lim\def\lim{\svlim\limits}

%Make implies and impliedby shorter
\let\implies\Rightarrow
\let\impliedby\Leftarrow
\let\iff\Leftrightarrow
\let\epsilon\varepsilon

% Add \contra symbol to denote contradiction
\usepackage{stmaryrd} % for \lightning
\newcommand\contra{\scalebox{1.5}{$\lightning$}}

% \let\phi\varphi

% Command for short corrections
% Usage: 1+1=\correct{3}{2}

\definecolor{correct}{HTML}{009900}
\newcommand\correct[2]{\ensuremath{\:}{\color{red}{#1}}\ensuremath{\to }{\color{correct}{#2}}\ensuremath{\:}}
\newcommand\green[1]{{\color{correct}{#1}}}

% horizontal rule
\newcommand\hr{
    \noindent\rule[0.5ex]{\linewidth}{0.5pt}
}

% hide parts
\newcommand\hide[1]{}

% si unitx
\usepackage{siunitx}
\sisetup{locale = FR}

% Environments
\makeatother
% For box around Definition, Theorem, \ldots
\usepackage[framemethod=TikZ]{mdframed}
\mdfsetup{skipabove=1em,skipbelow=0em}

%definition
\newenvironment{defn}[1][]{%
\ifstrempty{#1}%
{\mdfsetup{%
frametitle={%
\tikz[baseline=(current bounding box.east),outer sep=0pt]
\node[anchor=east,rectangle,fill=Emerald]
{\strut Definition};}}
}%
{\mdfsetup{%
frametitle={%
\tikz[baseline=(current bounding box.east),outer sep=0pt]
\node[anchor=east,rectangle,fill=Emerald]
{\strut Definition:~#1};}}%
}%
\mdfsetup{innertopmargin=10pt,linecolor=Emerald,%
linewidth=2pt,topline=true,%
frametitleaboveskip=\dimexpr-\ht\strutbox\relax
}
\begin{mdframed}[]\relax%
\label{#1}}{\end{mdframed}}


%theorem
%\newcounter{thm}[section]\setcounter{thm}{0}
%\renewcommand{\thethm}{\arabic{section}.\arabic{thm}}
\newenvironment{thm}[1][]{%
%\refstepcounter{thm}%
\ifstrempty{#1}%
{\mdfsetup{%
frametitle={%
\tikz[baseline=(current bounding box.east),outer sep=0pt]
\node[anchor=east,rectangle,fill=blue!20]
%{\strut Theorem~\thethm};}}
{\strut Theorem};}}
}%
{\mdfsetup{%
frametitle={%
\tikz[baseline=(current bounding box.east),outer sep=0pt]
\node[anchor=east,rectangle,fill=blue!20]
%{\strut Theorem~\thethm:~#1};}}%
{\strut Theorem:~#1};}}%
}%
\mdfsetup{innertopmargin=10pt,linecolor=blue!20,%
linewidth=2pt,topline=true,%
frametitleaboveskip=\dimexpr-\ht\strutbox\relax
}
\begin{mdframed}[]\relax%
\label{#1}}{\end{mdframed}}


%lemma
\newenvironment{lem}[1][]{%
\ifstrempty{#1}%
{\mdfsetup{%
frametitle={%
\tikz[baseline=(current bounding box.east),outer sep=0pt]
\node[anchor=east,rectangle,fill=Dandelion]
{\strut Lemma};}}
}%
{\mdfsetup{%
frametitle={%
\tikz[baseline=(current bounding box.east),outer sep=0pt]
\node[anchor=east,rectangle,fill=Dandelion]
{\strut Lemma:~#1};}}%
}%
\mdfsetup{innertopmargin=10pt,linecolor=Dandelion,%
linewidth=2pt,topline=true,%
frametitleaboveskip=\dimexpr-\ht\strutbox\relax
}
\begin{mdframed}[]\relax%
\label{#1}}{\end{mdframed}}

%corollary
\newenvironment{coro}[1][]{%
\ifstrempty{#1}%
{\mdfsetup{%
frametitle={%
\tikz[baseline=(current bounding box.east),outer sep=0pt]
\node[anchor=east,rectangle,fill=CornflowerBlue]
{\strut Corollary};}}
}%
{\mdfsetup{%
frametitle={%
\tikz[baseline=(current bounding box.east),outer sep=0pt]
\node[anchor=east,rectangle,fill=CornflowerBlue]
{\strut Corollary:~#1};}}%
}%
\mdfsetup{innertopmargin=10pt,linecolor=CornflowerBlue,%
linewidth=2pt,topline=true,%
frametitleaboveskip=\dimexpr-\ht\strutbox\relax
}
\begin{mdframed}[]\relax%
\label{#1}}{\end{mdframed}}

%proof
\newenvironment{prf}[1][]{%
\ifstrempty{#1}%
{\mdfsetup{%
frametitle={%
\tikz[baseline=(current bounding box.east),outer sep=0pt]
\node[anchor=east,rectangle,fill=SpringGreen]
{\strut Proof};}}
}%
{\mdfsetup{%
frametitle={%
\tikz[baseline=(current bounding box.east),outer sep=0pt]
\node[anchor=east,rectangle,fill=SpringGreen]
{\strut Proof:~#1};}}%
}%
\mdfsetup{innertopmargin=10pt,linecolor=SpringGreen,%
linewidth=2pt,topline=true,%
frametitleaboveskip=\dimexpr-\ht\strutbox\relax
}
\begin{mdframed}[]\relax%
\label{#1}}{\qed\end{mdframed}}


\theoremstyle{definition}

\newmdtheoremenv[nobreak=true]{definition}{Definition}
\newmdtheoremenv[nobreak=true]{prop}{Proposition}
\newmdtheoremenv[nobreak=true]{theorem}{Theorem}
\newmdtheoremenv[nobreak=true]{corollary}{Corollary}
\newtheorem*{eg}{Example}
\theoremstyle{remark}
\newtheorem*{case}{Case}
\newtheorem*{notation}{Notation}
\newtheorem*{remark}{Remark}
\newtheorem*{note}{Note}
\newtheorem*{problem}{Problem}
\newtheorem*{observe}{Observe}
\newtheorem*{property}{Property}
\newtheorem*{intuition}{Intuition}


% End example and intermezzo environments with a small diamond (just like proof
% environments end with a small square)
\usepackage{etoolbox}
\AtEndEnvironment{vb}{\null\hfill$\diamond$}%
\AtEndEnvironment{intermezzo}{\null\hfill$\diamond$}%
% \AtEndEnvironment{opmerking}{\null\hfill$\diamond$}%

% Fix some spacing
% http://tex.stackexchange.com/questions/22119/how-can-i-change-the-spacing-before-theorems-with-amsthm
\makeatletter
\def\thm@space@setup{%
  \thm@preskip=\parskip \thm@postskip=0pt
}

% Fix some stuff
% %http://tex.stackexchange.com/questions/76273/multiple-pdfs-with-page-group-included-in-a-single-page-warning
\pdfsuppresswarningpagegroup=1


% My name
\author{Jaden Wang}



\begin{document}
\centerline {\textsf{\textbf{\LARGE{Homework 2}}}}
\centerline {Jaden Wang}
\vspace{.15in}

\begin{problem}[3.2]
Yes.
\begin{enumerate}[label=\roman*)]
	\item injective: Given $x,y \in \zz$, we have $\phi(x) =-x,\phi(y) =-y$, if $\phi(x) =\phi(y) $, then $-x = -y \implies x =y$.
	\item surjective: Given $b \in \zz$, choose $a =-b \in \zz$. We see that $\phi(a) = -(-b) = b$.
	\item $\phi(x*y) = -(x+y) = -x + (-y) = \phi(x) *' \phi(y) $. \\

Hence $\phi $ satisfies the definition of an isomorphism.
\end{enumerate}
\end{problem}

\begin{problem}[3.3]
No. Because $\phi$ is not surjective: $\phi(a) = 2a \neq 1 \quad \forall a \in \zz$.
\end{problem}

\begin{problem}[3.11]
No. Consider $f,g \in F$ where $f(x) = 1$ and  $g(x) = 2$. Clearly  $\phi(f(x)) =0 = \phi(g(x)) $. However, $f(x) \neq g(x)$.  $\phi$ is then not injective and thus cannot be an isomorphism.
\end{problem}

\begin{problem}[3.12]
	No. Consider the same $f,g$ as above. $\phi(f(x)) =f'(0) = 0 = g'(0) = \phi(g(x)) $, yet $f(x) \neq g(x)$. Thus injectivity failed.
\end{problem}

\begin{problem}[4.1]
Not a group. $\mathcal{G}_3$ doesn't hold because $0$ doesn't have an inverse such that $0 * 0^{-1} = 1$.
\end{problem}

\begin{problem}[4.2]
It is a group.
\begin{enumerate}[label=(\roman*)]
	\item scalar addition is associative.
	\item The identity $e = 0 \in 2\zz$, since $0 + a = a+ 0 = a \quad \forall a \in 2\zz$.
	\item Given $a \in 2\zz$, let $a^{-1} = - a \in 2\zz$ so that 
\[
		a*a^{-1}=a+(-a)=0=-a+a=a^{-1} * a
.\]
\end{enumerate}
\end{problem}

\begin{problem}[4.8]
Consider the set $\{1,3,5,7\} $ and the operation $*=\cdot_{8} $:
\begin{table}[htpb]
	\centering
	\begin{tabular}{c||c|c|c|c}
		*&1&3&5&7\\
		\hline
		\hline
		1&1&3&5&7\\
		\hline
		3&3&1&7&5\\
		\hline
		5&5&7&1&3\\
		\hline
		7&7&5&3&1
	\end{tabular}
\end{table}
\end{problem}

\begin{problem}[4.11]
Yes. Let's denote this set as $D_n$. Given $A \in D_n$, 
\begin{enumerate}[label=(\roman*)]
	\item matrix addition is associative.
	\item Let $O_n$ be the $n \times n$ matrix with all $0$ entries. Given  $A \in D_n$, $A+ O_n=O_n+A =A$. Hence $O_n$ is the identity.
	\item Since $A+(-A) = -A +A = O_n$, the inverse $A^{-1} = -A$ exists for all $ A \in D_n$.
\end{enumerate}
\end{problem}

\begin{problem}[4.12]
	No. Notice that $O_n$ doesn't have an inverse such that $O_n \times O_n^{-1} = I_n$.
\end{problem}

\begin{problem}[4.13]
Yes. 
\begin{enumerate}[label=(\roman*)]
	\item matrix multiplication is associative.
	\item Let $I_n$ be the $n \times n$ identity matrix. Then given $A \in D_n$ $I_n \times  A = A \times I_n = A$, so $I_n$ is the identity.
	\item Let $B_{ii} = \frac{1}{A_{ii}}$ for $i \in [1,n] and i \in \nn$. Then $A_{ii} \cdot B_{ii} = B_{ii} \cdot A_{ii} = 1$, and $A \cdot B = B \cdot A = I_n$. So $A^{-1} = B$.
\end{enumerate}
\end{problem}

\begin{problem}[4.14]
Yes. It's trivially true because this is a special case of Problem 4.13.
\end{enumerate}
\end{problem}

\begin{problem}[4.19]
~\begin{enumerate}[label=\alph*)]
	\item Given $a,b \in S$, we want to show that $a*b \in S$. Instead we will prove the contrapositive. That is, if $a*b \not \in S$, then  $a \not \in S$ or  $b \not \in S$. Since  $-1$ is the only real number not in $S$, and that $+$ is only defined on real numbers, we only need to check this one case by setting  $a*b = -1$:
		 \begin{align*}
			 a+b+ab &= -1\\
			 a+b+ab+1 &= 0 \\
			 a(b+1)+b+1 &= 0 \\
			 (a+1)(b+1) &= 0
		\end{align*}
	The solution is that either $a = -1 \not \in S$ or  $b=-1 \not \in S$ as required. Hence by the contrapositive, we prove that  $a*b \in S$, which makes $*$ a binary operation. 
\item 
	~\begin{enumerate}[label=(\roman*)]
		\item Since scalar multiplication and addition are associative and commutative, we can do the following:
			\begin{align*}
				(a*b)*c&=(a+b+ab)*c\\
				&= a+b+ab+c+ac+bc+abc \\
				a*(b*c) &= a*(b+c+bc) \\
				&= a+b + c+bc+ab+ac+abc \\
				&= a+b+ab+c+ac+bc+abc \\
				&= (a*b)*c 
			\end{align*}
		\item Let $e=0$, we see that  $a*e=a+0+a\cdot 0=a = 0+a+0\cdot a= e*a$. Hence $0$ is the identity.
		\item Let  $b = -\frac{a}{1+a}$. Note that since $a \neq -1$,  $\frac{a}{1+a} \neq 1$ so $b \neq -1$. Thus  $b \in S$.
			\begin{align*}
				a*b &= a- \frac{a}{1+a}-\frac{a^2}{1+a} \\
				&= \frac{a+a^2-a-a^2}{1+a} \\
				&= 0 \\
b*a &= -\frac{a}{1+a} + a -\frac{a^2}{1+a} \\
&= \frac{-a+a+a^2-a^2}{1+a} \\
&= 0
			\end{align*}
	Hence $a*b=b*a=0$, so $a$ has an inverse $a^{-1}=b=-\frac{a}{1+a}$.
	\end{enumerate}
	Hence $(S,*)$ is a group.
\item 
	\begin{align*}
		2*x*3&= 7 \\
		(2+x+2x)*3 &= 7 \\
		2+x+2x+3+6+3x+6x &= 7 \\
		12x&= -4 \\
		x&= -\frac{1}{3} \in S
	\end{align*}
\end{enumerate}
\end{problem}

\begin{problem}[4.25]
~\begin{enumerate}[label=\alph*)]
	\item False. Given a group $(G,*)$, suppose there exist two identities  $e_1,e_2$ such that  given $a \in G$, $a*e_1=e_1*a=a$, $a*e_2=e_2*a=a$. Then $e_1*e_2 = e_1 = e_2$, so the identity is unique.
	\item True. In class we see that there is only one way to arrange the table so that it is a valid group.
	\item True. Because linear equations can be represented by $Mx = b$, where  $M$ is a matrix. Since matrices in a group have to be invertible, there always exists a solution  $x = M^{-1} b$.
	\item False. The proper attitude should be to understand instead of memorize definitions through struggling with challenging problems, so after engaging with them so deeply you would know them by heart without spending effort memorizing them.
	\item False. This only shows one direction. It is possible that there exists $a \in G_{text}$ such that $a \not \in G_{person}$. Then $G_{person}$ would not be the correction definition.
	\item True. If both definitions describe the exact objects, then they can be used interchangeably. If and only if is equivalent to an definition.
	\item True. We know from class that for a group with at most three elements, there is only one way to arrange them in a table per distinct number of elements. And these arrangements happen to be commutative. Hence these groups are abelian. 
	\item True. Since this equation is in the context of a group, the identity $e$ and $a^{-1}$, $b^{-1}$ exist.
		 \begin{align*}
			 a^{-1} * a*x*b&=a^{-1} * c \\
			 e*x*b*b ^{-1} &= a^{-1} * c*b ^{-1}\\
			 x*e &= a^{-1} * c * b ^{-1} \\
			 x&= a^{-1} * c * b ^{-1} \\
		\end{align*}
		Which yields a unique element as the solution as $*$ is a function.
	\item False. We also need a binary operation to construct a group.
	\item True. A group is a set with a binary operation $*$, which satisfies the definition of  a binary algebraic structure.
\end{enumerate}
\end{problem}
\end{document}

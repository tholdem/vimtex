\documentclass[class=article,crop=false]{standalone} 
%Fall 2020
% Some basic packages
\usepackage{standalone}[subpreambles=true]
\usepackage[utf8]{inputenc}
\usepackage[T1]{fontenc}
\usepackage{textcomp}
\usepackage[english]{babel}
\usepackage{url}
\usepackage{graphicx}
\usepackage{float}
\usepackage{enumitem}


\pdfminorversion=7

% Don't indent paragraphs, leave some space between them
\usepackage{parskip}

% Hide page number when page is empty
\usepackage{emptypage}
\usepackage{subcaption}
\usepackage{multicol}
\usepackage[dvipsnames]{xcolor}


% Math stuff
\usepackage{amsmath, amsfonts, mathtools, amsthm, amssymb}
% Fancy script capitals
\usepackage{mathrsfs}
\usepackage{cancel}
% Bold math
\usepackage{bm}
% Some shortcuts
\newcommand{\rr}{\ensuremath{\mathbb{R}}}
\newcommand{\zz}{\ensuremath{\mathbb{Z}}}
\newcommand{\qq}{\ensuremath{\mathbb{Q}}}
\newcommand{\nn}{\ensuremath{\mathbb{N}}}
\newcommand{\ff}{\ensuremath{\mathbb{F}}}
\newcommand{\cc}{\ensuremath{\mathbb{C}}}
\renewcommand\O{\ensuremath{\emptyset}}
\newcommand{\norm}[1]{{\left\lVert{#1}\right\rVert}}
\renewcommand{\vec}[1]{{\mathbf{#1}}}
\newcommand\allbold[1]{{\boldmath\textbf{#1}}}

% Put x \to \infty below \lim
\let\svlim\lim\def\lim{\svlim\limits}

%Make implies and impliedby shorter
\let\implies\Rightarrow
\let\impliedby\Leftarrow
\let\iff\Leftrightarrow
\let\epsilon\varepsilon

% Add \contra symbol to denote contradiction
\usepackage{stmaryrd} % for \lightning
\newcommand\contra{\scalebox{1.5}{$\lightning$}}

% \let\phi\varphi

% Command for short corrections
% Usage: 1+1=\correct{3}{2}

\definecolor{correct}{HTML}{009900}
\newcommand\correct[2]{\ensuremath{\:}{\color{red}{#1}}\ensuremath{\to }{\color{correct}{#2}}\ensuremath{\:}}
\newcommand\green[1]{{\color{correct}{#1}}}

% horizontal rule
\newcommand\hr{
    \noindent\rule[0.5ex]{\linewidth}{0.5pt}
}

% hide parts
\newcommand\hide[1]{}

% si unitx
\usepackage{siunitx}
\sisetup{locale = FR}

% Environments
\makeatother
% For box around Definition, Theorem, \ldots
\usepackage[framemethod=TikZ]{mdframed}
\mdfsetup{skipabove=1em,skipbelow=0em}

%definition
\newenvironment{defn}[1][]{%
\ifstrempty{#1}%
{\mdfsetup{%
frametitle={%
\tikz[baseline=(current bounding box.east),outer sep=0pt]
\node[anchor=east,rectangle,fill=Emerald]
{\strut Definition};}}
}%
{\mdfsetup{%
frametitle={%
\tikz[baseline=(current bounding box.east),outer sep=0pt]
\node[anchor=east,rectangle,fill=Emerald]
{\strut Definition:~#1};}}%
}%
\mdfsetup{innertopmargin=10pt,linecolor=Emerald,%
linewidth=2pt,topline=true,%
frametitleaboveskip=\dimexpr-\ht\strutbox\relax
}
\begin{mdframed}[]\relax%
\label{#1}}{\end{mdframed}}


%theorem
%\newcounter{thm}[section]\setcounter{thm}{0}
%\renewcommand{\thethm}{\arabic{section}.\arabic{thm}}
\newenvironment{thm}[1][]{%
%\refstepcounter{thm}%
\ifstrempty{#1}%
{\mdfsetup{%
frametitle={%
\tikz[baseline=(current bounding box.east),outer sep=0pt]
\node[anchor=east,rectangle,fill=blue!20]
%{\strut Theorem~\thethm};}}
{\strut Theorem};}}
}%
{\mdfsetup{%
frametitle={%
\tikz[baseline=(current bounding box.east),outer sep=0pt]
\node[anchor=east,rectangle,fill=blue!20]
%{\strut Theorem~\thethm:~#1};}}%
{\strut Theorem:~#1};}}%
}%
\mdfsetup{innertopmargin=10pt,linecolor=blue!20,%
linewidth=2pt,topline=true,%
frametitleaboveskip=\dimexpr-\ht\strutbox\relax
}
\begin{mdframed}[]\relax%
\label{#1}}{\end{mdframed}}


%lemma
\newenvironment{lem}[1][]{%
\ifstrempty{#1}%
{\mdfsetup{%
frametitle={%
\tikz[baseline=(current bounding box.east),outer sep=0pt]
\node[anchor=east,rectangle,fill=Dandelion]
{\strut Lemma};}}
}%
{\mdfsetup{%
frametitle={%
\tikz[baseline=(current bounding box.east),outer sep=0pt]
\node[anchor=east,rectangle,fill=Dandelion]
{\strut Lemma:~#1};}}%
}%
\mdfsetup{innertopmargin=10pt,linecolor=Dandelion,%
linewidth=2pt,topline=true,%
frametitleaboveskip=\dimexpr-\ht\strutbox\relax
}
\begin{mdframed}[]\relax%
\label{#1}}{\end{mdframed}}

%corollary
\newenvironment{coro}[1][]{%
\ifstrempty{#1}%
{\mdfsetup{%
frametitle={%
\tikz[baseline=(current bounding box.east),outer sep=0pt]
\node[anchor=east,rectangle,fill=CornflowerBlue]
{\strut Corollary};}}
}%
{\mdfsetup{%
frametitle={%
\tikz[baseline=(current bounding box.east),outer sep=0pt]
\node[anchor=east,rectangle,fill=CornflowerBlue]
{\strut Corollary:~#1};}}%
}%
\mdfsetup{innertopmargin=10pt,linecolor=CornflowerBlue,%
linewidth=2pt,topline=true,%
frametitleaboveskip=\dimexpr-\ht\strutbox\relax
}
\begin{mdframed}[]\relax%
\label{#1}}{\end{mdframed}}

%proof
\newenvironment{prf}[1][]{%
\ifstrempty{#1}%
{\mdfsetup{%
frametitle={%
\tikz[baseline=(current bounding box.east),outer sep=0pt]
\node[anchor=east,rectangle,fill=SpringGreen]
{\strut Proof};}}
}%
{\mdfsetup{%
frametitle={%
\tikz[baseline=(current bounding box.east),outer sep=0pt]
\node[anchor=east,rectangle,fill=SpringGreen]
{\strut Proof:~#1};}}%
}%
\mdfsetup{innertopmargin=10pt,linecolor=SpringGreen,%
linewidth=2pt,topline=true,%
frametitleaboveskip=\dimexpr-\ht\strutbox\relax
}
\begin{mdframed}[]\relax%
\label{#1}}{\qed\end{mdframed}}


\theoremstyle{definition}

\newmdtheoremenv[nobreak=true]{definition}{Definition}
\newmdtheoremenv[nobreak=true]{prop}{Proposition}
\newmdtheoremenv[nobreak=true]{theorem}{Theorem}
\newmdtheoremenv[nobreak=true]{corollary}{Corollary}
\newtheorem*{eg}{Example}
\theoremstyle{remark}
\newtheorem*{case}{Case}
\newtheorem*{notation}{Notation}
\newtheorem*{remark}{Remark}
\newtheorem*{note}{Note}
\newtheorem*{problem}{Problem}
\newtheorem*{observe}{Observe}
\newtheorem*{property}{Property}
\newtheorem*{intuition}{Intuition}


% End example and intermezzo environments with a small diamond (just like proof
% environments end with a small square)
\usepackage{etoolbox}
\AtEndEnvironment{vb}{\null\hfill$\diamond$}%
\AtEndEnvironment{intermezzo}{\null\hfill$\diamond$}%
% \AtEndEnvironment{opmerking}{\null\hfill$\diamond$}%

% Fix some spacing
% http://tex.stackexchange.com/questions/22119/how-can-i-change-the-spacing-before-theorems-with-amsthm
\makeatletter
\def\thm@space@setup{%
  \thm@preskip=\parskip \thm@postskip=0pt
}

% Fix some stuff
% %http://tex.stackexchange.com/questions/76273/multiple-pdfs-with-page-group-included-in-a-single-page-warning
\pdfsuppresswarningpagegroup=1


% My name
\author{Jaden Wang}



\begin{document}
\begin{eg}[]
	Subgroups of $ S_3$: $ \{e, (1\ 2)\}, \{e,(1\ 3)\}, \{e,(2\ 3)\}, \{e\}, S_3, A_3$. The first three are not normal. The rest are normal. $ A_3$ is 1. index is two. 2. kernel of the sign homomorphism. 
\end{eg}

Let $ g \in G$, we have seen that if $ H \leq G$, then so is  $ gHg^{-1}$. Furthermore, the map $ \iota_g (x) = gxg^{-1}$ (injective map) is called conjugating by $ g$.  
\begin{prop}[]
	$ \iota_g$ is a group homomorphism (in fact an isomorphism) from  $ G$ to itself.
\end{prop}
\begin{prf}
\begin{align*}
	\iota_g(xy)=gxyg^{-1} = gxg^{-1}gyg^{-1}=\iota_g(x) \iota_g(y)
\end{align*}
\end{prf}
\begin{eg}[]
Isomorphism of $ V_4 \to V_4$. Identity needs to go to itself, but there are 3! different isomorphism. Then the group of isomorphisms are just $ S_3$. 

But since it's abelian, conjugation is trivial.
\end{eg}

\begin{claim}[]
	The inverse of $ \iota_g = \iota_{g^{-1}}$ (conjugation by  $ g^{-1}$ ).
\end{claim}

\begin{defn}[automorphism]
An isomorphism from $ G$ to itself is called an  \allbold{automorphism}. 
\end{defn}

\begin{defn}[inner automorphism]
Automorphisms that come from conjugation are called \allbold{inner automorphism}. 
\end{defn}
\begin{note}[]
Inner automorphism is a subgroup of the group of automorphisms
\end{note}

\begin{note}[]
	$ \iota_g(H)$ is a subgroup of  $ G$ because it is the image of  $ \iota_g$.
\end{note}
\begin{defn}[]
\[
H \simeq gH g^{-1}
.\] 
is conjugate subgroup.
\end{defn}

\begin{eg}[]
	$ G=S_3, H=\{e, (1\ 2)\}, g=(1\ 3) $. Then
\begin{align*}
	gHg^{-1} &= \{geg^{-1}, g(1\ 2)g^{-1}\}\\
		 &= \{e, (2\ 3)\}  \\
	gHg^{-1} &\simeq H \text{ but } H \neq gHg^{-1} 
\end{align*}
Which proves that $ H$ is not normal.
\end{eg}

\begin{thm}[]
If $ G$ is abelian, then  $ G / N$ is abelian.
\end{thm}
\begin{prf}
Let $ xN,yN \in G / N$. Then
\begin{align*}
	xN*yN= (xy) N = (yx) N = yN*xN
\end{align*}
Since $ x=y \implies xN = yN$.
\end{prf}
\begin{note}[]
The converse is false. Example is $ S_3 / A_3$. Or the trivial $ N$.  
\end{note}
\begin{note}[]
$ G / N$ is cyclic doesn't imply $ G$ is cyclic.
\end{note}

What is the order of $ xN \in G / N$? It is the smallest $ n>0$  such that $ (xN)^{n}=N \implies x^{n} N= eN \implies e^{-1}x^{n} \in N \implies x^{n} \in N$.

\begin{defn}[]
The order of a coset $ xN \in G / N$ is the smallest positive integer $ n$  such that $ x^{n} \in N$. 
\end{defn}

\begin{eg}[15.7]
$ G= \zz_4 \times \zz_6$. Order is 24. If $ G_1$ and $ G_2$ are abelian so is $ G_1 \times G_2$. If one group is not abelian, then the product isn't abelian. So $ G$ is abelian. $ G$ isn't cyclic since it isn't isomorphism to $ \zz_{24}$. 

$ H=\langle (0,1) \rangle = \{(0,0), (0,1),\ldots,(0,5)\} $. The order is 6. 

Is $ H$ normal in  $ \zz_4 \times \zz_6$? Yes because $ G$ is abelian.

Then  $ G / H$ is abelian with order 4. So it's either $ \zz_4$ or $ V_4$. We can show $ \zz_4$ if we find an element of order 4 (a generator). Then a coset looks like  $ (1,0)+ \langle (0,1) \rangle$. 

What is the order of that? 

First find all elements of $ N$. Then repeat operation on the representative of coset until it's in  $ N$.  It takes 4 steps to get $ (0,0) \in N$. Thus it has order 4.
\end{eg}
\begin{eg}[]
	$ \qq / \zz$ is an infinite group where every element has finite order.

	$ \rr / \qq$ is an infinite group that has no element of finite order apart from the identity.

	$ \rr / \zz \simeq U$. Since $ \phi: \rr \to C^*, r\mapsto e^{2\pi i r} $.
\end{eg}
\end{document}

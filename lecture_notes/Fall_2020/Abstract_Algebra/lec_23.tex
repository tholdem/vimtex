\documentclass[class=article,crop=false]{standalone} 
%Fall 2020
% Some basic packages
\usepackage{standalone}[subpreambles=true]
\usepackage[utf8]{inputenc}
\usepackage[T1]{fontenc}
\usepackage{textcomp}
\usepackage[english]{babel}
\usepackage{url}
\usepackage{graphicx}
\usepackage{float}
\usepackage{enumitem}


\pdfminorversion=7

% Don't indent paragraphs, leave some space between them
\usepackage{parskip}

% Hide page number when page is empty
\usepackage{emptypage}
\usepackage{subcaption}
\usepackage{multicol}
\usepackage[dvipsnames]{xcolor}


% Math stuff
\usepackage{amsmath, amsfonts, mathtools, amsthm, amssymb}
% Fancy script capitals
\usepackage{mathrsfs}
\usepackage{cancel}
% Bold math
\usepackage{bm}
% Some shortcuts
\newcommand{\rr}{\ensuremath{\mathbb{R}}}
\newcommand{\zz}{\ensuremath{\mathbb{Z}}}
\newcommand{\qq}{\ensuremath{\mathbb{Q}}}
\newcommand{\nn}{\ensuremath{\mathbb{N}}}
\newcommand{\ff}{\ensuremath{\mathbb{F}}}
\newcommand{\cc}{\ensuremath{\mathbb{C}}}
\renewcommand\O{\ensuremath{\emptyset}}
\newcommand{\norm}[1]{{\left\lVert{#1}\right\rVert}}
\renewcommand{\vec}[1]{{\mathbf{#1}}}
\newcommand\allbold[1]{{\boldmath\textbf{#1}}}

% Put x \to \infty below \lim
\let\svlim\lim\def\lim{\svlim\limits}

%Make implies and impliedby shorter
\let\implies\Rightarrow
\let\impliedby\Leftarrow
\let\iff\Leftrightarrow
\let\epsilon\varepsilon

% Add \contra symbol to denote contradiction
\usepackage{stmaryrd} % for \lightning
\newcommand\contra{\scalebox{1.5}{$\lightning$}}

% \let\phi\varphi

% Command for short corrections
% Usage: 1+1=\correct{3}{2}

\definecolor{correct}{HTML}{009900}
\newcommand\correct[2]{\ensuremath{\:}{\color{red}{#1}}\ensuremath{\to }{\color{correct}{#2}}\ensuremath{\:}}
\newcommand\green[1]{{\color{correct}{#1}}}

% horizontal rule
\newcommand\hr{
    \noindent\rule[0.5ex]{\linewidth}{0.5pt}
}

% hide parts
\newcommand\hide[1]{}

% si unitx
\usepackage{siunitx}
\sisetup{locale = FR}

% Environments
\makeatother
% For box around Definition, Theorem, \ldots
\usepackage[framemethod=TikZ]{mdframed}
\mdfsetup{skipabove=1em,skipbelow=0em}

%definition
\newenvironment{defn}[1][]{%
\ifstrempty{#1}%
{\mdfsetup{%
frametitle={%
\tikz[baseline=(current bounding box.east),outer sep=0pt]
\node[anchor=east,rectangle,fill=Emerald]
{\strut Definition};}}
}%
{\mdfsetup{%
frametitle={%
\tikz[baseline=(current bounding box.east),outer sep=0pt]
\node[anchor=east,rectangle,fill=Emerald]
{\strut Definition:~#1};}}%
}%
\mdfsetup{innertopmargin=10pt,linecolor=Emerald,%
linewidth=2pt,topline=true,%
frametitleaboveskip=\dimexpr-\ht\strutbox\relax
}
\begin{mdframed}[]\relax%
\label{#1}}{\end{mdframed}}


%theorem
%\newcounter{thm}[section]\setcounter{thm}{0}
%\renewcommand{\thethm}{\arabic{section}.\arabic{thm}}
\newenvironment{thm}[1][]{%
%\refstepcounter{thm}%
\ifstrempty{#1}%
{\mdfsetup{%
frametitle={%
\tikz[baseline=(current bounding box.east),outer sep=0pt]
\node[anchor=east,rectangle,fill=blue!20]
%{\strut Theorem~\thethm};}}
{\strut Theorem};}}
}%
{\mdfsetup{%
frametitle={%
\tikz[baseline=(current bounding box.east),outer sep=0pt]
\node[anchor=east,rectangle,fill=blue!20]
%{\strut Theorem~\thethm:~#1};}}%
{\strut Theorem:~#1};}}%
}%
\mdfsetup{innertopmargin=10pt,linecolor=blue!20,%
linewidth=2pt,topline=true,%
frametitleaboveskip=\dimexpr-\ht\strutbox\relax
}
\begin{mdframed}[]\relax%
\label{#1}}{\end{mdframed}}


%lemma
\newenvironment{lem}[1][]{%
\ifstrempty{#1}%
{\mdfsetup{%
frametitle={%
\tikz[baseline=(current bounding box.east),outer sep=0pt]
\node[anchor=east,rectangle,fill=Dandelion]
{\strut Lemma};}}
}%
{\mdfsetup{%
frametitle={%
\tikz[baseline=(current bounding box.east),outer sep=0pt]
\node[anchor=east,rectangle,fill=Dandelion]
{\strut Lemma:~#1};}}%
}%
\mdfsetup{innertopmargin=10pt,linecolor=Dandelion,%
linewidth=2pt,topline=true,%
frametitleaboveskip=\dimexpr-\ht\strutbox\relax
}
\begin{mdframed}[]\relax%
\label{#1}}{\end{mdframed}}

%corollary
\newenvironment{coro}[1][]{%
\ifstrempty{#1}%
{\mdfsetup{%
frametitle={%
\tikz[baseline=(current bounding box.east),outer sep=0pt]
\node[anchor=east,rectangle,fill=CornflowerBlue]
{\strut Corollary};}}
}%
{\mdfsetup{%
frametitle={%
\tikz[baseline=(current bounding box.east),outer sep=0pt]
\node[anchor=east,rectangle,fill=CornflowerBlue]
{\strut Corollary:~#1};}}%
}%
\mdfsetup{innertopmargin=10pt,linecolor=CornflowerBlue,%
linewidth=2pt,topline=true,%
frametitleaboveskip=\dimexpr-\ht\strutbox\relax
}
\begin{mdframed}[]\relax%
\label{#1}}{\end{mdframed}}

%proof
\newenvironment{prf}[1][]{%
\ifstrempty{#1}%
{\mdfsetup{%
frametitle={%
\tikz[baseline=(current bounding box.east),outer sep=0pt]
\node[anchor=east,rectangle,fill=SpringGreen]
{\strut Proof};}}
}%
{\mdfsetup{%
frametitle={%
\tikz[baseline=(current bounding box.east),outer sep=0pt]
\node[anchor=east,rectangle,fill=SpringGreen]
{\strut Proof:~#1};}}%
}%
\mdfsetup{innertopmargin=10pt,linecolor=SpringGreen,%
linewidth=2pt,topline=true,%
frametitleaboveskip=\dimexpr-\ht\strutbox\relax
}
\begin{mdframed}[]\relax%
\label{#1}}{\qed\end{mdframed}}


\theoremstyle{definition}

\newmdtheoremenv[nobreak=true]{definition}{Definition}
\newmdtheoremenv[nobreak=true]{prop}{Proposition}
\newmdtheoremenv[nobreak=true]{theorem}{Theorem}
\newmdtheoremenv[nobreak=true]{corollary}{Corollary}
\newtheorem*{eg}{Example}
\theoremstyle{remark}
\newtheorem*{case}{Case}
\newtheorem*{notation}{Notation}
\newtheorem*{remark}{Remark}
\newtheorem*{note}{Note}
\newtheorem*{problem}{Problem}
\newtheorem*{observe}{Observe}
\newtheorem*{property}{Property}
\newtheorem*{intuition}{Intuition}


% End example and intermezzo environments with a small diamond (just like proof
% environments end with a small square)
\usepackage{etoolbox}
\AtEndEnvironment{vb}{\null\hfill$\diamond$}%
\AtEndEnvironment{intermezzo}{\null\hfill$\diamond$}%
% \AtEndEnvironment{opmerking}{\null\hfill$\diamond$}%

% Fix some spacing
% http://tex.stackexchange.com/questions/22119/how-can-i-change-the-spacing-before-theorems-with-amsthm
\makeatletter
\def\thm@space@setup{%
  \thm@preskip=\parskip \thm@postskip=0pt
}

% Fix some stuff
% %http://tex.stackexchange.com/questions/76273/multiple-pdfs-with-page-group-included-in-a-single-page-warning
\pdfsuppresswarningpagegroup=1


% My name
\author{Jaden Wang}



\begin{document}
\begin{defn}[simple group]
A group is \allbold{simple} if it is nontrivial and it has no \allbold{NORMAL} subgroups other than itself and the trivial subgroup. 
\end{defn}

\begin{claim}[]
$ \zz_p$ is a simple group if $ p$ is prime. By Lagrange, there aren't any subgroups other than the trivial group and itself. These are the only abelian simple groups!

\end{claim}
\begin{thm}[15.15]
	The alternating groups $ A_n$ for $ n \geq 5$ are simple and (nonabelian).
\end{thm}
\begin{note}[]
	$ A_4$ has a normal subgroup of order 4: $ \{e, (1\ 2)(3\ 4),(1\ 3)(2\ 4),(1\ 4)(2\ 3)\} $
\end{note}
\begin{note}[]
	$ (1\ 2\ 3\ 4\ 5) \in A_5$, 5-cycles are even, $ (1\ 5)(1\ 4)(1\ 3)(1\ 2)$.  $ \langle (1\ 2\ 3\ 4\ 5) \rangle$ is a subgroup of $ A_5$ of order 5. $ A_5$ has many subgroups but only $ A_5, e$ are normal.
\end{note}

\begin{remark}
Finite simple groups were classified in 1981. The largest "sporadic group" is the Monster. It is NOT the largest nonabelian group because $ A_n$ can as large as we like. 
\end{remark}

\begin{defn}[center]
	Let $ G$ be a group. Then  \allbold{center} (zentrum) of $ G$,  $ Z(G)$, is the set of elements of $ G$ that commute with everything. That is,
	\[
		Z(G) = \{x \in G: xg=gx \ \forall \ g \in G\} 
	.\] 
\end{defn}
\begin{eg}[]
	$ Z(D_4) = \{ \text{ identity}, \text{ rotation by 180}\} $ 
	\begin{equation*}
		Z(D_n)=
	\begin{cases}
		\text{ trivial group if n is odd}\\
		\{e, \text{ identity, rotation by 180} \} \\ 
	\end{cases}
	\end{equation*}
\end{eg}
\begin{note}[]
	If $ G$ is abelian then  $ Z(G)=G$.
\end{note}
\begin{note}[]
	$ Z(S_n)$ if $ n\leq 3$ is trivial.
\end{note}
\begin{thm}[]
 \[
	 Z(G) \leq G
.\] 
\end{thm}
\begin{prf}
	$ e \in Z(G)$. If $ x \in Z(G)$ then $ x^{-1} \in Z(G)$.
	\begin{align*}
		xg&= gx \\
		x^{-1}xgx^{-1}&=x^{-1}gx x^{-1}\\
		gx^{-1}&= x^{-1}g \\
	\end{align*}
	If $ x,y \in Z(G)$, then show it's closed:
	\[
	xyg=xgy=gxy
	.\] 

	To show it's normal, let $ g \in G$ and $ x \in Z(G)$. Show $ gx^{-1}g^{-1} \in Z(G)$. Need to know it is a subgroup which we just proved. 
	\[
		gxg^{-1}= xg g^{-1}= x \in Z(G)
	.\] 
	Thus, $ Z(G) \trianglelefteq G$.
\end{prf}
\begin{eg}[]
	What is $ Z(A_5)$? It is $ \{e\} $ or $ A_5$ because $ A_5$ is simple and $ Z(A_5) \trianglelefteq A_5$. Since $ A_5$ is nonabelian, so $ Z(A_5) \neq A_5$, so $ Z(A_5) = \{e\} $.
\end{eg}
\begin{note}[]
If a nonabelian group, the center is definitely not itself.
\end{note}

\begin{defn}[commutator]
	A \allbold{commutator} is an element of for $ aba^{-1}b^{-1}$, $ [a,b]$. 
\end{defn}
\begin{note}[]
Inverse can either be on the left or right. It doesn't matter.
\end{note}

\begin{defn}[commutator subgroup]
	The \allbold{commutator subgroup} (derived subgroup) $ C(G)$, (or $ G'$) of $ G$, is the subgroup generated by the commutators. 
\end{defn}
\begin{note}[]
There is no guarantee that products of commutators are commutators.
\end{note}
\begin{thm}[]
\[
	C(G) \trianglelefteq G.
.\] 
\end{thm}

\begin{prf}
\[
	gaba^{-1}b^{-1}g^{-1} = gag^{-1}gbg^{-1}ga^{-1}g^{-1}gb^{-1}g^{-1} = [gag^{-1},gbg^{-1}]
.\] 
\end{prf}
\begin{thm}[IMPORTANT]
	The commutator subgroup, $ C(G)$ of  $ G$, is the smallest normal subgroup with abelian quotient. That is, $ G /C(G)$ is abelian, and if  $ N \trianglelefteq G $ with $ G /N$ abelian, then  $ C(G) \leq N$.
\end{thm}
\begin{eg}[]
$ G=D_4$, normal subgroups of $ G$:  $ \{ \text{ rotations} \} $ because index is 2, $ \{e\}, D_4, \{e, \text{ rotation by 180} \}$ because it's the center .

\begin{itemize}
	\item $ D_4 /D_4$ is trivial group. Abelian
	\item $ D_4 / \{e\} $ is isomorphic to $ D_4$. Nonabelian.
	\item $ D_4 / Z(G)$ has order 4, it's in fact $ V_4$. So abelian.
	\item $ D_4 / \{ \text{ rotations} \} $ has order 2, so isomorphic to $ \zz_2$ abelian.
\end{itemize}
Now among the three abelian ones, which is the smallest? Guess: $ C(D_4) = \{e, \text{ rotations by 180} \} $. 

Let's check. Normal? Yes because it's the center.

Abelian quotient? Yes because has order 4.

Smallest? If not, the smallest would have to be one of its subgroups. The subgroup of this is just the identity. But identity doesn't work as it yields nonabelian group.
\end{eg}
\begin{intuition}
	What if $ [a,b]=e$?
	\[
		aba^{-1}b^{-1} = e \implies aba^{-1}=b \implies ab=ba
	.\] 
\end{intuition}
\begin{note}[]
	If $ G$ is abelian,  $ C(G) = \{e\} $ and vice versa. A group is abelian iff it's center is the whole thing.
\end{note}

\begin{eg}[]
	What is $ C(A_5)$? It's $ A_5$ since $ \{e\} $ yields $ A_5$ which is nonabelian.
\end{eg}



\end{document}

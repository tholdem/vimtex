\documentclass[class=article,crop=false]{standalone} 
%Fall 2020
% Some basic packages
\usepackage{standalone}[subpreambles=true]
\usepackage[utf8]{inputenc}
\usepackage[T1]{fontenc}
\usepackage{textcomp}
\usepackage[english]{babel}
\usepackage{url}
\usepackage{graphicx}
\usepackage{float}
\usepackage{enumitem}


\pdfminorversion=7

% Don't indent paragraphs, leave some space between them
\usepackage{parskip}

% Hide page number when page is empty
\usepackage{emptypage}
\usepackage{subcaption}
\usepackage{multicol}
\usepackage[dvipsnames]{xcolor}


% Math stuff
\usepackage{amsmath, amsfonts, mathtools, amsthm, amssymb}
% Fancy script capitals
\usepackage{mathrsfs}
\usepackage{cancel}
% Bold math
\usepackage{bm}
% Some shortcuts
\newcommand{\rr}{\ensuremath{\mathbb{R}}}
\newcommand{\zz}{\ensuremath{\mathbb{Z}}}
\newcommand{\qq}{\ensuremath{\mathbb{Q}}}
\newcommand{\nn}{\ensuremath{\mathbb{N}}}
\newcommand{\ff}{\ensuremath{\mathbb{F}}}
\newcommand{\cc}{\ensuremath{\mathbb{C}}}
\renewcommand\O{\ensuremath{\emptyset}}
\newcommand{\norm}[1]{{\left\lVert{#1}\right\rVert}}
\renewcommand{\vec}[1]{{\mathbf{#1}}}
\newcommand\allbold[1]{{\boldmath\textbf{#1}}}

% Put x \to \infty below \lim
\let\svlim\lim\def\lim{\svlim\limits}

%Make implies and impliedby shorter
\let\implies\Rightarrow
\let\impliedby\Leftarrow
\let\iff\Leftrightarrow
\let\epsilon\varepsilon

% Add \contra symbol to denote contradiction
\usepackage{stmaryrd} % for \lightning
\newcommand\contra{\scalebox{1.5}{$\lightning$}}

% \let\phi\varphi

% Command for short corrections
% Usage: 1+1=\correct{3}{2}

\definecolor{correct}{HTML}{009900}
\newcommand\correct[2]{\ensuremath{\:}{\color{red}{#1}}\ensuremath{\to }{\color{correct}{#2}}\ensuremath{\:}}
\newcommand\green[1]{{\color{correct}{#1}}}

% horizontal rule
\newcommand\hr{
    \noindent\rule[0.5ex]{\linewidth}{0.5pt}
}

% hide parts
\newcommand\hide[1]{}

% si unitx
\usepackage{siunitx}
\sisetup{locale = FR}

% Environments
\makeatother
% For box around Definition, Theorem, \ldots
\usepackage[framemethod=TikZ]{mdframed}
\mdfsetup{skipabove=1em,skipbelow=0em}

%definition
\newenvironment{defn}[1][]{%
\ifstrempty{#1}%
{\mdfsetup{%
frametitle={%
\tikz[baseline=(current bounding box.east),outer sep=0pt]
\node[anchor=east,rectangle,fill=Emerald]
{\strut Definition};}}
}%
{\mdfsetup{%
frametitle={%
\tikz[baseline=(current bounding box.east),outer sep=0pt]
\node[anchor=east,rectangle,fill=Emerald]
{\strut Definition:~#1};}}%
}%
\mdfsetup{innertopmargin=10pt,linecolor=Emerald,%
linewidth=2pt,topline=true,%
frametitleaboveskip=\dimexpr-\ht\strutbox\relax
}
\begin{mdframed}[]\relax%
\label{#1}}{\end{mdframed}}


%theorem
%\newcounter{thm}[section]\setcounter{thm}{0}
%\renewcommand{\thethm}{\arabic{section}.\arabic{thm}}
\newenvironment{thm}[1][]{%
%\refstepcounter{thm}%
\ifstrempty{#1}%
{\mdfsetup{%
frametitle={%
\tikz[baseline=(current bounding box.east),outer sep=0pt]
\node[anchor=east,rectangle,fill=blue!20]
%{\strut Theorem~\thethm};}}
{\strut Theorem};}}
}%
{\mdfsetup{%
frametitle={%
\tikz[baseline=(current bounding box.east),outer sep=0pt]
\node[anchor=east,rectangle,fill=blue!20]
%{\strut Theorem~\thethm:~#1};}}%
{\strut Theorem:~#1};}}%
}%
\mdfsetup{innertopmargin=10pt,linecolor=blue!20,%
linewidth=2pt,topline=true,%
frametitleaboveskip=\dimexpr-\ht\strutbox\relax
}
\begin{mdframed}[]\relax%
\label{#1}}{\end{mdframed}}


%lemma
\newenvironment{lem}[1][]{%
\ifstrempty{#1}%
{\mdfsetup{%
frametitle={%
\tikz[baseline=(current bounding box.east),outer sep=0pt]
\node[anchor=east,rectangle,fill=Dandelion]
{\strut Lemma};}}
}%
{\mdfsetup{%
frametitle={%
\tikz[baseline=(current bounding box.east),outer sep=0pt]
\node[anchor=east,rectangle,fill=Dandelion]
{\strut Lemma:~#1};}}%
}%
\mdfsetup{innertopmargin=10pt,linecolor=Dandelion,%
linewidth=2pt,topline=true,%
frametitleaboveskip=\dimexpr-\ht\strutbox\relax
}
\begin{mdframed}[]\relax%
\label{#1}}{\end{mdframed}}

%corollary
\newenvironment{coro}[1][]{%
\ifstrempty{#1}%
{\mdfsetup{%
frametitle={%
\tikz[baseline=(current bounding box.east),outer sep=0pt]
\node[anchor=east,rectangle,fill=CornflowerBlue]
{\strut Corollary};}}
}%
{\mdfsetup{%
frametitle={%
\tikz[baseline=(current bounding box.east),outer sep=0pt]
\node[anchor=east,rectangle,fill=CornflowerBlue]
{\strut Corollary:~#1};}}%
}%
\mdfsetup{innertopmargin=10pt,linecolor=CornflowerBlue,%
linewidth=2pt,topline=true,%
frametitleaboveskip=\dimexpr-\ht\strutbox\relax
}
\begin{mdframed}[]\relax%
\label{#1}}{\end{mdframed}}

%proof
\newenvironment{prf}[1][]{%
\ifstrempty{#1}%
{\mdfsetup{%
frametitle={%
\tikz[baseline=(current bounding box.east),outer sep=0pt]
\node[anchor=east,rectangle,fill=SpringGreen]
{\strut Proof};}}
}%
{\mdfsetup{%
frametitle={%
\tikz[baseline=(current bounding box.east),outer sep=0pt]
\node[anchor=east,rectangle,fill=SpringGreen]
{\strut Proof:~#1};}}%
}%
\mdfsetup{innertopmargin=10pt,linecolor=SpringGreen,%
linewidth=2pt,topline=true,%
frametitleaboveskip=\dimexpr-\ht\strutbox\relax
}
\begin{mdframed}[]\relax%
\label{#1}}{\qed\end{mdframed}}


\theoremstyle{definition}

\newmdtheoremenv[nobreak=true]{definition}{Definition}
\newmdtheoremenv[nobreak=true]{prop}{Proposition}
\newmdtheoremenv[nobreak=true]{theorem}{Theorem}
\newmdtheoremenv[nobreak=true]{corollary}{Corollary}
\newtheorem*{eg}{Example}
\theoremstyle{remark}
\newtheorem*{case}{Case}
\newtheorem*{notation}{Notation}
\newtheorem*{remark}{Remark}
\newtheorem*{note}{Note}
\newtheorem*{problem}{Problem}
\newtheorem*{observe}{Observe}
\newtheorem*{property}{Property}
\newtheorem*{intuition}{Intuition}


% End example and intermezzo environments with a small diamond (just like proof
% environments end with a small square)
\usepackage{etoolbox}
\AtEndEnvironment{vb}{\null\hfill$\diamond$}%
\AtEndEnvironment{intermezzo}{\null\hfill$\diamond$}%
% \AtEndEnvironment{opmerking}{\null\hfill$\diamond$}%

% Fix some spacing
% http://tex.stackexchange.com/questions/22119/how-can-i-change-the-spacing-before-theorems-with-amsthm
\makeatletter
\def\thm@space@setup{%
  \thm@preskip=\parskip \thm@postskip=0pt
}

% Fix some stuff
% %http://tex.stackexchange.com/questions/76273/multiple-pdfs-with-page-group-included-in-a-single-page-warning
\pdfsuppresswarningpagegroup=1


% My name
\author{Jaden Wang}



\begin{document}
\begin{intuition}
	For linear map T, $ T$ is injective iff  $ \ker T= \{0\} $
\end{intuition}


\begin{thm}[]
If $ \phi:G \to H$ homomorphism, then
$ \phi$ is injective $ \iff \ker  \phi = \{e_G\} $
\end{thm}

\begin{prf}
Let's prove the contrapositives:

($ \impliedby$):If $ \ker \phi \neq \{e_G\} $, then there exists $ x \in G \setminus \{e_G\} $ such that $ \phi(x)=e_H$. Since $ e_G \neq x$ and  $ \phi(x) = \phi(e_G) =e_H$, this implies $ \phi$ is not injective.

($ \implies$): We want to use the same idea from linear algebra. If $ \phi$ is not injective, then there exist $ x \neq y$ with  $ \phi(x)=\phi(y)$. Then
\begin{align*}
	\phi(x^{-1} y) &= \phi(x^{-1}) \phi(y) \\
		       &= \phi(x)^{-1} \phi(y) \\
		       &= e_H 
\end{align*}
However, $ x^{-1} y \neq e_G$ since $ x\neq y$. Hence,  $ e_G$ and  $ x^{-1}y$ are distinct elements of $ \ker \phi \implies \ker \neq \{e_G\} $.
\end{prf}

\begin{eg}[]
Read differentiation example in textbook.
\end{eg}

\begin{eg}[]
	$ \phi: \cc^* \to \rr*$, $ \phi(z)=|z|$. We claim this is a homomorphism. 
	\[
		\phi(z *_{ \cc} w) = |zw| = |z||w| = \phi(z) *_{ \rr} \phi(w)
	.\]
$ \im \phi = \rr^+$. This is a subgroup because it is the image of a known homomorphism!

$ \ker \phi = U$, the unit circle $ \{z \in \cc^* : |z|=1\} $. Hence $ U$ is a normal subgroup of  $ \cc^* $. It follows from that the subgroup of an abelian group is normal. Or it's the kernel of a known homomorphism.
\end{eg}

\begin{note}[]
	T/F: If $ H \leq G $ and  $ H$ is abelian, is  $ H$ a normal subgroup? NO, think  $ \{e, (1\ 2)\} \leq S_3 $.
\end{note}

\begin{notation}
	$ \pi$ stands for surjective/projection. $ \iota$ stands for injective.
\end{notation}
\begin{eg}[]
	$ \pi_1: G_1 \times G_2 \to G_1$, $ \pi((g_1,g_2))=g_1$. "Projection to first component".
	\begin{align*}
		\pi_1((x_1,x_2)*(y_1,y_2)) &= \pi_1((x_1*y_1,x_2*y_2)) \\
					   &= (x_1*y_1) \\
					   &= \phi((x_1,x_2)) \phi((y_1,y_2)) 
	\end{align*}
	$ \im \pi_1 = G_1$. 

	$ \ker \pi_1  = \{(g_1,g_2): g_1=e_1\} $. $ K \trianglelefteq G_1\times G_2$.
\end{eg}

\section{Factor/Quotient Groups}
\begin{note}[]
	Quotient groups are NOT a special kind of subgroup. We try to find a group of cosets. See screenshot for the motivation. For some cases, if you call the inputs by different names, we obtain the same result.

	Well-defined: the operation is not confused by picking different representations for the inputs.
\end{note}

Goal: $ H \leq G$, try to make the left cosets of  $ H $ in  $ G$ into a group. That is, $ (xH)*(yH) = xyH$. Here we should think of each coset as a single object.

Problem: the result seems to depend on the representatives  $ x,y$ that were chosen. If $ x,x'$ in the same left coset, $ xH=x'H \iff h=x^{-1}x' \in H, x'=xh$. So $ xH=xhH$. Then
\[
	(xhH)*(yh'H) = xhyh'H
.\] 
We need $ xhyh'H $ and  $ xyH$ to be the representations of the same things for all $ x,y,h,h'$. Again this means
\begin{align*}
	xyH = xhyh'H &\iff (xy)^{-1} xhyh'H \in H\\
	& \iff y^{-1}x^{-1}xhyh'H \in H \\
	& \iff y^{-1}hyh' =h_0 \in H \\
	& \iff y^{-1}hy = h_0(h')^{-1} \in H  
\end{align*}
Summary: this would work iff $ y^{-1}hy \in H$ for all $ h \in H, y \in G$.



\end{document}

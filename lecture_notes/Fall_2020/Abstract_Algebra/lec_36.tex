\documentclass[class=article,crop=false]{standalone} 
%Fall 2020
% Some basic packages
\usepackage{standalone}[subpreambles=true]
\usepackage[utf8]{inputenc}
\usepackage[T1]{fontenc}
\usepackage{textcomp}
\usepackage[english]{babel}
\usepackage{url}
\usepackage{graphicx}
\usepackage{float}
\usepackage{enumitem}


\pdfminorversion=7

% Don't indent paragraphs, leave some space between them
\usepackage{parskip}

% Hide page number when page is empty
\usepackage{emptypage}
\usepackage{subcaption}
\usepackage{multicol}
\usepackage[dvipsnames]{xcolor}


% Math stuff
\usepackage{amsmath, amsfonts, mathtools, amsthm, amssymb}
% Fancy script capitals
\usepackage{mathrsfs}
\usepackage{cancel}
% Bold math
\usepackage{bm}
% Some shortcuts
\newcommand{\rr}{\ensuremath{\mathbb{R}}}
\newcommand{\zz}{\ensuremath{\mathbb{Z}}}
\newcommand{\qq}{\ensuremath{\mathbb{Q}}}
\newcommand{\nn}{\ensuremath{\mathbb{N}}}
\newcommand{\ff}{\ensuremath{\mathbb{F}}}
\newcommand{\cc}{\ensuremath{\mathbb{C}}}
\renewcommand\O{\ensuremath{\emptyset}}
\newcommand{\norm}[1]{{\left\lVert{#1}\right\rVert}}
\renewcommand{\vec}[1]{{\mathbf{#1}}}
\newcommand\allbold[1]{{\boldmath\textbf{#1}}}

% Put x \to \infty below \lim
\let\svlim\lim\def\lim{\svlim\limits}

%Make implies and impliedby shorter
\let\implies\Rightarrow
\let\impliedby\Leftarrow
\let\iff\Leftrightarrow
\let\epsilon\varepsilon

% Add \contra symbol to denote contradiction
\usepackage{stmaryrd} % for \lightning
\newcommand\contra{\scalebox{1.5}{$\lightning$}}

% \let\phi\varphi

% Command for short corrections
% Usage: 1+1=\correct{3}{2}

\definecolor{correct}{HTML}{009900}
\newcommand\correct[2]{\ensuremath{\:}{\color{red}{#1}}\ensuremath{\to }{\color{correct}{#2}}\ensuremath{\:}}
\newcommand\green[1]{{\color{correct}{#1}}}

% horizontal rule
\newcommand\hr{
    \noindent\rule[0.5ex]{\linewidth}{0.5pt}
}

% hide parts
\newcommand\hide[1]{}

% si unitx
\usepackage{siunitx}
\sisetup{locale = FR}

% Environments
\makeatother
% For box around Definition, Theorem, \ldots
\usepackage[framemethod=TikZ]{mdframed}
\mdfsetup{skipabove=1em,skipbelow=0em}

%definition
\newenvironment{defn}[1][]{%
\ifstrempty{#1}%
{\mdfsetup{%
frametitle={%
\tikz[baseline=(current bounding box.east),outer sep=0pt]
\node[anchor=east,rectangle,fill=Emerald]
{\strut Definition};}}
}%
{\mdfsetup{%
frametitle={%
\tikz[baseline=(current bounding box.east),outer sep=0pt]
\node[anchor=east,rectangle,fill=Emerald]
{\strut Definition:~#1};}}%
}%
\mdfsetup{innertopmargin=10pt,linecolor=Emerald,%
linewidth=2pt,topline=true,%
frametitleaboveskip=\dimexpr-\ht\strutbox\relax
}
\begin{mdframed}[]\relax%
\label{#1}}{\end{mdframed}}


%theorem
%\newcounter{thm}[section]\setcounter{thm}{0}
%\renewcommand{\thethm}{\arabic{section}.\arabic{thm}}
\newenvironment{thm}[1][]{%
%\refstepcounter{thm}%
\ifstrempty{#1}%
{\mdfsetup{%
frametitle={%
\tikz[baseline=(current bounding box.east),outer sep=0pt]
\node[anchor=east,rectangle,fill=blue!20]
%{\strut Theorem~\thethm};}}
{\strut Theorem};}}
}%
{\mdfsetup{%
frametitle={%
\tikz[baseline=(current bounding box.east),outer sep=0pt]
\node[anchor=east,rectangle,fill=blue!20]
%{\strut Theorem~\thethm:~#1};}}%
{\strut Theorem:~#1};}}%
}%
\mdfsetup{innertopmargin=10pt,linecolor=blue!20,%
linewidth=2pt,topline=true,%
frametitleaboveskip=\dimexpr-\ht\strutbox\relax
}
\begin{mdframed}[]\relax%
\label{#1}}{\end{mdframed}}


%lemma
\newenvironment{lem}[1][]{%
\ifstrempty{#1}%
{\mdfsetup{%
frametitle={%
\tikz[baseline=(current bounding box.east),outer sep=0pt]
\node[anchor=east,rectangle,fill=Dandelion]
{\strut Lemma};}}
}%
{\mdfsetup{%
frametitle={%
\tikz[baseline=(current bounding box.east),outer sep=0pt]
\node[anchor=east,rectangle,fill=Dandelion]
{\strut Lemma:~#1};}}%
}%
\mdfsetup{innertopmargin=10pt,linecolor=Dandelion,%
linewidth=2pt,topline=true,%
frametitleaboveskip=\dimexpr-\ht\strutbox\relax
}
\begin{mdframed}[]\relax%
\label{#1}}{\end{mdframed}}

%corollary
\newenvironment{coro}[1][]{%
\ifstrempty{#1}%
{\mdfsetup{%
frametitle={%
\tikz[baseline=(current bounding box.east),outer sep=0pt]
\node[anchor=east,rectangle,fill=CornflowerBlue]
{\strut Corollary};}}
}%
{\mdfsetup{%
frametitle={%
\tikz[baseline=(current bounding box.east),outer sep=0pt]
\node[anchor=east,rectangle,fill=CornflowerBlue]
{\strut Corollary:~#1};}}%
}%
\mdfsetup{innertopmargin=10pt,linecolor=CornflowerBlue,%
linewidth=2pt,topline=true,%
frametitleaboveskip=\dimexpr-\ht\strutbox\relax
}
\begin{mdframed}[]\relax%
\label{#1}}{\end{mdframed}}

%proof
\newenvironment{prf}[1][]{%
\ifstrempty{#1}%
{\mdfsetup{%
frametitle={%
\tikz[baseline=(current bounding box.east),outer sep=0pt]
\node[anchor=east,rectangle,fill=SpringGreen]
{\strut Proof};}}
}%
{\mdfsetup{%
frametitle={%
\tikz[baseline=(current bounding box.east),outer sep=0pt]
\node[anchor=east,rectangle,fill=SpringGreen]
{\strut Proof:~#1};}}%
}%
\mdfsetup{innertopmargin=10pt,linecolor=SpringGreen,%
linewidth=2pt,topline=true,%
frametitleaboveskip=\dimexpr-\ht\strutbox\relax
}
\begin{mdframed}[]\relax%
\label{#1}}{\qed\end{mdframed}}


\theoremstyle{definition}

\newmdtheoremenv[nobreak=true]{definition}{Definition}
\newmdtheoremenv[nobreak=true]{prop}{Proposition}
\newmdtheoremenv[nobreak=true]{theorem}{Theorem}
\newmdtheoremenv[nobreak=true]{corollary}{Corollary}
\newtheorem*{eg}{Example}
\theoremstyle{remark}
\newtheorem*{case}{Case}
\newtheorem*{notation}{Notation}
\newtheorem*{remark}{Remark}
\newtheorem*{note}{Note}
\newtheorem*{problem}{Problem}
\newtheorem*{observe}{Observe}
\newtheorem*{property}{Property}
\newtheorem*{intuition}{Intuition}


% End example and intermezzo environments with a small diamond (just like proof
% environments end with a small square)
\usepackage{etoolbox}
\AtEndEnvironment{vb}{\null\hfill$\diamond$}%
\AtEndEnvironment{intermezzo}{\null\hfill$\diamond$}%
% \AtEndEnvironment{opmerking}{\null\hfill$\diamond$}%

% Fix some spacing
% http://tex.stackexchange.com/questions/22119/how-can-i-change-the-spacing-before-theorems-with-amsthm
\makeatletter
\def\thm@space@setup{%
  \thm@preskip=\parskip \thm@postskip=0pt
}

% Fix some stuff
% %http://tex.stackexchange.com/questions/76273/multiple-pdfs-with-page-group-included-in-a-single-page-warning
\pdfsuppresswarningpagegroup=1


% My name
\author{Jaden Wang}



\begin{document}
\begin{thm}[uniqueness of factorization]
	Let $ F$ be a field and let  $ f(x) \in F[x]$ be a non-constant polynomial. Then we can express $ f(x)$ as a product of irreducible polynomials
	 \[
		 f(x)=p_1(x) p_2(x) \ldots p_r(x)
	,\]
	unique up to changing order and multiplication by units.
\end{thm}

\begin{prop}[]
	In $ \rr[x]$, all irreducible polynomials have degree 1 or 2. 
\end{prop}
\begin{note}[]
 In $ \cc[x]$, all irreducible polynomials have degree 1.
\end{note}

\begin{prop}[]
	If $ \alpha \in \cc$ is a root of $ f(x) \in \rr[x]$, then so is $ \overline{\alpha}$.
\end{prop}

\begin{note}[]
Sum and product of pair of conjugates are real. That is,
\[
	(x-\alpha)(x- \overline{\alpha}) = x^2 - (\alpha + \overline{\alpha})x + \alpha \overline{\alpha}
.\] 
This can help us find roots in $ \cc[x]$.
\end{note}

\begin{remark}
It's better to work with monic polynomials and ignore multiply by units.
\end{remark}


\begin{thm}[23.11: Gauss's Lemma special case]
	Let $ f(x) \in \qq[x]$ (but with integer coefficients). If $ f(x)=g(x)h(x)$, where  $ g,h \in q[x]$ with lower degrees, then it is possible to factor $ f(x) = a(x)b(x)$ with  $ a(x),b(x) \in \zz[x]$ with lower degrees.
\end{thm}
\begin{eg}[]
	$ x^{4} + 1 \in \qq[x]$ is irreducible. Reduce to degree 2 in $ \rr[x]$ and to degree 1 in $ \cc[x]$. In general for degree 4 polynomial, we can have irreducible quartic, irreducible cubic+linear, irreducible quadratic, 1 quadratic two linear, and 4 linear.
\end{eg}
\begin{eg}[]
	Consider $ x^2-5x+6 \in \qq[x]$. The lemma ensures that we can factor into $ (x-2)(x-3)$ with integer coefficients.
\end{eg}
\begin{eg}[]
	There is a quick way to show $ x^{4}+1 \in \qq[x]$ is irreducible using Gauss's lemma.

	If $ x^{4}+1$ can be factorized, it can be factorized over $ \zz$.

	\begin{align*}
		x^{4}+1 &= (x^2+ax\pm 1)(x^2 -ax \pm 1) \text{ since } a+b=0 \\
			&= x^2 \pm (2\mp a^2) x + 1 \\
	\end{align*}
	But either case wouldn't work because $ 2-a^2 \neq 0, -2-a^2\neq 0$ since coefficient of $ x^3$, $ a$, is zero.
\end{eg}

\begin{thm}[]
	Let $ f(x) \in \qq[x]$ and coefficients are integers (we can always obtain this by multiplying by units to find roots). Suppose
	\[
		f(x) = a_n x^{n}+ a_{n-1}x^{n-1}+ \ldots+ a_1 x + a_0
	.\] 
	Suppose $ \frac{p}{q} \in f(x)$ is a root in $ \qq$ of $ f(x)$ and that  $ \frac{p}{q}$ is in lowest terms \emph{i.e.} $ \gcd ( p,q)=1 $. Then the numerator of the root divides the constant term and the denominator of the root divides the leading coefficient.
\end{thm}

\begin{prf}
\begin{align*}
	a_n \left( \frac{p}{q} \right) ^{n} + \ldots + a_0 &= 0 \\
	a_n p^{n} + a_{n-1}p^{n-1}q+ \ldots + a_1 p q^{n-1} + a_0 q^{n}&= 0 \text{ multiply by }q^{n}
\end{align*}
1st term is a multiple of q since everything else is multiple of q. Similarly, last term is a multiple of $ p$ since everything else is a multiple of  $ p$. So  $ q / a_n p^{n}$, since $ \gcd ( q,p)=1 \implies \gcd ( (q,p^{n})=1 $. 
\begin{claim}[]
If $ a / bc$ and  $ \gcd ( a,b) =1$, then $ a /c$.
\end{claim}
So we have $ q / a_n$. Similarly, $ \gcd ( (p,q^{n}), p / a_0 q^{n} \implies p / a_0$.
\end{prf}

\begin{eg}[]
	$ 3x^{3}-4x+6 \in \qq[x]$. Prove this is irreducible. We can think of roots because it has degree 3.

	Suppose $ \frac{p}{q} \in \qq$ is a root. Then $ q /3, p /6, \gcd ( p,q)=1 $. Then $ q \in \{\pm 1,\pm 3 \} $ but we can assume $ q>0$. And  $ p \in \{\pm 1,\pm 2,\pm 3,\pm 6\} $. So the candidates for roots are
	\[
	\pm 1, \pm 2, \pm 3, \pm 6, \pm \frac{1}{3}, \pm \frac{2}{3}
	.\] 
\end{eg}

~\begin{thm}[Eisenstein Criterion]
	Let $ f(x) \in \qq[x]$ with integer coefficients:
	\[
		f(x) = a_n x^{n}+ \ldots + a_1 x + a_0
	.\] 
	If there exists a prime such that $ p $ doesn't divide  $ a_n$, $ p^2$ doesn't divide $ a_0$, but $ p$ divides every other coefficients, then  $ f(x) $ is irreducible over  $ \qq[x]$. 
\end{thm}
\begin{note}[]
Eisenstein works for any degree.
\end{note}
\begin{eg}[]
Using Eisenstein for the above example, we can try $ p=2$ and it works.
\end{eg}

\begin{eg}[]
$ 25x^{5} - 9x^{4} - 3x^2-12$. Take $ p=3$ and it works so it's irreducible. "It's Eisenstein by  $ p=3 $.
\end{eg}

\begin{eg}[]
	$ x^{4}+x^3+ x^2 + x +1 \in \qq[x]$ is irreducible. This equals to $ \frac{x^{5}-1}{x-1 }$. Change $ x-1$ to  $ y$, so  $ x $ becomes  $ y+1$. And
	 \begin{align*}
		 \frac{x^{5}-1}{x-1 }&= \frac{(y+1)^{5}-1}{y } \\
		 &= y^{4}+ 5 y^3 + 10 y^2 + 10 y + 5 \text{ using binomial theorem} 
	\end{align*}
	This works because if $ p$ is prime, then the  $ p$th line of Pascal triangle are all multiples of  $ p$.
\end{eg}

\begin{thm}[]
	$ x^{p-1}+ x^{p-2}+ \ldots+ x+1 \in \qq[x]$ is irreducible for $ p$ prime.
\end{thm}

\begin{eg}[]
\[
	\frac{x^{6}-1}{x-1 } = x^{5} + x^{4}+ x^{3}+ x^{2}+ x+ 1 = (x+1) (x^{4}+ x^2 + 1)
.\] 
Since we can always group them into two. This doesn't work.
\end{eg}

\begin{claim}[]
	Over $ \rr$, there is no irreducible polynomials of degree $ \geq 3$. For odd degree it's because of Calculus. For even degree we use complex conjugate, so two linear factors of complex conjugates already give us a degree two polynomial in $ \rr[x]$, and any even degree $ \geq 4$ would have some degree 2 polynomials as factors if we consider the complex roots.
\end{claim}
\end{document}

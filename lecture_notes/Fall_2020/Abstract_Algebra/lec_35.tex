\documentclass[class=article,crop=false]{standalone} 
%Fall 2020
% Some basic packages
\usepackage{standalone}[subpreambles=true]
\usepackage[utf8]{inputenc}
\usepackage[T1]{fontenc}
\usepackage{textcomp}
\usepackage[english]{babel}
\usepackage{url}
\usepackage{graphicx}
\usepackage{float}
\usepackage{enumitem}


\pdfminorversion=7

% Don't indent paragraphs, leave some space between them
\usepackage{parskip}

% Hide page number when page is empty
\usepackage{emptypage}
\usepackage{subcaption}
\usepackage{multicol}
\usepackage[dvipsnames]{xcolor}


% Math stuff
\usepackage{amsmath, amsfonts, mathtools, amsthm, amssymb}
% Fancy script capitals
\usepackage{mathrsfs}
\usepackage{cancel}
% Bold math
\usepackage{bm}
% Some shortcuts
\newcommand{\rr}{\ensuremath{\mathbb{R}}}
\newcommand{\zz}{\ensuremath{\mathbb{Z}}}
\newcommand{\qq}{\ensuremath{\mathbb{Q}}}
\newcommand{\nn}{\ensuremath{\mathbb{N}}}
\newcommand{\ff}{\ensuremath{\mathbb{F}}}
\newcommand{\cc}{\ensuremath{\mathbb{C}}}
\renewcommand\O{\ensuremath{\emptyset}}
\newcommand{\norm}[1]{{\left\lVert{#1}\right\rVert}}
\renewcommand{\vec}[1]{{\mathbf{#1}}}
\newcommand\allbold[1]{{\boldmath\textbf{#1}}}

% Put x \to \infty below \lim
\let\svlim\lim\def\lim{\svlim\limits}

%Make implies and impliedby shorter
\let\implies\Rightarrow
\let\impliedby\Leftarrow
\let\iff\Leftrightarrow
\let\epsilon\varepsilon

% Add \contra symbol to denote contradiction
\usepackage{stmaryrd} % for \lightning
\newcommand\contra{\scalebox{1.5}{$\lightning$}}

% \let\phi\varphi

% Command for short corrections
% Usage: 1+1=\correct{3}{2}

\definecolor{correct}{HTML}{009900}
\newcommand\correct[2]{\ensuremath{\:}{\color{red}{#1}}\ensuremath{\to }{\color{correct}{#2}}\ensuremath{\:}}
\newcommand\green[1]{{\color{correct}{#1}}}

% horizontal rule
\newcommand\hr{
    \noindent\rule[0.5ex]{\linewidth}{0.5pt}
}

% hide parts
\newcommand\hide[1]{}

% si unitx
\usepackage{siunitx}
\sisetup{locale = FR}

% Environments
\makeatother
% For box around Definition, Theorem, \ldots
\usepackage[framemethod=TikZ]{mdframed}
\mdfsetup{skipabove=1em,skipbelow=0em}

%definition
\newenvironment{defn}[1][]{%
\ifstrempty{#1}%
{\mdfsetup{%
frametitle={%
\tikz[baseline=(current bounding box.east),outer sep=0pt]
\node[anchor=east,rectangle,fill=Emerald]
{\strut Definition};}}
}%
{\mdfsetup{%
frametitle={%
\tikz[baseline=(current bounding box.east),outer sep=0pt]
\node[anchor=east,rectangle,fill=Emerald]
{\strut Definition:~#1};}}%
}%
\mdfsetup{innertopmargin=10pt,linecolor=Emerald,%
linewidth=2pt,topline=true,%
frametitleaboveskip=\dimexpr-\ht\strutbox\relax
}
\begin{mdframed}[]\relax%
\label{#1}}{\end{mdframed}}


%theorem
%\newcounter{thm}[section]\setcounter{thm}{0}
%\renewcommand{\thethm}{\arabic{section}.\arabic{thm}}
\newenvironment{thm}[1][]{%
%\refstepcounter{thm}%
\ifstrempty{#1}%
{\mdfsetup{%
frametitle={%
\tikz[baseline=(current bounding box.east),outer sep=0pt]
\node[anchor=east,rectangle,fill=blue!20]
%{\strut Theorem~\thethm};}}
{\strut Theorem};}}
}%
{\mdfsetup{%
frametitle={%
\tikz[baseline=(current bounding box.east),outer sep=0pt]
\node[anchor=east,rectangle,fill=blue!20]
%{\strut Theorem~\thethm:~#1};}}%
{\strut Theorem:~#1};}}%
}%
\mdfsetup{innertopmargin=10pt,linecolor=blue!20,%
linewidth=2pt,topline=true,%
frametitleaboveskip=\dimexpr-\ht\strutbox\relax
}
\begin{mdframed}[]\relax%
\label{#1}}{\end{mdframed}}


%lemma
\newenvironment{lem}[1][]{%
\ifstrempty{#1}%
{\mdfsetup{%
frametitle={%
\tikz[baseline=(current bounding box.east),outer sep=0pt]
\node[anchor=east,rectangle,fill=Dandelion]
{\strut Lemma};}}
}%
{\mdfsetup{%
frametitle={%
\tikz[baseline=(current bounding box.east),outer sep=0pt]
\node[anchor=east,rectangle,fill=Dandelion]
{\strut Lemma:~#1};}}%
}%
\mdfsetup{innertopmargin=10pt,linecolor=Dandelion,%
linewidth=2pt,topline=true,%
frametitleaboveskip=\dimexpr-\ht\strutbox\relax
}
\begin{mdframed}[]\relax%
\label{#1}}{\end{mdframed}}

%corollary
\newenvironment{coro}[1][]{%
\ifstrempty{#1}%
{\mdfsetup{%
frametitle={%
\tikz[baseline=(current bounding box.east),outer sep=0pt]
\node[anchor=east,rectangle,fill=CornflowerBlue]
{\strut Corollary};}}
}%
{\mdfsetup{%
frametitle={%
\tikz[baseline=(current bounding box.east),outer sep=0pt]
\node[anchor=east,rectangle,fill=CornflowerBlue]
{\strut Corollary:~#1};}}%
}%
\mdfsetup{innertopmargin=10pt,linecolor=CornflowerBlue,%
linewidth=2pt,topline=true,%
frametitleaboveskip=\dimexpr-\ht\strutbox\relax
}
\begin{mdframed}[]\relax%
\label{#1}}{\end{mdframed}}

%proof
\newenvironment{prf}[1][]{%
\ifstrempty{#1}%
{\mdfsetup{%
frametitle={%
\tikz[baseline=(current bounding box.east),outer sep=0pt]
\node[anchor=east,rectangle,fill=SpringGreen]
{\strut Proof};}}
}%
{\mdfsetup{%
frametitle={%
\tikz[baseline=(current bounding box.east),outer sep=0pt]
\node[anchor=east,rectangle,fill=SpringGreen]
{\strut Proof:~#1};}}%
}%
\mdfsetup{innertopmargin=10pt,linecolor=SpringGreen,%
linewidth=2pt,topline=true,%
frametitleaboveskip=\dimexpr-\ht\strutbox\relax
}
\begin{mdframed}[]\relax%
\label{#1}}{\qed\end{mdframed}}


\theoremstyle{definition}

\newmdtheoremenv[nobreak=true]{definition}{Definition}
\newmdtheoremenv[nobreak=true]{prop}{Proposition}
\newmdtheoremenv[nobreak=true]{theorem}{Theorem}
\newmdtheoremenv[nobreak=true]{corollary}{Corollary}
\newtheorem*{eg}{Example}
\theoremstyle{remark}
\newtheorem*{case}{Case}
\newtheorem*{notation}{Notation}
\newtheorem*{remark}{Remark}
\newtheorem*{note}{Note}
\newtheorem*{problem}{Problem}
\newtheorem*{observe}{Observe}
\newtheorem*{property}{Property}
\newtheorem*{intuition}{Intuition}


% End example and intermezzo environments with a small diamond (just like proof
% environments end with a small square)
\usepackage{etoolbox}
\AtEndEnvironment{vb}{\null\hfill$\diamond$}%
\AtEndEnvironment{intermezzo}{\null\hfill$\diamond$}%
% \AtEndEnvironment{opmerking}{\null\hfill$\diamond$}%

% Fix some spacing
% http://tex.stackexchange.com/questions/22119/how-can-i-change-the-spacing-before-theorems-with-amsthm
\makeatletter
\def\thm@space@setup{%
  \thm@preskip=\parskip \thm@postskip=0pt
}

% Fix some stuff
% %http://tex.stackexchange.com/questions/76273/multiple-pdfs-with-page-group-included-in-a-single-page-warning
\pdfsuppresswarningpagegroup=1


% My name
\author{Jaden Wang}



\begin{document}
\begin{defn}[simplied irreducible]
	An element $ f(x) \in F[x]$ is \allbold{irreducible} if
	\begin{enumerate}[label=(\roman*)]
		\item It is not constant;
		\item It cannot be factorized into two polynomials both of strictly smaller degree.
	\end{enumerate}
\end{defn}

\begin{eg}[]
	In $ \qq[x]$, $ 3x+3$ is irreducible because  $ 3$ is a unit. Also the degree wouldn't work if  $ 3x+3 = g(x)h(x)$ where $ g(x),h(x)$ are nonconstant.
\end{eg}


\begin{coro}[]
	In $ F[x]$, a polynomial of degree 1 is irreducible. 
\end{coro}

\begin{remark}
	There is always a question about degree 4 factorization on the final.
\end{remark}

Let's now look at degree 2 or 3.

Suppose $ f(x) \in F[x]$ has degree 2, \emph{i.e.} $ f(x) = ax^2 +b x + c, a,b,c \in F, a \neq 0$. Suppose $ f(x)=g(x)h(x)$. Then we have three cases. When $ g(x)$ or  $ h(x)$ has degree 0, they are units and we ignore these cases. So if  $ f$ is not irreducible, then  $ \dg g = \dg h =1$. $ g(x) = ax+b, a \neq 0$, since  $ a^{-1} \in F$, $ g(x)$ has a root in  $ F$, which is  $ -\frac{b}{a}$. Then $ f(x)$ has a root in  $ F$ (the same one).

Suppose  $ f(x)$ has degree 3,  \emph{i.e.} $ f(x) = ax^3+bx^2+cx+d, a,b,c,d \in F, a\neq 0$. Then $ f(x) = g(x)h(x)$, ignoring degree 0 cases, implies two cases.  When $ \dg g=1$, then $ g$ has a root in  $ F$, so  $ f(x)$ does as well by evaluation homomorphism. When  $ \dg h=1$, then $ h$ has a root in  $ F$, so  $ f(x)$ does as well.

\begin{eg}[]
	$ x^2+1 \in \rr[x]$ has a zero $ i$ but it's not in  $ \rr$.
\end{eg}

\begin{thm}[23.10]
	If $ f(x) \in F[x]$ has degree 2 or 3, then either  $ f(x) $ is irreducible or  $ f(x)$ has a zero in  $ F$. 
\end{thm}
\begin{coro}[]
	If $ f(x) \in F[x]$ is a polynomial of degree 2 or 3 with no roots in $ F$, then  $ f(x)$ is irreducible.
\end{coro}

\begin{eg}[]
	$ x^2 + 1 \in R[x]$ is irreducible.
\end{eg}
\begin{note}[]
This is not true in degree 4 or higher.
\end{note}
\begin{eg}[counterexample in degree 4]
	Consider $ x^{4}+1 \in \qq[x]$. Does $ f(x)$ have any roots in  $ \qq$? No. What about $ \rr[x]$? No. Suppose $ f(x) = g(x)h(x)$. Ignoring degree 0 cases, for degree 1 cases there is a root. But for degree 2 case we might have 2 irreducible quadratics (no root). If  $ f$ is not irreducible we must have 2 irreducible quadratics case. WLOG, we can pretend the first term is monic because we can convert to monic anyway:
	 \begin{align*}
		 x^{4}+1 = (x^2+ ax + b)(x^2+cx+d) 
	\end{align*}
Now we can equate coefficients:

Constant terms: $ 1=bd$.

Coefficient of  $ x^3$: $ 0 = a+c$.

 $ x$: $ 0= ad+bc$.

  $ x^2$: $ 0 = ac+b+d$. 

  Let's write everything in terms of  $ a,b$:  $ d=\frac{1}{b}, c=-a$. So we have
  \begin{equation*}
  \begin{cases}
	  0&=\frac{a}{b} - ab\\
	  0&=b+\frac{1}{b} -a^2\\
  \end{cases}
  \end{equation*}

  First gives $ a(1-b^2)=0 \implies a=0 \text{ or }  b=\pm 1$. If $ a=0$, then  $ b+\frac{1}{b}=0$ (no solution). If $ b=-1$,  $ a^2 = -2$. If $ b=1$, then  $ a^2=2 \implies a=\sqrt{2}, b =1, c= -\sqrt{2}, d=1  $. So
  \[
	  x^{4}+1 = (x^2 + \sqrt{2}x+1 )(x^2 -\sqrt{2}x +1 )
  .\]
  These two factors are not units and irreducible because they are degree 2 and have no roots by determinant $ <0$ of quadratic. 

  Therefore, $ x^{4}+1 \in R[x]$ is reducible (two irreducible quadratics). But $ x^{4}+1 \in \qq[x]$ is irreducible. The other parts of the question can be $ \zz_p$. 

  Consider $ x^{4}+1 \in \cc[x]$. Since $ \cc$ is algebraically closed. We can pull out 4 factors:
  \[
	  ^{x^{4}+1} = (x-\alpha)(x-\beta)(x-\gamma)(x-\delta)
  .\] 
  Over $ \cc$, any $ f(x) \in  \cc[x]$ is a product of degree 1 polynomials. 

  So what is the solution $ x^{4}=-1 $? Think of unit circle. Then
  \[
	  x^{4}+1 = (x-e^{\frac{i\pi}{4 }})\ldots
  .\] 
\end{eg}

\begin{thm}[Fundamental Theorem of Algebra]
Every polynomial over $ \cc$ has a root in $ \cc$.
\end{thm}

\begin{thm}[]
	Any nonconstant polynomial $ f(x) \in F[x]$ can be written as a product of irreducible polynomials. This is unique up to changing the order and multiplying by units.
\end{thm}
\begin{eg}[]
	$ x^2-5x+6 \in \qq[x] = (x-2)(x-3) = (x-3)(x-2) = (5x-15)(\frac{x}{5}-\frac{2}{5})$.
\end{eg}
\end{document}

\documentclass[class=article,crop=false]{standalone} 
%Fall 2020
% Some basic packages
\usepackage{standalone}[subpreambles=true]
\usepackage[utf8]{inputenc}
\usepackage[T1]{fontenc}
\usepackage{textcomp}
\usepackage[english]{babel}
\usepackage{url}
\usepackage{graphicx}
\usepackage{float}
\usepackage{enumitem}


\pdfminorversion=7

% Don't indent paragraphs, leave some space between them
\usepackage{parskip}

% Hide page number when page is empty
\usepackage{emptypage}
\usepackage{subcaption}
\usepackage{multicol}
\usepackage[dvipsnames]{xcolor}


% Math stuff
\usepackage{amsmath, amsfonts, mathtools, amsthm, amssymb}
% Fancy script capitals
\usepackage{mathrsfs}
\usepackage{cancel}
% Bold math
\usepackage{bm}
% Some shortcuts
\newcommand{\rr}{\ensuremath{\mathbb{R}}}
\newcommand{\zz}{\ensuremath{\mathbb{Z}}}
\newcommand{\qq}{\ensuremath{\mathbb{Q}}}
\newcommand{\nn}{\ensuremath{\mathbb{N}}}
\newcommand{\ff}{\ensuremath{\mathbb{F}}}
\newcommand{\cc}{\ensuremath{\mathbb{C}}}
\renewcommand\O{\ensuremath{\emptyset}}
\newcommand{\norm}[1]{{\left\lVert{#1}\right\rVert}}
\renewcommand{\vec}[1]{{\mathbf{#1}}}
\newcommand\allbold[1]{{\boldmath\textbf{#1}}}

% Put x \to \infty below \lim
\let\svlim\lim\def\lim{\svlim\limits}

%Make implies and impliedby shorter
\let\implies\Rightarrow
\let\impliedby\Leftarrow
\let\iff\Leftrightarrow
\let\epsilon\varepsilon

% Add \contra symbol to denote contradiction
\usepackage{stmaryrd} % for \lightning
\newcommand\contra{\scalebox{1.5}{$\lightning$}}

% \let\phi\varphi

% Command for short corrections
% Usage: 1+1=\correct{3}{2}

\definecolor{correct}{HTML}{009900}
\newcommand\correct[2]{\ensuremath{\:}{\color{red}{#1}}\ensuremath{\to }{\color{correct}{#2}}\ensuremath{\:}}
\newcommand\green[1]{{\color{correct}{#1}}}

% horizontal rule
\newcommand\hr{
    \noindent\rule[0.5ex]{\linewidth}{0.5pt}
}

% hide parts
\newcommand\hide[1]{}

% si unitx
\usepackage{siunitx}
\sisetup{locale = FR}

% Environments
\makeatother
% For box around Definition, Theorem, \ldots
\usepackage[framemethod=TikZ]{mdframed}
\mdfsetup{skipabove=1em,skipbelow=0em}

%definition
\newenvironment{defn}[1][]{%
\ifstrempty{#1}%
{\mdfsetup{%
frametitle={%
\tikz[baseline=(current bounding box.east),outer sep=0pt]
\node[anchor=east,rectangle,fill=Emerald]
{\strut Definition};}}
}%
{\mdfsetup{%
frametitle={%
\tikz[baseline=(current bounding box.east),outer sep=0pt]
\node[anchor=east,rectangle,fill=Emerald]
{\strut Definition:~#1};}}%
}%
\mdfsetup{innertopmargin=10pt,linecolor=Emerald,%
linewidth=2pt,topline=true,%
frametitleaboveskip=\dimexpr-\ht\strutbox\relax
}
\begin{mdframed}[]\relax%
\label{#1}}{\end{mdframed}}


%theorem
%\newcounter{thm}[section]\setcounter{thm}{0}
%\renewcommand{\thethm}{\arabic{section}.\arabic{thm}}
\newenvironment{thm}[1][]{%
%\refstepcounter{thm}%
\ifstrempty{#1}%
{\mdfsetup{%
frametitle={%
\tikz[baseline=(current bounding box.east),outer sep=0pt]
\node[anchor=east,rectangle,fill=blue!20]
%{\strut Theorem~\thethm};}}
{\strut Theorem};}}
}%
{\mdfsetup{%
frametitle={%
\tikz[baseline=(current bounding box.east),outer sep=0pt]
\node[anchor=east,rectangle,fill=blue!20]
%{\strut Theorem~\thethm:~#1};}}%
{\strut Theorem:~#1};}}%
}%
\mdfsetup{innertopmargin=10pt,linecolor=blue!20,%
linewidth=2pt,topline=true,%
frametitleaboveskip=\dimexpr-\ht\strutbox\relax
}
\begin{mdframed}[]\relax%
\label{#1}}{\end{mdframed}}


%lemma
\newenvironment{lem}[1][]{%
\ifstrempty{#1}%
{\mdfsetup{%
frametitle={%
\tikz[baseline=(current bounding box.east),outer sep=0pt]
\node[anchor=east,rectangle,fill=Dandelion]
{\strut Lemma};}}
}%
{\mdfsetup{%
frametitle={%
\tikz[baseline=(current bounding box.east),outer sep=0pt]
\node[anchor=east,rectangle,fill=Dandelion]
{\strut Lemma:~#1};}}%
}%
\mdfsetup{innertopmargin=10pt,linecolor=Dandelion,%
linewidth=2pt,topline=true,%
frametitleaboveskip=\dimexpr-\ht\strutbox\relax
}
\begin{mdframed}[]\relax%
\label{#1}}{\end{mdframed}}

%corollary
\newenvironment{coro}[1][]{%
\ifstrempty{#1}%
{\mdfsetup{%
frametitle={%
\tikz[baseline=(current bounding box.east),outer sep=0pt]
\node[anchor=east,rectangle,fill=CornflowerBlue]
{\strut Corollary};}}
}%
{\mdfsetup{%
frametitle={%
\tikz[baseline=(current bounding box.east),outer sep=0pt]
\node[anchor=east,rectangle,fill=CornflowerBlue]
{\strut Corollary:~#1};}}%
}%
\mdfsetup{innertopmargin=10pt,linecolor=CornflowerBlue,%
linewidth=2pt,topline=true,%
frametitleaboveskip=\dimexpr-\ht\strutbox\relax
}
\begin{mdframed}[]\relax%
\label{#1}}{\end{mdframed}}

%proof
\newenvironment{prf}[1][]{%
\ifstrempty{#1}%
{\mdfsetup{%
frametitle={%
\tikz[baseline=(current bounding box.east),outer sep=0pt]
\node[anchor=east,rectangle,fill=SpringGreen]
{\strut Proof};}}
}%
{\mdfsetup{%
frametitle={%
\tikz[baseline=(current bounding box.east),outer sep=0pt]
\node[anchor=east,rectangle,fill=SpringGreen]
{\strut Proof:~#1};}}%
}%
\mdfsetup{innertopmargin=10pt,linecolor=SpringGreen,%
linewidth=2pt,topline=true,%
frametitleaboveskip=\dimexpr-\ht\strutbox\relax
}
\begin{mdframed}[]\relax%
\label{#1}}{\qed\end{mdframed}}


\theoremstyle{definition}

\newmdtheoremenv[nobreak=true]{definition}{Definition}
\newmdtheoremenv[nobreak=true]{prop}{Proposition}
\newmdtheoremenv[nobreak=true]{theorem}{Theorem}
\newmdtheoremenv[nobreak=true]{corollary}{Corollary}
\newtheorem*{eg}{Example}
\theoremstyle{remark}
\newtheorem*{case}{Case}
\newtheorem*{notation}{Notation}
\newtheorem*{remark}{Remark}
\newtheorem*{note}{Note}
\newtheorem*{problem}{Problem}
\newtheorem*{observe}{Observe}
\newtheorem*{property}{Property}
\newtheorem*{intuition}{Intuition}


% End example and intermezzo environments with a small diamond (just like proof
% environments end with a small square)
\usepackage{etoolbox}
\AtEndEnvironment{vb}{\null\hfill$\diamond$}%
\AtEndEnvironment{intermezzo}{\null\hfill$\diamond$}%
% \AtEndEnvironment{opmerking}{\null\hfill$\diamond$}%

% Fix some spacing
% http://tex.stackexchange.com/questions/22119/how-can-i-change-the-spacing-before-theorems-with-amsthm
\makeatletter
\def\thm@space@setup{%
  \thm@preskip=\parskip \thm@postskip=0pt
}

% Fix some stuff
% %http://tex.stackexchange.com/questions/76273/multiple-pdfs-with-page-group-included-in-a-single-page-warning
\pdfsuppresswarningpagegroup=1


% My name
\author{Jaden Wang}



\begin{document}
\begin{defn}[Cartesian product]
	Given groups $ (G_1,*), \ldots, (G_n,*)$. Then $ G_1 \times \ldots \times G_n$ are the Cartesian product. The element in this product is the $ n$-tuple $ (g_1,\ldots,g_n): g_i \in G_i$. The product of the elements is componentwise. 
\end{defn}

\begin{note}[]
	\begin{itemize}
		\item a finite group is finitely generated. It's generated by itself.
		\item $ \zz$ is finitely generated. It's generated by 1.
	\end{itemize}
\end{note}

WARNING: $ \zz_2 \times \zz_2 \not \simeq \zz_4$. Because the former is isomorphic to $ V_4$.

Recall the fundamental theorem of arithmetic claims that any number can be written as a product of primes. The factorization is unique up to permutation. We will generalize this to groups.

\begin{note}[]
	Rational numbers are not finitely generated.
\end{note}
\begin{thm}[The Fundamental Theorem of Finitely Generated Abelian Groups]
	Any finitely generated abelian group $ G$ is isomorphic to
	 \[
	G_1 \times G_2 \times \ldots \times G_n
	\] 
	where each $ G_i$ is either isomorphic to
	\begin{itemize}
		\item a cyclic group of prime power order, $ \zz_{p^r}$, where $ p$ is prime and  $ r \in \nn$.
		\item or $ \zz$.
	\end{itemize}
	Two such groups $ G_1 \times \ldots \times G_n$ and $ H_1 \times \ldots \times H_m$ are isomorphic if and only if $ m=n$ and the factors are rearrangements of each other.
\end{thm}

\begin{note}[]
	Rearrangement is the only way to have isomorphism. Unlike general groups. This theorem is powerful to tell when two finitely generated abelian groups are not isomorphic.
\end{note}

\begin{eg}[abelian group of order 8]
~\begin{enumerate}[label=\arabic*)]
	\item $ \zz_8$. The building blocks are $ \zz_8, \zz_4, \zz_2$.
	\item $ \zz_2 \times \zz_2 \times \zz_2$ not isomorphic to $ \zz_8$ by the fundamental theorem, as they are all power of prime numbers but they are not rearrangements. 
	\item $ \zz_4 \times \zz_2$.
	\item $ \zz_2 \times \zz_4$ is isomorphic to 3) by theorem.
\end{enumerate}

\begin{claim}[]
Direct products of abelian groups are abelian.
\end{claim}

\begin{enumerate}[label=\arabic*)]
	\item $ \zz_8 \zz_{2^3}$ 3
	\item $ \zz_4 \times \zz_2 \zz_{2^2} \times \zz_{2^1}$ 2+1
	\item $ \zz_2 \times \zz_2\times \zz_2 \zz_{2^1} \times \zz_{2^1} \times \zz_{2^1}$ 1+1+1
\end{enumerate}
These are "partitions of 3": Sequences of positive integers that sums to 3 and are decreasing. Partitions of $ n$ control abelian groups of order  $ 2^{n}$.
\end{eg}

\begin{eg}[partition of 4]
$ 4, 3+1,2+2,2+1+1,1+1+1+1$. Use this to investigate abelian groups of order 81= $ 3^4$. 
\begin{enumerate}[label=\arabic*)]
	\item $ \zz_{3^{4}}$ 
	\item $ \zz_{3^3} \times \zz_{3}$ 
	\item $ \zz_{3^2} \times \zz_{3^2}$ 
	\item $ \zz_{3^2}\times \zz_{3} \times \zz_{3}$ 
	\item $ \zz_{3} \times \zz_3 \times \zz_3 \times \zz_3$
\end{enumerate}
\end{eg}

\begin{eg}[]
Classify all abelian group of order 360.

$ 360 = 2^3 \times 3^2 \times 5^1$. Consider each power of prime term individually.
\begin{enumerate}[label=\arabic*)]
	\item abelian groups of order 8.
	\item order 9: $ \zz_9$ and $ \zz_3 \times \zz_3$
	\item order 5: $ \zz_5$.
\end{enumerate}
Then we just pick one out of each and do direct product. Note that $ \zz_{360} \simeq \zz_{72} \times \zz_5 \simeq \zz_8 \times \zz_9 \times \zz_5 $.
\end{eg}
\begin{eg}[]
Does $ \zz \simeq \zz \times \zz_2$? No use theorem and observe these are not rearrangement.
\end{eg}
\begin{eg}[11.18]
Is $ \zz_8 \times \zz_{10} \times \zz_{24}$ isomorphic to $ \zz_4 \times \zz_{12} \times \zz_{40}$?

Former $\simeq \zz_8 \times \zz_2 \times \zz_5 \times \zz_3 \times \zz_8$.\\
Latter $ \simeq \zz_4 \times \zz_3 \times \zz_4 \zz_5 \times \zz_8$.
Not isomorphic. This is the repeated application of $ \zz_m \times \zz_n \simeq \zz_{mn}$ if $  \gcd ( m,n) =1$.
\end{eg}

\begin{note}[]
Abelian group has subgroups of every allowed orders by Lagrange.
\end{note}
\begin{eg}[11.10]
	$ (8,4,10) \in \zz_{12} \times \zz_{60} \times \zz_{24}$. What is the order?
	Break it down to each. 
	Order of 8 in $ \zz_{12}$? 3. 4:15. 10:12.
	The answer $ \lcm ( 3,15,12) =  \frac{15 \times 12}{ \gcd ( 15,12) }=60 $.
\end{eg}
\end{document}

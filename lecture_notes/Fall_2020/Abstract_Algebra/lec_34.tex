\documentclass[class=article,crop=false]{standalone} 
%Fall 2020
% Some basic packages
\usepackage{standalone}[subpreambles=true]
\usepackage[utf8]{inputenc}
\usepackage[T1]{fontenc}
\usepackage{textcomp}
\usepackage[english]{babel}
\usepackage{url}
\usepackage{graphicx}
\usepackage{float}
\usepackage{enumitem}


\pdfminorversion=7

% Don't indent paragraphs, leave some space between them
\usepackage{parskip}

% Hide page number when page is empty
\usepackage{emptypage}
\usepackage{subcaption}
\usepackage{multicol}
\usepackage[dvipsnames]{xcolor}


% Math stuff
\usepackage{amsmath, amsfonts, mathtools, amsthm, amssymb}
% Fancy script capitals
\usepackage{mathrsfs}
\usepackage{cancel}
% Bold math
\usepackage{bm}
% Some shortcuts
\newcommand{\rr}{\ensuremath{\mathbb{R}}}
\newcommand{\zz}{\ensuremath{\mathbb{Z}}}
\newcommand{\qq}{\ensuremath{\mathbb{Q}}}
\newcommand{\nn}{\ensuremath{\mathbb{N}}}
\newcommand{\ff}{\ensuremath{\mathbb{F}}}
\newcommand{\cc}{\ensuremath{\mathbb{C}}}
\renewcommand\O{\ensuremath{\emptyset}}
\newcommand{\norm}[1]{{\left\lVert{#1}\right\rVert}}
\renewcommand{\vec}[1]{{\mathbf{#1}}}
\newcommand\allbold[1]{{\boldmath\textbf{#1}}}

% Put x \to \infty below \lim
\let\svlim\lim\def\lim{\svlim\limits}

%Make implies and impliedby shorter
\let\implies\Rightarrow
\let\impliedby\Leftarrow
\let\iff\Leftrightarrow
\let\epsilon\varepsilon

% Add \contra symbol to denote contradiction
\usepackage{stmaryrd} % for \lightning
\newcommand\contra{\scalebox{1.5}{$\lightning$}}

% \let\phi\varphi

% Command for short corrections
% Usage: 1+1=\correct{3}{2}

\definecolor{correct}{HTML}{009900}
\newcommand\correct[2]{\ensuremath{\:}{\color{red}{#1}}\ensuremath{\to }{\color{correct}{#2}}\ensuremath{\:}}
\newcommand\green[1]{{\color{correct}{#1}}}

% horizontal rule
\newcommand\hr{
    \noindent\rule[0.5ex]{\linewidth}{0.5pt}
}

% hide parts
\newcommand\hide[1]{}

% si unitx
\usepackage{siunitx}
\sisetup{locale = FR}

% Environments
\makeatother
% For box around Definition, Theorem, \ldots
\usepackage[framemethod=TikZ]{mdframed}
\mdfsetup{skipabove=1em,skipbelow=0em}

%definition
\newenvironment{defn}[1][]{%
\ifstrempty{#1}%
{\mdfsetup{%
frametitle={%
\tikz[baseline=(current bounding box.east),outer sep=0pt]
\node[anchor=east,rectangle,fill=Emerald]
{\strut Definition};}}
}%
{\mdfsetup{%
frametitle={%
\tikz[baseline=(current bounding box.east),outer sep=0pt]
\node[anchor=east,rectangle,fill=Emerald]
{\strut Definition:~#1};}}%
}%
\mdfsetup{innertopmargin=10pt,linecolor=Emerald,%
linewidth=2pt,topline=true,%
frametitleaboveskip=\dimexpr-\ht\strutbox\relax
}
\begin{mdframed}[]\relax%
\label{#1}}{\end{mdframed}}


%theorem
%\newcounter{thm}[section]\setcounter{thm}{0}
%\renewcommand{\thethm}{\arabic{section}.\arabic{thm}}
\newenvironment{thm}[1][]{%
%\refstepcounter{thm}%
\ifstrempty{#1}%
{\mdfsetup{%
frametitle={%
\tikz[baseline=(current bounding box.east),outer sep=0pt]
\node[anchor=east,rectangle,fill=blue!20]
%{\strut Theorem~\thethm};}}
{\strut Theorem};}}
}%
{\mdfsetup{%
frametitle={%
\tikz[baseline=(current bounding box.east),outer sep=0pt]
\node[anchor=east,rectangle,fill=blue!20]
%{\strut Theorem~\thethm:~#1};}}%
{\strut Theorem:~#1};}}%
}%
\mdfsetup{innertopmargin=10pt,linecolor=blue!20,%
linewidth=2pt,topline=true,%
frametitleaboveskip=\dimexpr-\ht\strutbox\relax
}
\begin{mdframed}[]\relax%
\label{#1}}{\end{mdframed}}


%lemma
\newenvironment{lem}[1][]{%
\ifstrempty{#1}%
{\mdfsetup{%
frametitle={%
\tikz[baseline=(current bounding box.east),outer sep=0pt]
\node[anchor=east,rectangle,fill=Dandelion]
{\strut Lemma};}}
}%
{\mdfsetup{%
frametitle={%
\tikz[baseline=(current bounding box.east),outer sep=0pt]
\node[anchor=east,rectangle,fill=Dandelion]
{\strut Lemma:~#1};}}%
}%
\mdfsetup{innertopmargin=10pt,linecolor=Dandelion,%
linewidth=2pt,topline=true,%
frametitleaboveskip=\dimexpr-\ht\strutbox\relax
}
\begin{mdframed}[]\relax%
\label{#1}}{\end{mdframed}}

%corollary
\newenvironment{coro}[1][]{%
\ifstrempty{#1}%
{\mdfsetup{%
frametitle={%
\tikz[baseline=(current bounding box.east),outer sep=0pt]
\node[anchor=east,rectangle,fill=CornflowerBlue]
{\strut Corollary};}}
}%
{\mdfsetup{%
frametitle={%
\tikz[baseline=(current bounding box.east),outer sep=0pt]
\node[anchor=east,rectangle,fill=CornflowerBlue]
{\strut Corollary:~#1};}}%
}%
\mdfsetup{innertopmargin=10pt,linecolor=CornflowerBlue,%
linewidth=2pt,topline=true,%
frametitleaboveskip=\dimexpr-\ht\strutbox\relax
}
\begin{mdframed}[]\relax%
\label{#1}}{\end{mdframed}}

%proof
\newenvironment{prf}[1][]{%
\ifstrempty{#1}%
{\mdfsetup{%
frametitle={%
\tikz[baseline=(current bounding box.east),outer sep=0pt]
\node[anchor=east,rectangle,fill=SpringGreen]
{\strut Proof};}}
}%
{\mdfsetup{%
frametitle={%
\tikz[baseline=(current bounding box.east),outer sep=0pt]
\node[anchor=east,rectangle,fill=SpringGreen]
{\strut Proof:~#1};}}%
}%
\mdfsetup{innertopmargin=10pt,linecolor=SpringGreen,%
linewidth=2pt,topline=true,%
frametitleaboveskip=\dimexpr-\ht\strutbox\relax
}
\begin{mdframed}[]\relax%
\label{#1}}{\qed\end{mdframed}}


\theoremstyle{definition}

\newmdtheoremenv[nobreak=true]{definition}{Definition}
\newmdtheoremenv[nobreak=true]{prop}{Proposition}
\newmdtheoremenv[nobreak=true]{theorem}{Theorem}
\newmdtheoremenv[nobreak=true]{corollary}{Corollary}
\newtheorem*{eg}{Example}
\theoremstyle{remark}
\newtheorem*{case}{Case}
\newtheorem*{notation}{Notation}
\newtheorem*{remark}{Remark}
\newtheorem*{note}{Note}
\newtheorem*{problem}{Problem}
\newtheorem*{observe}{Observe}
\newtheorem*{property}{Property}
\newtheorem*{intuition}{Intuition}


% End example and intermezzo environments with a small diamond (just like proof
% environments end with a small square)
\usepackage{etoolbox}
\AtEndEnvironment{vb}{\null\hfill$\diamond$}%
\AtEndEnvironment{intermezzo}{\null\hfill$\diamond$}%
% \AtEndEnvironment{opmerking}{\null\hfill$\diamond$}%

% Fix some spacing
% http://tex.stackexchange.com/questions/22119/how-can-i-change-the-spacing-before-theorems-with-amsthm
\makeatletter
\def\thm@space@setup{%
  \thm@preskip=\parskip \thm@postskip=0pt
}

% Fix some stuff
% %http://tex.stackexchange.com/questions/76273/multiple-pdfs-with-page-group-included-in-a-single-page-warning
\pdfsuppresswarningpagegroup=1


% My name
\author{Jaden Wang}



\begin{document}

\begin{eg}[]
	$ R[x,y]$ is a polynomial ring over  $ R$ in two commuting indeterminates,  $ x$ and  $ y$ that satisfies
	\[
		(x^{i} y^{j}) (x^{k} y^{l}) = x^{i+k} y ^{j+l}
	\] and distributivity.

	\begin{note}[]
	Noncommutative indeterminates version is denoted $ R\langle x,y \rangle$.
	\end{note}

	We can thin of $ R[x,y]$ as  $ (R[x])[y]$ or as  $ (R[y])[x]$. For example, in  $ \zz[x,y]$:
	\begin{align*}
		5+6y-3xy^2+2x^3y-7x^2y^2+4x \in \zz[x,y]\\
		(5+4x)+ (6+2x^3)y-(3x+7x^2)y^2 \in (\zz[x])[y]\\
		(5+6y) + (4-3y^2)x - (7y^2)x^2 + 2y(x^3) \in (\zz[y])[x]\\
	\end{align*}
\end{eg}

\begin{thm}[]
	$ R=F[x]$,  $ F$ is a field, $ f(x), g(x) \in F[x]$. There exist unique $ q(x), r(x) \in F[x]$ such that 
	\[
		f(x) = q(x)g(x) + r(x)
	\] 
	where either $ r(x) =0$ or deg$r <$ deg$g $.
\end{thm}

\begin{prf}
Uniqueness: Suppose
\begin{align*}
	q_1(x)g(x) + r_1(x) &= q_2 (x) g(x) + r_2(x) \\
	(q_1(x) - q_2(x)) g(x)&= r_2(x) - r_1(x) \\
\end{align*}
Notice the LHS is zero polynomial or has degree $ \geq $deg$g $, and the RHS is either zero polynomial or has degree  $ <$ deg$g $. The only way they can equal is to be both zero polynomial. Thus they are unique. 
\end{prf}

\begin{thm}[Factor Theorem]
	$ f(x) \in F[x], a \in F$. $ a$ is a root of  $ f(x) $ if and only if  $ (x-a)$ is a factor of  $ f(x)$. That is,
	 \[
		 f(x) = (x-a)h(x), h(x) \in F[x]
	.\] 
	We have $ \deg f = 1 + \deg h$.
\end{thm}

\begin{prf}
	Use division algorithm with $ f(x)$ and with  $ g(x)=x-a$:
	 \[
		 f(x)=q(x)(x-a) + r(x)
	.\] 
	where $ r(x) =0$ or  $ \deg r< \deg (x-a)$ which means $ r(x)$ is a nonzero constant.

	Thus we get
	 \[
		 f(x) = q(x)(x-a)+c
	.\] 
	Consider $ \phi_{a}: F[x] \to F$. So
	\begin{align*}
		\phi_a(f(x))& = \phi_a (q(x)(x-a)+c)\\
		f(a)&= q(a)(a-a) + c \\
	\end{align*}
	There are two cases.
	\begin{case}[]
		$ c=0$, then  $ f(a)=0$ and  $ f(x) = q(x)(x-a)$.
	\end{case}
	\begin{case}[]
		$ c\neq 0$, then  $ f(a) \neq 0$ and  $ f(x)= q(x)(x-a) + c$. Here  $ a$ is not a root and  $ c$ is a nonzero remainder.
	\end{case}
\end{prf}

\begin{coro}[23.6]
	Let $ F$  be a field and $ F^* $ be its group of units. Any finite subgroup of $ F^* $ is cyclic. 
\end{coro}
\begin{prf}[]
	Let $ G$ be a finite subgroup of  $ F^* $. Then $ G \simeq \zz_{d_1} \times \ldots\times \zz_{d_r}$, where $ d_i$ are prime powers. Let $ m= \lcm \left( d_1,\ldots,d_r \right)  $, thus $ m\leq d_1 \cdot  \ldots \cdot d_r$. In $ \zz_{d_i}, a_i^{d_i} = 1 \implies a_i^{m} = 1$ because $ d_i /m$. This means $ a^{m} =1$ for any $ a \in G$. So every element is a root of $ x^{m}-1 \in F^*[x]$, and there are at most $ m$ such roots by factor theorem and induction. Since $ d_1 \cdot  \ldots \cdot d_r$ is the order of $ G$ and every element of  $ G$ is such root, and there are at most  $ m$ of them in  $ F^*$, we have $ d_1 \cdot \ldots \cdot  d_r \leq m$. Hence, $ m = d_1 \cdot  \ldots \cdot  d_r$. Since the $ d_i$ are relatively prime, then the direct product is cyclic.
\end{prf}


\begin{note}[important special case]
	If $ F$ is a finite field, then  $ F^* $ is cyclic. For example, $\zz_{11}^* $ is cyclic of order 10. Q2 Section 20. 

	2 is a generator for $ \zz_{11}^* $. Since $ \zz_{11}^* \simeq (\zz_{10},+_{ 10} )$. Then the possible orders of elements are 1,2,5,10. Since 2 is not order 1,2,5, it must be order 10. Thus 2 is a primitive root.

	How many generators does $ \zz_{10}$ have? 1,3,7,9. They are pairs of inverses. So we start from 2 and find them: $ 2 \to 8 \to 7 \to 6$. 8 is cube of 2, 7 is inverse of cube, 6 is inverse of 2.

\end{note}
\begin{eg}[23.4]
	$ x^{4}+ 3x^3+2x+4 \in \zz_5[x]$. We can see that $ 1$ is a root of  $ f(x) \in \zz_5$. To factor, we can do it by tacking the easy terms first. But this isn't always feasible.

	We get $ (x-1)(x^3+4x^2-x+1)$. Now does $ x^3 + 4x^2 + 4x + 1$ have any roots in $ \zz_5$? We can find them exhaustively. So $ 1$ is a root, and we get $ (x-1)(x-1)(x+1)$. So 
	 \[
		 f(x) = (x-1)^3 (x+1)
	.\]
\end{eg}

\begin{defn}[irreducible]
	Let $ R$ be a ring with  $ 1$.  $ f(x) \in R[x]$ is \allbold{irreducible} if 
	\begin{enumerate}[label=(\roman*)]
		\item $ f(x)$ is not zero or a unit.
		\item  If $ f(x) = g(x)h(x)$ then either $ g(x)$ or  $ h(x)$ is a unit. 
	\end{enumerate}
\end{defn}

\begin{note}[]
	$ 3x+3 \in \zz[x]$ is not irreducible, since units in $ \zz$ are $ \pm 1$ so we can factor  $ 3(x+1)$. This works because  $ \zz$ is not a field. 
\end{note}
\end{document}

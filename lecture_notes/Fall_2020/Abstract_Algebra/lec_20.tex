\documentclass[class=article,crop=false]{standalone} 
%Fall 2020
% Some basic packages
\usepackage{standalone}[subpreambles=true]
\usepackage[utf8]{inputenc}
\usepackage[T1]{fontenc}
\usepackage{textcomp}
\usepackage[english]{babel}
\usepackage{url}
\usepackage{graphicx}
\usepackage{float}
\usepackage{enumitem}


\pdfminorversion=7

% Don't indent paragraphs, leave some space between them
\usepackage{parskip}

% Hide page number when page is empty
\usepackage{emptypage}
\usepackage{subcaption}
\usepackage{multicol}
\usepackage[dvipsnames]{xcolor}


% Math stuff
\usepackage{amsmath, amsfonts, mathtools, amsthm, amssymb}
% Fancy script capitals
\usepackage{mathrsfs}
\usepackage{cancel}
% Bold math
\usepackage{bm}
% Some shortcuts
\newcommand{\rr}{\ensuremath{\mathbb{R}}}
\newcommand{\zz}{\ensuremath{\mathbb{Z}}}
\newcommand{\qq}{\ensuremath{\mathbb{Q}}}
\newcommand{\nn}{\ensuremath{\mathbb{N}}}
\newcommand{\ff}{\ensuremath{\mathbb{F}}}
\newcommand{\cc}{\ensuremath{\mathbb{C}}}
\renewcommand\O{\ensuremath{\emptyset}}
\newcommand{\norm}[1]{{\left\lVert{#1}\right\rVert}}
\renewcommand{\vec}[1]{{\mathbf{#1}}}
\newcommand\allbold[1]{{\boldmath\textbf{#1}}}

% Put x \to \infty below \lim
\let\svlim\lim\def\lim{\svlim\limits}

%Make implies and impliedby shorter
\let\implies\Rightarrow
\let\impliedby\Leftarrow
\let\iff\Leftrightarrow
\let\epsilon\varepsilon

% Add \contra symbol to denote contradiction
\usepackage{stmaryrd} % for \lightning
\newcommand\contra{\scalebox{1.5}{$\lightning$}}

% \let\phi\varphi

% Command for short corrections
% Usage: 1+1=\correct{3}{2}

\definecolor{correct}{HTML}{009900}
\newcommand\correct[2]{\ensuremath{\:}{\color{red}{#1}}\ensuremath{\to }{\color{correct}{#2}}\ensuremath{\:}}
\newcommand\green[1]{{\color{correct}{#1}}}

% horizontal rule
\newcommand\hr{
    \noindent\rule[0.5ex]{\linewidth}{0.5pt}
}

% hide parts
\newcommand\hide[1]{}

% si unitx
\usepackage{siunitx}
\sisetup{locale = FR}

% Environments
\makeatother
% For box around Definition, Theorem, \ldots
\usepackage[framemethod=TikZ]{mdframed}
\mdfsetup{skipabove=1em,skipbelow=0em}

%definition
\newenvironment{defn}[1][]{%
\ifstrempty{#1}%
{\mdfsetup{%
frametitle={%
\tikz[baseline=(current bounding box.east),outer sep=0pt]
\node[anchor=east,rectangle,fill=Emerald]
{\strut Definition};}}
}%
{\mdfsetup{%
frametitle={%
\tikz[baseline=(current bounding box.east),outer sep=0pt]
\node[anchor=east,rectangle,fill=Emerald]
{\strut Definition:~#1};}}%
}%
\mdfsetup{innertopmargin=10pt,linecolor=Emerald,%
linewidth=2pt,topline=true,%
frametitleaboveskip=\dimexpr-\ht\strutbox\relax
}
\begin{mdframed}[]\relax%
\label{#1}}{\end{mdframed}}


%theorem
%\newcounter{thm}[section]\setcounter{thm}{0}
%\renewcommand{\thethm}{\arabic{section}.\arabic{thm}}
\newenvironment{thm}[1][]{%
%\refstepcounter{thm}%
\ifstrempty{#1}%
{\mdfsetup{%
frametitle={%
\tikz[baseline=(current bounding box.east),outer sep=0pt]
\node[anchor=east,rectangle,fill=blue!20]
%{\strut Theorem~\thethm};}}
{\strut Theorem};}}
}%
{\mdfsetup{%
frametitle={%
\tikz[baseline=(current bounding box.east),outer sep=0pt]
\node[anchor=east,rectangle,fill=blue!20]
%{\strut Theorem~\thethm:~#1};}}%
{\strut Theorem:~#1};}}%
}%
\mdfsetup{innertopmargin=10pt,linecolor=blue!20,%
linewidth=2pt,topline=true,%
frametitleaboveskip=\dimexpr-\ht\strutbox\relax
}
\begin{mdframed}[]\relax%
\label{#1}}{\end{mdframed}}


%lemma
\newenvironment{lem}[1][]{%
\ifstrempty{#1}%
{\mdfsetup{%
frametitle={%
\tikz[baseline=(current bounding box.east),outer sep=0pt]
\node[anchor=east,rectangle,fill=Dandelion]
{\strut Lemma};}}
}%
{\mdfsetup{%
frametitle={%
\tikz[baseline=(current bounding box.east),outer sep=0pt]
\node[anchor=east,rectangle,fill=Dandelion]
{\strut Lemma:~#1};}}%
}%
\mdfsetup{innertopmargin=10pt,linecolor=Dandelion,%
linewidth=2pt,topline=true,%
frametitleaboveskip=\dimexpr-\ht\strutbox\relax
}
\begin{mdframed}[]\relax%
\label{#1}}{\end{mdframed}}

%corollary
\newenvironment{coro}[1][]{%
\ifstrempty{#1}%
{\mdfsetup{%
frametitle={%
\tikz[baseline=(current bounding box.east),outer sep=0pt]
\node[anchor=east,rectangle,fill=CornflowerBlue]
{\strut Corollary};}}
}%
{\mdfsetup{%
frametitle={%
\tikz[baseline=(current bounding box.east),outer sep=0pt]
\node[anchor=east,rectangle,fill=CornflowerBlue]
{\strut Corollary:~#1};}}%
}%
\mdfsetup{innertopmargin=10pt,linecolor=CornflowerBlue,%
linewidth=2pt,topline=true,%
frametitleaboveskip=\dimexpr-\ht\strutbox\relax
}
\begin{mdframed}[]\relax%
\label{#1}}{\end{mdframed}}

%proof
\newenvironment{prf}[1][]{%
\ifstrempty{#1}%
{\mdfsetup{%
frametitle={%
\tikz[baseline=(current bounding box.east),outer sep=0pt]
\node[anchor=east,rectangle,fill=SpringGreen]
{\strut Proof};}}
}%
{\mdfsetup{%
frametitle={%
\tikz[baseline=(current bounding box.east),outer sep=0pt]
\node[anchor=east,rectangle,fill=SpringGreen]
{\strut Proof:~#1};}}%
}%
\mdfsetup{innertopmargin=10pt,linecolor=SpringGreen,%
linewidth=2pt,topline=true,%
frametitleaboveskip=\dimexpr-\ht\strutbox\relax
}
\begin{mdframed}[]\relax%
\label{#1}}{\qed\end{mdframed}}


\theoremstyle{definition}

\newmdtheoremenv[nobreak=true]{definition}{Definition}
\newmdtheoremenv[nobreak=true]{prop}{Proposition}
\newmdtheoremenv[nobreak=true]{theorem}{Theorem}
\newmdtheoremenv[nobreak=true]{corollary}{Corollary}
\newtheorem*{eg}{Example}
\theoremstyle{remark}
\newtheorem*{case}{Case}
\newtheorem*{notation}{Notation}
\newtheorem*{remark}{Remark}
\newtheorem*{note}{Note}
\newtheorem*{problem}{Problem}
\newtheorem*{observe}{Observe}
\newtheorem*{property}{Property}
\newtheorem*{intuition}{Intuition}


% End example and intermezzo environments with a small diamond (just like proof
% environments end with a small square)
\usepackage{etoolbox}
\AtEndEnvironment{vb}{\null\hfill$\diamond$}%
\AtEndEnvironment{intermezzo}{\null\hfill$\diamond$}%
% \AtEndEnvironment{opmerking}{\null\hfill$\diamond$}%

% Fix some spacing
% http://tex.stackexchange.com/questions/22119/how-can-i-change-the-spacing-before-theorems-with-amsthm
\makeatletter
\def\thm@space@setup{%
  \thm@preskip=\parskip \thm@postskip=0pt
}

% Fix some stuff
% %http://tex.stackexchange.com/questions/76273/multiple-pdfs-with-page-group-included-in-a-single-page-warning
\pdfsuppresswarningpagegroup=1


% My name
\author{Jaden Wang}



\begin{document}
Midterm 1: HW 1-4. Midterm 2: HW 5-9.

VERY IMPORTANT: the easiest example of nonnormal subgroup: $ G=S_3$, and $ H$ has order 2,  \emph{i.e.} $ H= \{e, \{1\ 2\} \} $.

True: any subgroup of an abelian group is normal.
False: any abelian subgroup of a group is normal. The example above!

The group $ G$ and the trivial subgroup are normal.

Any subgroup of index 2 is normal.

The kernel of a homomorphism is normal.

False: subgroup of index 3 is normal. The example above again!

True: $ 3 \zz \trianglelefteq \zz$ because $ \zz$ is abelian.

\begin{note}[]
$ H \trianglelefteq G$ is equivalent to:
\begin{itemize}
	\item $ gH = Hg$ for all  $ g \in G$. The left/right cosets containing $ g$. Because  $ e$ is in  $ H$. 
	\item  $ gHg^{-1}=H$ for all $ g \in G$. $ gHg^{-1} = \{ghg^{-1}: h \in H\} $.
		\begin{claim}
			$ gHg^{-1}$ is a subgroup of $ G$ (even if  $ H$ is not normal).
		\end{claim}
		\begin{prf}
		It is clearly a subset. 

		Identity: since $ e \in H$, $ geg^{-1} = g g^{-1} = e \in gHg^{-1}$.

		Closure: $ g h_1 g^{-1} g h_2 g^{-1} = g h_1h_2 g^{-1} \in g H g^{-1}$.

		Inverse: $ (g h g^{-1})^{-1} = ghg^{-1} \in gHg^{-1}$.

		\end{prf}

\begin{eg}[]
	$ G=S_3, H= \{e, (1\ 2) \} $. Let $ g=(1\ 2\ 3)$.
	 \[
		 gHg^{-1}= \{(1\ 2\ 3)e(1\ 3\ 2), (1\ 2\ 3)(1\ 2)(1\ 3\ 2)\} =\{e, (2\ 3)\} \neq H
	.\] 
\end{eg}
This is a subgroup of $ S_3$, this proves that it is not normal.

This form is called \allbold{conjugation}. We conjugated $ H$ by  $ g$ to get  $ gHg^{-1}$. This might not give us the same subgroup but it would have the same order.

Then $ gH=Hg \iff gH g^{-1}=H g g^{-1} =H$.
\item $ ghg^{-1} \in H$  for all $ g \in G$, $ h \in H$. This is very useful if everything else doesn't work.

	Warning: this only checks that a known subgroup is normal. It doesn't prove that something is a subgroup.

	Recall last time we tried to put a group structure on the (left) cosets of $ H$ in  $ G$. That is,
	 \[
		 (xH)*(yH) = xyH
	.\] 
	However, this is not well-defined unless $ y^{-1} h y \in H$ for all $ y \in G, h \in H$. Let $ y=g^{-1}$, then $ ghg^{-1} \in H \ \forall \ h \in H, g \in G$.

	\begin{eg}[not well-defined function]
		$ f\left( \frac{a}{b} \right) =a $ is not well-defined because by choosing different representations we get different answers.
	\end{eg}
\end{itemize}
\end{note}

\begin{defn}[quotient group]
	Let $ G$ be a group and  $ N \trianglelefteq G$. We define a new group, $ G / N$ (read $ G \mod N$), where $ G / N$ is the set of cosets of $ N$ in  $ G$, and the operation is $ (xN)*(yN)=xyN$.
\end{defn}
\begin{intuition}
	$ N$ is normal guarantees that if we choose different elements from the same cosets, we would get answers in another same coset.
\end{intuition}
To show that the quotient group is indeed a group, 
\begin{prf}
~\begin{enumerate}[label=(\roman*)]
	\item identity: $ eH = N$ so that  $ (eN)*(xN) = exN=xN=(xN)*(eN)$.
	\item inverses:  $ (xN)*(x^{-1}N) = x x^{-1} N = N = (x^{-1}N)*(xN)$.
	\item associativity: $ (xN)*(yN)*(zN)= (xyN)*(zN) = xyzN =x(yz)N = (xN)*((yN)*(zN))$ by associativity in  $ G$.
\end{enumerate}
\end{prf}
\begin{eg}[]
$ G= \zz, N = 6 \zz$. Note $ N$ is normal because it is a subgroup of an abelian group. Then $ G / N = \zz / 6 \zz$ is the definition of the integers mod 6, $ \zz_6$. It follows that $ \zz_6$ is a group and $ +_{ n} $ is associative.
\[
G /N = \{ 0+ 6 \zz, 1+ 6 \zz, \ldots, 5+6 \zz\} 
.\] 
Then an example is
\[
	(3+6 \zz)+ (5 + 6 \zz) = 8+  6 \zz = 2 + 6 \zz
.\] 
This is because $ 8-2 \in 6 \zz$, so $ 8+ 6 \zz = 2 + 6 \zz$.
\end{eg}

\begin{thm}[fundamental homomorphsim theorem (1st isomorphism theorem]
Let $ \phi: G \to H $ be a homomorphism. Then $ \ker \phi \trianglelefteq G$ and $ \im \phi \leq H$, and
\[
G / \ker \phi \simeq \im \phi
.\] 
Furthermore, an isomorphism is given by 
 \[
	\psi: g( \ker \phi) \to \phi(g) 
.\]
\end{thm}
WARNING: the input of $ \psi$ is coset. So we need to prove that it is a function first. 
\end{document}

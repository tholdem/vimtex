\documentclass[12pt]{article}
%Fall 2020
% Some basic packages
\usepackage{standalone}[subpreambles=true]
\usepackage[utf8]{inputenc}
\usepackage[T1]{fontenc}
\usepackage{textcomp}
\usepackage[english]{babel}
\usepackage{url}
\usepackage{graphicx}
\usepackage{float}
\usepackage{enumitem}


\pdfminorversion=7

% Don't indent paragraphs, leave some space between them
\usepackage{parskip}

% Hide page number when page is empty
\usepackage{emptypage}
\usepackage{subcaption}
\usepackage{multicol}
\usepackage[dvipsnames]{xcolor}


% Math stuff
\usepackage{amsmath, amsfonts, mathtools, amsthm, amssymb}
% Fancy script capitals
\usepackage{mathrsfs}
\usepackage{cancel}
% Bold math
\usepackage{bm}
% Some shortcuts
\newcommand{\rr}{\ensuremath{\mathbb{R}}}
\newcommand{\zz}{\ensuremath{\mathbb{Z}}}
\newcommand{\qq}{\ensuremath{\mathbb{Q}}}
\newcommand{\nn}{\ensuremath{\mathbb{N}}}
\newcommand{\ff}{\ensuremath{\mathbb{F}}}
\newcommand{\cc}{\ensuremath{\mathbb{C}}}
\renewcommand\O{\ensuremath{\emptyset}}
\newcommand{\norm}[1]{{\left\lVert{#1}\right\rVert}}
\renewcommand{\vec}[1]{{\mathbf{#1}}}
\newcommand\allbold[1]{{\boldmath\textbf{#1}}}

% Put x \to \infty below \lim
\let\svlim\lim\def\lim{\svlim\limits}

%Make implies and impliedby shorter
\let\implies\Rightarrow
\let\impliedby\Leftarrow
\let\iff\Leftrightarrow
\let\epsilon\varepsilon

% Add \contra symbol to denote contradiction
\usepackage{stmaryrd} % for \lightning
\newcommand\contra{\scalebox{1.5}{$\lightning$}}

% \let\phi\varphi

% Command for short corrections
% Usage: 1+1=\correct{3}{2}

\definecolor{correct}{HTML}{009900}
\newcommand\correct[2]{\ensuremath{\:}{\color{red}{#1}}\ensuremath{\to }{\color{correct}{#2}}\ensuremath{\:}}
\newcommand\green[1]{{\color{correct}{#1}}}

% horizontal rule
\newcommand\hr{
    \noindent\rule[0.5ex]{\linewidth}{0.5pt}
}

% hide parts
\newcommand\hide[1]{}

% si unitx
\usepackage{siunitx}
\sisetup{locale = FR}

% Environments
\makeatother
% For box around Definition, Theorem, \ldots
\usepackage[framemethod=TikZ]{mdframed}
\mdfsetup{skipabove=1em,skipbelow=0em}

%definition
\newenvironment{defn}[1][]{%
\ifstrempty{#1}%
{\mdfsetup{%
frametitle={%
\tikz[baseline=(current bounding box.east),outer sep=0pt]
\node[anchor=east,rectangle,fill=Emerald]
{\strut Definition};}}
}%
{\mdfsetup{%
frametitle={%
\tikz[baseline=(current bounding box.east),outer sep=0pt]
\node[anchor=east,rectangle,fill=Emerald]
{\strut Definition:~#1};}}%
}%
\mdfsetup{innertopmargin=10pt,linecolor=Emerald,%
linewidth=2pt,topline=true,%
frametitleaboveskip=\dimexpr-\ht\strutbox\relax
}
\begin{mdframed}[]\relax%
\label{#1}}{\end{mdframed}}


%theorem
%\newcounter{thm}[section]\setcounter{thm}{0}
%\renewcommand{\thethm}{\arabic{section}.\arabic{thm}}
\newenvironment{thm}[1][]{%
%\refstepcounter{thm}%
\ifstrempty{#1}%
{\mdfsetup{%
frametitle={%
\tikz[baseline=(current bounding box.east),outer sep=0pt]
\node[anchor=east,rectangle,fill=blue!20]
%{\strut Theorem~\thethm};}}
{\strut Theorem};}}
}%
{\mdfsetup{%
frametitle={%
\tikz[baseline=(current bounding box.east),outer sep=0pt]
\node[anchor=east,rectangle,fill=blue!20]
%{\strut Theorem~\thethm:~#1};}}%
{\strut Theorem:~#1};}}%
}%
\mdfsetup{innertopmargin=10pt,linecolor=blue!20,%
linewidth=2pt,topline=true,%
frametitleaboveskip=\dimexpr-\ht\strutbox\relax
}
\begin{mdframed}[]\relax%
\label{#1}}{\end{mdframed}}


%lemma
\newenvironment{lem}[1][]{%
\ifstrempty{#1}%
{\mdfsetup{%
frametitle={%
\tikz[baseline=(current bounding box.east),outer sep=0pt]
\node[anchor=east,rectangle,fill=Dandelion]
{\strut Lemma};}}
}%
{\mdfsetup{%
frametitle={%
\tikz[baseline=(current bounding box.east),outer sep=0pt]
\node[anchor=east,rectangle,fill=Dandelion]
{\strut Lemma:~#1};}}%
}%
\mdfsetup{innertopmargin=10pt,linecolor=Dandelion,%
linewidth=2pt,topline=true,%
frametitleaboveskip=\dimexpr-\ht\strutbox\relax
}
\begin{mdframed}[]\relax%
\label{#1}}{\end{mdframed}}

%corollary
\newenvironment{coro}[1][]{%
\ifstrempty{#1}%
{\mdfsetup{%
frametitle={%
\tikz[baseline=(current bounding box.east),outer sep=0pt]
\node[anchor=east,rectangle,fill=CornflowerBlue]
{\strut Corollary};}}
}%
{\mdfsetup{%
frametitle={%
\tikz[baseline=(current bounding box.east),outer sep=0pt]
\node[anchor=east,rectangle,fill=CornflowerBlue]
{\strut Corollary:~#1};}}%
}%
\mdfsetup{innertopmargin=10pt,linecolor=CornflowerBlue,%
linewidth=2pt,topline=true,%
frametitleaboveskip=\dimexpr-\ht\strutbox\relax
}
\begin{mdframed}[]\relax%
\label{#1}}{\end{mdframed}}

%proof
\newenvironment{prf}[1][]{%
\ifstrempty{#1}%
{\mdfsetup{%
frametitle={%
\tikz[baseline=(current bounding box.east),outer sep=0pt]
\node[anchor=east,rectangle,fill=SpringGreen]
{\strut Proof};}}
}%
{\mdfsetup{%
frametitle={%
\tikz[baseline=(current bounding box.east),outer sep=0pt]
\node[anchor=east,rectangle,fill=SpringGreen]
{\strut Proof:~#1};}}%
}%
\mdfsetup{innertopmargin=10pt,linecolor=SpringGreen,%
linewidth=2pt,topline=true,%
frametitleaboveskip=\dimexpr-\ht\strutbox\relax
}
\begin{mdframed}[]\relax%
\label{#1}}{\qed\end{mdframed}}


\theoremstyle{definition}

\newmdtheoremenv[nobreak=true]{definition}{Definition}
\newmdtheoremenv[nobreak=true]{prop}{Proposition}
\newmdtheoremenv[nobreak=true]{theorem}{Theorem}
\newmdtheoremenv[nobreak=true]{corollary}{Corollary}
\newtheorem*{eg}{Example}
\theoremstyle{remark}
\newtheorem*{case}{Case}
\newtheorem*{notation}{Notation}
\newtheorem*{remark}{Remark}
\newtheorem*{note}{Note}
\newtheorem*{problem}{Problem}
\newtheorem*{observe}{Observe}
\newtheorem*{property}{Property}
\newtheorem*{intuition}{Intuition}


% End example and intermezzo environments with a small diamond (just like proof
% environments end with a small square)
\usepackage{etoolbox}
\AtEndEnvironment{vb}{\null\hfill$\diamond$}%
\AtEndEnvironment{intermezzo}{\null\hfill$\diamond$}%
% \AtEndEnvironment{opmerking}{\null\hfill$\diamond$}%

% Fix some spacing
% http://tex.stackexchange.com/questions/22119/how-can-i-change-the-spacing-before-theorems-with-amsthm
\makeatletter
\def\thm@space@setup{%
  \thm@preskip=\parskip \thm@postskip=0pt
}

% Fix some stuff
% %http://tex.stackexchange.com/questions/76273/multiple-pdfs-with-page-group-included-in-a-single-page-warning
\pdfsuppresswarningpagegroup=1


% My name
\author{Jaden Wang}



\begin{document}
\centerline {\textsf{\textbf{\LARGE{Homework 4}}}}
\centerline {Jaden Wang}
\vspace{.15in}

\begin{problem}[7.1]
Recall that $ 1$ is a generator of  $ \zz_{12}$. If $ 1$ can be represented by a word of $ \{2,3\} $, then we can just let the  $ \{2,3\} $ representation of $ 1$ generate $ \zz_{12}$. This is indeed possible here:
\[
1 = 3 +_{ 12} 3+_{ 12} 3+_{ 12} 2+_{ 12}  2 
.\]
Therefore, $ \{2,3\} $ should generate the whole $ \zz_{12}$.
\end{problem}
\begin{problem}[7.2]
Since even numbers under $ +_{ 12} $ can only output even numbers, we know that the odd numbers $ \{1,3,5,7,9,11\} $ cannot be in the subgroup. Let's check the rest:
\begin{align*}
	0 &= 6 +_{ 12} 6\\
	2 &= 6 +_{ 12} 4 +_{ 12} 4 \\
	4 &= 4 \\
	6 &= 6 \\
	8 &= 4 +_{ 12} 4 \\
	10 &= 4 +_{ 12} 6 
\end{align*}
Hence, the elements of the subgroup are $ \{0,2,4,6,8,10\} $.
\end{problem}

\begin{problem}[7.7]
~\begin{enumerate}[label=\alph*)]
	\item $ (a^2 b)  a^3 = eaabaaa = a^3 b$
	\item $ (ab)(a^3 b) = eabaaab = a^2$ 
	\item $ b(a^2 b) = ebaab = a^2$
\end{enumerate}
\end{problem}
\begin{problem}[7.8]
	To fill out the table, we simply need to represent $ c$ in terms of  $ a,b$. The graph tells us  $ c = ab = ba$. Hence we have the following: $ aa=e,ab=c,ac=aab=eb=b,ba=c,bb=e,bc=bba=ea = a, ca=b,cb=a,cc=c(ab)=ce=c$. That is,
 \begin{table}[H]
	\centering
	\begin{tabular}{c||c|c|c|c}
		*&e&a&b&c\\
		\hline
		\hline
		e&e&a&b&c\\
		\hline
		a&a&e&c&b\\
		\hline
		b&b&c&e&a\\
		\hline
		c&c&b&a&e
	\end{tabular}
\end{table}
\end{problem}

\begin{problem}[7.10]
Let's represent other elements using $ a,b$:  $c=aa d=ba, f=ab$. Thus we have the following:  $ aa=c,ab=f,ac=aaa=e, ad=aba=b, af = aab = d, ba=d,bb=e,bc=baa=f,bd=bba=a,bf=bab=c,ca=e,cb=d,cc=caa=a,cd=cba=f,cf=cab=b,da=f,db=c,dc=daa=b,dd=dba=e,df=dab=a,fa=b,fb=a,fc=faa=d,fd=fba=c,ff=fab=e$.
\begin{table}[H]
	\centering
	\begin{tabular}{c||c|c|c|c|c|c}
		*&e&a&b&c&d&f\\
		\hline
		\hline
		e&e&a&b&c&d&f\\
		\hline
		a&a&c&f&e&b&d\\
		\hline
		b&b&d&e&f&a&c\\
		\hline
		c&c&e&d&a&f&b\\
		\hline
		d&d&f&c&b&e&a\\
		\hline
		f&f&b&a&d&c&e	
	\end{tabular}
\end{table}
\end{problem}

\begin{problem}[8,1]
\begin{align*}
	\tau \sigma &= \begin{pmatrix} 1&2&3&4&5&6\\2&4&1&3&6&5 \end{pmatrix}  \begin{pmatrix} 1&2&3&4&5&6\\2&4&1&3&6&5 \end{pmatrix}\\ 
	&=   \begin{pmatrix} 1&2&3&4&5&6\\1&2&3&6&5&4 \end{pmatrix} 
\end{align*}
\end{problem}

\begin{problem}[8.2]
\begin{align*}
	\tau^2 \sigma = \tau(\tau \sigma) &=   \begin{pmatrix} 1&2&3&4&5&6\\2&4&1&3&6&5 \end{pmatrix}\begin{pmatrix} 1&2&3&4&5&6\\1&2&3&6&5&4 \end{pmatrix}  \\
	&= \begin{pmatrix} 1&2&3&4&5&6\\2&4&1&5&6&3 \end{pmatrix}  
\end{align*}
\end{problem}

\begin{problem}[8.6]
\begin{align*}
	\sigma^2 &=  \begin{pmatrix} 1&2&3&4&5&6\\3&1&4&5&6&2 \end{pmatrix} \begin{pmatrix} 1&2&3&4&5&6\\3&1&4&5&6&2 \end{pmatrix} \\
	&= \begin{pmatrix} 1&2&3&4&5&6\\4&3&5&6&2&1 \end{pmatrix}  \\
	\sigma^3 &= \begin{pmatrix} 1&2&3&4&5&6\\3&1&4&5&6&2\end{pmatrix} \begin{pmatrix} 1&2&3&4&5&6\\4&3&5&6&2&1 \end{pmatrix}  \\
           &= \begin{pmatrix} 1&2&3&4&5&6\\5&4&6&2&1&3 \end{pmatrix}  \\
	   \sigma^{4}&= \begin{pmatrix} 1&2&3&4&5&6\\3&1&4&5&6&2\end{pmatrix} \begin{pmatrix} 1&2&3&4&5&6\\5&4&6&2&1&3 \end{pmatrix} \\
	   &= \begin{pmatrix} 1&2&3&4&5&6\\6&5&2&1&3&4 \end{pmatrix}  \\
	   \sigma^{5} &= \begin{pmatrix} 1&2&3&4&5&6\\ 3&1&4&5&6&2\end{pmatrix}  \begin{pmatrix} 1&2&3&4&5&6\\6&5&2&1&3&4 \end{pmatrix}  \\
	   &= \begin{pmatrix} 1&2&3&4&5&6\\2&6&1&3&4&5 \end{pmatrix}  \\
	   \sigma^{6} &= \begin{pmatrix} 1&2&3&4&5&6\\3&1&4&5&6&2\end{pmatrix} \begin{pmatrix} 1&2&3&4&5&6\\2&6&1&3&4&5 \end{pmatrix}  \\
	   &= \begin{pmatrix} 1&2&3&4&5&6\\1&2&3&4&5&6 \end{pmatrix}  
\end{align*}
Since $ \sigma^{6}$ is clearly the identity, we know that $ \sigma^{6}= \sigma^{0}$. Therefore, $ |\langle \sigma \rangle| = 6$.
\end{problem}
\begin{problem}[8.9]
Notice
\begin{align*}
	\mu^2 &= \begin{pmatrix} 1&2&3&4&5&6\\5&2&4&3&1&6 \end{pmatrix}  \begin{pmatrix} 1&2&3&4&5&6\\5&2&4&3&1&6 \end{pmatrix}\\ 
	&=   \begin{pmatrix} 1&2&3&4&5&6\\1&2&3&4&5&6 \end{pmatrix} 
\end{align*}
Hence $ \mu^2=\mu^0$. It is easy to see that $ \mu^{100} = \mu^0$ which is the identity above. 
\end{problem}

\begin{problem}[8.20]
~\begin{table}[H]
	\centering
	\begin{tabular}{c||c|c|c|c|c|c}
		& $ \rho^0$ & $ \rho$ & $ \rho^2$ & $ \rho^3$ & $ \rho^{4}$ & $ \rho^{5}$\\
		\hline
		\hline
		$ \rho^0$&$ \rho^0$&$ \rho$&$ \rho^2$&$ \rho^3$&$ \rho^{4}$&$ \rho^{5}$\\
		\hline
	        $ \rho$&$ \rho$&$ \rho^2$&$ \rho^3$&$ \rho^{4}$&$ \rho^{5}$ & $ \rho^{0}$\\
		\hline
		$ \rho^2$ &$ \rho^2$&$ \rho^3$&$ \rho^{4}$&$ \rho^{5}$ & $ \rho^{0}$ & $ \rho$\\
		\hline
$ \rho^3$ &$ \rho^3$&$ \rho^{4}$&$ \rho^{5}$ & $ \rho^{0}$ & $ \rho$&$ \rho^2$\\
\hline
$ \rho^4$ &$ \rho^{4}$&$ \rho^{5}$ & $ \rho^{0}$ & $ \rho$&$ \rho^2$&$ \rho^3$\\
\hline
$ \rho^5$ &$ \rho^{5}$ & $ \rho^{0}$ & $ \rho$&$ \rho^2$&$ \rho^3$&$ \rho^{4}$\\
	\end{tabular}
\end{table}
This cannot be isomorphic to $ S_3$, since it is symmetric about the diagonal and thus abelian yet $ S_3$ is nonabelian.
\end{problem}
\begin{problem}[8.32]
	It suffices to show that $ f_3$ has a two-sided inverse to determine it is a permutation. Let $ g: \rr \to \rr$ where $g(x) = -x^{\frac{1}{3}}$. It is clear that given $ x \in \rr$, $ f_3 \circ g (x) = -\left( -x^{\frac{1}{3}} \right)^3 = x = -\left( -x^3 \right) ^{\frac{1}{3}} = g \circ f_3(x)$. Therefore, $ g(x) = f^{-1}(x)$ thus $ f_3$ is bijective and hence is a permutation.
\end{problem}

\begin{problem}[8.33]
	$ f_4$ is not a permutation because it is not bijective on  $ \rr \to \rr$. Choose $ y=-1 \in \rr$, notice that since $ e^{x}>0$ for all $ x \in \rr$, there is no $ x \in \rr$ such that $ f_4(x) = y<0$. This violates the surjectivity requirement. 
\end{problem}

\begin{problem}[8.34]
	$ f_5$ is not a permutation. $ f_5(x)=x^3-x^2-2x = x(x-2)(x+1)$. We can immediately see that it is not bijective because although $ f_4(0)=f_4(-1)=f_4(2)=0$, we have  $ -1 \neq 0 \neq 2$, violating the injective requirement.
\end{problem}
\end{document}

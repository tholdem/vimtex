\documentclass[class=article,crop=false]{standalone} 
%Fall 2020
% Some basic packages
\usepackage{standalone}[subpreambles=true]
\usepackage[utf8]{inputenc}
\usepackage[T1]{fontenc}
\usepackage{textcomp}
\usepackage[english]{babel}
\usepackage{url}
\usepackage{graphicx}
\usepackage{float}
\usepackage{enumitem}


\pdfminorversion=7

% Don't indent paragraphs, leave some space between them
\usepackage{parskip}

% Hide page number when page is empty
\usepackage{emptypage}
\usepackage{subcaption}
\usepackage{multicol}
\usepackage[dvipsnames]{xcolor}


% Math stuff
\usepackage{amsmath, amsfonts, mathtools, amsthm, amssymb}
% Fancy script capitals
\usepackage{mathrsfs}
\usepackage{cancel}
% Bold math
\usepackage{bm}
% Some shortcuts
\newcommand{\rr}{\ensuremath{\mathbb{R}}}
\newcommand{\zz}{\ensuremath{\mathbb{Z}}}
\newcommand{\qq}{\ensuremath{\mathbb{Q}}}
\newcommand{\nn}{\ensuremath{\mathbb{N}}}
\newcommand{\ff}{\ensuremath{\mathbb{F}}}
\newcommand{\cc}{\ensuremath{\mathbb{C}}}
\renewcommand\O{\ensuremath{\emptyset}}
\newcommand{\norm}[1]{{\left\lVert{#1}\right\rVert}}
\renewcommand{\vec}[1]{{\mathbf{#1}}}
\newcommand\allbold[1]{{\boldmath\textbf{#1}}}

% Put x \to \infty below \lim
\let\svlim\lim\def\lim{\svlim\limits}

%Make implies and impliedby shorter
\let\implies\Rightarrow
\let\impliedby\Leftarrow
\let\iff\Leftrightarrow
\let\epsilon\varepsilon

% Add \contra symbol to denote contradiction
\usepackage{stmaryrd} % for \lightning
\newcommand\contra{\scalebox{1.5}{$\lightning$}}

% \let\phi\varphi

% Command for short corrections
% Usage: 1+1=\correct{3}{2}

\definecolor{correct}{HTML}{009900}
\newcommand\correct[2]{\ensuremath{\:}{\color{red}{#1}}\ensuremath{\to }{\color{correct}{#2}}\ensuremath{\:}}
\newcommand\green[1]{{\color{correct}{#1}}}

% horizontal rule
\newcommand\hr{
    \noindent\rule[0.5ex]{\linewidth}{0.5pt}
}

% hide parts
\newcommand\hide[1]{}

% si unitx
\usepackage{siunitx}
\sisetup{locale = FR}

% Environments
\makeatother
% For box around Definition, Theorem, \ldots
\usepackage[framemethod=TikZ]{mdframed}
\mdfsetup{skipabove=1em,skipbelow=0em}

%definition
\newenvironment{defn}[1][]{%
\ifstrempty{#1}%
{\mdfsetup{%
frametitle={%
\tikz[baseline=(current bounding box.east),outer sep=0pt]
\node[anchor=east,rectangle,fill=Emerald]
{\strut Definition};}}
}%
{\mdfsetup{%
frametitle={%
\tikz[baseline=(current bounding box.east),outer sep=0pt]
\node[anchor=east,rectangle,fill=Emerald]
{\strut Definition:~#1};}}%
}%
\mdfsetup{innertopmargin=10pt,linecolor=Emerald,%
linewidth=2pt,topline=true,%
frametitleaboveskip=\dimexpr-\ht\strutbox\relax
}
\begin{mdframed}[]\relax%
\label{#1}}{\end{mdframed}}


%theorem
%\newcounter{thm}[section]\setcounter{thm}{0}
%\renewcommand{\thethm}{\arabic{section}.\arabic{thm}}
\newenvironment{thm}[1][]{%
%\refstepcounter{thm}%
\ifstrempty{#1}%
{\mdfsetup{%
frametitle={%
\tikz[baseline=(current bounding box.east),outer sep=0pt]
\node[anchor=east,rectangle,fill=blue!20]
%{\strut Theorem~\thethm};}}
{\strut Theorem};}}
}%
{\mdfsetup{%
frametitle={%
\tikz[baseline=(current bounding box.east),outer sep=0pt]
\node[anchor=east,rectangle,fill=blue!20]
%{\strut Theorem~\thethm:~#1};}}%
{\strut Theorem:~#1};}}%
}%
\mdfsetup{innertopmargin=10pt,linecolor=blue!20,%
linewidth=2pt,topline=true,%
frametitleaboveskip=\dimexpr-\ht\strutbox\relax
}
\begin{mdframed}[]\relax%
\label{#1}}{\end{mdframed}}


%lemma
\newenvironment{lem}[1][]{%
\ifstrempty{#1}%
{\mdfsetup{%
frametitle={%
\tikz[baseline=(current bounding box.east),outer sep=0pt]
\node[anchor=east,rectangle,fill=Dandelion]
{\strut Lemma};}}
}%
{\mdfsetup{%
frametitle={%
\tikz[baseline=(current bounding box.east),outer sep=0pt]
\node[anchor=east,rectangle,fill=Dandelion]
{\strut Lemma:~#1};}}%
}%
\mdfsetup{innertopmargin=10pt,linecolor=Dandelion,%
linewidth=2pt,topline=true,%
frametitleaboveskip=\dimexpr-\ht\strutbox\relax
}
\begin{mdframed}[]\relax%
\label{#1}}{\end{mdframed}}

%corollary
\newenvironment{coro}[1][]{%
\ifstrempty{#1}%
{\mdfsetup{%
frametitle={%
\tikz[baseline=(current bounding box.east),outer sep=0pt]
\node[anchor=east,rectangle,fill=CornflowerBlue]
{\strut Corollary};}}
}%
{\mdfsetup{%
frametitle={%
\tikz[baseline=(current bounding box.east),outer sep=0pt]
\node[anchor=east,rectangle,fill=CornflowerBlue]
{\strut Corollary:~#1};}}%
}%
\mdfsetup{innertopmargin=10pt,linecolor=CornflowerBlue,%
linewidth=2pt,topline=true,%
frametitleaboveskip=\dimexpr-\ht\strutbox\relax
}
\begin{mdframed}[]\relax%
\label{#1}}{\end{mdframed}}

%proof
\newenvironment{prf}[1][]{%
\ifstrempty{#1}%
{\mdfsetup{%
frametitle={%
\tikz[baseline=(current bounding box.east),outer sep=0pt]
\node[anchor=east,rectangle,fill=SpringGreen]
{\strut Proof};}}
}%
{\mdfsetup{%
frametitle={%
\tikz[baseline=(current bounding box.east),outer sep=0pt]
\node[anchor=east,rectangle,fill=SpringGreen]
{\strut Proof:~#1};}}%
}%
\mdfsetup{innertopmargin=10pt,linecolor=SpringGreen,%
linewidth=2pt,topline=true,%
frametitleaboveskip=\dimexpr-\ht\strutbox\relax
}
\begin{mdframed}[]\relax%
\label{#1}}{\qed\end{mdframed}}


\theoremstyle{definition}

\newmdtheoremenv[nobreak=true]{definition}{Definition}
\newmdtheoremenv[nobreak=true]{prop}{Proposition}
\newmdtheoremenv[nobreak=true]{theorem}{Theorem}
\newmdtheoremenv[nobreak=true]{corollary}{Corollary}
\newtheorem*{eg}{Example}
\theoremstyle{remark}
\newtheorem*{case}{Case}
\newtheorem*{notation}{Notation}
\newtheorem*{remark}{Remark}
\newtheorem*{note}{Note}
\newtheorem*{problem}{Problem}
\newtheorem*{observe}{Observe}
\newtheorem*{property}{Property}
\newtheorem*{intuition}{Intuition}


% End example and intermezzo environments with a small diamond (just like proof
% environments end with a small square)
\usepackage{etoolbox}
\AtEndEnvironment{vb}{\null\hfill$\diamond$}%
\AtEndEnvironment{intermezzo}{\null\hfill$\diamond$}%
% \AtEndEnvironment{opmerking}{\null\hfill$\diamond$}%

% Fix some spacing
% http://tex.stackexchange.com/questions/22119/how-can-i-change-the-spacing-before-theorems-with-amsthm
\makeatletter
\def\thm@space@setup{%
  \thm@preskip=\parskip \thm@postskip=0pt
}

% Fix some stuff
% %http://tex.stackexchange.com/questions/76273/multiple-pdfs-with-page-group-included-in-a-single-page-warning
\pdfsuppresswarningpagegroup=1


% My name
\author{Jaden Wang}



\begin{document}
$ x$ commutes with everything.
Multiplication:
\[
	\left( \sum_{ i= 0}^{ n} a_i x^{i} \right) \left( \sum_{ j= 0}^{ n} b_j x^{j} \right) = \sum_{ k= 0}^{ n} c_k x^{k} 
\]
where 
\[
c_k = \sum_{ i= 0}^{ k} a_i b_{k-i}
.\] 

\begin{eg}[]
The coefficient of $ x^3$ is contributed by terms with degree up to 3. So $ c_3 = a_0 b_3 + a_1 b_2 + a_2 b_1 + a_3 b_0$.
\end{eg}

\begin{prop}[]
	The zero element in $ R[x]$ is the zero polynomial  $ 0_R$. If  $ R$ has identity, then so does  $ R[x]$: constant polynomial  $ 1_R$. If  $ R$ has zero divisors, then so does  $ R[x]$.
\end{prop}

\begin{eg}[]
	$ R=\zz_6, \zz_6[x]$. $ 2\times 3=0$, then think of these as constant polynomials, so they are zero divisors of $ R[x]$.
\end{eg}

\begin{thm}[]
	If $ R$ is a domain, then  $ R[x]$ is a domain.
\end{thm}
\begin{prf}
\begin{enumerate}[label=(\roman*)]
	\item $ R[x]$ is commutative, we can show by doing multiplication the other way around.
	\item  $ R[x]$ has identity: we can check  $ 1_R$ is the identity.
	\item  $ R[x]$ has no zero divisors: Suppose  $ f(x),g(x) \in R[x]$ and $ f(x) \neq 0, g(x)\neq 0$. We need to show that  $ f(x)g(x)\neq 0$.

		Let  $ f(x)$ have degree  $ n$ and  $ g(x)$ have degree  $ m$. So  $ f(x)=a_n x^{n}+ \ldots, a_n\neq 0$, $ g(x) = b_m x^{m}+ \ldots, b_n \neq 0$. So
		\[
			f(x)g(x)=a_n b_m x^{m+n}+\ldots
		.\] 
		Then $ a_n b_m \neq 0$ because $ R$ is a domain and  $ a_n, b_m \neq 0$.
\end{enumerate}
\end{prf}

\begin{coro}[]
	If $ R$ is a domain and  $ f(x),g(x) \in R[x]$ are nonzero, then deg$(f(x)g(x)) $ = deg$f(x) $+deg$g(x) $.
\end{coro}

\begin{eg}[counterexample when $ R$ is not a domain]
	$ \zz_6[x]$. Consider
	\begin{align*}
		(1+2x)(1-x+3x^2) &= 1-x+3x^2+2x-2x^2 +6x^3 \\
		&= 1+x+x^2 \\
	\end{align*}

	Is $ x^{6} = x^{0}$? No because coefficient of $ x^{6}$ is 1 but that of $ x^{0}$ is 0. 
	\begin{note}[]
	The exponents are not ring elements. They are natural numbers.
	\end{note}
\end{eg}

What are the units in $ R[x]$? 

If  $ R$ is not a domain, there is no easy answer.

If  $ R$ is a domain, the units in  $ R[x]$ are the constant units in  $ R$.

 \begin{eg}[]
	 $In \qq[x]$, the units are the nonzero constants because units in $ \qq $ are $ \qq \setminus \{0\} $.
\end{eg}
\begin{eg}[]
	The units in $ \zz_5[x]$ are $ \{1,2,3,4\} $. The units in $ \zz[x]$ are $ \pm 1$.
\end{eg}

\begin{thm}[]
	If $ R$ is a domain, then  $ U(R[x])=U(R)$.
\end{thm}

\begin{prf}
	Suppose $ f(x)g(x)=1$ (they are units). The degree 0 polynomials are the nonzero constant polynomials. The degree of 1 is 0, and equals to the sum of degrees of  $ f,g$, thus  $ f,g$ has degree 0 too. This means that  $ f(x),g(x)$ are nonzero constants, so they are units in  $ R$.
\end{prf}
\end{document}

\documentclass[class=article,crop=false]{standalone} 
%Fall 2020
% Some basic packages
\usepackage{standalone}[subpreambles=true]
\usepackage[utf8]{inputenc}
\usepackage[T1]{fontenc}
\usepackage{textcomp}
\usepackage[english]{babel}
\usepackage{url}
\usepackage{graphicx}
\usepackage{float}
\usepackage{enumitem}


\pdfminorversion=7

% Don't indent paragraphs, leave some space between them
\usepackage{parskip}

% Hide page number when page is empty
\usepackage{emptypage}
\usepackage{subcaption}
\usepackage{multicol}
\usepackage[dvipsnames]{xcolor}


% Math stuff
\usepackage{amsmath, amsfonts, mathtools, amsthm, amssymb}
% Fancy script capitals
\usepackage{mathrsfs}
\usepackage{cancel}
% Bold math
\usepackage{bm}
% Some shortcuts
\newcommand{\rr}{\ensuremath{\mathbb{R}}}
\newcommand{\zz}{\ensuremath{\mathbb{Z}}}
\newcommand{\qq}{\ensuremath{\mathbb{Q}}}
\newcommand{\nn}{\ensuremath{\mathbb{N}}}
\newcommand{\ff}{\ensuremath{\mathbb{F}}}
\newcommand{\cc}{\ensuremath{\mathbb{C}}}
\renewcommand\O{\ensuremath{\emptyset}}
\newcommand{\norm}[1]{{\left\lVert{#1}\right\rVert}}
\renewcommand{\vec}[1]{{\mathbf{#1}}}
\newcommand\allbold[1]{{\boldmath\textbf{#1}}}

% Put x \to \infty below \lim
\let\svlim\lim\def\lim{\svlim\limits}

%Make implies and impliedby shorter
\let\implies\Rightarrow
\let\impliedby\Leftarrow
\let\iff\Leftrightarrow
\let\epsilon\varepsilon

% Add \contra symbol to denote contradiction
\usepackage{stmaryrd} % for \lightning
\newcommand\contra{\scalebox{1.5}{$\lightning$}}

% \let\phi\varphi

% Command for short corrections
% Usage: 1+1=\correct{3}{2}

\definecolor{correct}{HTML}{009900}
\newcommand\correct[2]{\ensuremath{\:}{\color{red}{#1}}\ensuremath{\to }{\color{correct}{#2}}\ensuremath{\:}}
\newcommand\green[1]{{\color{correct}{#1}}}

% horizontal rule
\newcommand\hr{
    \noindent\rule[0.5ex]{\linewidth}{0.5pt}
}

% hide parts
\newcommand\hide[1]{}

% si unitx
\usepackage{siunitx}
\sisetup{locale = FR}

% Environments
\makeatother
% For box around Definition, Theorem, \ldots
\usepackage[framemethod=TikZ]{mdframed}
\mdfsetup{skipabove=1em,skipbelow=0em}

%definition
\newenvironment{defn}[1][]{%
\ifstrempty{#1}%
{\mdfsetup{%
frametitle={%
\tikz[baseline=(current bounding box.east),outer sep=0pt]
\node[anchor=east,rectangle,fill=Emerald]
{\strut Definition};}}
}%
{\mdfsetup{%
frametitle={%
\tikz[baseline=(current bounding box.east),outer sep=0pt]
\node[anchor=east,rectangle,fill=Emerald]
{\strut Definition:~#1};}}%
}%
\mdfsetup{innertopmargin=10pt,linecolor=Emerald,%
linewidth=2pt,topline=true,%
frametitleaboveskip=\dimexpr-\ht\strutbox\relax
}
\begin{mdframed}[]\relax%
\label{#1}}{\end{mdframed}}


%theorem
%\newcounter{thm}[section]\setcounter{thm}{0}
%\renewcommand{\thethm}{\arabic{section}.\arabic{thm}}
\newenvironment{thm}[1][]{%
%\refstepcounter{thm}%
\ifstrempty{#1}%
{\mdfsetup{%
frametitle={%
\tikz[baseline=(current bounding box.east),outer sep=0pt]
\node[anchor=east,rectangle,fill=blue!20]
%{\strut Theorem~\thethm};}}
{\strut Theorem};}}
}%
{\mdfsetup{%
frametitle={%
\tikz[baseline=(current bounding box.east),outer sep=0pt]
\node[anchor=east,rectangle,fill=blue!20]
%{\strut Theorem~\thethm:~#1};}}%
{\strut Theorem:~#1};}}%
}%
\mdfsetup{innertopmargin=10pt,linecolor=blue!20,%
linewidth=2pt,topline=true,%
frametitleaboveskip=\dimexpr-\ht\strutbox\relax
}
\begin{mdframed}[]\relax%
\label{#1}}{\end{mdframed}}


%lemma
\newenvironment{lem}[1][]{%
\ifstrempty{#1}%
{\mdfsetup{%
frametitle={%
\tikz[baseline=(current bounding box.east),outer sep=0pt]
\node[anchor=east,rectangle,fill=Dandelion]
{\strut Lemma};}}
}%
{\mdfsetup{%
frametitle={%
\tikz[baseline=(current bounding box.east),outer sep=0pt]
\node[anchor=east,rectangle,fill=Dandelion]
{\strut Lemma:~#1};}}%
}%
\mdfsetup{innertopmargin=10pt,linecolor=Dandelion,%
linewidth=2pt,topline=true,%
frametitleaboveskip=\dimexpr-\ht\strutbox\relax
}
\begin{mdframed}[]\relax%
\label{#1}}{\end{mdframed}}

%corollary
\newenvironment{coro}[1][]{%
\ifstrempty{#1}%
{\mdfsetup{%
frametitle={%
\tikz[baseline=(current bounding box.east),outer sep=0pt]
\node[anchor=east,rectangle,fill=CornflowerBlue]
{\strut Corollary};}}
}%
{\mdfsetup{%
frametitle={%
\tikz[baseline=(current bounding box.east),outer sep=0pt]
\node[anchor=east,rectangle,fill=CornflowerBlue]
{\strut Corollary:~#1};}}%
}%
\mdfsetup{innertopmargin=10pt,linecolor=CornflowerBlue,%
linewidth=2pt,topline=true,%
frametitleaboveskip=\dimexpr-\ht\strutbox\relax
}
\begin{mdframed}[]\relax%
\label{#1}}{\end{mdframed}}

%proof
\newenvironment{prf}[1][]{%
\ifstrempty{#1}%
{\mdfsetup{%
frametitle={%
\tikz[baseline=(current bounding box.east),outer sep=0pt]
\node[anchor=east,rectangle,fill=SpringGreen]
{\strut Proof};}}
}%
{\mdfsetup{%
frametitle={%
\tikz[baseline=(current bounding box.east),outer sep=0pt]
\node[anchor=east,rectangle,fill=SpringGreen]
{\strut Proof:~#1};}}%
}%
\mdfsetup{innertopmargin=10pt,linecolor=SpringGreen,%
linewidth=2pt,topline=true,%
frametitleaboveskip=\dimexpr-\ht\strutbox\relax
}
\begin{mdframed}[]\relax%
\label{#1}}{\qed\end{mdframed}}


\theoremstyle{definition}

\newmdtheoremenv[nobreak=true]{definition}{Definition}
\newmdtheoremenv[nobreak=true]{prop}{Proposition}
\newmdtheoremenv[nobreak=true]{theorem}{Theorem}
\newmdtheoremenv[nobreak=true]{corollary}{Corollary}
\newtheorem*{eg}{Example}
\theoremstyle{remark}
\newtheorem*{case}{Case}
\newtheorem*{notation}{Notation}
\newtheorem*{remark}{Remark}
\newtheorem*{note}{Note}
\newtheorem*{problem}{Problem}
\newtheorem*{observe}{Observe}
\newtheorem*{property}{Property}
\newtheorem*{intuition}{Intuition}


% End example and intermezzo environments with a small diamond (just like proof
% environments end with a small square)
\usepackage{etoolbox}
\AtEndEnvironment{vb}{\null\hfill$\diamond$}%
\AtEndEnvironment{intermezzo}{\null\hfill$\diamond$}%
% \AtEndEnvironment{opmerking}{\null\hfill$\diamond$}%

% Fix some spacing
% http://tex.stackexchange.com/questions/22119/how-can-i-change-the-spacing-before-theorems-with-amsthm
\makeatletter
\def\thm@space@setup{%
  \thm@preskip=\parskip \thm@postskip=0pt
}

% Fix some stuff
% %http://tex.stackexchange.com/questions/76273/multiple-pdfs-with-page-group-included-in-a-single-page-warning
\pdfsuppresswarningpagegroup=1


% My name
\author{Jaden Wang}



\begin{document}
\section{}
\begin{eg}[]
	Consider $(\zz,+)$, $\langle 5 \rangle = \{ \ldots, -5,0,5,10,\ldots\} $. This is called $5\zz$ (integer multiple of 5). Note that this is not $\zz_5$. The latter doesn't even have the same operation.
\end{eg}

Is $\zz$ generated by 5? No. But 1 would do.

$\zz = \langle 1 \rangle = \langle -1 \rangle$.

\begin{lem}[]
The inverse of a generator of a group is also a generator.
\end{lem}

Is $\zz$ cyclic? Yes, it is generated by 1 (or -1).

$5\zz$ is a cyclic group generated by 5. $\zz_5$ is generated by 1.

Is $\zz_n$ cyclic? Yes it is generated by 1. 

\begin{thm}[]
	Any cyclic group is either isomorphic to $(\zz_n,+_{n} )$ or to $(\zz,+)$.
\end{thm}

Question: Is $(\rr,+)$ cyclic?

No. Since $\rr$ is uncountable, so there is no bijection between $\rr$ and $\zz$ or $\zz_n$.

\begin{defn}[greatest common divisors (gcd)]

\end{defn}

\begin{note}[]
The gcd of $a,b$ can be written as  $ra+sb$ with  $r,s \in \zz$.
\end{note}

\begin{eg}[]
	$28r+40s=4 \implies 28 \times 3 + 40 \times (-2) =4$.
\end{eg}

\begin{eg}[]
	In $\zz_{40}$, what is $\langle 28 \rangle$?

	This is controlled by the gcd(28,40). The key is $r=3$. What else is in $\langle 28 \rangle$? $\{0, 28, 16,4\} $. So $4 \in \langle 28 \rangle$. Then we have $\{0,4,8,\ldots,36\} $ with $\frac{40}{4}=10$ elements !

	In $\zz_{40}$, $\langle 28 \rangle=\langle 4 \rangle$.
\end{eg}

\begin{thm}[]
	In $\zz_n$, the subgroup $\langle r \rangle$ is equal to $\langle d \rangle$, where $d$=gcd$(r,n)$. Then number of elements in  $\langle d \rangle$ is $\frac{n}{d}=\frac{n}{ \gcd (r,n) }$.
\end{thm}

\begin{thm}[]
Every subgroup of a cyclic group is cyclic.
\end{thm}

\begin{coro}[]
Every subgroup of $\zz_n$ is of form $\langle r \rangle$, and in fact we can take $r$ to be a divisor of  $n$. 
\end{coro}

\begin{eg}[]
	What are the subgroups of $\zz_{18}$?

	We just need to choose an appropriate generator from the divisors of $18$. 
	$\langle 1 \rangle = \zz_{18}$\\
	$\langle 2 \rangle = \{0,2,4,\ldots\} $ 9 elements.\\
	$\langle 3 \rangle = \{0,3,\ldots\} $ 6 elements.\\
	$\langle 6 \rangle = \{0,6,12\} $ 3 elements.\\
	$\langle 9 \rangle = \{0,9\} $ 2 elements.\\
	$\langle 18 \rangle = \{18\} $ 1 element.\\
	$\langle 10 \rangle=\langle 2 \rangle$.\\
	$\langle 7 \rangle = \langle 1 \rangle$ because 7 and 18 are coprime.\\

	See iPad for subgroup lattice.
\end{eg}

\begin{eg}[]
Subgroup lattice of $\zz_4$. See iPad.
\end{eg}

\begin{notation}
	Let $(G,*)$ be a group, and let  $g \in G$. For multiplication, we define $g^2 = g*g$,\ldots,$g^{n}=g*\ldots*g$ with $n$ occurrence of  $g$s.  $g^{0}=e$. $g^{-1}$ is the inverse. $g^{-2}= \left( g^{-1} \right)^2 = \left( g^2 \right) ^{-1}$ (this is easy to check). $g^{-n} = \left( g^{-1} \right) ^{n} = \left( g^{n} \right) ^{-1}$. 

	The subgroup $\langle g \rangle$ is given by 
	\[
	\left\{ g^{n}: n \in \zz \right\} 
	.\] 
It is true that $g^{m}*g^{n}=g^{m+n}$ for all $m,n \in \zz$. Caution: $m,n$ are not elements of  $G$.

If the operation is addition. we write $2g = g+g, 0g=e, -1g=g^{-1}$,\ldots
then 
\[
\langle g \rangle = \left\{ ng:n \in \zz \right\} 
.\] 

\end{notation}

\begin{thm}[]
	Every cyclic group is abelian.
\end{thm}
\begin{note}[]
The converse is false. $ V_4$ is a counterexample.
\end{note}

\begin{prf}
If $ G$ is cyclic then $ G=\left\{ g^{n}:n \in \zz \right\} $. For some generator $ g$, let  $ x,y \in G$. Then $ x=g^{n}$ and $ y=g^{m}$. Then
\[
	x*y = g^{n} * g^{m} = g^{n+m} = g^{m+n} = g^{m} * g^{n} = y*x
.\]
So $ G$ is abelian.
\end{prf}

\end{document}

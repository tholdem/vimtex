\documentclass[12pt]{article}
%Fall 2020
% Some basic packages
\usepackage{standalone}[subpreambles=true]
\usepackage[utf8]{inputenc}
\usepackage[T1]{fontenc}
\usepackage{textcomp}
\usepackage[english]{babel}
\usepackage{url}
\usepackage{graphicx}
\usepackage{float}
\usepackage{enumitem}


\pdfminorversion=7

% Don't indent paragraphs, leave some space between them
\usepackage{parskip}

% Hide page number when page is empty
\usepackage{emptypage}
\usepackage{subcaption}
\usepackage{multicol}
\usepackage[dvipsnames]{xcolor}


% Math stuff
\usepackage{amsmath, amsfonts, mathtools, amsthm, amssymb}
% Fancy script capitals
\usepackage{mathrsfs}
\usepackage{cancel}
% Bold math
\usepackage{bm}
% Some shortcuts
\newcommand{\rr}{\ensuremath{\mathbb{R}}}
\newcommand{\zz}{\ensuremath{\mathbb{Z}}}
\newcommand{\qq}{\ensuremath{\mathbb{Q}}}
\newcommand{\nn}{\ensuremath{\mathbb{N}}}
\newcommand{\ff}{\ensuremath{\mathbb{F}}}
\newcommand{\cc}{\ensuremath{\mathbb{C}}}
\renewcommand\O{\ensuremath{\emptyset}}
\newcommand{\norm}[1]{{\left\lVert{#1}\right\rVert}}
\renewcommand{\vec}[1]{{\mathbf{#1}}}
\newcommand\allbold[1]{{\boldmath\textbf{#1}}}

% Put x \to \infty below \lim
\let\svlim\lim\def\lim{\svlim\limits}

%Make implies and impliedby shorter
\let\implies\Rightarrow
\let\impliedby\Leftarrow
\let\iff\Leftrightarrow
\let\epsilon\varepsilon

% Add \contra symbol to denote contradiction
\usepackage{stmaryrd} % for \lightning
\newcommand\contra{\scalebox{1.5}{$\lightning$}}

% \let\phi\varphi

% Command for short corrections
% Usage: 1+1=\correct{3}{2}

\definecolor{correct}{HTML}{009900}
\newcommand\correct[2]{\ensuremath{\:}{\color{red}{#1}}\ensuremath{\to }{\color{correct}{#2}}\ensuremath{\:}}
\newcommand\green[1]{{\color{correct}{#1}}}

% horizontal rule
\newcommand\hr{
    \noindent\rule[0.5ex]{\linewidth}{0.5pt}
}

% hide parts
\newcommand\hide[1]{}

% si unitx
\usepackage{siunitx}
\sisetup{locale = FR}

% Environments
\makeatother
% For box around Definition, Theorem, \ldots
\usepackage[framemethod=TikZ]{mdframed}
\mdfsetup{skipabove=1em,skipbelow=0em}

%definition
\newenvironment{defn}[1][]{%
\ifstrempty{#1}%
{\mdfsetup{%
frametitle={%
\tikz[baseline=(current bounding box.east),outer sep=0pt]
\node[anchor=east,rectangle,fill=Emerald]
{\strut Definition};}}
}%
{\mdfsetup{%
frametitle={%
\tikz[baseline=(current bounding box.east),outer sep=0pt]
\node[anchor=east,rectangle,fill=Emerald]
{\strut Definition:~#1};}}%
}%
\mdfsetup{innertopmargin=10pt,linecolor=Emerald,%
linewidth=2pt,topline=true,%
frametitleaboveskip=\dimexpr-\ht\strutbox\relax
}
\begin{mdframed}[]\relax%
\label{#1}}{\end{mdframed}}


%theorem
%\newcounter{thm}[section]\setcounter{thm}{0}
%\renewcommand{\thethm}{\arabic{section}.\arabic{thm}}
\newenvironment{thm}[1][]{%
%\refstepcounter{thm}%
\ifstrempty{#1}%
{\mdfsetup{%
frametitle={%
\tikz[baseline=(current bounding box.east),outer sep=0pt]
\node[anchor=east,rectangle,fill=blue!20]
%{\strut Theorem~\thethm};}}
{\strut Theorem};}}
}%
{\mdfsetup{%
frametitle={%
\tikz[baseline=(current bounding box.east),outer sep=0pt]
\node[anchor=east,rectangle,fill=blue!20]
%{\strut Theorem~\thethm:~#1};}}%
{\strut Theorem:~#1};}}%
}%
\mdfsetup{innertopmargin=10pt,linecolor=blue!20,%
linewidth=2pt,topline=true,%
frametitleaboveskip=\dimexpr-\ht\strutbox\relax
}
\begin{mdframed}[]\relax%
\label{#1}}{\end{mdframed}}


%lemma
\newenvironment{lem}[1][]{%
\ifstrempty{#1}%
{\mdfsetup{%
frametitle={%
\tikz[baseline=(current bounding box.east),outer sep=0pt]
\node[anchor=east,rectangle,fill=Dandelion]
{\strut Lemma};}}
}%
{\mdfsetup{%
frametitle={%
\tikz[baseline=(current bounding box.east),outer sep=0pt]
\node[anchor=east,rectangle,fill=Dandelion]
{\strut Lemma:~#1};}}%
}%
\mdfsetup{innertopmargin=10pt,linecolor=Dandelion,%
linewidth=2pt,topline=true,%
frametitleaboveskip=\dimexpr-\ht\strutbox\relax
}
\begin{mdframed}[]\relax%
\label{#1}}{\end{mdframed}}

%corollary
\newenvironment{coro}[1][]{%
\ifstrempty{#1}%
{\mdfsetup{%
frametitle={%
\tikz[baseline=(current bounding box.east),outer sep=0pt]
\node[anchor=east,rectangle,fill=CornflowerBlue]
{\strut Corollary};}}
}%
{\mdfsetup{%
frametitle={%
\tikz[baseline=(current bounding box.east),outer sep=0pt]
\node[anchor=east,rectangle,fill=CornflowerBlue]
{\strut Corollary:~#1};}}%
}%
\mdfsetup{innertopmargin=10pt,linecolor=CornflowerBlue,%
linewidth=2pt,topline=true,%
frametitleaboveskip=\dimexpr-\ht\strutbox\relax
}
\begin{mdframed}[]\relax%
\label{#1}}{\end{mdframed}}

%proof
\newenvironment{prf}[1][]{%
\ifstrempty{#1}%
{\mdfsetup{%
frametitle={%
\tikz[baseline=(current bounding box.east),outer sep=0pt]
\node[anchor=east,rectangle,fill=SpringGreen]
{\strut Proof};}}
}%
{\mdfsetup{%
frametitle={%
\tikz[baseline=(current bounding box.east),outer sep=0pt]
\node[anchor=east,rectangle,fill=SpringGreen]
{\strut Proof:~#1};}}%
}%
\mdfsetup{innertopmargin=10pt,linecolor=SpringGreen,%
linewidth=2pt,topline=true,%
frametitleaboveskip=\dimexpr-\ht\strutbox\relax
}
\begin{mdframed}[]\relax%
\label{#1}}{\qed\end{mdframed}}


\theoremstyle{definition}

\newmdtheoremenv[nobreak=true]{definition}{Definition}
\newmdtheoremenv[nobreak=true]{prop}{Proposition}
\newmdtheoremenv[nobreak=true]{theorem}{Theorem}
\newmdtheoremenv[nobreak=true]{corollary}{Corollary}
\newtheorem*{eg}{Example}
\theoremstyle{remark}
\newtheorem*{case}{Case}
\newtheorem*{notation}{Notation}
\newtheorem*{remark}{Remark}
\newtheorem*{note}{Note}
\newtheorem*{problem}{Problem}
\newtheorem*{observe}{Observe}
\newtheorem*{property}{Property}
\newtheorem*{intuition}{Intuition}


% End example and intermezzo environments with a small diamond (just like proof
% environments end with a small square)
\usepackage{etoolbox}
\AtEndEnvironment{vb}{\null\hfill$\diamond$}%
\AtEndEnvironment{intermezzo}{\null\hfill$\diamond$}%
% \AtEndEnvironment{opmerking}{\null\hfill$\diamond$}%

% Fix some spacing
% http://tex.stackexchange.com/questions/22119/how-can-i-change-the-spacing-before-theorems-with-amsthm
\makeatletter
\def\thm@space@setup{%
  \thm@preskip=\parskip \thm@postskip=0pt
}

% Fix some stuff
% %http://tex.stackexchange.com/questions/76273/multiple-pdfs-with-page-group-included-in-a-single-page-warning
\pdfsuppresswarningpagegroup=1


% My name
\author{Jaden Wang}



\begin{document}
\centerline {\textsf{\textbf{\LARGE{Homework 7}}}}
\centerline {Jaden Wang}
\vspace{.15in}

\begin{problem}[13.2]
Not a homomorphism. Consider $ 1.5, 0.5 \in \rr$, then
\[
	\phi(1.5+0.5) = \phi(2)=2 \neq 1 = 1+0= \phi(1.5) + \phi(0.5)
.\] 
\end{problem}

\begin{problem}[13.3]
Yes. Given $ a,b \in \rr^* $, 
\[
\phi(ab) = |ab| = |a||b| = \phi(a) \phi(b)   
.\] 
\end{problem}
\begin{problem}[13.4]
Given $ a,b \in \zz_6$,
\[
	\phi(a+_{ 6} b) = (a+_{ 6} b) \mod 2
.\]
Note $ \mod 2$ is a homomorphism, so we get
\[
	= (a \mod 2) +_{ 2} (b \mod 2) = \phi(a) +_{ 2} \phi(b)  
.\]
\end{problem}

\begin{problem}[13.5]
No. Notice
\[
\phi(2+_{ 9} 7) = \phi(0) = 0 \neq 1 = 0+_{ 2} 1 = \phi(2)+_{ 2} \phi(7)      
.\] 
\end{problem}

\begin{problem}[13.6]
Yes. Given $ a,b \in \rr$,
\[
\phi(a+b)= 2^{a+b}=2^{a} \cdot 2^{b} = \phi(a) \phi(b)   
.\] 
\end{problem}

\begin{problem}[13.8]
No. Consider $ \mu_1, \rho_2 \in D_3$,
\[
	\phi(\mu_1 \rho_2) = (\mu_1 \rho_2)^{-1} = \rho_2 ^{-1} \mu_1 ^{-1} = \rho_1 \mu_1 = \mu_3 
.\] 
However,
\[
	\phi(\mu_1) \phi(\rho_2) = \mu_1 ^{-1} \rho_2 ^{-1} = \mu_1 \rho_1 = \mu_2 
\] 
and they clearly don't equal to each other.
\end{problem}
\begin{problem}[13.12]
	No. Consider $ \begin{pmatrix} 1&0\\0&0 \end{pmatrix}, \begin{pmatrix} 0&0\\0&1 \end{pmatrix} \in M_2 $,
	\[
		\phi\left(  \begin{pmatrix} 1&0\\0&0 \end{pmatrix} + \begin{pmatrix} 0&0\\0&1 \end{pmatrix}  \right) = \det \begin{pmatrix}  1&0\\0&1  \end{pmatrix} = 1
	.\] 
	However,
	\[
		\phi \left(\begin{pmatrix} 1&0\\0&0 \end{pmatrix}\right) + \phi \left(\begin{pmatrix} 0&0\\0&1 \end{pmatrix} \right) = \det \begin{pmatrix} 1&0\\0&0 \end{pmatrix} + \det \begin{pmatrix} 0&0\\0&1 \end{pmatrix} = 0+0=0
	.\] 
	And they clearly don't equal.
\end{problem}
\begin{problem}[13.13]
Yes. Given $ A,B \in M_n$, consider the diagonal elements $ A_{ii} $ and $ B_{ii} $, where $ i=1,\ldots,n$. Since matrix addition is elementwise, we have
\[
	(A+B)_{ii}=A_{ii} + B_{ii} 
.\] 
Recall 
\begin{align*}
	\tr(A+B) &= \sum_{ i= 1}^{ n} (A+B)_{ii}\\
		 &= \sum_{ i= 1}^{ n} (A_{ii} + B_{ii} )\\
		 &= \sum_{ i= 1}^{ n} A_{ii} + \sum_{ i= 1}^{ n} B_{ii} \\
		 &= \tr A + \tr B \\
\end{align*}
Therefore,
\[
	\phi(A+B)=\tr (A+B) = \tr A + \tr B = \phi(A) + \phi(B)   
.\] 
\end{problem}

\begin{problem}[13.17]
The identity of both groups is 0. The kernel here is
\[
\ker \phi = \{x \in \zz: \phi(x) = 0 \} 
.\] 
Since $ \zz_7$ is a cyclic group, and since $ \gcd ( 4,7)=1 $, we expect $ \phi(1)=4 $ to be a generator of $ \zz_7$ with order 7. Therefore, we expect to arrive at the identity after adding $ 7n$ numbers of $ \phi(1)$ where $ n \in \zz$. Since $ \phi$ is a homomorphism, $7n \cdot_7 \phi(1)= \phi(7n) = 0$. Therefore,
\[
\ker \phi = \{7n: n \in \zz\} = 7 \zz 
.\] 
Again by homomorphism of $ \phi$ and $ \mod 7$, 
\[
\phi(25) = 25 \cdot_7 \phi(1) = 25 \cdot \phi(1) \mod 7 = 25 \cdot 4 \mod 7 = 2   
.\] 
\end{problem}

\begin{problem}[13.18]
By similar argument as above, we have $ \gcd ( 6,10) = 2 $, so we expect $ \phi(1)=6 $ to have order $ \frac{10}{2}=5$. Hence $ 5 n \cdot_{10} \phi(1) = \phi(5n) =0 $ by homomorphism. It follows that
\[
\ker \phi = 5 \zz 
.\]
By homomorphism of $ \phi$ and $ \mod 10$,
\begin{align*}
	\phi(18) &= 18 \cdot_{10} \phi(1)  \\
	&= 18 \cdot 6 \mod 10   \\
	&=  8
\end{align*}
\end{problem}

\begin{problem}[13.25]
	There are two: $ \phi(x) =x $ and $ \phi(x) = -x $. The identity function is clearly a bijective homomorphism. For the latter function, given $ x,y \in \zz$, $ \phi(x+y) = -(x+y) = -x - y = \phi(x) + \phi(y) $ so it is a homomorphism. We can also find $ \phi(x)^{-1} = -x $ so it is also bijective.
\end{problem}

\begin{problem}[13.26]
There are infinitely many. By Corollary 13.18, if $ \ker \phi = \{e\} $, then the homomorphism $ \phi $ is injective. Let  $ \phi(1) = n $, where $ |n| \geq 1$ and  $ n \in \zz$. Then since $1$ is a generator of $ \zz$, using homomorphism we can generate a set in $ \zz$ using $ \phi(1) $. And since $ |\phi(1)|\geq 1 $, we know that $ \phi(1)^{k} \neq 0 \ \forall \ k \in \zz^* $. Hence, the only element that maps to 0 is 0, which gives us that $ \phi $ is injective. Since we have infinitely many ways to choose $ n$, we have infinitely many injective homomorphisms. 
\end{problem}

\begin{problem}[13.32]
~\begin{enumerate}[label=\alph*)]
	\item True. Every subgroup of index 2 is normal.
	\item True. The trivial homomorphism.
	\item False. The trivial homomorphism is clearly not one-to-one.
	\item True. By Corollary 13.18.
	\item False. By exercise 44.
	\item False. This requires at least one element in the first group to map to more than one output, making it not a function.
	\item True. Let $ G= \zz_{12}$ and $ H = \langle 2 \rangle$. Then $ |\langle 2 \rangle|=\frac{12}{ \gcd ( 12,2) } = 6$.  Now let $ \phi(x)=x $. In general, we can always find a $ H \leq G$ where  $ H$ has order 6 and  $ G$ has order 12. This is allowed by Lagrange, since 6 is a divisor of 12.
	\item False. No such subgroup described above exist here because this isn't allowed by Lagrange. 
	\item False. The identity is always in there, because $ \phi(e_G) = \phi(a \cdot a^{-1}) = \phi(a) \phi(a^{-1}) = \phi(a) \phi(a)^{-1} = e_H $.
	\item False. Let $ \phi: \zz_n \to \cc^* $, $ \phi(x) = e^{i 2\pi x/n}  $. 
\end{enumerate}
\end{problem}
\end{document}

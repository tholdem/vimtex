\documentclass[class=article,crop=false]{standalone} 
%Fall 2020
% Some basic packages
\usepackage{standalone}[subpreambles=true]
\usepackage[utf8]{inputenc}
\usepackage[T1]{fontenc}
\usepackage{textcomp}
\usepackage[english]{babel}
\usepackage{url}
\usepackage{graphicx}
\usepackage{float}
\usepackage{enumitem}


\pdfminorversion=7

% Don't indent paragraphs, leave some space between them
\usepackage{parskip}

% Hide page number when page is empty
\usepackage{emptypage}
\usepackage{subcaption}
\usepackage{multicol}
\usepackage[dvipsnames]{xcolor}


% Math stuff
\usepackage{amsmath, amsfonts, mathtools, amsthm, amssymb}
% Fancy script capitals
\usepackage{mathrsfs}
\usepackage{cancel}
% Bold math
\usepackage{bm}
% Some shortcuts
\newcommand{\rr}{\ensuremath{\mathbb{R}}}
\newcommand{\zz}{\ensuremath{\mathbb{Z}}}
\newcommand{\qq}{\ensuremath{\mathbb{Q}}}
\newcommand{\nn}{\ensuremath{\mathbb{N}}}
\newcommand{\ff}{\ensuremath{\mathbb{F}}}
\newcommand{\cc}{\ensuremath{\mathbb{C}}}
\renewcommand\O{\ensuremath{\emptyset}}
\newcommand{\norm}[1]{{\left\lVert{#1}\right\rVert}}
\renewcommand{\vec}[1]{{\mathbf{#1}}}
\newcommand\allbold[1]{{\boldmath\textbf{#1}}}

% Put x \to \infty below \lim
\let\svlim\lim\def\lim{\svlim\limits}

%Make implies and impliedby shorter
\let\implies\Rightarrow
\let\impliedby\Leftarrow
\let\iff\Leftrightarrow
\let\epsilon\varepsilon

% Add \contra symbol to denote contradiction
\usepackage{stmaryrd} % for \lightning
\newcommand\contra{\scalebox{1.5}{$\lightning$}}

% \let\phi\varphi

% Command for short corrections
% Usage: 1+1=\correct{3}{2}

\definecolor{correct}{HTML}{009900}
\newcommand\correct[2]{\ensuremath{\:}{\color{red}{#1}}\ensuremath{\to }{\color{correct}{#2}}\ensuremath{\:}}
\newcommand\green[1]{{\color{correct}{#1}}}

% horizontal rule
\newcommand\hr{
    \noindent\rule[0.5ex]{\linewidth}{0.5pt}
}

% hide parts
\newcommand\hide[1]{}

% si unitx
\usepackage{siunitx}
\sisetup{locale = FR}

% Environments
\makeatother
% For box around Definition, Theorem, \ldots
\usepackage[framemethod=TikZ]{mdframed}
\mdfsetup{skipabove=1em,skipbelow=0em}

%definition
\newenvironment{defn}[1][]{%
\ifstrempty{#1}%
{\mdfsetup{%
frametitle={%
\tikz[baseline=(current bounding box.east),outer sep=0pt]
\node[anchor=east,rectangle,fill=Emerald]
{\strut Definition};}}
}%
{\mdfsetup{%
frametitle={%
\tikz[baseline=(current bounding box.east),outer sep=0pt]
\node[anchor=east,rectangle,fill=Emerald]
{\strut Definition:~#1};}}%
}%
\mdfsetup{innertopmargin=10pt,linecolor=Emerald,%
linewidth=2pt,topline=true,%
frametitleaboveskip=\dimexpr-\ht\strutbox\relax
}
\begin{mdframed}[]\relax%
\label{#1}}{\end{mdframed}}


%theorem
%\newcounter{thm}[section]\setcounter{thm}{0}
%\renewcommand{\thethm}{\arabic{section}.\arabic{thm}}
\newenvironment{thm}[1][]{%
%\refstepcounter{thm}%
\ifstrempty{#1}%
{\mdfsetup{%
frametitle={%
\tikz[baseline=(current bounding box.east),outer sep=0pt]
\node[anchor=east,rectangle,fill=blue!20]
%{\strut Theorem~\thethm};}}
{\strut Theorem};}}
}%
{\mdfsetup{%
frametitle={%
\tikz[baseline=(current bounding box.east),outer sep=0pt]
\node[anchor=east,rectangle,fill=blue!20]
%{\strut Theorem~\thethm:~#1};}}%
{\strut Theorem:~#1};}}%
}%
\mdfsetup{innertopmargin=10pt,linecolor=blue!20,%
linewidth=2pt,topline=true,%
frametitleaboveskip=\dimexpr-\ht\strutbox\relax
}
\begin{mdframed}[]\relax%
\label{#1}}{\end{mdframed}}


%lemma
\newenvironment{lem}[1][]{%
\ifstrempty{#1}%
{\mdfsetup{%
frametitle={%
\tikz[baseline=(current bounding box.east),outer sep=0pt]
\node[anchor=east,rectangle,fill=Dandelion]
{\strut Lemma};}}
}%
{\mdfsetup{%
frametitle={%
\tikz[baseline=(current bounding box.east),outer sep=0pt]
\node[anchor=east,rectangle,fill=Dandelion]
{\strut Lemma:~#1};}}%
}%
\mdfsetup{innertopmargin=10pt,linecolor=Dandelion,%
linewidth=2pt,topline=true,%
frametitleaboveskip=\dimexpr-\ht\strutbox\relax
}
\begin{mdframed}[]\relax%
\label{#1}}{\end{mdframed}}

%corollary
\newenvironment{coro}[1][]{%
\ifstrempty{#1}%
{\mdfsetup{%
frametitle={%
\tikz[baseline=(current bounding box.east),outer sep=0pt]
\node[anchor=east,rectangle,fill=CornflowerBlue]
{\strut Corollary};}}
}%
{\mdfsetup{%
frametitle={%
\tikz[baseline=(current bounding box.east),outer sep=0pt]
\node[anchor=east,rectangle,fill=CornflowerBlue]
{\strut Corollary:~#1};}}%
}%
\mdfsetup{innertopmargin=10pt,linecolor=CornflowerBlue,%
linewidth=2pt,topline=true,%
frametitleaboveskip=\dimexpr-\ht\strutbox\relax
}
\begin{mdframed}[]\relax%
\label{#1}}{\end{mdframed}}

%proof
\newenvironment{prf}[1][]{%
\ifstrempty{#1}%
{\mdfsetup{%
frametitle={%
\tikz[baseline=(current bounding box.east),outer sep=0pt]
\node[anchor=east,rectangle,fill=SpringGreen]
{\strut Proof};}}
}%
{\mdfsetup{%
frametitle={%
\tikz[baseline=(current bounding box.east),outer sep=0pt]
\node[anchor=east,rectangle,fill=SpringGreen]
{\strut Proof:~#1};}}%
}%
\mdfsetup{innertopmargin=10pt,linecolor=SpringGreen,%
linewidth=2pt,topline=true,%
frametitleaboveskip=\dimexpr-\ht\strutbox\relax
}
\begin{mdframed}[]\relax%
\label{#1}}{\qed\end{mdframed}}


\theoremstyle{definition}

\newmdtheoremenv[nobreak=true]{definition}{Definition}
\newmdtheoremenv[nobreak=true]{prop}{Proposition}
\newmdtheoremenv[nobreak=true]{theorem}{Theorem}
\newmdtheoremenv[nobreak=true]{corollary}{Corollary}
\newtheorem*{eg}{Example}
\theoremstyle{remark}
\newtheorem*{case}{Case}
\newtheorem*{notation}{Notation}
\newtheorem*{remark}{Remark}
\newtheorem*{note}{Note}
\newtheorem*{problem}{Problem}
\newtheorem*{observe}{Observe}
\newtheorem*{property}{Property}
\newtheorem*{intuition}{Intuition}


% End example and intermezzo environments with a small diamond (just like proof
% environments end with a small square)
\usepackage{etoolbox}
\AtEndEnvironment{vb}{\null\hfill$\diamond$}%
\AtEndEnvironment{intermezzo}{\null\hfill$\diamond$}%
% \AtEndEnvironment{opmerking}{\null\hfill$\diamond$}%

% Fix some spacing
% http://tex.stackexchange.com/questions/22119/how-can-i-change-the-spacing-before-theorems-with-amsthm
\makeatletter
\def\thm@space@setup{%
  \thm@preskip=\parskip \thm@postskip=0pt
}

% Fix some stuff
% %http://tex.stackexchange.com/questions/76273/multiple-pdfs-with-page-group-included-in-a-single-page-warning
\pdfsuppresswarningpagegroup=1


% My name
\author{Jaden Wang}



\begin{document}
Let $ \phi: D \to F, a \mapsto  [(a,1)]$ is a ring homomorphism. Clearly,
\begin{align*}
	\phi(a+b) = [(a+b),1] = [(a1+1b,1)] = [(a,1)]+[(b,1)] = \phi(a)+\phi(b)
\end{align*}
and
\[
	\phi(a\times b) = [(ab,1)] = [(a,1)]\times [(b,1)] = \phi(a)\phi(b)
.\] 
To show injectivity, let's look at $ \ker \phi$:
\[
	\ker \phi = \{a \in D: \phi(a)=0\} = \{a \in D: [(a,1)]=[(0,1)]\}  
.\] 
iff $ a=0$. So  $ \ker \phi = \{e\} $.

\begin{thm}[]
	If $ F$ is the field of fraction of  $ D$, and  $ K$ is some other field containing  $ D$, then  $ K$ has a subfield isomorphic to  $ F$.
\end{thm}

\begin{intuition}[]
If $ \iota:D \to K$ is an injective map.
\[
	\iota\left( \frac{a}{b} \right) \to \frac{\iota(a)}{\iota(b) } = \iota(a)(\iota(b))^{-1}
.\] 
\end{intuition}

How do we check that $ F$ is the field of fractions of  $ D$?

We need to check
 \begin{enumerate}[label=\arabic*)]
	 \item $ F$ is a field.
	\item "$ F$ is not too small". 
	\item "$ F$ is not too big". $ \rr$ isn't the field of fractions of $ \zz$ because $ \pi \in \rr$ but $ \pi \neq \frac{a}{b}$ for $ a,b \in \zz, b \neq 0$.
\end{enumerate}

\begin{eg}[21.1]
	$ D=\{n+m1,: n,m \in \zz\} $ Gaussian integers. 

	Why is this a ring? Prove it's a subring of $ \cc$ : closed under $ +,-,\times $ and nonempty.

	Why is this a domain? $ D$ is commutative inherited from complex numbers.  $ D$ has identity  $ (1+0i)$. And  $ D$ has no zero divisors because  $ \cc$ is a field and doesn't. 

	The field of fraction of $ D$ must be a subring of  $ \cc$. We guess that
	\[
	F=\{p+qi: p,q \in \qq\} 
	\] 
	is the field of fractions.

	$ F$ is a field (done in previous homeowkr).  $ F$ is a subring. It is commutative and has identity. Every element has inverse.

	 $ F$ needs to contain  $ D$. Clearly  $ D \subseteq F$ because $ \zz \subseteq \qq$. 

	 $ F$ is not too big: Every element of $ F$ is of form  $ \frac{a}{b}, a,b \in D, b\neq 0$. Given  $ p+qi \in F$, express this as $ \frac{a+bi}{c+di }$ where $ a,b,c,d \in \zz$. If $ p+qi = \frac{r}{s} + \frac{t}{u}i$ with $ r,s,t,u \in \zz$ and $ s,u \neq 0$.
	  \[
	 \frac{r}{s}+ \frac{t}{u}i = \frac{ru+st i}{su+0i }
	 .\] 
\end{eg}

\begin{eg}[]
What is the field of fraction of $ \qq$? It is $ \qq$ itself.
\end{eg}

\section*{22: Polynomial Rings}

Start with a ring $ R$ (probability commutative with identity). Invent a new ring  $ R[x]$ (polynomials with coefficients in  $ R$ in the indeterminate $ x$).  

Warning: $ x$ is NOT a variable. It is not waiting for you to plug in a value. "Plugging in values" is only allowed in the context of evaluate homomorphisms.  $ x$ is just  $ x$.

A typical element of  $ R[x]$ looks like
 \[
a_0+a_1x+a_2 x^2 + \ldots + a_n x^{n}
.\]
where $ n \in \nn, a_i \in R$. We don't allow infinite terms because we don't want infinitely many coefficients. 

Note
\[
a_0+\ldots +a_n x^{n} = b_0 + \ldots + b_n x^{n} \iff a_i=b_i \ \forall \ i
.\]
In fact, $ 1-x^2 = 1+0x-x^2+0x^3$. 

These polynomials are not functions! Consider $ \zz_3[x]$. $x^3-x+1, 1 \in \zz_3[x]$. If we illegally plug in values 0,1,2, then the outputs equal but they aren't functions. This is happening because every element of $ \zz_3$ is a root of $ x^3 -x$. This cannot happen in infinite fields like $ \zz$. We need to check coefficients instead and they clearly aren't the same.

Addition:
\[
	\sum_{ i= 0}^{ n} a_i x^{i} + \sum_{ i= 0}^{ n} b_i x^{i} = \sum_{ i= 0}^{ n} (a_i + b_i) x^{i} 
.\] 
\end{document}

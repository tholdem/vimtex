\documentclass[class=article,crop=false]{standalone} 
%Fall 2020
% Some basic packages
\usepackage{standalone}[subpreambles=true]
\usepackage[utf8]{inputenc}
\usepackage[T1]{fontenc}
\usepackage{textcomp}
\usepackage[english]{babel}
\usepackage{url}
\usepackage{graphicx}
\usepackage{float}
\usepackage{enumitem}


\pdfminorversion=7

% Don't indent paragraphs, leave some space between them
\usepackage{parskip}

% Hide page number when page is empty
\usepackage{emptypage}
\usepackage{subcaption}
\usepackage{multicol}
\usepackage[dvipsnames]{xcolor}


% Math stuff
\usepackage{amsmath, amsfonts, mathtools, amsthm, amssymb}
% Fancy script capitals
\usepackage{mathrsfs}
\usepackage{cancel}
% Bold math
\usepackage{bm}
% Some shortcuts
\newcommand{\rr}{\ensuremath{\mathbb{R}}}
\newcommand{\zz}{\ensuremath{\mathbb{Z}}}
\newcommand{\qq}{\ensuremath{\mathbb{Q}}}
\newcommand{\nn}{\ensuremath{\mathbb{N}}}
\newcommand{\ff}{\ensuremath{\mathbb{F}}}
\newcommand{\cc}{\ensuremath{\mathbb{C}}}
\renewcommand\O{\ensuremath{\emptyset}}
\newcommand{\norm}[1]{{\left\lVert{#1}\right\rVert}}
\renewcommand{\vec}[1]{{\mathbf{#1}}}
\newcommand\allbold[1]{{\boldmath\textbf{#1}}}

% Put x \to \infty below \lim
\let\svlim\lim\def\lim{\svlim\limits}

%Make implies and impliedby shorter
\let\implies\Rightarrow
\let\impliedby\Leftarrow
\let\iff\Leftrightarrow
\let\epsilon\varepsilon

% Add \contra symbol to denote contradiction
\usepackage{stmaryrd} % for \lightning
\newcommand\contra{\scalebox{1.5}{$\lightning$}}

% \let\phi\varphi

% Command for short corrections
% Usage: 1+1=\correct{3}{2}

\definecolor{correct}{HTML}{009900}
\newcommand\correct[2]{\ensuremath{\:}{\color{red}{#1}}\ensuremath{\to }{\color{correct}{#2}}\ensuremath{\:}}
\newcommand\green[1]{{\color{correct}{#1}}}

% horizontal rule
\newcommand\hr{
    \noindent\rule[0.5ex]{\linewidth}{0.5pt}
}

% hide parts
\newcommand\hide[1]{}

% si unitx
\usepackage{siunitx}
\sisetup{locale = FR}

% Environments
\makeatother
% For box around Definition, Theorem, \ldots
\usepackage[framemethod=TikZ]{mdframed}
\mdfsetup{skipabove=1em,skipbelow=0em}

%definition
\newenvironment{defn}[1][]{%
\ifstrempty{#1}%
{\mdfsetup{%
frametitle={%
\tikz[baseline=(current bounding box.east),outer sep=0pt]
\node[anchor=east,rectangle,fill=Emerald]
{\strut Definition};}}
}%
{\mdfsetup{%
frametitle={%
\tikz[baseline=(current bounding box.east),outer sep=0pt]
\node[anchor=east,rectangle,fill=Emerald]
{\strut Definition:~#1};}}%
}%
\mdfsetup{innertopmargin=10pt,linecolor=Emerald,%
linewidth=2pt,topline=true,%
frametitleaboveskip=\dimexpr-\ht\strutbox\relax
}
\begin{mdframed}[]\relax%
\label{#1}}{\end{mdframed}}


%theorem
%\newcounter{thm}[section]\setcounter{thm}{0}
%\renewcommand{\thethm}{\arabic{section}.\arabic{thm}}
\newenvironment{thm}[1][]{%
%\refstepcounter{thm}%
\ifstrempty{#1}%
{\mdfsetup{%
frametitle={%
\tikz[baseline=(current bounding box.east),outer sep=0pt]
\node[anchor=east,rectangle,fill=blue!20]
%{\strut Theorem~\thethm};}}
{\strut Theorem};}}
}%
{\mdfsetup{%
frametitle={%
\tikz[baseline=(current bounding box.east),outer sep=0pt]
\node[anchor=east,rectangle,fill=blue!20]
%{\strut Theorem~\thethm:~#1};}}%
{\strut Theorem:~#1};}}%
}%
\mdfsetup{innertopmargin=10pt,linecolor=blue!20,%
linewidth=2pt,topline=true,%
frametitleaboveskip=\dimexpr-\ht\strutbox\relax
}
\begin{mdframed}[]\relax%
\label{#1}}{\end{mdframed}}


%lemma
\newenvironment{lem}[1][]{%
\ifstrempty{#1}%
{\mdfsetup{%
frametitle={%
\tikz[baseline=(current bounding box.east),outer sep=0pt]
\node[anchor=east,rectangle,fill=Dandelion]
{\strut Lemma};}}
}%
{\mdfsetup{%
frametitle={%
\tikz[baseline=(current bounding box.east),outer sep=0pt]
\node[anchor=east,rectangle,fill=Dandelion]
{\strut Lemma:~#1};}}%
}%
\mdfsetup{innertopmargin=10pt,linecolor=Dandelion,%
linewidth=2pt,topline=true,%
frametitleaboveskip=\dimexpr-\ht\strutbox\relax
}
\begin{mdframed}[]\relax%
\label{#1}}{\end{mdframed}}

%corollary
\newenvironment{coro}[1][]{%
\ifstrempty{#1}%
{\mdfsetup{%
frametitle={%
\tikz[baseline=(current bounding box.east),outer sep=0pt]
\node[anchor=east,rectangle,fill=CornflowerBlue]
{\strut Corollary};}}
}%
{\mdfsetup{%
frametitle={%
\tikz[baseline=(current bounding box.east),outer sep=0pt]
\node[anchor=east,rectangle,fill=CornflowerBlue]
{\strut Corollary:~#1};}}%
}%
\mdfsetup{innertopmargin=10pt,linecolor=CornflowerBlue,%
linewidth=2pt,topline=true,%
frametitleaboveskip=\dimexpr-\ht\strutbox\relax
}
\begin{mdframed}[]\relax%
\label{#1}}{\end{mdframed}}

%proof
\newenvironment{prf}[1][]{%
\ifstrempty{#1}%
{\mdfsetup{%
frametitle={%
\tikz[baseline=(current bounding box.east),outer sep=0pt]
\node[anchor=east,rectangle,fill=SpringGreen]
{\strut Proof};}}
}%
{\mdfsetup{%
frametitle={%
\tikz[baseline=(current bounding box.east),outer sep=0pt]
\node[anchor=east,rectangle,fill=SpringGreen]
{\strut Proof:~#1};}}%
}%
\mdfsetup{innertopmargin=10pt,linecolor=SpringGreen,%
linewidth=2pt,topline=true,%
frametitleaboveskip=\dimexpr-\ht\strutbox\relax
}
\begin{mdframed}[]\relax%
\label{#1}}{\qed\end{mdframed}}


\theoremstyle{definition}

\newmdtheoremenv[nobreak=true]{definition}{Definition}
\newmdtheoremenv[nobreak=true]{prop}{Proposition}
\newmdtheoremenv[nobreak=true]{theorem}{Theorem}
\newmdtheoremenv[nobreak=true]{corollary}{Corollary}
\newtheorem*{eg}{Example}
\theoremstyle{remark}
\newtheorem*{case}{Case}
\newtheorem*{notation}{Notation}
\newtheorem*{remark}{Remark}
\newtheorem*{note}{Note}
\newtheorem*{problem}{Problem}
\newtheorem*{observe}{Observe}
\newtheorem*{property}{Property}
\newtheorem*{intuition}{Intuition}


% End example and intermezzo environments with a small diamond (just like proof
% environments end with a small square)
\usepackage{etoolbox}
\AtEndEnvironment{vb}{\null\hfill$\diamond$}%
\AtEndEnvironment{intermezzo}{\null\hfill$\diamond$}%
% \AtEndEnvironment{opmerking}{\null\hfill$\diamond$}%

% Fix some spacing
% http://tex.stackexchange.com/questions/22119/how-can-i-change-the-spacing-before-theorems-with-amsthm
\makeatletter
\def\thm@space@setup{%
  \thm@preskip=\parskip \thm@postskip=0pt
}

% Fix some stuff
% %http://tex.stackexchange.com/questions/76273/multiple-pdfs-with-page-group-included-in-a-single-page-warning
\pdfsuppresswarningpagegroup=1


% My name
\author{Jaden Wang}



\begin{document}
\begin{coro}[9.12]
	Any permutation of a finite set of at least two elements can be written as a product of its transpositions.
\end{coro}
\begin{eg}[]
\[
	(1\ 2)(1\ 3)(1\ 4)(1\ 5) = (1\ 5\ 4\ 3\ 2) \in S_5
.\] 
We start from right to left. Then for (1 2 3 4 5) we can just use (1 5)(1 4)(1 3)(1 2)
\end{eg}

\begin{eg}[]
\begin{align*}
	\sigma &= \text{ (1 3 6)(2 8)(4 7 5)} \\
	       &= \text{ (1 6)(1 3)(2 8)(4 5)(4 7)}  \\
\end{align*}
\end{eg}

\begin{eg}[]
\begin{align*}
	\text{ (1 2)(2 3)(1 2)(2 3)(1 2)} &= \text{ (1)(2 3)}  \\
					  &= \text{ (2 3)}  \\
\end{align*}
A product of 5 transpositions = product of 1 transposition.
\end{eg}
\begin{eg}[]
Use the same $ \sigma$ as above, it is a product of 5 transpositions. We can write it as 7 transposition.
\[
	\sigma = \text{(1 6)(1 3)(2 8)(4 5)(4 7)(1 2)(1 2) } 
.\] 
Could $ \sigma$ be the product of 10 transpositions? No!
\end{eg}

\begin{defn}[even permutation]
An \allbold{even permutation} is a product of even number of transpositions. 
\end{defn}
Likewise for odd permutation.
\begin{note}
So far any permutation is even or odd. Also, $ k$-cycle are odd if  $ k$ is even, and are even if  $ k$ is odd.
\end{note}
\begin{eg}[]
	Consider (1 2) and (1 2 3) in $ S_3$
\end{eg}

\begin{claim}[]
No permutation is both even and odd.
\end{claim}
Let's prove this using permutation matrix from linear algebra. Note that ith column of the permutation matrix tells you where the $ e_i$ basis goes.
\begin{prf}
\begin{claim}[]
	A permutation is even or odd if its permutation matrix has determinant 1 or -1, respectively.
\end{claim}
Since every time swapping rows flips the sign of the determinant. Since it cannot have determinant to be both 1 and -1 at the same time, the permutation cannot be both even and odd.
\end{prf}

\begin{claim}[]
There exists an isomorphism between $ S_n$ and $ n\times n$ permutation matrices under matrix multiplication.
\end{claim}

\begin{note}[]
	The identity is even $ (1 2)(1 2)$ and has determinant 1.
\end{note}

\begin{eg}[]
If $ \sigma$ is a product of 5, $ \tau$ is a product of 4, so $ \sigma \tau$ is a product of 9 transposition.

\begin{table}[H]
	\centering
	\begin{tabular}{c||c|c}
		*& even & odd\\
		\hline
		\hline
		even & even & odd\\
		\hline
		odd& odd & even
	\end{tabular}
\end{table}
So it adds like even and odd numbers.
\end{eg}
\begin{eg}[]
	If $ \alpha = \text{ (1 2)(2 4)(3 4)} $, then $ \alpha^{-1}=\text{(3 4)(2 4)(1 2)}$.
\end{eg}
Then clearly an even permutation's inverse is also even.

\begin{thm}[]
The even permutations in $ S_n$ form a subgroup.
\end{thm}

\begin{prf}
~\begin{enumerate}[label=(\roman*)]
	\item the identity is even.
	\item the product of two even permutations is even.
	\item the inverse of an even permutation is even.
\end{enumerate}
\end{prf}

\begin{defn}[]
This is called the \allbold{alternating group on $ n$ letters}, denoted $ A_n$.
\end{defn}

\begin{claim}[]
If $ n\geq 2$, then exactly half the elements in  $ S_n$ are even,
\end{claim}

\begin{prf}
	The map $ x \mapsto x \text{ (1 2)} $ from $ S_n$ to $ S_n$ sends even to odd and vice versa.
	And if $ n\geq 2$, then  $ |A_n|=\frac{n!}{2}$.
\end{prf}
\begin{note}[]
If $ n=1$,  $ S_n$ is trivial, and so is $ A_n$, so $ A_n=S_n$.

If $ n=2$,  $ S_n = \{id, \text{ (1 2)} \} $, and $ A_n$ has an order 1.

If $ n=3$,  $ S_n$ has order 6, and $ A_n$ has order 3. So $ S_n$ is nonabelian but $ A_n$ is abelian! This is the only time it happens.

If $ n=4$,  $ A_n$ has order 12, is $ \{1, \text{ (1 2 3)}, \text{ (1 3 2)}, \text{ (1 2 4)}, \text{ (1 4 2)}, \text{ (1 3 4)}, \text{ (1 4 3)}, \text{ (2 3 4)}, \text{ (2 4 3)}, \text{ (1 2)(3 4)}, \text{ (1 3)(2 4)}, \text{ (1 4)(2 3)}        \} $

This is nonabelian because
\begin{align*}
	\text{ (1 2 3)(1 2 4)} &= \text{ (1 3)(2 4)}  \\
	\text{ (1 2 4)(1 2 3)} &= \text{ (1 4)(2 3)}  \\ 
\end{align*}
The same counterexample can be used for $ A_n$ showing that $ A_n$ is nonabelian for $ n \geq 4$.
\end{note}

\begin{note}[]
$ \zz_n$ is abelian. $ D_n$ is nonabelian for $ n\geq 3$ of  order 6, $ A_n$ is nonabelian for $ n\geq 4$ of order $ 12$, $ S_n$ is nonabelian for $ n\geq 3$ of order  $ 6$.
\end{note}


\end{document}

\documentclass[class=article,crop=false]{standalone} 
%Fall 2020
% Some basic packages
\usepackage{standalone}[subpreambles=true]
\usepackage[utf8]{inputenc}
\usepackage[T1]{fontenc}
\usepackage{textcomp}
\usepackage[english]{babel}
\usepackage{url}
\usepackage{graphicx}
\usepackage{float}
\usepackage{enumitem}


\pdfminorversion=7

% Don't indent paragraphs, leave some space between them
\usepackage{parskip}

% Hide page number when page is empty
\usepackage{emptypage}
\usepackage{subcaption}
\usepackage{multicol}
\usepackage[dvipsnames]{xcolor}


% Math stuff
\usepackage{amsmath, amsfonts, mathtools, amsthm, amssymb}
% Fancy script capitals
\usepackage{mathrsfs}
\usepackage{cancel}
% Bold math
\usepackage{bm}
% Some shortcuts
\newcommand{\rr}{\ensuremath{\mathbb{R}}}
\newcommand{\zz}{\ensuremath{\mathbb{Z}}}
\newcommand{\qq}{\ensuremath{\mathbb{Q}}}
\newcommand{\nn}{\ensuremath{\mathbb{N}}}
\newcommand{\ff}{\ensuremath{\mathbb{F}}}
\newcommand{\cc}{\ensuremath{\mathbb{C}}}
\renewcommand\O{\ensuremath{\emptyset}}
\newcommand{\norm}[1]{{\left\lVert{#1}\right\rVert}}
\renewcommand{\vec}[1]{{\mathbf{#1}}}
\newcommand\allbold[1]{{\boldmath\textbf{#1}}}

% Put x \to \infty below \lim
\let\svlim\lim\def\lim{\svlim\limits}

%Make implies and impliedby shorter
\let\implies\Rightarrow
\let\impliedby\Leftarrow
\let\iff\Leftrightarrow
\let\epsilon\varepsilon

% Add \contra symbol to denote contradiction
\usepackage{stmaryrd} % for \lightning
\newcommand\contra{\scalebox{1.5}{$\lightning$}}

% \let\phi\varphi

% Command for short corrections
% Usage: 1+1=\correct{3}{2}

\definecolor{correct}{HTML}{009900}
\newcommand\correct[2]{\ensuremath{\:}{\color{red}{#1}}\ensuremath{\to }{\color{correct}{#2}}\ensuremath{\:}}
\newcommand\green[1]{{\color{correct}{#1}}}

% horizontal rule
\newcommand\hr{
    \noindent\rule[0.5ex]{\linewidth}{0.5pt}
}

% hide parts
\newcommand\hide[1]{}

% si unitx
\usepackage{siunitx}
\sisetup{locale = FR}

% Environments
\makeatother
% For box around Definition, Theorem, \ldots
\usepackage[framemethod=TikZ]{mdframed}
\mdfsetup{skipabove=1em,skipbelow=0em}

%definition
\newenvironment{defn}[1][]{%
\ifstrempty{#1}%
{\mdfsetup{%
frametitle={%
\tikz[baseline=(current bounding box.east),outer sep=0pt]
\node[anchor=east,rectangle,fill=Emerald]
{\strut Definition};}}
}%
{\mdfsetup{%
frametitle={%
\tikz[baseline=(current bounding box.east),outer sep=0pt]
\node[anchor=east,rectangle,fill=Emerald]
{\strut Definition:~#1};}}%
}%
\mdfsetup{innertopmargin=10pt,linecolor=Emerald,%
linewidth=2pt,topline=true,%
frametitleaboveskip=\dimexpr-\ht\strutbox\relax
}
\begin{mdframed}[]\relax%
\label{#1}}{\end{mdframed}}


%theorem
%\newcounter{thm}[section]\setcounter{thm}{0}
%\renewcommand{\thethm}{\arabic{section}.\arabic{thm}}
\newenvironment{thm}[1][]{%
%\refstepcounter{thm}%
\ifstrempty{#1}%
{\mdfsetup{%
frametitle={%
\tikz[baseline=(current bounding box.east),outer sep=0pt]
\node[anchor=east,rectangle,fill=blue!20]
%{\strut Theorem~\thethm};}}
{\strut Theorem};}}
}%
{\mdfsetup{%
frametitle={%
\tikz[baseline=(current bounding box.east),outer sep=0pt]
\node[anchor=east,rectangle,fill=blue!20]
%{\strut Theorem~\thethm:~#1};}}%
{\strut Theorem:~#1};}}%
}%
\mdfsetup{innertopmargin=10pt,linecolor=blue!20,%
linewidth=2pt,topline=true,%
frametitleaboveskip=\dimexpr-\ht\strutbox\relax
}
\begin{mdframed}[]\relax%
\label{#1}}{\end{mdframed}}


%lemma
\newenvironment{lem}[1][]{%
\ifstrempty{#1}%
{\mdfsetup{%
frametitle={%
\tikz[baseline=(current bounding box.east),outer sep=0pt]
\node[anchor=east,rectangle,fill=Dandelion]
{\strut Lemma};}}
}%
{\mdfsetup{%
frametitle={%
\tikz[baseline=(current bounding box.east),outer sep=0pt]
\node[anchor=east,rectangle,fill=Dandelion]
{\strut Lemma:~#1};}}%
}%
\mdfsetup{innertopmargin=10pt,linecolor=Dandelion,%
linewidth=2pt,topline=true,%
frametitleaboveskip=\dimexpr-\ht\strutbox\relax
}
\begin{mdframed}[]\relax%
\label{#1}}{\end{mdframed}}

%corollary
\newenvironment{coro}[1][]{%
\ifstrempty{#1}%
{\mdfsetup{%
frametitle={%
\tikz[baseline=(current bounding box.east),outer sep=0pt]
\node[anchor=east,rectangle,fill=CornflowerBlue]
{\strut Corollary};}}
}%
{\mdfsetup{%
frametitle={%
\tikz[baseline=(current bounding box.east),outer sep=0pt]
\node[anchor=east,rectangle,fill=CornflowerBlue]
{\strut Corollary:~#1};}}%
}%
\mdfsetup{innertopmargin=10pt,linecolor=CornflowerBlue,%
linewidth=2pt,topline=true,%
frametitleaboveskip=\dimexpr-\ht\strutbox\relax
}
\begin{mdframed}[]\relax%
\label{#1}}{\end{mdframed}}

%proof
\newenvironment{prf}[1][]{%
\ifstrempty{#1}%
{\mdfsetup{%
frametitle={%
\tikz[baseline=(current bounding box.east),outer sep=0pt]
\node[anchor=east,rectangle,fill=SpringGreen]
{\strut Proof};}}
}%
{\mdfsetup{%
frametitle={%
\tikz[baseline=(current bounding box.east),outer sep=0pt]
\node[anchor=east,rectangle,fill=SpringGreen]
{\strut Proof:~#1};}}%
}%
\mdfsetup{innertopmargin=10pt,linecolor=SpringGreen,%
linewidth=2pt,topline=true,%
frametitleaboveskip=\dimexpr-\ht\strutbox\relax
}
\begin{mdframed}[]\relax%
\label{#1}}{\qed\end{mdframed}}


\theoremstyle{definition}

\newmdtheoremenv[nobreak=true]{definition}{Definition}
\newmdtheoremenv[nobreak=true]{prop}{Proposition}
\newmdtheoremenv[nobreak=true]{theorem}{Theorem}
\newmdtheoremenv[nobreak=true]{corollary}{Corollary}
\newtheorem*{eg}{Example}
\theoremstyle{remark}
\newtheorem*{case}{Case}
\newtheorem*{notation}{Notation}
\newtheorem*{remark}{Remark}
\newtheorem*{note}{Note}
\newtheorem*{problem}{Problem}
\newtheorem*{observe}{Observe}
\newtheorem*{property}{Property}
\newtheorem*{intuition}{Intuition}


% End example and intermezzo environments with a small diamond (just like proof
% environments end with a small square)
\usepackage{etoolbox}
\AtEndEnvironment{vb}{\null\hfill$\diamond$}%
\AtEndEnvironment{intermezzo}{\null\hfill$\diamond$}%
% \AtEndEnvironment{opmerking}{\null\hfill$\diamond$}%

% Fix some spacing
% http://tex.stackexchange.com/questions/22119/how-can-i-change-the-spacing-before-theorems-with-amsthm
\makeatletter
\def\thm@space@setup{%
  \thm@preskip=\parskip \thm@postskip=0pt
}

% Fix some stuff
% %http://tex.stackexchange.com/questions/76273/multiple-pdfs-with-page-group-included-in-a-single-page-warning
\pdfsuppresswarningpagegroup=1


% My name
\author{Jaden Wang}



\begin{document}
\begin{thm}[Euler's Theorem]
	$ G=U(\zz_n), |G|=\phi(n)$. If $ x \in U(\zz_n)$, then $ x^{|G|}=e$. If $ \gcd ( a,n)=1 $ then $ a^{\phi(n)}=1 \mod n$.
\end{thm}
\begin{note}[]
It's the same thing as Fermat with $ \zz_n^* $. The units of 
\end{note}

How many elements in $ U(\zz_n)$?

\begin{defn}[Euler's phi-function]
	$ \phi(p) = p-1$ in $ U(\zz_p)$. $ \phi(1)=1$.
\end{defn}

\begin{eg}[]
	$ \phi(9) =3$.  
\end{eg}
\begin{prop}[]
	$ \phi(p^{n}) = p^{n-1}(p-1)$. 
\end{prop}
\begin{eg}[]
	$ \phi(1000)$. Even numbers and multiples of 5 are factors. The units are the ones that don't have 2 or 5 as factors.
\end{eg}
\begin{eg}[solving congruence]
Solving $ ax=b \mod m$, $ a,b,m \in \zz$. "Congruence mod m": $ ax-b = km$ for some  $ k \in \zz$. So $ b=ax-km$.

$ d=\gcd ( a,m) $. So $ a$ is a multiple of  $ d$,  $ m$ is a multiple of  $ d$. So  $ b= ax-km$ is multiple of $ d$. So by contrapositive, if $ d $ doesn't divide  $ b$ then there is no solution. 

If we assume $ d / b$. Define  $ b' = \frac{b}{d}, a'=\frac{a}{d}, m'=\frac{m}{d}$ all integers. Then
\begin{align*}
	b'&=a'x-km'\\
	a'x &= b' \mod m'\\
\end{align*}

This is better because let $ a=30, m=42$, then  $ d=6, a'=\frac{30}{6}=5, m'=\frac{42}{6}=7$. Then $ \gcd ( a',m')=1 $ always coprime! 

Now, $ a'$ is a unit in  $ \zz_{m'}$. This means $ a'x=b'$ has a unique solution in  $ \zz_{m'}$ by multiplying by the inverse.
\end{eg}

\begin{eg}[20.14]
$ 12x=27 \mod 18$. $ a=12, b=27,m=18$.  $ d=6$. Since  $ d$ doesn't divide  $ b$, no solution.

$ 15x=27 \mod 18$. $ d=3$. $ d / b$. So  $ a'=5, m'=6, b'= 9$. So  $ 5x=9 \mod 6 \implies 5x=3 \mod 6$. Since 5 and 6 are coprime, 5 is a unit in $ \zz_6$. The inverse is also 5. 

So \[
5 \times 5 x = 5 \times 3 \mod 6 \implies x = 3 \mod 6
.\]

\end{eg}

What is the relationship between mod 18 and mod 6?

Since 18 is a multiple of 6, so mod 18 implies mod 6. So if $ x=3 \mod 6 \implies x=3, 9, 15 \mod 18$. There are $ d$ solutions here, evenly spaced mod 18, similar to  $ U_n$ complex root of unity..

If $ c=5 \mod 9$ what is $ c \mod 2$? Not well-defined.

\begin{coro}[20.13]
$ ax=b \mod m, d=\gcd ( a,m) $. 
\begin{itemize}
	\item If $ d$ doesn't divide $b$, no solutions.
	\item If  $ d / b$, there are  $ d$ solutions  $ \mod m$, and they are evenly spaced. 
\end{itemize}
\end{coro}
\begin{note}[]
	This allows us to find all solutions if we find one.
\end{note}

\section*{21}
Question: Is a subring of a field always a domain?
\begin{eg}[]
$ \zz \leq \qq$. It's a subset, nonempty, closed under addition, subtraction, and multiplication. It is also a domain.
\end{eg}

Suppose $ S \leq F$ a field.  $ S$ inherits commutativity and no zero divisors. However,  $ 2\zz \leq \qq$ doesn't have identity.

\begin{thm}[]
	If $ S$ is a subring of a field and  $ S$ has identity, then  $ S$ is a domain.
\end{thm}

Goal: start with a domain and try to find the smallest field that contains it. (This always works!). This smallest field is called the field of fractions of the domain.

\begin{eg}[]
	$ \zz$ is a domain. $ \zz\leq F$. What could $ F$ be? It can be  $ \qq, \rr, \cc, \qq(\sqrt{2} )$. There is a smallest one: $ \qq$. 
\end{eg}

To say two fractions are equal without mentioning division: $ ad=bc$.
\end{document}


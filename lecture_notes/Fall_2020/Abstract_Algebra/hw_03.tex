\documentclass[12pt]{article}
%Fall 2020
% Some basic packages
\usepackage{standalone}[subpreambles=true]
\usepackage[utf8]{inputenc}
\usepackage[T1]{fontenc}
\usepackage{textcomp}
\usepackage[english]{babel}
\usepackage{url}
\usepackage{graphicx}
\usepackage{float}
\usepackage{enumitem}


\pdfminorversion=7

% Don't indent paragraphs, leave some space between them
\usepackage{parskip}

% Hide page number when page is empty
\usepackage{emptypage}
\usepackage{subcaption}
\usepackage{multicol}
\usepackage[dvipsnames]{xcolor}


% Math stuff
\usepackage{amsmath, amsfonts, mathtools, amsthm, amssymb}
% Fancy script capitals
\usepackage{mathrsfs}
\usepackage{cancel}
% Bold math
\usepackage{bm}
% Some shortcuts
\newcommand{\rr}{\ensuremath{\mathbb{R}}}
\newcommand{\zz}{\ensuremath{\mathbb{Z}}}
\newcommand{\qq}{\ensuremath{\mathbb{Q}}}
\newcommand{\nn}{\ensuremath{\mathbb{N}}}
\newcommand{\ff}{\ensuremath{\mathbb{F}}}
\newcommand{\cc}{\ensuremath{\mathbb{C}}}
\renewcommand\O{\ensuremath{\emptyset}}
\newcommand{\norm}[1]{{\left\lVert{#1}\right\rVert}}
\renewcommand{\vec}[1]{{\mathbf{#1}}}
\newcommand\allbold[1]{{\boldmath\textbf{#1}}}

% Put x \to \infty below \lim
\let\svlim\lim\def\lim{\svlim\limits}

%Make implies and impliedby shorter
\let\implies\Rightarrow
\let\impliedby\Leftarrow
\let\iff\Leftrightarrow
\let\epsilon\varepsilon

% Add \contra symbol to denote contradiction
\usepackage{stmaryrd} % for \lightning
\newcommand\contra{\scalebox{1.5}{$\lightning$}}

% \let\phi\varphi

% Command for short corrections
% Usage: 1+1=\correct{3}{2}

\definecolor{correct}{HTML}{009900}
\newcommand\correct[2]{\ensuremath{\:}{\color{red}{#1}}\ensuremath{\to }{\color{correct}{#2}}\ensuremath{\:}}
\newcommand\green[1]{{\color{correct}{#1}}}

% horizontal rule
\newcommand\hr{
    \noindent\rule[0.5ex]{\linewidth}{0.5pt}
}

% hide parts
\newcommand\hide[1]{}

% si unitx
\usepackage{siunitx}
\sisetup{locale = FR}

% Environments
\makeatother
% For box around Definition, Theorem, \ldots
\usepackage[framemethod=TikZ]{mdframed}
\mdfsetup{skipabove=1em,skipbelow=0em}

%definition
\newenvironment{defn}[1][]{%
\ifstrempty{#1}%
{\mdfsetup{%
frametitle={%
\tikz[baseline=(current bounding box.east),outer sep=0pt]
\node[anchor=east,rectangle,fill=Emerald]
{\strut Definition};}}
}%
{\mdfsetup{%
frametitle={%
\tikz[baseline=(current bounding box.east),outer sep=0pt]
\node[anchor=east,rectangle,fill=Emerald]
{\strut Definition:~#1};}}%
}%
\mdfsetup{innertopmargin=10pt,linecolor=Emerald,%
linewidth=2pt,topline=true,%
frametitleaboveskip=\dimexpr-\ht\strutbox\relax
}
\begin{mdframed}[]\relax%
\label{#1}}{\end{mdframed}}


%theorem
%\newcounter{thm}[section]\setcounter{thm}{0}
%\renewcommand{\thethm}{\arabic{section}.\arabic{thm}}
\newenvironment{thm}[1][]{%
%\refstepcounter{thm}%
\ifstrempty{#1}%
{\mdfsetup{%
frametitle={%
\tikz[baseline=(current bounding box.east),outer sep=0pt]
\node[anchor=east,rectangle,fill=blue!20]
%{\strut Theorem~\thethm};}}
{\strut Theorem};}}
}%
{\mdfsetup{%
frametitle={%
\tikz[baseline=(current bounding box.east),outer sep=0pt]
\node[anchor=east,rectangle,fill=blue!20]
%{\strut Theorem~\thethm:~#1};}}%
{\strut Theorem:~#1};}}%
}%
\mdfsetup{innertopmargin=10pt,linecolor=blue!20,%
linewidth=2pt,topline=true,%
frametitleaboveskip=\dimexpr-\ht\strutbox\relax
}
\begin{mdframed}[]\relax%
\label{#1}}{\end{mdframed}}


%lemma
\newenvironment{lem}[1][]{%
\ifstrempty{#1}%
{\mdfsetup{%
frametitle={%
\tikz[baseline=(current bounding box.east),outer sep=0pt]
\node[anchor=east,rectangle,fill=Dandelion]
{\strut Lemma};}}
}%
{\mdfsetup{%
frametitle={%
\tikz[baseline=(current bounding box.east),outer sep=0pt]
\node[anchor=east,rectangle,fill=Dandelion]
{\strut Lemma:~#1};}}%
}%
\mdfsetup{innertopmargin=10pt,linecolor=Dandelion,%
linewidth=2pt,topline=true,%
frametitleaboveskip=\dimexpr-\ht\strutbox\relax
}
\begin{mdframed}[]\relax%
\label{#1}}{\end{mdframed}}

%corollary
\newenvironment{coro}[1][]{%
\ifstrempty{#1}%
{\mdfsetup{%
frametitle={%
\tikz[baseline=(current bounding box.east),outer sep=0pt]
\node[anchor=east,rectangle,fill=CornflowerBlue]
{\strut Corollary};}}
}%
{\mdfsetup{%
frametitle={%
\tikz[baseline=(current bounding box.east),outer sep=0pt]
\node[anchor=east,rectangle,fill=CornflowerBlue]
{\strut Corollary:~#1};}}%
}%
\mdfsetup{innertopmargin=10pt,linecolor=CornflowerBlue,%
linewidth=2pt,topline=true,%
frametitleaboveskip=\dimexpr-\ht\strutbox\relax
}
\begin{mdframed}[]\relax%
\label{#1}}{\end{mdframed}}

%proof
\newenvironment{prf}[1][]{%
\ifstrempty{#1}%
{\mdfsetup{%
frametitle={%
\tikz[baseline=(current bounding box.east),outer sep=0pt]
\node[anchor=east,rectangle,fill=SpringGreen]
{\strut Proof};}}
}%
{\mdfsetup{%
frametitle={%
\tikz[baseline=(current bounding box.east),outer sep=0pt]
\node[anchor=east,rectangle,fill=SpringGreen]
{\strut Proof:~#1};}}%
}%
\mdfsetup{innertopmargin=10pt,linecolor=SpringGreen,%
linewidth=2pt,topline=true,%
frametitleaboveskip=\dimexpr-\ht\strutbox\relax
}
\begin{mdframed}[]\relax%
\label{#1}}{\qed\end{mdframed}}


\theoremstyle{definition}

\newmdtheoremenv[nobreak=true]{definition}{Definition}
\newmdtheoremenv[nobreak=true]{prop}{Proposition}
\newmdtheoremenv[nobreak=true]{theorem}{Theorem}
\newmdtheoremenv[nobreak=true]{corollary}{Corollary}
\newtheorem*{eg}{Example}
\theoremstyle{remark}
\newtheorem*{case}{Case}
\newtheorem*{notation}{Notation}
\newtheorem*{remark}{Remark}
\newtheorem*{note}{Note}
\newtheorem*{problem}{Problem}
\newtheorem*{observe}{Observe}
\newtheorem*{property}{Property}
\newtheorem*{intuition}{Intuition}


% End example and intermezzo environments with a small diamond (just like proof
% environments end with a small square)
\usepackage{etoolbox}
\AtEndEnvironment{vb}{\null\hfill$\diamond$}%
\AtEndEnvironment{intermezzo}{\null\hfill$\diamond$}%
% \AtEndEnvironment{opmerking}{\null\hfill$\diamond$}%

% Fix some spacing
% http://tex.stackexchange.com/questions/22119/how-can-i-change-the-spacing-before-theorems-with-amsthm
\makeatletter
\def\thm@space@setup{%
  \thm@preskip=\parskip \thm@postskip=0pt
}

% Fix some stuff
% %http://tex.stackexchange.com/questions/76273/multiple-pdfs-with-page-group-included-in-a-single-page-warning
\pdfsuppresswarningpagegroup=1


% My name
\author{Jaden Wang}



\begin{document}
\centerline {\textsf{\textbf{\LARGE{Homework 3}}}}
\centerline {Jaden Wang}
\vspace{.15in}

\begin{problem}[5.9]
	Yes. By Problem 4.13, we have already shown that $D_n^* $ is a group, where $D_n^* $ denotes the set of diagonal $n\times n$ matrices with no zeros on the diagonal. Since we are told that elements in this set are invertible (or by definition of a group), it follows that $D_n^* \subseteq GL(n, \rr)$.Since $D_n^* $ is a group under the same operation as $GL(n, \rr)$, namely matrix multiplication, by the definition of subgroup, $D_n^* $ is a subgroup of $GL(n, \rr)$.
\end{problem}

\begin{problem}[5.10]
	Yes. Denote this set as $U_n^* $, since we are told that its elements are invertible (or by full-rankness), $U_n^* \subseteq GL(n, \rr)$. Given $A,B \in U_n^* $,
	\begin{enumerate}[label=(\roman*)]
		\item Notice that $I_n$ is the identity of $GL(n, \rr)$, and $I_n$ is an upper triangle matrix with no zeros on the diagonal, so $I_n \in U_n^* $.
		\item We want to show that $A^{-1}$ is also an upper triangle matrix with no zeros on the diagonal, so that $A^{-1} \in U_n^* $. 

			First let's show that it is upper triangular via the inversion process using its adjugate matrix. Since $A_n$ is upper triangular, $A^{T}$ must be lower triangular. That is, $A_{pq}=A^{T}_{pq} = 0 \quad \forall 1 \leq q < p \leq n$. Now let's consider its adjugate matrix Adj$(A)$. For its lower triangular entires, \emph{i.e.} $1\leq j<i\leq n$, Adj$(A)_{ij}=$det$(M^{T}_{ij})=$det$(M_{ji})$, where $M_{ji}$ is the minor matrix of $A$ after removing  $j$th row and  $i$th column. Since $j<i$,  $A_{ji}$ is one of the upper triangular entries of $A$, and eliminating its row and column would necessarily yields  det$(M_{ji})=$Adj$(A)=0$. Since the rest of the inversion process only involves scalar multiplication on each entry, this 0 will carry over to $A^{-1}$, \emph{i.e.} $A^{-1}_{ij}=0 \quad \forall 1\leq j<i\leq n$. Therefore, $A^{-1}$ is upper triangular by definition.  
			When $j=i$, it is easy to see that $M_{ji}$ still has full rank and cannot equal to 0. Hence after nonzero scalar multiplication it is still nonzero. Hence, we establish that $A^{-1}$ is an upper triangular matrix with no zeros on the diagonal, so $A^{-1} \in U_n^* $.
		\item Let $C=A \times B$, so $c_{ij}=\sum_{ k= 1}^{ n} a_{ik} b_{kj} $. Since $A,B$ are upper triangular,  $a_{ik}=0 \quad \forall i>k$ and $b_{kj}=0 \quad \forall k>j$. Now consider the lower triangular entries of $C$,  \emph{i.e.} $c_{ij}$ when $i>j$. 
			\begin{align*}
				c_{ij}=\sum_{ k= 1}^{ n} a_{ik}b_{kj} &= \sum_{ k= 1}^{ i-1} a_{ik}b_{kj} + \sum_{ k= i}^{ n} a_{ik}b_{jk}  \\
				&=  \sum_{ k= 1}^{ i-1} 0 \cdot b_{kj} + \sum_{ k= i}^{ n} a_{ik} \cdot 0 \\
				&= 0 
			\end{align*}
			Hence $C$ is upper triangular. When $i=j$, $c_{ij}=c_{ii} = a_{ii}b_{ii}\neq 0$. Therefore, $C \in U_n^* $. 
	\end{enumerate}
	Then by Theorem 5.14, $U_n^* $ is a subgroup of $GL(n, \rr)$.
\end{problem}

\begin{problem}[5.11]
	No. $I_n$ is the identity of $GL(n, \rr)$, yet det$(I_n)=1 \neq -1$, so $I_n \not\in GL(n, \rr)$. Thus it cannot be a subgroup.
\end{problem}

\begin{problem}[5.12]
	Yes. Denote this set as $H$. Again by assumption $H \subseteq GL(n, \rr)$. Given $A,B \in H$, 
	\begin{enumerate}[label=(\roman*)]
		\item det$(I_n)=1 \implies I_n \in H$.
		\item det$(A)$ det$(A^{-1}) =$det$(I_n) =1 \implies$ det$(A^{-1}) = \pm 1 \implies A^{-1} \in H$.
		\item det$(A)$ det$(B) = \pm 1 =$ det$(C) \implies C \in H$. 
	\end{enumerate}
	Hence by Theorem, 5.14,  $H$ is a subgroup of  $GL(n, \rr)$.
\end{problem}

\begin{problem}[5.22]
	Let $a= \begin{pmatrix} 0&-1\\-1&0 \end{pmatrix} $. Then $a^2 = \begin{pmatrix} 1&0\\0&1 \end{pmatrix} $, and $a^3 = \begin{pmatrix} 0&-1\\-1&0 \end{pmatrix} = a$. Therefore, 
	\[
		\langle a \rangle = \left\{ \begin{pmatrix} 0&-1\\-1&0 \end{pmatrix} , \begin{pmatrix} 1&0\\0&1 \end{pmatrix}  \right\} 
	.\] 
\end{problem}

\begin{problem}[5.23]
	Let $a= \begin{pmatrix} 1&1\\0&1 \end{pmatrix} $. $a^2 = \begin{pmatrix} 1&2\\0&1 \end{pmatrix} $. And by induction we can easy show that $a^{n} = \begin{pmatrix} 1&n\\0&1 \end{pmatrix} $. Since $a \in GL(2, \rr)$, $\langle a \rangle$ is a subgroup by Theorem 5.17. Thus, $a^{0} = \begin{pmatrix} 1&0\\0&1 \end{pmatrix} = I_2 \in \langle a \rangle$ and $(a^{n})^{-1} = \begin{pmatrix} 1&-n\\0&1 \end{pmatrix} = a^{-n} \in \langle a \rangle$. Putting them together,
\[
	\langle a \rangle = \begin{pmatrix} 1&n\\0&1 \end{pmatrix} \quad \forall n \in \zz
.\] 
\end{problem}

\begin{problem}[5.39]
~\begin{enumerate}[label=\alph*)]
	\item True. It's in the definition of a group.
	\item False, By the contrapositive of Theorem 4.15, if cancellation law doesn't hold in  $(G,*)$, then $(G,*)$ is not a group.
	\item True.  $G \subseteq G$ under $*$, so it satisfies the definition of a subgroup.
	\item False. By definition of an improper subgroup, only  $G$ itself is defined as such.
	\item False. For the cyclic group  $ \zz_4 $, $\langle 0 \rangle = \{0\} \neq \{0,1,2,3\} $. Hence $0$ is not a generator for  $ \zz_4$.
	\item False. In $ \zz_4$, both 1 and 3 are generators hence it's not unique.
	\item False. $ ( \zz, +)$ is a group but $ ( \zz,\times )$ is not a group because $0 \in \zz$ yet it doesn't have an inverse.
	\item False. The subset also has to be a group itself.
	\item True. $1$ is a generator for  $ \zz_4$.
	\item False. Consider $H=(\{1\} ,+)$. Clearly $H \subseteq \zz$, yet since the identity $0 \not\in H$, $H$ cannot be a subgroup by Theorem 5.14.
\end{enumerate}
\end{problem}

\begin{problem}[5.44]
Consider the empty set $ \O$. Since it has no element, it trivially satisfies condition 1 and 3. However, it is not a group since the identity is not in $ \O$, so it cannot be a subgroup. Hence, 2 is necessary to exclude this unwanted edge case.
\end{problem}

\begin{problem}[6.5]
Listing all the positive divisors:
	\begin{align*}
	32: 1,2,4,8,16,32\\
	24:1,2,3,4,6,8,12,24
\end{align*}
Clearly $ \gcd ( 32,24) =8$.
\end{problem}

\begin{problem}[6.6]
Listing  all the positive divisors:
\begin{align*}
	48: 1,2,3,4,6,8,12,16,24,48\\
	88:1,2,4,8,11,22,44,88
\end{align*}
Clearly $ \gcd ( 48,88) =8$.
\end{problem}

\begin{problem}[6.9]
	By Theorem 6.10, all cyclic group of order $ 8$ is isomorphic to  $ ( \zz_8,+_{ 8} )$. By Theorem 6.14, we simply need to find the number of elements that are relatively prime with $ n=8$ in $ \zz_8$. This yields $ \{1,3,5,7\} $. Hence the number of generators is $ 4$. 
\end{problem}

\begin{problem}[6.10]
Similarly, the elements of $ \zz_{12}$ that is relatively prime with $n=12$ is $ \{1,5,7,11\} $. Hence the number of generators is 4.
\end{problem}

\begin{problem}[6.17]
It is easy to see that $ d= \gcd ( 30,25) =5 $. By Theorem 6.14, the subgroup contains
\[
\frac{n}{d} = \frac{30}{5}=6
\]
elements.
\end{problem}

\begin{problem}[6.18]
Similarly, $ d= \gcd ( 42,30) =3 $. The subgroup contains:
\[
\frac{n}{d} = \frac{42}{3} =14
\] 
elements.
\end{problem}

\begin{problem}[6.20]
	Notice that $ \zeta = \frac{1+i}{\sqrt{2} }$ is in $ U_8 \subseteq \cc^* $. Additionally, it is a generator of $ U_8$, since  $U_8 = \{\zeta^{n}, 0\leq n \leq 7\}$. Hence we know this subgroup has 8 elements.
\end{problem}

\begin{problem}[6.21]
Rewrite $ 1+i$ in its polar form $a=\sqrt{2}  e^{\frac{\pi}{4}i}$ and we can see that $ a^{n} = 2^{\frac{n}{2}} e^{\frac{n\pi}{4}i}$, where the argument is repeating but the modulus keeps growing as $ n$ increases. Since $ \langle a \rangle$ is a subgroup, the identity $ a^{0}$ and inverses $ a^{-n}$ are all in the subgroup just like Problem 5.23. Therefore, this subgroup is infinite since there is no repeating elements and $ n \in \zz$.
\end{problem}

\begin{problem}[6.32]
~\begin{enumerate}[label=\alph*)]
	\item True. By Theorem 6.1.
	\item False. Consider the example in Problem 4.19. It is an abelian group because it is clearly commutative and we proved that it is a group. However, it is not cyclic since the set is uncountable and cannot be isomorphic to $ \zz$ or $ \zz_n$.
	\item True. Since $ \qq$ is countably infinite and has one-to-one correspondence with $ \zz$, it is a group and isomorphic to $ \zz$ under addition and hence is cyclic.
	\item False. In $ \zz_4$, $ 2\in \zz_4$ is not a generator.
	\item True. There exists a group $ \zz_n$ for every finite group of order $ n>0$, and we know it is cyclic and therefore abelian.
	\item False. The Klein 4 group $V_4$ is not cyclic yet has an order 4.
	\item True. Elements of $ \zz_{20}$ that are relatively prime with 20 are $ \{3,7,11,13,17,19\} $ which are all prime numbers. By Theorem 6.14 they are also generators.
	\item False. The binary operation of $ G \cap G'$ is ambiguous since $ G$ and  $ G'$ might not have the same operation. Then  $ G \cap G'$ wouldn't even be well-defined as a group.
	\item True. First  $ H \cap K \subseteq H \subseteq G$. Given $ a,b \in H \cap K$,
		\begin{enumerate}[label=(\roman*)]
			\item Since $ H,K$ are subgroups of  $ G$, the identity $e_G \in H$ and $ K$, which is equivalent to  $ e_G = H \cap K$.
			\item Since $ a \in H \cap K$, $ a$ is in both  $ H$ and  $ K$. So  $ a^{-1} \in H \text{ and } K $ which is equivalent to $ a^{-1} \in H \cap K$.
			\item Since $ H,k$ are subgroups,  $ a * b \in H$ and $a*b \in K $, which is equivalent to $ a*b \in H \cap K$.
		\end{enumerate}
		Hence $ H \cap K$ is a subgroup of $ G$.
	\item True. Consider all the finite cyclic groups. They are all isomorphic to $ \zz_n$ which has at least 1 and a prime between 2 and $ n-1$ as generators for  $ n\geq 3$. Then for infinite cyclic groups, they are all isomorphic to  $ \zz$ which has at least 1 and -1 as generators. This covers all cyclic groups.

\end{enumerate}

\end{problem}
\end{document}

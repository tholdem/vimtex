\documentclass[class=article,crop=false]{standalone} 
%Fall 2020
% Some basic packages
\usepackage{standalone}[subpreambles=true]
\usepackage[utf8]{inputenc}
\usepackage[T1]{fontenc}
\usepackage{textcomp}
\usepackage[english]{babel}
\usepackage{url}
\usepackage{graphicx}
\usepackage{float}
\usepackage{enumitem}


\pdfminorversion=7

% Don't indent paragraphs, leave some space between them
\usepackage{parskip}

% Hide page number when page is empty
\usepackage{emptypage}
\usepackage{subcaption}
\usepackage{multicol}
\usepackage[dvipsnames]{xcolor}


% Math stuff
\usepackage{amsmath, amsfonts, mathtools, amsthm, amssymb}
% Fancy script capitals
\usepackage{mathrsfs}
\usepackage{cancel}
% Bold math
\usepackage{bm}
% Some shortcuts
\newcommand{\rr}{\ensuremath{\mathbb{R}}}
\newcommand{\zz}{\ensuremath{\mathbb{Z}}}
\newcommand{\qq}{\ensuremath{\mathbb{Q}}}
\newcommand{\nn}{\ensuremath{\mathbb{N}}}
\newcommand{\ff}{\ensuremath{\mathbb{F}}}
\newcommand{\cc}{\ensuremath{\mathbb{C}}}
\renewcommand\O{\ensuremath{\emptyset}}
\newcommand{\norm}[1]{{\left\lVert{#1}\right\rVert}}
\renewcommand{\vec}[1]{{\mathbf{#1}}}
\newcommand\allbold[1]{{\boldmath\textbf{#1}}}

% Put x \to \infty below \lim
\let\svlim\lim\def\lim{\svlim\limits}

%Make implies and impliedby shorter
\let\implies\Rightarrow
\let\impliedby\Leftarrow
\let\iff\Leftrightarrow
\let\epsilon\varepsilon

% Add \contra symbol to denote contradiction
\usepackage{stmaryrd} % for \lightning
\newcommand\contra{\scalebox{1.5}{$\lightning$}}

% \let\phi\varphi

% Command for short corrections
% Usage: 1+1=\correct{3}{2}

\definecolor{correct}{HTML}{009900}
\newcommand\correct[2]{\ensuremath{\:}{\color{red}{#1}}\ensuremath{\to }{\color{correct}{#2}}\ensuremath{\:}}
\newcommand\green[1]{{\color{correct}{#1}}}

% horizontal rule
\newcommand\hr{
    \noindent\rule[0.5ex]{\linewidth}{0.5pt}
}

% hide parts
\newcommand\hide[1]{}

% si unitx
\usepackage{siunitx}
\sisetup{locale = FR}

% Environments
\makeatother
% For box around Definition, Theorem, \ldots
\usepackage[framemethod=TikZ]{mdframed}
\mdfsetup{skipabove=1em,skipbelow=0em}

%definition
\newenvironment{defn}[1][]{%
\ifstrempty{#1}%
{\mdfsetup{%
frametitle={%
\tikz[baseline=(current bounding box.east),outer sep=0pt]
\node[anchor=east,rectangle,fill=Emerald]
{\strut Definition};}}
}%
{\mdfsetup{%
frametitle={%
\tikz[baseline=(current bounding box.east),outer sep=0pt]
\node[anchor=east,rectangle,fill=Emerald]
{\strut Definition:~#1};}}%
}%
\mdfsetup{innertopmargin=10pt,linecolor=Emerald,%
linewidth=2pt,topline=true,%
frametitleaboveskip=\dimexpr-\ht\strutbox\relax
}
\begin{mdframed}[]\relax%
\label{#1}}{\end{mdframed}}


%theorem
%\newcounter{thm}[section]\setcounter{thm}{0}
%\renewcommand{\thethm}{\arabic{section}.\arabic{thm}}
\newenvironment{thm}[1][]{%
%\refstepcounter{thm}%
\ifstrempty{#1}%
{\mdfsetup{%
frametitle={%
\tikz[baseline=(current bounding box.east),outer sep=0pt]
\node[anchor=east,rectangle,fill=blue!20]
%{\strut Theorem~\thethm};}}
{\strut Theorem};}}
}%
{\mdfsetup{%
frametitle={%
\tikz[baseline=(current bounding box.east),outer sep=0pt]
\node[anchor=east,rectangle,fill=blue!20]
%{\strut Theorem~\thethm:~#1};}}%
{\strut Theorem:~#1};}}%
}%
\mdfsetup{innertopmargin=10pt,linecolor=blue!20,%
linewidth=2pt,topline=true,%
frametitleaboveskip=\dimexpr-\ht\strutbox\relax
}
\begin{mdframed}[]\relax%
\label{#1}}{\end{mdframed}}


%lemma
\newenvironment{lem}[1][]{%
\ifstrempty{#1}%
{\mdfsetup{%
frametitle={%
\tikz[baseline=(current bounding box.east),outer sep=0pt]
\node[anchor=east,rectangle,fill=Dandelion]
{\strut Lemma};}}
}%
{\mdfsetup{%
frametitle={%
\tikz[baseline=(current bounding box.east),outer sep=0pt]
\node[anchor=east,rectangle,fill=Dandelion]
{\strut Lemma:~#1};}}%
}%
\mdfsetup{innertopmargin=10pt,linecolor=Dandelion,%
linewidth=2pt,topline=true,%
frametitleaboveskip=\dimexpr-\ht\strutbox\relax
}
\begin{mdframed}[]\relax%
\label{#1}}{\end{mdframed}}

%corollary
\newenvironment{coro}[1][]{%
\ifstrempty{#1}%
{\mdfsetup{%
frametitle={%
\tikz[baseline=(current bounding box.east),outer sep=0pt]
\node[anchor=east,rectangle,fill=CornflowerBlue]
{\strut Corollary};}}
}%
{\mdfsetup{%
frametitle={%
\tikz[baseline=(current bounding box.east),outer sep=0pt]
\node[anchor=east,rectangle,fill=CornflowerBlue]
{\strut Corollary:~#1};}}%
}%
\mdfsetup{innertopmargin=10pt,linecolor=CornflowerBlue,%
linewidth=2pt,topline=true,%
frametitleaboveskip=\dimexpr-\ht\strutbox\relax
}
\begin{mdframed}[]\relax%
\label{#1}}{\end{mdframed}}

%proof
\newenvironment{prf}[1][]{%
\ifstrempty{#1}%
{\mdfsetup{%
frametitle={%
\tikz[baseline=(current bounding box.east),outer sep=0pt]
\node[anchor=east,rectangle,fill=SpringGreen]
{\strut Proof};}}
}%
{\mdfsetup{%
frametitle={%
\tikz[baseline=(current bounding box.east),outer sep=0pt]
\node[anchor=east,rectangle,fill=SpringGreen]
{\strut Proof:~#1};}}%
}%
\mdfsetup{innertopmargin=10pt,linecolor=SpringGreen,%
linewidth=2pt,topline=true,%
frametitleaboveskip=\dimexpr-\ht\strutbox\relax
}
\begin{mdframed}[]\relax%
\label{#1}}{\qed\end{mdframed}}


\theoremstyle{definition}

\newmdtheoremenv[nobreak=true]{definition}{Definition}
\newmdtheoremenv[nobreak=true]{prop}{Proposition}
\newmdtheoremenv[nobreak=true]{theorem}{Theorem}
\newmdtheoremenv[nobreak=true]{corollary}{Corollary}
\newtheorem*{eg}{Example}
\theoremstyle{remark}
\newtheorem*{case}{Case}
\newtheorem*{notation}{Notation}
\newtheorem*{remark}{Remark}
\newtheorem*{note}{Note}
\newtheorem*{problem}{Problem}
\newtheorem*{observe}{Observe}
\newtheorem*{property}{Property}
\newtheorem*{intuition}{Intuition}


% End example and intermezzo environments with a small diamond (just like proof
% environments end with a small square)
\usepackage{etoolbox}
\AtEndEnvironment{vb}{\null\hfill$\diamond$}%
\AtEndEnvironment{intermezzo}{\null\hfill$\diamond$}%
% \AtEndEnvironment{opmerking}{\null\hfill$\diamond$}%

% Fix some spacing
% http://tex.stackexchange.com/questions/22119/how-can-i-change-the-spacing-before-theorems-with-amsthm
\makeatletter
\def\thm@space@setup{%
  \thm@preskip=\parskip \thm@postskip=0pt
}

% Fix some stuff
% %http://tex.stackexchange.com/questions/76273/multiple-pdfs-with-page-group-included-in-a-single-page-warning
\pdfsuppresswarningpagegroup=1


% My name
\author{Jaden Wang}



\begin{document}
\begin{eg}[]

Question: Are 19 and 6 in the same coset?

No, because $ x+H=y+H \iff x-y \in H$. And $19-6=13 \not\in H$. 

Are 19 and 7 in the same coset?

Yes, because $ 7-19=-12 \in H$.

\end{eg}

\begin{eg}[easiest one: left/right cosets are different, IMPORTANT]
	$ G=S_3$,  $ H=\{e, (1\ 2)\} $. 

	What are the left cosets of $ H$ in  $ G$?
	 \begin{enumerate}[label=\arabic*)]
		\item H itself.
		\item The left coset containing $ (1\ 2\ 3)H = \{(1\ 2\ 3)e, (1\ 2\ 3)(1\ 2)\} =\{(1\ 2\ 3), (1\ 3)\}  $.
		\item by elimination, we know the last left coset is $ \{(2\ 3),(1\ 3\ 2)\} $.
	
	\end{enumerate}
			Question: is $ (1\ 3\ 2)H=(2\ 3)H$? Yes because let  $ x= (1\ 3\ 2)$ and  $ y=(2\ 3)$, then  $ x^{-1}= (1\ 2\ 3)$, so $ x^{-1}y=(1\ 2\ 3)(2\ 3)=(1\ 2) \in H$.
			QUestion: is $ (1\ 2\ 3)H=(2\ 3)H$?  $ x^{-1}y=(1\ 3\ 2)(2\ 3) = (1\ 3) \not\in H$, so no.

We can just take every element of the left cosets and individually invert it to convert to right cosets. 

Show that $ H(1\ 3)=H(1\ 3\ 2)$.  $ xy^{-1}=(1\ 3)(1\ 2\ 3)=(1\ 2) \in H$.

\end{eg}
\begin{eg}[]
	$ G=S_3, H=\{e\}  $. There are six cosets. Left/right cosets are the same even if the group is nonabelian.
\end{eg}

\begin{eg}[]
	$ G=S_3,H=S_3$. There is only one coset (left/right).
\end{eg}
\begin{eg}[]
	$ G=S_3, H=A_3$. There are two cosets. Left and right again agree because there is a subgroup and everything else.
\end{eg}
\begin{claim}[]
	Left and right cosets always agree if there are only two cosets (because one must a subgroup).
\end{claim}

\begin{thm}[]
	If $ G$ is finite, then the order of  $ x$, $o(x)$ divides  $ |G|$.
\end{thm}
\begin{prf}
$ \langle x \rangle$ is a subgroup of $ G$, so  $ |\langle x \rangle|$ divides $ G$ by Lagrange.
\end{prf}

\begin{thm}[10.11]
Every group of prime order is cyclic.
\end{thm}
\begin{eg}[]
We know this doesn't have to be true for non-prime number: $ V_4$.
\end{eg}
\begin{prf}
Let $ G$ be a group of order  $ p$ and let  $ H\geq g$. Then  $ |H|$ divides  $ |G|$. But since  $ |G|$ is a prime, so  $ |H|=1 \implies H=\{e\} $ or  $ |H|=p \implies H=G$.
\end{prf}

\begin{claim}[]
Let $ x \in G\setminus \{e\} $. Then $ \langle x \rangle =G$. Not only is $ G$ cyclic, but any nonidentity element is a generator.
\end{claim}

\begin{defn}[direct products]
	Let $ (G,*_G)$ and  $ (H,*_H)$ be two groups. The \allbold{direct product} $ G \times H$ as
	\[
		G \times H= \{(g,h):g \in G, h \in H\} 
	.\] 
	Operation: componentwise
	\[
		(g_1,h_2) * (g_2,h_2) = (g_1*g_2,h_1*h_2)
	.\] 
	where closure follows immediately.
	
	Identity: $ e = (e_G,e_H)$.  $ (e_G,e_H)*(g,h) = (e_G*g,e_H*H)=(g,h)$. Same for the other way.

	Inverse:  $ (g^{-1},h^{-1})$.

	Associativity: see book.
\end{defn}

\begin{eg}[]
$ \zz_2 \times \zz_2$ under $ +_2$. See iPad screenshots for the table. $ |A \times B|=|A| \times |B|$. It is isomorphic to $ V_4$. This is the proof that $ V_4$ is a group and is associative. 

Therefore, $ \zz_2 \times \zz_2$ is NOT isomorphic to $ \zz_4$!
\end{eg}
\begin{eg}[]
	$ \zz_2 \times \zz_3$ this is an abelian group of order 6. $ x=(1,1) \in \zz_2 \times \zz_3$. $ x + x = (0,2)$.  $ x+x+x=(1,0)$.  $ 4x = (0,1)$.  $ 5x=(1,2)$.  $ 6x=(0,0)$. Hence $ \zz_2 \times \zz_3$ is generated by $(1,1)$. So it is a cyclic group of order 6. Then it must be isomorphic to  $ \zz_6$! 
\end{eg}
\begin{claim}[]
Direct product of abelian groups are abelian.
\end{claim}
\begin{claim}[]
	$ \zz_m \times \zz_n \simeq \zz_{mn}$ if and only if $ \gcd ( m,n)=1 $. If $ \gcd ( m,n) =1$, then the order of $ (1,1)$ is  lcm$(m,n)=mn $.
\end{claim}

Goal: classify all abelian groups of order $ n$.

\end{document}

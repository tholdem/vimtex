\documentclass[12pt]{article}
%Fall 2020
% Some basic packages
\usepackage{standalone}[subpreambles=true]
\usepackage[utf8]{inputenc}
\usepackage[T1]{fontenc}
\usepackage{textcomp}
\usepackage[english]{babel}
\usepackage{url}
\usepackage{graphicx}
\usepackage{float}
\usepackage{enumitem}


\pdfminorversion=7

% Don't indent paragraphs, leave some space between them
\usepackage{parskip}

% Hide page number when page is empty
\usepackage{emptypage}
\usepackage{subcaption}
\usepackage{multicol}
\usepackage[dvipsnames]{xcolor}


% Math stuff
\usepackage{amsmath, amsfonts, mathtools, amsthm, amssymb}
% Fancy script capitals
\usepackage{mathrsfs}
\usepackage{cancel}
% Bold math
\usepackage{bm}
% Some shortcuts
\newcommand{\rr}{\ensuremath{\mathbb{R}}}
\newcommand{\zz}{\ensuremath{\mathbb{Z}}}
\newcommand{\qq}{\ensuremath{\mathbb{Q}}}
\newcommand{\nn}{\ensuremath{\mathbb{N}}}
\newcommand{\ff}{\ensuremath{\mathbb{F}}}
\newcommand{\cc}{\ensuremath{\mathbb{C}}}
\renewcommand\O{\ensuremath{\emptyset}}
\newcommand{\norm}[1]{{\left\lVert{#1}\right\rVert}}
\renewcommand{\vec}[1]{{\mathbf{#1}}}
\newcommand\allbold[1]{{\boldmath\textbf{#1}}}

% Put x \to \infty below \lim
\let\svlim\lim\def\lim{\svlim\limits}

%Make implies and impliedby shorter
\let\implies\Rightarrow
\let\impliedby\Leftarrow
\let\iff\Leftrightarrow
\let\epsilon\varepsilon

% Add \contra symbol to denote contradiction
\usepackage{stmaryrd} % for \lightning
\newcommand\contra{\scalebox{1.5}{$\lightning$}}

% \let\phi\varphi

% Command for short corrections
% Usage: 1+1=\correct{3}{2}

\definecolor{correct}{HTML}{009900}
\newcommand\correct[2]{\ensuremath{\:}{\color{red}{#1}}\ensuremath{\to }{\color{correct}{#2}}\ensuremath{\:}}
\newcommand\green[1]{{\color{correct}{#1}}}

% horizontal rule
\newcommand\hr{
    \noindent\rule[0.5ex]{\linewidth}{0.5pt}
}

% hide parts
\newcommand\hide[1]{}

% si unitx
\usepackage{siunitx}
\sisetup{locale = FR}

% Environments
\makeatother
% For box around Definition, Theorem, \ldots
\usepackage[framemethod=TikZ]{mdframed}
\mdfsetup{skipabove=1em,skipbelow=0em}

%definition
\newenvironment{defn}[1][]{%
\ifstrempty{#1}%
{\mdfsetup{%
frametitle={%
\tikz[baseline=(current bounding box.east),outer sep=0pt]
\node[anchor=east,rectangle,fill=Emerald]
{\strut Definition};}}
}%
{\mdfsetup{%
frametitle={%
\tikz[baseline=(current bounding box.east),outer sep=0pt]
\node[anchor=east,rectangle,fill=Emerald]
{\strut Definition:~#1};}}%
}%
\mdfsetup{innertopmargin=10pt,linecolor=Emerald,%
linewidth=2pt,topline=true,%
frametitleaboveskip=\dimexpr-\ht\strutbox\relax
}
\begin{mdframed}[]\relax%
\label{#1}}{\end{mdframed}}


%theorem
%\newcounter{thm}[section]\setcounter{thm}{0}
%\renewcommand{\thethm}{\arabic{section}.\arabic{thm}}
\newenvironment{thm}[1][]{%
%\refstepcounter{thm}%
\ifstrempty{#1}%
{\mdfsetup{%
frametitle={%
\tikz[baseline=(current bounding box.east),outer sep=0pt]
\node[anchor=east,rectangle,fill=blue!20]
%{\strut Theorem~\thethm};}}
{\strut Theorem};}}
}%
{\mdfsetup{%
frametitle={%
\tikz[baseline=(current bounding box.east),outer sep=0pt]
\node[anchor=east,rectangle,fill=blue!20]
%{\strut Theorem~\thethm:~#1};}}%
{\strut Theorem:~#1};}}%
}%
\mdfsetup{innertopmargin=10pt,linecolor=blue!20,%
linewidth=2pt,topline=true,%
frametitleaboveskip=\dimexpr-\ht\strutbox\relax
}
\begin{mdframed}[]\relax%
\label{#1}}{\end{mdframed}}


%lemma
\newenvironment{lem}[1][]{%
\ifstrempty{#1}%
{\mdfsetup{%
frametitle={%
\tikz[baseline=(current bounding box.east),outer sep=0pt]
\node[anchor=east,rectangle,fill=Dandelion]
{\strut Lemma};}}
}%
{\mdfsetup{%
frametitle={%
\tikz[baseline=(current bounding box.east),outer sep=0pt]
\node[anchor=east,rectangle,fill=Dandelion]
{\strut Lemma:~#1};}}%
}%
\mdfsetup{innertopmargin=10pt,linecolor=Dandelion,%
linewidth=2pt,topline=true,%
frametitleaboveskip=\dimexpr-\ht\strutbox\relax
}
\begin{mdframed}[]\relax%
\label{#1}}{\end{mdframed}}

%corollary
\newenvironment{coro}[1][]{%
\ifstrempty{#1}%
{\mdfsetup{%
frametitle={%
\tikz[baseline=(current bounding box.east),outer sep=0pt]
\node[anchor=east,rectangle,fill=CornflowerBlue]
{\strut Corollary};}}
}%
{\mdfsetup{%
frametitle={%
\tikz[baseline=(current bounding box.east),outer sep=0pt]
\node[anchor=east,rectangle,fill=CornflowerBlue]
{\strut Corollary:~#1};}}%
}%
\mdfsetup{innertopmargin=10pt,linecolor=CornflowerBlue,%
linewidth=2pt,topline=true,%
frametitleaboveskip=\dimexpr-\ht\strutbox\relax
}
\begin{mdframed}[]\relax%
\label{#1}}{\end{mdframed}}

%proof
\newenvironment{prf}[1][]{%
\ifstrempty{#1}%
{\mdfsetup{%
frametitle={%
\tikz[baseline=(current bounding box.east),outer sep=0pt]
\node[anchor=east,rectangle,fill=SpringGreen]
{\strut Proof};}}
}%
{\mdfsetup{%
frametitle={%
\tikz[baseline=(current bounding box.east),outer sep=0pt]
\node[anchor=east,rectangle,fill=SpringGreen]
{\strut Proof:~#1};}}%
}%
\mdfsetup{innertopmargin=10pt,linecolor=SpringGreen,%
linewidth=2pt,topline=true,%
frametitleaboveskip=\dimexpr-\ht\strutbox\relax
}
\begin{mdframed}[]\relax%
\label{#1}}{\qed\end{mdframed}}


\theoremstyle{definition}

\newmdtheoremenv[nobreak=true]{definition}{Definition}
\newmdtheoremenv[nobreak=true]{prop}{Proposition}
\newmdtheoremenv[nobreak=true]{theorem}{Theorem}
\newmdtheoremenv[nobreak=true]{corollary}{Corollary}
\newtheorem*{eg}{Example}
\theoremstyle{remark}
\newtheorem*{case}{Case}
\newtheorem*{notation}{Notation}
\newtheorem*{remark}{Remark}
\newtheorem*{note}{Note}
\newtheorem*{problem}{Problem}
\newtheorem*{observe}{Observe}
\newtheorem*{property}{Property}
\newtheorem*{intuition}{Intuition}


% End example and intermezzo environments with a small diamond (just like proof
% environments end with a small square)
\usepackage{etoolbox}
\AtEndEnvironment{vb}{\null\hfill$\diamond$}%
\AtEndEnvironment{intermezzo}{\null\hfill$\diamond$}%
% \AtEndEnvironment{opmerking}{\null\hfill$\diamond$}%

% Fix some spacing
% http://tex.stackexchange.com/questions/22119/how-can-i-change-the-spacing-before-theorems-with-amsthm
\makeatletter
\def\thm@space@setup{%
  \thm@preskip=\parskip \thm@postskip=0pt
}

% Fix some stuff
% %http://tex.stackexchange.com/questions/76273/multiple-pdfs-with-page-group-included-in-a-single-page-warning
\pdfsuppresswarningpagegroup=1


% My name
\author{Jaden Wang}



\begin{document}
\centerline {\textsf{\textbf{\LARGE{Homework 9}}}}
\centerline {Jaden Wang}
\vspace{.15in}
\begin{problem}[18.4]
\[
	20 \times_{26} (-8) = -160 \mod 26 = -4
.\] 
\end{problem}

\begin{problem}[18.5]
\[
	(2,3)(3,5)= (2 \times_5 3, 3 \times_9 5) = (1,6)
.\] 
\end{problem}
\begin{problem}[18.11]
First let's show that $ R = \{a+b \sqrt{2}:a , b \in \zz \} $ is a ring. Take $ a+b\sqrt{2} $ and $ \c+d \sqrt{2} $ where $ a,b,c,d \in \zz$.

Since the addition operation is the addition of complex numbers which is commutative and associative, we have
\[
	(a+b\sqrt{2} ) + (c+ d \sqrt{2} ) = (a+c) + (b+d) \sqrt{2} 
.\] 
Since $ a+c, b+d \in \zz$, $ (a+c)+(b+d) \sqrt{2} \in R $. So $ R$ is closed under addition.

Since the identity of addition of complex numbers is 0, and $ 0= 0+0 \sqrt{2} \in R $, $ R$ contains the additive identity.

Notice
\[
	(-a)+(-b) \sqrt{2} + a+b \sqrt{2} = a+b \sqrt{2} + (-a) + (-b)\sqrt{2} = (a-a) + (b-b) \sqrt{2} = 0 + 0 \sqrt{2} = 0 
\]
so $ (-a)+(-b)\sqrt{2} $ is its inverse. Since $(-a),(-b) \in \zz \implies (-a)+(-b) \sqrt{2}  \in R$. $ R$ is closed under inverses.

Hence, we just showed that $ R$ is an abelian group under addition.

Since the multiplication is the multiplication of complex numbers which is commutative, associative, and left and right distributive with addition, we have
 \[
	 (a+b\sqrt{2} )(c+d\sqrt{2} )=(ac+bc\sqrt{2}) + (ad\sqrt{2} + 2bd)=(ac+2bd)+(bc+ad)\sqrt{2}  
.\] 
Since $ ad+2bd, bc+ad \in \zz$, $ (ac+2bd)+(bc+ad)\sqrt{2} \in R $. So $ R$ is closed under associative multiplication.

Therefore, by satisfying an abelian group under addition, closed under associative multiplication, and left and right distributive laws, $ R$ is a ring.

Since multiplication is commutative, $ R$ is a commutative ring.

Since the multiplicative identity is  $ 1$, and  $ 1 = 1 + 0\sqrt{2} \in R $, $ R$ contains the multiplicative identity.

However,  $ R$ is not a field. Consider  $ (0+1 \sqrt{2} )$. Since
\[
	(0+1\sqrt{2} )(0+\frac{1}{2}\sqrt{2} )=(0+\frac{1}{2}\sqrt{2} )(0+1\sqrt{2} ) = 1
,\] 
$ 0+\frac{1}{2}\sqrt{2} $ is its multiplicative inverse. Yet $ 0+\frac{1}{2}\sqrt{2} \not\in R $. Thus $ 0+1\sqrt{2} \in R$ is not a unit and $ R$ is not a field.
\end{problem}
\begin{problem}[18.12]
First let's show that $ R = \{a+b \sqrt{2}:a , b \in \qq \} $ is a ring. Take $ a+b\sqrt{2} $ and $ \c+d \sqrt{2} $ where $ a,b,c,d \in \qq$.

Since the addition operation is the addition of complex numbers which is commutative and associative, we have
\[
	(a+b\sqrt{2} ) + (c+ d \sqrt{2} ) = (a+c) + (b+d) \sqrt{2} 
.\] 
Since $ a+c, b+d \in \qq$, $ (a+c)+(b+d) \sqrt{2} \in R $. So $ R$ is closed under addition.

Since the identity of addition of complex numbers is 0, and $ 0= 0+0 \sqrt{2} \in R $, $ R$ contains the additive identity.

Notice
\[
	(-a)+(-b) \sqrt{2} + a+b \sqrt{2} = a+b \sqrt{2} + (-a) + (-b)\sqrt{2} = (a-a) + (b-b) \sqrt{2} = 0 + 0 \sqrt{2} = 0 
\]
so $ (-a)+(-b)\sqrt{2} $ is its inverse. Since $(-a),(-b) \in \qq \implies (-a)+(-b) \sqrt{2}  \in R$. $ R$ is closed under inverses.

Hence, we just showed that $ R$ is an abelian group under addition.

Since the multiplication is the multiplication of complex numbers which is commutative, associative, and left and right distributive with addition, we have
 \[
	 (a+b\sqrt{2} )(c+d\sqrt{2} )=(ac+bc\sqrt{2}) + (ad\sqrt{2} + 2bd)=(ac+2bd)+(bc+ad)\sqrt{2}  
.\] 
Since $ ad+2bd, bc+ad \in \qq$, $ (ac+2bd)+(bc+ad)\sqrt{2} \in R $. So $ R$ is closed under associative multiplication.

Therefore, by satisfying an abelian group under addition, closed under associative multiplication, and left and right distributive laws, $ R$ is a ring.

Since multiplication is commutative, $ R$ is a commutative ring.

Since the multiplicative identity is  $ 1$, and  $ 1 = 1 + 0\sqrt{2} \in R $, $ R$ contains the multiplicative identity.

To show $ R$ is a field, consider  $ a+b\sqrt{2} \in R $, so $ a,b \in \qq$.
\begin{case}[]
$a\neq 0, b=0$, then clearly its inverse is  $ \frac{1}{a}+0\sqrt{2} \in R $.
\end{case}
\begin{case}[]
$a=0, b\neq 0$, then clearly its inverse is $ \frac{1}{2b} \sqrt{2} \in R $.
\end{case}
\begin{case}[]
	$ a\neq 0, b\neq 0$, then its inverse is  $ -\frac{a}{2b^2-a^2}+\frac{b}{2b^2-a^2} \sqrt{2} $, because $ a\neq b\sqrt{2} \ \forall \ a,b \in \qq \implies 2b^2-a^2 \neq 0 $ and 
	\begin{align*}
		&\qquad (a+b\sqrt{2} )\left( -\frac{a}{2b^2-a^2}+\frac{b}{2b^2-a^2} \sqrt{2} \right) \\
		&=\left( -\frac{a^2}{2b^2-a^2} + \frac{2b^2}{2b^2-a^2} \right) +\left( \frac{ab}{2b^2-a^2} - \frac{ab}{2b^2-a^2}  \right)  \sqrt{2}   \\
		&= \left( \frac{2b^2-a^2}{2b^2-a^2 } \right) + 0 \sqrt{2}   \\
		&=  1\\
	\end{align*}
	The other direction follows from commutativity. Moreover, since rational numbers are closed under addition and multiplication, the two coefficients of the inverse are in $ \qq$ and thus the inverse is in $ R$.
\begin{case}[]
$ a=0, b=0$, then we get the zero element which we disregard.
\end{case}
Therefore, for all possible cases of $ a,b \in \qq$, we show that nonzero $ a+b\sqrt{2} $ is a unit, proving $ R$ is a field.
\end{case}
\end{problem}
\begin{problem}[18.13]
Given purely imaginary numbers $ ri,si$ where $ r,s \in \rr$. It is closed under addition:
\[
	ri + si = (r+s) i
\] 
where $ r+s \in \rr$. However, it is not closed under multiplication:
\[
1i \cdot 2i = 2i^2 = -2 
\]
which is not purely imaginary. Since it's not closed under multiplication, it is not a ring.
\end{problem}
\begin{problem}[18.15]
	Let $ a,b\in \zz$, so $ (a,b)\in \zz\times \zz$. Since the identity of $ \zz\times \zz$ is $ (1,1)$, for $ (a,b)$ to be a unit, we need  $ (c,d) \in \zz\times \zz$ such that 
	\begin{align*}
		(a,b)(c,d)&=(1,1)\\
		(ac,bd)&= (1,1) \\
		ac=1&,bd=1  \\
		c=\frac{1}{a} \in \zz &, d=\frac{1}{b} \in \zz, a,b\neq 0
	\end{align*}
	This means that $ a,b$ are divisors of  $ 1$, implying that $ a,b=\pm 1$. Therefore, the units are $(1,1),(-1,1),(-1,-1),(1,-1) $.
\end{problem}

\begin{problem}[18.17]
	Taken $ a \in \qq \setminus \{0\}$, then $ \frac{1}{a} = a^{-1}= \in \qq$,  thus the units of $ \qq$ are $ \qq \setminus \{0\}$.
\end{problem}
\begin{problem}[18.18]
Since direct product has elementwise operations, the units of $ \zz\times \qq\times \zz$ are simply the direct product of units of each set. By 18.15 and 18.17, the units are 
\[
	\{(a,b,c): a,c = \pm 1, b \in \qq\setminus \{0\}\}   
.\] 
\end{problem}
\begin{problem}[18.23]
	Recall that the only functions that satisfy injective group homomorphism from $ \zz \to \zz$ are $ \phi(x)=x$ and $ \phi(x)=-x$. Since ring homomorphism must satisfy group homomorphism, these two are our only candidates. For the identity map, it clearly still works, since
	\[
		\phi(a+b) = a+b = \phi(a)+\phi(b) \text{ and } \phi(ab)=ab=\phi(a)\phi(b) 
	.\] 
	But $ \phi(x)=-x$ no longer works:
	\[
		\phi(1 \times 2)=-(2)=-2\neq 2 =(-1)(-2) = \phi(1)\times \phi(2) 
	.\]
	So the only injective ring homomorphism $ \phi: \zz \to \zz$ is $ \phi(x)=x$.
\end{problem}
\begin{problem}[18.33]
~\begin{enumerate}[label=\alph*)]
	\item True. By definition.
	\item False. $ 2\zz$ under usual $ +,\times $ has no multiplicative identity, yet is a ring.
	\item False. $ \zz_2 $ is a ring, but the multiplicative identity $ 1$ is the only unit.
	\item False.  $ \rr$ has infinite units and is a ring.
	\item True. Consider $ \zz \subseteq \qq$, $ \zz$ is clearly a ring but not a field, and $ \qq$ is a field.
	\item False. They govern how $ +,\times $ interact!
	\item True. By definition.
	\item True. We get closure and associativity of the definition of ring, and identity and inverses from the definition of field.
	\item True. By definition.
	\item True. By definition.

\end{enumerate}
\end{problem}
\end{document}


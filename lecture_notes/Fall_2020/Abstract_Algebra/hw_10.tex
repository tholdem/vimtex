\documentclass[12pt]{article}
%Fall 2020
% Some basic packages
\usepackage{standalone}[subpreambles=true]
\usepackage[utf8]{inputenc}
\usepackage[T1]{fontenc}
\usepackage{textcomp}
\usepackage[english]{babel}
\usepackage{url}
\usepackage{graphicx}
\usepackage{float}
\usepackage{enumitem}


\pdfminorversion=7

% Don't indent paragraphs, leave some space between them
\usepackage{parskip}

% Hide page number when page is empty
\usepackage{emptypage}
\usepackage{subcaption}
\usepackage{multicol}
\usepackage[dvipsnames]{xcolor}


% Math stuff
\usepackage{amsmath, amsfonts, mathtools, amsthm, amssymb}
% Fancy script capitals
\usepackage{mathrsfs}
\usepackage{cancel}
% Bold math
\usepackage{bm}
% Some shortcuts
\newcommand{\rr}{\ensuremath{\mathbb{R}}}
\newcommand{\zz}{\ensuremath{\mathbb{Z}}}
\newcommand{\qq}{\ensuremath{\mathbb{Q}}}
\newcommand{\nn}{\ensuremath{\mathbb{N}}}
\newcommand{\ff}{\ensuremath{\mathbb{F}}}
\newcommand{\cc}{\ensuremath{\mathbb{C}}}
\renewcommand\O{\ensuremath{\emptyset}}
\newcommand{\norm}[1]{{\left\lVert{#1}\right\rVert}}
\renewcommand{\vec}[1]{{\mathbf{#1}}}
\newcommand\allbold[1]{{\boldmath\textbf{#1}}}

% Put x \to \infty below \lim
\let\svlim\lim\def\lim{\svlim\limits}

%Make implies and impliedby shorter
\let\implies\Rightarrow
\let\impliedby\Leftarrow
\let\iff\Leftrightarrow
\let\epsilon\varepsilon

% Add \contra symbol to denote contradiction
\usepackage{stmaryrd} % for \lightning
\newcommand\contra{\scalebox{1.5}{$\lightning$}}

% \let\phi\varphi

% Command for short corrections
% Usage: 1+1=\correct{3}{2}

\definecolor{correct}{HTML}{009900}
\newcommand\correct[2]{\ensuremath{\:}{\color{red}{#1}}\ensuremath{\to }{\color{correct}{#2}}\ensuremath{\:}}
\newcommand\green[1]{{\color{correct}{#1}}}

% horizontal rule
\newcommand\hr{
    \noindent\rule[0.5ex]{\linewidth}{0.5pt}
}

% hide parts
\newcommand\hide[1]{}

% si unitx
\usepackage{siunitx}
\sisetup{locale = FR}

% Environments
\makeatother
% For box around Definition, Theorem, \ldots
\usepackage[framemethod=TikZ]{mdframed}
\mdfsetup{skipabove=1em,skipbelow=0em}

%definition
\newenvironment{defn}[1][]{%
\ifstrempty{#1}%
{\mdfsetup{%
frametitle={%
\tikz[baseline=(current bounding box.east),outer sep=0pt]
\node[anchor=east,rectangle,fill=Emerald]
{\strut Definition};}}
}%
{\mdfsetup{%
frametitle={%
\tikz[baseline=(current bounding box.east),outer sep=0pt]
\node[anchor=east,rectangle,fill=Emerald]
{\strut Definition:~#1};}}%
}%
\mdfsetup{innertopmargin=10pt,linecolor=Emerald,%
linewidth=2pt,topline=true,%
frametitleaboveskip=\dimexpr-\ht\strutbox\relax
}
\begin{mdframed}[]\relax%
\label{#1}}{\end{mdframed}}


%theorem
%\newcounter{thm}[section]\setcounter{thm}{0}
%\renewcommand{\thethm}{\arabic{section}.\arabic{thm}}
\newenvironment{thm}[1][]{%
%\refstepcounter{thm}%
\ifstrempty{#1}%
{\mdfsetup{%
frametitle={%
\tikz[baseline=(current bounding box.east),outer sep=0pt]
\node[anchor=east,rectangle,fill=blue!20]
%{\strut Theorem~\thethm};}}
{\strut Theorem};}}
}%
{\mdfsetup{%
frametitle={%
\tikz[baseline=(current bounding box.east),outer sep=0pt]
\node[anchor=east,rectangle,fill=blue!20]
%{\strut Theorem~\thethm:~#1};}}%
{\strut Theorem:~#1};}}%
}%
\mdfsetup{innertopmargin=10pt,linecolor=blue!20,%
linewidth=2pt,topline=true,%
frametitleaboveskip=\dimexpr-\ht\strutbox\relax
}
\begin{mdframed}[]\relax%
\label{#1}}{\end{mdframed}}


%lemma
\newenvironment{lem}[1][]{%
\ifstrempty{#1}%
{\mdfsetup{%
frametitle={%
\tikz[baseline=(current bounding box.east),outer sep=0pt]
\node[anchor=east,rectangle,fill=Dandelion]
{\strut Lemma};}}
}%
{\mdfsetup{%
frametitle={%
\tikz[baseline=(current bounding box.east),outer sep=0pt]
\node[anchor=east,rectangle,fill=Dandelion]
{\strut Lemma:~#1};}}%
}%
\mdfsetup{innertopmargin=10pt,linecolor=Dandelion,%
linewidth=2pt,topline=true,%
frametitleaboveskip=\dimexpr-\ht\strutbox\relax
}
\begin{mdframed}[]\relax%
\label{#1}}{\end{mdframed}}

%corollary
\newenvironment{coro}[1][]{%
\ifstrempty{#1}%
{\mdfsetup{%
frametitle={%
\tikz[baseline=(current bounding box.east),outer sep=0pt]
\node[anchor=east,rectangle,fill=CornflowerBlue]
{\strut Corollary};}}
}%
{\mdfsetup{%
frametitle={%
\tikz[baseline=(current bounding box.east),outer sep=0pt]
\node[anchor=east,rectangle,fill=CornflowerBlue]
{\strut Corollary:~#1};}}%
}%
\mdfsetup{innertopmargin=10pt,linecolor=CornflowerBlue,%
linewidth=2pt,topline=true,%
frametitleaboveskip=\dimexpr-\ht\strutbox\relax
}
\begin{mdframed}[]\relax%
\label{#1}}{\end{mdframed}}

%proof
\newenvironment{prf}[1][]{%
\ifstrempty{#1}%
{\mdfsetup{%
frametitle={%
\tikz[baseline=(current bounding box.east),outer sep=0pt]
\node[anchor=east,rectangle,fill=SpringGreen]
{\strut Proof};}}
}%
{\mdfsetup{%
frametitle={%
\tikz[baseline=(current bounding box.east),outer sep=0pt]
\node[anchor=east,rectangle,fill=SpringGreen]
{\strut Proof:~#1};}}%
}%
\mdfsetup{innertopmargin=10pt,linecolor=SpringGreen,%
linewidth=2pt,topline=true,%
frametitleaboveskip=\dimexpr-\ht\strutbox\relax
}
\begin{mdframed}[]\relax%
\label{#1}}{\qed\end{mdframed}}


\theoremstyle{definition}

\newmdtheoremenv[nobreak=true]{definition}{Definition}
\newmdtheoremenv[nobreak=true]{prop}{Proposition}
\newmdtheoremenv[nobreak=true]{theorem}{Theorem}
\newmdtheoremenv[nobreak=true]{corollary}{Corollary}
\newtheorem*{eg}{Example}
\theoremstyle{remark}
\newtheorem*{case}{Case}
\newtheorem*{notation}{Notation}
\newtheorem*{remark}{Remark}
\newtheorem*{note}{Note}
\newtheorem*{problem}{Problem}
\newtheorem*{observe}{Observe}
\newtheorem*{property}{Property}
\newtheorem*{intuition}{Intuition}


% End example and intermezzo environments with a small diamond (just like proof
% environments end with a small square)
\usepackage{etoolbox}
\AtEndEnvironment{vb}{\null\hfill$\diamond$}%
\AtEndEnvironment{intermezzo}{\null\hfill$\diamond$}%
% \AtEndEnvironment{opmerking}{\null\hfill$\diamond$}%

% Fix some spacing
% http://tex.stackexchange.com/questions/22119/how-can-i-change-the-spacing-before-theorems-with-amsthm
\makeatletter
\def\thm@space@setup{%
  \thm@preskip=\parskip \thm@postskip=0pt
}

% Fix some stuff
% %http://tex.stackexchange.com/questions/76273/multiple-pdfs-with-page-group-included-in-a-single-page-warning
\pdfsuppresswarningpagegroup=1


% My name
\author{Jaden Wang}



\begin{document}
\centerline {\textsf{\textbf{\LARGE{Homework 10}}}}
\centerline {Jaden Wang}
\vspace{.15in}

\begin{problem}[19.2]
	For both cases, we wish to find the multiplicative inverse of $ 3$  to solve for $ x$. It exists because both cases are fields.
\begin{case}[]
	$ \zz_7$: to find $ a \in \zz_7$, the inverse of $ 3$, we know that $ 3 \times_7 a = 3 a \mod 7= 1$. So $ 3a = 7n +1 $ for some  $ n \in \zz$. $ n=1$ doesn't work but  $ n=2$ does, and we have  $ 3a=2 \times 7 + 1 = 15 \implies a = 5 \in \zz_7$. Thus,
	\begin{align*}
		3x &= 2 \\ 
		5 \times_7 3 x &= 5 \times_7 2 \\ 
		1 \cdot x &= 10 \mod 7\\
		x&= 3 \\
\end{align*}
\end{case}

\begin{case}[]
	$ \zz_{23}$ : Similarly, we wish to find the inverse of 3, $ a$,  such that $ 3a=23n+1$.  $ n=1$ works and gives us  $ a=8$. Thus,
	 \begin{align*}
		3x&= 2 \\
		8 \times_7 3 x&= 8 \times_{23} 2 \\
		1 \cdot x&= 16 \mod 23 \\
		x &= 16 \\
	\end{align*}
\end{case}
\end{problem}
\begin{problem}[19.3]
In $ \zz_6$:
\begin{align*}
	x^2+2x+2 &= 0 \\
	x^2 + 2x + 1 &= -1 \\
	(x+1)^2 & =5 \\
\end{align*}
Let's find the solution(s) by exhaustion:
\begin{align*}
	(0+1)^2 &= 1\\
	(1+1)^2 &= 4 \\
	(2+1)^2 &= 3 \\
	(3+1)^2 &= 4 \\
	(4+1)^2 &= 1 \\
	(5+1)^2 &= 0 \\
\end{align*}
None of them equals to 5. Hence there is no solution.
\end{problem}

\begin{problem}[19.11]
Since $ R$ is commutative, has multiplicative identity, and has characteristic 4, we have  $ 4a=0$, and
 \begin{align*}
	 (a+b)^4 &= (a+b)^2 (a+b)^2 \\
		 &= (a^2+2ab+b^2)(a^2+2ab+b^2) \\
		 &= a^{4}+ 2a^3b+a^2 b^2 + 2a^3b+4a^2b^2+2ab^3+a^2b^2+2ab^3+b^{4} \text{ by distributivity} \\
		 &= a^{4} + 4a^3b+4a^2b^2+2a^2b^2+4ab^3+b^{4} \\
		 &= a^{4}+2a^2b^2 +b^4 \text{ by }4a=0 \\
\end{align*}
\end{problem}

\begin{problem}[19.14]
	Note $ \begin{pmatrix} -2&-2\\1&1\\ \end{pmatrix} \in M_2(\zz)$, we have
	\[
		\begin{pmatrix} 1&2\\2&4\\ \end{pmatrix} \begin{pmatrix} -2&-2\\1&1\\ \end{pmatrix} = \begin{pmatrix} 0&0\\0&0\\ \end{pmatrix} 
	.\] 
	which is the additive identity of $ M_2(\zz)$. Thus, it is a zero divisor.
\end{problem}

\begin{problem}[19.15]
If $ a,b$ are the non-zero elements of the ring  $ R$ such that $ ab=0$, then  $ a,b$ are zero divisors of  $ R$.
\end{problem}

\begin{problem}[19.16]
If $ n \cdot a=0$ for all elements $ a$ in a ring  $ R$, then such smallest integer  $ n>0$ is the characteristic of $ R$. If no such positive integer exists, then the characteristic of  $ R$ is 0.
\end{problem}

\begin{problem}[19.17]
~\begin{enumerate}[label=\alph*)]
	\item False. Let $ a,b \in n\zz$, if $ ab=0$ under real multiplication, we have  $ a=0$ or  $ b=0$. Thus,  $ a,b$ cannot be zero divisors.
	\item True. By theorem.
	\item False. It is 0. Suppose there exists a  $ n$  such that $ n \cdot a=0 \ \forall \ a \in n\zz$, which is true iff $ n=0$ which isn't positive, so such  $ n$ doesn't exist. Then by definition it's 0.
	\item False. If  $ n=2$, then  $ 2\zz$ doesn't have multiplicative identity, and this is a structural difference to $ \zz$.
	\item True. If $ R$ is isomorphic to an integral domain,  $ R$ must have no zero divisors. Thus Theorem 19.5 applies.
	\item True. Suppose that the integral domain is finite. Given  $ a \neq 0 \in R$, it follows that $ |a|< \infty$ under addition. Therefore, there exists an $ n \in \nn$ such that $ na=0$, and by Theorem 19.15, $ R$ cannot have characteristic 0. By the contrapositive, if $ R$ has characteristic 0, it has to be infinite.
	\item False. Consider $ (1,0) \in D_1, (0,1) \in D_2$. Notice the additivity identity in $ D_1 \times D_2$ is $ (0,0)$ which neither equals to. But $(1,0) * (0,1) = (0,0)$, so both are zero divisors. Then the direct product cannot be an integral domain.
	\item False. Let's show the contrapositive: An element $ a$ in a commutative ring with unity  that has a multiplicative inverse $ a^{-1} \in R$ cannot be a divisor of zero.  Suppose $ a \neq 0$ is a zero divisor, by commutativity there exists $ b\neq 0$  such that $ ab=0$. We know that $ a a^{-1}=1$ and $ ab=0$, so
		\begin{align*}
			a a^{-1}-1 &= ab =0\\
			a(a^{-1} - b)&= 1 \\
			a^{-1}a (a^{-1} - b)&= a^{-1} \cdot 1 \\
			a^{-1} - b&= a^{-1} \\
			-b &= 0 \\
			b &= 0 \\
		\end{align*}
		which is a contradiction!
	\item False. Let $ n=2$,  $ 2\zz$ doesn't have identity and thus cannot be an integral domain.
	\item False. The inverse of $ 2$ is  $ \frac{1}{2} \not\in \zz$, so $ \zz$ is not a field.
\end{enumerate}
\end{problem}

\begin{problem}[19.18]
~\begin{enumerate}[label=\arabic*)]
	\item $ \rr$ is a field.
	\item $ \zz$ is an integral domain.
	\item $ \zz_{12}$ is a commutative ring with 1, but has zero divisors $ 3\times_{12} 4 =0 $.
	\item $ 2\zz$ doesn't have 1 but is commutative.
	\item $ M_2(\rr)$ isn't commutative but has the identity matrix.
	\item $ 2\zz \times M_2(\rr)$ is just a ring because direct product of rings is still a ring, but since $ 2\zz$ doesn't have identity and $ M_2(\rr)$ isn't commutative, their direct product can't either.
\end{enumerate}
\end{problem}

\begin{problem}[19.23]
Let $ R$ be a divisor ring and  $ a \in R$.
\begin{case}[]
$ a\neq 0$, then by definition of division ring, $ a^{-1} \in R$, and
\begin{align*}
	a^2&= a \\
	a^{-1} a^2 &= a^{-1} a \\
	a &= 1 \in R \\
\end{align*}
\begin{case}[]
$ a=0$, then  $ a^2=0^2=0 \in R$.
\end{case}
Thus, we have considered all cases of  $ a$, and only found exactly two elements, 0 and 1, that are idempotent.
\end{case}
\end{problem}
\end{document}

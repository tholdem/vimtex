\documentclass[class=article,crop=false]{standalone} 
%Fall 2020
% Some basic packages
\usepackage{standalone}[subpreambles=true]
\usepackage[utf8]{inputenc}
\usepackage[T1]{fontenc}
\usepackage{textcomp}
\usepackage[english]{babel}
\usepackage{url}
\usepackage{graphicx}
\usepackage{float}
\usepackage{enumitem}


\pdfminorversion=7

% Don't indent paragraphs, leave some space between them
\usepackage{parskip}

% Hide page number when page is empty
\usepackage{emptypage}
\usepackage{subcaption}
\usepackage{multicol}
\usepackage[dvipsnames]{xcolor}


% Math stuff
\usepackage{amsmath, amsfonts, mathtools, amsthm, amssymb}
% Fancy script capitals
\usepackage{mathrsfs}
\usepackage{cancel}
% Bold math
\usepackage{bm}
% Some shortcuts
\newcommand{\rr}{\ensuremath{\mathbb{R}}}
\newcommand{\zz}{\ensuremath{\mathbb{Z}}}
\newcommand{\qq}{\ensuremath{\mathbb{Q}}}
\newcommand{\nn}{\ensuremath{\mathbb{N}}}
\newcommand{\ff}{\ensuremath{\mathbb{F}}}
\newcommand{\cc}{\ensuremath{\mathbb{C}}}
\renewcommand\O{\ensuremath{\emptyset}}
\newcommand{\norm}[1]{{\left\lVert{#1}\right\rVert}}
\renewcommand{\vec}[1]{{\mathbf{#1}}}
\newcommand\allbold[1]{{\boldmath\textbf{#1}}}

% Put x \to \infty below \lim
\let\svlim\lim\def\lim{\svlim\limits}

%Make implies and impliedby shorter
\let\implies\Rightarrow
\let\impliedby\Leftarrow
\let\iff\Leftrightarrow
\let\epsilon\varepsilon

% Add \contra symbol to denote contradiction
\usepackage{stmaryrd} % for \lightning
\newcommand\contra{\scalebox{1.5}{$\lightning$}}

% \let\phi\varphi

% Command for short corrections
% Usage: 1+1=\correct{3}{2}

\definecolor{correct}{HTML}{009900}
\newcommand\correct[2]{\ensuremath{\:}{\color{red}{#1}}\ensuremath{\to }{\color{correct}{#2}}\ensuremath{\:}}
\newcommand\green[1]{{\color{correct}{#1}}}

% horizontal rule
\newcommand\hr{
    \noindent\rule[0.5ex]{\linewidth}{0.5pt}
}

% hide parts
\newcommand\hide[1]{}

% si unitx
\usepackage{siunitx}
\sisetup{locale = FR}

% Environments
\makeatother
% For box around Definition, Theorem, \ldots
\usepackage[framemethod=TikZ]{mdframed}
\mdfsetup{skipabove=1em,skipbelow=0em}

%definition
\newenvironment{defn}[1][]{%
\ifstrempty{#1}%
{\mdfsetup{%
frametitle={%
\tikz[baseline=(current bounding box.east),outer sep=0pt]
\node[anchor=east,rectangle,fill=Emerald]
{\strut Definition};}}
}%
{\mdfsetup{%
frametitle={%
\tikz[baseline=(current bounding box.east),outer sep=0pt]
\node[anchor=east,rectangle,fill=Emerald]
{\strut Definition:~#1};}}%
}%
\mdfsetup{innertopmargin=10pt,linecolor=Emerald,%
linewidth=2pt,topline=true,%
frametitleaboveskip=\dimexpr-\ht\strutbox\relax
}
\begin{mdframed}[]\relax%
\label{#1}}{\end{mdframed}}


%theorem
%\newcounter{thm}[section]\setcounter{thm}{0}
%\renewcommand{\thethm}{\arabic{section}.\arabic{thm}}
\newenvironment{thm}[1][]{%
%\refstepcounter{thm}%
\ifstrempty{#1}%
{\mdfsetup{%
frametitle={%
\tikz[baseline=(current bounding box.east),outer sep=0pt]
\node[anchor=east,rectangle,fill=blue!20]
%{\strut Theorem~\thethm};}}
{\strut Theorem};}}
}%
{\mdfsetup{%
frametitle={%
\tikz[baseline=(current bounding box.east),outer sep=0pt]
\node[anchor=east,rectangle,fill=blue!20]
%{\strut Theorem~\thethm:~#1};}}%
{\strut Theorem:~#1};}}%
}%
\mdfsetup{innertopmargin=10pt,linecolor=blue!20,%
linewidth=2pt,topline=true,%
frametitleaboveskip=\dimexpr-\ht\strutbox\relax
}
\begin{mdframed}[]\relax%
\label{#1}}{\end{mdframed}}


%lemma
\newenvironment{lem}[1][]{%
\ifstrempty{#1}%
{\mdfsetup{%
frametitle={%
\tikz[baseline=(current bounding box.east),outer sep=0pt]
\node[anchor=east,rectangle,fill=Dandelion]
{\strut Lemma};}}
}%
{\mdfsetup{%
frametitle={%
\tikz[baseline=(current bounding box.east),outer sep=0pt]
\node[anchor=east,rectangle,fill=Dandelion]
{\strut Lemma:~#1};}}%
}%
\mdfsetup{innertopmargin=10pt,linecolor=Dandelion,%
linewidth=2pt,topline=true,%
frametitleaboveskip=\dimexpr-\ht\strutbox\relax
}
\begin{mdframed}[]\relax%
\label{#1}}{\end{mdframed}}

%corollary
\newenvironment{coro}[1][]{%
\ifstrempty{#1}%
{\mdfsetup{%
frametitle={%
\tikz[baseline=(current bounding box.east),outer sep=0pt]
\node[anchor=east,rectangle,fill=CornflowerBlue]
{\strut Corollary};}}
}%
{\mdfsetup{%
frametitle={%
\tikz[baseline=(current bounding box.east),outer sep=0pt]
\node[anchor=east,rectangle,fill=CornflowerBlue]
{\strut Corollary:~#1};}}%
}%
\mdfsetup{innertopmargin=10pt,linecolor=CornflowerBlue,%
linewidth=2pt,topline=true,%
frametitleaboveskip=\dimexpr-\ht\strutbox\relax
}
\begin{mdframed}[]\relax%
\label{#1}}{\end{mdframed}}

%proof
\newenvironment{prf}[1][]{%
\ifstrempty{#1}%
{\mdfsetup{%
frametitle={%
\tikz[baseline=(current bounding box.east),outer sep=0pt]
\node[anchor=east,rectangle,fill=SpringGreen]
{\strut Proof};}}
}%
{\mdfsetup{%
frametitle={%
\tikz[baseline=(current bounding box.east),outer sep=0pt]
\node[anchor=east,rectangle,fill=SpringGreen]
{\strut Proof:~#1};}}%
}%
\mdfsetup{innertopmargin=10pt,linecolor=SpringGreen,%
linewidth=2pt,topline=true,%
frametitleaboveskip=\dimexpr-\ht\strutbox\relax
}
\begin{mdframed}[]\relax%
\label{#1}}{\qed\end{mdframed}}


\theoremstyle{definition}

\newmdtheoremenv[nobreak=true]{definition}{Definition}
\newmdtheoremenv[nobreak=true]{prop}{Proposition}
\newmdtheoremenv[nobreak=true]{theorem}{Theorem}
\newmdtheoremenv[nobreak=true]{corollary}{Corollary}
\newtheorem*{eg}{Example}
\theoremstyle{remark}
\newtheorem*{case}{Case}
\newtheorem*{notation}{Notation}
\newtheorem*{remark}{Remark}
\newtheorem*{note}{Note}
\newtheorem*{problem}{Problem}
\newtheorem*{observe}{Observe}
\newtheorem*{property}{Property}
\newtheorem*{intuition}{Intuition}


% End example and intermezzo environments with a small diamond (just like proof
% environments end with a small square)
\usepackage{etoolbox}
\AtEndEnvironment{vb}{\null\hfill$\diamond$}%
\AtEndEnvironment{intermezzo}{\null\hfill$\diamond$}%
% \AtEndEnvironment{opmerking}{\null\hfill$\diamond$}%

% Fix some spacing
% http://tex.stackexchange.com/questions/22119/how-can-i-change-the-spacing-before-theorems-with-amsthm
\makeatletter
\def\thm@space@setup{%
  \thm@preskip=\parskip \thm@postskip=0pt
}

% Fix some stuff
% %http://tex.stackexchange.com/questions/76273/multiple-pdfs-with-page-group-included-in-a-single-page-warning
\pdfsuppresswarningpagegroup=1


% My name
\author{Jaden Wang}



\begin{document}
\begin{problem}[22.3]
	In $ \zz_6[x]$ (note $ +_{ 6},\times _6 $ are implied):

	\begin{align*}
		f(x)g(x)&=(2+3 )x^2 + (3+2 ) x + (4+ 3)\\
		&= 5x^2 + 5x + 1 
	\end{align*}
	\begin{align*}
		f(x)g(x) &= (2 \times  3)x^{4} + (2\times  2 + 3 \times  3 ) x^3 + (2 \times 3 + 3\times 2 + 4\times 3 ) x^2 + (3\times 3+ 4 \times 2) x + 4\times 3\\
		&= x^3 + 5x 
	\end{align*}
\end{problem}

\begin{problem}[22.4]
	In $ \zz_5[x]$ :
\begin{align*}
	f(x)+g(x) &= 3x^{4} + 2x^3 + 4x^2 + (3+2) x + (2+4) \\
	&= 3x^{4} + 2x^3 + 4x^2 + 1 
\end{align*}

\begin{align*}
	f(x)g(x) &= (2\times 3) x^{7} + (4\times 3) x^{6} + (3 \times 3) x^{5} + (2 \times 3 + 2\times 2) x^{4} + (2\times 4 + 4\times 2) x^{3}\\ 
		 & \quad + (4\times 4 + 3\times 2) x^2 + (3\times 4 + 2\times 2)x + (2\times 4)\\
	&= x^{7}+ 2x^{6} + 4x^{5} + x^3 + 2x^2 + x +3 
\end{align*}
\end{problem}

\begin{problem}[22.8]
	In $ \cc$:
\begin{align*}
	\phi_{i} (2x^3 - x^2 + 3x + 2) &= 2 i^{3} - i^2 + 3i + 2 \\
	&= -2 i + 1 + 3i + 2 \\
	&= 3+ i  \\
\end{align*}
\end{problem}

\begin{problem}[22.9]
	In $ \zz_7[x]$:
\begin{align*}
	\phi_3 [(x^{4} + 2x )(x^3-3x^2+3)] &= \phi_3(x^{4}+ 2x) \times \phi_3(x^3 - 3x^2 + 3) \\
					   &= (3^{4} \bmod 7 + 2\times 3) (3^{3} \bmod 7 - 3^{3} \bmod 7 + 3) \\
					   &= (4+6) 3 \\
					   &= 2 
\end{align*}
\end{problem}

\begin{problem}[22.12]
	In $ \zz_2[x]$, trying exhaustively:
\begin{align*}
	\phi_0(x^2+1) = 0^2 + 1 = 1\\
	\phi_1(x^2+1) = 1^2 + 1 = 0\\
\end{align*}
Thus $ 1$ is the zero of  $ x^2+1$ in $ \zz_2[x]$.
\end{problem}

\begin{problem}[22.13]
	In $ \zz_7[x]$, trying exhaustively:
\begin{align*}
	\phi_0 = 0 ^3 + 2 \times 0 + 2 &= 2 \\
	\phi_1 = 1^3 + 2\times 1+ 2 &= 5 \\
	\phi_2 = 2^3 + 2 \times 2 + 2 &= 0 \\
	\phi_3 = 3^3 + 2\times 3 + 2 &= 0 \\ 
	\phi_4 = 4^3 + 2\times 4 + 2 &=4\\
	\phi_5 = 5^{3} + 2\times 5 +2 &= 4 \\
	\phi_6 = 6^{3} + 2\times 6 + 2 &=6
\end{align*}
Thus $ 2,3$ are the zeros of  $ x^3 + 2x+2 \in \zz_7[x]$.
\end{problem}

\begin{problem}[22.16]
	In $ \zz_5[x]$, since $ \gcd ( 3,5)=1 $, by Fermat $ 3^{4} = 1 \bmod 5$. 
	\begin{align*}
		\phi_3(x^{231}+ 3x^{117}-2x^{117}-2x^{53}+ 1) &= 3^{231} + 3^{118} - 2 \times 3^{53} + 1 \\
							      &= (3^{4})^{57} \times 3^3 + (3^{4})^{29} \times 3^2 - 2\times (3^{4})^{13} \times 3 \\
							      &= 3^{3} + 3^2 - 2\times 3 \\
							      &= 2+4-1  \\
							      &= 0 
	\end{align*}	
\end{problem}
\begin{problem}[22.18]
	A polynomial with coefficients in a ring R is an infinite formal sum
	\[
	\sum_{ i= 0}^{\infty} a_i x^{i}= a_0+a_1 x+ a_2 x^2 + \ldots + a_n x^{n} +\ldots
	.\] 
	where $ a_i \in \rr$ and only finitely many $ a_i \neq 0$.
\end{problem}

\begin{problem}[22.20]
\begin{align*}
	f(x,y) &= 3x^3y^3 + 2xy^3 + x^2y^2 - 6xy^2 + y^2 + x^{4} y - 2xy + x^{4} -3x^2 + 2 \\
	       &= (1+y)x^{4} + (3y^3)x^3 + (y^2 -3) x^2 + (2y^3 - 6 y^2- 2y)x+ (y^2 + 2)
\end{align*}
\end{problem}

\begin{problem}[22.22]
	In $ \zz_4[x]$, consider $ 2x+1$:
	 \[
		 (2x+1)(1-2x) = 1^2 - (2x)^2 = 1- 0x^2 = 1
	.\] 
	Since $ 2x+1\neq 0$, it is a unit in  $ \zz_4[x]$. 
\end{problem}

\begin{problem}[22.23]
\begin{enumerate}[label=\alph*)]
	\item True.
	\item True.
	\item True. By theorem.
	\item True.
	\item False. It cannot exceed 7.
	\item False. $ f(x)=2x^{3}, g(x)=2 x^{4} \in \zz_4[x]$, $ f(x)g(x) = 0$ doesn't have degree 7.  
	\item True. By theorem.
	\item True. By theorem.
	\item True. Given $ f(x)\neq 0 \in R[x]$, then $ xf(x)$ keeps all coefficients the same and add 1 degree to each term. Thus  $ xf(x) \neq 0$.
	\item False. In  $ \zz_{12}[x]$, $ 3x \times 4x = 0$ so they are zero divisors but they are not in $ \zz_{12}$. 
\end{enumerate}
\end{problem}

\begin{problem}[23.1]
\begin{figure}[H]
	\centering
	\includegraphics[width=0.8\textwidth]{}
	\caption{}
\end{figure}
\end{problem}

\begin{problem}[23.2]
	\begin{figure}[H]
		\centering
		\includegraphics[width=0.8\textwidth]{}
		\caption{}
	\end{figure}
\end{problem}

\begin{problem}[23.6]
	By theorem since 7 is prime, $ U(\zz_7) \simeq (\zz_6,+_{ 6} )$. We know the generators of $ \zz_6$ are 1,5 since they are coprime to 6, and notice $ 5 = -1$ which is the inverse of 1. Since isomorphism preserves structure, we know that  $ U(\zz_7)$ must have only 2 generators that are a pair of inverses. It suffices to find one generator in $ U(\zz_7)$ and the other is the inverse. Notice:
	\begin{align*}
		3^{1}&= 3 \\
		3^2 &= 2 \\
		3^{3} &= 6 \\
		3^{4} &= 4 \\
		3^{5} &= 5 \\
		3^{6} &= 1 
	\end{align*}
	Thus $ 3$ is a generator of  $ U(\zz_7)$. The other generator is thus its inverse $ 5$.
\end{problem}

\begin{problem}[23.9]
	In $ \zz_5[x]$:
	\begin{align*}
		x^{4}+ 4 &= x^{4} -1 \\
			 &= (x^2-1)(x^2+1) \\
			 &= (x-1)(x+1) (x^2-4)\\
			 &= (x-1)(x+1)(x-2)(x+2) 
	\end{align*}
\end{problem}

\begin{problem}[23.12]
	In $ \zz_5[x]$:
	\[
		\phi_4(x^3+ 2x+3) = (-1)^3 + (-2)+ 3 = 0
	.\] 
	Thus 4 is a zero. Since $ f(x)$ has degree 3 and  $ \zz_5$ is a field, by Theorem 23.10 $ f(x)$ is not irreducible. Thus by inspection,
\[
	x^3+ 2x + 3 = (x-1)(x^2-x + 3) = (x-1)(x+1)(x-3)
\] 
	since  $ \phi_{-1}(x^2 - x +3) =0$. 
\end{problem}

\begin{problem}[23.14]
Using the quadratic formula:
\[
r_{\pm} = \frac{-8 \pm \sqrt{8^2+8} }{2 } = -4 \pm 3\sqrt{2} \not\in \qq 
.\] 
So the roots of $ f(x)$ are not in  $ \qq$, and since $ f(x)$ has degree 2 and no zeros in  $ \qq$, it is irreducible over $ \qq$ by Theorem 23.10.

And since $ -4 \pm 3\sqrt{2} \in \rr \subseteq \cc $, $ f(x)$ has zeros in  $ \rr$ and $ \cc$, so it is not irreducible over $ \rr$ and $ \cc$.
\end{problem}

\begin{problem}[23.15]
Using the quadratic formula:
\[
r_{\pm} = \frac{-6 \pm \sqrt{36 - 48} }{2 } = -3 \pm i \sqrt{3} \not\in \qq \text{ or } \rr 
.\] 
Since $ f(x)$ has degree 2, by Theorem 23.10 it is irreducible over $ \qq$ and $ \rr$.

Since $ -3 \pm i \sqrt{3} \in \cc$, it is not irreducible over $ \cc$.
\end{problem}

\begin{problem}[23.16]
	Let $ p=3$ which is a prime. Notice  $ f(x) \in \zz[x], 1 \neq 0 \bmod 3, 8 \neq 0 \bmod 3^2, 3=0 \bmod 3$, by Eisenstein Criterion $ f(x)$ is irreducible over  $ \qq$.
\end{problem}
\begin{problem}[23.18]
	Let $ p=5$ which is a prime. Notice $ f(x) \in \zz[x], 1 \neq 0 \bmod 5, -12 \neq 0 \bmod 5^2, 0 = 0 \bmod 5$, by Eisenstein Criterion $ f(x)$ is irreducible over  $ \qq$. 
\end{problem}

\begin{problem}[23.19]
	Let $ p=3$ which is a prime. Notice  $ f(x) \in \zz[x], 8 \neq 0 \bmod 3, 6 = 0 \bmod 3, 9 = 0 \bmod 3, 24 \neq 0 \bmod 3^2$, by Eisenstein Criterion $ f(x)$ is irreducible over  $ \qq$.
\end{problem}

\begin{problem}[23.25]
	~\begin{enumerate}[label=\alph*)]
		\item True. Since $ \qq$ is a field, all degree 1 polynomials in $ \qq[x]$ are irreducible.
		\item True. By the same reasoning.
		\item True. Since the roots $\pm \sqrt{3} \not\in \qq $.
		\item False. In $ \zz_7$, $ x^2 + 3= x^2 -4 = (x+2)(x-2)$.
		\item True. By theorem.
		\item True (repeat).
		\item True. By Corollary 23.5.
		\item True. By factor theorem.
		\item True. Because its coefficient has inverse since $ F$ is a field.
		\item True. Because we only have finitely many nonzero terms, and we can only at most have as many zeros as the leading degree of the polynomial.
	\end{enumerate}
\end{problem}
\end{document}

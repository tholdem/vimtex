\documentclass[class=article,crop=false]{standalone} 
%Fall 2020
% Some basic packages
\usepackage{standalone}[subpreambles=true]
\usepackage[utf8]{inputenc}
\usepackage[T1]{fontenc}
\usepackage{textcomp}
\usepackage[english]{babel}
\usepackage{url}
\usepackage{graphicx}
\usepackage{float}
\usepackage{enumitem}


\pdfminorversion=7

% Don't indent paragraphs, leave some space between them
\usepackage{parskip}

% Hide page number when page is empty
\usepackage{emptypage}
\usepackage{subcaption}
\usepackage{multicol}
\usepackage[dvipsnames]{xcolor}


% Math stuff
\usepackage{amsmath, amsfonts, mathtools, amsthm, amssymb}
% Fancy script capitals
\usepackage{mathrsfs}
\usepackage{cancel}
% Bold math
\usepackage{bm}
% Some shortcuts
\newcommand{\rr}{\ensuremath{\mathbb{R}}}
\newcommand{\zz}{\ensuremath{\mathbb{Z}}}
\newcommand{\qq}{\ensuremath{\mathbb{Q}}}
\newcommand{\nn}{\ensuremath{\mathbb{N}}}
\newcommand{\ff}{\ensuremath{\mathbb{F}}}
\newcommand{\cc}{\ensuremath{\mathbb{C}}}
\renewcommand\O{\ensuremath{\emptyset}}
\newcommand{\norm}[1]{{\left\lVert{#1}\right\rVert}}
\renewcommand{\vec}[1]{{\mathbf{#1}}}
\newcommand\allbold[1]{{\boldmath\textbf{#1}}}

% Put x \to \infty below \lim
\let\svlim\lim\def\lim{\svlim\limits}

%Make implies and impliedby shorter
\let\implies\Rightarrow
\let\impliedby\Leftarrow
\let\iff\Leftrightarrow
\let\epsilon\varepsilon

% Add \contra symbol to denote contradiction
\usepackage{stmaryrd} % for \lightning
\newcommand\contra{\scalebox{1.5}{$\lightning$}}

% \let\phi\varphi

% Command for short corrections
% Usage: 1+1=\correct{3}{2}

\definecolor{correct}{HTML}{009900}
\newcommand\correct[2]{\ensuremath{\:}{\color{red}{#1}}\ensuremath{\to }{\color{correct}{#2}}\ensuremath{\:}}
\newcommand\green[1]{{\color{correct}{#1}}}

% horizontal rule
\newcommand\hr{
    \noindent\rule[0.5ex]{\linewidth}{0.5pt}
}

% hide parts
\newcommand\hide[1]{}

% si unitx
\usepackage{siunitx}
\sisetup{locale = FR}

% Environments
\makeatother
% For box around Definition, Theorem, \ldots
\usepackage[framemethod=TikZ]{mdframed}
\mdfsetup{skipabove=1em,skipbelow=0em}

%definition
\newenvironment{defn}[1][]{%
\ifstrempty{#1}%
{\mdfsetup{%
frametitle={%
\tikz[baseline=(current bounding box.east),outer sep=0pt]
\node[anchor=east,rectangle,fill=Emerald]
{\strut Definition};}}
}%
{\mdfsetup{%
frametitle={%
\tikz[baseline=(current bounding box.east),outer sep=0pt]
\node[anchor=east,rectangle,fill=Emerald]
{\strut Definition:~#1};}}%
}%
\mdfsetup{innertopmargin=10pt,linecolor=Emerald,%
linewidth=2pt,topline=true,%
frametitleaboveskip=\dimexpr-\ht\strutbox\relax
}
\begin{mdframed}[]\relax%
\label{#1}}{\end{mdframed}}


%theorem
%\newcounter{thm}[section]\setcounter{thm}{0}
%\renewcommand{\thethm}{\arabic{section}.\arabic{thm}}
\newenvironment{thm}[1][]{%
%\refstepcounter{thm}%
\ifstrempty{#1}%
{\mdfsetup{%
frametitle={%
\tikz[baseline=(current bounding box.east),outer sep=0pt]
\node[anchor=east,rectangle,fill=blue!20]
%{\strut Theorem~\thethm};}}
{\strut Theorem};}}
}%
{\mdfsetup{%
frametitle={%
\tikz[baseline=(current bounding box.east),outer sep=0pt]
\node[anchor=east,rectangle,fill=blue!20]
%{\strut Theorem~\thethm:~#1};}}%
{\strut Theorem:~#1};}}%
}%
\mdfsetup{innertopmargin=10pt,linecolor=blue!20,%
linewidth=2pt,topline=true,%
frametitleaboveskip=\dimexpr-\ht\strutbox\relax
}
\begin{mdframed}[]\relax%
\label{#1}}{\end{mdframed}}


%lemma
\newenvironment{lem}[1][]{%
\ifstrempty{#1}%
{\mdfsetup{%
frametitle={%
\tikz[baseline=(current bounding box.east),outer sep=0pt]
\node[anchor=east,rectangle,fill=Dandelion]
{\strut Lemma};}}
}%
{\mdfsetup{%
frametitle={%
\tikz[baseline=(current bounding box.east),outer sep=0pt]
\node[anchor=east,rectangle,fill=Dandelion]
{\strut Lemma:~#1};}}%
}%
\mdfsetup{innertopmargin=10pt,linecolor=Dandelion,%
linewidth=2pt,topline=true,%
frametitleaboveskip=\dimexpr-\ht\strutbox\relax
}
\begin{mdframed}[]\relax%
\label{#1}}{\end{mdframed}}

%corollary
\newenvironment{coro}[1][]{%
\ifstrempty{#1}%
{\mdfsetup{%
frametitle={%
\tikz[baseline=(current bounding box.east),outer sep=0pt]
\node[anchor=east,rectangle,fill=CornflowerBlue]
{\strut Corollary};}}
}%
{\mdfsetup{%
frametitle={%
\tikz[baseline=(current bounding box.east),outer sep=0pt]
\node[anchor=east,rectangle,fill=CornflowerBlue]
{\strut Corollary:~#1};}}%
}%
\mdfsetup{innertopmargin=10pt,linecolor=CornflowerBlue,%
linewidth=2pt,topline=true,%
frametitleaboveskip=\dimexpr-\ht\strutbox\relax
}
\begin{mdframed}[]\relax%
\label{#1}}{\end{mdframed}}

%proof
\newenvironment{prf}[1][]{%
\ifstrempty{#1}%
{\mdfsetup{%
frametitle={%
\tikz[baseline=(current bounding box.east),outer sep=0pt]
\node[anchor=east,rectangle,fill=SpringGreen]
{\strut Proof};}}
}%
{\mdfsetup{%
frametitle={%
\tikz[baseline=(current bounding box.east),outer sep=0pt]
\node[anchor=east,rectangle,fill=SpringGreen]
{\strut Proof:~#1};}}%
}%
\mdfsetup{innertopmargin=10pt,linecolor=SpringGreen,%
linewidth=2pt,topline=true,%
frametitleaboveskip=\dimexpr-\ht\strutbox\relax
}
\begin{mdframed}[]\relax%
\label{#1}}{\qed\end{mdframed}}


\theoremstyle{definition}

\newmdtheoremenv[nobreak=true]{definition}{Definition}
\newmdtheoremenv[nobreak=true]{prop}{Proposition}
\newmdtheoremenv[nobreak=true]{theorem}{Theorem}
\newmdtheoremenv[nobreak=true]{corollary}{Corollary}
\newtheorem*{eg}{Example}
\theoremstyle{remark}
\newtheorem*{case}{Case}
\newtheorem*{notation}{Notation}
\newtheorem*{remark}{Remark}
\newtheorem*{note}{Note}
\newtheorem*{problem}{Problem}
\newtheorem*{observe}{Observe}
\newtheorem*{property}{Property}
\newtheorem*{intuition}{Intuition}


% End example and intermezzo environments with a small diamond (just like proof
% environments end with a small square)
\usepackage{etoolbox}
\AtEndEnvironment{vb}{\null\hfill$\diamond$}%
\AtEndEnvironment{intermezzo}{\null\hfill$\diamond$}%
% \AtEndEnvironment{opmerking}{\null\hfill$\diamond$}%

% Fix some spacing
% http://tex.stackexchange.com/questions/22119/how-can-i-change-the-spacing-before-theorems-with-amsthm
\makeatletter
\def\thm@space@setup{%
  \thm@preskip=\parskip \thm@postskip=0pt
}

% Fix some stuff
% %http://tex.stackexchange.com/questions/76273/multiple-pdfs-with-page-group-included-in-a-single-page-warning
\pdfsuppresswarningpagegroup=1


% My name
\author{Jaden Wang}



\begin{document}
\section{Homomorphism}

\begin{note}[]
\allbold{Homomorphism} is a structure preserving map. 
\end{note}
In linear algebra: it preserves vector addition and scalar multiplication (linear maps).

In group theory: preserves group operation.

\begin{defn}[]
	Let $ (G,*)$ and  $ (H,*)$ be groups. A  \allbold{homomorphism} (of groups) from $ G$ to  $ H$ is a function $ \phi: G \to H$ such that 
	\[
		\phi(g_1 *_G g_2) = \phi(g_1) *_H \phi(g_2)
	.\] 
\end{defn}
\begin{note}[]
	This resembles linear maps $ T(u+v)=T(u)+T(v)$. Any linear map is a group homomorphism. An isomorphism is a bijective homomorphism.
\end{note}

\begin{eg}[uninteresting]
	$ T:V \to W$? Let $ T(v)=0$ for all  $ v \in V$. Similarly, let $ (G,*_G), (H,*_H)$ be groups. Define $ \phi: G \to H$ by $ \phi(g) =e_H$ for all $ g \in G$. Then $ \phi$ is a homomorphism.

	\begin{prf}
		$ \phi(x*_G y) = e_H = e_H *_H e_H = \phi(x) *_H \phi(y)$
	\end{prf}
\end{eg}
\begin{eg}[]
	$ T: V \to V$. Another trivial example of homomorphism is the identity map $ T(v) = v$. Similarly, $ (G,*_G)$ is a group. Let  $ \phi: G \to G$ be defined as $ \phi(g)=g$. The proof is trivial.
\end{eg}
\begin{eg}[interesting, not isomorphism]
	Consider $ GL_n( \rr)$: $ n\times n$ invertible matrices with entries from $ \rr$ under matrix multiplication. 

	Is it abelian? Counterexample: use $ 2 \times 2$ upper and lower triangle of all 1 as non-zero entries. Then we can just insert this to any $ n \times n$ identity matrices to the top left.

	$ GL_1( \rr) \simeq \rr^* $. This is abelian.

	So it is abelian if and only if $ n=1$.

	It is an infinite group. Just change one element of an identity matrix with infinite number of choices.

	$\det: GL_n( \rr) \to \rr^* $, $ A \mapsto \det(A)$.
\begin{claim}[]
$ \det$ is a homomorphism of groups. 
\end{claim}
	 It is surjective but not injective.
It isn't an isomorphism because $ \det$ is abelian. 

$ \det(AB) = \det (A) \det (B)$ by a Theorem from linear algebra. This satisfies the definition of homomorphism.
\end{eg}

\begin{eg}[sign map]
$ \varepsilon: S_n \to \zz_2$.
\begin{equation*}
	\varepsilon(g)
\begin{cases}
	0 & \text{ if }g \text{ is even}  \\
	1 & \text{ if } g \text{ is odd}  \\
\end{cases}
\end{equation*}
\begin{table}[H]
	\centering
	\begin{tabular}{c||c|c}
	&even&odd\\
	\hline
	\hline
		even&even&odd\\
		\hline
		odd&odd&even\\
	\end{tabular}
\end{table}
\begin{table}[H]
	\centering
	\begin{tabular}{c||c|c}
	$+_2$&0&1\\
	\hline
	\hline
		0&0&1\\
		\hline
		1&1&0\\
	\end{tabular}
\end{table}
WLOG assume $ x$ is even,  $ y$ is odd. Then
 \begin{align*}
	 \varepsilon(x*y) &=? \varepsilon(x) * \varepsilon(y) \\
	 1&= 0 +_2 1 \\
\end{align*}
True by table.

\end{eg}

\begin{note}[]
	Given linear map $ T: V \to W$. Then $ T(0_V) = 0_W$.  $ T(-v) = -T(v)$.

\end{note}
Similarly, for group homomorphism $\phi: G \to H$.
\begin{claim}[]
	$ \phi(e_G)=e_H$.
\end{claim}
\begin{prf}
\begin{align*}
	\phi(x*_G y ) &= \phi(x) *_H \phi(y)\\
	\phi(e_G) &= \phi(e_G * e_G) = \phi(e_G) *_H \phi(e_G) \\
	Y &= Y *_H Y \implies Y=\phi(e_G) = e_H \\
\end{align*}
\end{prf}
\begin{claim}[]
	$ \phi(g^{-1}) = \phi(g)^{-1}$.
\end{claim}
\begin{prf}
\begin{align*}
	\phi(x*_G x^{-1}) &= \phi(x) *_H \phi(x^{-1}) \\
	e_H = \phi(e_G) &=  \\
\end{align*}
Thus $ \phi(x^{-1}) = \phi(x)^{-1}$.
\end{prf}

\begin{eg}[]
	Consider $ \det: GL_n( \rr) \to \rr^* $. By the theorem above: 
	\[\det(A^{-1}) = \frac{1}{\det A}\]. 
\end{eg}

\begin{note}[]
	In linear algebra, $ T:V \to W$ linear map, \[\ker T=\{v \in V: T(v) = 0_W\} \] and \[\im T=\{w \in W: T(v)=w, v \in V\} \]
\end{note}
\begin{defn}[kernel and image]
	$ \phi: G \to H$ a homomorphism of groups. 
	\[
		\ker \phi = \{g \in G : \phi(g) = e_H\} 
	.\] 
	\[
		\im \phi=\{ h \in H: \phi(g) =h, g \in G\} 
	.\] 
	$ \ker \phi \leq G$, $ \im \phi \leq H$.
\end{defn}
See screenshot for illustration. 
\begin{prf}
$ \ker \phi \subseteq G$ by definition. Then
\begin{enumerate}[label=\arabic*)]
	\item $ \phi(e_G) = e_H$ by previous proof.
	\item closure: If $ x,y \in \ker \phi$, $ \phi(x)=e_H, \phi(y) = e_H$, and $ \phi(x*y) = e_H * e_H = e_H$.
	\item If $ x \in \ker \phi$, $ \phi(x^{-1}) = \phi(x)^{-1} = e_H ^{-1} = e_H$.
\end{enumerate}
\end{prf}
\end{document}
